% \iffalse meta-comment
%
%% File: l3draw-softpath.dtx Copyright(C) 2018 The LaTeX3 Project
%
% It may be distributed and/or modified under the conditions of the
% LaTeX Project Public License (LPPL), either version 1.3c of this
% license or (at your option) any later version.  The latest version
% of this license is in the file
%
%    http://www.latex-project.org/lppl.txt
%
% This file is part of the "l3trial bundle" (The Work in LPPL)
% and all files in that bundle must be distributed together.
%
% -----------------------------------------------------------------------
%
% The development version of the bundle can be found at
%
%    https://github.com/latex3/latex3
%
% for those people who are interested.
%
%<*driver>
\RequirePackage{expl3}
\documentclass[full]{l3doc}
\begin{document}
  \DocInput{\jobname.dtx}
\end{document}
%</driver>
% \fi
%
% \title{^^A
%   The \pkg{l3draw-softpath} package\\ Soft paths^^A
% }
%
% \author{^^A
%  The \LaTeX3 Project\thanks
%    {^^A
%      E-mail:
%        \href{mailto:latex-team@latex-project.org}
%          {latex-team@latex-project.org}^^A
%    }^^A
% }
%
% \date{Released 2018/02/05}
%
% \maketitle
%
% \begin{implementation}
%
% \section{\pkg{l3draw-softpath} implementation}
%
%    \begin{macrocode}
%<*initex|package>
%    \end{macrocode}
%
%    \begin{macrocode}
%<@@=draw>
%    \end{macrocode}
%
% There are two linked aims in the code here. The most significant is to
% provide a way to modify paths, for example to shorten the ends or round
% the corners.  This means that the path cannot be written piecemeal as
% specials, but rather needs to be held in macros. The second aspect that
% follows from this is performance: simply adding to a single macro a piece
% at a time will have poor performance as the list gets long. Paths need to
% be global (as specials are), so we cannot use \pkg{l3tl-build} or a similar
% approach. Instead, we use the same idea as \pkg{pgf}: use a series of buffer
% macros such that in most cases we don't add tokens to the main list. This
% will get slow only for \emph{enormous} paths.
%
% Each marker (operation) token takes two arguments, which makes processing
% more straight-forward. As such, some operations have dummy arguments, whilst
% others have to be split over several tokens. As the code here is at a low
% level, all dimension arguments are assumed to be explicit and fully-expanded.
%
% \begin{variable}
%   {
%     \g_@@_softpath_main_tl     ,
%     \g_@@_softpath_buffer_a_tl ,
%     \g_@@_softpath_buffer_b_tl
%   }
%   The soft path itself.
%    \begin{macrocode}
\tl_new:N \g_@@_softpath_main_tl
\tl_new:N \g_@@_softpath_buffer_a_tl
\tl_new:N \g_@@_softpath_buffer_b_tl
%    \end{macrocode}
% \end{variable}
%
% \begin{variable}
%   {
%     \g_@@_softpath_buffer_a_int ,
%     \g_@@_softpath_buffer_b_int
%   }
%   Tracking data.
%    \begin{macrocode}
\int_new:N \g_@@_softpath_buffer_a_int
\int_new:N \g_@@_softpath_buffer_b_int
%    \end{macrocode}
% \end{variable}
%
% \begin{macro}{\@@_softpath_add:n, \@@_softpath_add:x}
% \begin{macro}{\@@_softpath_concat:n}
% \begin{macro}{\@@_softpath_reset_buffers:}
%   The softpath itself is quite simple. We use three token lists to hold the
%   data: two buffers of limited length, and the main list of arbitrary size.
%   Most of the time this will mean that we don't add to the full list, so
%   performance will be acceptable.
%    \begin{macrocode}
\cs_new_protected:Npn \@@_softpath_add:n #1
  {
    \int_compare:nNnTF \g_@@_softpath_buffer_a_int < { 40 }
      {
        \int_gincr:N \g_@@_softpath_buffer_a_int
        \tl_gput_right:Nn \g_@@_softpath_buffer_a_tl {#1}
      }
      {
        \int_compare:nNnTF \g_@@_softpath_buffer_b_int < { 40 }
          {
            \int_gincr:N \g_@@_softpath_buffer_b_int
            \tl_gset:Nx \g_@@_softpath_buffer_b_tl
              {
                \exp_not:V \g_@@_softpath_buffer_b_tl
                \exp_not:V \g_@@_softpath_buffer_a_tl
                \exp_not:n {#1}
              }
            \int_gzero:N \g_@@_softpath_buffer_a_int
            \tl_gclear:N \g_@@_softpath_buffer_a_tl
          }
          { \@@_softpath_concat:n {#1} }
      }
  }
\cs_generate_variant:Nn \@@_softpath_add:n { x }
\cs_new_protected:Npn \@@_softpath_concat:n #1
  {
    \tl_gset:Nx \g_@@_softpath_main_tl
      {
        \exp_not:V \g_@@_softpath_main_tl
        \exp_not:V \g_@@_softpath_buffer_b_tl
        \exp_not:V \g_@@_softpath_buffer_a_tl
        \exp_not:n {#1}
      }
    \@@_softpath_reset_buffers:
  }
\cs_new_protected:Npn \@@_softpath_reset_buffers:
  {
    \int_gzero:N \g_@@_softpath_buffer_a_int
    \tl_gclear:N \g_@@_softpath_buffer_a_tl
    \int_gzero:N \g_@@_softpath_buffer_b_int
    \tl_gclear:N \g_@@_softpath_buffer_b_tl
  }
%    \end{macrocode}
% \end{macro}
%
% \begin{macro}{\@@_softpath_get:N, \@@_softpath_set_eq:N}
%   Save and restore functions.
%    \begin{macrocode}
\cs_new_protected:Npn \@@_softpath_get:N #1
  {
    \@@_softpath_concat:n { }
    \tl_set_eq:NN #1 \g_@@_softpath_main_tl
  }
\cs_new_protected:Npn \@@_softpath_set_eq:N #1
  {
    \tl_gset_eq:NN \g_@@_softpath_main_tl #1
    \@@_softpath_reset_buffers:
  }
%    \end{macrocode}
% \end{macro}
%
% \begin{macro}
%   {\@@_softpath_use:, \@@_softpath_clear:, \@@_softpath_use_clear:}
%   Using and clearing is trivial.
%    \begin{macrocode}
\cs_new_protected:Npn \@@_softpath_use:
  {
    \g_@@_softpath_main_tl
    \g_@@_softpath_buffer_b_tl
    \g_@@_softpath_buffer_a_tl
  }
\cs_new_protected:Npn \@@_softpath_clear:
  {
    \tl_gclear:N \g_@@_softpath_main_tl
    \tl_gclear:N \g_@@_softpath_buffer_a_tl
    \tl_gclear:N \g_@@_softpath_buffer_b_tl
  }
\cs_new_protected:Npn \@@_softpath_use_clear:
  {
    \@@_softpath_use:
    \@@_softpath_clear:
  }
%    \end{macrocode}
% \end{macro}
%
% \begin{variable}{\g_@@_softpath_lastx_dim, \g_@@_softpath_lasty_dim}
%   For tracking the end of the path (to close it).
%    \begin{macrocode}
\dim_new:N \g_@@_softpath_lastx_dim
\dim_new:N \g_@@_softpath_lasty_dim
%    \end{macrocode}
% \end{variable}
%
% \begin{variable}{\g_@@_softpath_move_bool}
%   Track if moving a point should update the close position.
%    \begin{macrocode}
\bool_new:N \g_@@_softpath_move_bool
\bool_gset_true:N \g_@@_softpath_move_bool
%    \end{macrocode}
% \end{variable}
%
% \begin{macro}{\@@_softpath_curveto:nnnnnn}
% \begin{macro}
%   {
%     \@@_softpath_lineto:nn,
%     \@@_softpath_moveto:nn
%   }
% \begin{macro}{\@@_softpath_rectangle:nnnn}
% \begin{macro}{\@@_softpath_roundpoint:nn, \@@_softpath_roundpoint:VV}
%   The various parts of a path expressed as the appropriate soft path
%   functions.
%    \begin{macrocode}
\cs_new_protected:Npn \@@_softpath_closepath:
  {
    \@@_softpath_add:x
      {
        \@@_softpath_close_op:nn
          { \dim_use:N \g_@@_softpath_lastx_dim }
          { \dim_use:N \g_@@_softpath_lasty_dim }
      }
  }
\cs_new_protected:Npn \@@_softpath_curveto:nnnnnn #1#2#3#4#5#6
  {
    \@@_softpath_add:n
      {
        \@@_softpath_curveto_opi:nn {#1} {#2}
        \@@_softpath_curveto_opii:nn {#3} {#4}
        \@@_softpath_curveto_opiii:nn {#5} {#6}
      }
  }
\cs_new_protected:Npn \@@_softpath_lineto:nn #1#2
  {
    \@@_softpath_add:n
      { \@@_softpath_lineto_op:nn {#1} {#2} }
  }
\cs_new_protected:Npn \@@_softpath_moveto:nn #1#2
  {
    \@@_softpath_add:n
      { \@@_softpath_moveto_op:nn {#1} {#2} }
    \bool_if:NT \g_@@_softpath_move_bool
      {
        \dim_gset:Nn \g_@@_softpath_lastx_dim {#1}
        \dim_gset:Nn \g_@@_softpath_lasty_dim {#2}
      }
  }
\cs_new_protected:Npn \@@_softpath_rectangle:nn #1#2#3#4
  {
    \@@_softpath_add:n
      {
        \@@_softpath_rectangle_opi:nn {#1} {#2}
        \@@_softpath_rectangle_opii:nn {#3} {#4}
      }
  }
\cs_new_protected:Npn \@@_softpath_roundpoint:nn #1#2
  {
    \@@_softpath_add:n
      { \@@_softpath_round_op:nn {#1} {#2} }
  }
\cs_generate_variant:Nn \@@_softpath_roundpoint:nn { VV }
%    \end{macrocode}
% \end{macro}
%
% \begin{macro}
%   {
%     \@@_softpath_close_op:nn      ,
%     \@@_softpath_curveto_opi:nn   ,
%     \@@_softpath_curveto_opii:nn  ,
%     \@@_softpath_curveto_opiii:nn ,
%     \@@_softpath_lineto_op:nn     ,
%     \@@_softpath_moveto_op:nn     ,
%     \@@_softpath_roundpoint_op:nn ,
%     \@@_softpath_rectangle_opi:nn ,
%     \@@_softpath_rectangle_opii:nn
%   }
% \begin{macro}{\@@_softpath_curveto_opi:nnNnnNnn}
% \begin{macro}{\@@_softpath_curveto_opi:nnNnn}
%   The markers for operations: all the top-level ones take two arguments.
%    \begin{macrocode}
\cs_new_protected:Npn \@@_softpath_close_op:nn #1#2
  { \driver_draw_closepath: }
\cs_new_protected:Npn \@@_softpath_curveto_opi:nn #1#2 
  { \@@_softpath_curveto_opi:nnNnnNnn {#1} {#2} }
\cs_new_protected:Npn \@@_softpath_curveto_opi:nnNnnNnn #1#2#3#4#5#6#7#8
  { \driver_draw_curveto:nnnnnn {#1} {#2} {#4} {#5} {#7} {#8} }
\cs_new_protected:Npn \@@_softpath_curveto_opii:nn #1#2 { }
\cs_new_protected:Npn \@@_softpath_curveto_opiii:nn #1#2 { }
\cs_new_protected:Npn \@@_softpath_lineto_op:nn #1#2
  { \driver_draw_lineto:nn {#1} {#2} }
\cs_new_protected:Npn \@@_softpath_moveto_op:nn #1#2
  { \driver_draw_moveto:nn {#1} {#2} }
\cs_new_protected:Npn \@@_softpath_roundpoint_op:nn #1#2 { }
\cs_new_protected:Npn \@@_softpath_rectangle_opi:nn #1#2 
  { \@@_softpath_rectangle_opi:nnNnn{#1} {#2} }
\cs_new_protected:Npn \@@_softpath_rectangle_opi:nnNnnNnn #1#2#3#4#5
  { \driver_draw_rectangle:nnnn {#1} {#2} {#4} {#5} }
  \cs_new_protected:Npn \@@_softpath_rectangle_opii:nn #1#2 { }
%    \end{macrocode}
% \end{macro}
% \end{macro}
% \end{macro}
%
%    \begin{macrocode}
%</initex|package>
%    \end{macrocode}
%
% \end{implementation}
%
% \PrintIndex
