% \iffalse meta-comment
%
%% File: l3draw-state.dtx Copyright(C) 2018 The LaTeX3 Project
%
% It may be distributed and/or modified under the conditions of the
% LaTeX Project Public License (LPPL), either version 1.3c of this
% license or (at your option) any later version.  The latest version
% of this license is in the file
%
%    http://www.latex-project.org/lppl.txt
%
% This file is part of the "l3trial bundle" (The Work in LPPL)
% and all files in that bundle must be distributed together.
%
% -----------------------------------------------------------------------
%
% The development version of the bundle can be found at
%
%    https://github.com/latex3/latex3
%
% for those people who are interested.
%
%<*driver>
\RequirePackage{expl3}
\documentclass[full]{l3doc}
\begin{document}
  \DocInput{\jobname.dtx}
\end{document}
%</driver>
% \fi
%
% \title{^^A
%   The \pkg{l3draw-state} package\\ Drawing graphics state^^A
% }
%
% \author{^^A
%  The \LaTeX3 Project\thanks
%    {^^A
%      E-mail:
%        \href{mailto:latex-team@latex-project.org}
%          {latex-team@latex-project.org}^^A
%    }^^A
% }
%
% \date{Released 2018/02/05}
%
% \maketitle
%
% \begin{implementation}
%
% \section{\pkg{l3draw-state} implementation}
%
%    \begin{macrocode}
%<*initex|package>
%    \end{macrocode}
%
%    \begin{macrocode}
%<@@=draw>
%    \end{macrocode}
%
% \begin{variable}{\g_@@_linewidth_dim, \g_@@_inner_linewidth_dim}
%   Linewidth for strokes: global as the scope for this relies on the graphics
%   state. The inner line width is used for places where two lines are used.
%    \begin{macrocode}
\dim_new:N \g_@@_linewidth_dim
\dim_new:N \g_@@_inner_linewidth_dim
%    \end{macrocode}
% \end{variable}
%
% \begin{variable}{\l_draw_default_linewidth_dim}
%   A default: this is used at the start of every drawing.
%    \begin{macrocode}
\dim_new:N \l_draw_default_linewidth_dim
\dim_set:Nn \l_draw_default_linewidth_dim { 0.4pt }
%    \end{macrocode}
% \end{variable}
%
% \begin{macro}{\draw_linewidth:n, \draw_inner_linewidth:n}
%   Set the linewidth: we need a wrapper as this has to pass to the driver
%   layer. The inner version is handled at the macro layer but is given a
%   consistent interface here.
%    \begin{macrocode}
\cs_new_protected:Npn \draw_linewidth:n #1
  {
    \dim_gset:Nn \g_@@_linewidth_dim { \fp_to_dim:n {#1} }
    \driver_draw_linewidth:n \g_@@_linewidth_dim
  }
\cs_new_protected:Npn \draw_inner_linewidth:n #1
  { \dim_gset:Nn \g_@@_inner_linewidth_dim { \fp_to_dim:n {#1} }  }
%    \end{macrocode}
% \end{macro}
%
% \begin{macro}{\draw_miterlimit:n}
%   Pass through to the driver layer.
%    \begin{macrocode}
\cs_new_protected:Npn \draw_miterlimit:n #1
  { \driver_draw_miterlimit:n { \fp_to_dim:n {#1} } }
%    \end{macrocode}
% \end{macro}
%
% \begin{macro}
%   {
%     \draw_cap_butt:, \draw_cap_rectangle:, \draw_cap_round:,
%     \draw_evenodd_rule:, \draw_nonzero_rule:,
%     \draw_join_bevel:, \draw_join_miter:, \draw_join_round:
%   }
%   All straight wrappers.
%    \begin{macrocode}
\cs_new_protected:Npn \draw_cap_butt: { \driver_draw_cap_butt: }
\cs_new_protected:Npn \draw_cap_rectangle: { \driver_draw_cap_rectangle: }
\cs_new_protected:Npn \draw_cap_round: { \driver_draw_cap_round: }
\cs_new_protected:Npn \draw_evenodd_rule: { \driver_draw_evenodd_rule: }
\cs_new_protected:Npn \draw_nonzero_rule: { \driver_draw_nonzero_rule: }
\cs_new_protected:Npn \draw_join_bevel: { \driver_draw_join_bevel: }
\cs_new_protected:Npn \draw_join_miter: { \driver_draw_join_miter: }
\cs_new_protected:Npn \draw_join_round: { \driver_draw_join_round: }
%    \end{macrocode}
% \end{macro}
%
%    \begin{macrocode}
%</initex|package>
%    \end{macrocode}
%
% \end{implementation}
%
% \PrintIndex
