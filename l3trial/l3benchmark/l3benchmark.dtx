% \iffalse
%
%% File l3benchmark.dtx Copyright (C) 2011,2012,2014-2018 The LaTeX3 Project
%%
%% It may be distributed and/or modified under the conditions of the
%% LaTeX Project Public License (LPPL), either version 1.3c of this
%% license or (at your option) any later version.  The latest version
%% of this license is in the file
%%
%%    http://www.latex-project.org/lppl.txt
%%
%% This file is part of the "l3trial bundle" (The Work in LPPL)
%% and all files in that bundle must be distributed together.
%%
%% The released version of this bundle is available from CTAN.
%%
%% -----------------------------------------------------------------------
%%
%% The development version of the bundle can be found at
%%
%%    http://www.latex-project.org/svnroot/experimental/trunk/
%%
%% for those people who are interested.
%%
%%%%%%%%%%%
%% NOTE: %%
%%%%%%%%%%%
%%
%%   Snapshots taken from the repository represent work in progress and may
%%   not work or may contain conflicting material!  We therefore ask
%%   people _not_ to put them into distributions, archives, etc. without
%%   prior consultation with the LaTeX Project Team.
%%
%% -----------------------------------------------------------------------
%%
%
%<*driver|package>
\RequirePackage{expl3}
%</driver|package>
%<*driver>
\documentclass[full]{l3doc}
\begin{document}
  \DocInput{\jobname.dtx}
\end{document}
%</driver>
% \fi
%
% \title{^^A
%   The \pkg{l3benchmark} package\\ Benchmark^^A
% }
%
% \author{^^A
%  The \LaTeX3 Project\thanks
%    {^^A
%      E-mail:
%        \href{mailto:latex-team@latex-project.org}
%          {latex-team@latex-project.org}^^A
%    }^^A
% }
%
% \date{Released 2018-04-30}
%
% \maketitle
%
% \begin{documentation}
%
% \section{Additions to \pkg{l3sys}: elapsed time}
%
% \begin{function}{\sys_timer_gzero:}
%   \begin{syntax}
%     \cs{sys_timer_gzero:}
%   \end{syntax}
%   Resets the timer to zero.
% \end{function}
%
% \begin{function}{\sys_timer_apply:N}
%   \begin{syntax}
%     \cs{sys_timer_apply:N} \meta{function}
%   \end{syntax}
%   Calls \meta{function} \Arg{time} where \meta{time} is the time
%   elapsed since the last \cs{sys_timer_gzero:}, which is also called
%   while loading the package.  The \meta{time} is an integer, measured
%   in scaled seconds, namely $2^{-16}$ seconds.  The integer may
%   overflow after about $9$ hours in some engines.
% \end{function}
%
% \section{Benchmark}
%
% \begin{variable}{\g_benchmark_duration_target_fp}
%   This variable controls roughly for how long \cs{benchmark:n} will
%   repeat code to more accurately benchmark it.  The actual duration of
%   one call to \cs{benchmark:n} typically lasts between half and twice
%   \cs{g_benchmark_duration_target_fp} seconds, unless of course
%   running the code only once already lasts longer than this.
% \end{variable}
%
% \begin{variable}{\g_benchmark_time_fp, \g_benchmark_ops_fp}
%   Functions such as \cs{benchmark:n} store the measured time in
%   \cs{g_benchmark_time_fp} (in seconds).  Functions such as
%   \cs{benchmark_normalized:n} store the estimated number of operations
%   in \cs{g_benchmark_ops_fp}.
% \end{variable}
%
% \begin{function}{\benchmark_display:, \benchmark_display_in_ops:}
%   \begin{syntax}
%     \cs{benchmark_display:}
%   \end{syntax}
%   Prints the time \cs{g_benchmark_time_fp} (in seconds) or the
%   estimated number of operations \cs{g_benchmark_ops_fp} to the
%   terminal.  These functions are called by functions such as
%   \cs{benchmark:n} and can be redefined by the user.
% \end{function}
%
% \begin{function}{\benchmark_once:n, \benchmark_once_in_ops:n, \benchmark_once_silent:n}
%   \begin{syntax}
%     \cs{benchmark_once:n} \Arg{code}
%   \end{syntax}
%   Measures the time taken by \TeX{} to run the \meta{code} once, sets
%   \cs{g_benchmark_time_fp} and \cs{g_benchmark_ops_fp}, and calls
%   \cs{benchmark_display:}.  The \meta{code} is run only once so the
%   time may be quite inaccurate for fast code.  The
%   \cs{benchmark_once_silent:n} function omits the call to
%   \cs{benchmark_display:}, while \cs{benchmark_once_in_ops:n} calls
%   \cs{benchmark_display_in_ops:} instead.
% \end{function}
%
% \begin{function}{\benchmark:n, \benchmark_in_ops:n, \benchmark_silent:n}
%   \begin{syntax}
%     \cs{benchmark:n} \Arg{code}
%   \end{syntax}
%   Measures the time taken by \TeX{} to run the \meta{code}, sets
%   \cs{g_benchmark_time_fp} and \cs{g_benchmark_ops_fp}, and calls
%   \cs{benchmark_display:}.  The \meta{code} may be run many times and
%   not within a group, thus code with side-effects may cause problems.
%   The \cs{benchmark_silent:n} function omits the call to
%   \cs{benchmark_display:}, while \cs{benchmark_in_ops:n} calls
%   \cs{benchmark_display_in_ops:} instead.
% \end{function}
%
% \begin{function}{\benchmark_tic:, \benchmark_toc:}
%   \begin{syntax}
%     \cs{benchmark_tic:} \meta{slow code} \cs{benchmark_toc:}
%   \end{syntax}
%   When it is not possible to run \cs{benchmark:n} (e.g., the code is
%   part of the execution of a package which cannot be looped) the
%   tic/toc commands can be used instead to time between two points in
%   the code.  When executed, \cs{benchmark_tic:} will print a line to the
%   terminal, and \cs{benchmark_toc:} will print a matching line with a
%   time to indicate the duration between them in seconds.  Note that
%   these commands can be nested.
% \end{function}
%
% \end{documentation}
%
% \begin{implementation}
%
% \section{\pkg{l3benchmark} implementation}
%
% Our working unit is the scaled second, namely $2^{-16}$ seconds.
%
% There are four cases: the elapsed time is available as the primitive
% \tn{pdfelapsedtime}, or through Heiko Oberdiek's \pkg{pdftexcmds} as
% \tn{pdf@elapsedtime}, or by accessing system time through
% \texttt{texlua} using unrestricted shell escape, or the elapsed time
% is not available.  The first two cases behave in the same way, while
% the third requires using floating point numbers.  The third way uses
% the following Lua code to print a large floating point number: the
% system time in scaled seconds; it is around $10^{14}$.
%    \begin{macrocode}
%<*lua>
if os.gettimeofday then
  io.write(65536 * os.gettimeofday())
else
  io.write(65536 * os.date("%s"))
end
%</lua>
%    \end{macrocode}
%
%    \begin{macrocode}
%<*initex|package>
%    \end{macrocode}
%
%    \begin{macrocode}
%<*package>
\ProvidesExplPackage{l3benchmark}{2018-04-30}{}
  {L3 Experimental benchmarking}
%</package>
%    \end{macrocode}
%
% \subsection{Additions to \pkg{l3sys}: elapsed time}
%
%    \begin{macrocode}
%<@@=sys>
%    \end{macrocode}
%
% \begin{macro}{\@@_timer:}
% \begin{macro}[pTF]{\sys_if_timer_exist:}
%   In \LuaTeX{}, load Heiko's \pkg{pdftexcmds} to get
%   \tn{pdf@elapsedtime}, otherwise try to locate the primitive.  The
%   elapsed time will be available if this succeeds or if unrestricted
%   shell escape is available.
%    \begin{macrocode}
\sys_if_engine_luatex:TF
  {
    \RequirePackage{pdftexcmds} \scan_stop:
    \cs_new_eq:NN \@@_timer: \pdf@elapsedtime
  }
  {
    \cs_if_exist:NTF \tex_elapsedtime:D
      { \cs_new_eq:NN \@@_timer: \tex_elapsedtime:D }
      {
        \cs_if_exist:NT \pdfelapsedtime
          { \cs_new_eq:NN \@@_timer: \pdfelapsedtime }
      }
  }
\@@_const:nn { sys_if_timer_exist }
  { \cs_if_exist_p:N \@@_timer: || \sys_if_shell_unrestricted_p: }
%    \end{macrocode}
% \end{macro}
% \end{macro}
%
% \begin{macro}{\sys_timer_gzero:, \sys_timer_apply:N, \@@_timer:, \@@_timer_fp:}
% \begin{variable}{\g_@@_timer_base_int, \g_@@_timer_base_fp}
%   Three case, and in each case we define \cs{sys_timer_gzero:} then
%   \cs{sys_timer_apply:N}.  If \cs{@@_timer:} exists then it is safe to
%   use the corresponding \enquote{resettimer} code for
%   \cs{sys_timer_gzero:}, or as a fall-back use
%   \cs{g_@@_timer_base_int} as an offset.
%    \begin{macrocode}
\cs_if_exist:NTF \@@_timer:
  {
    \cs_if_exist:NTF \pdf@resettimer
      { \cs_new_eq:NN \sys_timer_gzero: \pdf@resettimer }
      {
        \cs_if_exist:NTF \tex_resettimer:D
          { \cs_new_eq:NN \sys_timer_gzero: \tex_resettimer:D }
          {
            \cs_if_exist:NTF \pdfresettimer
              { \cs_new_eq:NN \sys_timer_gzero: \pdfresettimer }
              {
                \int_new:N \g_@@_timer_base_int
                \cs_new_protected:Npn \sys_timer_gzero:
                  { \int_gset:Nn \g_@@_timer_base_int { \@@_timer: } }
              }
          }
      }
    \int_if_exist:NTF \g_@@_timer_base_int
      {
        \cs_new_protected:Npn \sys_timer_apply:N #1
          {
            \exp_args:Nf #1
              { \int_eval:n { \@@_timer: - \g_@@_timer_base_int } }
          }
      }
      {
        \cs_new_protected:Npn \sys_timer_apply:N #1
          { \exp_args:No #1 { \int_value:w \@@_timer: } }
      }
  }
%    \end{macrocode}
%   Otherwise we use unrestricted shell escape to define
%   \cs{@@_timer_fp:} (an analogue of \cs{@@_timer:} that produces a
%   floating point number), and we store the offset in
%   \cs{g_@@_timer_base_fp}.  We neutralize the end of file marker using
%   \cs{exp_not:N}.
%    \begin{macrocode}
  {
    \sys_if_shell_unrestricted:TF
      {
        \cs_new_protected:Npn \@@_timer_fp:
          {
            \int_eval:n
              { \tex_input:D " | texlua ~ l3sys-time.lua " ~ }
            \exp_stop_f:
          }
        \fp_new:N \g_@@_timer_base_fp
        \cs_new_protected:Npn \sys_timer_gzero:
          { \fp_gset:Nn \g_@@_timer_base_fp { \@@_timer_fp: } }
        \cs_new_protected:Npn \sys_timer_apply:N #1
          {
            \group_begin:
            \tex_everyeof:D { \exp_not:N }
            \exp_args:NNf \group_end:
              #1
              { \fp_to_int:n { \@@_timer_fp: - \g_@@_timer_base_fp } }
          }
      }
%    \end{macrocode}
%   Finally if the elapsed time cannot be accessed then we define user
%   commands to produce errors.
%    \begin{macrocode}
      {
        \__kernel_msg_new:nnnn { kernel } { no-elapsed-time }
          { No~clock~detected~for~#1. }
          {
            The~current~engine~provides~no~way~to~access~the~system~time,~
            hence~no~way~to~know~the~time~elapsed~without~shell-escape.~
            Please~use~pdfTeX,~LuaTeX,~or~call~other~engines~with~the~
            --shell-escape~option.
          }
        \cs_new_protected:Npn \sys_timer_gzero:
          {
            \__kernel_msg_error:nnn { kernel } { no-elapsed-time }
              { \sys_timer_gzero: }
          }
        \cs_new_protected:Npn \sys_timer_apply:N #1
          {
            \__kernel_msg_error:nnn { kernel } { no-elapsed-time }
              { \sys_timer_apply:N #1 }
          }
      }
  }
%    \end{macrocode}
% \end{variable}
% \end{macro}
%
% The following line is needed in engines where elapsed time comes from
% system time, because the starting value of the clock is otherwise very
% large (time is counted from epoch).  For consistency, we call
% \cs{sys_timer_gzero:} in all engines.
%    \begin{macrocode}
\sys_if_timer_exist:T { \sys_timer_gzero: }
%    \end{macrocode}
%
% \subsection{Benchmarking code}
%
%    \begin{macrocode}
%<@@=benchmark>
%    \end{macrocode}
%
% \begin{variable}{\g_benchmark_time_fp, \g_benchmark_ops_fp}
%   Functions such as \cs{benchmark:n} store the measured time in
%   \cs{g_benchmark_time_fp} (in seconds) and the estimated number of
%   operations in \cs{g_benchmark_ops_fp}.
%    \begin{macrocode}
\fp_new:N \g_benchmark_time_fp
\fp_new:N \g_benchmark_ops_fp
%    \end{macrocode}
% \end{variable}
%
% \begin{macro}{\benchmark_display:, \benchmark_display_in_ops:}
%   Function to display the time that was measured or the estimated
%   number of operations.  This can be redefined by the user.
%    \begin{macrocode}
\cs_new_protected:Npn \benchmark_display:
  { \iow_term:x { \fp_to_tl:N \g_benchmark_time_fp ~ seconds } }
\cs_new_protected:Npn \benchmark_display_in_ops:
  {
    \iow_term:x
      {
        \fp_compare:nTF { \g_benchmark_ops_fp > 1e9 }
          { \fp_to_tl:n { \g_benchmark_ops_fp * 1e-9 } ~ G~ops }
          {
            \fp_compare:nTF { \g_benchmark_ops_fp > 1e6 }
              { \fp_to_tl:n { \g_benchmark_ops_fp * 1e-6 } ~ M~ops }
              {
                \fp_compare:nTF { \g_benchmark_ops_fp > 1e3 }
                  { \fp_to_tl:n { \g_benchmark_ops_fp * 1e-3 } ~ K~ops }
                  { \fp_to_tl:n { \g_benchmark_ops_fp } ~ ops }
              }
          }
      }
  }
%    \end{macrocode}
% \end{macro}
%
%    \begin{macrocode}
\sys_if_timer_exist:F
  {
    \msg_new:nnnn { benchmark } { no-time }
      { The~l3benchmark~package~failed~to~access~a~clock. }
      {
        The~current~engine~provides~no~way~to~access~the~system~time,~
        hence~making~benchmarking~impossible~without~shell-escape.~
        Please~use~pdfTeX,~LuaTeX,~or~call~other~engines~with~the~
        --shell-escape~option.
      }
    \cs_new_protected:Npn \benchmark_once:n #1
      { \msg_error:nn { benchmark } { no-time } }
    \cs_new_protected:Npn \benchmark_once_in_ops:n #1
      { \msg_error:nn { benchmark } { no-time } }
    \cs_new_protected:Npn \benchmark_once_silent:n #1
      { \msg_error:nn { benchmark } { no-time } }
    \cs_new_protected:Npn \benchmark:n #1
      { \msg_error:nn { benchmark } { no-time } }
    \cs_new_protected:Npn \benchmark_in_ops:n #1
      { \msg_error:nn { benchmark } { no-time } }
    \cs_new_protected:Npn \benchmark_silent:n #1
      { \msg_error:nn { benchmark } { no-time } }
    \fp_gset:Nn \g_benchmark_time_fp { nan }
    \fp_gset:Nn \g_benchmark_ops_fp { nan }
    \msg_critical:nn { benchmark } { no-time }
  }
%    \end{macrocode}
%
% \begin{variable}{\g_benchmark_duration_target_fp, \g_@@_duration_int}
%   The benchmark is constrained to take roughly (from half to twice)
%   \cs{g_benchmark_duration_target_fp} seconds, more precisely
%   \cs{g_@@_duration_int} scaled seconds, unless one iteration of the
%   code takes longer.
%    \begin{macrocode}
\fp_new:N \g_benchmark_duration_target_fp
\fp_gset:Nn \g_benchmark_duration_target_fp { 1 }
\int_new:N \g_@@_duration_int
%    \end{macrocode}
% \end{variable}
% 
% \begin{variable}{\g_@@_time_int, \g_@@_time_a_int, \g_@@_time_b_int, \g_@@_time_c_int, \g_@@_time_d_int, \g_@@_time_median_int}
%   These variables hold the time for running a piece of code, as an
%   integer in scaled seconds.
%    \begin{macrocode}
\int_new:N \g_@@_time_int
\int_new:N \g_@@_time_a_int
\int_new:N \g_@@_time_b_int
\int_new:N \g_@@_time_c_int
\int_new:N \g_@@_time_d_int
\int_new:N \g_@@_time_median_int
%    \end{macrocode}
% \end{variable}
%
% \begin{variable}{\g_@@_repeat_int}
%   Holds the number of times that the piece of code was
%   repeated when timing.
%    \begin{macrocode}
\int_new:N \g_@@_repeat_int
%    \end{macrocode}
% \end{variable}
%
% \begin{variable}{\g_@@_nesting_int}
% \begin{macro}{\@@_raw:nN, \@@_raw_aux:N, \@@_raw_end:nN}
%   Store in the given integer variable the time it took to perform a given
%   piece of code, in scaled seconds.  We call \cs{sys_timer_apply:N} as
%   close before and after the code as possible.  We store the
%   intermediate result in a new integer when \cs{@@_raw:nN} is
%   nested.
%    \begin{macrocode}
\int_new:N \g_@@_nesting_int
\cs_new_protected:Npn \@@_raw:nN #1
  {
    \int_gincr:N \g_@@_nesting_int
    \exp_args:Nc \@@_raw_aux:N
      { g_@@_ \int_use:N \g_@@_nesting_int _int }
    \sys_timer_apply:N \@@_raw_aux:n
    #1
    \sys_timer_apply:N \@@_raw_end:nN
  }
\cs_new_protected:Npn \@@_raw_aux:N #1
  {
    \int_gzero_new:N #1
    \cs_gset_protected:Npn \@@_raw_aux:n { \int_gset:Nn #1 }
  }
\cs_new_protected:Npn \@@_raw_end:nN #1#2
  {
    \int_gset:Nn #2
      {
        #1 -
        \int_use:c { g_@@_ \int_use:N \g_@@_nesting_int _int }
      }
    \int_gdecr:N \g_@@_nesting_int
  }
%    \end{macrocode}
% \end{macro}
% \end{variable}
%
% \begin{macro}{\@@_raw_replicate:nnN, \@@_tmp:w}
%   Here, we wish to measure the time it takes for the piece of code
%   |#2| to be run |#1| times, and store the result in the
%   integer~|#3|.
%
%   If the number of copies required is large (here ${}>1024$), it may
%   exhaust \TeX{}'s main memory. In that case, we replicate $1024$
%   times less the code |\@@_replicate_kibi_fold:n {#1}|.  Of course the
%   division by $1024$ rounds to an integer, so that step introduces a
%   relative error of order $1/1000$, much less than many other sources
%   of variability.
%
%   We subtract the time for another call to \cs{@@_tmp:w}, with the
%   same arguments (to capture the time it takes to read the argument)
%   but empty expansion.
%    \begin{macrocode}
\cs_new_eq:NN \@@_tmp:w ?
\cs_new_protected:Npn \@@_raw_replicate:nnN #1
  {
    \int_compare:nNnTF {#1} > { 1024 }
      { \@@_raw_replicate_large:nnN {#1} }
      { \@@_raw_replicate_small:nnN {#1} }
  }
\cs_new_protected:Npn \@@_raw_replicate_large:nnN #1#2
  {
    \exp_args:Nno \@@_raw_replicate:nnN { #1 / 1024 }
      { \@@_replicate_kibi_fold:n {#2} }
  }
\cs_new_protected:Npn \@@_raw_replicate_small:nnN #1#2
  {
    \cs_set:Npx \@@_tmp:w ##1##2 { \prg_replicate:nn {#1} {##1} }
    \@@_raw:nN { \@@_tmp:w {#2} { } } \g_@@_time_int
    \exp_args:No \@@_raw_replicate_aux:nnN
      { \int_use:N \g_@@_time_int } {#2}
  }
\cs_new_protected:Npn \@@_raw_replicate_aux:nnN #1#2#3
  {
    \@@_raw:nN { \@@_tmp:w { } {#2} } \g_@@_time_int
    \int_gset:Nn #3 { #1 - \g_@@_time_int }
  }
\cs_new:Npx \@@_replicate_kibi_fold:n #1
  { \prg_replicate:nn {1024} {#1} }
%    \end{macrocode}
% \end{macro}
%
% \begin{macro}{\@@_set_median:, \@@_set_median:NNNN}
%    \begin{macrocode}
\cs_new_protected:Npn \@@_set_median:
  {
    \@@_set_median:NNNN \g_@@_time_a_int \g_@@_time_b_int
      \g_@@_time_c_int \g_@@_time_d_int
  }
\cs_new_protected:Npn \@@_set_median:NNNN #1#2#3#4
  {
    \int_gset:Nn \g_@@_time_median_int
      {
        ( #1 + #2 + #3 + #4
        - \int_min:nn { \int_min:nn #1 #2 } { \int_min:nn #3 #4 }
        - \int_max:nn { \int_max:nn #1 #2 } { \int_max:nn #3 #4 } ) / 2
      }
  }
%    \end{macrocode}
% \end{macro}
%
% \begin{macro}{\benchmark:n, \benchmark_silent:n}
%   The main timing function. First time the user code once.  If that
%   took more than a third of a second we're done.  If that took much
%   less than a second, quadruple the number of copies until it takes a
%   reasonable amount of time (this is to avoid division by a possibly
%   zero time).  Once we reach a reasonable time, compute a number of
%   times that can fit in one quarter of a second and measure that four
%   times.  To save time we reuse the result of the first pass if
%   \cs{g_@@_repeat_int} is one.  Once we have four results, take their
%   median and display that, divided by $65536$ and by the number of
%   repetitions.
%    \begin{macrocode}
\fp_new:N \g_@@_one_op_fp
\tl_new:N \g_@@_code_tl
\cs_new_protected:Npn \benchmark:n #1
  { \benchmark_silent:n {#1} \benchmark_display: }
\cs_new_protected:Npn \benchmark_in_ops:n #1
  { \benchmark_silent:n {#1} \benchmark_display_in_ops: }
\cs_new_protected:Npn \@@_measure_op:
  {
    \int_gset:Nn \g_@@_duration_int { 256 }
    \tl_gset:Nn \g_@@_code_tl
      { \int_gset:Nn \g_@@_duration_int { 256 } } % arbitrary single operation
    \@@_aux:
    \fp_gset_eq:NN \g_@@_one_op_fp \g_benchmark_time_fp
  }
\cs_new_protected:Npn \benchmark_silent:n #1
  {
    \@@_measure_op:
    \int_gset:Nn \g_@@_duration_int
      { \fp_to_int:n { 65536 * \g_benchmark_duration_target_fp } }
    \tl_gset:Nn \g_@@_code_tl {#1}
    \@@_aux:
    \fp_gset:Nn \g_benchmark_ops_fp
      { \g_benchmark_time_fp / \g_@@_one_op_fp }
  }
\cs_new_protected:Npn \@@_aux:
  {
    \int_gset:Nn \g_@@_repeat_int { 1 }
    \@@_raw:nN { \g_@@_code_tl } \g_@@_time_int
    \int_compare:nNnF \g_@@_time_int < { \g_@@_duration_int / 2 }
      { \prg_break: }
    \int_while_do:nNnn \g_@@_time_int < { \g_@@_duration_int / 100 }
      {
        \int_compare:nNnT \g_@@_repeat_int > { \c_max_int / 4 }
          {
            \int_gset:Nn \g_@@_time_median_int { 0 }
            \prg_break:
          }
        \int_gset:Nn \g_@@_repeat_int { 4 * \g_@@_repeat_int }
        \@@_run:N \g_@@_time_int
      }
    \int_gset:Nn \g_@@_repeat_int
      {
        \int_max:nn { 1 }
          {
            \g_@@_duration_int * \g_@@_repeat_int
            / ( \g_@@_time_int * 4 )
          }
      }
    \int_compare:nNnTF \g_@@_repeat_int = 1
      { \int_gset_eq:NN \g_@@_time_a_int \g_@@_time_int }
      { \@@_run:N \g_@@_time_a_int }
    \@@_run:N \g_@@_time_b_int
    \@@_run:N \g_@@_time_c_int
    \@@_run:N \g_@@_time_d_int
    \@@_set_median:
    \prg_break_point:
    \fp_gset:Nn \g_benchmark_time_fp
      { \g_@@_time_median_int / \g_@@_repeat_int / 65536 }
  }
\cs_new_protected:Npn \@@_run:N
  { \exp_args:NNo \@@_raw_replicate:nnN \g_@@_repeat_int { \g_@@_code_tl } }
%    \end{macrocode}
% \end{macro}
%
% \begin{macro}{\benchmark_once:n, \benchmark_once_in_ops:n, \benchmark_once_silent:n}
%   Convert from scaled seconds to seconds.
%    \begin{macrocode}
\cs_new_protected:Npn \benchmark_once:n #1
  { \benchmark_once_silent:n {#1} \benchmark_display: }
\cs_new_protected:Npn \benchmark_once_in_ops:n #1
  { \benchmark_once_silent:n {#1} \benchmark_display_in_ops: }
\cs_new_protected:Npn \benchmark_once_silent:n #1
  {
    \@@_measure_op:
    \@@_raw:nN {#1} \g_@@_time_int
    \fp_gset:Nn \g_benchmark_time_fp { \g_@@_time_int / 65536 }
    \fp_gset:Nn \g_benchmark_ops_fp
      { \g_benchmark_time_fp / \g_@@_one_op_fp }
  }
%    \end{macrocode}
% \end{macro}
%
% \subsection{Benchmark tic toc}
%
% \begin{variable}{\g_@@_tictoc_int, \g_@@_tictoc_seq}
%    \begin{macrocode}
\int_new:N \g_@@_tictoc_int
\seq_new:N \g_@@_tictoc_seq
%    \end{macrocode}
% \end{variable}
%
% \begin{macro}{\benchmark_tic:, \@@_tic:n}
%    \begin{macrocode}
\cs_new_protected:Npn \benchmark_tic:
  {
    \int_compare:nTF { \g_@@_tictoc_int == 0 }
      {
        \sys_timer_gzero:
        \@@_tic:n { 0 }
      }
      { \sys_timer_apply:N \@@_tic:n }
  }
\cs_new_protected:Npn \@@_tic:n #1
  {
    \seq_put_right:Nn \g_@@_tictoc_seq {#1}
    \int_gincr:N \g_@@_tictoc_int
    \iow_term:x
      {
        \prg_replicate:nn {\g_@@_tictoc_int} {---+} \space
        TIC
      }
  }
%    \end{macrocode}
% \end{macro}
%
% \begin{macro}{\benchmark_toc:, \@@_toc:n}
%    \begin{macrocode}
\cs_new:Npn \benchmark_toc:
  {
    \int_compare:nT { \g_@@_tictoc_int == 0 }
      { \msg_error:nn {benchmark} {toc-first} }
    \seq_pop_right:NN \g_@@_tictoc_seq \l_@@_tictoc_pop_tl
    \sys_timer_apply:N \@@_toc:n
  }
\cs_new_protected:Npn \@@_toc:n #1
  {
    \tl_set:Nx \l_@@_tictoc_tl
      { \fp_to_decimal:n { round( (#1 - \l_@@_tictoc_pop_tl) / 65536 , 3 )} }
    \iow_term:x
      {
        \prg_replicate:nn {\g_@@_tictoc_int} {---+} \c_space_tl
        TOC: \c_space_tl
        \l_@@_tictoc_tl \c_space_tl s
      }
    \int_gdecr:N \g_@@_tictoc_int
  }
\msg_new:nnn {benchmark} {toc-first}
  {
    \token_to_str:N \benchmark_toc: \space without~
    \token_to_str:N \benchmark_tic: \space !
  }
%    \end{macrocode}
% \end{macro}
%
%    \begin{macrocode}
%</initex|package>
%    \end{macrocode}
%
% \end{implementation}
%
% \PrintIndex
