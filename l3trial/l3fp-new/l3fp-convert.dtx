% \iffalse meta-comment
% 
%% File: l3fp-convert.dtx Copyright(C) 2011 The LaTeX3 Project
%%
%% It may be distributed and/or modified under the conditions of the
%% LaTeX Project Public License (LPPL), either version 1.3c of this
%% license or (at your option) any later version.  The latest version
%% of this license is in the file
%%
%%    http://www.latex-project.org/lppl.txt
%%
%% This file is part of the "l3trial bundle" (The Work in LPPL)
%% and all files in that bundle must be distributed together.
%%
%% The released version of this bundle is available from CTAN.
%%
%% -----------------------------------------------------------------------
%%
%% The development version of the bundle can be found at
%%
%%    http://www.latex-project.org/svnroot/experimental/trunk/
%%
%% for those people who are interested.
%%
%%%%%%%%%%%
%% NOTE: %%
%%%%%%%%%%%
%%
%%   Snapshots taken from the repository represent work in progress and may
%%   not work or may contain conflicting material!  We therefore ask
%%   people _not_ to put them into distributions, archives, etc. without
%%   prior consultation with the LaTeX Project Team.
%%
%% -----------------------------------------------------------------------
%%
%
%<*driver|package>
\RequirePackage{expl3,l3fp-aux}
\GetIdInfo$Id: l3fp-convert.dtx 0000 0000-00-00 00:00:00Z bruno $
  {L3 Experimental floating point conversion}
%</driver|package>
%<*driver>
\documentclass[full]{l3doc}
\begin{document}
  \DocInput{\jobname.dtx}
\end{document}
%</driver>
% \fi
% 
% \title{^^A
%   The \textsf{l3fp-convert} package\\ Floating point conversion^^A
%   \thanks{This file describes v\fileversion, last revised \filedate.}^^A
% }
%         
% \author{^^A
%  The \LaTeX3 Project\thanks
%    {^^A
%      E-mail:
%        \href{mailto:latex-team@latex-project.org}
%          {latex-team@latex-project.org}^^A
%    }^^A
% }
%
% \date{Released \filedate}
%
% \maketitle
%
% \begin{documentation}
%
%   The internal representation of floating point numbers is quite
%   untypable, and we provide here the tools to convert from a more
%   user-friendly representation to internal floating point numbers,
%   and for various other conversions. Every floating point operation
%   calls those functions to normalize the input, so they must be
%   optimized.
%
%   \section{Internal representation}
%
%   Internally, a floating point number \meta{X} is a
%   protected token list containing
%   \begin{quote}
%     \cs{s_fp} \cs{fp_use:w} \meta{case} \meta{sign} \meta{body} |;|
%   \end{quote}
%   Let us explain each piece separately.
%
%   Internal floating point numbers will be used in expressions,
%   and in this context will be subject to f-expansion. They must
%   leave a recognizable mark after f-expansion, to prevent the
%   floating point number from being re-parsed. Thus, \cs{s_fp}
%   is simply another name for \cs{tex_relax:D}.
%
%   Since floating point numbers are always accessed by the various
%   operations using f-expansion, we can safely let them be protected:
%   x-expansion will then leave them untouched. However, when typeset,
%   \cs{s_fp} will do nothing, and \cs{fp_use:w} will take care of
%   typesetting the number, with whichever typesetting setting is
%   currently active.
%
%   The (decimal part of the) IEEE-754-2008 standard requires the
%   format to be able to represent special floating point numbers
%   besides the usual positive and negative cases. The various
%   possibilities will be distinguished by their \meta{case}, which
%   is a single digit:\footnote{Bruno: I need to implement subnormal
%     numbers. Also, quiet and signalling NaN must be better
%     distinguished.}
%   \begin{itemize}
%   \item[0] zeros: |+0| and |-0|,
%   \item[1] \enquote{normal} numbers (positive and negative),
%   \item[2] infinities: |+inf| and |-inf|,
%   \item[3] quiet and signalling |NaN|.
%   \end{itemize}
%   The \meta{sign} is |0| (positive) or |2| (negative),
%   except in the case of |NaN|, which have $\meta{sign} = 1$.
%   This ensures that changing the \meta{sign} digit to $2-\meta{sign}$
%   is exactly equivalent to changing the sign of the number.
%
%   Special floating point numbers have the form
%   \begin{quote}
%     \cs{s_fp} \cs{fp_use:w} \meta{case} \meta{sign} \cs{fp_info:...} |;|
%   \end{quote}
%   where \cs{fp_info:...} is a control sequence let to
%   \cs{tex_relax:D}, carrying an information string
%   \enquote{\texttt{...}} which describes where the
%   number was formed (useful for debugging).
%
%   Normal floating point numbers ($\meta{case} = 1$) have the form
%   \begin{quote}
%     \cs{s_fp} \cs{fp_use:w} 1 \meta{sign} \Arg{exponent}
%     \Arg{X1} \Arg{X2} \Arg{X3} \Arg{X4} |;|
%   \end{quote}
%   Here, the \meta{exponent} is an integer in the range
%   [\cs{c_fp_min_exponent_int},\cs{c_fp_max_exponent_int}].
%   The body consists in four blocks of exactly $4$ digits,
%   $ 0000 \leq \meta{X$\sb{i}$} \leq 9999 $, such that
%   \[
%   \meta{X}
%   = (-1)^{\meta{sign}} 10^{-\meta{exponent}}
%   \sum_{i=1}^{4} \meta{X$\sb{i}$} 10^{-4i}
%   \]
%   and such that the \meta{exponent} is minimal. This implies
%   $ 1000 \leq \meta{X1} \leq 9999 $.
%
%   \begin{table}\centering
%     \caption{Internal representation of floating point numbers.}
%     \label{tab:fp-convert-special}
%     \begin{tabular}{ll}
%       \toprule
%       \multicolumn{1}{c}{Representation} & Meaning \\
%       \midrule
%       0 0 \cs{fp_info:...}  \texttt{;} & Positive zero.      \\
%       0 2 \cs{fp_info:...}  \texttt{;} & Negative zero.      \\
%       1 0 \Arg{exponent} \Arg{X1} \Arg{X2} \Arg{X3} \Arg{X4} \texttt{;}
%                                        & Positive floating point. \\
%       1 2 \Arg{exponent} \Arg{X1} \Arg{X2} \Arg{X3} \Arg{X4} \texttt{;}
%                                        & Negative floating point. \\
%       2 0 \cs{fp_info:...}  \texttt{;} & Positive infinity.  \\
%       2 2 \cs{fp_info:...}  \texttt{;} & Negative infinity.  \\
%       3 1 \cs{fp_info:q...} \texttt{;} & Quiet |NaN|.        \\
%       3 1 \cs{fp_info:s...} \texttt{;} & Signalling |NaN|.   \\
%       \bottomrule
%     \end{tabular}
%   \end{table}
%
%   \section{Conversion functions}
%
%   \begin{function}[EXP]{\fp_convert_from_str:n}
%     \begin{syntax}
%       \cs{fp_convert_from_str:n} \Arg{token list}
%     \end{syntax}
%     Reads the \meta{token list}, and expands to the internal
%     representation of the corresponding floating point number.
%     The \meta{token list} may be an internal floating point
%     number, or any reasonable input.\footnote{Bruno: clarify.}
%     This function is f-expandable.
%   \end{function}
%
%   \begin{function}[EXP]{\fp_convert_from_old:N,\fp_convert_from_old:w}
%     \begin{syntax}
%       \cs{fp_convert_from_old:N} \meta{old fp var}
%       \cs{fp_convert_from_old:w} \meta{old floating point} |;|
%     \end{syntax}
%     Converts the \meta{old fp var} to the new floating
%     point format. Leaves the result in the input stream.
%     These functions are f-expandable. Note the trailing
%     semicolon (with catcode other) in the |w| case.
%   \end{function}
%
%   \begin{function}[EXP]{\fp_convert_to_str:N,\fp_convert_to_str:w}
%     \begin{syntax}
%       \cs{fp_convert_to_str:N} \meta{fp var}
%       \cs{fp_convert_to_str:w} \meta{floating point}
%     \end{syntax}
%     These functions f-expand to a string representation of the
%     internal floating point number, in the form
%     \texttt{0.}\meta{16 digits}\texttt{e}\meta{exponent}.
%     Note that the \meta{floating point} is naturally delimited
%     by a semicolon.
%   \end{function}
%
%   \begin{function}[EXP]{\fp_convert_to_old:n,
%       \fp_convert_to_old:N,\fp_convert_to_old:w}
%     \begin{syntax}
%       \cs{fp_convert_to_old:n} \Arg{token list}
%       \cs{fp_convert_to_old:N} \meta{fp var}
%       \cs{fp_convert_to_old:w} \meta{floating point}
%     \end{syntax}
%     These functions convert an internal floating point number to the old
%     format (\meta{sign}\meta{digit}.\meta{9~digits}e\meta{exponent}).
%     Note that since the new internal floating point number format includes
%     a trailing semicolon, it is not needed for \cs{fp_convert_to_old:w}.
%
%     This should probably be made into some generic rounding procedure.
%   \end{function}
%
% \end{documentation}
% 
% \begin{implementation}
%
% \section{\texttt{l3fp-convert} implementation}
% 
% We start by ensuring that the required packages are loaded.
%    \begin{macrocode}
%<*package>
\ProvidesExplPackage
  {\filename}{\filedate}{\fileversion}{\filedescription}
%</package>
%<*initex|package>
%    \end{macrocode}
%
% \subsection{Conversion to string}
%
% It is easy to convert an internal floating point to a string:
% no rounding, and the format is very well defined. We simply
% have to filter out the various special cases.
%
% \begin{macro}[EXP]{
%     \fp_convert_to_str:n,
%     \fp_convert_to_str:N,
%     \fp_convert_to_str:w}
% \begin{macro}[aux,EXP]{
%     \fp_convert_to_str_zero:Nw,
%     \fp_convert_to_str_normal:Nnnnnn,
%     \fp_convert_to_str_inf:Nw,
%     \fp_convert_to_str_nan:w}
%   \begin{syntax}
%     \cs{fp_convert_to_str:w}
%     ~~\cs{s_fp} \cs{fp_use:w} \meta{case} \meta{sign} \meta{body} |;|
%   \end{syntax}
%   The expanding versions must leave their argument unbraced.
%    \begin{macrocode}
\cs_new:Npn \fp_convert_to_str:n
  { \exp_last_unbraced:Nf \fp_convert_to_str:w }
\cs_new:Npn \fp_convert_to_str:N
  { \exp_after:wN \fp_convert_to_str:w }
\cs_new:Npn \fp_convert_to_str:w \s_fp \fp_use:w #1
  {
    \if_case:w #1 \exp_stop_f:
         \exp_after:wN \fp_convert_to_str_zero:Nw
    \or: \exp_after:wN \fp_convert_to_str_normal:Nnnnnn
    \or: \exp_after:wN \fp_convert_to_str_inf:Nw
    \or: \exp_after:wN \fp_convert_to_str_nan:w
    \fi:
  }
\cs_new:Npn \fp_convert_to_str_zero:Nw #1 #2;
  {
    \if:w 0 #1
      \exp_after:wN +
    \else:
      \exp_after:wN -
    \fi:
    0
  }
\cs_new:Npn \fp_convert_to_str_inf:Nw #1 #2;
  {
    \if:w 0 #1
      \exp_after:wN +
    \else:
      \exp_after:wN -
    \fi:
    inf
  }
\cs_new:Npn \fp_convert_to_str_nan:w #1; { nan }
\cs_new:Npn \fp_convert_to_str_normal:Nnnnnn #1 #2 #3#4#5#6 ;
  {
    \if:w 0 #1
      \exp_after:wN +
    \else:
      \exp_after:wN -
    \fi:
    0 . #3 #4 #5 #6 e #2
  }
%    \end{macrocode}
% \end{macro}
% \end{macro}
%
% \begin{macro}{\fp_use:N,\fp_use:n}
%    \begin{macrocode}
\cs_new_eq:NN \fp_convert_use:N \fp_convert_to_str:N
\cs_new:Npn \fp_convert_use:n #1
  { \fp_convert_to_str:n { \fp_convert_from_str:n {#1} } }
%    \end{macrocode}
% \end{macro}
%
% \subsection{Conversion from old floating points}
%
% To ease the transition, we provide a conversion from the old
% floating point implementation to the new one. The input is a token
% list of the form \meta{sign} \meta{1~digit}|.|\meta{9~digits}
% |e|\meta{exponent}.
%
% \begin{macro}[EXP]{
%     \fp_convert_from_old:N,
%     \fp_convert_from_old:w
%   }
% \begin{macro}[EXP,aux]{
%     \fp_convert_from_old_special:Nw,
%     \fp_convert_from_old_normal:NnNNNNww,
%     \fp_convert_from_old_sign:N
%   }
%
%   The old module only has one special case, \cs{c_undefined_fp},
%   distinguished by its sign. It is converted to an \enquote{empty}
%   \texttt{qNaN}. All other numbers are converted with no
%   precision loss, just by adding zero at the end. The exponent
%   is shifted by $1$ (because the mantissa of new floating point
%   numbers starts with $0.\cdots$). For instance,
%   \texttt{+1.234512345e7} is stored as \cs{s_fp} \cs{fp_use:w}
%   |10{8}{1234}{5123}{4500}{0000};|.
%
%   \begin{macrocode}
\cs_new_nopar:Npn \fp_convert_from_old:N #1
  { \exp_after:wN \fp_convert_from_old:w #1 ; }
\cs_new_nopar:Npn \fp_convert_from_old:w #1 #2.#3#4#5
  {
    \if_meaning:w 0 #2
      \exp_after:wN \fp_convert_from_old_special:Nw
    \else:
      \exp_after:wN \fp_convert_from_old_normal:NnNNNNww
    \fi:
    #1 {#2#3#4#5}
  }
\cs_new_nopar:Npn \fp_convert_from_old_special:Nw #1 #2 ;
  {
    \if:w X #1
      \exp_after:wN \c_empty_qnan_fp
    \else:
      \exp_after:wN \c_plus_zero_fp
    \fi:
  }
\cs_new_nopar:Npn \fp_convert_from_old_normal:NnNNNNww
    #1 #2 #3#4#5#6 #7 e #8 ;
  {
    \fp_convert_from_old_sign:N #1
    { \int_use:N \int_eval:w #8 + 1 }
    {#2} {#3#4#5#6} {#700} {0000} ;
  }
\cs_new_nopar:Npn \fp_convert_from_old_sign:N #1
  {
    \exp_after:wN \s_fp
    \exp_after:wN \fp_use:w
    \int_value:w 1 \if_meaning:w + #1 0 \else: 2 \fi: \exp_after:wN
  }
%    \end{macrocode}
% \end{macro}
% \end{macro}
%
%
% \subsection{Conversion to old floating points}
%
% Converting from the internal representation to the old format for fp
% is slightly harder, as it involves rounding.
%
% \begin{macro}[EXP]{
%     \fp_convert_to_old:n,
%     \fp_convert_to_old:N,
%     \fp_convert_to_old:w
%   }
% \begin{macro}[aux,EXP]{
%     \fp_convert_to_old_zero:w,
%     \fp_convert_to_old_undefined:w
%   }
% \begin{macro}[aux,EXP]{
%     \fp_convert_to_old_normal:Nnnn,
%     \fp_convert_to_old_aux_i:NNN,
%     \fp_convert_to_old_aux_ii:N,
%     \fp_convert_to_old_aux_iii:NN,
%     \fp_convert_to_old_carry:w
%     \fp_convert_to_old_sanitize:NNw,
%   }
%   \begin{syntax}
%     \cs{fp_convert_to_old:w}
%     ~~\cs{s_fp} \cs{fp_use:w} \meta{case} \meta{sign} \meta{body} |;|
%   \end{syntax}
%   The expanding versions must leave their argument unbraced.
%    \begin{macrocode}
\cs_new:Npn \fp_convert_to_old:n
    { \exp_last_unbraced:Nf \fp_convert_to_old:w }
\cs_new:Npn \fp_convert_to_old:N
    { \exp_after:wN \fp_convert_to_old:w }
%    \end{macrocode}
%
%   The conversion then starts by examining the \meta{case}, and
%   branching accordingly. Infinities and NaN are treated identically.
%    \begin{macrocode}
\cs_new:Npn \fp_convert_to_old:w \s_fp \fp_use:w #1
  {
    \if_case:w #1 \exp_stop_f:
         \exp_after:wN \fp_convert_to_old_zero:w
    \or: \exp_after:wN \fp_convert_to_old_normal:Nnnn
    \else: \exp_after:wN \fp_convert_to_old_undefined:w
    \fi:
  }
\cs_new:Npn \fp_convert_to_old_zero:w #1; { \c_zero_old_fp }
\cs_new:Npn \fp_convert_to_old_undefined:w #1; { \c_undefined_old_fp }
%    \end{macrocode}
%
%   In the case of normal numbers ($\meta{case}=1$), we do the
%   rounding within some \cs{int_eval:w}. Afterwards,
%   check for a carry. This only ever happens to yield $1.0\cdots0$
%   as a result, so \cs{fp_convert_to_old_carry:w} knows
%   precisely what follows.
%
%   The whole thing is wrapped in a calculation of the exponent.
%   This will be ended by $-1$ (unless there is a carry). At the end,
%   \emph{i.e.}, after the carry, \cs{fp_convert_to_old_sanitize:NNw}
%   tests for overflow or underflow, and reorders the various pieces.
%    \begin{macrocode}
\cs_new:Npn \fp_convert_to_old_normal:Nnnn #1 #2 #3#4
  {
    \exp_after:wN \fp_convert_to_old_sanitize:NNw
    \exp_after:wN #1
    \int_use:N \int_eval:w #2
      \exp_after:wN \fp_convert_to_old_aux_iii:NN
      \int_use:N \int_eval:w 1#3
        \exp_after:wN \fp_convert_to_old_aux_ii:N
        \int_use:N \int_eval:w
          1 #4 \fp_convert_to_old_aux_i:NNN #1
  }
\cs_new:Npn \fp_convert_to_old_aux_i:NNN #1 #2#3
  { % <sign> <X9> <X10> <X11-X16> ;
    #2 #3
    + \fp_aux_round_s:NNNw #1#3
  }
\cs_new:Npn \fp_convert_to_old_aux_ii:N #1
  { % <1 or 2> <X'6-10> ;
    \if_meaning:w 2 #1 + 1 \fi: \int_eval_end:
  }
\cs_new:Npn \fp_convert_to_old_aux_iii:NN #1 #2
  { % <1 or 2> <X'1>
    \if_meaning:w 2 #1
      \exp_after:wN \fp_convert_to_old_carry:w
    \fi:
    - 1 ;
    #2.
  }
\cs_new:Npn \fp_convert_to_old_carry:w - 1 ; 0.000000000 ;
  { ; 1.000000000 ; }
%    \end{macrocode}
%   If (the absolute value of) the exponent is less than $100$,
%   then the number fits in the \enquote{old} range of exponents:
%   simply set the sign correctly. Underflow gives \cs{c_zero_old_fp}
%   and  overflow gives \cs{c_undefined_old_fp}.
%    \begin{macrocode}
\cs_new:Npn \fp_convert_to_old_sanitize:NNw #1 #2#3 ;
  { % <sign digit> <exponent> ; <body> ;
    \if_num:w \if:w - #2 - \fi: #2#3 < \c_one_hundred
      \exp_after:wN \fp_convert_to_old_sane:Nww
    \else:
      \exp_after:wN \fp_convert_to_old_unsane:NNww
    \fi:
    #1 #2#3 ;
  }
\cs_new:Npn \fp_convert_to_old_sane:Nww #1 #2; #3;
  { \if:w 0 #1 \exp_after:wN + \else: \exp_after:wN - \fi: #3 e #2 }
\cs_new:Npn \fp_convert_to_old_unsane:NNww #1 #2#3; #4;
  {
    \if:w - #2
      \exp_after:wN \c_zero_old_fp
    \else:
      \exp_after:wN \c_undefined_old_fp
    \fi:
  }
%    \end{macrocode}
% \end{macro}
% \end{macro}
% \end{macro}
%
% 
%
%
% \subsection{Conversion from string}
%
% In this section, we write some code to convert user input to
% an internal floating point. The following forms are accepted:
% \begin{itemize}
% \item \meta{signs} \meta{floating point}
% \item \meta{signs} \meta{integer} |.| \meta{decimal} |e| \meta{exponent}
% \end{itemize}
% In both cases, \meta{signs} is a (possibly empty) string of
% |+| and |-| (with any category code\footnote{Bruno: except
%   1, 2, 4, 10, 13, and those which cannot be tokens (0, 5, 9),
%   so really, just 3, 6, 7, 8, 11, 12.}).\footnote{Bruno:
%   test (and implement) non-other digits.}
%
% In the second form, the \meta{integer} is a sequence of digits,
% whose length is not limited by constraints \TeX{}'s integer
% registers. It stops at the first non-digit character. The
% \meta{decimal} part is formed by all digits from the dot
% (if it exists) until the first non-digit character. The
% \meta{exponent} part has the form \meta{exponent sign}
% \meta{exponent body}, where \meta{exponent sign} is any string
% of |+| or |-|, and \meta{exponent body} is a string of digits,
% stopping, as usual, at the first non-digit.
%
% The code attempts to match the longest possible string. If some
% tokens remain, no error is raised.\footnote{Bruno: this \emph{must}
%   be fixed, and \texttt{sNaN} be returned!}
%
% Any missing part will take the appropriate default value.
% \begin{itemize}
% \item A missing \meta{exponent} is considered to be zero.
% \item A number with no dot has zero decimal part.
% \item An empty \meta{integer} part or decimal part is zero.
% \end{itemize}
%
% Border cases:
% \begin{itemize}
% \item \texttt{e1} is considered as invalid input, and gives
%   \texttt{qNaN}.\footnote{Bruno: now just gives an error.}
%   This will be important once parsing expressions is
%   implemented, since \texttt{e-1} would be ambiguous otherwise.
% \item \texttt{.e3} and \texttt{.} are zero.
% \end{itemize}
% 
% Bruno: expansion, not yet. Only f-expansion at the start, and
% unpacking of registers after signs.
%
%
% Work-plan.
% \begin{itemize}
% \item Remove any leading sign and build the \meta{sign} as we go.
%   If the next character is a letter, go to the \enquote{special}
%   branch, discussed later.
% \item Drop leading zeros.
% \item If the next character is a dot, drop some more zeros,
%   keeping track of how many were dropped after the dot.
%   Counting those gives $\meta{exp1}<0$. Then read the decimal part
%   with the \cs{fp_convert_from_str_small} functions.
% \item Otherwise, $\meta{exp1}=0$, and first read the integer part,
%   then the decimal part. This is implemented through the more
%   elaborate \cs{fp_convert_from_str_large} functions.
% \item Continuing in the same line of expansion, read the exponent
%   \meta{exp2}.
% \item Finally check that nothing is left.\footnote{Bruno: not done yet.}
% \end{itemize}
%
%
% 
% \begin{macro}{\fp_convert_from_str:n}
%   \begin{syntax}
%     \cs{fp_convert_from_str:n} \Arg{token list}
%   \end{syntax}
%   This f-expands to an internal floating point number.
%    \begin{macrocode}
\cs_new:Npn \fp_convert_from_str:n #1
  {
    \exp_after:wN \fp_convert_from_str_after:ww
    \tex_romannumeral:D -`0 \fp_cfs_sign:w #1 \s_fp_mark \s_fp_stop
  }
\cs_new:Npn \fp_convert_from_str_after:ww #1; #2\s_fp_mark \s_fp_stop
  {
    \tl_if_blank:nTF {#2}
      { #1; }
      {
        \exp_after:wN \fp_aux_snan_fp:N
        \cs:w fp_info: ~ '\tl_to_str:n {#2}'; \cs_end:
      }
  }
%    \end{macrocode}
% \end{macro}
%
% 
% \subsubsection{Signs}
%
% \begin{macro}{\fp_cfs_sign:w,
%     \fp_cfs_sign_loop:N,
%     \fp_cfs_sign_end:w}
%
%   Finding the sign is done with an \cs{int_eval:w} of the form
%   $1-\pm\cdots\pm 1$, which is $0$ if $\pm\cdots\pm = +$
%   (positive input) and $2$ for a negative input.
%   The conditionals are a bit tricky.
%   \begin{itemize}
%   \item If |#1| is |+|, the inner \cs{if:w} is false, but the second
%     |#1| makes the outer \cs{if:w} true, and we continue the loop.
%   \item If |#1| is |-|, the inner \cs{if:w} is true, and leaves |+|,
%     making the outer \cs{if:w} true as well. Thus, the second |#1|
%     (which is |-|) remains, changing the sign in the \cs{int_eval:w}
%     appropriately.
%   \item Otherwise, the inner and outer \cs{if:w} are false,
%     and the loop ends.
%   \end{itemize}
%   Note that \cs{token_to_str:N} ensures that a non-catcode-other
%   minus sign does not break anything.
%
%    \begin{macrocode}
\cs_new:Npn \fp_cfs_sign:w
  {
    \exp_after:wN \fp_cfs_sign_after:Nwn
    \int_use:N \int_eval:w 1 -
      \exp_after:wN \fp_cfs_sign_loop:N
      \tex_romannumeral:D \fp_aux_expand:w
  }
\cs_new:Npn \fp_cfs_sign_loop:N #1
  {
      \if:w + \if:w - #1 + \fi: \token_to_str:N #1
        \exp_after:wN \fp_cfs_sign_loop:N
        \tex_romannumeral:D
      \else:
        \fp_cfs_lead_test:w #1
      \fi:
      \fp_aux_expand:w
  }
%    \end{macrocode}
% \end{macro}
%
%
% \begin{macro}{\fp_cfs_sign_after:Nwn}
%   \begin{syntax}
%     \cs{fp_cfs_sign_after:Nwn} \meta{sign} E \meta{exp1} |;| \meta{function}
%   \end{syntax}
%   Puts \meta{function} back in front. This function will be
%   \begin{itemize}
%   \item \cs{fp_cfs_premature_end:w} when the sign is
%     immediately followed by the end of the number (gives a sNaN).
%   \item \cs{fp_cfs_internal:NNNw} when parsing an internal floating point,
%   \item \cs{fp_cfs_zero:Nw} when parsing a zero,
%   \item \cs{fp_cfs_small:NwN} when the mantissa is less than $1$,
%   \item \cs{fp_cfs_large:NwN} when the mantissa is greater than $1$.
%   \item \cs{fp_cfs_letter:NwN} when what follows the sign is a letter.
%   \end{itemize}
%   \footnote{Bruno: the \texttt{E} is useless here.}
%    \begin{macrocode}
\cs_new:Npn \fp_cfs_sign_after:Nwn #1 E #2 ; #3 { #3 #1 #2 ; }
%    \end{macrocode}
% \end{macro}
%
%
% \subsubsection{Filter special cases}
%
% \begin{macro}{\fp_cfs_lead_test:w}
%   Dirty trick to nest \TeX{}'s conditionals properly: remove
%   unwanted tokens in front of us, and put \cs{fi:} back.
%   That's probably quicker than the corresponding
%   \cs{exp_after:wN} construction.\footnote{Bruno: must test.}
%
%   Non-characters are either \cs{tex_relax:D} or registers
%   (everything else is invalid in the context of floating
%   point expressions, since we've f-expanded).
%   A leading character can be a digit, a dot, or a letter.
%   Letters need special treatment: used for \texttt{-+inf}
%   or \texttt{qNaN}.\footnote{Bruno: not implemented yet.}
%    \begin{macrocode}
\cs_new:Npn \fp_cfs_lead_test:w #1 \fi: \fp_aux_expand:w
  {
      \fi:
      \if_catcode:w \tex_relax:D #1
        \if_meaning:w \tex_relax:D #1
          \exp_after:wN \exp_after:wN \exp_after:wN
          \fp_cfs_lead_relax:N
        \else:
          \exp_after:wN \fp_cfs_sign_loop:N
          \tex_the:D
        \fi:
      \else:
        \c_one \exp_after:wN E
        \int_use:N \int_eval:w \c_zero
          \if_num:w 9 < 1 #1 \exp_stop_f:
            \exp_after:wN \exp_after:wN \exp_after:wN
            \fp_cfs_trim_loop:N
          \else:
            \exp_after:wN \exp_after:wN \exp_after:wN
            \fp_cfs_lead_test_dot:N
          \fi:
      \fi:
      #1
  }
\cs_new:Npn \fp_cfs_lead_test_dot:N #1
  {
      \if:w . #1
        \exp_after:wN \fp_cfs_strim_loop:N
        \tex_romannumeral:D
      \else:
        \fp_cfs_lead_letter:w #1
      \fi:
      \fp_aux_expand:w
  }
\cs_new:Npn \fp_cfs_lead_letter:w #1 \fi: \fp_aux_expand:w
  { \fi: ; \fp_cfs_letter:NwN #1 }
%    \end{macrocode}
% \end{macro}
%
% 
% \subsubsection{Sign followed by a letter}
%
% \begin{macro}{\fp_cfs_letter:NwN}
%   \begin{syntax}
%     \cs{fp_cfs_letter:NwN} \meta{sign} |0;| \meta{letter}
%   \end{syntax}
%   Here, \meta{sign} is |0| or |2| depending on what signs were
%   found before the leading letter. In fact, the name here is
%   misleading: we only know that |#2| is a character which is
%   neither a digit, nor a dot.
%    \begin{macrocode}
\cs_new:Npn \fp_cfs_letter:NwN #1 0; #2
  {
    \exp_after:wN \fp_cfs_letter_after:NNw
    \exp_after:wN #1
    \cs:w fp_cfs_
      \if_num:w \int_eval:w \tex_uccode:D `#2 / \c_twenty_six = \c_three
        letter: ~ #2
        \exp_after:wN \fp_aux_grab_letters:N
        \tex_romannumeral:D
      \else:
        non_letter: ~ #2 ;
        \exp_after:wN \cs_end:
        \tex_romannumeral:D
      \fi:
      \fp_aux_expand:w
  }
\cs_new:Npn \fp_cfs_letter_after:NNw #1 #2 % #3 \s_fp_mark \s_fp_stop
  {
    \if_meaning:w \tex_relax:D #2
      \exp_after:wN \use_i:nn
    \else:
      \exp_after:wN \use_ii:nn
    \fi:
    {
      \exp_after:wN \fp_aux_snan_fp:N
      \cs:w
        fp_info: ~ %
          \fp_aux_extract_info:N #2
      \cs_end:
    }
    { #2 #1 }
  }
%    \end{macrocode}
%   A few known special cases.\footnote{Bruno: the conversion should
%     not expand beyond its scope, contrarily to the binary operations.}
%    \begin{macrocode}
\cs_new:cpn { fp_cfs_letter: ~ nan ; } #1
  { \c_empty_qnan_fp }
\cs_new:cpn { fp_cfs_letter: ~ inf ; } #1
  { \fp_aux_inf_fp:N #1 }
 % \cs_new_eq:cc { fp_cfs_letter: ~ NaN ; }      { fp_cfs_letter:'nan' }
 % \cs_new_eq:cc { fp_cfs_letter: ~ Inf ; }      { fp_cfs_letter:'inf' }
\cs_new_eq:cc { fp_cfs_letter: ~ infty ; }    { fp_cfs_letter: ~ inf ; }
\cs_new_eq:cc { fp_cfs_letter: ~ infinity ; } { fp_cfs_letter: ~ inf ; }
%    \end{macrocode}
% \end{macro}
% 
% \subsubsection{Trimming leading zeros}
%
% \begin{macro}{\fp_cfs_trim_loop:N,\fp_cfs_trim_end:w}
%
%   In this branch, we expect a number with a non-empty mantissa
%   (between the sign part and the exponent part), starting with
%   a digit (those starting with a dot are handled by
%   \cs{fp_cfs_strim_loop:N}. The goal here is to trim zeros from
%   the beginning of the number, reaching the most significant
%   digit before starting to really read the number (this is required
%   to round properly).
%
%   There are two cases. If we find a non-zero digit before any dot
%   or exponent part, the mantissa is \enquote{large} ($\geq 1$).
%   If we find no non-zero digit before a dot, we need to continue
%   removing zeros afterwards (through the \cs{fp_cfs_strim} functions,
%   resulting in a \enquote{small} mantissa ($<1$). In the latter case,
%   we have to keep track of the shift $\meta{exp1}<0$ as we remove
%   zeros after the dot. This exponent is \cs{c_zero} for large
%   mantissas.
%
%    \begin{macrocode}
\cs_new:Npn \fp_cfs_trim_loop:N #1
  {
      \if:w 0 #1
        \exp_after:wN \fp_cfs_trim_loop:N
        \tex_romannumeral:D
      \else:
        \if:w . #1
          \exp_after:wN \fp_cfs_strim_loop:N
          \tex_romannumeral:D
        \else:
          \fp_cfs_trim_end:w #1
        \fi:
      \fi:
      \fp_aux_expand:w
  }
\cs_new:Npn \fp_cfs_trim_end:w #1 \fi: \fi: \fp_aux_expand:w
  {
        \fi:
      \fi:
      \if_num:w 9 < 1#1 \exp_stop_f:
        \exp_after:wN ; \exp_after:wN \fp_cfs_large:NwN
      \else:
        \exp_after:wN ; \exp_after:wN \fp_cfs_zero:Nw
      \fi:
      #1
  }
%    \end{macrocode}
% \end{macro}
%
%
% \begin{macro}{\fp_cfs_strim_loop:N,\fp_cfs_strim_end:w}
%   If we have removed all digits until a period (or if the body
%   started with a period), then enter the
%   \enquote{\texttt{small\_trim}} loop which outputs $-1$ for each
%   removed $0$.
%    \begin{macrocode}
\cs_new:Npn \fp_cfs_strim_loop:N #1
  {
      \if:w 0 #1
        - \c_one
        \exp_after:wN \fp_cfs_strim_loop:N
        \tex_romannumeral:D
      \else: 
        \fp_cfs_strim_end:w #1
      \fi:
      \fp_aux_expand:w
  }
%    \end{macrocode}
%
%   As we cannot round properly before the sign is known,
%   we need to take a rest when reaching a non-zero, stop the
%   \cs{int_eval:w} which is computing the \meta{exp1},
%   and simply leave instructions after the semicolon on what to
%   do next. This will be seen by \cs{fp_cfs_sign_after:Nwn}
%   and put in front of everything to continue parsing.
%
%
% %   and the best we can do at this stage is to read up to $16$
% %   digits, and use any remaining digit to produce the correct
% %   \meta{rounding} digit, as described for \cs{fp_aux_round:NNN}.
% %   This can be done at most $9$ digits at a time, so we need two
% %   steps.
% %
% %   If the number of digits in front of us is less than $16$,
% %   simply fill with more zeros. If there are more digits, start
% %   a loop to remove them, doing almost nothing (taking care of
% %   the \meta{rounding}).
%    \begin{macrocode}
\cs_new:Npn \fp_cfs_strim_end:w #1 \fi: \fp_aux_expand:w
  {
      \fi:
      \if_num:w 9 < 1#1 \exp_stop_f:
        \exp_after:wN ; \exp_after:wN \fp_cfs_small:NwN
      \else:
        \exp_after:wN ; \exp_after:wN \fp_cfs_zero:Nw
      \fi:
      #1
  }
%    \end{macrocode}
% \end{macro}
%
%
% \subsubsection{Special cases}
%
% \begin{macro}{\fp_cfs_lead_relax:N}
%   The argument is a token equal to \cs{tex_relax:D}.
%   This can be \cs{s_fp}, \cs{s_fp_mark}, or a badly
%   initialized register.\footnote{Bruno: this is not
%     caught yet.}
%    \begin{macrocode}
\cs_new:Npn \fp_cfs_lead_relax:N #1
  {
    \reverse_if:N \if_num:w \pdftex_strcmp:D {\s_fp} {#1} = \c_zero
      \fp_cfs_premature_end_i:w
    \fi:
    \fp_cfs_internal:w
    #1
  }
%    \end{macrocode}
%
%   In the case of an internal floating point number, |#1| is
%   \cs{s_fp}, and what follows looks as follows.
%   We terminate the \cs{int_eval:w} for the sign, and
%   put the tokens in the right place to fool
%   \cs{fp_cfs_sign_after:Nwn}.\footnote{Bruno: everywhere,
%     check whether the input was completely parsed (i.e.,
%     nothing left before \cs{s_fp_mark}).}
%   ^^A Need consistency check: |#4| must be blank.
%   ^^A When parsing expressions, we must pursue after the fp.
%   \begin{syntax}
%     \cs{fp_cfs_internal:w} \cs{s_fp} \cs{fp_use:w}
%     ~~\meta{case} \meta{sign} \meta{body} |;|
%     ~~\meta{remaining input} \cs{s_fp_mark} \cs{s_fp_stop}
%   \end{syntax}
%    \begin{macrocode}
\cs_new:Npn \fp_cfs_internal:w
    \s_fp \fp_use:w #1 #2 #3 ; %#4 \s_fp_mark \s_fp_stop
  {
    \int_eval:w 1 - #2 E #3 ; { \s_fp \fp_use:w #1 }
  }
%    \end{macrocode}
%
%   All other \cs{tex_relax:D} are invalid input, and we
%   produce a signalling \texttt{NaN}, whose \meta{info}
%   field is the remaining part of the input.
%    \begin{macrocode}
\cs_new:Npn \fp_cfs_premature_end_i:w \fi: \fp_cfs_internal:w
  { \fi: \c_one E 0 ; \fp_cfs_premature_end_ii:w }
\cs_new:Npn \fp_cfs_premature_end_ii:w #1 0; #2% \s_fp_mark \s_fp_stop
  {
    \exp_after:wN \fp_aux_snan_fp:N
    \cs:w
      fp_info:~ premature~ end~ '
      \if_meaning:w 0 #1 + \else: - \fi:
      \token_to_str:N #2' ;
    \cs_end:
    #2
  }
%    \end{macrocode}
% \end{macro}
%
% \begin{macro}{\fp_cfs_zero:Nw}
%   \begin{syntax}
%     \cs{fp_cfs_zero:Nw} \meta{sign} \meta{exp1} ; \meta{non-digit}
%   \end{syntax}
%    \begin{macrocode}
\cs_new:Npn \fp_cfs_zero:Nw #1 #2 ;
  {
    \exp_after:wN \fp_cfs_zero_aux:Nw
    \exp_after:wN #1
    \int_value:w \fp_cfs_exponent_test:N
  }
\cs_new:Npn \fp_cfs_zero_aux:Nw #1 #2 ; %\s_fp_mark \s_fp_stop
  { \fp_aux_zero_fp:N #1 }
%    \end{macrocode}
% \end{macro}
%
% 
% \subsubsection{Small mantissa}
%
% \begin{macro}{\fp_cfs_small:NwN}
%   \begin{syntax}
%     \cs{fp_cfs_small:NwN} \meta{sign} \meta{exp1} |;| \meta{non-zero digit}
%   \end{syntax}
%   The goal here is to grab up to $16$ digits. But we can't do
%   that all at once, because \cs{int_value:w} (which allows us
%   to collect digits and continue expanding) can only go up to
%   $9$ digits. We do it in two steps of (almost) $8$ digits.
%   Using $8$ rather than $9$ helps us repack more easily. In fact,
%   we've already read one digit, and there is no need to re-read it,
%   hence the use of \cs{fp_aux_grab_digits_vii:N}. As usual,
%   brace-removing f-expansion of the following. The goal is
%   \begin{syntax}
%     \cs{fp_cfs_sanitize:Nw} \meta{sign} \meta{exp} |;|
%     ~~\Arg{\4 digits} \Arg{\4 digits} \Arg{\4 digits} \Arg{\4 digits} |;|
%   \end{syntax}
%    \begin{macrocode}
\cs_new:Npn \fp_cfs_small:NwN #1 #2 ; #3
  {
    \exp_after:wN \fp_cfs_sanitize:Nw
    \exp_after:wN #1
    \int_use:N \int_eval:w #2
      \exp_after:wN \fp_cfs_i_after:NNNNNww
      \int_use:N \int_eval:w 1 
        \exp_after:wN \fp_cfs_small_i:NwwNN
        \exp_after:wN #1
        \int_value:w 1 #3
          \exp_after:wN \fp_aux_grab_digits_vii:N
          \tex_romannumeral:D \fp_aux_expand:w
  }
%    \end{macrocode}
%
%   The next function appears within the \cs{int_eval:w}
%   for the final exponent.
%   \begin{syntax}
%     \cs{fp_cfs_small_i:NwwNN} \meta{sign}
%     ~~|1| \meta{digits} |,| \meta{zeros} |;| \meta{number~of~zeros}
%   \end{syntax}
%   Here, \meta{digits}\meta{zeros} is a string of $8$ digits,
%   and \meta{number~of~zeros} counts how many zeros we had to
%   add to complete the digits that we had already found.
%
%   Note that the next token, |#5|, is already f-expanded in all
%   cases (by the \cs{fp_aux_grab_digits} functions which looked
%   for a digit). If that next token is a digit, then we need to
%   grab more digits, until the $15$th digit (the digit number
%   $16$ plays a special role in rounding). Otherwise, we check
%   for an exponent (\cs{fp_cfs_exponent_test:N/w} is used in
%   all \enquote{normal} branches), and end.
%    \begin{macrocode}
\cs_new:Npn \fp_cfs_small_i:NwwNN #1 1 #2, #3; #4 #5
  {
        #2 #3
        \if_num:w 9 < 1 #5 \exp_stop_f:
          \exp_after:wN \fp_cfs_small_ii:NwwNN
          \exp_after:wN #1
          \int_value:w 1 #5
            \exp_after:wN \fp_aux_grab_digits_vi:N
            \tex_romannumeral:D
        \else:
          \exp_after:wN \fp_cfs_small_i_end:
          \int_value:w
            \fp_cfs_exponent_test:w #5
        \fi:
        \fp_aux_expand:w
  }
%    \end{macrocode}
%   Ending involves filling up with zeros, and bringing the exponent
%   back in the input stream, for \cs{fp_cfs_i_after:NNNNNww}.
%   Since \cs{fp_cfs_i_after:NNNNNww} is common to both the case
%   where there are few digits, and the cases with many digits, there
%   is a check for the carry which may arise from rounding up.
%   This carry can only arise to give a power of ten (and |#1| is |0000|).
%   \footnote{Bruno: explain by going down through the expansion, and
%     then up.}
%    \begin{macrocode}
\cs_new:Npn \fp_cfs_small_i_end: #1 ; %\s_fp_mark \s_fp_stop
  { ; #1 ; {0000} {0000} ; }
\cs_new:Npn \fp_cfs_ii_after:NNNNNNww
    #1 #2 #3#4#5#6 #7; #8 ; %\s_fp_mark \s_fp_stop
  {
    \if:w 2 #2
      + 1
    \fi:
    ; #8 + #1 ; {#3#4#5#6}{#7};
  }
\cs_new:Npn \fp_cfs_i_after:NNNNNww
    #1 #2#3#4#5 #6; #7 ;
  {
    + #7
    \if:w 2 #1
      + 1 \exp_after:wN \fp_cfs_small_carry:w
    \fi:
    ; {#2#3#4#5} {#6}
  }
\cs_new:Npn \fp_cfs_small_carry:w ; #1 { ; {1000} }
%    \end{macrocode}
%
%   \begin{syntax}
%     \cs{fp_cfs_small_ii:NwwNN} \meta{sign}
%     ~~|1| \meta{digits} |,| \meta{zeros} |;| \meta{number~of~zeros}
%     ~~\meta{next~token}
%   \end{syntax}
%   Let's get ready for more digits. The \meta{next~token} is
%   f-expanded. If it is a digit, it is the $16$th digit.
%   \cs{fp_cfs_small_ii:NwwNN} builds a number $1$\meta{8 digits},
%   rounded properly, and feeds that to
%   \cs{fp_cfs_large_ii_after:NNNNNNww}.
%   
%    \begin{macrocode}
\cs_new:Npn \fp_cfs_small_ii:NwwNN #1 1 #2, #3; #4 #5
  {
        \exp_after:wN \fp_cfs_ii_after:NNNNNNww
        \exp_after:wN \c_zero
        \int_use:N \int_eval:w 1 #2 #3 
          \if_num:w 9 < 1 #5 \exp_stop_f:
            #5
            +
            \exp_after:wN \fp_cfs_small_round:NNNw
            \exp_after:wN #1
            \exp_after:wN #5
            \tex_romannumeral:D
          \else:
            0
            \exp_after:wN ;
            \int_value:w
              \fp_cfs_exponent_test:w #5
          \fi:
          \fp_aux_expand:w
  }
%    \end{macrocode}
%   Just as \cs{fp_cfs_i_after:NNNNNww}, pack digits in blocks
%   of $4$, check for a carry, and increase the \cs{int_eval:w}
%   above if needed. No surprise there.
%    \begin{macrocode}
%    \end{macrocode}
% \end{macro}
%
%
%
% \subsubsection{Finding the exponent}
%
%
% \begin{macro}{\fp_cfs_exponent_test:N}
%   \begin{syntax}
%     \cs{int_value:w} \cs{fp_cfs_exponent_test:N}
%   \end{syntax}
%   Can also be invoked from within \cs{int_eval:w}, since
%   the exponent gets terminated by a semicolon.
%
%   Expansion is a little bit tricky here. When implementing
%   expression parsing, later, we may want to accept input where
%   multiplication is implicit.
% \begin{verbatim}
% \fp_parse:n { 3.2 erf(0.1) }
% \fp_parse:n { 3.2 e\l_my_int }
% \fp_parse:n { 3.2 \c_pi_fp }
% \end{verbatim}
%   The first case indicates that just looking one character ahead
%   for an \enquote{\texttt{e}} is not enough, since we would mistake
%   the function \texttt{erf} for an exponent of \enquote{\texttt{rf}}.
%   An alternative would be to look two tokens ahead and check if what
%   follows is a sign or a digit, considering in that case that we must
%   be finding an exponent. But taking care of the second case requires
%   that we unpack registers after \texttt{e}. However, blindly expanding
%   the two tokens ahead completely (plus unpacking) would break
%   the third example. Indeed, in the course of reading $3.2$,
%   \cs{c_pi_fp} is expanded to \cs{s_fp} \cs{fp_use:w} |1| |0| |{-1}|
%   |{3141}| $\cdots$ |;| and \cs{s_fp} stops the expansion. Expanding
%   two tokens ahead would then force the expansion of \cs{fp_use:w}
%   (despite it being protected), and that function typesets the floating
%   point number.
%
%   What can we do? Really, the reason why this last case breaks is that
%   just as \TeX{} does, we should read ahead as little as possible.
%   Here, the only case where there may be an exponent is if the first
%   token ahead is |e|.\footnote{The commented code below allows
%     both \texttt{e} and \texttt{E}, but seems ugly.} Then we expand
%   (and possibly unpack) the second token --- and hopefully that is safe.
%
%    \begin{macrocode}
\cs_new:Npn \fp_cfs_exponent_test:w
    #1 \fi: \fp_aux_expand:w
  {
    \fi:
    \fp_cfs_exponent_test:N #1
  }
\cs_new:Npn \fp_cfs_exponent_test:N #1
  {
    % \if_num:w \tex_uccode:D `e
    %     = \tex_uccode:D \exp_after:wN ` \token_to_str:N #1 \exp_stop_f:
    \if:w e #1
      \exp_after:wN \fp_cfs_exponent_test_aux:NN
      \exp_after:wN #1
      \tex_romannumeral:D \exp_after:wN \fp_aux_unpack:w
    \else:
      \exp_after:wN 0
      \exp_after:wN \fp_cfs_exponent_end:
      \exp_after:wN #1
    \fi:
  }
%    \end{macrocode}
%   The interested reader is invited to
%   check that |#2=-| gives the case |0|, |#2=+| gives |2|,
%   |#2=|digit gives the \cs{else:} branch, and all other case,
%   including |*| and |/|, and also |(| and |)| (which stop the
%   \cs{int_eval:w}) go to case |1|. 
%
%   Anyways, that test should be rewritten. Now that I've put
%   the expansion, it is ugly. Note that after the signs we
%   unpack registers, whereas between digits we don't.
%    \begin{macrocode}
\cs_new:Npn \fp_cfs_exponent_test_aux:NN #1 #2
  {
    \if_case:w \int_eval:w 1 #2 1 \int_eval_end:
      - \exp_after:wN \fp_cfs_exponent_sign:N
      \tex_romannumeral:D \exp_after:wN \fp_aux_unpack:w
    \or:
      \exp_after:wN 0
      \exp_after:wN \fp_cfs_exponent_end:
      \exp_after:wN #1
      \exp_after:wN #2
    \or:
      \exp_after:wN \fp_cfs_exponent_sign:N
      \tex_romannumeral:D \exp_after:wN \fp_aux_unpack:w
    \else:
      #2
      \exp_after:wN \fp_cfs_exponent_body:N
      \tex_romannumeral:D \exp_after:wN \fp_aux_expand:w
    \fi:
  }
\cs_new:Npn \fp_cfs_exponent_sign:N #1
  {
    \if:w + \if:w - #1 + \fi: #1
      \exp_after:wN \fp_cfs_exponent_sign:N
      \tex_romannumeral:D \exp_after:wN \fp_aux_unpack:w
    \else:
      \exp_after:wN \fp_cfs_exponent_body:N
      \exp_after:wN #1
    \fi:
  }
\cs_new:Npn \fp_cfs_exponent_body:N #1
  {
    \if_num:w 9 < 1 #1 \exp_stop_f:
      #1
      \exp_after:wN \fp_cfs_exponent_body:N
      \tex_romannumeral:D \exp_after:wN \fp_aux_expand:w
    \else:
      \exp_after:wN \fp_cfs_exponent_end: \exp_after:wN #1
    \fi:
  }
\cs_new:Npn \fp_cfs_exponent_end: %#1 \s_fp_mark \s_fp_stop
  {
   % \if_catcode:w  $ \tl_to_str:n {#1} $
   %   \exp_after:wN \use_none:nn
   % \fi:
   % \fp_aux_error:n {#1}
    ; %\s_fp_mark \s_fp_stop
  }
%    \end{macrocode}
% \end{macro}
%
% 
%
%
%
% \subsubsection{Large mantissa}
%
% \begin{macro}{\fp_cfs_large:NwN}
%   \begin{syntax}
%     \cs{fp_cfs_large:NwN} \meta{sign} \meta{exp1} |;| \meta{non-zero digit}
%   \end{syntax}
%   This function and subsequent \cs{fp_cfs_large_...} functions are
%   the most convoluted of this submodule. It is called after trimming
%   leading zeros, if this does not reach a dot yet. Because we are
%   reading the integer part of the number, digits must be counted.
%   But at the same time, digits must be stored, and expansion must
%   continue. We only have one place to store both digits and the
%   corresponding exponent: upstream. Hence, quite a lot of reshuffling.
%
%    \begin{macrocode}
\cs_new:Npn \fp_cfs_large:NwN #1 #2 ; #3
  {
    \exp_after:wN \fp_cfs_sanitize:Nw
    \exp_after:wN #1
    \int_use:N \int_eval:w #2
      \exp_after:wN \fp_cfs_large_i:NwwNN
      \exp_after:wN #1
      \int_value:w 1 #3
        \exp_after:wN \fp_aux_grab_digits_vii:N
        \tex_romannumeral:D \fp_aux_expand:w
  }
%    \end{macrocode}
%   Beyond |#3|, which is the first non-zero digit of the (large)
%   mantissa (already expanded), we grab seven additional digits.
%   Afterwards, we add to the exponent the shift corresponding to
%   how many digits were grabbed, and test whether |#5| is a digit
%   (the $9$th digit), a dot, or something else.
%
%   Then, if |#5| is a digit, it is the $9$th digit of the mantissa,
%   and we grab six more digits to reach a total of $15$, calling
%   \cs{fp_cfs_large_ii:NwwNN}.
%
%   Otherwise, |#5| may be a dot, in which case we switch to the
%   functions we used in the \enquote{small} case. Finally, |#5|
%   may be something else, in which case we just look for an exponent.
%    \begin{macrocode}
\cs_new:Npn \fp_cfs_large_i:NwwNN #1 1 #2, #3; #4 #5
  {
      + 8 - #4
      \exp_after:wN \fp_cfs_i_after:NNNNNww
      \int_use:N \int_eval:w 1 #2
        \if_num:w 9 < 1 #5 \exp_stop_f:
          \exp_after:wN \fp_cfs_large_ii:NwwNN
          \exp_after:wN #1
          \int_value:w 1 #5
            \exp_after:wN \fp_aux_grab_digits_vi:N
            \tex_romannumeral:D
        \else:
          \if:w . #5
            \exp_after:wN \fp_cfs_small_i:NwwNN
            \exp_after:wN #1
            \int_value:w 1
              \cs:w
                fp_aux_grab_digits_
                \tex_romannumeral:D #4
                :N \exp_after:wN
              \cs_end:
              \tex_romannumeral:D
          \else:
            #3
            \exp_after:wN \fp_cfs_small_i_end:
            \int_value:w
              \exp_after:wN \fp_cfs_exponent_test:w
              \exp_after:wN #5
          \fi:
        \fi:
        \fp_aux_expand:w
  }
%    \end{macrocode}
%   
%    \begin{macrocode}

\cs_new:Npn \fp_cfs_large_ii:NwwNN #1 1 #2, #3; #4 #5
  {
      \if_num:w 9 < 1 #5 \exp_stop_f:
        \exp_after:wN \fp_cfs_ii_after:NNNNNNww
        \exp_after:wN \c_eight
        \int_use:N \int_eval:w 1 #2 #5
          + \exp_after:wN \fp_cfs_large_round:NNNw
            \exp_after:wN #1
            \exp_after:wN #5
            \tex_romannumeral:D
      \else:
        \exp_after:wN \fp_cfs_ii_after:NNNNNNww
        \int_use:N \int_eval:w 7 - #4 \exp_after:wN \int_eval_end:
        \int_use:N \int_eval:w 1 #2
          \if:w . #5
            \exp_after:wN \fp_cfs_large_iii:NwwNN
            \exp_after:wN #1
            \int_value:w 1
              \cs:w
                fp_aux_grab_digits_
                \tex_romannumeral:D #4
                :N \exp_after:wN
              \cs_end:
              \tex_romannumeral:D
          \else:
            #3 0 \exp_after:wN ;
            \int_value:w
              \exp_after:wN \fp_cfs_exponent_test:w
              \exp_after:wN #5
          \fi:
      \fi:
      \fp_aux_expand:w
  }
\cs_new:Npn \fp_cfs_large_iii:NwwNN #1 1 #2, #3; #4 #5
  {
          #2 #3
          \if_num:w 9 < 1 #5 \exp_stop_f:
            #5
            +
            \exp_after:wN \fp_cfs_small_round:NNNw
            \exp_after:wN #1
            \exp_after:wN #5
            \tex_romannumeral:D
          \else:
            0
            \exp_after:wN ;
            \int_value:w
              \fp_cfs_exponent_test:w #5
          \fi:
          \fp_aux_expand:w
  }
%    \end{macrocode}
% \end{macro}
%
%
% \subsubsection{Beyond 16 digits: rounding}
%
% \begin{macro}{\fp_cfs_round_loop:N}
%   Used both for \cs{fp_cfs_small_round:NNNw} and
%   \cs{fp_cfs_large_round:NNNw}.
%   Should appear after a \cs{int_eval:w} |0|. Reads digits one by one,
%   until reaching a non-digit. Adds |+1| for each digit. If all digits
%   found are |0|, ends the \cs{int_eval:w} by |;\c_zero|, otherwise
%   by |;\c_one|. This is done by switching the loop to |round_up|
%   at the first non-zero digit.
%
%    \begin{macrocode}
\cs_new:Npn \fp_cfs_round_loop:N #1
  {
    \if_num:w 9 < 1 #1 \exp_stop_f:
      + \c_one
      \if:w 0 #1
        \exp_after:wN \fp_cfs_round_loop:N
        \tex_romannumeral:D
      \else:
        \exp_after:wN \fp_cfs_round_up:N
        \tex_romannumeral:D
      \fi:
    \else:
      \fp_aux_swap_fisuf:w ; \c_zero #1
    \fi:
    \fp_aux_expand:w
  }
\cs_new:Npn \fp_cfs_round_up:N #1
  {
    \if_num:w 9 < 1 #1 \exp_stop_f:
      + 1
      \exp_after:wN \fp_cfs_round_up:N
      \tex_romannumeral:D
    \else:
      \fp_aux_swap_fisuf:w ; \c_one #1
    \fi:
    \fp_aux_expand:w
  }
%    \end{macrocode}
% \end{macro}
%
%
% \begin{macro}{\fp_cfs_large_round:NNNw}
%   \begin{syntax}
%     \cs{fp_cfs_large_round:NNNw} \meta{sign} \meta{digit} \meta{stuff}
%   \end{syntax}
%   \meta{digit} is the digit that we are currently rounding (we only
%   care whether it is even or odd).
%
%   The goal is to get \cs{c_zero} or \cs{c_one}, check for an exponent
%   afterwards, and combine it to the number of digits before the decimal
%   point (which we thus need to keep track of).
%    \begin{macrocode}
\cs_new:Npn \fp_cfs_large_round:NNNw #1 #2 #3
  {
    \if_num:w 9 < 1 #3 \exp_stop_f:
      \exp_after:wN \fp_aux_round_s:NNNw
      \exp_after:wN #1
      \exp_after:wN #2
      \exp_after:wN #3
      \int_use:N \int_eval:w
        \exp_after:wN \fp_cfs_large_round_after:wNN
        \int_use:N \int_eval:w 1
          \exp_after:wN \fp_cfs_round_loop:N
    \else: %^^A could be dot, or e, or other
      \exp_after:wN \fp_cfs_large_round_dot_test:NNNw
      \exp_after:wN #1
      \exp_after:wN #2
      \exp_after:wN #3
    \fi:
  }
\cs_new:Npn \fp_cfs_large_round_dot_test:NNNw #1 #2 #3
  {
    \if:w . #3
      \exp_after:wN \fp_cfs_small_round:NNNw
      \exp_after:wN #1
      \exp_after:wN #2
      \tex_romannumeral:D
    \else:
      \exp_after:wN \c_zero \exp_after:wN ;
      \int_value:w
        \fp_cfs_exponent_test:w #3
    \fi:
    \fp_aux_expand:w
  }
%    \end{macrocode}
%   \begin{syntax}
%     \cs{fp_cfs_large_round_after:wNN} \meta{exp} |;|
%     ~~\meta{0 or 1} \meta{next~token}
%   \end{syntax}
%    \begin{macrocode}
\cs_new:Npn \fp_cfs_large_round_after:wNN #1 ; #2 #3
  {
    \if:w . #3
      \exp_after:wN \fp_cfs_large_round_after_ii:wN
      \int_use:N \int_eval:w #1 +
        \c_zero * \int_eval:w
          \exp_after:wN \fp_cfs_round_loop:N
          \tex_romannumeral:D
    \else:
      + #2
      \exp_after:wN ;
      \int_use:N \int_eval:w #1 +
        \fp_cfs_exponent_test:w #3
    \fi:
    \fp_aux_expand:w
  }
\cs_new:Npn \fp_cfs_large_round_after_ii:wN #1 ; #2
  {
    + #2
    \exp_after:wN ;
    \int_use:N \int_eval:w #1 +
      \fp_cfs_exponent_test:N
  }
%    \end{macrocode}
% \end{macro}
%
%
%
% \begin{macro}{\fp_cfs_small_round:NNNw}
%   \begin{syntax}
%     \cs{fp_cfs_small_round:NNNw} \meta{sign} \meta{digit} \meta{stuff}
%   \end{syntax}
%   \meta{digit} is the digit that we are currently rounding (we only
%   care whether it is even or odd).
%
%   The goal is to get \cs{c_zero} or \cs{c_one}
%    \begin{macrocode}
\cs_new:Npn \fp_cfs_small_round:NNNw #1 #2 #3
  {
    \if_num:w 9 < 1 #3 \exp_stop_f:
      \exp_after:wN \fp_aux_round_s:NNNw
      \exp_after:wN #1
      \exp_after:wN #2
      \exp_after:wN #3
      \int_use:N \int_eval:w
        \exp_after:wN \fp_cfs_small_round_after:wN
        \int_use:N \int_eval:w 0
          \exp_after:wN \fp_cfs_round_loop:N
          \tex_romannumeral:D
    \else:
      0 \exp_after:wN ;
      \int_value:w
        \fp_cfs_exponent_test:w #3
    \fi:
    \fp_aux_expand:w
  }
\cs_new:Npn \fp_cfs_small_round_after:wN #1; #2
  {
    + #2 \exp_after:wN ;
    \int_value:w \fp_cfs_exponent_test:N
  }
%    \end{macrocode}
% \end{macro}
%
%
% \subsubsection{Just a final check}
%
% \begin{macro}{\fp_cfs_sanitize:Nw}
%   \begin{syntax}
%     \cs{fp_cfs_sanitize:Nw} \meta{sign} \meta{exp} |;|
%     ~~\Arg{\4~digits} \Arg{\4~digits} \Arg{\4~digits} \Arg{\4~digits} |;|
%   \end{syntax}
%   check overflow
%    \begin{macrocode}
\cs_new:Npn \fp_cfs_sanitize:Nw #1 #2 ;
  {
    \if_num:w #2 > \c_fp_max_exponent_int
      \exp_after:wN \fp_aux_overflow:w
    \else:
      \if_num:w #2 < \c_fp_min_exponent_int
        \exp_after:wN \exp_after:wN
        \exp_after:wN \fp_aux_underflow:w
      \fi:
    \fi:
    \s_fp \fp_use:w 1 #1 {#2}
  }
%    \end{macrocode}
% \end{macro}
%
% \subsection{Int}
%
% Conversion from \TeX{}'s integers is mostly provided because it
% is involved in the conversion from \TeX{}'s dimensions.
%
% \begin{macro}{\fp_convert_from_int:n}
%   \begin{syntax}
%     \cs{fp_convert_from_int:n} \Arg{integer expression}
%   \end{syntax}
%   Of course, the input has to be within the bounds of \TeX{}'s
%   integer expressions.
%    \begin{macrocode}
\cs_new:Npn \fp_convert_from_int:n #1
  {
    \exp_after:wN \fp_convert_from_int_test:N
    \int_use:N \int_eval:w #1 ;
  }
\cs_new:Npn \fp_convert_from_int_test:N #1
  {
    \if_num:w \c_ten < 1 #1 \exp_stop_f:
      \exp_after:wN \fp_convert_from_int_ii:Nw
      \exp_after:wN 0
      \exp_after:wN #1
    \else:
      \if_meaning:w 0 #1
        \exp_after:wN \exp_after:wN
        \exp_after:wN \fp_convert_from_int_zero:w
      \else:
        \exp_after:wN \exp_after:wN
        \exp_after:wN \fp_convert_from_int_ii:Nw
        \exp_after:wN \exp_after:wN
        \exp_after:wN 2
      \fi:
    \fi:
  }
\cs_new:Npn \fp_convert_from_int_zero:w ;
  { \c_plus_zero_fp } %^^A the sign should depend on the rounding mode
\cs_new:Npn \fp_convert_from_int_ii:Nw #1 #2;
  { \fp_convert_from_int_iii:NNNNNNNN #2 000 0000 00{10}987654321; #1 }
\cs_new:Npn \fp_convert_from_int_iii:NNNNNNNN #1#2#3#4 #5#6#7#8
  { \fp_convert_from_int_iii:nNNNNNNN { {#1#2#3#4} {#5#6#7#8} } }
\cs_new:Npn \fp_convert_from_int_iii:nNNNNNNN #1 #2#3 #4#5#6#7#8
  { \fp_convert_from_int_iv:nnnnnnwN { #1 {#2#300} {0000} } }
\cs_new:Npn \fp_convert_from_int_iv:nnnnnnwN #1 #2#3#4#5 #6 #7; #8
  { \s_fp \fp_use:w 1 #8 {#6} #1 ; }
%    \end{macrocode}
% \end{macro}
%
%
% \subsection{Dimensions}
%
% \begin{macro}{\fp_convert_from_dim:n}
%   \begin{syntax}
%     \cs{fp_convert_from_dim:n} \Arg{dimension expression}
%   \end{syntax}
%   The dimension expression (which can in fact be a glue expression)
%   is evaluated, converted to a number (\emph{i.e.}, expressed in
%   scaled points), then multiplied by $2^{-16} = 0.0000152587890625$
%   to give a value expressed in points.
%
%   We use the auxiliary function
%   \begin{quote}
%     \cs{fp_basics_mul_npos:Nnwnw} \meta{sign} \newline
%     ~~\Arg{exp1}  \meta{body1} |;| \Arg{exp2} \meta{body2} |;|
%   \end{quote}
%
%    \begin{macrocode}
\cs_new:Npn \fp_convert_from_dim:n #1
  {
    \exp_after:wN \fp_convert_from_dim_aux:w
    \tex_romannumeral:D -`0
    \exp_after:wN \fp_convert_from_int_test:N
    \int_value:w \etex_glueexpr:D #1 ;
    {-4} {1525} {8789} {0625} {0000} ;
  }
\cs_new:Npn \fp_convert_from_dim_aux:w \s_fp \fp_use:w #1
  {
    \if_meaning:w 0 #1
      \exp_after:wN \fp_convert_from_dim_zero:w
    \else:
      \exp_after:wN \fp_basics_mul_npos:Nnwnw
    \fi:
  }
\cs_new:Npn \fp_convert_from_dim_zero:w #1; #2; { \c_plus_zero_fp }
%    \end{macrocode}
% \end{macro}
% 
%
% \begin{macro}{\fp_convert_to_dim:n}
%   \begin{syntax}
%     \cs{fp_convert_to_dim:n} \Arg{floating point string}
%   \end{syntax}
%   f-expands to a dimension (ending in \texttt{pt}), suitable
%   for \TeX{} to digest. Now, any non-normal number is converted
%   to the zero dimension.\footnote{Bruno:
%     should nan raise exceptions? Should infinities give maxdimen?}
%    \begin{macrocode}
\cs_new:Npn \fp_convert_to_dim:n
  {
    \exp_after:wN \fp_convert_to_dim:w
    \tex_romannumeral:D -`0 \fp_convert_from_str:n
  }
\cs_new:Npn \fp_convert_to_dim:w \s_fp \fp_use:w #1
  {
    \if_num:w #1 = \c_one
      \exp_after:wN \fp_convert_to_dim_normal:Nnnnnn
    \else:
      \exp_after:wN \fp_convert_to_dim_zero:w
    \fi:
  }
\cs_new:Npn \fp_convert_to_dim_zero:w #1; { \dim_use:N \c_zero_dim }
%    \end{macrocode}
%   In the case of normal numbers ($\meta{case}=1$), we need to shift,
%   and then insert the number.\footnote{Bruno: incorrect rounding now,
%     I must look into Knuth's code to be sure of how to produce exact
%     rounding in all cases (with various rounding modes.} The
%   arguments are the sign and the exponent, then the four groups of
%   digits and a semicolon. We unpack those digits.
%    \begin{macrocode}
\cs_new:Npn \fp_convert_to_dim_normal:Nnnnnn #1 #2 #3#4#5#6;
  {
    \if_num:w #2 > \c_zero
      \if_num:w #2 > \c_eight
        \exp_after:wN \exp_after:wN
        \exp_after:wN \fp_convert_to_dim_huge:Nnw
      \else:
        \exp_after:wN \exp_after:wN
        \exp_after:wN \fp_convert_to_dim_large:Nn
      \fi:
    \else:
      \exp_after:wN \fp_convert_to_dim_small:Nn
    \fi:
    #1 {#2} #3#4#5#6pt
  }
%    \end{macrocode}
%   This could perhaps be optimized by instead multiplying by
%   $65536$ and using a \cs{dim_eval:n} with the integer in
%   scaled points. Right now, we need to leave all the digits,
%   because in some very rare cases, \TeX{}'s rounding seems
%   to be affected by those digits.
%
%   For non-positive exponent, just add zeros. For slightly
%   bigger exponent, fill with zeros and grab eight digits
%   to repack (the \cs{tex_number:D} primitive does the work
%   of removing leading zeros). For exponents larger than
%   $8$, the dimension is too big anyways, so produce an
%   overflow (could be improved a lot).
%    \begin{macrocode}
\cs_new:Npn \fp_convert_to_dim_small:Nn #1 #2
  {
    \exp_after:wN \fp_convert_to_dim_sign:N
    \exp_after:wN #1
    \tex_number:D
      \exp_after:wN 0 \exp_after:wN .
      \tex_romannumeral:D -`0 \prg_replicate:nn {-#2} {0}
  }
\cs_new:Npn \fp_convert_to_dim_large:Nn #1 #2
  {
    \exp_after:wN \fp_convert_to_dim_sign:N
    \exp_after:wN #1
    \tex_number:D
      \exp_after:wN \fp_convert_to_dim_large_pack:NNNNNNNNw
      \tex_romannumeral:D -`0 \prg_replicate:nn {8-#2} {0}
  }
\cs_new:Npn \fp_convert_to_dim_sign:N #1
  {
    \if:w 0 #1
      \exp_after:wN +
    \else:
      \exp_after:wN -
    \fi:
  }
\cs_new:Npn \fp_convert_to_dim_large_pack:NNNNNNNNw #1#2#3#4#5#6#7#8 #9pt
  { #1#2#3#4#5#6#7#8 . #9pt }
\cs_new:Npn \fp_convert_to_dim_huge:Nnw #1 #2 #3pt
  { 100000000000000000000 pt }
%    \end{macrocode}
% \end{macro}
%
%    \begin{macrocode}
%</initex|package>
%    \end{macrocode}
%
%\end{implementation}
%
%\PrintChanges
%
%\PrintIndex