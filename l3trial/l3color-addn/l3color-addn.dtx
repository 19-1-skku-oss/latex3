% \iffalse meta-comment
%
%% File: l3color-addn.dtx Copyright(C) 2017 The LaTeX3 Project
%
% It may be distributed and/or modified under the conditions of the
% LaTeX Project Public License (LPPL), either version 1.3c of this
% license or (at your option) any later version.  The latest version
% of this license is in the file
%
%    http://www.latex-project.org/lppl.txt
%
% This file is part of the "l3kernel bundle" (The Work in LPPL)
% and all files in that bundle must be distributed together.
%
% -----------------------------------------------------------------------
%
% The development version of the bundle can be found at
%
%    https://github.com/latex3/latex3
%
% for those people who are interested.
%
%<*driver>
\documentclass[full]{l3doc}
\begin{document}
  \DocInput{\jobname.dtx}
\end{document}
%</driver>
% \fi
%
% \title{^^A
%   The \textsf{l3color-addn} package\\ Experimental additions to color support^^A
% }
%
% \author{^^A
%  The \LaTeX3 Project\thanks
%    {^^A
%      E-mail:
%        \href{mailto:latex-team@latex-project.org}
%          {latex-team@latex-project.org}^^A
%    }^^A
% }
%
% \date{Released 2017/09/18}
%
% \maketitle
%
% \begin{documentation}
%
% \section{Color expressions}
%
% \section{Named colors}
%
% \begin{function}{\color_set:nn}
%   \begin{syntax}
%     \cs{color_set:nn} \marg{name} \marg{color expression}
%   \end{syntax}
% \end{function}
%
% \begin{function}{\color_set:nnn}
%   \begin{syntax}
%     \cs{color_set:nnn} \marg{name} \marg{model} \marg{value(s)}
%   \end{syntax}
% \end{function}
%
% \begin{function}{\color_set_eq:nn}
%   \begin{syntax}
%     \cs{color_set_eq:nn} \meta{name1} \marg{name2}
%   \end{syntax}
% \end{function}
%
% \section{Selecting colors}
%
% \begin{function}{\color_select:n}
%   \begin{syntax}
%     \cs{color_select:n} \meta{color expression}
%   \end{syntax}
% \end{function}
%
% \begin{function}{\color_select:nn}
%   \begin{syntax}
%     \cs{color_select:nn} \meta{model} \meta{value(s)}
%   \end{syntax}
% \end{function}
%
% \end{documentation}
%
% \begin{implementation}
%
% \section{\pkg{l3color-addn} Implementation}
%
%    \begin{macrocode}
%<*initex|package>
%    \end{macrocode}
%
% \subsection{Predefined color names}
%
% The ability to predefine colors with a name is a key part of this module and
% means there has to be a method for storing the results. At first sight, it
% seems natural to follow the usual \pkg{expl3} model and create a
% \texttt{color} variable type for the process. That would then allow both
% local and global colors, constant colors and the like. However, these names
% need to be accessible in some form at the user level, for selection of colors
% either simply by name or as part of a more complex expression. This does not
% require that the full name is exposed but does require that they can be
% looked up in a predictable way. As such, it is more useful to expose just the
% color names as part of the interface, with the result that only local color
% names can be created. (This is also seen for example in key creation in
% \pkg{l3keys}.) As a result, color names are declarative (no \texttt{new}
% functions).
%
% Since there is no need to manipulate colors \emph{en masse}, each is stored
% in a separate token list variable, rather than the alternative of using a
% single property list for all names.
%
% \subsection{Setup}
%
% \begin{variable}{\l_@@_tmp_tl}
%    \begin{macrocode}
\tl_new:N \l_@@_tmp_tl
%    \end{macrocode}
% \end{variable}
%
% \subsection{Utility functions}
%
% \begin{macro}[int, TF, EXP]{\@@_if_defined:n}
%   A simple wrapper to avoid needing to have the lookup repeated in too many
%   places.
%    \begin{macrocode}
\prg_new_conditional:Npnn \@@_if_defined:n #1 { T, F, TF }
  {
    \tl_if_exist:cTF { l_@@_named_ #1 _tl }
      \prg_return_true:
      \prg_return_false:
  }
%    \end{macrocode}
% \end{macro}
%
% \begin{macro}[int]{\@@_extract:nNN}
% \begin{macro}[aux]{\@@_extract:NNw}
%   Split the model and color from a named color, and store the two. No test
%   for the existence of the color: that is assumed to be the case (this
%   is internal only). Somewhat \enquote{old-fashioned} but should be quite
%   fast.
%    \begin{macrocode}
\cs_new_protected:Npn \@@_extract:nNN #1#2#3
  {
    \exp_after:wN \exp_after:wN \exp_after:wN
      \@@_extract:NNw
      \exp_after:wN \exp_after:wN \exp_after:wN #2
      \exp_after:wN \exp_after:wN \exp_after:wN #3
        \cs:w l_@@_named_ #1 _tl \cs_end: \q_stop
  }
\cs_new_protected:Npn \@@_extract:NNw #1#2 #3 ~ #4 \q_stop
  {
    \tl_set:Nn #1 {#3}
    \tl_set:Nn #2 {#4}
  }
%    \end{macrocode}
% \end{macro}
% \end{macro}
%
% \subsection{Model conversion}
%
% \begin{macro}[int]{\@@_convert:nnN, \@@_convert:VVN}
% \begin{macro}[aux]{\@@_convert:nnnN}
% \begin{macro}[aux, EXP]
%   {
%     \@@_convert_gray_rgb:w
%     \@@_convert_gray_cmyk:w
%     \@@_convert_cmyk_gray:w
%     \@@_convert_cmyk_rgb:w
%     \@@_convert_rgb_gray:w
%     \@@_convert_rgb_cmyk:w
%   }
%  \begin{macro}[aux, EXP]{\@@_convert_rgb_cmyk:nnnnn}
%    Model conversion is carried out using standard formulae, as described in
%    the manual for \pkg{xcolor} (see also the \emph{PostScript Language
%    Reference Manual}).
%    \begin{macrocode}
\cs_new_protected:Npn \@@_convert:nnN #1#2#3
  { \@@_convert:nnVN {#1} {#2} #3 #3 }
\cs_generate_variant:Nn \@@_convert:nnN { VV }
\cs_new_protected:Npn \@@_convert:nnnN #1#2#3#4
  {
    \str_if_eq_x:nnT {#1} { spot } % TO DO!!!
      { }
    \tl_set:Nx #4
      { \use:c { @@_convert_ #1 _ #2 :w } #3 ~ 0 ~ 0 ~ 0 \q_stop }
  }
\cs_generate_variant:Nn \@@_convert:nnnN { nnV }
\cs_new:Npn \@@_convert_gray_rgb:w #1 ~ #2 \q_stop
  { #1 ~ #1 ~ #1 }
\cs_new:Npn \@@_convert_gray_cmyk:w #1 ~ #2 \q_stop
  { 0 ~ 0 ~ 0 ~ #1 }
%    \end{macrocode}
%   These rather odd values are based on \textsc{ntsc} television: the set are
%   used for the |cmyk| conversion.
%    \begin{macrocode}
\cs_new:Npn \@@_convert_rgb_gray:w #1 ~ #2 ~ #3 ~ #4 \q_stop
  { \fp_eval:n { 0.3 * #1 + 0.59 * #2 + 0.11 * #3 } }
%    \end{macrocode}
%   The conversion from |rgb| to |cmyk| is the most complex: a two-step
%   procedure which requires \emph{black generation} and \emph{undercolor
%   removal} functions. The PostScript reference describes them as
%   device-dependent, but following \pkg{xcolor} we assume they are linear.
%   Moreover, as the likelihood of anyone using a non-unitary matrix here is
%   tiny, we simplify and treat those two concepts as no-ops.
%    \begin{macrocode}
\cs_new:Npn \@@_convert_rgb_cmyk:w #1 ~ #2 ~ #3 ~ #4 \q_stop
  {
    \exp_args:Nf \@@_convert_rgb_cmyk:nnnnn
      { \fp_eval:n { min ( 1 - #1 , 1 - #2 , 1 - #3 ) } {#1} {#2} {#3}
  }

\cs_new:Npn \@@_convert_rgb_cmyk:nnnnn #1#2#3#4
  {
    \fp_eval:n { min ( 1 , max ( 0 , 1 - #2 - #1 ) ) } \c_space_tl
    \fp_eval:n { min ( 1 , max ( 0 , 1 - #3 - #1 ) ) } \c_space_tl
    \fp_eval:n { min ( 1 , max ( 0 , 1 - #4 - #1 ) ) } \c_space_tl
    #1
  }
\cs_new:Npn \@@_convert_cmyk_gray:w #1 ~ #2 ~ #3 ~ #4 ~ #5 \q_stop
  { \fp_eval:n { 1 - min ( 1 , 0.3 * #1 + 0.59 * #2 + 0.11 * #3 + #4 ) } } }
\cs_new:Npn \@@_convert_cmyk_rgb:w #1 ~ #2 ~ #3 ~ #4 ~ #5 \q_stop
  {
    \fp_eval:n { 1 - min ( 1 , #1 + #4 ) } \c_space_tl
    \fp_eval:n { 1 - min ( 1 , #2 + #4 ) } \c_space_tl
    \fp_eval:n { 1 - min ( 1 , #3 + #4 ) }
  }
%    \end{macrocode}
% \end{macro}
% \end{macro}
% \end{macro}
% \end{macro}
%
% \subsection{Color expressions}
%
% \begin{variable}
%   {\l_@@_model_tl, \l_@@_value_tl, \l_@@_next_model_tl, \l_@@_next_value_tl}
%   Working space to store the color data whilst doing calculations: keeping
%   it on the stack is attractive but gets tricky (return is non-trivial).
%    \begin{macrocode}
\tl_new:N \l_@@_model_tl
\tl_new:N \l_@@_value_tl
\tl_new:N \l_@@_next_model_tl
\tl_new:N \l_@@_next_value_tl
%    \end{macrocode}
% \end{variable}
%
% \begin{macro}[int]{\@@_parse:nn}
% \begin{macro}[aux]{\@@_parse:n}
% \begin{macro}[aux]{\@@_parse:w}
% \begin{macro}[aux]{\@@_parse_loop_init:nn}
% \begin{macro}[aux]{\@@_parse_loop:w}
% \begin{macro}[aux]{\@@_parse_loop:nn}
% \begin{macro}[aux]{\@@_parse_break:w}
% \begin{macro}[aux]{\@@_parse_end:}
% \begin{macro}[aux, EXP]{\@@_parse_mix:Nnnn}
% \begin{macro}[aux, EXP]{\@@_parse_mix:nNnn}
% \begin{macro}[aux, EXP]
%   {
%     \@@_parse_mix_gray:nw ,
%     \@@_parse_mix_rgb:nw  ,
%     \@@_parse_mix_cmyk:nw
%   }
%   The main function for parsing color expressions removes actives but
%   otherwise expands, then starts working through the expression itself.
%   At the end, we apply the payload.
%    \begin{macrocode}
\cs_new_protected:Npn \@@_parse:nn #1#2
  {
    \group_begin:
      \seq_map_inline:Nn \l_char_active_seq
        {
          \tl_set:Nx \l_@@_tmp_tl { \cs_to_str:N ##1 }
          \char_set_active_eq:NN ##1 \l_@@_tmp_tl
        }
      \tl_set:Nx \l_@@_tmp_tl {#2}
    \exp_args:NNV \group_end:
    \@@_parse:V \l_@@_tmp_tl
    #1
  }
%    \end{macrocode}
%   Before going to all of the effort of parsing an expression, these two
%   precursor functions look for a pre-defined name, either on its own or
%   with a trailing |!| (which is the same thing).
%    \begin{macrocode}
\cs_new_protected:Nn \@@_parse:n #1
  {
    \tl_if_exist:cTF { l_@@_named_ #1 _tl }
      { \tl_set_eq:Nc \l__color_current_tl { l_@@_named_ #1 _tl } }
      { \@@_parse:w #1 ! \q_stop }
  }
\cs_new_protected:Npn \@@_parse:w #1 ! #2 \q_stop
  {
    \@@_if_defined:nTF {#1}
      {
        \tl_if_blank:nTF {#2}
          { \tl_set_eq:Nc \l__color_current_tl { l_@@_named_ #1 _tl } }
          { \@@_parse_loop_init:nn {#1} {#2} }
      }
      {
        \msg_kernel_error:nnn { color } { unknown-color } {#2}
        \tl_set_eq:NN \l__color_current_tl \l_@@_named_black_tl
      }
  }
%    \end{macrocode}
%   Once we establish that a full parse is needed, the next job is to get the
%   detail of the first color. That will determine the model we use for the
%   calculation: splitting here makes checking that a bit easier.
%    \begin{macrocode}
\cs_new_protected:Npn \@@_parse_loop_init:nn #1#2
  {
    \group_begin:
      \@@_extract:nNN {#1} \l_@@_model_tl \l_@@_value_tl
      \@@_parse_loop:w #2 ! ! ! ! \q_stop
      \tl_set:Nx \l_@@_tmp_tl
        { \l_@@_model_tl \c_space_tl \l_@@_value_tl }
    \exp_args:NNNV \group_end:
    \tl_set:Nn \l__color_current_tl \l_@@_tmp_tl
  }
%    \end{macrocode}
%   This is the loop proper: there can be an open-ended set of colors to parse,
%   separated by |!| tokens. There are a few cases to look out for. At the end
%   of the expression and with we find a mix of $100$ then we simply skip the
%   next color entirely (we can't stop the loop as there might be a further
%   valid color to mix in). On the other hand, if we get a mix of $0$ then
%   drop everything so far and start again. There is also a trailing
%   |white| to \enquote{read in} if the final explicit data is a mix.
%   Those conditions are separate from actually looping, which is therefore
%   sorted out by checking if we have further data to process: in contrast
%   to \pkg{xcolor}, we don't allow |!!| so the test can be simplified.
%    \begin{macrocode}
\cs_new_protected:Npn \@@_parse_loop:w #1 ! #2 ! #3 ! #4 ! #5 \q_stop
  {
    \bool_lazy_or:nnF
      { \tl_if_blank_p:n {#1} }
      { ! \int_compare_p:nNn {#1} = 100 }
      {
        \int_compare:nNnTF {#1} = { 0 }
          {
            \tl_if_blank:nTF {#2}
              { \@@_extract:nNN { white } }
              { \@@_extract:nNN {#2} }
                \l_@@_model_tl \l_@@_value_tl
          }
          {
            \use:x
              {
                \@@_parse_loop:nn {#1}
                  { \tl_if_blank:nTF {#2} { white } {#2} }
              }
          }
      }
    \tl_if_blank:nF {#3}
      { \@@_parse_loop:w #3 ! #4 ! #5 \q_stop }
    \@@_parse_end:
  }
%    \end{macrocode}
%   The \enquote{payload} of calculation in the loop first. If the model for
%   the upcoming color is different from that of the existing (partial) color,
%   convert the model. For |gray| the two are flipped round so that the outcome
%   is something with \enquote{real} color. We are then in a position to do the
%   actual calculation itself. The two auxiliaries here give us a way to break
%   the loop should an invalid name be found.
%    \begin{macrocode}
\cs_new_protected:Npn \@@_parse_loop:nn #1#2
  {
    \@@_if_defined:nTF {#1}
      {
        \@@_extract:nNN {#1} \l_@@_next_model_tl \l_@@_next_value_tl
        \tl_if_eq:NNF \l_@@_model_tl \l_@@_next_model_tl
          {
            \str_if_eq_x:nnT { \l_@@_model_tl } { gray }
              {
                \use:x
                  {
                    \tl_set:Nn \exp_not:N \l_@@_model_tl
                      { \l_@@_next_model_tl }
                    \tl_set:Nn \exp_not:N \l_@@_value_tl
                      { \l_@@_next_value_tl }
                    \tl_set:Nn \exp_not:N \l_next_@@_model_tl
                      { \l_@@_model_tl }
                    \tl_set:Nn \exp_not:N \l_next_@@_value_tl
                      { \l_@@_value_tl }
                  }
              }
            \@@_convert:VVN
              \l_@@_next_model_tl
              \l_@@_model_tl
              \l_@@_next_value_tl
          }
        \tl_set:Nx \l_@@_value_tl
          {
            \@@_parse_mix:NVVn
              \l_@@_model_tl \l_@@_value_tl \l_@@_next_value_tl {#2}
          }
      }
      {
        \msg_kernel_error:nnn { color } { unknown-color } {#2}
        \@@_extract:nNN { black } \l_@@_model_tl \l_@@_value_tl
        \@@_parse_break:w
      }
  }
\cs_new_protected:Npn \@@_parse_break:w #1 \@@_parse_end: { }
\cs_new_protected:Npn \@@_parse_end: { }
%    \end{macrocode}
%   Do the vector arithmetic: mainly a question of shuffling input, along
%   with one pre-calculation to keep down the use of division.
%    \begin{macrocode}
\cs_new:Npn \@@_parse_mix:Nnnn #1#2#3#4
  {
    \exp_args:Nf \@@_parse_mix:nNnn
      { \fp_eval:n { #4 / 100 } }
      #1 {#2} {#3}
  }
\cs_new:Npn \@@_parse_mix:nNnn #1#2#3#4
  {
    \use:c { @@_parse_mix_ #2 :nnw } {#1}
      #3 \q_mark #4 \q_stop
  }
\cs_new:Npn \@@_parse_mix_gray:nw #1#2 \q_mark #3 \q_stop
  { \fp_eval:n { #2 * #1 + #3 * ( 1 - #1 ) } }
\cs_new:Npn \@@_parse_mix_rgb:nw
  #1#2 ~ #3 ~ #4 \q_mark #5 ~ #6 ~ #7 \q_stop
  {
    \fp_eval:n { #2 * #1 + #5 * ( 1 - #1 ) } \c_space_tl
    \fp_eval:n { #3 * #1 + #6 * ( 1 - #1 ) } \c_space_tl
    \fp_eval:n { #4 * #1 + #7 * ( 1 - #1 ) }
  }
\cs_new:Npn \@@_parse_mix_cmyk:nw
  #1#2 ~ #3 ~ #4 ~ #5 \q_mark #6 ~ #7 ~ #8 ~ #9 \q_stop
  {
    \fp_eval:n { #2 * #1 + #6 * ( 1 - #1 ) } \c_space_tl
    \fp_eval:n { #3 * #1 + #7 * ( 1 - #1 ) } \c_space_tl
    \fp_eval:n { #4 * #1 + #8 * ( 1 - #1 ) } \c_space_tl
    \fp_eval:n { #5 * #1 + #9 * ( 1 - #1 ) }
  }
%    \end{macrocode}
% \end{macro}
% \end{macro}
% \end{macro}
% \end{macro}
% \end{macro}
% \end{macro}
% \end{macro}
% \end{macro}
% \end{macro}
% \end{macro}
% \end{macro}
%
% \begin{macro}[EXP]
%   {\@@_parse_gray:w, \@@_parse_rgb:w, \@@_parse_cmyk:w, \@@_parse_spot:w}
% \begin{macro}[EXP]{\@@_parse_spot_aux:w}
%   Turn the input into internal form.
%    \begin{macrocode}
\cs_new:Npn \@@_parse_gray:w #1 , #2 \q_stop {#1}
\cs_new:Npn \@@_parse_rgb:w #1 , #2 , #3 , #4 \q_stop { #1 ~ #2 ~ #3 }
\cs_new:Npn \@@_parse_cmyk:w #1 , #2 , #3 , #4 , #5 \q_stop
  { #1 ~ #2 ~ #3 ~ #4 }
\cs_new:Npn \@@_parse_spot:w #1 , #2 \q_stop
  { \@@_parse_spot_aux:w #1 ! 100 ! \q_stop }
\cs_new:Npn \@@_parse_spot_aux:w #1 ! #2 ! #3 \q_stop
  { #1 ~ \fp_eval:n { #2 / 100 } }
%    \end{macrocode}
% \end{macro}
% \end{macro}
%
% \subsection{Direct model use}
%
% \begin{macro}{\@@_use:nnn}
%    \begin{macrocode}
\cs_new_protected:Npn \@@_use:nnn #1#2#3
  {
    \cs_if_exist:cTF { @@_parse_ #2 :w }
      {
        \tl_set:Nx \l__color_current_tl
          { #2 ~ \use:c {  @@_parse_ #2 :w } #3 , 0 , 0 , 0 , 0 \q_stop }
        #1
      }
      {
        \__msg_kernel_error:nnn { color } { invalid-model } {#2}
      }
  }
%    \end{macrocode}
% \end{macro}
%
% \begin{macro}{\color_set:nn}
% \begin{macro}{\color_set:nnn}
% \begin{macro}{\color_set_eq:nn}
%   Defining named colors has to include a step to force creation of the
%   underlying token list to avoid errors when checking is enabled.
%    \begin{macrocode}
\cs_new_protected:Npn \color_set:nn #1
  { \@@_parse:nn { \@@_store:n {#1} } }
\cs_new_protected:Npn \color_set:nnn #1
  { \@@_use:nnn { \@@_store:n {#1} } }
\cs_new_protected:Npn \color_set_eq:nn #1#2
  {
    \@@_if_defined:nTF {#2}
      {
        \tl_clear_new:c { l_@@_named_ #1 _tl }
        \str_if_eq:nnTF {#2} { . }
          { \tl_et_eq:cN { l_@@_named_ #1 _tl } \l__color_current_tl }
          { \tl_et_eq:cc { l_@@_named_ #1 _tl } { l_@@_named_ #2 _tl } }
      }
      {
        \msg_kernel_error:nnn { color } { unknown-color } {#2}
      }
  }
%    \end{macrocode}
% \end{macro}
% \end{macro}
% \end{macro}
% \end{macro}
% \end{macro}
% \end{macro}
%
% \begin{macro}{\color_select:n}
% \begin{macro}{\color_select:nn}
%   Parse the input expressions then get the driver to actually activate
%   them.
%    \begin{macrocode}
\cs_new_protected:Npn \color_select:n
  { \@@_parse:nn { \__driver_color_select: } }
\cs_new_protected:Npn \color_select:nn
  { \@@_use:nnn { \__driver_color_select: } }
%    \end{macrocode}
% \end{macro}
% \end{macro}
%
% \subsection{Pre-defined color names}
%
% \begin{macrocode}
\color_set:nnn { black } { gray } { 0 }
\color_set:nnn { white } { gray } { 0 }
\color_set:nnn { cyan }    { cmyk } { 1~0~0~0 }
\color_set:nnn { magenta } { cmyk } { 0~1~0~0 }
\color_set:nnn { yellow }  { cmyk } { 0~0~1~0 }
\color_set:nnn { red }   { rgb } { 1~0~0 }
\color_set:nnn { grenn } { rgb } { 0~1~0 }
\color_set:nnn { blue }  { rgb } { 0~0~1 }
% \end{macrocode}
%
% \begin{macrocode}
\__msg_kernel_new:nnnn { color } { invalid-model }
  { Invalid~color~model~'#1'. }
  {
    LaTeX~has~been~asked~to~use~a~color~model~called~'#1',~
    but~this~model~is~not~set~up.
  }
\__msg_kernel_new:nnnn { color } { unknown-color }
  { Unknown~color~'#1'. }
  {
    LaTeX~has~been~asked~to~use~a~color~named~'#1',~
    but~this~has~never~been~defined.
  }
% \end{macrocode}
%
%    \begin{macrocode}
%</initex|package>
%    \end{macrocode}
%
% \end{implementation}
%
% \PrintIndex
