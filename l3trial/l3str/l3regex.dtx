% \iffalse meta-comment
%
%% File: l3regex.dtx Copyright (C) 2011 The LaTeX3 Project
%%
%% It may be distributed and/or modified under the conditions of the
%% LaTeX Project Public License (LPPL), either version 1.3c of this
%% license or (at your option) any later version.  The latest version
%% of this license is in the file
%%
%%    http://www.latex-project.org/lppl.txt
%%
%% This file is part of the "l3trial bundle" (The Work in LPPL)
%% and all files in that bundle must be distributed together.
%%
%% The released version of this bundle is available from CTAN.
%%
%% -----------------------------------------------------------------------
%%
%% The development version of the bundle can be found at
%%
%%    http://www.latex-project.org/svnroot/experimental/trunk/
%%
%% for those people who are interested.
%%
%%%%%%%%%%%
%% NOTE: %%
%%%%%%%%%%%
%%
%%   Snapshots taken from the repository represent work in progress and may
%%   not work or may contain conflicting material!  We therefore ask
%%   people _not_ to put them into distributions, archives, etc. without
%%   prior consultation with the LaTeX3 Project.
%%
%% -----------------------------------------------------------------------
%
%<*driver|package>
\RequirePackage{expl3}
\GetIdInfo$Id$
  {L3 Experimental Regular Expressions}
%</driver|package>
%<*driver>
\documentclass[full]{l3doc}
\usepackage{amsmath}
\begin{document}
  \DocInput{\jobname.dtx}
\end{document}
%</driver>
% \fi
%
%
% \title{^^A
%   The \textsf{l3regex} package: regular expressions in \TeX{}^^A
%   \thanks{This file describes v\ExplFileVersion,
%     last revised \ExplFileDate.}^^A
% }
%
% \author{^^A
%  The \LaTeX3 Project\thanks
%    {^^A
%      E-mail:
%        \href{mailto:latex-team@latex-project.org}
%          {latex-team@latex-project.org}^^A
%    }^^A
% }
%
% \date{Released \ExplFileDate}
%
% \maketitle
%
% \begin{documentation}
%
%   \section{\pkg{l3regex} documentation}
%
%
%   \subsection{User commands and parameters}
%
%   \begin{function}[TF]{\regex_match:nn}
%     \begin{syntax}
%       \cs{regex_match:nnTF} \Arg{regular expression} \Arg{string}
%       ~~\Arg{true code} \Arg{false code}
%     \end{syntax}
%     Tests whether the \meta{string} matches the \meta{regular expression}.
%   \end{function}
%
%   \begin{function}[TF]{\regex_match_extract:nnN}
%     \begin{syntax}
%       \cs{regex_match_extract:nnN} \Arg{regular expression} \Arg{string}
%       ~~\meta{seq variable}
%     \end{syntax}
%     Tests whether the string matches the \meta{regular expression},
%     and stores the submatch information in the \meta{seq variable}.
%   \end{function}
%
%^^A bug: \match { (a)(b)* } { abb } yields "ab" and "b" as submatches:
%^^A the second group's paren moves the first group's one.
%^^A todo: check "a*b*"!
% \end{documentation}
%
% \begin{implementation}
% 
% \section{\pkg{l3regex} implementation}
%
%<*package>
%    \begin{macrocode}
\ProvidesExplPackage
  {\ExplFileName}{\ExplFileDate}{\ExplFileVersion}{\ExplFileDescription}
\RequirePackage{l3str}
%    \end{macrocode}
%
% Most regex engines use backtracking. This allows to provide very
% powerful features (back-references come to mind first), but it is
% costly. Since \TeX{} is not first and foremost a programming language,
% complicated code tends to run slowly, and we must use faster, albeit
% more restrictive, techniques, coming from automata theory.
%
% Given a regular expression of length $n$, we build a
% non-deterministic automaton with $n$ states, which accepts
% precisely those strings matching that regular expression.
% We then run the string through the automaton, and check the return
% value.
%
% \subsection{Helpers}
%
% \begin{macro}{\regex_cs_use:cF}
%   Use the csname provided if it exists, otherwise,
%   fall back to the default.
%    \begin{macrocode}
\cs_new:Npn \regex_cs_use:cF #1
  { \cs_if_exist:cTF {#1} { \use:c {#1} } }
%    \end{macrocode}
% \end{macro}
%
% \subsection{Constants and variables}
%
% \begin{variable}{\l_regex_regex_str,\l_regex_string_str}
%   Store the regex and the string that being tested.
%    \begin{macrocode}
\str_new:N \l_regex_regex_str
\str_new:N \l_regex_string_str
%    \end{macrocode}
% \end{variable}
%
%^^A todo: initialize properly.
%    \begin{macrocode}
\seq_new:N \l_regex_left_int_seq
\seq_new:N \l_regex_right_int_seq
\int_new:N \l_regex_capturing_group_int
\int_new:N \l_regex_left_int
\int_new:N \l_regex_right_int
\int_new:N \l_regex_max_int
\seq_new:N \l_regex_tmpa_seq
\tl_new:N \g_regex_maybe_escape_tl
\tl_new:N \g_regex_tmpa_tl
\tl_new:N \l_regex_tmpc_tl
\int_new:N \l_regex_current_char_int
\tl_new:N \l_regex_end_build_tl
%    \end{macrocode}
%
% \begin{macro}{\regex_setup:}
%    \begin{macrocode}
\cs_new_protected_nopar:Npn \regex_setup:
  {
    \int_set:Nn \tex_escapechar:D { 92 }
    \int_set:Nn \l_regex_left_int {1}
    \int_set:Nn \l_regex_right_int {1}
    \int_set:Nn \l_regex_max_int {1}
    \seq_clear:N \l_regex_left_int_seq
    \seq_clear:N \l_regex_right_int_seq
    \tl_gclear:N \g_regex_maybe_escape_tl
  }
%    \end{macrocode}
% \end{macro}
%
% \subsection{Regex matching}
%
% \begin{macro}[TF]{\regex_match:nn}
%    \begin{macrocode}
\cs_new:Npn \regex_if_track_submatches:T { \ERROR }
\prg_new_protected_conditional:Npnn \regex_match:nn #1#2 { T , F , TF }
  {
    \regex_match_aux_return:nnnn {#1} {#2}
      { \cs_set_eq:NN \regex_if_track_submatches:T \use_none:n }
      { }
  }
\prg_new_protected_conditional:Npnn
    \regex_match_extract:nnN #1#2#3 { T , F , TF }
  {
    \regex_match_aux_return:nnnn {#1} {#2}
      { \cs_set_eq:NN \regex_if_track_submatches:T \use:n }
      { \seq_set_eq:NN #3 \g_regex_submatches_seq }
  }
\cs_new_protected:Npn \regex_match_aux_return:nnnn #1#2#3#4
  {
    \group_begin:
      \regex_setup:
      #3
      \str_set:Nn \l_regex_regex_str  {#1}
      \str_set:Nn \l_regex_string_str {#2}
%<trace>      \regex_trace_input:
      \regex_build_automaton:
      \regex_run_automaton:
      \regex_if_track_submatches:T { \regex_extract_submatches: }
%<trace>      \regex_trace_success:
      \int_compare:nNnTF { \l_regex_successful_thread_int } = { 0 }
        {
          \group_end:
          \prg_return_false:
        }
        {
          \group_end:
          #4
          \prg_return_true:
        }
  }
%    \end{macrocode}
% \end{macro}
%
%
% \subsection{Building the automaton}
%
% \subsubsection{Adding transitions}
%
% \begin{macro}{\regex_toks_put_right:nx}
%    \begin{macrocode}
\cs_new_protected:Npn \regex_toks_put_right:nx #1#2
  {
    \use:x { \regex_toks_put_right:nn {#1} {#2} }
  }
\cs_new_protected:Npn \regex_toks_put_right:nn #1 #2
  {
    \tex_toks:D \int_eval:w #1 \int_eval_end:
      \exp_after:wN {
        \tex_the:D \tex_toks:D \int_eval:w #1 \int_eval_end: #2
      }
  }
%    \end{macrocode}
% \end{macro}
%
% \begin{macro}{\regex_put_free:nn}
%   Start and end state. We don't check whether the transition
%   is already there.
%    \begin{macrocode}
\cs_new_protected:Npn \regex_put_free:nn #1#2
  {
    \regex_toks_put_right:nx {#1}
      { \regex_thread_copy:nn { \int_eval:n {#1} } { \int_eval:n {#2} } }
  }
%    \end{macrocode}
% \end{macro}
%
% \subsubsection{Generic frame to build the automaton}
%
% \begin{macro}{\regex_build_automaton:}
%   Read characters one at a time
%    \begin{macrocode}
\cs_new_protected_nopar:Npn \regex_build_automaton:
  {
    \exp_after:wN \regex_build_loop:N \l_regex_regex_str
    \q_recursion_tail \q_recursion_stop
    \regex_toks_put_right:nx { \l_regex_right_int }
      { \regex_thread_success:n { \int_use:N \l_regex_right_int } }
    \l_regex_end_build_tl
%<*trace>
    \regex_trace:x { \\ Automaton~built~from:~\tl_to_str:N \l_regex_regex_str }
    \prg_stepwise_inline:nnnn {1} {1} {\l_regex_max_int}
      {
        \regex_trace_wrap:nnx { \ \ \ \ } {4}
          { ##1 ~ > ~ \tex_the:D \tex_toks:D ##1 ~ }
      }
%</trace>
  }
%    \end{macrocode}
% \end{macro}
%
% \begin{macro}{\regex_build_loop:N}
%   If we find \cs{q_recursion_tail} end the recursion.
%   A backslash also triggers a special behaviour.
%    \begin{macrocode}
\cs_new_protected_nopar:Npn \regex_build_loop:N #1
  {
    \quark_if_recursion_tail_stop:N #1
    \tl_gset_eq:NN \g_regex_tmpa_tl \g_regex_maybe_escape_tl
    \tl_gclear:N \g_regex_maybe_escape_tl
    \regex_cs_use:cF
      { regex_build_ \g_regex_tmpa_tl #1: }
      { \regex_build_normal:N #1 }
    \regex_build_loop:N
  }
%    \end{macrocode}
% \end{macro}
%
% \begin{macro}{\regex_build_\:}
%   Set the \enquote{maybe escape} token list to be
%   a slash (rather than a backslash, because it is easier
%   to get with \TeX{}'s usual catcode r\'egime), and loop.
%    \begin{macrocode}
\cs_new_protected_nopar:cpn { regex_build_ \c_backslash_str : }
  { \tl_gset:Nx \g_regex_maybe_escape_tl { / } }
%    \end{macrocode}
% \end{macro}
%
% \begin{macro}{\regex_build_normal:N}
%   When we see a normal letter or an escaped non-letter
%   (actually, any unknown combination), we simply add
%   a transition with that letter from the last node
%   to a new node.
%    \begin{macrocode}
\cs_new_protected_nopar:Npn \regex_build_normal:N #1
  {
    \regex_new_state:
    \regex_toks_put_right:nx { \l_regex_left_int }
      {
        \exp_not:N \if_num:w `#1 = \l_regex_current_char_int
          \regex_thread_goto:nn
            { \int_use:N \l_regex_left_int }
            { \int_use:N \l_regex_right_int }
        \exp_not:N \else:
          \regex_thread_fail:n
            { \int_use:N \l_regex_left_int }
        \exp_not:N \fi:
      }
  }
\cs_new_protected_nopar:Npn \regex_new_state:
  {
    \int_set_eq:NN \l_regex_left_int \l_regex_right_int
    \int_incr:N \l_regex_max_int
    \tex_toks:D   \l_regex_max_int = { }
    \tex_skip:D   \l_regex_max_int = 0 pt minus -1pt \scan_stop:
    \tex_muskip:D \l_regex_max_int = 0 mu minus -1mu \scan_stop:
    \int_set_eq:NN \l_regex_right_int \l_regex_max_int
  }
%    \end{macrocode}
% \end{macro}
%
% \subsubsection{Ignoring spaces}
%
% \begin{macro}{\regex_build_~:}
%   Spaces are ignored in regexes unless escaped.
%   This is necessary, as \eTeX{} inserts a space
%   after control sequences when detokenizing.
%    \begin{macrocode}
\cs_new_protected_nopar:cpn { regex_build_~: } { }
%    \end{macrocode}
% \end{macro}
%
% \subsubsection{Special characters}
%
% \begin{macro}{\regex_build_?:}
%   A question mark means that the last character or group
%   is optional. This is achieved by adding a free transition
%   which skips that group, going directly from the left node
%   to the right node.
%    \begin{macrocode}
\cs_new_protected_nopar:cpn { regex_build_?: }
  { \regex_put_free:nn { \l_regex_left_int } { \l_regex_right_int } }
%    \end{macrocode}
% \end{macro}
%
% \begin{macro}{\regex_build_+:}
%   A plus sign means that the last group must appear at least once,
%   but can be repeated arbitrarily many times. For this, we insert
%   a free transition from the end of the group to its beginning.
%    \begin{macrocode}
\cs_new_protected_nopar:cpn { regex_build_+: }
  { \regex_put_free:nn { \l_regex_right_int } { \l_regex_left_int } }
%    \end{macrocode}
% \end{macro}
%
% \begin{macro}{\regex_build_*:}
%   A star is a combination of \texttt{+} and \texttt{?}. Perhaps not
%   the most efficient implementation, though.
%    \begin{macrocode}
\cs_new_protected_nopar:cpx { regex_build_*: }
  {
    \exp_not:c { regex_build_?: }
    \exp_not:c { regex_build_+: }
  }
%    \end{macrocode}
% \end{macro}
%
% \begin{macro}{\regex_build_(:, \regex_build_):}
% \catcode`\|=12\relax
% \begin{macro}{\regex_build_|:}
% \catcode`\|=13\relax
% \begin{macro}[aux]{\regex_build_begin_alternation:,
%     \regex_build_end_alternation:}
%   Grouping and alternation goes together. First allocate the
%   next available number for the end vertex of the
%   alternation/group and store it on a stack (so that nested
%   alternations work). Then put free transitions to separate
%   all cases of the alternation, then build each branch
%   separately, and merge them to the common end-node.
%    \begin{macrocode}
\cs_new_protected_nopar:cpn { regex_build_(: }
  {
    \regex_new_state:
    \regex_if_track_submatches:T
      {
        \int_incr:N \l_regex_capturing_group_int
        \regex_toks_put_right:nx { \l_regex_left_int }
          {
            \regex_thread_submatch:nn
              { \int_use:N \l_regex_left_int }
              { \int_use:N \l_regex_capturing_group_int < }
          }
        \regex_toks_put_right:nx { \int_use:N \l_regex_right_int }
          {
            \regex_thread_submatch:nn
              { \int_use:N \l_regex_right_int }
              { \int_use:N \l_regex_capturing_group_int > }
          }
      }
    \seq_push:Nx \l_regex_left_int_seq { \int_use:N \l_regex_left_int }
    \seq_push:Nx \l_regex_right_int_seq { \int_use:N \l_regex_right_int }
    \regex_build_begin_alternation:
  }
\cs_new_protected_nopar:cpn { regex_build_|: }
  {
    \regex_build_end_alternation:
    \regex_build_begin_alternation:
  }
\cs_new_protected_nopar:cpn { regex_build_): }
  {
    \regex_build_end_alternation:
    \seq_pop:NN \l_regex_right_int_seq \l_regex_tmpb_tl
    \int_set:Nn \l_regex_right_int { \l_regex_tmpb_tl }
    \seq_pop:NN \l_regex_left_int_seq \l_regex_tmpb_tl
    \int_set:Nn \l_regex_left_int { \l_regex_tmpb_tl }
    \tl_put_right:Nx \l_regex_end_build_tl
      {
        \regex_toks_put_right:nx { \int_use:N \l_regex_left_int }
          { \regex_thread_obsolete:n { \int_use:N \l_regex_left_int } }
      }
  }
%    \end{macrocode}
%   Building each branch.
%    \begin{macrocode}
\cs_new_protected_nopar:Npn \regex_build_begin_alternation:
  {
    \regex_new_state:
    \seq_get:NN \l_regex_left_int_seq \l_regex_tmpb_tl
    \int_set:Nn \l_regex_left_int { \l_regex_tmpb_tl }
    \regex_put_free:nn { \l_regex_left_int } { \l_regex_right_int }
  }
\cs_new_protected_nopar:Npn \regex_build_end_alternation:
  {
    \seq_get:NN \l_regex_right_int_seq \l_regex_tmpb_tl
    \regex_toks_put_right:nx {\l_regex_right_int}
      {
        \regex_thread_move:nn
          { \int_use:N \l_regex_right_int }
          { \l_regex_tmpb_tl }
      }

  }
%    \end{macrocode}
% \end{macro}
% \end{macro}
% \end{macro}
%
% \subsubsection{Escaped letters: character classes}
%
% \begin{macro}{\regex_build_.:}
%   Similar to \cs{regex_build_normal:N} but accepts any character,
%   and refuses $-1$, which marks the end of the string.
%    \begin{macrocode}
\cs_new_protected_nopar:cpn { regex_build_.: }
  {
    \regex_new_state:
    \regex_toks_put_right:nx { \l_regex_left_int }
      {
        \exp_not:N \if_num:w -1 < \l_regex_current_char_int
          \regex_thread_goto:nn
            { \int_use:N \l_regex_left_int }
            { \int_use:N \l_regex_right_int }
        \exp_not:N \else:
          \regex_thread_fail:n
            { \int_use:N \l_regex_left_int }
        \exp_not:N \fi:
      }
  }
%    \end{macrocode}
% \end{macro}
%
% \begin{macro}{\regex_build_/d:,\regex_build_/D}
% \begin{macro}{\regex_build_/w:,\regex_build_/W}
% \begin{macro}[aux]{\regex_build_slash_aux:Nnn}
%   Similar to character classes with |[]|, but simpler.
%    \begin{macrocode}
\cs_new_protected_nopar:Npn \regex_build_slash_aux:Nnn #1#2#3
  {
    \regex_new_state:
    \regex_toks_put_right:nx { \l_regex_left_int }
      {
        \exp_not:o #1
        \exp_not:N \regex_break_point:TF
          #2
          {
            \regex_thread_goto:nn
              { \int_use:N \l_regex_left_int }
              { \int_use:N \l_regex_right_int }
          }
          #3
      }
  }
\tl_const:Nn \c_regex_d_tl { \regex_item_range:NN 0 9 }
\tl_const:Nn \c_regex_w_tl
  {
    \regex_item_range:NN A Z
    \regex_item_range:NN a z
    \regex_item_range:NN 0 9
    \regex_item_single:N _
  }
\cs_new_protected_nopar:cpn { regex_build_/d: }
  {
    \regex_build_slash_aux:Nnn \c_regex_d_tl
      { } { \regex_thread_fail:n { \int_use:N \l_regex_left_int } }
  }
\cs_new_protected_nopar:cpn { regex_build_/D: }
  {
    \regex_build_slash_aux:Nnn \c_regex_d_tl
      { \regex_thread_fail:n { \int_use:N \l_regex_left_int } } { }
  }
\cs_new_protected_nopar:cpn { regex_build_/w: }
  {
    \regex_build_slash_aux:Nnn \c_regex_w_tl
      { } { \regex_thread_fail:n { \int_use:N \l_regex_left_int } }
  }
\cs_new_protected_nopar:cpn { regex_build_/W: }
  {
    \regex_build_slash_aux:Nnn \c_regex_w_tl
      { \regex_thread_fail:n { \int_use:N \l_regex_left_int } } { }
  }
%    \end{macrocode}
% \end{macro}
% \end{macro}
% \end{macro}
%
% \subsubsection{Character classes}
%
% \begin{macro}{\regex_build_[:}
%   This starts a class. If the first character
%   is |^|, then the meaning of the class should be inverted.
%   We keep track of this in \cs{l_regex_class_bool}. The
%   code for the class is collected in \cs{l_regex_class_tl}.
%    \begin{macrocode}
\bool_new:N \l_regex_class_bool
\tl_new:N \l_regex_class_tl
\cs_new_protected_nopar:cpn { regex_build_[: } \regex_build_loop:N
  {
    \bool_set_true:N \l_regex_class_bool
    \tl_clear:N \l_regex_class_tl
    \regex_class_first:N
  }
%    \end{macrocode}
% \end{macro}
%
% \begin{macro}{\regex_class_first:N}
%   If the first character is |^|, then the class is inverted.
%    \begin{macrocode}
\cs_new_protected_nopar:Npn \regex_class_first:N #1
  {
    \token_if_eq_charcode:NNTF #1 ^
      {
        \bool_set_false:N \l_regex_class_bool
        \regex_class_loop:N
      }
      { \regex_class_loop:N #1 }
  }
%    \end{macrocode}
% \end{macro}
%
% \begin{macro}{\regex_class_loop:N}
%   We collect characters one by one using \cs{regex_class_loop:N}.
%    \begin{macrocode}
\cs_new_protected_nopar:Npn \regex_class_loop:N #1
  { %^^A todo: error recovery when \quark_if_recursion_tail_stop!
    \quark_if_recursion_tail_stop:N #1
    \tl_gset_eq:NN \g_regex_tmpa_tl \g_regex_maybe_escape_tl
    \tl_gclear:N \g_regex_maybe_escape_tl
    \regex_cs_use:cF
      { regex_class_ \g_regex_tmpa_tl #1: }
      { \regex_class_normal:N #1 }
  }
%    \end{macrocode}
% \end{macro}
%
% \begin{macro}{\regex_class_\string\:}
%   Escaping is done just as we do it outside classes.
%    \begin{macrocode}
\cs_new_protected_nopar:cpn { regex_class_ \c_backslash_str : }
  {
    \tl_gset:Nx \g_regex_maybe_escape_tl { / }
    \regex_class_loop:N
  }
%    \end{macrocode}
% \end{macro}
%
% \begin{macro}{\regex_class_normal:N}
% \begin{macro}{\regex_class_single:N}
%   Most characters are treated here. We look ahead for a dash.
%   If there is none, then the character matches itself.
%    \begin{macrocode}
\cs_new_protected_nopar:Npn \regex_class_normal:N #1 #2
  {
    \token_if_eq_charcode:NNTF #2 -
      { \regex_class_range:NN #1 }
      { \regex_class_single:N #1 #2 }
  }
\cs_new_protected_nopar:Npn \regex_class_single:N #1
  {
    \tl_put_right:Nn \l_regex_class_tl { \regex_item_single:N #1 }
    \regex_class_loop:N
  }
%    \end{macrocode}
% \end{macro}
% \end{macro}
%
% \begin{macro}{\regex_class_range:NN}
% \begin{macro}{\regex_class_range_interrupted:N}
% \begin{macro}{\regex_class_range_escaped:NNN}
%   If the character is followed by a dash, we look for
%   the end-point of the range. In most cases, that's
%   simply |#2|, except in the two special cases |]| and |\|.
%    \begin{macrocode}
\cs_new_protected_nopar:Npn \regex_class_range:NN #1 #2
  {
    \quark_if_recursion_tail_stop:N #2
    \prg_case_int:nnn { `#2 }
      {
        { `\] } { \regex_class_range_interrupted:N #1 #2 }
        { `\\ } { \regex_class_range_escaped:NNN #1 #2 }
      }
      {
        \tl_put_right:Nn \l_regex_class_tl { \regex_item_range:NN #1 #2 }
        \regex_class_loop:N
      }
  }
\cs_new_protected_nopar:Npn \regex_class_range_interrupted:N #1
  {
    \tl_put_right:Nn \l_regex_class_tl
      {
        \regex_item_single:N #1
        \regex_item_single:N -
      }
    \regex_class_loop:N
  }
\cs_new_protected_nopar:Npn \regex_class_range_escaped:NNN #1#2#3
  {
    \quark_if_recursion_tail_stop:N #3
    \cs_if_exist:cTF { regex_class_/#3: }
      { \regex_class_range_interrupted:N #1#2#3 }
      {
        \tl_put_right:Nn \l_regex_class_tl { \regex_item_range:NN #1 #3 }
        \regex_class_loop:N
      }
  }
%    \end{macrocode}
% \end{macro}
% \end{macro}
% \end{macro}
%
% \begin{macro}{\regex_class_]:}
%   At the end of the class, act just as for a single character,
%   but with a more complicated test.
%    \begin{macrocode}
\cs_new_protected_nopar:cpn { regex_class_]: }
  {
    \regex_new_state:
    \regex_toks_put_right:nx
      \l_regex_left_int
      {
        \exp_not:o \l_regex_class_tl
        \bool_if:NTF \l_regex_class_bool
          {
            \exp_not:N \regex_break_point:TF
              {
                \regex_thread_goto:nn
                  { \int_use:N \l_regex_left_int }
                  { \int_use:N \l_regex_right_int }
              }
              { \regex_thread_fail:n { \int_use:N \l_regex_left_int } }
          }
          {
            \exp_not:N \regex_break_point:TF
              { \regex_thread_fail:n { \int_use:N \l_regex_left_int } }
              {
                \regex_thread_goto:nn
                  { \int_use:N \l_regex_left_int }
                  { \int_use:N \l_regex_right_int }
              }
          }
      }
    \regex_build_loop:N
  }
%    \end{macrocode}
% \end{macro}
%
% \begin{macro}{\regex_class_/d:,\regex_class_/D:}
% \begin{macro}{\regex_class_/w:,\regex_class_/W:}
%   Similar to \cs{regex_class_single:N}, adding the appropriate
%   ranges of characters to the class.
%    \begin{macrocode}
\cs_new_protected_nopar:cpn { regex_class_/d: }
  {
    \tl_put_right:Nn \l_regex_class_tl { \regex_item_range:NN 0 9 }
    \regex_class_loop:N
  }
\cs_new_protected_nopar:cpn { regex_class_/D: }
  {
    \tl_put_right:Nn \l_regex_class_tl
      {
        \regex_item_range_number:nn {0}    {`0-1}
        \regex_item_range_number:nn {`9+1} {\c_max_int}
      }
    \regex_class_loop:N
  }
\cs_new_protected_nopar:cpn { regex_class_/w: }
  {
    \tl_put_right:Nn \l_regex_class_tl
      {
        \regex_item_range:NN A Z
        \regex_item_range:NN a z
        \regex_item_single:N \_
      }
    \regex_class_loop:N
  }
\cs_new_protected_nopar:cpn { regex_class_/W: }
  {
    \tl_put_right:Nn \l_regex_class_tl
      {
        \regex_item_range_number:nn {0} {`A-1}
        \regex_item_range_number:nn {`Z+1} {`\_-1}
        \regex_single:N \`
        \regex_item_range_number:nn {`z+1} { \c_max_int }
      }
    \regex_class_loop:N
  }
%    \end{macrocode}
% \end{macro}
%
%
% \begin{macro}{\regex_item_single:N,\regex_item_range:NN}
% \begin{macro}{\regex_item_range_number:nn}
%^^A todo: complete
%    \begin{macrocode}
\cs_new_nopar:Npn \regex_item_single:N #1
  {
    \if_num:w `#1 = \l_regex_current_char_int
      \exp_after:wN \regex_break_true:w
    \fi:
  }
\cs_new_nopar:Npn \regex_item_range:NN #1 #2
  {
    \reverse_if:N \if_num:w `#1 > \l_regex_current_char_int
      \reverse_if:N \if_num:w `#2 < \l_regex_current_char_int
        \exp_after:wN \exp_after:wN \exp_after:wN \regex_break_true:w
      \fi:
    \fi:
  }
\cs_new_nopar:Npn \regex_item_range_number:nn #1 #2
  {
    \reverse_if:N \if_num:w \int_eval:w #1 > \l_regex_current_char_int
      \reverse_if:N \if_num:w \int_eval:w #2 < \l_regex_current_char_int
        \exp_after:wN \exp_after:wN \exp_after:wN \regex_break_true:w
      \fi:
    \fi:
  }
\cs_new_eq:NN \regex_break_point:TF \use_ii:nn
\cs_new_nopar:Npn \regex_break_true:w #1 \regex_break_point:TF #2 #3 {#2}
%    \end{macrocode}
% \end{macro}
% \end{macro}
%
%
% \subsection{Running the NFA}
%
% Once the NFA is built, we read the string one character at a time,
% and run all possible threads through the automaton in parallel.
%
% \begin{macro}{\l_regex_step_int}
%   Counts the number of characters that have been read in the string.
%    \begin{macrocode}
\int_new:N \l_regex_step_int
%    \end{macrocode}
% \end{macro}
%
% \subsubsection{Skips}
%
% The information about which thread lies in which state at a given
% state is stored in skip and muskip registers, alternatively
% referred to as \enquote{A} or \enquote{B} registers. The
% \enquote{A} registers hold the status of threads before reading
% a character, and \enquote{B} registers hold the status after
% reading that character. Once the new state is built in the
% \enquote{B} registers, the meaning of \enquote{A} and \enquote{B}
% are exchanged. We carry on reading the next character of the string.
%
% Each skip register corresponds to one state in the automaton.
% The main component gives the label of the thread that is currently
% in that state of the automaton. The stretch component is a pointer
% to the state of the automaton in which we can find the next thread
% in order of priority. The shrink component is used to mark which
% states have been visited at a given step in the automaton.
%
% Precisely, for \enquote{A} registers,
% \begin{itemize}
% \item a shrink of \cs{l_regex_step_int} means that the
%   corresponding state has been treated already, either because of a
%   thread coming from the previous character, or because of a thread coming
%   from a free transition, which might have clobbered a thread coming from
%   the previous character,
% \item any smaller value means that the corresponding
%   state of the automaton either has not been used yet,
%   or hosts a thread that we haven't taken care of.
% \end{itemize}
% For \enquote{B} registers,
% \begin{itemize}
% \item a shrink of \cs{l_regex_step_int} means that a thread occupies
%   the state already: any later thread coming there will be discarded,
%   because it has lower precedence than the initial one.
% \item any smaller value means that the state is free and untouched at
%   this step yet.
% \end{itemize}
%
% \begin{macro}{\regex_swap_AB:}
% \begin{macro}{\regex_setup_Bmu:,\regex_setup_Amu:}
%   The goal is to swap A and B with respect to skips and muskips.
%    \begin{macrocode}
\cs_new_protected_nopar:Npn \regex_setup_Bmu:
  {
    \cs_set:Npn \regex_Aset:nw ##1 { \tex_skip:D ##1 = }
    \cs_set:Npn \regex_Bset:nw ##1 { \tex_muskip:D ##1 = \etex_gluetomu:D }
    \cs_set_eq:NN \regex_Ause:w \tex_skip:D
    \cs_set:Npn   \regex_Buse:w { \etex_mutoglue:D \tex_muskip:D }
    \cs_set_eq:NN \regex_swap_AB: \regex_setup_Amu:
  }
\cs_new_protected_nopar:Npn \regex_setup_Amu:
  {
    \cs_set:Npn \regex_Aset:nw ##1 { \tex_muskip:D ##1 = \etex_gluetomu:D }
    \cs_set:Npn \regex_Bset:nw ##1 { \tex_skip:D ##1 = }
    \cs_set:Npn   \regex_Ause:w { \etex_mutoglue:D \tex_muskip:D }
    \cs_set_eq:NN \regex_Buse:w \tex_skip:D
    \cs_set_eq:NN \regex_swap_AB: \regex_setup_Bmu:
  }
\cs_new_eq:NN \regex_Aset:nw ?
\cs_new_eq:NN \regex_Bset:nw ?
\cs_new_eq:NN \regex_Ause:w ?
\cs_new_eq:NN \regex_Buse:w ?
\cs_new_eq:NN \regex_swap_AB: ?
%    \end{macrocode}
% \end{macro}
% \end{macro}
%
% \begin{macro}{\regex_Athread:n,\regex_Anext:n}
%   Gets the thread and next step corresponding
%   to a given program step.
%    \begin{macrocode}
\cs_new:Npn \regex_Athread:n #1
  {
    \int_value:w \dim_eval:w
      1 \regex_Ause:w \int_eval:w #1 \int_eval_end:
    \dim_eval_end:
  }
\cs_new:Npn \regex_Anext:n #1
  {
    \int_value:w \dim_eval:w
      \etex_gluestretch:D \regex_Ause:w \int_eval:w #1 \int_eval_end:
    \dim_eval_end:
  }
%    \end{macrocode}
% \end{macro}
%
% \begin{macro}[TF]{\regex_B_if_used:n,\regex_A_if_used:n}
%   See discussion above on how to detect whether registers are free or not.
%    \begin{macrocode}
\prg_new_protected_conditional:Npnn \regex_B_if_used:n #1 { F, TF }
  {
    \if_dim:w \etex_glueshrink:D \regex_Buse:w #1
        = \l_regex_step_int sp \scan_stop:
      \prg_return_true:
    \else:
      \prg_return_false:
    \fi:
  }
\prg_new_protected_conditional:Npnn \regex_A_if_used:n #1 { T, F, TF }
  {
    \if_dim:w \etex_glueshrink:D \regex_Ause:w #1
        = \l_regex_step_int sp \scan_stop:
      \prg_return_true:
    \else:
      \prg_return_false:
    \fi:
  }
%    \end{macrocode}
% \end{macro}
%
% \begin{macro}{\regex_A_make_used:n}
%   Marks the state as being used at this step.
%    \begin{macrocode}
\cs_new_protected:Npn \regex_A_make_used:n #1
  {
    \regex_Aset:nw {#1}
      \regex_Athread:n {#1} sp
      plus \regex_Anext:n {#1} sp
      minus \l_regex_step_int sp
    \scan_stop:
  }
%    \end{macrocode}
% \end{macro}
%
% \begin{macro}{\regex_Bput_thread_in_state:nn}
%   Insert the last state--thread pair with a pointer to the new one:
%   the new pair then becomes the \enquote{last} one.
%    \begin{macrocode}
\cs_new_protected:Npn \regex_Bput_thread_in_state:nn #1 #2
  {
    \regex_Bset:nw { \l_regex_last_state_int }
      \l_regex_last_thread_int sp
      plus \int_eval:n {#2} sp
      minus \l_regex_step_int sp
    \scan_stop:
    \int_set:Nn \l_regex_last_thread_int {#1}
    \int_set:Nn \l_regex_last_state_int {#2}
  }
%    \end{macrocode}
% \end{macro}
%
% \begin{macro}{\regex_Aput_thread_in_state:nn}
%   This is quite different from the \enquote{B} counterpart.
%   Here, we shouldn't overwrite the pointer part of the
%   \enquote{A} register.
%   Set the thread part of a given \enquote{A} register,
%   without changing the pointer.
%    \begin{macrocode}
\cs_new_protected:Npn \regex_Aput_thread_in_state:nn #1 #2
  {
    \regex_Aset:nw {#2}
      #1 sp
      plus  \etex_gluestretch:D \regex_Ause:w #2
      minus \l_regex_step_int sp
    \scan_stop:
  }
%    \end{macrocode}
% \end{macro}
%
% \subsubsection{Framework}
%
% \begin{macro}{\regex_run_automaton:}
%   Read the string one char at a time.
%    \begin{macrocode}
\int_new:N \l_regex_max_thread_int
\int_new:N \l_regex_last_thread_int
\int_new:N \l_regex_last_state_int
\int_new:N \l_regex_successful_thread_int
\prop_new:N \l_regex_submatch_prop
\prop_new:N \l_regex_tmpa_prop
\cs_new_protected_nopar:Npn \regex_run_automaton:
  {
    % all skips have been set to zero
    \regex_setup_Bmu:
    \int_zero:N \l_regex_step_int
    \regex_if_track_submatches:T { \prop_clear:N \l_regex_submatch_prop }
    \int_zero:N \l_regex_successful_thread_int
    \int_zero:N \l_regex_last_thread_int
    \int_zero:N \l_regex_last_state_int
    \int_set:Nn \l_regex_max_thread_int {1}
    \regex_Bput_thread_in_state:nn {1} {1}
    \exp_after:wN \regex_run_loop:N \l_regex_string_str
    \q_recursion_tail \q_recursion_stop
  }
%    \end{macrocode}
% \end{macro}
%
% \begin{macro}{\regex_run_loop:N}
%    \begin{macrocode}
\cs_new_protected_nopar:Npn \regex_run_loop:N #1
  {
    \regex_Bput_thread_in_state:nn {0} {0}
    \regex_swap_AB:
    \int_incr:N \l_regex_step_int
%<*trace>
    \int_zero:N \l_regex_trace_nesting_int
    \tl_clear:N \l_regex_trace_tl
    \int_set:Nn \l_regex_trace_int { \regex_Anext:n {0} }
    \bool_until_do:nn { \int_compare_p:n { \l_regex_trace_int = 0 } }
      {
        \tl_put_right:Nx \l_regex_trace_tl
          {
            Thread~ \regex_Athread:n {\l_regex_trace_int} ~
            in~ state~  \int_use:N \l_regex_trace_int .
            \exp_not:N \\
          }
        \int_set:Nn \l_regex_trace_int
          { \regex_Anext:n { \l_regex_trace_int } }
      }
    \regex_trace:x
      {
        \\ ---- ~ Before~step ~ \int_use:N \l_regex_step_int . ~ ----
        \\ \l_regex_trace_tl
        ---- ~ Read ~ char ~ '\token_to_str:N #1' . ~ ----
      }
%</trace>
    \int_zero:N \l_regex_last_thread_int
    \int_zero:N \l_regex_last_state_int
    \token_if_eq_meaning:NNTF \q_recursion_tail #1
      { \int_set:Nn \l_regex_current_char_int {-1} }
      { \int_set:Nn \l_regex_current_char_int {`#1} }
    \tex_toks:D 0 { \use_none_delimit_by_q_stop:w }
    \regex_state_use:n { \regex_Anext:n {0} }
    \q_stop
    \quark_if_recursion_tail_stop:N #1
    \regex_run_loop:N
  }
%    \end{macrocode}
% \end{macro}
%
% \subsubsection{Let's run}
%
% \begin{macro}{\regex_state_use:n}
%   We make sure to expand \cs{regex_Anext:n} before doing all the stuff
%   in the toks register number \cs{l_regex_current_state_int}.
%    \begin{macrocode}
\int_new:N \l_regex_current_state_int
\cs_new_protected_nopar:Npn \regex_state_use:n #1
  {
    \int_set:Nn \l_regex_current_state_int {#1}
    \use:x
      {
        \regex_state_use_aux:nn
          { \int_use:N \l_regex_current_state_int }
          { \regex_Anext:n { \l_regex_current_state_int } }
      }
  }
\cs_new_protected_nopar:Npn \regex_state_use_aux:nn #1 #2
  {
%<*trace>
    \regex_A_if_used:nTF { \l_regex_current_state_int }
      { \regex_trace:x { State~#1~ignored. } }
      {
        \int_compare:nF { #1 = 0 }
          { \regex_trace:x { State~#1,~thread~\regex_Athread:n{#1}. } }
      }
    \int_incr:N \l_regex_trace_nesting_int
%</trace>
    \regex_A_if_used:nF { \l_regex_current_state_int }
      {
        \regex_A_make_used:n \l_regex_current_state_int
        \tex_the:D \tex_toks:D \l_regex_current_state_int
      }
%<trace>    \int_decr:N \l_regex_trace_nesting_int
    \regex_state_use:n {#2}
  }
%    \end{macrocode}
% \end{macro}
%
% \subsubsection{Instructions}
%
% Four basic instructions:
% \begin{itemize}
% \item \cs{regex_thread_goto:nn} for arrows which \enquote{cost}
%   one character, coming from single characters or character classes;
% \item \cs{regex_thread_fail:n} when the current character
%   does not match the required character for costly arrows;
% \item \cs{regex_thread_obsolete:n} similar to fail, but when
%   a thread is clobbered by one with higher precedence.
% \item \cs{regex_thread_copy:nn} for \enquote{free} arrows,
%   coming from modifiers such as |*| or from alternations;
% \item \cs{regex_thread_move:nn} similar to \enquote{copy},
%   but for cases where there is only one arrow, hence
%   no copying is needed;
% \item \cs{regex_thread_success:n} when reaching the end
%   of the regular expression.
% \end{itemize}
%
%
% \begin{macro}{\regex_thread_goto:nn}
%   Arguments: current state, new state. If the new state
%   has already appeared in a thread with higher priority,
%   do nothing, otherwise introduce the thread in the
%   \enquote{B} register corresponding to the state.
%    \begin{macrocode}
\int_new:N \l_regex_tmpa_int
\cs_new_protected_nopar:Npn \regex_thread_goto:nn #1 #2
  {
%<*trace>
    \regex_B_if_used:nTF {#2}
      { \regex_trace:x { skipped~'goto~#2' } }
      { \regex_trace:x { goto~#2 } }
%</trace>
    \regex_B_if_used:nF {#2}
      { \regex_Bput_thread_in_state:nn { \regex_Athread:n {#1} } {#2} }
  }
%    \end{macrocode}
% \end{macro}
%
% \begin{macro}{\regex_thread_fail:n}
%   When we are not tracking submatches, nothing needs to be done
%   for failed threads.
%    \begin{macrocode}
\cs_new_protected:Npn \regex_thread_fail:n #1
  {
%<trace>    \regex_trace:x { fail }
    \regex_if_track_submatches:T
      {
        \exp_args:NNf \prop_pop:NnN \l_regex_submatch_prop
          { \regex_Athread:n {#1} } \l_regex_tmpb_tl
      }
  }
%    \end{macrocode}
% \end{macro}
%
% \begin{macro}{\regex_thread_obsolete:n}
%   ^^A for efficiency, we should reuse the thread number in copy.
%    \begin{macrocode}
\cs_new_protected:Npn \regex_thread_obsolete:n #1
  {
%<trace>    \regex_trace:x { obsolete~thread~\regex_Athread:n{#1} }
    \regex_if_track_submatches:T
      {
        \exp_args:NNf \prop_pop:NnN \l_regex_submatch_prop
          { \regex_Athread:n {#1} } \l_regex_tmpb_tl
      }
  }
%    \end{macrocode}
% \end{macro}
%
% \begin{macro}{\regex_thread_copy:nn}
%   Arguments: \meta{current state}, \meta{new state}. If we have
%   already met a thread in the \meta{new state}, then going there
%   is now useless (since the earlier thread had higher precedence).
%   Otherwise, we should fail the current thread %^^A unfinished sentence
%   The new state may appear
%   later in the chained list, so we should not break the link
%   to the next step of the chain. First set the thread to a new
%   thread, do the action, and finally inhibit any later action
%   of that state.
%    \begin{macrocode}
\cs_new_protected:Npn \regex_thread_copy:nn #1 #2
  {
%<*trace>
    \regex_A_if_used:nTF {#2}
      { \regex_trace:x { skipped~'copy~to~#2' } }
      {
        \regex_trace:x
          { create~thread~\int_eval:n { \l_regex_max_thread_int + 1 } }
      }
%</trace>
    \regex_A_if_used:nF {#2}
      {
        \if_dim:w \etex_glueshrink:D \regex_Ause:w #2
            = \int_eval:w \l_regex_step_int - 1 sp \scan_stop:
          \regex_thread_obsolete:n {#2}
        \fi:
        \int_incr:N \l_regex_max_thread_int
        \regex_Aput_thread_in_state:nn { \l_regex_max_thread_int } {#2}
        \regex_if_track_submatches:T
          {
            \use:x
              {
                \regex_thread_copy_submatch:nn
                  { \regex_Athread:n {#1} }
                  { \int_use:N \l_regex_max_thread_int }
              }
          }
%<trace>        \regex_trace_the_toks:n {#2}
%<!trace>        \tex_the:D \tex_toks:D #2 \scan_stop:
      }
  }
\cs_new_protected:Npn \regex_thread_copy_submatch:nn #1 #2
  {
    \prop_split:NnTF \l_regex_submatch_prop {#1}
      { \regex_thread_copy_submatch_aux:nnnnn {#1} {#2} }
      { }
  }
\cs_new_protected:Npn \regex_thread_copy_submatch_aux:nnnnn #1#2#3#4#5
  {
    \tl_set:Nn \l_regex_submatch_prop
      { #3 #1 \q_prop {#4} \q_prop #2 \q_prop {#4} \q_prop #5 }
  }
%    \end{macrocode}
% \end{macro}
%
% \begin{macro}{\regex_thread_move:nn}
%   Similar to \cs{regex_thread_copy:nn} for the end of an alternation:
%   it is then useless to copy the thread, and we can simply move it
%   to the state at the end of the alternation, since there is only one
%   free transition.
%    \begin{macrocode}
\cs_new_protected:Npn \regex_thread_move:nn #1 #2
  {
%<*trace>
    \regex_A_if_used:nTF {#2}
      { \regex_trace:x { skipped~'move~to~#2' } }
      { \regex_trace:x { move~to~#2 } }
%</trace>
    \regex_A_if_used:nF {#2}
      {
        \regex_Aput_thread_in_state:nn { \regex_Athread:n {#1} } {#2}
%<trace>        \regex_trace_the_toks:n {#2}
%<!trace>        \tex_the:D \tex_toks:D #2 \scan_stop:
      }
  }
%    \end{macrocode}
% \end{macro}
%
% \begin{macro}{\regex_thread_success:n}
%   Mark the thread as successful, and forget any previous successful thread.
%^^A memory leak: using use_none_delimit_by_q_stop means that we
%^^A don't fail all subsequent threads.
%    \begin{macrocode}
\cs_new_protected_nopar:Npn \regex_thread_success:n #1
  {
%<trace>    \regex_trace:x { success }
    \int_set:Nn \l_regex_successful_thread_int { \regex_Athread:n {#1} }
    \use_none_delimit_by_q_stop:w
  }
%    \end{macrocode}
% \end{macro}
%
% \subsection{Extract submatches}
%
% \begin{macro}{\regex_thread_submatch:nn}
%   Nested props. Somewhat messy.
%    \begin{macrocode}
\cs_new_protected:Npn \regex_thread_submatch:nn #1 #2
  {
    \use:x
      {
        \regex_thread_submatch_aux:nnn
          { \regex_Athread:n {#1} }
          {#2}
          { \int_eval:n { \l_regex_step_int - 1 } }
      }
  }
\cs_new_protected:Npn \regex_thread_submatch_aux:nnn #1 #2 #3
  {
%<trace>    \regex_trace:x { submatch~thread~#1~nesting~#2~char~#3 }
    \prop_split:NnTF \l_regex_submatch_prop {#1}
      { \regex_thread_submatch_aux_ii:nnnnnn {#1} {#2} {#3} }
      {
        \tl_put_right:Nn \l_regex_submatch_prop
          { #1 \q_prop { \q_prop #2 \q_prop {#3} \q_prop } \q_prop }
      }
  }
\cs_new_protected:Npn \regex_thread_submatch_aux_ii:nnnnnn #1#2#3#4#5#6
  {
    \tl_set:Nn \l_regex_tmpa_prop {#5}
    \prop_put:Nnn \l_regex_tmpa_prop {#2} {#3}
    \tl_set:Nx \l_regex_submatch_prop
      {
        \exp_not:n {#4}
        #1
        \exp_not:N \q_prop
        { \exp_not:o \l_regex_tmpa_prop }
        \exp_not:N \q_prop
        \exp_not:n {#6}
      }
  }
%    \end{macrocode}
% \end{macro}
%
% \begin{macro}{\regex_extract_submatches:}
%    \begin{macrocode}
\seq_new:N \g_regex_submatches_seq
\cs_new_protected_nopar:Npn \regex_extract_submatches:
  {
    \seq_gclear:N \g_regex_submatches_seq
    \exp_args:NNV \prop_get:NnNT
      \l_regex_submatch_prop
      \l_regex_successful_thread_int
      \l_regex_tmpa_prop
      {
%<trace>        \prop_show:N \l_regex_tmpa_prop
        \prg_stepwise_inline:nnnn
          {1} {1} { \l_regex_capturing_group_int }
          {
            \prop_pop:NnNTF \l_regex_tmpa_prop { ##1 < } \l_regex_tmpb_tl
              {
                \prop_pop:NnNTF \l_regex_tmpa_prop { ##1 > } \l_regex_tmpc_tl
                  {
                    \seq_gput_right:Nx \g_regex_submatches_seq
                      {
                        \str_from_to:Nnn \l_regex_string_str
                          { \l_regex_tmpb_tl }
                          { \l_regex_tmpc_tl }
                      }
                  }
                  { \seq_gput_right:Nn \g_regex_submatches_seq { } }
              }
              { \seq_gput_right:Nn \g_regex_submatches_seq { } }
          }
      }
  }
%    \end{macrocode}
% \end{macro}
%
% \textbf{Old and wrong.}%^^A This whole section is a mess.
%
% At the end of the day, if the pattern matches the string,
% we want to know how it matched, and extract the various
% parts of the pattern. There are often many ways of matching,
% only the \enquote{leftmost longest} is kept.\footnote{Todo: clarify.}
% Again, we go through the string one character at a time,
% keeping track of \enquote{threads}: each thread corresponds
% to one possible way of matching the partial string with
% the beginning of the pattern. Threads are kept synchronous
% in the string, but are at different states of the NFA.
% In fact, at each step in the string, we only need to keep
% one thread per state in the NFA: since we do not support
% back-references, the previous execution will not influence
% the match, and we can keep the thread with highest priority.
%
% What do we have to keep track of? Threads are organized
% as a simply chained list. Each thread must thus know
% \begin{itemize}
% \item what the next thread is,
% \item what NFA instruction it is at,
% \item what submatches it has found.
% \end{itemize}
% We also need to keep track at each step of which
% program instructions already have a thread. Lastly,
% the NFA itself must be stored.
%
% There are several channels to store data:
% \begin{itemize}
% \item \cs{dimen} registers,
% \item \cs{skip} registers,
% \item \cs{muskip} registers,
% \item \cs{toks} registers,
% \item sequences or property lists.
% \end{itemize}
% How we organize our data will of course influence the
% performance of the algorithm. Let us denote by $n$ the
% size of the NFA and $L$ the length of the string. The
% number of threads $t$ is always less than or equal to $n$,
% and changes at each step. The number $g$ of capturing groups
% is also less than $n$, and often much smaller.
%
% The NFA must be stored either in \cs{toks} registers
% (access in $O(1)$), or in a property list (access in $O(n)$).
% We have the same alternative for submatches, either
% toks in $O(g)$ or property list in $O(gt)$.
%
% For each character in the string, we loop over the chained list
% of threads. For each of those threads, we read the corresponding
% instruction of the program. If this is a \enquote{split}
% instruction, copies of the current thread are created and inserted 
% in the chained list. Submatches found so far are also copied (maybe?).
% If this is a character instruction, the current character is tested,
% and if it matches, we move the thread to the next instruction in
% the program.
%
% Three main costs: looking up the program instruction,
% storing submatch information, and killing a thread
% (removing its submatch information is the slow bit).
%
% All the times below are wrong: I forgot the time to kill a thread
% and remove its submatch info. Let $l$ be the typical life-span
% of a thread. Then at each char $t/l$ threads die. Similarly, at
% each char, $gt/n$ threads require us to store submatch information,
% and the total amount of submatch information is $gt$.
% \begin{itemize}
% \item NFA as toks, submatches as prop.
%   Looking up the NFA costs $Lt$ in total.
%   Submatches storage costs $Lg^2t^2/n$.
%   Killing costs $Lgt^2/l$.
%   Copying costs $Lgt$.
% \item NFA as prop, submatches as toks.
%   [Impossible! the number of threads is unbounded.]
%   ^^A Looking up the NFA costs $Lnt$.
%   ^^A Submatches storage costs $Lg^2t/n$
%   ^^A Killing costs $Lgt/l$.
%   ^^A Copying costs $Lgt/l$.
% \item NFA as toks, submatches as toks, threadwise.
%   [Impossible! the number of threads is unbounded.]
%   ^^A Looking up the NFA costs $Lgt$.
%   ^^A Submatches storage costs $Lg^2t/n$
%   ^^A Killing costs $Lgt/l$
%   ^^A Copying costs $Lgt/l$
% \item NFA as toks, submatches at the corresponding program instruction.
%   Looking up the NFA costs $Lt$.
%   Submatches storage costs $Lgt^2/n$.
%   Killing costs $Lgt^2/l$.
%   Copying costs $Lgt^2/l$.
% \end{itemize}
% It looks as if the last one might be a faster option than the first,
% but it also looks quite messy, and requires to keep track of another
% list for each thread, so things are not too clear.
%
% I'll go for the first method, storing program instructions
% in \TeX{}'s toks registers, and submatch information
% in a property list. To each program instruction is associated:
% \begin{itemize}
% \item the thread number,
% \item the program instruction at which
%   the next thread in the chained list is.
% \end{itemize}
% This information is stored as the normal and stretch components
% of a \cs{skip} or \cs{muskip} register (alternating depending
% on the step).
%
%
%
%    \begin{macrocode}
%</package>
%    \end{macrocode}
%
% \section{Code specific to tracing}
%
%    \begin{macrocode}
%<*trace>
%    \end{macrocode}
%
% \begin{macro}{\l_regex_trace_tl,\l_regex_trace_nesting_int,\l_regex_trace_int}
%    \begin{macrocode}
\tl_new:N \l_regex_trace_tl
\int_new:N \l_regex_trace_int
\int_new:N \l_regex_trace_nesting_int
\tl_new:N \l_regex_trace_line_start_tl
%    \end{macrocode}
% \end{macro}
%
% \begin{macro}{\regex_trace:x}
%   Trace, adding \enquote{(regex)} to each new line,
%   and setting up the control space to produce a space.
%    \begin{macrocode}
\cs_new_protected:Npn \regex_trace:x #1
  {
    \group_begin:
      \cs_set_eq:NN \ \c_space_tl
      \tl_set:Nx \l_regex_trace_line_start_tl
        {
          (regex) ~
          \prg_replicate:nn { \l_regex_trace_nesting_int } { \ \  }
        }
      \cs_set_nopar:Npx \\ { \iow_newline: \l_regex_trace_line_start_tl }
      \iow_term:x { \l_regex_trace_line_start_tl #1}
    \group_end:
  }
\cs_new_protected:Npn \regex_trace_wrap:nnx #1 #2 #3
  { \iow_wrap:xnnnN { #3 } { (regex)~ #1 } { 8 + #2 } { } \regex_trace:x }
%    \end{macrocode}
% \end{macro}
%
% \begin{macro}{\regex_trace_the_toks:n}
%    \begin{macrocode}
\cs_new_protected:Npn \regex_trace_the_toks:n #1
  {
    \regex_trace:x { State~#1,~thread~\regex_Athread:n{#1}. }
    \int_incr:N \l_regex_trace_nesting_int
    \tex_the:D \tex_toks:D #1 \scan_stop:
    \int_decr:N \l_regex_trace_nesting_int
  }
%    \end{macrocode}
% \end{macro}
%
% \begin{macro}{\regex_trace_input:}
%    \begin{macrocode}
\cs_new_protected_nopar:Npn \regex_trace_input:
  {
    \regex_trace:x
      {
        Task:~match~the~regex~'\l_regex_regex_str'\\
        in~'\l_regex_string_str'.
      }
  }
%    \end{macrocode}
% \end{macro}
%
% \begin{macro}{\regex_trace_success:}
%   Used at one place to print which thread was successful.
%    \begin{macrocode}
\cs_new_protected_nopar:Npn \regex_trace_success:
  {
    \int_zero:N \l_regex_trace_nesting_int
    \regex_trace:x
      { \\ Successful~thread:~ \int_use:N \l_regex_successful_thread_int }
  }
%    \end{macrocode}
% \end{macro}
%
%    \begin{macrocode}
%</trace>
%    \end{macrocode}
%
% \end{implementation}
%
% \endinput
