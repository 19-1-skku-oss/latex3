% \iffalse meta-comment
%
%% File: l3regex.dtx Copyright (C) 2011 The LaTeX3 Project
%%
%% It may be distributed and/or modified under the conditions of the
%% LaTeX Project Public License (LPPL), either version 1.3c of this
%% license or (at your option) any later version.  The latest version
%% of this license is in the file
%%
%%    http://www.latex-project.org/lppl.txt
%%
%% This file is part of the "l3trial bundle" (The Work in LPPL)
%% and all files in that bundle must be distributed together.
%%
%% The released version of this bundle is available from CTAN.
%%
%% -----------------------------------------------------------------------
%%
%% The development version of the bundle can be found at
%%
%%    http://www.latex-project.org/svnroot/experimental/trunk/
%%
%% for those people who are interested.
%%
%%%%%%%%%%%
%% NOTE: %%
%%%%%%%%%%%
%%
%%   Snapshots taken from the repository represent work in progress and may
%%   not work or may contain conflicting material!  We therefore ask
%%   people _not_ to put them into distributions, archives, etc. without
%%   prior consultation with the LaTeX3 Project.
%%
%% -----------------------------------------------------------------------
%
%<*driver|package>
\RequirePackage{expl3}
\GetIdInfo$Id$
  {L3 Experimental Regular Expressions}
%</driver|package>
%<*driver>
\documentclass[full]{l3doc}
\usepackage{amsmath}
\begin{document}
  \DocInput{\jobname.dtx}
\end{document}
%</driver>
% \fi
%
% \title{^^A
%   The \textsf{l3regex} package: regular expressions in \TeX{}^^A
%   \thanks{This file describes v\ExplFileVersion,
%     last revised \ExplFileDate.}^^A
% }
%
% \author{^^A
%  The \LaTeX3 Project\thanks
%    {^^A
%      E-mail:
%        \href{mailto:latex-team@latex-project.org}
%          {latex-team@latex-project.org}^^A
%    }^^A
% }
%
% \date{Released \ExplFileDate}
%
% \maketitle
%
% \begin{documentation}
% \newenvironment{l3regex-syntax}
%   {\begin{itemize}\def\makelabel##1{\hss\llap{\ttfamily\string##1}}}
%   {\end{itemize}}
%
% \section{\pkg{l3regex} documentation}
%
% The \pkg{l3regex} package provides regular expression testing,
% extraction of submatches, splitting, and replacement, all acting on strings
% of characters. The syntax of regular expressions is mostly a subset
% of the PCRE syntax (and very close to POSIX). For performance
% reasons, only a limited set of features are implemented. Notably,
% back-references are not supported.
%
% Let us give a few examples. After
% \begin{verbatim}
%   \str_set:Nn \l_my_str { That~cat. }
%   \regex_replace_once:nnN { at } { is } \l_my_str
% \end{verbatim}
% the string variable \cs{l_my_str} holds the text
% \enquote{\texttt{This cat.}}, where the first
% occurrence of \enquote{\texttt{at}} was replaced
% by \enquote{\texttt{is}}. A more complicated example is
% a pattern to add a comma at the end of each word:
% \begin{verbatim}
%   \regex_replace_all:nnN { (\w+) } { \1 , } \l_my_str
% \end{verbatim}
% The |\w| sequence represents any \enquote{word} character,
% and |*| indicates that the |\w| sequence should be repeated
% as many times as possible, hence matching a word in the
% input string. The parentheses \enquote{capture} what their
% contents matched in the input string, and this can be used
% in the replacement text as |\1| (and higher numbers if several
% groups are used in the regular expression).
%
% \subsection{Syntax of regular expressions}
%
% Most characters match exactly themselves. Some characters are
% special and must be escaped with a backslash (\emph{e.g.}, |\*|
% matches an explicit star character). Some escape sequences of
% the form backslash--letter also have a special meaning
% (for instance |\d| matches any digit). As a rule,
% \begin{itemize}
% \item every alphanumeric character (\texttt{A}--\texttt{Z},
%   \texttt{a}--\texttt{z}, \texttt{0}--\texttt{9}) matches
%   exactly itself, none of them should be escaped, because
%   most of those escape sequences have special meanings;
% \item non-alphanumeric printable ascii characters can always
%   be safely escaped, and for highest portability, they should
%   always be escaped (this avoids problems if some currently
%   \enquote{normal} characters is given a meaning in later
%   releases);
% \item spaces should always be escaped (even in character
%   classes);
% \item any other character may be escaped or not, without any
%   effect: both versions will match exactly that character.
% \end{itemize}
% Note that these rules play nicely with the fact that many
% non-alphanumeric characters are difficult to input into \TeX{}
% under normal category codes. For instance, |\$\%\^\\abc\#|
% matches the literal string |$%^\abc#|.
% \begin{texnote}
%   When converting the regular expression to a string,
%   the value of the escape character is set to be a backslash.
% \end{texnote}
%
% Any special character which appears at a place where its special
% behaviour cannot apply matches itself instead (for instance,
% a quantifier appearing at the beginning of a string).
%
% Characters.
% \begin{l3regex-syntax}
% \item[\x\{hh\ldots{}\}]
%   Character with hex code \texttt{hh\ldots{}}
% \item[\xhh]
%   Character with hex code \texttt{hh}.
% \item[\a] Alarm (hex 07).
% \item[\e] Escape (hex 1B).
% \item[\f] Form-feed (hex 0C).
% \item[\n] New line (hex 0A).
% \item[\r] Carriage return (hex 0D).
% \item[\t] Horizontal tab (hex 09).
% \end{l3regex-syntax}
%
% Character types.
% \begin{l3regex-syntax}
% \item[.] A single period matches any character,
%   including newlines.\footnote{Should that be changed?}
% \item[\N] A character that is not
%   the |\n| character (hex 0A).\footnote{Is that right?}
% \item[\d] Any decimal digit.
% \item[\h] Any horizontal space character,
%   equivalent to |[\^^I\ ]|: space and tab.
% \item[\s] Any space character,
%   equivalent to |[\^^I\^^J\^^L\^^M\ ]|.
% \item[\v] Any vertical space character,
%   equivalent to |[\^^I-\^^M]|. Note that |\^^K| is a vertical space,
%   but not a space, for compatibility with perl.
% \item[\w] Any word character, \emph{i.e.},
%   alpha-numerics and underscore, equivalent to |[A-Za-z0-9\_]|.
% \item[\D] Any character not matched by |\d|.
% \item[\H] Any character not matched by |\h|.
% \item[\S] Any character not matched by |\s|.
% \item[\V] Any character not matched by |\v|.
% \item[\W] Any character not matched by |\w|.
% \end{l3regex-syntax}
%
% Character classes match exactly one character in the subject string.
% \begin{l3regex-syntax}
% \item[{[...]}] Positive character class.
% \item[{[\string^...]}] Negative character class.
% \item[{[x-y]}] Range (can be used with escaped characters).
% \end{l3regex-syntax}
%
% Quantifiers.
% \begin{l3regex-syntax}
% \item[?] $0$ or $1$, greedy.
% \item[??] $0$ or $1$, lazy.
% \item[*] $0$ or more, greedy.
% \item[*?] $0$ or more, lazy.
% \item[+] $1$ or more, greedy.
% \item[+?] $1$ or more, lazy.
% \end{l3regex-syntax}
% Not implemented yet: ^^A !
% \begin{l3regex-syntax}\def\makelabel#1{\hss\llap{\ttfamily#1}}
% \item[\{$n$\}] Exactly $n$.
% \item[\{$n,$\}] $n$ or more, greedy.
% \item[\{$n,$\}?] $n$ or more, lazy.
% \item[\{$n,m$\}] At least $n$, no more than $m$, greedy.
% \item[\{$n,m$\}?] At least $n$, no more than $m$, lazy.
% \end{l3regex-syntax}
%
% Anchors and simple assertions.
% \begin{l3regex-syntax}
% \item[\b] Word boundary.
% \item[\B] Not a word boundary.
% \item[^\string\A] Start of the subject string.\footnote{The
%     multiline mode is not implemented yet, so those are currently
%     identical.}
% \item[$\string\Z\string\z] End of the subject
%   string.\footnotemark[\thefootnote]
% \item[\G] Start of the current match. This is only different from |^|
%   in the case of multiple matches: for instance
%   |\regex_count:nnN { \G a } { aaba } \l_tmpa_int| yields $2$, but
%   replacing |\G| by |^| would result in \cs{l_tmpa_int} holding the
%   value $1$.
% \end{l3regex-syntax}
%
% Alternation and capturing groups.
% \begin{l3regex-syntax}
% \item[A\string|B\string|C] Either one of \texttt{A}, \texttt{B},
%   or \texttt{C}.
% \item[(\ldots{})] Capturing group.
% \item[(?:\ldots{})] Non-capturing group.
% \end{l3regex-syntax}
%
% In character classes, only |^|, |-|, |]|, |\| and spaces are special,
% and should be escaped. Other non-alphanumeric characters can
% still be escaped without harm. The escape sequences |\d|,
% |\D|, |\w|, |\W| are also supported in character classes.
% If the first character is |^|, then the meaning of the character
% class is inverted. Ranges of characters can be expressed using
% |-|, for instance, |[\D 0-5]| is equivalent to |[^6-9]|.
%
% Capturing groups are a means of extracting information about the
% match. Parenthesized groups are labelled in the order of their
% opening parenthesis, starting at $1$. The contents of those groups
% corresponding to the \enquote{best} match (leftmost longest)
% can be extracted and stored in a sequence of strings using for
% instance \cs{regex_extract:nnNTF}.
%
% \subsection{Syntax of in the replacement text}
%
% Most of the features described in regular expressions do not make sense
% within the replacement text. Escaped characters are supported as inside
% regular expressions. The various submatches are accessed with |\1|, |\2|,
% \emph{etc.}, and the whole match is accessed using |\0|.
%
% For instance,
% \begin{verbatim}
% \str_set:Nn \l_my_str { Hello,~world! }
% \regex_replace_all:nnN { ([er]?l|o) . } { \(\0\-\-\1\) } \l_my_str
% \end{verbatim}
% results in \cs{l_my_str} holding |H(ell--el)(o,--o) w(or--o)(ld--l)!|
%
% Submatches with numbers higher than $10$ are accessed in the same way,
% namely |\10|, |\11|, \emph{etc}. To insert in the replacement text
% a submatch followed by a digit, the digit must be entered using the
% |\x| escape sequence: for instance, to get the first submatch followed
% by the digit $7$, use |\1\x37|, because $7$ has character code |37|
% (in hexadecimal).
%
% \subsection{Precompiling regular expressions}
%
% If a regular expression is to be used several times,
% it is better to compile it once rather than doing it
% each time the regular expression is used. The precompiled
% regular expression is stored as a token list variable. All
% of the \pkg{l3regex} module's functions can be given their
% regular expression argument either as an explicit string
% or as a precompiled regular expression.
%
% \begin{function}{\regex_set:Nn}
% \begin{function}{\regex_gset:Nn}
%   \begin{syntax}
%     \cs{regex_set:Nn} \meta{tl var} \Arg{regex}
%   \end{syntax}
%   Stores a precompiled version of the \meta{regular expression}
%   in the \meta{tl var}. For instance, this function can be used
%   as
%   \begin{verbatim}
%     \tl_new:N \l_my_regex_tl
%     \regex_set:Nn \l_my_regex_tl { my\ (simple\ )? reg(ex|ular\ expression) }
%   \end{verbatim}
%   The assignment is local for \cs{regex_set:Nn}
%   and global for \cs{regex_gset:Nn}.
%   \begin{texnote}
%     Precompiled regular expressions can be safely written to a file
%     and read when the \LaTeX3 syntax is active (as triggered by
%     \cs{ExplSyntaxOn}).
%   \end{texnote}
% \end{function}
% \end{function}
%
% \subsection{String matching}
%
% \begin{function}[TF]{\regex_match:nn}
% \begin{function}[TF]{\regex_match:Nn}
%   \begin{syntax}
%     \cs{regex_match:nnTF} \Arg{regex} \Arg{string} \Arg{true code} \Arg{false code}
%   \end{syntax}
%   Tests whether the \meta{regular expression} matches any substring
%   of \meta{string}. For instance,
%   \begin{verbatim}
%     \regex_match:nnTF { b [cde]* } { abecdcx } { TRUE } { FALSE }
%     \regex_match:nnTF { [b-dq-w] } { example } { TRUE } { FALSE }
%   \end{verbatim}
%   leaves \texttt{TRUE FALSE} in the input stream.
% \end{function}
% \end{function}
%
% \begin{function}{\regex_count:nnN}
% \begin{function}{\regex_count:NnN}
%   \begin{syntax}
%     \cs{regex_count:nnN} \Arg{regex} \Arg{string} \meta{int var}
%   \end{syntax}
%   Sets \meta{int var} equal to the number of times
%   \meta{regular expression} appears in \meta{string}.
%   The search starts by finding the left-most longest match,
%   respecting greedy and ungreedy operators. Then the search
%   starts again from the character following the last character
%   of the previous match, until reaching the end of the string.
%   For instance,
%   \begin{verbatim}
%     \int_new:N \l_foo_int
%     \regex_count:nnN { (b+|c) } { abbababcbb } \l_foo_int
%   \end{verbatim}
%   results in \cs{l_foo_int} taking the value $5$.
% \end{function}
% \end{function}
%
% \subsection{Submatch extraction}
%
% \begin{function}{\regex_extract:nnN}
% \begin{function}{\regex_extract:NnN}
% \begin{function}[TF]{\regex_extract:nnN}
% \begin{function}[TF]{\regex_extract:NnN}
%   \begin{syntax}
%     \cs{regex_extract:nnNTF} \Arg{regex} \Arg{string} \meta{seq~var} \Arg{true code} \Arg{false code}
%   \end{syntax}
%   Finds the first match of the \meta{regular expression}
%   in the \meta{string}. If it exists, the match is stored
%   as the zeroeth item of the \meta{seq~var}, and further
%   items are the contents of capturing groups, in the order
%   of their opening left parenthesis. The \meta{seq~var}
%   is assigned locally. If there is no match,
%   the \meta{seq~var} is not altered.
%   The testing versions return \meta{true} if a match was found,
%   and \meta{false} otherwise.
%   For instance, assume that you type
%   \begin{verbatim}
%     \regex_extract:nnNTF { ^(La)?TeX(!*)$ } { LaTeX!!! }
%       \l_foo_seq { true } { false }
%   \end{verbatim}
%   Then the regular expression (anchored at the start with |^| and
%   at the end with |$|) will match the whole string. The first
%   capturing group, |(La)?|, matches |La|, and the second capturing
%   group, |(!*)|, matches |!!!|. Thus, |\l_foo_seq| will contain
%   the items |{LaTeX!!!}|, |{La}|, and |{!!!}|, and the \texttt{true}
%   branch is left in the input stream.
% \end{function}
% \end{function}
% \end{function}
% \end{function}
%
% \subsection{String splitting}
%
% \begin{function}{\regex_split:nnN}
% \begin{function}{\regex_split:NnN}
%   \begin{syntax}
%     \cs{regex_split:nnN} \Arg{regular expression} \meta{string} \meta{seq~var}
%   \end{syntax}
%   Searches the \meta{string} into a sequence of substrings, delimited by
%   matches of the \meta{regular expression}. If the \meta{regular expression}
%   has capturing groups, then the substrings that they match are stored as
%   items of the sequence as well. The assignment to \meta{seq~var} is local.
%   If no match is found the resulting \meta{seq~var} has the \meta{string} as
%   its sole item. If the \meta{regular expression} matches the empty string,
%   then the \meta{string} is split into single character substrings.
% \end{function}
% \end{function}
%
% \subsection{String replacement}
%
% \begin{function}{\regex_replace_once:nnN}
% \begin{function}{\regex_replace_once:NnN}
% \begin{function}[TF]{\regex_replace_once:nnN}
% \begin{function}[TF]{\regex_replace_once:NnN}
%   \begin{syntax}
%     \cs{regex_replace_once:nnN} \Arg{regular expression} \Arg{replacement} \meta{str~var}
%   \end{syntax}
%   Searches for the \meta{regular expression} in the \meta{string}
%   and replaces the matching part with the \meta{replacement}. The result
%   is assigned locally to \meta{str~var}. In the \meta{replacement},
%   |\0| represents the full match, |\1| represent the contents
%   of the first capturing group, |\2| of the second, \emph{etc.}
% \end{function}
% \end{function}
% \end{function}
% \end{function}
%
% \begin{function}{\regex_replace_all:nnN}
% \begin{function}{\regex_replace_all:NnN}
%   \begin{syntax}
%     \cs{regex_replace_all:nnN} \Arg{regular expression} \Arg{replacement} \meta{str~var}
%   \end{syntax}
%   Replaces all occurrences of the \cs{regular expression}
%   in the \meta{string} by the \meta{replacement}, where
%   |\0| represents the full match, |\1|
%   represent the contents of the first capturing group,
%   |\2| of the second, \emph{etc.} Every match
%   is treated independently. The result is assigned
%   locally to \meta{str~var}.
% \end{function}
% \end{function}
%
% \subsection{Bugs, misfeatures, future work, and other possibilities}
%
% The following things should preferably be sorted out before
% shipping the code.
% \begin{itemize}
% \item Move submatch tracking to the \tn{toks} registers along-side
%   with the states of the NFA (a mess, but faster than the current
%   approach). Understand how to avoid keeping submatch information
%   around for longer than strictly necessary (no \enquote{memory leak}).
% \item Bug where mismatched parentheses are wrongly accepted ("a)b(c").
% \item Better error recovery in the case of missing closing parentheses.
% \item Check that all initializing steps are correct.
% \item Actually implement the \texttt{\{\}} quantifiers.
% \end{itemize}
%
% The following need discussion.
% \begin{itemize}
% \item Newline conventions are not done.
%   In particular, |.| should not match newlines.
%   Also, |\A| should differ from |^|, and |\Z|, |\z| and |$| should
%   differ.
% \item Caseless matching and more generally
%   |(*..)| and |(?..)| sequences to set some options.
% \item General look-ahead/behind assertions.
% \item Idea of |#| as a synonym of |.*?|.
% \item Do we need a facility for balanced groups?
% \end{itemize}
%
% The following features are likely to be implemented at some point
% in the future.
% \begin{itemize}
% \item Optimize simple strings: use less states
%   (|abcade| should give two states, for |abc| and |ade|).
% \item Optimize groups with no alternative.
% \item Implement regex matching on external files.
% \item Conditional subpatterns with look ahead/behind: \enquote{if
%     what follows is [\ldots{}], then [\ldots{}]}.
% \item "(?|..|..)" to reset the capturing group number at the start
%   of each alternative.
% \end{itemize}
%
% The following features not present in PCRE nor perl (because they can
% be done in other ways in those languages) might be added.
% \begin{itemize}
% \item |#| as a synonym of |.*?|, in a way similar to macro parameters.
% \item Facility to match balanced groups: this is non-regular, but quite
%   useful, and it can be implemented without too much performance loss.
%   (\emph{cf.} callout?)
% \end{itemize}
%
% The following features of PCRE or perl will probably not be implemented.
% \begin{itemize}
% \item Other forms of escapes, and character classes.
%   |\cx|$\simeq$|\^^x|, |\ddd| (octal \texttt{ddd}).
%   POSIX character classes.
%   |\p{..}| and |\P{..}| for having/not having a Unicode property.
%   |\X| for \enquote{extended} Unicode sequence.
% \item Callout with |(?C...)|.
% \item Conditional subpatterns (other than with a look-ahead
%   or look-behind condition): this is non-regular, isn't it?
% \end{itemize}
%
% The following features of PCRE or perl will not be implemented.
% \begin{itemize}
% \item Comments: \TeX{} already has its own system for comments.
% \item Named subpatterns: \TeX{} programmers have lived so far without
%   any need for named macro parameters.
% \item |\Q...\E| escaping: this would require to read the argument
%   verbatim, which is not in the scope of this module.
% \item Atomic grouping, possessive quantifiers: those tools, mostly
%   meant to fix catastrophic backtracking, are unnecessary in a
%   non-backtracking algorithm, and difficult to implement.
% \item Subroutine calls: this syntactic sugar is difficult to
%   include in a non-backtracking algorithm, in particular because
%   the corresponding group should be treated as atomic.
% \item Recursion: this is a non-regular feature.
% \item Back-references: non-regular feature, this requires backtracking,
%   which is prohibitively slow.
% \item Backtracking control verbs: intrinsically tied to backtracking.
% \item |\K| for resetting the beginning of the match: tied to backtracking.
% \item |\C| single byte in UTF-8 mode: well, UTF-8 mode is not
%   implemented, and that seems like a very rarely useful feature anyways.
% \end{itemize}
%
% \end{documentation}
%
% \begin{implementation}
%
% \section{\pkg{l3regex} implementation}
%
%<*package>
%    \begin{macrocode}
\ProvidesExplPackage
  {\ExplFileName}{\ExplFileDate}{\ExplFileVersion}{\ExplFileDescription}
\RequirePackage{l3str}
%    \end{macrocode}
%
% Most regex engines use backtracking. This allows to provide very
% powerful features (back-references come to mind first), but it is
% costly. Since \TeX{} is not first and foremost a programming language,
% complicated code tends to run slowly, and we must use faster, albeit
% slightly more restrictive, techniques, coming from automata theory.
%
% Given a regular expression of $n$ characters, we build a
% non-deterministic finite automaton (NFA) with roughly $n$ states,
% which accepts precisely those strings matching that regular expression.
% We then run the string through the NFA, and check the return value.
%
% The code is structured as follows. Various helper functions are
% introduced in the next subsection, to limit the clutter in later
% parts. Then functions pertaining to parsing the regular expression
% are introduced: that part is rather long because of the many bells
% and whistles that we need to cater for. The next subsection takes
% care of running the NFA, and describes how the various \TeX{}
% registers are (ab)used in this module. Finally, user functions.
%
% \subsection{Constants and variables}
%
% \begin{macro}{\regex_tmp:w}
% \begin{variable}{\g_regex_tmpa_tl,\l_regex_tmpa_tl,\l_regex_tmpb_tl}
% \begin{variable}{\l_regex_tmpa_int}
% \begin{variable}{\l_regex_tmpa_prop}
%   Temporary variables.
%    \begin{macrocode}
\cs_new:Npn \regex_tmp:w { }
\tl_new:N   \g_regex_tmpa_tl
\tl_new:N   \l_regex_tmpa_tl
\tl_new:N   \l_regex_tmpb_tl
\int_new:N  \l_regex_tmpa_int
\prop_new:N \l_regex_tmpa_prop
%    \end{macrocode}
% \end{variable}
% \end{variable}
% \end{variable}
% \end{macro}
%
% \subsubsection{Variables used while building}
%
% \begin{variable}{\l_regex_pattern_str}
%   The \meta{pattern} is stored in a string variable.
%    \begin{macrocode}
\tl_new:N  \l_regex_pattern_str
%    \end{macrocode}
% \end{variable}
%
% \begin{variable}{\l_regex_state_max_int}
% \begin{variable}{\l_regex_state_left_int,\l_regex_state_right_int}
% \begin{variable}{\l_regex_state_left_seq,\l_regex_state_right_seq}
%   The last state that was allocated is stored in \cs{l_regex_state_max_int},
%   and \cs{l_regex_state_left/right_int} point to both end-points of the
%   last group (which any quantifier would repeat). For simple strings of
%   characters, the left and right pointers only differ by one.
%    \begin{macrocode}
\int_new:N  \l_regex_state_max_int
\int_new:N  \l_regex_state_left_int
\int_new:N  \l_regex_state_right_int
\seq_new:N  \l_regex_state_left_seq
\seq_new:N  \l_regex_state_right_seq
%    \end{macrocode}
% \end{variable}
% \end{variable}
% \end{variable}
%
% \begin{variable}{\l_regex_capturing_group_int}
% \begin{variable}{\l_regex_capturing_group_seq}
%   \cs{l_regex_capturing_group_int} is the id number of the current
%   capturing group, starting at $0$ for a group enclosing the full
%   regular expression, and counting in the order of their left parenthesis.
%   This number is used when a branch of the alternation ends.
%   Capturing groups can be arbitrarily nested, and we keep track of
%   the stack of id numbers in \cs{l_regex_capturing_group_seq}.
%    \begin{macrocode}
\int_new:N  \l_regex_capturing_group_int
\seq_new:N  \l_regex_capturing_group_seq
%    \end{macrocode}
% \end{variable}
% \end{variable}
%
% \begin{variable}{\l_regex_one_or_group_tl}
%   When looking for quantifiers, this variable holds either
%   \enquote{one} or \enquote{group} depending on whether the
%   object to which the quantifier applies matches one character
%   (\emph{i.e.}, is a character or character class), or is a group.
%    \begin{macrocode}
\tl_new:N   \l_regex_one_or_group_tl
%    \end{macrocode}
% \end{variable}
%
% \begin{variable}{l_regex_repetition_int}
%   Used when to distinguish threads at the same state but with
%   different repetitions of some quantifiers.
%    \begin{macrocode}
\int_new:N  \l_regex_repetition_int
%    \end{macrocode}
% \end{variable}
%
% \begin{variable}{\l_regex_look_behind_bool}
% \begin{variable}{\l_regex_look_behind_str}
%   The boolean \cs{l_regex_look_behind_bool} indicates whether
%   a look-behind assertion appears within the regular expression
%   (currently only |\b| or |\B|). When matching, we will keep
%   track of the part of the string before the current character
%   in \cs{l_regex_look_behind_str}, stored backwards.
%    \begin{macrocode}
\bool_new:N \l_regex_look_behind_bool
\tl_new:N \l_regex_look_behind_str
%    \end{macrocode}
% \end{variable}
% \end{variable}
%
% \subsubsection{Character classes}
%
% \begin{macro}{\regex_build_tmp_class:n}
% \begin{variable}{\l_regex_class_bool,\l_regex_class_tl}
%   Used when building character classes.
%    \begin{macrocode}
\cs_new_eq:NN \regex_build_tmp_class:n \use_none:n
\bool_new:N \l_regex_class_bool
\tl_new:N   \l_regex_class_tl
%    \end{macrocode}
% \end{variable}
% \end{macro}
%
% \begin{variable}{\c_regex_d_tl,\c_regex_D_tl}
% \begin{variable}{\c_regex_h_tl,\c_regex_H_tl}
% \begin{variable}{\c_regex_s_tl,\c_regex_S_tl}
% \begin{variable}{\c_regex_v_tl,\c_regex_V_tl}
% \begin{variable}{\c_regex_w_tl,\c_regex_W_tl}
% \begin{variable}{\c_regex_N_tl}
%   These constant token lists encode which characters
%   are recognized by |\d|, |\D|, |\w|, \emph{etc.}
%   in regular expressions. Namely, |\d=[0-9]|,
%   |\w=[0-9A-Z_a-z]|, |\s=[\ \^^I\^^J\^^L\^^M]|,
%   |\h=[\ \^^I]|, |\v=[\^^J-\^^M]|, and the upper-case
%   counterparts match anything that the lowercase
%   does not match.
%   The order in which the various ranges appear is
%   optimized for usual mostly lowercase letter text.
%    \begin{macrocode}
\tl_const:Nn \c_regex_d_tl
  {
    \regex_item_range:nn {48} {57} % 0--9
  }
\tl_const:Nn \c_regex_D_tl
  {
    \regex_item_more:n {57} % 9
    \regex_item_range:nn {0} {47} % 0
  }
\tl_const:Nn \c_regex_h_tl
  {
    \regex_item_equal:n {32} % space
    \regex_item_equal:n {9}  % tab
  }
\tl_const:Nn \c_regex_H_tl
  {
    \regex_item_neq:n {32} % space
    \regex_item_neq:n {9}  % tab
    \regex_break_true:w
  }
\tl_const:Nn \c_regex_s_tl
  {
    \regex_item_equal:n  {32}     % space
    \regex_item_neq:n    {11}     % vtab
    \regex_item_range:nn {9} {13} % tab, lf, vtab, ff, cr
  }
\tl_const:Nn \c_regex_S_tl
  {
    \regex_item_more:n   {32}      % > space
    \regex_item_range:nn {14} {31} % tab < ... < space
    \regex_item_range:nn {0}  {8}  % < tab
    \regex_item_equal:n  {11}      % vtab
  }
\tl_const:Nn \c_regex_v_tl
  {
    \regex_item_range:nn {10} {13} % lf, vtab, ff, cr
  }
\tl_const:Nn \c_regex_V_tl
  {
    \regex_item_more:n   {13}    % cr
    \regex_item_range:nn {0} {9} % < lf
  }
\tl_const:Nn \c_regex_w_tl
  {
    \regex_item_range:nn {97} {122} % a--z
    \regex_item_range:nn {65}  {90} % A--Z
    \regex_item_range:nn {48}  {57} % 0--9
    \regex_item_equal:n  {95}       % _
  }
\tl_const:Nn \c_regex_W_tl
  {
    \regex_item_range:nn  {0} {47} % <`0
    \regex_item_range:nn {58} {64} % (`9+1)--(`A-1)
    \regex_item_range:nn {91} {94} % (`Z+1)--(`_-1)
    \regex_item_equal:n  {96}      % `
    \regex_item_more:n  {122}      % z
  }
\tl_const:Nn \c_regex_N_tl
  {
    \regex_item_neq:n {10} % lf
    \regex_break_true:w
  }
%    \end{macrocode}
% \end{variable}
% \end{variable}
% \end{variable}
% \end{variable}
% \end{variable}
% \end{variable}
%
% \subsubsection{Variables used when matching}
%
% \begin{variable}{\l_regex_query_str}
%   The string that is being matched.
%    \begin{macrocode}
\tl_new:N \l_regex_query_str
%    \end{macrocode}
% \end{variable}
%
% \begin{variable}{\l_regex_current_char_int}
% \begin{variable}{\l_regex_current_step_int}
%   The character at the current position in the string,
%   and the current position.
%    \begin{macrocode}
\int_new:N \l_regex_current_char_int
\int_new:N \l_regex_current_step_int
%    \end{macrocode}
% \end{variable}
% \end{variable}
%
% \begin{variable}{\l_regex_start_step_int}
%   In the case of multiple matches, \cs{l_regex_start_step_int}
%   is equal to the position where the current match
%   attempt began.
%   \begin{macrocode}
\int_new:N \l_regex_start_step_int
%    \end{macrocode}
% \end{variable}
%
% \begin{variable}{\l_regex_max_thread_int}
%   Every thread is assigned a unique \texttt{id} when created,
%   and the current maximum thread \texttt{id} is stored in
%   \cs{l_regex_max_thread_int}.
%    \begin{macrocode}
\int_new:N \l_regex_max_thread_int
%    \end{macrocode}
% \end{variable}
%
% \begin{variable}{\l_regex_current_thread_int,\l_regex_current_state_int}
%   For every character in the string, each of the active threads is
%   considered in turn. Every thread is in a given state of the NFA.
%   The variables \cs{l_regex_current_thread_int} and
%   \cs{l_regex_current_state_int} hold the \texttt{id} of the thread
%   currently considered, and the state of the NFA in which it lies.
%    \begin{macrocode}
\int_new:N \l_regex_current_thread_int
\int_new:N \l_regex_current_state_int
%    \end{macrocode}
% \end{variable}
%
% \begin{variable}{\l_regex_max_index_int}
%   All the currently active threads are kept in order of precedence
%   in the \tn{skip} registers, which for our purpose serve as an array:
%   the $i$-th item of the array is \tn{skip}$i$. The largest index used
%   after treating the previous character is \cs{l_regex_max_index_int}.
%   At the start of every step, the whole array is unpacked, so that the
%   space can immediately be reused, and \cs{l_regex_max_index_int} reset
%   to zero, effectively clearing the array.
%    \begin{macrocode}
\int_new:N \l_regex_max_index_int
%    \end{macrocode}
% \end{variable}
%
% \begin{macro}[int]{\regex_if_track_submatches:T}
%   Not tracking submatches comes with a large performance gain,
%   by omitting some expensive operations on property lists.
%   The function \cs{regex_if_track_submatches:T} is set to
%   either \cs{use:n} or \cs{use_none:n} to enable or disable
%   tracking. We could use a boolean, but this method is faster,
%   and speed really matters.
%    \begin{macrocode}
\cs_new_eq:NN \regex_if_track_submatches:T \use_none:n
%    \end{macrocode}
% \end{macro}
%
% \begin{macro}[int]{\l_regex_every_match_tl}
%   Every time a match is found, this token list is used.
%   For single matching, the token list is set to removing
%   the remainder of the query string. For multiple matching,
%   the token list is set to repeat the matching.
%    \begin{macrocode}
\tl_new:N \l_regex_every_match_tl
%    \end{macrocode}
% \end{macro}
%
% \begin{macro}[int]{\regex_obsolete_all:F}
% \begin{macro}[aux]{\regex_obsolete_all_no:F,\regex_obsolete_all_yes:F}
%   Again a function that should be a boolean, but is not
%   for performance reasons. When a thread succeeds, all threads
%   with lower priority should be removed (in particular removed
%   from the submatch tracking).
%    \begin{macrocode}
\cs_new_protected:Npn \regex_obsolete_all_no:F #1 {#1}
\cs_new_protected:Npn \regex_obsolete_all_yes:F #1
  { \regex_action_fail: }
\cs_new_eq:NN \regex_obsolete_all:F \regex_obsolete_all_no:F
%    \end{macrocode}
% \end{macro}
% \end{macro}
%
% \begin{macro}{\regex_last_match_empty:F}
%   This function is most often \cs{use:n}, unless the current
%   regular expression can match an empty string at the current
%   position, and has already done so at the previous match attempt.
%   The goal is to break infinite loops.
%    \begin{macrocode}
\cs_new_protected:Npn \regex_last_match_empty_no:F #1 {#1}
\cs_new_protected:Npn \regex_last_match_empty_yes:F
  { \int_compare:nNnF \l_regex_start_step_int = \l_regex_current_step_int }
\cs_new_eq:NN \regex_last_match_empty:F \regex_last_match_empty_no:F
%    \end{macrocode}
% \end{macro}
%
% \begin{variable}{\l_regex_success_thread_int}
% \begin{variable}{\l_regex_success_step_int}
% \begin{variable}{\l_regex_success_submatches_prop}
% \begin{variable}{\l_regex_success_empty_bool}
%   The thread number corresponding to successful thread with highest
%   priority found so far, the step at which it was successfull, and
%   the corresponding submatch information.
%    \begin{macrocode}
\int_new:N  \l_regex_success_thread_int
\int_new:N  \l_regex_success_step_int
\prop_new:N \l_regex_success_submatches_prop
\bool_new:N \l_regex_success_empty_bool
%    \end{macrocode}
% \end{variable}
% \end{variable}
% \end{variable}
% \end{variable}
%
% \begin{variable}{\l_regex_submatches_prop}
%   Maps threads \texttt{id}s to their submatch information.
%    \begin{macrocode}
\prop_new:N \l_regex_submatches_prop
%    \end{macrocode}
% \end{variable}
%
% \begin{variable}{\l_regex_fresh_thread_bool}
%   This boolean marks when the current thread has started from
%   the beginning of the regular expression at this character,
%   in other words, it is true if the current thread has matched
%   an empty string so far (and we only care about this boolean
%   when a thread succeeds, hence matching an empty string overall).
%    \begin{macrocode}
\bool_new:N \l_regex_fresh_thread_bool
%    \end{macrocode}
% \end{variable}
%
% \subsubsection{Variables used for user functions}
%
% \begin{variable}{\g_regex_submatches_seq}
%   This holds temporarily a sequence of submatches,
%   global so that it exits the group.
%    \begin{macrocode}
\seq_new:N  \g_regex_submatches_seq
%    \end{macrocode}
% \end{variable}
%
% \begin{variable}{\g_regex_match_count_int}
%   The number of matches found so far is stored
%   in \cs{g_regex_match_count_int}. This is only used
%   in the \cs{regex_count:nnN} functions.
%    \begin{macrocode}
\int_new:N \g_regex_match_count_int
%    \end{macrocode}
% \end{variable}
%
% \begin{variable}{\g_regex_split_seq}
%   The \cs{regex_split:nnN} function stores its result
%   in that sequence variable before assigning it to the
%   variable provided by the user.
%    \begin{macrocode}
\seq_new:N \g_regex_split_seq
%    \end{macrocode}
% \end{variable}
%
% \begin{variable}{\l_regex_replacement_tl}
% \begin{variable}{\g_regex_replaced_str}
%   The replacement code stores a processed version
%   of the user's argument in \cs{l_regex_replacement_tl}.
%   The result of the replacement is stored in \cs{g_regex_replaced_str},
%   global to exit the group.
%    \begin{macrocode}
\tl_new:N \l_regex_replacement_tl
\tl_new:N \g_regex_replaced_str
%    \end{macrocode}
% \end{variable}
% \end{variable}
%
% \subsection{Helpers}
%
% \subsubsection{General}
%
% \begin{macro}[aux]{\prop_pop:NoNF,\prop_del:No}
%   Some variants.
%    \begin{macrocode}
\cs_generate_variant:Nn \prop_pop:NnNF { No }
\cs_generate_variant:Nn \prop_del:Nn { No } %^^A not used!
%    \end{macrocode}
% \end{macro}
%
% \begin{macro}[int]{\regex_toks_put_left:Nx}
% \begin{macro}[int]{\regex_toks_put_right:Nx}
%   Adding expanded material to toks registers, on the left or right.
%   The normal \LaTeX3 would be slower than this approach \enquote{by hand}.
%    \begin{macrocode}
\cs_new_protected:Npn \regex_toks_put_left:Nx #1#2
  {
    \cs_set_nopar:Npx \regex_tmp:w {#2}
    \tex_toks:D #1 \exp_after:wN \exp_after:wN \exp_after:wN
      { \exp_after:wN \regex_tmp:w \tex_the:D \tex_toks:D #1 }
  }
\cs_new_protected:Npn \regex_toks_put_right:Nx #1#2
  {
    \cs_set_nopar:Npx \regex_tmp:w {#2}
    \tex_toks:D #1 \exp_after:wN
      { \tex_the:D \tex_toks:D \exp_after:wN #1 \regex_tmp:w }
  }
%    \end{macrocode}
% \end{macro}
% \end{macro}
%
% \subsubsection{Interrupting recursions}
%
% \begin{macro}[int]{\regex_if_tail_stop:N}
% \begin{macro}[int]{\regex_if_tail_error:Nn}
%   Test for the end of the \meta{string}, and either stop
%   or cause an error if it is reached.
%    \begin{macrocode}
\cs_new_eq:NN \regex_if_tail_stop:N \quark_if_recursion_tail_stop:N
\cs_new_protected_nopar:Npn \regex_if_tail_error:Nn #1#2
  { \quark_if_recursion_tail_stop_do:Nn #1 { \regex_build_error:n {#2} } }
%    \end{macrocode}
% \end{macro}
% \end{macro}
%
% \begin{macro}[int]{\regex_build_error:n}
%   This macro is called if anything goes wrong when building
%   the NFA corresponding to a given regular expression
%   Negative codes are specific to the \LaTeX3 implementation.
%   Other error codes match with the PCRE codes (not all of
%   the PCRE errors can occur, since some constructions are
%   not supported).
%    \begin{macrocode}
\cs_new_protected_nopar:Npn \regex_build_error:n #1
  {
    \msg_error:nnxx { regex } { build-error } {\int_eval:n{#1}}
      {
        \prg_case_int:nnn {#1}
          {
            {-999} {File~not~found}
            {-998} {Unsupported~construct}
            {-997} {The~regular~expression~is~too~large~(32768~states).}
            {1}  {\iow_char:N\\~at~end~of~pattern }
%           {2}  {\iow_char:N\\c~at~end~of~pattern }
            {4}  {Numbers~out~of~order~in~\iow_char:N\{\iow_char\}~quantifier.}
            {6}  {Missing~terminating~\iow_char:N\]~for~character~class }
            {7}  {Invalid~escape~sequence~in~character~class}
            {8}  {Range~out~of~order~in~character~class}
            {22} {Unmatched~parentheses}
            {34} {Character~value~in~\iow_char:N\\x{...}~sequence~is~too~large}
%           {44} {Invalid~UTF-8~string}
%           {46} {Malformed~\iow_char:N\\P~or\iow_char:N\\p~sequence}
%           {47} {Unknown~property~after~\iow_char:N\\P~or\iow_char:N\\p}
%           {68} {\iow_char:N\\c~must~be~followed~by~an~ASCII~character}
          }
          { Internal~bug. }
      }
  }
\msg_new:nnn { regex } { build-error } { (error~#1): ~ #2 }
%    \end{macrocode}
% \end{macro}
%
% \subsubsection{Testing characters}
%
% \begin{macro}[int]{\regex_break_point:TF}
% \begin{macro}[int]{\regex_break_true:w,\regex_break_false:w}
%   When testing whether a character of the query string matches
%   a given character class in the regular expression, we often
%   have to test it against several ranges of characters, checking
%   if any one of those matches. This is done with a structure like
%   \begin{quote}
%     \meta{test1} \ldots{} \meta{test$\sb{n}$} \\
%     \cs{regex_break_point:TF} \Arg{true code} \Arg{false code}
%   \end{quote}
%   If any of the tests succeeds, it calls \cs{regex_break_true:w},
%   which cleans up and leaves \meta{true code} in the input stream.
%   Otherwise, \cs{regex_break_point:TF} leaves the \meta{false code}
%   in the input stream.
%    \begin{macrocode}
\cs_new_nopar:Npn \regex_break_true:w #1 \regex_break_point:TF #2 #3 {#2}
\cs_new_nopar:Npn \regex_break_false:w #1 \regex_break_point:TF #2 #3 {#3}
\cs_new_eq:NN \regex_break_point:TF \use_ii:nn
%    \end{macrocode}
% \end{macro}
% \end{macro}
%
% \begin{macro}[int]{\regex_item_equal:n,\regex_item_range:nn}
% \begin{macro}[int]{\regex_item_less:n,\regex_item_more:n}
% \begin{macro}[int]{\regex_item_neq:n}
%   Simple comparisons triggering \cs{regex_break_true:w} when true.
%    \begin{macrocode}
\cs_new_nopar:Npn \regex_item_equal:n #1
  {
    \if_num:w #1 = \l_regex_current_char_int
      \exp_after:wN \regex_break_true:w
    \fi:
  }
\cs_new_nopar:Npn \regex_item_range:nn #1 #2
  {
    \reverse_if:N \if_num:w #1 > \l_regex_current_char_int
      \reverse_if:N \if_num:w #2 < \l_regex_current_char_int
        \exp_after:wN \exp_after:wN \exp_after:wN \regex_break_true:w
      \fi:
    \fi:
  }
\cs_new_nopar:Npn \regex_item_less:n #1
  {
    \if_num:w #1 > \l_regex_current_char_int
      \exp_after:wN \regex_break_true:w
    \fi:
  }
\cs_new_nopar:Npn \regex_item_more:n #1
  {
    \if_num:w #1 < \l_regex_current_char_int
      \exp_after:wN \regex_break_true:w
    \fi:
  }
\cs_new_nopar:Npn \regex_item_neq:n #1
  {
    \if_num:w #1 = \l_regex_current_char_int
      \exp_after:wN \regex_break_false:w
    \fi:
  }
%    \end{macrocode}
% \end{macro}
% \end{macro}
% \end{macro}
%
% \subsubsection{Grabbing digits}
%
% \begin{macro}[int]{\regex_get_digits:nw}
% \begin{macro}[aux]{\regex_get_digits_loop:N,\regex_get_digits_end:w}
%   Grabs digits (of category code other), skipping any intervening
%   space, until encountering a non-digit, and places the result
%   in a brace group after |#1|. This is used when parsing the \texttt{\{}
%   quantifier.
%    \begin{macrocode}
\cs_new_protected:Npn \regex_get_digits:nw #1
  {
    \tex_afterassignment:D \regex_tmp:w
    \cs_set_nopar:Npx \regex_tmp:w
      {
        \exp_not:n {#1}
        { \if_false: } } \fi:
        \regex_get_digits_aux:N
  }
\cs_new_nopar:Npn \regex_get_digits_aux:N #1
  {
    \if_num:w 9 < 1 \exp_not:N #1 \exp_stop_f:
    \else:
      \if_charcode:w \c_space_token \exp_not:N #1
      \else:
        \regex_get_digits_end:w
      \fi:
    \fi:
    #1 \regex_get_digits_aux:N
  }
\cs_new_nopar:Npn \regex_get_digits_end:w
    \fi: \fi: #1 \regex_get_digits_aux:N
  {
    \fi: \fi:
    \if_false: { { \fi: } }
    #1
  }
%    \end{macrocode}
% \end{macro}
% \end{macro}
%
% \subsubsection{More char testing}
%
% \begin{macro}[EXP,aux]{\regex_aux_char_if_alphanumeric:NTF}
% \begin{macro}[EXP,aux]{\regex_aux_char_if_special:NTF}
%   These two tests are used in the first pass when parsing a
%   regular expression. That pass is responsible for finding
%   escaped and non-escaped characters, and recognizing which
%   ones have special meanings and which should be interpreted
%   as \enquote{raw} characters. Namely,
%   \begin{itemize}
%   \item alphanumerics are \enquote{raw} if they are not escaped,
%     and may have a special meaning when escaped;
%   \item non-alphanumeric printable ascii characters are \enquote{raw}
%     if they are escaped, and may have a special meaning when not escaped;
%   \item characters other than printable ascii are always \enquote{raw}.
%   \end{itemize}
%   The code is ugly, and highly based on magic numbers and the ascii
%   codes of characters. This is mostly unavoidable for performance
%   reasons: testing for instance with \cs{str_if_contains_char:nN}
%   would be much slower. Maybe the tests can be optimized a little
%   bit more.
%   Here, \enquote{alphanumeric} means \texttt{0}--\texttt{9},
%   \texttt{A}--\texttt{Z}, \texttt{a}--\texttt{z};
%   \enquote{special} character means non-alphanumeric
%   but printable ascii, from space (hex \texttt{20}) to
%   \texttt{del} (hex \texttt{7E}).
%    \begin{macrocode}
\prg_new_conditional:Npnn \regex_aux_char_if_special:N #1 { TF }
  {
    \if_num:w `#1 < 97 \exp_stop_f:
      \if_num:w `#1 < 58 \exp_stop_f:
        \if_num:w \int_eval:w (`#1 - \c_eight)/\c_sixteen = \c_two
          \prg_return_true:
        \else:
          \prg_return_false:
        \fi:
      \else:
        \if_num:w \int_eval:w `#1 / 26 = \c_three
          \prg_return_false:
        \else:
          \prg_return_true:
        \fi:
      \fi:
    \else:
      \if_num:w \int_eval:w `#1 / \c_five = 25 \exp_stop_f:
        \prg_return_true:
      \else:
        \prg_return_false:
      \fi:
    \fi:
  }
\prg_new_conditional:Npnn \regex_aux_char_if_alphanumeric:N #1 { TF }
  {
    \if_num:w `#1 < 91 \exp_stop_f:
      \if_num:w `#1 < 65 \exp_stop_f:
        \if_num:w \c_nine < 1 #1 \exp_stop_f:
          \prg_return_true:
        \else:
          \prg_return_false:
        \fi:
      \else:
        \prg_return_true:
      \fi:
    \else:
      \if_num:w \int_eval:w (`#1-\c_six)/26 = \c_four
        \prg_return_true:
      \else:
        \prg_return_false:
      \fi:
    \fi:
  }
%    \end{macrocode}
% \end{macro}
% \end{macro}
%
% \subsection{Building}
%
% \subsubsection{Helpers for building an NFA}
%
% \begin{macro}[int]{\regex_build_new_state:}
%   Here, we add a new state to the NFA:
%   empty the corresponding toks now, then set
%   \cs{l_regex_state_left/right_int} to their
%   new values.
%    \begin{macrocode}
\cs_new_protected_nopar:Npn \regex_build_new_state:
  {
    \int_compare:nNnTF \l_regex_state_max_int > {32766}
      { \regex_build_error:n {-997} }
      {
        \int_incr:N \l_regex_state_max_int
        \tex_toks:D \l_regex_state_max_int { }
      }
    \int_set_eq:NN \l_regex_state_left_int \l_regex_state_right_int
    \int_set_eq:NN \l_regex_state_right_int \l_regex_state_max_int
  }
%    \end{macrocode}
% \end{macro}
%
% \begin{macro}[aux]{\regex_build_transition_aux:NN}
% \begin{macro}[aux]{\regex_build_transitions_aux:NNNN}
%   These functions create a new state, and put one or two transitions
%   starting from the old current state.
%    \begin{macrocode}
\cs_new_protected_nopar:Npn \regex_build_transition_aux:NN #1#2
  {
    \regex_build_new_state:
    \regex_toks_put_right:Nx \l_regex_state_left_int
      { #1 { \int_use:N #2 } }
  }
\cs_new_protected_nopar:Npn \regex_build_transitions_aux:NNNN #1#2#3#4
  {
    \regex_build_new_state:
    \regex_toks_put_right:Nx \l_regex_state_left_int
      {
        #1 { \int_use:N #2 }
        #3 { \int_use:N #4 }
      }
  }
%    \end{macrocode}
% \end{macro}
% \end{macro}
%
% \subsubsection{From regex to NFA: framework}
%
% In order for the construction \texttt{ab\string|cd} to work, we enclose
% the whole pattern within parentheses. These have the added benefit
% to form a capturing group: hence we get the data of the whole match
% for free.
%
% \begin{macro}[int]{\regex_build:n}
%   First, reset a few variables. Then use the generic framework defined
%   in \pkg{l3str} to parse the regular expression once, recognizing
%   which characters are raw characters, and which have special meanings.
%   The result is stored in \cs{g_str_tmpa_tl}, and can be run directly.
%   The trailing \cs{prg_do_nothing:} ensure that the look-ahead done by
%   some of the operations is harmless.
%   Finally, \cs{regex_build_end:} adds the finishing code
%   (checking that parentheses are properly nested, for instance).
%    \begin{macrocode}
\cs_new_protected:Npn \regex_build:n #1
  {
    \regex_build_setup:
    \str_aux_escape:NNNn
      \regex_build_i_unescaped:N
      \regex_build_i_escaped:N
      \regex_build_i_raw:N
      { (#1) }
    \g_str_tmpa_tl \prg_do_nothing: \prg_do_nothing:
    \regex_build_end:
%<trace>    \regex_trace_nfa:
  }
%    \end{macrocode}
% \end{macro}
%
% \begin{macro}[aux]{\regex_build_i_unescaped:N}
% \begin{macro}[aux]{\regex_build_i_escaped:N}
% \begin{macro}[aux]{\regex_build_i_raw:N}
%   The \pkg{l3str} function \cs{str_aux_escape:NNNn} goes through
%   the regular expression and finds the |\a|, |\e|, |\f|, |\n|, |\r|,
%   |\t|, and |\x| escape sequences, then distinguishes three cases:
%   non-escaped characters, escaped characters, and \enquote{raw}
%   characters coming from one of the escape sequences.
%   In the particular case of regular expressions, escaped alphanumerics
%   and non-escaped non-alphanumeric printable ascii characters may have
%   special meanings, while everything else should be treated as a raw
%   character.
%    \begin{macrocode}
\cs_new_nopar:Npn \regex_build_i_unescaped:N #1
  {
    \regex_aux_char_if_special:NTF #1
      { \exp_not:N \regex_build_control:N #1 }
      { \exp_not:N \regex_build_raw:N #1 }
  }
\cs_new_nopar:Npn \regex_build_i_escaped:N #1
  {
    \regex_aux_char_if_alphanumeric:NTF #1
      { \exp_not:N \regex_build_control:N #1 }
      { \exp_not:N \regex_build_raw:N #1 }
  }
\cs_new_nopar:Npn \regex_build_i_raw:N #1
  { \exp_not:N \regex_build_raw:N #1 }
%    \end{macrocode}
% \end{macro}
% \end{macro}
% \end{macro}
%
% \begin{macro}[aux]{\regex_build_default_control:N}
%   If the control character has a particular meaning in regular expressions,
%   the corresponding function is used. Otherwise, it is interpreted as a raw
%   character. The \cs{regex_build_default_raw:N} function is defined later.
%    \begin{macrocode}
\cs_new_protected_nopar:Npn \regex_build_default_control:N #1
  {
    \cs_if_exist_use:cF { regex_build_#1: }
      { \regex_build_default_raw:N #1 }
  }
%    \end{macrocode}
% \end{macro}
%
% \begin{macro}[int]{\regex_build_setup:}
%   Hopefully, we didn't forget to initialize anything here.
%   The search is not anchored: to acheive that, we insert state(s)
%   responsible for repeating the match attempt on every character
%   of the string: the first state has a free transition to the second
%   state, where the regular expression really begins, and a costly
%   transition to itself, to try again at the next character.
%    \begin{macrocode}
\cs_new_protected_nopar:Npn \regex_build_setup:
  {
    \cs_set_eq:NN \regex_build_control:N \regex_build_default_control:N
    \cs_set_eq:NN \regex_build_raw:N \regex_build_default_raw:N
    \int_set_eq:NN \l_regex_capturing_group_int \c_minus_one
    \int_zero:N \l_regex_state_left_int
    \int_zero:N \l_regex_state_right_int
    \int_zero:N \l_regex_state_max_int
    \regex_build_new_state:
    \regex_build_new_state:
    \regex_toks_put_right:Nx \l_regex_state_left_int
      {
        \regex_action_start_wildcard:nn
          { \int_use:N \l_regex_state_left_int }
          { \int_use:N \l_regex_state_right_int }
      }
  }
%    \end{macrocode}
% \end{macro}
%
% \begin{macro}[int]{\regex_build_end:}
%   If parentheses are not nested properly, an error is raised,
%   and the correct number of parentheses is closed.
%   After that, we insert an instruction for the match to succeed.
%    \begin{macrocode}
\cs_new_protected_nopar:Npn \regex_build_end:
  {
    \seq_if_empty:NF \l_regex_capturing_group_seq
      {
        \regex_build_error:n {22}
        \prg_replicate:nn
          { \seq_length:N \l_regex_capturing_group_seq } % (
          { \use:c { regex_build_): } }
        \prg_do_nothing: \prg_do_nothing:
      }
    \regex_toks_put_right:Nx \l_regex_state_right_int
      { \regex_action_success: }
  }
%    \end{macrocode}
% \end{macro}
%
% \subsubsection{Anchoring and simple assertions}
%
% \begin{macro}[int]{\regex_build_A:}
% \begin{macro}[int]+\regex_build_^:+
% \begin{macro}[int]{\regex_build_G:}
% \begin{macro}[aux]{\regex_build_anchor_start:N}
%   Anchoring at the start corresponds to checking that the current
%   character is the first in the string. Anchoring to the beginning
%   of the match attempt uses \cs{l_regex_start_step_int} instead of
%   \cs{c_zero}.
%    \begin{macrocode}
\cs_new_protected_nopar:cpn { regex_build_^: }
  { \regex_build_anchor_start:N \c_zero }
\cs_new_protected_nopar:Npn \regex_build_A:
  { \regex_build_anchor_start:N \c_zero }
\cs_new_protected_nopar:Npn \regex_build_G:
  { \regex_build_anchor_start:N \l_regex_start_step_int }
\cs_new_protected_nopar:Npn \regex_build_anchor_start:N #1
  {
    \regex_build_new_state:
    \regex_toks_put_right:Nx \l_regex_state_left_int
      {
        \exp_not:N \if_num:w #1 = \l_regex_current_step_int
          \regex_action_free_move:n
            { \int_use:N \l_regex_state_right_int }
        \exp_not:N \else:
          \regex_action_fail:
        \exp_not:N \fi:
      }
  }
%    \end{macrocode}
% \end{macro}
% \end{macro}
% \end{macro}
% \end{macro}
%
% \begin{macro}[aux]{\regex_build_Z:}
% \begin{macro}[aux]{\regex_build_z:}
% \begin{macro}[aux]+\regex_build_$:+
%   This matches the end of the string, marked by a character code of $-1$.
%    \begin{macrocode}
\cs_new_protected_nopar:cpn { regex_build_$: } % $
  {
    \regex_build_new_state:
    \regex_toks_put_right:Nx \l_regex_state_left_int
      {
        \exp_not:N \if_num:w \c_minus_one = \l_regex_current_char_int
          \regex_action_free_move:n
            { \int_use:N \l_regex_state_right_int }
        \exp_not:N \else:
          \regex_action_fail:
        \exp_not:N \fi:
      }
  }
\cs_new_eq:Nc \regex_build_Z: { regex_build_$: } %$
\cs_new_eq:Nc \regex_build_z: { regex_build_$: } %$
%    \end{macrocode}
% \end{macro}
% \end{macro}
% \end{macro}
%
% \begin{macro}[int]{\regex_build_b:}
% \begin{macro}[int]{\regex_build_B:}
% \begin{macro}[aux]{\regex_if_word_boundary:TF}
%   Contrarily to |^| and |$|, which could be implemented without
%   really knowing what preceeds in the string, this requires
%   more information. We request it by setting
%   \cs{l_regex_look_behind_bool}. Then the matching code will
%   keep store the characters that were already read, backwards,
%   in \cs{l_regex_look_behind_str}, and we can analyse the first
%   character of that string.
%    \begin{macrocode}
\cs_new_protected_nopar:Npn \regex_build_b:
  {
    \bool_set_true:N \l_regex_look_behind_bool
    \regex_build_new_state:
    \regex_toks_put_right:Nx \l_regex_state_left_int
      {
        \exp_not:N \regex_if_word_boundary:TF
          {
            \regex_action_free_move:n
              { \int_use:N \l_regex_state_right_int }
          }
          { \regex_action_fail: }
      }
  }
\cs_new_protected_nopar:Npn \regex_build_B:
  {
    \bool_set_true:N \l_regex_look_behind_bool
    \regex_build_new_state:
    \regex_toks_put_right:Nx \l_regex_state_left_int
      {
        \exp_not:N \regex_if_word_boundary:TF
          { \regex_action_fail: }
          {
            \regex_action_free_move:n
              { \int_use:N \l_regex_state_right_int }
          }
      }
  }
\cs_new_protected_nopar:Npn \regex_if_word_boundary:TF
  {
    \tl_if_empty:NTF \l_regex_look_behind_str
      { \c_regex_w_tl }
      {
        \group_begin:
          \cs_set_nopar:Npx \regex_tmp:w
            {
              \int_set:Nn \l_regex_current_char_int
                { ` \str_head:N \l_regex_look_behind_str }
            }
          \regex_tmp:w
          \c_regex_w_tl
          \regex_break_point:TF
            { \group_end: \c_regex_W_tl }
            { \group_end: \c_regex_w_tl }
      }
    \regex_break_point:TF
  }
%    \end{macrocode}
% \end{macro}
% \end{macro}
% \end{macro}
%
% \subsubsection{Normal character, and simple character classes}
%
% \begin{macro}[aux]{\regex_build_default_raw:N}
%   A normal alphanumeric or an escaped non-alphanumeric
%   (actually, any unknown combination) will match itself
%   and the thread will fail otherwise. We prepare
%   \cs{regex_build_tmp_class:n} with the relevant test and
%   commands. The program steps to be inserted in those
%   commands will come as |##1| and |##2|: we don't know
%   yet what those will be before checking for quantifiers.
%    \begin{macrocode}
\cs_new_protected_nopar:Npn \regex_build_default_raw:N #1
  {
    \cs_set:Npx \regex_build_tmp_class:n ##1
      {
        \exp_not:n { \exp_not:N \if_num:w }
            \int_value:w `#1 = \l_regex_current_char_int
          \regex_action_cost:n { ##1 }
        \exp_not:n { \exp_not:N \else: }
          \regex_action_fail:
        \exp_not:n { \exp_not:N \fi: }
      }
    \regex_build_one_quantifier:
  }
%    \end{macrocode}
% \end{macro}
%
% \begin{macro}[aux]{\regex_build_.:}
%   Similar to \cs{regex_build_default_raw:N} but accepts any character,
%   and refuses $-1$, which marks the end of the string.
%    \begin{macrocode}
\cs_new_protected_nopar:cpn { regex_build_.: }
  {
    \cs_set:Npn \regex_build_tmp_class:n ##1
      {
        \exp_not:N \if_num:w \c_minus_one = \l_regex_current_char_int
          \regex_action_fail:
        \exp_not:N \else:
          \regex_action_cost:n {##1}
        \exp_not:N \fi:
      }
    \regex_build_one_quantifier:
  }
%    \end{macrocode}
% \end{macro}
%
% \begin{macro}[aux]{\regex_build_d:,\regex_build_D}
% \begin{macro}[aux]{\regex_build_h:,\regex_build_H}
% \begin{macro}[aux]{\regex_build_s:,\regex_build_S}
% \begin{macro}[aux]{\regex_build_v:,\regex_build_V}
% \begin{macro}[aux]{\regex_build_w:,\regex_build_W}
% \begin{macro}[aux]{\regex_build_N:}
% \begin{macro}[aux]{\regex_build_char_type:N}
%   The constants \cs{c_regex_d_tl}, \emph{etc.} hold
%   a list of tests which match the corresponding character
%   class, and jump to the \cs{regex_break_point:TF} marker.
%   As for a normal character, we check for quantifiers.
%    \begin{macrocode}
\cs_new_protected_nopar:Npn \regex_build_char_type:N #1
  {
    \cs_set:Npn \regex_build_tmp_class:n ##1
      {
        \exp_not:N #1
        \exp_not:N \regex_break_point:TF
          { \regex_action_cost:n {##1} }
          { \regex_action_fail: }
      }
    \regex_build_one_quantifier:
  }
\tl_map_inline:nn { dDhHsSvVwWN }
  {
    \cs_new_protected_nopar:cpx { regex_build_#1: }
      {
        \exp_not:N \regex_build_char_type:N
        \exp_not:c { c_regex_#1_tl }
      }
  }
%    \end{macrocode}
% \end{macro}
% \end{macro}
% \end{macro}
% \end{macro}
% \end{macro}
% \end{macro}
% \end{macro}
%
% \subsubsection{Character classes}
%
% \begin{macro}[aux]{\regex_build_[:}
%   This starts a class. The code for the class is collected
%   in \cs{l_regex_class_tl}. The first character is special.
%    \begin{macrocode}
\cs_new_protected_nopar:cpn { regex_build_[: }
  {
    \tl_clear:N \l_regex_class_tl
    \cs_set_eq:NN \regex_build_control:N \regex_class_control:N
    \cs_set_eq:NN \regex_build_raw:N \regex_class_raw:N
    \regex_class_first:NN
  }
%    \end{macrocode}
% \end{macro}
%
% \begin{macro}[aux]{\regex_class_control:N}
%   This function is similar to \cs{regex_build_control:N}. If the control
%   character has a meaning in character classes, call the corresponding
%   function, otherwise, treat it as a raw character, with the
%   \cs{regex_class_raw:N} function, defined later.
%    \begin{macrocode}
\cs_new_protected_nopar:Npn \regex_class_control:N #1
  {
    \cs_if_exist_use:cF { regex_class_#1: }
      { \regex_class_raw:N #1 }
  }
%    \end{macrocode}
% \end{macro}
%
% \begin{macro}[aux]{\regex_class_]:}
%   If \texttt{]} appears as the first item of a class, then
%   it doesn't end the class. Otherwise, it's the end,
%   act just as for a single character, but with a more
%   complicated test. And restore \cs{regex_build_control:N}
%   and \cs{regex_build_raw:N}.
%    \begin{macrocode}
\cs_new_protected_nopar:cpn { regex_class_]: }
  {
    \tl_if_empty:NTF \l_regex_class_tl %[
      { \regex_class_raw:N ] }
      {
        \cs_set_eq:NN \regex_build_control:N \regex_build_default_control:N
        \cs_set_eq:NN \regex_build_raw:N  \regex_build_default_raw:N
        \cs_set:Npn \regex_build_tmp_class:n ##1
          {
            \exp_not:o \l_regex_class_tl
            \bool_if:NTF \l_regex_class_bool
              {
                \exp_not:N \regex_break_point:TF
                  { \regex_action_cost:n {##1} }
                  { \regex_action_fail: }
              }
              {
                \exp_not:N \regex_break_point:TF
                  { \regex_action_fail: }
                  { \regex_action_cost:n {##1} }
              }
          }
        \regex_build_one_quantifier:
      }
  }
%    \end{macrocode}
% \end{macro}
%
% \begin{macro}[aux]{\regex_class_first:NN}
%   If the first non-space character is |^|, then the class is inverted.
%   We keep track of this in \cs{l_regex_class_bool}.
%    \begin{macrocode}
\cs_new_protected_nopar:Npn \regex_class_first:NN #1#2
  {
    \str_if_eq:nnTF {#1#2} { \regex_build_control:N ^ }
      { \bool_set_false:N \l_regex_class_bool }
      {
        \bool_set_true:N \l_regex_class_bool
        #1 #2
      }
  }
%    \end{macrocode}
% \end{macro}
%
% \begin{macro}[aux]{\regex_class_raw:N}
% \begin{macro}[aux]{\regex_class_single:N}
%   Most characters are treated here. We look ahead for an unescaped dash.
%   If there is none, then the character matches itself.
%    \begin{macrocode}
\cs_new_protected_nopar:Npn \regex_class_raw:N #1#2#3
  {
    \str_if_eq:nnTF {#2#3} { \regex_build_control:N - }
      { \regex_class_range:Nw #1 }
      {
        \regex_class_single:N #1
        #2 #3
      }
  }
\cs_new_protected_nopar:Npn \regex_class_single:N #1
  {
    \tl_put_right:Nx \l_regex_class_tl
      { \exp_not:N \regex_item_equal:n { \int_value:w `#1 } }
  }
%    \end{macrocode}
% \end{macro}
% \end{macro}
%
% \begin{macro}[aux]{\regex_class_range:Nw}
% \begin{macro}[aux]{\regex_class_range_put:NN}
%   If the character is followed by a dash, we look for
%   the end-point of the range. For \enquote{raw} characters,
%   that's simply |#3|. Most \enquote{control} characters also
%   have no meaning, and can serve as an end-point, but those
%   with a meaning interrupt the range.
%   In the case of a true range, check whether the end-points
%   are in the right order, and optimize in the case of equal
%   end-points.
%    \begin{macrocode}
\cs_new_protected_nopar:Npn \regex_class_range:Nw #1#2#3
  {
    \token_if_eq_meaning:NNTF #2 \regex_build_control:N
      {
        \cs_if_exist:cTF { regex_class_#3: }
          {
            \regex_class_single:N #1
            \regex_class_single:N -
            #2#3
          }
          { \regex_class_range_put:NN #1#3 }
      }
      { \regex_class_range_put:NN #1#3 }
  }
\cs_new_protected_nopar:Npn \regex_class_range_put:NN #1#2
  {
    \if_num:w `#1 > `#2 \exp_stop_f:
      \regex_build_error:n {8}
    \else:
      \tl_put_right:Nx \l_regex_class_tl
        {
          \if_num:w `#1 = `#2 \exp_stop_f:
            \exp_not:N \regex_item_equal:n
          \else:
            \exp_not:N \regex_item_range:nn { \int_value:w `#1 }
          \fi:
          { \int_value:w `#2 }
        }
    \fi:
  }
%    \end{macrocode}
% \end{macro}
% \end{macro}
%
% \begin{macro}[aux]{\regex_class_d:,\regex_class_D:}
% \begin{macro}[aux]{\regex_class_h:,\regex_class_H:}
% \begin{macro}[aux]{\regex_class_s:,\regex_class_S:}
% \begin{macro}[aux]{\regex_class_v:,\regex_class_V:}
% \begin{macro}[aux]{\regex_class_w:,\regex_class_W:}
%   Similar to \cs{regex_class_single:N}, adding the appropriate
%   ranges of characters to the class. The token lists are not
%   expanded because it is more memory efficient, with a tiny
%   overhead on execution.
%    \begin{macrocode}
\tl_map_inline:nn { dDhHsSvVwWN }
  {
    \cs_new_protected_nopar:cpx { regex_class_#1: }
      {
        \tl_put_right:Nn \exp_not:N \l_regex_class_tl
          { \exp_not:c { c_regex_#1_tl } }
      }
  }
%    \end{macrocode}
% \end{macro}
% \end{macro}
% \end{macro}
% \end{macro}
% \end{macro}
%
% \subsubsection{Quantifiers}
%
% \begin{macro}[int]{\regex_build_quantifier:w}
%   This looks ahead and finds any quantifier (control character
%   equal to either of |?+*{|). ^^A}
%   When all characters for the quantifier are found, the corresponding
%   function is called.
%    \begin{macrocode}
\cs_new_protected_nopar:Npn \regex_build_quantifier:w #1#2
  {
    \token_if_eq_meaning:NNTF #1 \regex_build_control:N
      {
        \cs_if_exist_use:cF { regex_build_quantifier_#2:w }
          {
            \regex_build_quantifier_end:n { }
            #1 #2
          }
      }
      {
        \regex_build_quantifier_end:n { }
        #1 #2
      }
  }
%    \end{macrocode}
% \end{macro}
%
% \begin{macro}[aux]{\regex_build_quantifier_?:w}
% \begin{macro}[aux]{\regex_build_quantifier_*:w}
% \begin{macro}[aux]{\regex_build_quantifier_+:w}
%   For each \enquote{basic} quantifier, |?|, |*|, |+|, feed the correct
%   arguments to \cs{regex_build_quantifier_aux:nnNN}.
%    \begin{macrocode}
\cs_new_protected_nopar:cpn { regex_build_quantifier_?:w }
  { \regex_build_quantifier_aux:nnNN { } { ? } }
\cs_new_protected_nopar:cpn { regex_build_quantifier_*:w }
  { \regex_build_quantifier_aux:nnNN { } { * } }
\cs_new_protected_nopar:cpn { regex_build_quantifier_+:w }
  { \regex_build_quantifier_aux:nnNN { } { + } }
%    \end{macrocode}
% \end{macro}
% \end{macro}
% \end{macro}
%
% \begin{macro}[aux]{\regex_build_quantifier_aux:nnNN}
%   Once the \enquote{main} quantifier (\texttt{?}, \texttt{*},
%   \texttt{+} or a braced construction) is found, we check
%   whether it is lazy (followed by a question mark),
%   and calls the appropriate function. Here |#1| holds some extra
%   arguments that the final function needs in the case of braced
%   constructions, and is empty otherwise.
%    \begin{macrocode}
\cs_new_protected_nopar:Npn \regex_build_quantifier_aux:nnNN #1#2#3#4
  {
    \str_if_eq:nnTF { #3 #4 } { \regex_build_control:N ? }
      { \regex_build_quantifier_end:n { #2 #4 } #1 }
      {
        \regex_build_quantifier_end:n { #2 } #1
        #3 #4
      }
  }
%    \end{macrocode}
% \end{macro}
%
% \begin{macro}[aux]+\regex_build_quantifier_{:w+ ^^A}
% \begin{macro}[aux]{\regex_build_quantifier_lbrace:n}
% \begin{macro}[aux]{\regex_build_quantifier_lbrace:nw}
% \begin{macro}[aux]{\regex_build_quantifier_lbrace:nnw}
%   Three possible syntaxes: \texttt{\{\meta{int}\}},
%   \texttt{\{\meta{int},\}}, or \texttt{\{\meta{int},\meta{int}\}}.
%    \begin{macrocode}
\cs_new_protected_nopar:cpn { regex_build_quantifier_ \c_lbrace_str :w }
  { \regex_get_digits:nw { \regex_build_quantifier_lbrace:n } }
\cs_new_protected_nopar:Npn \regex_build_quantifier_lbrace:n #1
  {
    \tl_if_empty:nTF {#1}
      {
        \regex_build_quantifier_end:n { }
        \exp_after:wN \regex_build_raw:N \c_lbrace_str
      }
      { \regex_build_quantifier_lbrace:nw {#1} }
  }
\cs_new_protected_nopar:Npx \regex_build_quantifier_lbrace:nw #1#2#3
  {
    \exp_not:N \prg_case_str:nnn { #2 #3 }
      {
        { \exp_not:N \regex_build_control:N , }
          {
            \exp_not:N \regex_get_digits:nw
              { \exp_not:N \regex_build_quantifier_lbrace:nnw {#1} }
          }
        { \exp_not:N \regex_build_control:N \c_rbrace_str }
          { \exp_not:N \regex_build_quantifier_end:n {n} {#1} }
      }
      {
        \exp_not:N \regex_build_quantifier_end:n { }
        \exp_not:N \regex_build_raw:N \c_lbrace_str #1#2
      }
  }
\cs_new_protected_nopar:Npn \regex_build_quantifier_lbrace:nnw #1#2#3
  {
    \str_if_eq:xxTF { \exp_not:N #3 } { \c_rbrace_str }
      {
        \tl_if_empty:nTF {#2}
          { \regex_build_quantifier_aux:nnN { {#1} {\c_max_int} } {nn} }
          {
            \int_compare:nNnT {#1} > {#2}
              {
                \regex_build_error:n {4}
                %^^A insert "fail"
              }
            \regex_build_quantifier_aux:nnN { {#1} {#2} } {nn}
          }
      }
      {
        \regex_build_quantifier_end:n { }
        \use:x
          {
            \exp_not:n { \exp_args:No \tl_map_function:nN }
              { \c_lbrace_str #1 #2 , }
            \regex_build_raw:N
          }
      }
  }
%    \end{macrocode}
% \end{macro}
% \end{macro}
% \end{macro}
% \end{macro}
%
% \begin{macro}[aux]{\regex_build_quantifier_end:n}
%   When all quantifiers are found, we will call the relevant
%   \cs{regex_build_one/group_\meta{quantifiers}:} function.
%    \begin{macrocode}
\cs_new_protected_nopar:Npn \regex_build_quantifier_end:n #1
  { \use:c { regex_build_ \l_regex_one_or_group_tl _ #1 : } }
%    \end{macrocode}
% \end{macro}
%
% \subsubsection{Quantifiers for one character or character class}
%
% \begin{macro}[aux]{\regex_build_one_quantifier:}
%   Used for one single character, or a character class.
%   Contrarily to \cs{regex_build_group_quantifier:},
%   we don't need to keep track of submatches, and no thread
%   can be created within one repetition, so things are relatively easy.
%    \begin{macrocode}
\cs_new_protected_nopar:Npn \regex_build_one_quantifier:
  {
    \tl_set:Nx \l_regex_one_or_group_tl { one }
    \regex_build_quantifier:w
  }
%    \end{macrocode}
% \end{macro}
%
% \begin{macro}[aux]{\regex_build_one_:}
%   If no quantifier is found, then the character or character class
%   should just be built into a transition from the current
%   \enquote{right} state to a new state.
%    \begin{macrocode}
\cs_new_protected_nopar:Npn \regex_build_one_:
  {
    \regex_build_transition_aux:NN
      \regex_build_tmp_class:n \l_regex_state_right_int
  }
%    \end{macrocode}
% \end{macro}
%
% \begin{macro}[aux]{\regex_build_one_?:}
% \begin{macro}[aux]{\regex_build_one_??:}
%   The two transitions are a costly transition controlled by
%   the character class, and a free transition, both going to
%   a common new state. The only difference between the greedy
%   and lazy operators is the order of transitions.
%    \begin{macrocode}
\cs_new_protected_nopar:cpn { regex_build_one_?: }
  {
    \regex_build_transitions_aux:NNNN
      \regex_build_tmp_class:n  \l_regex_state_right_int
      \regex_action_free_copy:n \l_regex_state_right_int
  }
\cs_new_protected_nopar:cpn { regex_build_one_??: }
  {
    \regex_build_transitions_aux:NNNN
      \regex_action_free_copy:n \l_regex_state_right_int
      \regex_build_tmp_class:n  \l_regex_state_right_int
  }
%    \end{macrocode}
% \end{macro}
% \end{macro}
%
% \begin{macro}[aux]{\regex_build_one_*:}
% \begin{macro}[aux]{\regex_build_one_*?:}
%   Build a costly transition going from the current state to itself,
%   and a free transition moving to a new state.
%    \begin{macrocode}
\cs_new_protected_nopar:cpn { regex_build_one_*: }
  {
    \regex_build_transitions_aux:NNNN
      \regex_build_tmp_class:n  \l_regex_state_left_int
      \regex_action_free_copy:n \l_regex_state_right_int
  }
\cs_new_protected_nopar:cpn { regex_build_one_*?: }
  {
    \regex_build_transitions_aux:NNNN
      \regex_action_free_copy:n \l_regex_state_right_int
      \regex_build_tmp_class:n  \l_regex_state_left_int
  }
%    \end{macrocode}
% \end{macro}
% \end{macro}
%
% \begin{macro}[aux]{\regex_build_one_+:}
% \begin{macro}[aux]{\regex_build_one_+?:}
%   Build a transition from the current state to a new state,
%   controlled by the character class, then build two transitions
%   from this new state to the original state (for repetition)
%   and to another new state (to move on to the rest of the pattern).
%    \begin{macrocode}
\cs_new_protected_nopar:cpn { regex_build_one_+: }
  {
    \regex_build_one_:
    \int_set_eq:NN \l_regex_tmpa_int \l_regex_state_left_int
    \regex_build_transitions_aux:NNNN
      \regex_action_free_copy:n \l_regex_tmpa_int
      \regex_action_free_move:n \l_regex_state_right_int
  }
\cs_new_protected_nopar:cpn { regex_build_one_+?: }
  {
    \regex_build_one_:
    \int_set_eq:NN \l_regex_tmpa_int \l_regex_state_left_int
    \regex_build_transitions_aux:NNNN
      \regex_action_free_copy:n \l_regex_state_right_int
      \regex_action_free_move:n \l_regex_tmpa_int
  }
%    \end{macrocode}
% \end{macro}
% \end{macro}
%
% \begin{macro}[aux]{\regex_build_one_n:}
% \begin{macro}[aux]{\regex_build_one_n?:}
%   This function is called in case the syntax is
%   \texttt{\{\meta{int}\}}. Greedy and lazy operators
%   are identical, since the number of repetitions is fixed.
%    \begin{macrocode}
\cs_new_protected_nopar:Npn \regex_build_one_n: #1
  {
    \int_set_eq:NN \l_regex_tmpa_int \l_regex_state_right_int
    \regex_build_new_state:
    \regex_build_new_state:
    \regex_toks_put_right:Nx \l_regex_tmpa_int
      {
        \exp_not:N \if_num:w #1 > \l_regex_repetition_int
          \regex_action_repeat_move:n
            { \int_use:N \l_regex_state_left_int }
        \exp_not:N \else:
          \regex_action_no_repeat_move:n
            { \int_use:N \l_regex_state_right_int }
        \exp_not:N \fi:
      }
    \regex_toks_put_right:Nx \l_regex_state_left_int
      {
        \regex_build_tmp_class:n
          { \int_use:N \l_regex_tmpa_int }
      }
  }
\cs_new_eq:cN { regex_build_one_n?: } \regex_build_one_n:
%    \end{macrocode}
% \end{macro}
% \end{macro}
%
% \begin{macro}[aux]{\regex_build_one_nn:}
% \begin{macro}[aux]{\regex_build_one_nn?:}
% \begin{macro}[aux]{\regex_build_one_nn_aux:nn}
%   This function is called when the syntax is either
%   \texttt{\{\meta{int},\}} or \texttt{\{\meta{int},\meta{int}\}}.
%    \begin{macrocode}
\cs_new_protected_nopar:Npn \regex_build_one_nn: #1#2
  {
    \regex_build_one_nn_aux:nn {#1}
      {
        \exp_not:N \if_num:w #2 > \l_regex_repetition_int
          \regex_action_repeat_copy:n
            { \int_use:N \l_regex_state_left_int }
        \exp_not:N \fi:
        \regex_action_no_repeat_move:n
          { \int_use:N \l_regex_state_right_int }
      }
  }
\cs_new_protected_nopar:cpn { regex_build_one_nn?: } #1#2
  {
    \regex_build_one_nn_aux:nn {#1}
      {
        \regex_action_no_repeat_copy:n
          { \int_use:N \l_regex_state_right_int }
        \exp_not:N \if_num:w #2 > \l_regex_repetition_int
          \regex_action_repeat_move:n
            { \int_use:N \l_regex_state_left_int }
        \exp_not:N \fi:
      }
  }
\cs_new_protected_nopar:Npn \regex_build_one_nn_aux:nn #1#2
  {
    \int_set_eq:NN \l_regex_tmpa_int \l_regex_state_right_int
    \regex_build_new_state:
    \regex_build_new_state:
    \regex_toks_put_right:Nx \l_regex_tmpa_int
      {
        \exp_not:N \if_num:w #1 > \l_regex_repetition_int
          \regex_action_repeat_move:n
            { \int_use:N \l_regex_state_left_int }
        \exp_not:N \else:
          #2
        \exp_not:N \fi:
      }
    \regex_toks_put_right:Nx \l_regex_state_left_int
      {
        \regex_build_tmp_class:n
          { \int_use:N \l_regex_tmpa_int }
      }
  }
%    \end{macrocode}
% \end{macro}
% \end{macro}
% \end{macro}
%
% \subsubsection{Groups and alternation}
%
% We support the syntax \texttt{(\meta{expr1}|\ldots{}%^^A
%   |\meta{expr$\sb{n}$})\meta{quantifier}} for alternations.
%
% \begin{macro}[aux]{\regex_build_(:, \regex_build_):}
% \begin{macro}[aux]+\regex_build_|:+
% \begin{macro}[aux]{\regex_build_begin_alternation:,
%     \regex_build_end_alternation:}
%   Grouping and alternation go together.
%   \begin{itemize}
%   \item Allocate the next available number for the end vertex
%     of the alternation/group and store it on a stack (so that nested
%     alternations work).
%   \item Put free transitions to separate all cases of the alternation.
%   \item Build each branch separately, and merge them to the common
%     end-node.
%   \item Test for a quantifier, and if needed, transfer the initial
%     vertex to a new vertex.
%   \end{itemize}
%    \begin{macrocode}
\cs_new_protected_nopar:cpn { regex_build_(: } #1#2
  {
    \str_if_eq:nnTF { #1 #2 } { \regex_build_control:N ? }
      { \regex_build_special_group:NN }
      {
        \regex_build_new_state:
        \int_incr:N \l_regex_capturing_group_int
        \seq_push:Nx \l_regex_capturing_group_seq
          { \int_use:N \l_regex_capturing_group_int }
        \seq_push:Nx \l_regex_state_left_seq
          { \int_use:N \l_regex_state_left_int }
        \seq_push:Nx \l_regex_state_right_seq
          { \int_use:N \l_regex_state_right_int }
        \regex_build_begin_alternation:
        #1 #2
      }
  }
\cs_new_protected_nopar:cpn { regex_build_|: }
  {
    \regex_build_end_alternation:
    \regex_build_begin_alternation:
  }
\cs_new_protected_nopar:cpn { regex_build_): }
  {
    \seq_if_empty:NTF \l_regex_capturing_group_seq
      { \regex_build_error:n {22} }
      {
        \regex_build_end_alternation:
        \seq_pop:NN \l_regex_state_left_seq  \l_regex_tmpa_tl
        \int_set:Nn \l_regex_state_left_int  \l_regex_tmpa_tl
        \seq_pop:NN \l_regex_state_right_seq \l_regex_tmpa_tl
        \int_set:Nn \l_regex_state_right_int \l_regex_tmpa_tl
            %^^A obsolete the correct thread.
            % \regex_toks_put_right:Nx \l_regex_left_int
            %   { \regex_thread_obsolete:n { \int_use:N \l_regex_left_int } }
        \regex_build_group_quantifier:
      }
  }
%    \end{macrocode}
%   Building each branch.
%    \begin{macrocode}
\cs_new_protected_nopar:Npn \regex_build_begin_alternation:
  {
    \regex_build_new_state:
    \seq_get:NN \l_regex_state_left_seq \l_regex_tmpa_tl
    \int_set:Nn \l_regex_state_left_int \l_regex_tmpa_tl
    \regex_toks_put_right:Nx \l_regex_state_left_int
      {
        \regex_action_free_copy:n
          { \int_use:N \l_regex_state_right_int }
      }
  }
\cs_new_protected_nopar:Npn \regex_build_end_alternation:
  {
    \seq_get:NN \l_regex_state_right_seq \l_regex_tmpa_tl
    \regex_toks_put_right:Nx \l_regex_state_right_int
      {
        \regex_action_free_move:n
          { \l_regex_tmpa_tl }
      }
  }
%    \end{macrocode}
% \end{macro}
% \end{macro}
% \end{macro}
%
% \begin{macro}{\regex_build_special_group:NN}
%   Same method as elsewhere: if the combination |(?#1| ^^A )
%   is known, then use that. Otherwise, treat the question mark
%   as if it had been escaped.
%    \begin{macrocode}
\cs_new_protected_nopar:Npn \regex_build_special_group:NN #1#2
  {
    \cs_if_exist_use:cF { regex_build_special_group_\token_to_str:N #2 : }
      {
        \regex_build_error:n { -998 }
        \regex_build_control:N ( % )
        \regex_build_raw:N ?
        #1 #2
      }
  }
%    \end{macrocode}
% \end{macro}
%
% \begin{macro}{\regex_build_special_group_::}
%   Non-capturing groups are like capturing groups, except that
%   we set the group id to \texttt{*}, which will then inhibit
%   submatching in \cs{regex_build_group_submatches:NN}.
%   The group number is not increased.
%    \begin{macrocode}
\cs_new_protected_nopar:cpn { regex_build_special_group_:: }
  {
    \regex_build_new_state:
    \seq_push:Nx \l_regex_capturing_group_seq { * }
    \seq_push:Nx \l_regex_state_left_seq
      { \int_use:N \l_regex_state_left_int }
    \seq_push:Nx \l_regex_state_right_seq
      { \int_use:N \l_regex_state_right_int }
    \regex_build_begin_alternation:
  }
%    \end{macrocode}
% \end{macro}
%
% \subsubsection{Quantifiers for groups}
%
% \begin{macro}[aux]{\regex_build_group_quantifier:}
%   Used for one group. We need to keep track of submatches,
%   threads can be created within one repetition, so things are hard.
%   The code for the group that was just built starts
%   at \cs{l_regex_state_left_int} and ends at
%   \cs{l_regex_state_right_int}.
%    \begin{macrocode}
\cs_new_protected_nopar:Npn \regex_build_group_quantifier:
  {
    \tl_set:Nn \l_regex_one_or_group_tl { group }
    \regex_build_quantifier:w
  }
%    \end{macrocode}
% \end{macro}
%
% \begin{macro}[aux]{\regex_build_group_submatches:NN}
%   Once the quantifier is found by \cs{regex_build_quantifier:w},
%   we insert the code for tracking submatches.
%    \begin{macrocode}
\cs_new_protected_nopar:Npn \regex_build_group_submatches:NN #1#2
  {
    \seq_pop:NN \l_regex_capturing_group_seq \l_regex_tmpa_tl
    \str_if_eq:xxF { \l_regex_tmpa_tl } { * }
      {
        \regex_toks_put_left:Nx #1
          { \regex_action_submatch:n { \l_regex_tmpa_tl < } }
        \regex_toks_put_left:Nx #2
          { \regex_action_submatch:n { \l_regex_tmpa_tl > } }
      }
  }
%    \end{macrocode}
% \end{macro}
%
% \begin{macro}[aux]{\regex_build_group_:}
%   When there is no quantifier, the group is simply inserted as is,
%   and we only need to track submatches.
%    \begin{macrocode}
\cs_new_protected_nopar:cpn { regex_build_group_: }
  {
    \regex_build_group_submatches:NN
      \l_regex_state_left_int \l_regex_state_right_int
    \regex_build_transition_aux:NN
      \regex_action_free_move:n \l_regex_state_right_int
  }
%    \end{macrocode}
% \end{macro}
%
% \begin{macro}[aux]{\regex_build_group_shift:N}
%   Most quantifiers require to add an extra state before the group.
%   This is done by shifting the current contents of the \cs{tex_toks:D}
%   \cs{l_regex_tmpa_int} to a new state.
%    \begin{macrocode}
\cs_new_protected_nopar:Npn \regex_build_group_shift:N #1
  {
    \int_set_eq:NN \l_regex_tmpa_int \l_regex_state_left_int
    \regex_build_new_state:
    \tex_toks:D \l_regex_state_right_int = \tex_toks:D \l_regex_tmpa_int
    \use:x
      {
        \tex_toks:D \l_regex_tmpa_int
          { #1 { \int_use:N \l_regex_state_right_int } }
          % ^^A incorrect? Does "move" have to come after "copy"?
      }
    \regex_build_group_submatches:NN
      \l_regex_state_right_int \l_regex_state_left_int
    \int_set_eq:NN \l_regex_state_right_int \l_regex_state_left_int
  }
%    \end{macrocode}
% \end{macro}
%
% \begin{macro}[aux]{\regex_build_group_qs_aux:NN}
% \begin{macro}[aux]{\regex_build_group_?:}
% \begin{macro}[aux]{\regex_build_group_??:}
% \begin{macro}[aux]{\regex_build_group_*:}
% \begin{macro}[aux]{\regex_build_group_*?:}
%   Shift the state at which the group begins using
%   \cs{regex_build_group_shift:N}, then add two transitions.
%   The first transition is taken once the group has been
%   traversed: in the case of \texttt{?} and \texttt{??},
%   we should exit by going to \cs{l_regex_state_right_int},
%   while for \texttt{*} and \texttt{*?} we loop by going to
%   \cs{l_regex_tmpa_int}.
%   The second transition corresponds to skipping the group;
%   it has lower priority (\texttt{put_right}) for greedy
%   operators, and higher priority (\texttt{put_left}) for
%   lazy operators.
%    \begin{macrocode}
\cs_new_protected_nopar:Npn \regex_build_group_qs_aux:NN #1#2
  {
    \regex_build_group_shift:N \regex_action_free_copy:n
    \regex_build_transition_aux:NN \regex_action_free_move:n #1
    #2 \l_regex_tmpa_int
      {
        \regex_action_free_move:n
          { \int_use:N \l_regex_state_right_int }
      }
  }
\cs_new_protected_nopar:cpn { regex_build_group_?: }
  {
    \regex_build_group_qs_aux:NN
      \l_regex_state_right_int \regex_toks_put_right:Nx
  }
\cs_new_protected_nopar:cpn { regex_build_group_??: }
  {
    \regex_build_group_qs_aux:NN
      \l_regex_state_right_int \regex_toks_put_left:Nx
  }
\cs_new_protected_nopar:cpn { regex_build_group_*: }
  {
    \regex_build_group_qs_aux:NN
      \l_regex_tmpa_int \regex_toks_put_right:Nx
  }
\cs_new_protected_nopar:cpn { regex_build_group_*?: }
  {
    \regex_build_group_qs_aux:NN
      \l_regex_tmpa_int \regex_toks_put_left:Nx
  }
%    \end{macrocode}
% \end{macro}
% \end{macro}
% \end{macro}
% \end{macro}
% \end{macro}
%
% \begin{macro}[aux]{\regex_build_group_+:}
% \begin{macro}[aux]{\regex_build_group_+?:}
%   Insert the submatch tracking code, then add two transitions
%   from the current state to the left end of the group (repeating the group),
%   and to a new state (to carry on with the rest of the regular expression).
%    \begin{macrocode}
\cs_new_protected_nopar:cpn { regex_build_group_+: }
  {
    \regex_build_group_submatches:NN
      \l_regex_state_left_int \l_regex_state_right_int
    \int_set_eq:NN \l_regex_tmpa_int \l_regex_state_left_int
    \regex_build_transitions_aux:NNNN
      \regex_action_free_copy:n \l_regex_tmpa_int
      \regex_action_free_move:n \l_regex_state_right_int
  }
\cs_new_protected_nopar:cpn { regex_build_group_+?: }
  {
    \regex_build_group_submatches:NN
      \l_regex_state_left_int \l_regex_state_right_int
    \int_set_eq:NN \l_regex_tmpa_int \l_regex_state_left_int
    \regex_build_transitions_aux:NNNN
      \regex_action_free_copy:n \l_regex_state_right_int
      \regex_action_free_move:n \l_regex_tmpa_int
  }
%    \end{macrocode}
% \end{macro}
% \end{macro}
%
% \begin{macro}[aux]{\regex_build_group_n:}
% \begin{macro}[aux]{\regex_build_group_n?:}
%   These functions are called in case the syntax is
%   \texttt{\{\meta{int}\}}. Greedy and lazy operators
%   are identical, since the number of repetitions is fixed.
%    \begin{macrocode}
\cs_new_protected_nopar:Npn \regex_build_group_n: #1
  {
    \regex_build_group_shift:N \regex_action_repeat_move:n
    \regex_build_transition_aux:NN
      \regex_action_free_move:n \l_regex_tmpa_int
    \use:x
      {
        \tex_toks:D \l_regex_tmpa_int
          {
            \exp_not:N \if_num:w #1 > \l_regex_repetition_int
              \tex_the:D \tex_toks:D \l_regex_tmpa_int
            \exp_not:N \else:
              \regex_action_no_repeat_move:n
                { \int_use:N \l_regex_state_right_int }
            \exp_not:N \fi:
          }
      }
  }
\cs_new_eq:cN { regex_build_group_n?: } \regex_build_group_n:
%    \end{macrocode}
% \end{macro}
% \end{macro}
%
% \begin{macro}[aux]{\regex_build_group_nn:}
% \begin{macro}[aux]{\regex_build_group_nn?:}
%   These functions are called when the syntax is either
%   \texttt{\{\meta{int},\}} or \texttt{\{\meta{int},\meta{int}\}}.
%    \begin{macrocode}
\cs_new_protected_nopar:Npn \regex_build_group_nn: #1#2
  {
    \regex_build_group_shift:N \regex_action_repeat_move:n
    \regex_build_transition_aux:NN
      \regex_action_free_move:n \l_regex_tmpa_int
    \use:x
      {
        \tex_toks:D \l_regex_tmpa_int
          {
            \exp_not:N \if_num:w #1 > \l_regex_repetition_int
              \tex_the:D \tex_toks:D \l_regex_tmpa_int
            \exp_not:N \else:
              \exp_not:N \if_num:w #2 > \l_regex_repetition_int
                \tex_the:D \tex_toks:D \l_regex_tmpa_int
              \exp_not:N \fi:
              \regex_action_no_repeat_copy:n
                { \int_use:N \l_regex_state_right_int }
            \exp_not:N \fi:
          }
      }
  }
\cs_new_protected_nopar:cpn { regex_build_group_nn?: } #1#2
  {
    \regex_build_group_shift:N \regex_action_repeat_move:n
    \regex_build_transition_aux:NN
      \regex_action_free_move:n \l_regex_tmpa_int
    \use:x
      {
        \tex_toks:D \l_regex_tmpa_int
          {
            \exp_not:N \if_num:w #1 > \l_regex_repetition_int
              \tex_the:D \tex_toks:D \l_regex_tmpa_int
            \exp_not:N \else:
              \regex_action_no_repeat_copy:n
                { \int_use:N \l_regex_state_right_int }
              \exp_not:N \if_num:w #2 > \l_regex_repetition_int
                \tex_the:D \tex_toks:D \l_regex_tmpa_int
              \exp_not:N \fi:
            \exp_not:N \fi:
          }
      }
  }
%    \end{macrocode}
% \end{macro}
% \end{macro}
%
% \subsection{Matching}
%
% \subsubsection{Use of \TeX{} registers when matching}
%
% The first step in matching a regular expression is to build
% the corresponding NFA and store its states in the \tn{toks}
% registers. Then loop through the query string one character
% (one \enquote{step}) at a time. While going through the string,
% keep track of various \enquote{threads} of execution through
% the NFA.
%
% For each thread, we must keep track of the state in which it
% is at this step, as well as submatches.
% Most of the instructions of the NFA are transitions from the
% current state to a new state.
% The NFA has two main types of instructions: instruction which
% \enquote{consume} the current character (such as explicit
% characters, or character classes), and instructions which
% match $0$ characters (such as branching instructions to take
% care of alternation, or quantifiers).
% In the first case, the thread is moved to the new state, and
% put in an array to be considered at the next step.
% In the second case, the thread moves to the target state, and
% the instructions for that state are performed immediately,
% since they pertain to the current character.
%
% The presence of $\epsilon$-transitions (transitions which
% consume no character) leads to potential infinite loops;
% for instance the regular expression |(a??)*| could lead to
% an infinite recursion, where |a??| matches no character,
% |*| loops back to the start of the group, and |a??| matches
% no character again. Therefore, we need to keep track of
% the states of the NFA visited at the current step. More
% precisely, a state is marked as \enquote{visited} if the
% instructions for that state have been inserted in the input
% stream, by setting the corresponding \tn{dimen} register to
% \texttt{1sp}.
%
% We store the various threads in an array in order of precedence
% (where precedence depends for instance on whether quantifiers
% are greedy or lazy). This is done using the \tn{skip} registers,
% which give us $3$ integers per register. We only use two of those.
% Namely, the main part of \tn{skip}$i$ is the thread \texttt{id}
% which we put in that slot of the array, and the stretch component
% is the state of the NFA in which that thread is (both are converted
% from a dimension to an integer with $1\mathtt{sp}\to 1$).
%
% The current approach means that the shrink component of \tn{skip}s,
% as well as all \tn{muskip} registers are unused. It could seem that
% \tn{count} registers are also free for use, but we still want to be
% able to safely use integers, which are implemented as \tn{count}
% registers.
%
% \subsubsection{Helpers for running the NFA}
%
% \begin{macro}[aux]{\regex_if_state_free:nTF}
%   A state is free if it is not marker as taken, namely
%   if the corresponding \tn{dimen} register is $0\mathtt{sp}$
%   rather than $1\mathtt{sp}$.
%    \begin{macrocode}
\cs_new_protected:Npn \regex_if_state_free:nTF #1
  {
    \if_int_odd:w \tex_dimen:D #1 \scan_stop:
      \exp_after:wN \use_ii:nn
    \else:
      \exp_after:wN \use_i:nn
    \fi:
  }
%    \end{macrocode}
% \end{macro}
%
% \begin{macro}[aux]{\regex_put_thread_in_state:Nn}
%   Put the given thread in the array of \tn{skip} registers,
%   in a given state. This is done by increasing the pointer
%   \cs{l_regex_max_index_int}, and converting both integers
%   to dimensions (suitable for a \tn{skip} assignment) in
%   scaled points (\texttt{sp}).
%    \begin{macrocode}
\cs_new_protected:Npn \regex_put_thread_in_state:Nn #1 #2
  {
    \int_incr:N \l_regex_max_index_int
    \tex_skip:D \l_regex_max_index_int #1 sp plus #2 sp \scan_stop:
  }
%    \end{macrocode}
% \end{macro}
%
% \begin{macro}[aux]{\regex_state_use:N}
%   Use a given program instruction, unless it has already been
%   executed at this step.
%    \begin{macrocode}
\cs_new_protected_nopar:Npn \regex_state_use:N #1
  {
    \regex_if_state_free:nTF { #1 }
      {
        \tex_dimen:D #1 = \c_one sp \scan_stop:
        \tex_the:D \tex_toks:D #1 \scan_stop:
      }
      {
%<trace> \regex_trace:x { Already~seen~state~\int_value:w #1. }
      } % ^^A should it be fail instead ?
  }
%    \end{macrocode}
% \end{macro}
%
% \subsubsection{Matching: framework}
%
% \begin{macro}[int]{\regex_match:n}
%   Store the query string in \cs{l_regex_query_str}.
%   Then reset a few variables which should be set only once,
%   before the first match, even in the case of multiple matches.
%   Then run the NFA once: in the case of multiple matches, the function
%   \cs{regex_match_once:} takes care of calling itself again.
%    \begin{macrocode}
\cs_new_protected:Npn \regex_match:n #1
  {
    \tl_set:Nx \l_regex_query_str { \tl_to_other_str:n {#1} }
    \regex_match_initial_setup:
    \regex_match_once:
  }
%    \end{macrocode}
% \end{macro}
%
% \begin{macro}[int]{\regex_match_once:}
% \begin{macro}[aux]{\regex_match_once_aux:}
%   After setting up more variables in \cs{regex_match_setup:},
%   skip the \cs{l_regex_start_step_int} first characters of the
%   query string, and loop over it.
%    \begin{macrocode}
\cs_new_protected_nopar:Npn \regex_match_once:
  {
    \regex_match_setup:
    \exp_after:wN \regex_match_once_aux: \l_regex_query_str
      \q_recursion_tail \q_recursion_stop
    \int_compare:nNnF \l_regex_success_thread_int = \c_zero
      { \l_regex_every_match_tl }
  }
\cs_new_protected_nopar:Npn \regex_match_once_aux:
  { \str_skip_do:nn { \l_regex_start_step_int } { \regex_match_loop:N } }
%    \end{macrocode}
% \end{macro}
% \end{macro}
%
% \begin{macro}[aux]{\regex_match_initial_setup:}
% \begin{macro}[aux]{\regex_match_loop_setup:}
% \begin{macro}[aux]{\regex_match_loop_setup_aux:n}
%   This function holds the setup that should be done
%   only once for one given pattern matching on a given
%   string. It is called only once for the whole string.
%   On the other hand, \cs{regex_match_setup:}
%   is called for every match in the string in case of
%   repeated matches, and \cs{regex_match_loop_setup:}
%   is called at every step.
%   We define \cs{regex_match_loop_setup:} here with
%   \texttt{x}-expansion to hold an explicit list of
%   all the dimensions that must be taken to zero at
%   every character by \cs{regex_match_loop_setup:}. Otherwise,
%   we would need to expand it at every character, and the
%   \cs{prg_stepwise_\ldots{}} functions are rather slow.
%    \begin{macrocode}
\cs_new_protected_nopar:Npn \regex_match_initial_setup:
  {
    \tl_clear:N \l_regex_look_behind_str
    \int_set_eq:NN \l_regex_start_step_int \c_minus_one
    \int_set_eq:NN \l_regex_current_step_int \c_zero
    \int_set_eq:NN \l_regex_success_step_int \c_zero
    \bool_set_false:N \l_regex_success_empty_bool
    \cs_set_protected_nopar:Npx \regex_match_loop_setup:
      {
        \prg_stepwise_function:nnnN
          {1} {1} { \l_regex_state_max_int }
          \regex_match_loop_setup_aux:n
        \scan_stop:
        \int_incr:N \l_regex_current_step_int
        \cs_set_eq:NN \regex_obsolete_all:F \regex_obsolete_all_no:F
      }
  }
\cs_new_protected_nopar:Npn \regex_match_loop_setup: { }
\cs_new_nopar:Npn \regex_match_loop_setup_aux:n #1
  { \tex_dimen:D #1 \c_zero pt }
%    \end{macrocode}
% \end{macro}
% \end{macro}
% \end{macro}
%
% \begin{macro}[aux]{\regex_match_setup:}
%   Hopefully, we didn't forget to initialize anything here.
%    \begin{macrocode}
\cs_new_protected_nopar:Npn \regex_match_setup:
  {
%<trace>    \regex_trace:x { Match~setup... }
    \prop_clear:N \l_regex_submatches_prop
    \bool_if:NTF \l_regex_success_empty_bool
      { \cs_set_eq:NN \regex_last_match_empty:F \regex_last_match_empty_yes:F }
      { \cs_set_eq:NN \regex_last_match_empty:F \regex_last_match_empty_no:F }
    \int_set_eq:NN \l_regex_start_step_int \l_regex_success_step_int
    \int_set_eq:NN \l_regex_current_step_int \l_regex_start_step_int
    \int_decr:N \l_regex_current_step_int
    \int_zero:N \l_regex_max_index_int
    \int_set_eq:NN \l_regex_max_thread_int \c_one
    \int_zero:N \l_regex_success_thread_int
    \regex_put_thread_in_state:Nn \c_one {1}
  }
%    \end{macrocode}
% \end{macro}
%
% \begin{macro}[aux]{\regex_match_loop:N}
% \begin{macro}[aux]{\regex_match_one_index:n}
% \begin{macro}[aux]{\regex_match_one_index_aux:nn}
%   Setup what needs to be reset at every character,
%   then set \cs{l_regex_current_char_int} to the
%   character code of the character that is read
%   (and $-1$ for the end of the string), and loop
%   over the elements of the \texttt{A} array.
%    \begin{macrocode}
\cs_new_protected_nopar:Npn \regex_match_loop:N #1
  {
    \regex_match_loop_setup:
    \token_if_eq_meaning:NNTF #1 \q_recursion_tail
      { \int_set_eq:NN \l_regex_current_char_int \c_minus_one }
      { \int_set:Nn \l_regex_current_char_int {`#1} }
%<trace>    \regex_trace:x { Read~'\token_to_str:N #1',
%<trace>      ~charcode~\int_use:N \l_regex_current_char_int . }
    \cs_set_nopar:Npx \regex_tmp:w
      {
        \int_zero:N \l_regex_max_index_int
        \prg_stepwise_function:nnnN
          {1} {1} { \l_regex_max_index_int }
          \regex_match_one_index:n
      }
    \regex_tmp:w
    \if_num:w \l_regex_max_index_int = \c_zero
      \exp_after:wN \use_none_delimit_by_q_recursion_stop:w
    \fi:
    \quark_if_recursion_tail_stop:N #1
    \bool_if:NT \l_regex_look_behind_bool
      { \tl_put_left:Nx \l_regex_look_behind_str {#1} }
    \regex_match_loop:N
  }
\cs_new_nopar:Npn \regex_match_one_index:n #1
  {
%<trace>    \regex_trace_ist:n {Begin~index~#1}
    \regex_match_one_index_aux:nn
      { \int_value:w \tex_skip:D #1 }
      { \int_value:w \etex_gluestretch:D \tex_skip:D #1 }
%<trace>    \regex_trace_ist:n {End~index~#1}
  }
\cs_new_protected_nopar:Npn \regex_match_one_index_aux:nn #1#2
  {
    \int_set:Nn \l_regex_current_thread_int {#1}
    \int_set:Nn \l_regex_current_state_int {#2}
    \regex_obsolete_all:F
      { \regex_state_use:N \l_regex_current_state_int }
  }
%    \end{macrocode}
% \end{macro}
% \end{macro}
% \end{macro}
%
% \subsubsection{Actions when matching}
%
% \begin{macro}[aux]{\regex_action_start_wildcard:nn}
%   The search is made unanchored at the start by putting
%   a free transition to the real start of the NFA, and a
%   costly transition to the same state, waiting for the
%   next character in the query string. This combination
%   could be reused (with some changes). We sometimes need
%   to know that the match for a given thread starts at
%   this character. For that, we use the boolean
%   \cs{l_regex_fresh_thread_bool}.
%    \begin{macrocode}
\cs_new_protected_nopar:Npn \regex_action_start_wildcard:nn #1#2
  {
    \bool_set_true:N \l_regex_fresh_thread_bool
    \regex_action_free_copy:n {#2}
    \bool_set_false:N \l_regex_fresh_thread_bool
    \regex_action_cost:n {#1}
  }
%    \end{macrocode}
% \end{macro}
%
% \begin{macro}[aux]{\regex_action_cost:n}
%    \begin{macrocode}
\cs_new_protected_nopar:Npn \regex_action_cost:n #1
  {
%<trace> \regex_trace:x { Cost~t=\thread\ (s=\state\ ->~#1). }
    \regex_obsolete_all:F
      { \regex_put_thread_in_state:Nn \l_regex_current_thread_int {#1} }
  }
%    \end{macrocode}
% \end{macro}
%
% \begin{macro}[aux]{\regex_action_fail:}
%    \begin{macrocode}
\cs_new_protected_nopar:Npn \regex_action_fail:
  {
%<trace> \regex_trace:x { Fail~t=\thread. }
    \regex_if_track_submatches:T
      {
        % %^^A todo: move that to the "end-of-toks" marker!?
        % \prop_del:No \l_regex_submatches_prop
        %   { \int_use:N \l_regex_current_thread_int }
        % \prop_show:N \l_regex_submatches_prop
      }
  }
%    \end{macrocode}
% \end{macro}
%
% \begin{macro}[aux]{\regex_action_success:}
%    \begin{macrocode}
\cs_new_protected_nopar:Npn \regex_action_success:
  {
    \regex_obsolete_all:F
      {
        \regex_last_match_empty:F
          {
%<trace>             \regex_trace:x { Success~(t=\thread). }
            \cs_set_eq:NN \regex_obsolete_all:F \regex_obsolete_all_yes:F
            \int_set_eq:NN \l_regex_success_thread_int
              \l_regex_current_thread_int
            \bool_set_eq:NN \l_regex_success_empty_bool
              \l_regex_fresh_thread_bool
            \int_set_eq:NN \l_regex_success_step_int
              \l_regex_current_step_int
            \regex_if_track_submatches:T
              {
                \prop_pop:NoNF \l_regex_submatches_prop
                  { \int_use:N \l_regex_current_thread_int }
                  \l_regex_success_submatches_prop
                  { \prop_clear:N \l_regex_success_submatches_prop }
%<trace>                \prop_show:N \l_regex_submatches_prop
%<trace>                \prop_show:N \l_regex_success_submatches_prop
              }
          }
      }
  }
%    \end{macrocode}
% \end{macro}
%
% \begin{macro}[aux]{\regex_action_free_move:n}
%   To move a thread, change the current program state,
%   but not the current thread.
%    \begin{macrocode}
\cs_new_protected_nopar:Npn \regex_action_free_move:n #1
  {
%<trace>    \regex_trace:x { Move~t=\thread\ (s=\state\ ->~#1). }
    \int_set:Nn \l_regex_current_state_int {#1}
    \regex_obsolete_all:F
      { \regex_state_use:N \l_regex_current_state_int }
  }
%    \end{macrocode}
% \end{macro}
%
% \begin{macro}[aux]{\regex_action_free_copy:n}
% \begin{macro}[aux]{\regex_action_free_copy_aux:}
%   To copy a thread, check whether the program state has already
%   been used at this character. If not, create a new thread
%   in that new state, copy submatches, and insert the instructions
%   for that state in the input stream.
%   Then restore the old values of \cs{l_regex_current_thread_int}
%   and \cs{l_regex_current_state_int}.
%    \begin{macrocode}
\cs_new_protected_nopar:Npn \regex_action_free_copy:n #1
  {
%<trace> \regex_trace:x { Copy~t=\thread\ (s=\state\ ->~#1). }
    \regex_obsolete_all:F
      {
        \regex_if_state_free:nTF {#1}
          {
            \int_incr:N \l_regex_max_thread_int
            \use:x
              {
                \int_set:Nn \l_regex_current_state_int {#1}
                \int_set_eq:NN \l_regex_current_thread_int
                  \l_regex_max_thread_int
                \regex_action_free_copy_aux:n
                  { \int_use:N \l_regex_current_thread_int }
                \int_set:Nn \l_regex_current_state_int
                  { \int_use:N \l_regex_current_state_int }
                \int_set:Nn \l_regex_current_thread_int
                  { \int_use:N \l_regex_current_thread_int }
              }
          }
          {
%<trace> \regex_trace:x { [Ignored:~state~#1~already~encountered.] }
          } %^^A  should it be fail?
      }
  }
\cs_new_protected_nopar:Npn \regex_action_free_copy_aux:n #1
  {
    \regex_if_track_submatches:T
      {
        \prop_get:NoNT \l_regex_submatches_prop {#1} \l_regex_tmpa_tl
          {
            \prop_put:Noo \l_regex_submatches_prop
              { \int_use:N \l_regex_current_thread_int }
              { \l_regex_tmpa_tl }
          }
%<trace> \regex_trace:x { [New~thread~\thread] }
%<trace> \prop_show:N \l_regex_submatches_prop
      }
    \regex_obsolete_all:F
      { \regex_state_use:N \l_regex_current_state_int }
  }
%    \end{macrocode}
% \end{macro}
% \end{macro}
%
% \begin{macro}[aux]{\regex_action_submatch:n}
%    \begin{macrocode}
\cs_new_protected_nopar:Npn \regex_action_submatch:n #1
  {
    \regex_if_track_submatches:T
      {
        \prop_pop:NoNF \l_regex_submatches_prop
          { \int_use:N \l_regex_current_thread_int }
          \l_regex_tmpa_prop
          { \prop_clear:N \l_regex_tmpa_prop }
        \prop_put:Nno \l_regex_tmpa_prop {#1}
          { \int_use:N \l_regex_current_step_int }
        \prop_put:Noo \l_regex_submatches_prop
          { \int_use:N \l_regex_current_thread_int }
          { \l_regex_tmpa_prop }
%<trace> \regex_trace:x { Submatch:~s=\state,~t=\thread,~m=#1,~c=\charstep. }
%<trace> \prop_show:N \l_regex_submatches_prop
      }
  }
%    \end{macrocode}
% \end{macro}
%
%    \begin{macrocode}
\cs_new_protected_nopar:Npn \regex_action_repeat_copy:n { \ERROR }
\cs_new_protected_nopar:Npn \regex_action_repeat_move:n { \ERROR }
\cs_new_protected_nopar:Npn \regex_action_no_repeat_copy:n { \ERROR }
\cs_new_protected_nopar:Npn \regex_action_no_repeat_move:n { \ERROR }
%    \end{macrocode}
%
% \subsection{Submatches, once the correct match is found}
%
% \begin{macro}[int]{\regex_extract:}
% \begin{macro}[aux]{\regex_extract_aux:nTF}
%    \begin{macrocode}
\cs_new_protected_nopar:Npn \regex_extract:
  {
    \seq_gclear:N \g_regex_submatches_seq
%<trace>    \prop_show:N \l_regex_success_submatches_prop
    \prg_stepwise_inline:nnnn
      {0} {1} { \l_regex_capturing_group_int }
      {
        \regex_extract_aux:nTF { ##1 }
          {
            \seq_gput_right:Nx \g_regex_submatches_seq
              {
                \str_from_to:Nnn \l_regex_query_str
                  { \l_regex_tmpa_tl }
                  { \l_regex_tmpb_tl }
              }
          }
          { \seq_gput_right:Nn \g_regex_submatches_seq { } }
      }
  }
\cs_new_protected_nopar:Npn \regex_extract_aux:nTF #1#2#3
  {
    \prop_get:NnNTF \l_regex_success_submatches_prop
      { #1 < } \l_regex_tmpa_tl
      {
        \prop_get:NnNTF \l_regex_success_submatches_prop
          { #1 > } \l_regex_tmpb_tl
          {#2}
          {#3}
      }
      {#3}
  }
%    \end{macrocode}
% \end{macro}
% \end{macro}
%
% \subsection{User commands}
%
% \subsubsection{Precompiled pattern}
%
% A given pattern is often reused to match many different strings.
% We thus give a means of storing the NFA corresponding to a given
% pattern in a token list variable of the form
% \begin{quote}
%   \cs{regex_nfa:Nw} \meta{variable~name} \\
%   \meta{assignments} \\
%   \cs{tex_toks:D} 0 \{ \meta{instruction0} \} \\
%   \ldots{}                              \\
%   \cs{tex_toks:D} $n$ \{ \meta{instruction$\sb{n}$} \} \\
%   \cs{scan_stop:}
% \end{quote}
% where $n$ is the number of states in the NFA,
% and the various \meta{instruction$\sb{i}$} control
% how the NFA behaves in state $i$. The \cs{regex_nfa:Nw}
% function removes the whole NFA from the input stream
% and produces an error: the \meta{nfa var} should only be
% accessed through dedicated functions. This rather drastic
% approach is taken because assignments triggered by the
% contents of \meta{nfa var} may overwrite data which is used
% elsewhere, unless everything is done carefully in a group.
%
% \begin{macro}{\regex_set:Nn}
% \begin{macro}{\regex_gset:Nn}
% \begin{macro}[aux]{\regex_set_aux:NNn}
% \begin{macro}[aux]{\regex_set_aux:n}
% \begin{macro}[aux]{\regex_nfa:Nw}
%   Within a group, build the NFA corresponding to the given regular
%   expression, with submatch tracking. Then save the contents of all
%   relevant \tn{toks} registers into \cs{g_regex_tmpa_tl}, then
%   transferred to the user's tl variable.
%   The auxiliary \cs{regex_nfa:Nw} is not protected: this ensures that
%   the NFA will properly be replaced by an error message in expansion
%   contexts.
%    \begin{macrocode}
\cs_new_protected_nopar:Npn \regex_set:Nn
  { \regex_set_aux:NNn \tl_set_eq:NN }
\cs_new_protected_nopar:Npn \regex_gset:Nn
  { \regex_set_aux:NNn \tl_gset_eq:NN }
\cs_new_protected:Npn \regex_set_aux:NNn #1#2#3
  {
    \group_begin:
      \cs_set_eq:NN \regex_if_track_submatches:T \use:n
      \regex_build:n {#3}
      \tl_gset:Nx \g_regex_tmpa_tl
        {
          \exp_not:N \regex_nfa:Nw \exp_not:N #2
          \l_regex_state_max_int
            = \int_use:N \l_regex_state_max_int
          \l_regex_capturing_group_int
            = \int_use:N \l_regex_capturing_group_int
          \bool_if:NTF \l_regex_look_behind_bool
            { \bool_set_true:N \l_regex_look_behind_bool }
            { \bool_set_false:N \l_regex_look_behind_bool }
          \prg_stepwise_function:nnnN
            {1} {1} {\l_regex_state_max_int}
            \regex_set_aux:n
          \scan_stop:
        }
    \group_end:
    #1 #2 \g_regex_tmpa_tl
  }
\cs_new_nopar:Npn \regex_set_aux:n #1
  { \tex_toks:D #1 { \tex_the:D \tex_toks:D #1 } }
\cs_new:Npn \regex_nfa:Nw #1 #2 \scan_stop:
  { \msg_expandable_error:n { Automaton~#1 used~incorrectly. } }
%    \end{macrocode}
% \end{macro}
% \end{macro}
% \end{macro}
% \end{macro}
% \end{macro}
%
% \begin{macro}[int,TF]{\regex_check_nfa:N}
%   If a token list variable starts with \cs{regex_nfa:Nw},
%   then it most likely holds the data for a precompiled pattern.
%    \begin{macrocode}
\prg_new_protected_conditional:Npnn \regex_check_nfa:N #1 { TF }
  { \exp_after:wN \regex_check_nfa_aux:Nw #1 \q_stop }
\cs_new:Npn \regex_check_nfa_aux:Nw #1 #2 \q_stop
  {
    \if_meaning:w \regex_nfa:Nw #1
      \prg_return_true:
    \else:
      \msg_error:nnx { regex } { not-nfa } { \token_to_str:N #1 }
      \prg_return_false:
    \fi:
  }
\msg_new:nnn { regex } { not-nfa }
  {
    I~was~expecting~a~regular~expression~variable.\\
    Instead,~I~got~#1.
  }
%    \end{macrocode}
% \end{macro}
%
% \begin{macro}[int]{\regex_use:N}
%   No error-checking.
%    \begin{macrocode}
\cs_new_protected_nopar:Npn \regex_use:N #1
  { \exp_after:wN \use_none:nn #1 }
%    \end{macrocode}
% \end{macro}
%
% \subsubsection{Generic auxiliary functions}
%
% \begin{macro}[aux]{\regex_user_aux:Nn}
%   This is an auxiliary used by most user functions.
%   The first and second arguments control whether we should track
%   submatches, and whether we track one or multiple submatches.
%   Everything is done within a group, so that |#2| can perform
%   \enquote{unsafe} assignments. Most user functions return a
%   result using \cs{group_insert_after:N}.
%    \begin{macrocode}
\cs_new_protected:Npn \regex_user_aux:Nn #1#2
  {
    \group_begin:
      \cs_set_eq:NN \regex_if_track_submatches:T #1
      \tl_clear:N \l_regex_every_match_tl
      #2
    \group_end:
  }
%    \end{macrocode}
% \end{macro}
%
% \begin{macro}[aux]{\regex_return_after_group:}
%   Most of \pkg{l3regex}'s work is done within a group.
%   This function triggers either \cs{prg_return_false:}
%   or \cs{prg_return_true:} as appropriate to whether a
%   match was found or not.
%    \begin{macrocode}
\cs_new_protected_nopar:Npn \regex_return_after_group:
  {
    \if_num:w \l_regex_success_thread_int = \c_zero
      \group_insert_after:N \prg_return_false:
    \else:
      \group_insert_after:N \prg_return_true:
    \fi:
  }
%    \end{macrocode}
% \end{macro}
%
% \begin{macro}[aux]{\regex_extract_after_group:N}
%   Extract submatches, and store them in the user-given variable
%   after the group has ended.
%    \begin{macrocode}
\cs_new_protected_nopar:Npn \regex_extract_after_group:N #1
  {
    \if_num:w \l_regex_success_thread_int > \c_zero
      \regex_extract:
      \group_insert_after:N \seq_set_eq:NN
      \group_insert_after:N #1
      \group_insert_after:N \g_regex_submatches_seq
    \fi:
  }
%    \end{macrocode}
% \end{macro}
%
% \begin{macro}[aux]{\regex_count_after_group:N}
%   Same procedure as \cs{regex_extract_after_group:N}, but simpler
%   since the match counting has already taken place.
%    \begin{macrocode}
\cs_new_protected_nopar:Npn \regex_count_after_group:N #1
  {
    \group_insert_after:N \int_set_eq:NN
    \group_insert_after:N #1
    \group_insert_after:N \g_regex_match_count_int
  }
%    \end{macrocode}
% \end{macro}
%
% \subsubsection{Matching}
%
% \begin{macro}[TF]{\regex_match:nn}
% \begin{macro}[TF]{\regex_match:Nn}
%   We don't track submatches. Then either build the NFA corresponding
%   to the regular expression, or use a precompiled pattern. Then match,
%   using the internal \cs{regex_match:n}. Finally return the result
%   after closing the group.
%    \begin{macrocode}
\prg_new_protected_conditional:Npnn \regex_match:nn #1#2 { T , F , TF }
  {
    \regex_user_aux:Nn \use_none:n
      {
        \regex_build:n {#1}
        \regex_match:n {#2}
        \regex_return_after_group:
      }
  }
\prg_new_protected_conditional:Npnn \regex_match:Nn #1#2 { T , F , TF }
  {
    \regex_check_nfa:NTF #1
      {
        \regex_user_aux:Nn \use_none:n
          {
            \regex_use:N #1
            \regex_match:n {#2}
            \regex_return_after_group:
          }
      }
      { \prg_return_false: }
  }
%    \end{macrocode}
% \end{macro}
% \end{macro}
%
% \begin{macro}{\regex_count:nnN}
% \begin{macro}{\regex_count:NnN}
%   Instead of aborting once the first \enquote{best match} is found,
%   we repeat the search. The code is such that the search will not
%   start on the same character, hence avoiding infinite loops.
%    \begin{macrocode}
\cs_new_protected:Npn \regex_count:nnN #1#2#3
  {
    \regex_user_aux:Nn \use_none:n
      {
        \int_gzero:N \g_regex_match_count_int
        \tl_set:Nn \l_regex_every_match_tl
          {
            \int_gincr:N \g_regex_match_count_int
            \regex_match_once:
          }
        \regex_build:n {#1}
        \regex_match:n {#2}
        \regex_count_after_group:N #3
      }
  }
\cs_new_protected:Npn \regex_count:NnN #1#2#3
  {
    \regex_check_nfa:NTF #1
      {
        \regex_user_aux:Nn \use_none:n
          {
            \int_gzero:N \g_regex_match_count_int
            \tl_set:Nn \l_regex_every_match_tl
              {
                \int_gincr:N \g_regex_match_count_int
                \regex_match_once:
              }
            \regex_use:N #1
            \regex_match:n {#2}
            \regex_count_after_group:N #3
          }
      }
      { }
  }
%    \end{macrocode}
% \end{macro}
% \end{macro}
%
% \subsubsection{Submatch extraction}
%
% \begin{macro}{\regex_extract:nnN}
% \begin{macro}{\regex_extract:NnN}
% \begin{macro}[TF]{\regex_extract:nnN}
% \begin{macro}[TF]{\regex_extract:NnN}
%    \begin{macrocode}
\cs_new_protected:Npn \regex_extract:nnN #1#2#3
  {
    \regex_user_aux:Nn \use:n
      {
        \regex_build:n {#1}
        \regex_match:n {#2}
        \regex_extract_after_group:N #3
      }
  }
\prg_new_protected_conditional:Npnn \regex_extract:nnN #1#2#3 { T , F , TF }
  {
    \regex_user_aux:Nn \use:n
      {
        \regex_build:n {#1}
        \regex_match:n {#2}
        \regex_extract_after_group:N #3
        \regex_return_after_group:
      }
  }
\cs_new_protected:Npn \regex_extract:NnN #1#2#3
  {
    \regex_check_nfa:NTF #1
      {
        \regex_user_aux:Nn \use:n
          {
            \regex_use:N #1
            \regex_match:n {#2}
            \regex_extract_after_group:N #3
          }
      }
      { }
  }
\prg_new_protected_conditional:Npnn \regex_extract:NnN #1#2#3 { T , F , TF }
  {
    \regex_check_nfa:NTF #1
      {
        \regex_user_aux:Nn \use:n
          {
            \regex_use:N #1
            \regex_match:n {#2}
            \regex_extract_after_group:N #3
            \regex_return_after_group:
          }
      }
      { \prg_return_false: }
  }
%    \end{macrocode}
% \end{macro}
% \end{macro}
% \end{macro}
% \end{macro}
%
% \subsubsection{Splitting a string by matches of a regex}
%
% \begin{macro}{\regex_split:nnN}
% \begin{macro}{\regex_split:NnN}
% \begin{macro}[aux]{\regex_split_aux:}
% \begin{macro}[aux]{\regex_split_after_group:N}
%   Similarly to \cs{regex_count:nnN} functions,
%   recurse through matches of the pattern. Then we do
%   something slightly different, extracting submatches.
%   Submatches are not extracted if the pattern matched
%   an empty string at the start of the match attempt
%   (to avoid adding spurious empty items to the resulting
%   sequence).
%    \begin{macrocode}
\cs_new_protected:Npn \regex_split:nnN #1#2#3
  {
    \regex_user_aux:Nn \use:n
      {
        \seq_gclear:N \g_regex_split_seq
        \tl_set:Nn \l_regex_every_match_tl { \regex_split_aux: }
        \regex_build:n {#1}
        \regex_match:n {#2}
        \regex_split_after_group:N #3
      }
  }
\cs_new_protected:Npn \regex_split:NnN #1#2#3
  {
    \regex_check_nfa:NTF #1
      {
        \regex_user_aux:Nn \use:n
          {
            \seq_gclear:N \g_regex_split_seq
            \tl_set:Nn \l_regex_every_match_tl { \regex_split_aux: }
            \regex_use:N #1
            \regex_match:n {#2}
            \regex_split_after_group:N #3
          }
      }
      { }
  }
\cs_new_protected_nopar:Npn \regex_split_aux:
  {
    \int_compare:nNnF \l_regex_start_step_int = \l_regex_success_step_int
      {
        \regex_extract:
        \seq_pop:NN \g_regex_submatches_seq \l_regex_tmpa_tl
        \regex_extract_aux:nTF {0}
          {
            \seq_gput_left:Nx \g_regex_submatches_seq
              {
                \str_from_to:Nnn \l_regex_query_str
                  { \int_use:N \l_regex_start_step_int }
                  { \l_regex_tmpa_tl }
              }
          }
          { \msg_error:nn { regex } { internal } }
        \seq_gconcat:NNN \g_regex_split_seq
          \g_regex_split_seq \g_regex_submatches_seq
      }
    \regex_match_once:
  }
\cs_new_protected_nopar:Npn \regex_split_after_group:N #1
  {
    \int_compare:nNnTF \l_regex_start_step_int = \l_regex_current_step_int
      {
        \bool_if:NF \l_regex_success_empty_bool
          { \seq_gput_right:Nn \g_regex_split_seq { } }
      }
      {
        \seq_gput_right:Nx \g_regex_split_seq
          {
            \str_from_to:Nnn \l_regex_query_str
              { \int_use:N \l_regex_start_step_int }
              { \int_use:N \l_regex_current_step_int }
          }
      }
    \group_insert_after:N \seq_set_eq:NN
    \group_insert_after:N #1
    \group_insert_after:N \g_regex_split_seq
  }
%    \end{macrocode}
% \end{macro}
% \end{macro}
% \end{macro}
% \end{macro}
%
% \subsubsection{String replacement}
%
% \begin{macro}[int]{\regex_replacement:n}
% \begin{macro}[aux]{\regex_replacement_loop:N}
% \begin{macro}[aux]{\regex_replacement_normal:N}
%    \begin{macrocode}
\cs_new_protected:Npn \regex_replacement:n #1
  {
    \str_aux_escape:NNNn
      \prg_do_nothing:
      \regex_replacement_escaped:N
      \regex_replacement_raw:N
      {#1}
    \tl_set_eq:NN \l_regex_replacement_tl \g_str_tmpa_tl
    \tl_set:Nx \l_regex_replacement_tl
      { \l_regex_replacement_tl \prg_do_nothing: }
  }
\cs_new_nopar:Npn \regex_replacement_raw:N { \exp_not:N \prg_do_nothing: }
\cs_new_nopar:Npn \regex_replacement_escaped:N #1
  {
    \if_num:w 9 < 1 #1 \exp_stop_f:
      \exp_not:N \regex_replacement_escaped_ii:nN {#1}
    \else:
      #1 %^^A todo: check space's catcode.
    \fi:
  }
\cs_new_nopar:Npn \regex_replacement_escaped_ii:nN #1#2
  {
    \regex_token_if_other_digit:NTF #2
      { \regex_replacement_escaped_ii:nN {#1#2} }
      { \exp_not:N \seq_item:Nn \exp_not:N \g_regex_submatches_seq {#1} #2 }
  }
\prg_new_conditional:Npnn \regex_token_if_other_digit:N #1 { TF }
  {
    \if_num:w 9 < 1 \exp_not:N #1 \exp_stop_f:
      \prg_return_true:
    \else:
      \prg_return_false:
    \fi:
  }
%    \end{macrocode}
% \end{macro}
% \end{macro}
% \end{macro}
%
% \begin{macro}[aux]{\regex_replace_after_group:N}
%    \begin{macrocode}
\cs_new_protected_nopar:Npn \regex_replace_after_group:N #1
  {
    \group_insert_after:N \tl_set_eq:NN
    \group_insert_after:N #1
    \group_insert_after:N \g_regex_replaced_str
  }
%    \end{macrocode}
% \end{macro}
%
% \begin{macro}{\regex_replace_once:nnN}
% \begin{macro}{\regex_replace_once:NnN}
% \begin{macro}[TF]{\regex_replace_once:nnN}
% \begin{macro}[TF]{\regex_replace_once:NnN}
% \begin{macro}[aux]{\regex_replace_once_aux:Nn}
%    \begin{macrocode}
\cs_new_protected:Npn \regex_replace_once_aux:Nn #1#2
  {
    \group_begin:
      \regex_replace_after_group:N #1
      \cs_set_eq:NN \regex_if_track_submatches:T \use:n
      \tl_clear:N \l_regex_every_match_tl
      #2
      \exp_args:No \regex_match:n {#1}
      \regex_extract:
      \regex_extract_aux:nTF {0}
        {
          \tl_gset:Nx \g_regex_replaced_str
            {
              \str_from_to:Nnn \l_regex_query_str {0} { \l_regex_tmpa_tl }
              \l_regex_replacement_tl
              \str_from_to:Nnn \l_regex_query_str
                { \l_regex_tmpb_tl } { \c_max_int }
            }
        }
        { \tl_gset_eq:NN \g_regex_replaced_str \l_regex_query_str }
    \group_end:
  }
\cs_new_protected:Npn \regex_replace_once:nnN #1#2#3
  {
    \regex_replace_once_aux:Nn #3
      {
        \regex_build:n {#1}
        \regex_replacement:n {#2}
      }
  }
\cs_new_protected:Npn \regex_replace_once:NnN #1#2#3
  {
    \regex_replace_once_aux:Nn #3
      {
        \regex_use:N #1
        \regex_replacement:n {#2}
      }
  }
\prg_new_protected_conditional:Npnn \regex_replace_once:nnN #1#2#3 {T,F,TF}
  {
    \regex_replace_once_aux:Nn #3
      {
        \regex_build:n {#1}
        \regex_replacement:n {#2}
        \regex_return_after_group:
      }
  }
\prg_new_protected_conditional:Npnn \regex_replace_once:NnN #1#2#3 {T,F,TF}
  {
    \regex_replace_once_aux:Nn #3
      {
        \regex_use:N #1
        \regex_replacement:n {#2}
        \regex_return_after_group:
      }
  }
%    \end{macrocode}
% \end{macro}
% \end{macro}
% \end{macro}
% \end{macro}
% \end{macro}
%
% \begin{macro}{\regex_replace_all:nnN}
% \begin{macro}{\regex_replace_all:NnN}
% \begin{macro}[aux]{\regex_replace_all_aux:Nn}
% \begin{macro}[aux]{\regex_replace_all_aux:}
%    \begin{macrocode}
\cs_new_protected:Npn \regex_replace_all_aux:Nn #1#2
  {
    \group_begin:
      \regex_replace_after_group:N #1
      \cs_set_eq:NN \regex_if_track_submatches:T \use:n
      \tl_set:Nn \l_regex_every_match_tl { \regex_replace_all_aux: }
      \tl_gclear:N \g_regex_replaced_str
      #2
      \exp_args:No \regex_match:n {#1}
      \tl_gput_right:Nx \g_regex_replaced_str
        {
          \str_from_to:Nnn \l_regex_query_str
            { \int_use:N \l_regex_start_step_int }
            { \c_max_int }
        }
    \group_end:
  }
\cs_new_protected_nopar:Npn \regex_replace_all_aux:
  {
    \regex_extract:
    \regex_extract_aux:nTF {0}
      {
        \tl_gput_right:Nx \g_regex_replaced_str
          {
            \str_from_to:Nnn \l_regex_query_str
              { \int_use:N \l_regex_start_step_int }
              { \l_regex_tmpa_tl }
            \l_regex_replacement_tl
          }
      }
      { \msg_error:nn { regex } { internal } }
    \regex_match_once:
  }
\cs_new_protected:Npn \regex_replace_all:nnN #1#2#3
  {
    \regex_replace_all_aux:Nn #3
      {
        \regex_build:n {#1}
        \regex_replacement:n {#2}
      }
  }
\cs_new_protected:Npn \regex_replace_all:NnN #1#2#3
  {
    \regex_replace_all_aux:Nn #3
      {
        \regex_use:N #1
        \regex_replacement:n {#2}
      }
  }
%    \end{macrocode}
% \end{macro}
% \end{macro}
% \end{macro}
% \end{macro}
%
%    \begin{macrocode}
%</package>
%    \end{macrocode}
%
% \section{Code specific to tracing}
%
%    \begin{macrocode}
%<*trace>
%    \end{macrocode}
%
% \begin{variable}{\l_regex_trace_tl}
% \begin{variable}{\l_regex_trace_line_start_tl}
%    \begin{macrocode}
\tl_new:N \l_regex_trace_tl
\tl_new:N \l_regex_trace_line_start_tl
%    \end{macrocode}
% \end{variable}
% \end{variable}
%
% \begin{macro}{\regex_trace:x}
%   Trace, adding \enquote{(regex)} to each new line,
%   and setting up the control space to produce a space.
%    \begin{macrocode}
\cs_new_protected:Npn \regex_trace:x #1
  {
    \group_begin:
      \cs_set_eq:NN \ \c_space_tl
      \cs_set:Npn \thread { \int_use:N \l_regex_current_thread_int }
      \cs_set:Npn \state  { \int_use:N \l_regex_current_state_int }
      \cs_set:Npn \charstep { \int_use:N \l_regex_current_step_int }
      \tl_set:Nx \l_regex_trace_line_start_tl
        {
          (regex) ~
%          \prg_replicate:nn { \l_regex_trace_nesting_int } { \ \  }
        }
      \cs_set_nopar:Npx \\ { \iow_newline: \l_regex_trace_line_start_tl }
      \iow_term:x { \l_regex_trace_line_start_tl #1}
    \group_end:
  }
\cs_new_protected:Npn \regex_trace_wrap:nnx #1 #2 #3
  { \iow_wrap:xnnnN { #3 } { (regex)~ #1 } { 8 + #2 } { } \regex_trace:x }
%    \end{macrocode}
% \end{macro}
%
% \subsection{Trace-related functions}
%
% \begin{macro}[aux]{\regex_trace_nfa:}
%    \begin{macrocode}
\cs_new_nopar:Npn \regex_trace_nfa:
  {
    \regex_trace:x
      {
        NFA~from~\l_regex_pattern_str\ has~
        \int_use:N \l_regex_state_max_int \ states:
      }
    \prg_stepwise_inline:nnnn {1} {1} {\l_regex_state_max_int}
      {
        \regex_trace_wrap:nnx { \ \ \ \ } {4}
          {
            \iow_char:N \{ ##1 \iow_char:N \}
            \ = \ \iow_char:N \{ \tex_the:D \tex_toks:D ##1 \iow_char:N \}
          }
      }
  }
%    \end{macrocode}
% \end{macro}
%
% \begin{macro}[aux]{\regex_trace_ist:n}
%    \begin{macrocode}
\cs_new_protected_nopar:Npn \regex_trace_ist:n #1
  { \regex_trace:x { #1,~s=\state,~t=\thread. } }
%    \end{macrocode}
% \end{macro}
%
%    \begin{macrocode}
%</trace>
%    \end{macrocode}
%
% \end{implementation}
%
% \endinput

%^^A NOT IMPLEMENTED
%^^A    \cx        "control-x", where x is any ASCII character
%^^A    \C         one byte, even in UTF-8 mode (best avoided)
%^^A    \p{xx}     a character with the xx property
%^^A    \P{xx}     a character without the xx property
%^^A    \R         a newline sequence
%^^A    \X         an extended Unicode sequence
%^^A    [[:xxx:]]   positive POSIX named set
%^^A    [[:^xxx:]]  negative POSIX named set
%^^A    ?+          0 or 1, possessive
%^^A    *+          0 or more, possessive
%^^A    ++          1 or more, possessive
%^^A    {n,m}+      at least n, no more than m, possessive
%^^A    {n,}+       n or more, possessive
%^^A    \K          reset start of match
%^^A    (?<name>...)    named capturing group (Perl)
%^^A    (?'name'...)    named capturing group (Perl)
%^^A    (?P<name>...)   named capturing group (Python)
%^^A    (?:...)         non-capturing group
%^^A    (?|...)         non-capturing group; reset group numbers for
%^^A                     capturing groups in each alternative
%^^A    (?>...)         atomic, non-capturing group
%^^A    (?#....)        comment (not nestable)
%^^A    (?i)            caseless
%^^A    (?J)            allow duplicate names
%^^A    (?m)            multiline
%^^A    (?s)            single line (dotall)
%^^A    (?U)            default ungreedy (lazy)
%^^A    (?x)            extended (ignore white space)
%^^A    (?-...)         unset option(s)
%^^A    (*NO_START_OPT) no start-match optimization (PCRE_NO_START_OPTIMIZE)
%^^A    (*UTF8)         set UTF-8 mode (PCRE_UTF8)
%^^A    (*UCP)          set PCRE_UCP (use Unicode properties for \d etc)
%^^A    (?=...)         positive look ahead
%^^A    (?!...)         negative look ahead
%^^A    (?<=...)        positive look behind
%^^A    (?<!...)        negative look behind
%^^A    \n              reference by number (can be ambiguous)
%^^A    \gn             reference by number
%^^A    \g{n}           reference by number
%^^A    \g{-n}          relative reference by number
%^^A    \k<name>        reference by name (Perl)
%^^A    \k'name'        reference by name (Perl)
%^^A    \g{name}        reference by name (Perl)
%^^A    \k{name}        reference by name (.NET)
%^^A    (?P=name)       reference by name (Python)
%^^A    (?R)            recurse whole pattern
%^^A    (?n)            call subpattern by absolute number
%^^A    (?+n)           call subpattern by relative number
%^^A    (?-n)           call subpattern by relative number
%^^A    (?&name)        call subpattern by name (Perl)
%^^A    (?P>name)       call subpattern by name (Python)
%^^A    \g<name>        call subpattern by name (Oniguruma)
%^^A    \g'name'        call subpattern by name (Oniguruma)
%^^A    \g<n>           call subpattern by absolute number (Oniguruma)
%^^A    \g'n'           call subpattern by absolute number (Oniguruma)
%^^A    \g<+n>          call subpattern by relative number (PCRE extension)
%^^A    \g'+n'          call subpattern by relative number (PCRE extension)
%^^A    \g<-n>          call subpattern by relative number (PCRE extension)
%^^A    \g'-n'          call subpattern by relative number (PCRE extension)
%^^A    (?(n)...        absolute reference condition
%^^A    (?(+n)...       relative reference condition
%^^A    (?(-n)...       relative reference condition
%^^A    (?(<name>)...   named reference condition (Perl)
%^^A    (?('name')...   named reference condition (Perl)
%^^A    (?(name)...     named reference condition (PCRE)
%^^A    (?(R)...        overall recursion condition
%^^A    (?(Rn)...       specific group recursion condition
%^^A    (?(R&name)...   specific recursion condition
%^^A    (?(DEFINE)...   define subpattern for reference
%^^A    (?(assert)...   assertion condition
%^^A    (*ACCEPT)       force successful match
%^^A    (*FAIL)         force backtrack; synonym (*F)
%^^A    (*COMMIT)       overall failure, no advance of starting point
%^^A    (*PRUNE)        advance to next starting character
%^^A    (*SKIP)         advance start to current matching position
%^^A    (*THEN)         local failure, backtrack to next alternation
%^^A    (*CR)           carriage return only
%^^A    (*LF)           linefeed only
%^^A    (*CRLF)         carriage return followed by linefeed
%^^A    (*ANYCRLF)      all three of the above
%^^A    (*ANY)          any Unicode newline sequence
%^^A    (*BSR_ANYCRLF)  CR, LF, or CRLF
%^^A    (*BSR_UNICODE)  any Unicode newline sequence
%^^A    (?C)      callout
%^^A    (?Cn)     callout with data n


%^^A % [todo: put that explanation somewhere]
%^^A % In fact, at each step in the string, we only need to keep
%^^A % one thread per state in the NFA: since we do not support
%^^A % back-references, the previous execution will not influence
%^^A % the match, and we can keep the thread with highest priority.

%^^A \begin{function}{\regex_extract_map_variable:nnNn}
%^^A \begin{function}{\regex_extract_map_variable:NnNn}