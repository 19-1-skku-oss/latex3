% \iffalse
%% File l3ldb.dtx (C) Copyright 2011 Frank Mittelbach, LaTeX3 Project
%%
%% It may be distributed and/or modified under the conditions of the
%% LaTeX Project Public License (LPPL), either version 1.3c of this
%% license or (at your option) any later version.  The latest version
%% of this license is in the file
%%
%%    http://www.latex-project.org/lppl.txt
%%
%% This file is part of the ``l3ldb bundle'' (The Work in LPPL)
%% and all files in that bundle must be distributed together.
%%
%% The released version of this bundle is available from CTAN.
%%
%% -----------------------------------------------------------------------
%%
%% The development version of the bundle can be found at
%%
%%    http://www.latex-project.org/svnroot/experimental/trunk/
%%
%% for those people who are interested.
%%
%%%%%%%%%%%
%% NOTE: %% 
%%%%%%%%%%%
%%
%%   Snapshots taken from the repository represent work in progress and may
%%   not work or may contain conflicting material!  We therefore ask
%%   people _not_ to put them into distributions, archives, etc. without
%%   prior consultation with the LaTeX Project Team.
%%
%% -----------------------------------------------------------------------
%%
%<*driver|package|signature>
\RequirePackage{expl3}
%</driver|package|signature>
%\fi
\GetIdInfo$Id: l3ldb.dtx 2601 2011-08-16 23:20:27Z bruno $
          {l3ldb}
%\iffalse
%<*signature>
\ProvidesFile{\jobname.tex}
  [\ExplFileDate\space v\ExplFileVersion\space\ExplFileDescription]
%</signature>
%<*driver>
%\fi
\ProvidesFile{\ExplFileName.\ExplFileExtension}
  [\ExplFileDate\space v\ExplFileVersion\space\ExplFileDescription]
%\iffalse
\documentclass{l3doc}
\usepackage{textcomp}
% \usepackage{xparse,xlists,xlists-samples}
% \usepackage{ldcdoc}

% next three definitions are big hacks to run the file
% without the above packages
%
\renewcommand\NoValue{\texttt{\textbackslash NoValue}}
\begin{document}
\DocInput{l3ldb.dtx}
\end{document}
%</driver>
% \fi
%



% 
% \section{The \LaTeX3 Resource Database}
% 
%
% The \LaTeX3 Resource Database (or Ldb for short) is a system that
% provide parameter settings depending on the ``context'' where this
% context is defined by the current nesting and sequencing of structural elements
% in the document.
%
% Rougly speaking certain (but not all) document elements (typically
% environments) are declared to belong to one or more element
% classes. Whenever such an element is encountered in the document the
% LDB is consulted to provide parameter updates due to the changed
% context.
%
% In case of environments both the start and the end of the
% environment result in a context change so that sequences of begin
% or end tags  can be specially handled.
%
% \subsection{Context}
%
% When setting up the database the following conventions are used
% \begin{description}
% \item[\texttt{<foo}] denotes the begin tag of class \texttt{foo}.
% \item[\texttt{foo>}] denotes the end tag of class \texttt{foo}.
% \item[\texttt{<foo>}] denotes a the occurrence  of a whole
%    environment for use in sequencing specs. 
% \item[\texttt{<-}]  is a begin elision tag spec and denotes a 
%    possibly empty sequence
%   of begin tags (not implemented).
% \item[\texttt{->}]  is an end elision tag spec  (also not implemented).
% \end{description}
% 
% Tags are related through each other through nesting or
% sequencing. This is denoted by a ``\texttt{.}'' (dot) for direct
% nesting, ``\texttt{*} (star) for nesting at arbitrary depth, and the
% absence of a binding denotes sequencing (i.e. when two tags occur
% with nothing in between).
%
% In BNF notation this would be represented as
%\begin{verbatim}
%       context	== spec [[binding] spec]*
%       binding	== "." | "*"
%       spec	== ["<"] id [">"] | <- | ->
%\end{verbatim}
%
% There are additional restriction on the form of admissible context
% specifications. The context matching approach requires that we keep
% a stack of tags to represent the current situation so that we can
% match it with entries in the database.  The desire to keep the size
% of this stack under control leads us to formulate a number of
% restrictions wich will allow us to trim the stack as much and as
% soon as possible.  Also there are semantic difficulties which also
% suggest certain restrictions.
% \begin{enumerate}
% \item The scope of \texttt{<foo} extends to the end of the context
%   specification.
% \item Specs to the right of \texttt{foo>} can only involve sequencing.
% \item  \texttt{<foo>} occurs right-most, or must be followed by `<baz' or
%   \texttt{<baz>}.
% \item  \texttt{<foo>} cannot occur to the right of \texttt{baz>}.
% \end{enumerate}
%
% With these restrictions, we arrive at the following formulation for
% admissible context specifications.  A context specification has 3
% parts:
% \begin{enumerate} 
% \item The 1st part consists solely of begin tags \texttt{<foo},
%    begin elision tags
% \texttt{<-}, and completed tags \texttt{<foo>} interspersed here and
%    there with tight and loose bindings.
% 
% \item The 2nd part consists only of a sequence of end tags
%    \texttt{foo>} and end elision tags \texttt{->}.  If both the 1st
%    and the 2nd parts are non-empty,
% then they must be separated by a loose binding.
% 
% \item The 3rd part consists only of a sequence of begin tags
%    \texttt{<foo} and begin elision tags \texttt{<-}.
% \end{enumerate}
% 
% Here is an example in which the 3 parts have been deliberately
% separated for ease of recognition:
%\begin{verbatim}
% 	<A1.<A2><A3><-<A4*<A5   *   B1>B2>->B3>   <C1<-<C2
%\end{verbatim}
% The reason we need a loose binding between the 1st and 2nd part is
% that a context like \texttt{<A.B>A>C>} could never match anything
% since the scope of \texttt{<A} extends all the way to the right, yet
% the scope of the environment \texttt{A} doesn't extend past
% \texttt{A>} (which in this situation must obviously correspond to
% \texttt{<A}). \texttt{<A*B>A>C>} may match certain situations where
% we have another environment \texttt{A} indirectly nested in the
% first one.
% 
% The reason end tags can only appear as a sequence in the 2nd part is
% that, as soon as possible, we want to collapse the sequence
% \texttt{B1>B2>B3>} into the completed env \texttt{<B3>}.  When we do
% this, we just throw away \texttt{B1>} and \texttt{B2>} and forget
% everything about them.  As soon as possible is when we cannot extend
% the sequence of tags (e.g., because the new tag wasn't in sequence,
% or because we are in the situation where we found a sequence of end
% tags followed by a sequence of begin tags and the new tag is an end
% tag again).
%
%
%
% \subsection{The issue of specificity}
% 
% Whenever there are several LDB entries matching the current situation
% and proposing different assignments to the same parameter, we must
% choose the most specific entry.  Here are some rules that govern the
% issue of specificity:
% \begin{enumerate}
% \item  Any spec is more specific than another one to its left.  In other
%     words, specificity increases from left to right.  In terms of
%     nesting: specificity decreases from the inside out.
% 
% \item  Sequencing is more specific than nesting.
% 
% \item  \texttt{<foo} is more specific than \texttt{<-}.
% 
% \item \texttt{foo>} is more specific than \texttt{->} which is more
%    specific than \texttt{<foo>}.
% 
% \item  \texttt{.foo} is more specific than \texttt{*foo}.
% \end{enumerate}
%
% \subsection{Tag classes}
%
% The identifier \texttt{foo} in a spec like \texttt{<foo} names a
% class rather than an environment called \texttt{foo}.  Each
% environment is associated with a list of classes.  The order is
% meaningful: it goes from most to least specific.  Typically, the
% most specific class has the same name as the environment.
% 
% The purpose of classes is to allow a form of inheritance.  The idea
% is that an environment like \texttt{enumerate} is also an instance
% of a \texttt{list}.  Thus \texttt{enumerate} should be able to
% inherit the LDB entries for \texttt{list} but also be allowed to
% have additional and/or overriding entries.  The list of classes for
% \texttt{enumerate} probably ought to be the list of classes for
% \texttt{list} augmented at the front with the class
% \texttt{enumerate}.
% 
%
%
% \StopEventually{}
% 
% 
% 
%
% \section{The LDB (FMi version)}
%
% \textbf{WARNING:} This is a more or less straight adaption of code
% written about 20 years ago. Yes there was some version of expl3 back
% then but it looked sufficiently different so that it was hard going
% to get this working without rewriting it from scratch.
% Consequently a lot of the documentation is wrong or strange as back
% then we experimented with several versions and traces of this can be
% found in the code still, e.g., in form of references to commands and
% variables that no longer exist in the code.
%
% \textbf{All of that needs to be rewritten!} These days
% \texttt{expl3} has much better data and control structures, so that
% a lot of the manual code could be streamlined (and some of the speed
% optimizations that make the code very difficult to read and
% understand aren't that important which much faster processors.
% \textbf{:WARNING END}
%
%
% \subsection{Package start}
%
% Set up certain defaults including to ignore white space
% within the body of this package.
%    \begin{macrocode}
%<*package>
\RequirePackage{expl3}
\ProvidesExplPackage
  {\ExplFileName}{\ExplFileDate}{\ExplFileVersion}{\ExplFileDescription}
%    \end{macrocode}
%


% \subsection{old code tmp converted}
%
%    \begin{macrocode}
\cs_new:Npn\file_show_lineno: {on~line~ \msg_line_number:}
%    \end{macrocode}
%

%    \begin{macrocode}
\cs_new:Npn \tl_new:Nn #1 {\tl_new:N {#1} \tl_set:Nn #1}
%    \end{macrocode}
%
%    This one is extremely inefficient (2000 lines of trace and we
%    should have closed the stream in addition ...
%    \begin{macrocode}
\cs_new:Npn \file_not_found:nTF #1{
  \ior_open:Nn \c_ior_test_stream {#1}
  \ior_if_eof:NTF \c_ior_test_stream
}
%    \end{macrocode}
%

% \subsection{Precompiling and debugging}
%
%    \begin{macrocode}
% \iffalse meta-comment
%
%% File: l3precom.dtx Copyright (C) 2012,2013,2015-2017 The LaTeX3 Project
%%
%% It may be distributed and/or modified under the conditions of the
%% LaTeX Project Public License (LPPL), either version 1.3c of this
%% license or (at your option) any later version.  The latest version
%% of this license is in the file
%%
%%    http://www.latex-project.org/lppl.txt
%%
%% This file is part of the "l3trial bundle" (The Work in LPPL)
%% and all files in that bundle must be distributed together.
%%
%% The released version of this bundle is available from CTAN.
%%
%% -----------------------------------------------------------------------
%%
%% The development version of the bundle can be found at
%%
%%    http://www.latex-project.org/svnroot/experimental/trunk/
%%
%% for those people who are interested.
%%
%%%%%%%%%%%
%% NOTE: %%
%%%%%%%%%%%
%%
%%   Snapshots taken from the repository represent work in progress and may
%%   not work or may contain conflicting material!  We therefore ask
%%   people _not_ to put them into distributions, archives, etc. without
%%   prior consultation with the LaTeX3 Project.
%%
%% -----------------------------------------------------------------------
%
%<*driver|package>
\RequirePackage{expl3}
%</driver|package>
%<*driver>
\documentclass[full]{l3doc}
\usepackage{amsmath}
\begin{document}
  \DocInput{\jobname.dtx}
\end{document}
%</driver>
% \fi
%
% \title{^^A
%   The \textsf{l3precom} package: precompiling functions^^A
% }
%
% \author{^^A
%  The \LaTeX3 Project\thanks
%    {^^A
%      E-mail:
%        \href{mailto:latex-team@latex-project.org}
%          {latex-team@latex-project.org}^^A
%    }^^A
% }
%
% \date{Released 2017/12/05}
%
% \maketitle
%
% \begin{documentation}
%
% \section{\pkg{l3precom} documentation}
%
% It might speed up the processing of documents when certain parts of
% the document style file are `precompiled' and stored in a separate
% file. How much of this is still useful these days is open to some
% experiments I guess. If a large number of LDB settings are being used
% then perhaps this will indeed safe some considerable time as all the
% parsing isn't necessary (some initial test show 1 second extra user
% time per 1000 additional LDB entries).
%
% I've largely added it to have \cs{dumpLaTeXstate} available while
% debugging the LDB pointer structure.
%
% \end{documentation}
%
% \begin{implementation}
%
% \section{\pkg{l3precom} implementation}
%
%    \begin{macrocode}
%<*package>
\RequirePackage{expl3}
\ProvidesExplPackage{l3precom}{2017/12/05}{}
  {L3 Experimental precompilation of functions}
%    \end{macrocode}
%
%    \begin{macrocode}
\tex_message:D { LaTeX:~ precompiling~ code, }
%    \end{macrocode}
%
%    \begin{macrocode}
%<*precompile>
%    \end{macrocode}
%
% \begin{variable}{\c_cs_dump_iow}
%   We need to allocate an output stream for the dump in order to be
%   able to write the precompiled code out.
%    \begin{macrocode}
\iow_new:N \c_cs_dump_iow
%    \end{macrocode}
% \end{variable}
%
% \begin{variable}{\g_cs_dump_name_tl}
%   This \meta{tl} is used to store the name of the file.
%    \begin{macrocode}
\tl_new:N  \g_cs_dump_name_tl
\tl_gset:Nn \g_cs_dump_name_tl { \c_sys_jobname_str -default.dmp }
%    \end{macrocode}
% \end{variable}
%
% \begin{variable}{\g_cs_dump_seq}
%   While processing the document class we build up a list of
%   control sequence names to be dumped. For this purpose
%   we use the |\g_cs_dump_seq| sequence.
%    \begin{macrocode}
\seq_new:N \g_cs_dump_seq
%    \end{macrocode}
% \end{variable}
%
% \begin{macro}{\cs_record_name:N, \cs_record_name:c}
%   These functions mark a control sequence for dumping into a
%   precompiled class.  When the \textsf{trace} `module' is included in
%   the code we also write information about the control sequence into a
%   |.dmp| file.
%    \begin{macrocode}
\cs_new_protected:Npn \cs_record_name:N #1
  {
    \seq_gput_left:Nn \g_cs_dump_seq {#1}
%<trace>    \seq_gput_left:Nn \g_cs_trace_seq {#1}
  }
\cs_generate_variant:Nn \cs_record_name:N { c }
%    \end{macrocode}
% \end{macro}
%
% \begin{macro}{\cs_dump:}
%   When a document styleclass calls \cs{cs_dump:} it triggers this code
%   to write all the precompilation information out to a file.
%
%   Before dumping, we write a message to the terminal informing the
%   \enquote{user} of this fact.
%    \begin{macrocode}
\cs_new_protected:Npn \cs_dump:
  {
    \iow_term:x { Precompiling~class~into~(\g_cs_dump_name_tl) }
    \iow_open:Nn \c_cs_dump_iow { \g_cs_dump_name_tl }
%    \end{macrocode}
%   The first thing we write on a `dump' file is a command that allows
%   us to use |*| in control sequences. We also need to be able to write
%   to (and read from) the file internal control sequences, containing
%   |_| and |:|.
%    \begin{macrocode}
    \iow_now:Nn \c_cs_dump_iow
      {
        \group_begin:
          \char_set_catcode_letter:N \*
          \ExplSyntaxOn
      }
%    \end{macrocode}
%   We use a direct |\cs_gset:Npn| to disable any type of local/global
%   check on the pointers.
%
%   Such dumping should probably ensure that the line length is not going
%   to be too long or else it may not be possible to read such a file
%   back in (not done).
%
%   One other thing: dumping of the LDB doesn't work of course because
%   dumping of property lists doesn't work (in all cases) as the keys
%   these days are specially handled and become strings.
%    \begin{macrocode}
    \seq_map_inline:Nn \g_cs_dump_seq
      {
        \tex_message:D {.}
        \iow_now:Nx \c_cs_dump_iow
          {
            \token_to_str:N \cs_gset:Npn
            \token_to_str:N ##1
              { \tl_to_str:N ##1 }
          }
      }
    \iow_now:Nn \c_cs_dump_iow { \group_end: }
    \iow_close:N \c_cs_dump_iow
    \tex_message:D { ~ finished }
  }
%</precompile>
%    \end{macrocode}
% \end{macro}
%
% \begin{macro}{\cs_load_dump:n}
%   A function to read a precompiled file into memory and skip until a
%   \cs{cs_dump:} command is found. If no such file is found, processing
%   continues and a subsequent \cs{cs_dump:} command will then create
%   the dump file.
%    \begin{macrocode}
\cs_new_protected:Npn \cs_load_dump:n #1
  {
    \file_if_exist:nTF { #1.cmp }
      {
        \iow_term:x { Loading~ precompiled~ format~ in~ #1.cmp ~... }
        \file_input:n { #1.cmp }
        \cs_set_eq:NN \cs_dump: \fi:
        \if_false:
      }
%<precompile>      { \tl_gset:Nn \g_cs_dump_name_tl { #1.cmp } }
%<-precompile>      { \tex_errmessage:D { Cannot~ dump~ with~ this~ format } }
  }
%    \end{macrocode}
% \end{macro}
%
% \begin{variable}{\g_gen_sym_int, \g_ggen_sym_int}
%    Two counters to make up new local or global \textsl{short} names
%    in pointer structures like the \textsc{ldb}.
%    \begin{macrocode}
\int_new:N   \g_gen_sym_int
\int_new:N   \g_ggen_sym_int
%    \end{macrocode}
% \end{variable}
%
% \begin{macro}{\cs_gen_sym:N, \cs_ggen_sym:N}
%   We need to be able to generate control sequences on the fly.  They
%   will exist of a prefix, either |l*| or |g*|, followed by the value
%   of the counter \cs{g_gen_sym_int} (\cs{g_ggen_sym_int}) in roman
%   numeral representation. The generated control sequence is locally
%   stored in the token that was passed in |#1|.
%
%    \begin{macrocode}
\cs_new_protected:Npn \cs_gen_sym:N #1
  {
    \int_gincr:N \g_gen_sym_int
    \tl_set:Nx #1 { \exp_not:c { l*\__int_to_roman:w \g_gen_sym_int } }
%<precompile>    \exp_after:wN \cs_record_name:N #1
%    \end{macrocode}
%   We still want to define the initial value for the new symbol
%   globally to make sure that during compilation something is written
%   to the output file.
%    \begin{macrocode}
    \exp_after:wN \tl_clear_new:N #1
  }
%    \end{macrocode}
%   The global variant %^^A todo: why so different?
%    \begin{macrocode}
\cs_new_protected:Npn \cs_ggen_sym:N #1
  {
%<trace>    \trace_ldb_internal:x { \exp_not:n { \cs_ggen_sym:N { #1 } } }
    \int_gincr:N \g_ggen_sym_int
    \tl_set:Nx #1 { \exp_not:c { g*\__int_to_roman:w \g_ggen_sym_int } }
%<trace>    \trace_ldb_internal:x { \token_to_meaning:N #1 ~ ... ~ allocated }
%<precompile>    \exp_after:wN \cs_record_name:N #1
    \exp_after:wN \tl_clear_new:N #1
  }
%    \end{macrocode}
% \end{macro}
%
%
% \begin{macro}{\cs_ggen_prop:N}
%   We really need to allocate property lists these days for nodes of
%   the LDB.
%    \begin{macrocode}
\cs_new_protected:Npn \cs_ggen_prop:N #1
  {
    \cs_ggen_sym:N #1
    \exp_after:wN \prop_gclear:N #1
  }
%    \end{macrocode}
% \end{macro}
%
%
%
% \subsection{Dumping data structure states for tracing}
%
% \begin{variable}{\g_cs_trace_seq}
%   A sequence which holds the control sequence names that are to be
%   dumped. They are stored together with their meaning.
%    \begin{macrocode}
%<*trace>
\seq_new:N \g_cs_trace_seq
%    \end{macrocode}
% \end{variable}
%
% \begin{variable}{\g_register_trace_seq}
%   Sequence holding the register names to be dumped with their
%   corresponding values.
%    \begin{macrocode}
\seq_new:N \g_register_trace_seq
%    \end{macrocode}
% \end{variable}
%
% \begin{macro}{\cs_record_meaning:N}
%   Function marking a control sequence for dumping with meaning.
%    \begin{macrocode}
\cs_new_protected:Npn \cs_record_meaning:N #1
  { \seq_gput_left:Nn \g_cs_trace_seq {#1} }
%    \end{macrocode}
% \end{macro}
%
% \begin{macro}{\register_record_name:N}
%   Function marking a register for dumping with value.
%    \begin{macrocode}
\cs_new_protected:Npn \register_record_name:N #1
  { \seq_gput_left:Nn \g_register_trace_seq {#1} }
%    \end{macrocode}
% \end{macro}
%
%
% \begin{macro}{\dumpLaTeXstate}
%   The function \cs{dumpLaTeXstate} is used to write control sequences
%   and registers, together with their meaning or value in the |.dmp|
%   file. We write informational messages to the terminal during the
%   dump.
%    \begin{macrocode}
\cs_new_protected:Npn \dumpLaTeXstate #1
  {
    \iow_term:x { Dumping~commands~into~(#1.dmp) }
    \iow_open:Nn \c_cs_dump_iow { #1.dmp }
    \tex_message:D { ~ commands }
    \seq_map_inline:Nn \g_cs_trace_seq
      {
        \tex_message:D {.}
        \iow_now:Nx \c_cs_dump_iow
          {
            \token_to_str:N ##1~
            \token_to_meaning:N ##1
          }
      }
    \tex_message:D { ~ registers }
    \seq_map_inline:Nn \g_register_trace_seq
      {
        \tex_message:D {.}
        \iow_now:Nx \c_cs_dump_iow
          {
            \token_to_str:N ##1
            \tex_the:D ##1
          }
      }
    \tex_message:D { ~ finished }
    \iow_close:N \c_cs_dump_iow
  }
%</trace>
%    \end{macrocode}
% \end{macro}
%
%    \begin{macrocode}
%</package>
%    \end{macrocode}
%
% \end{implementation}
%
% \endinput

%    \end{macrocode}
%

% \subsection {Parsing A Context}
%
%  This is now done in l3ldbparse.
%
%    \begin{macrocode}
%------------------------------------------------------------------------
% \subsection {Parsing LDB Context Specifications}
%
% A context specification consists of 3 segments.  The 1st segment
% contains begin tags and completed tags only, respectively of the
% form |<foo| and |<foo>|.  Two consecutive tags can be optionally
% separated by a tight or loose binding (i.e.\ a dot or a star),
% except that a completed tag may never be followed by a binding.  The
% second segment consists of end tags (i.e.\ of the form |foo>|)
% occurring in sequence; no bindings are allowed.  If both the 1st and
% 2nd segments are non-empty, then there must be a loose binding
% between them.  The last segment consists of begin tags occurring in
% sequence.  There must no binding between the 2nd and 3rd segments.
%
% A context specification is parsed as follows: first, abbreviations
% such as |>...<| are expanded, and various special cases are
% normalized --- for example, |*<-| is turned into the simpler, but
% semantically equivalent, |*|.  Then, the context specification is
% examined and broken into basic constituents, and turned into a list
% of triples of the form (bdg,type,tag), where tag is the identifier
% naming the tag, type is 0 (begin), 1 (end), or 2 (completed),
% and bdg indicates whether this tag was preceded by a tight (1) or a
% loose binding (2), or by no binding at all (0) (i.e.\ it occurred in
% sequence).
%
% For example, |<A.<B<C><D*E>| is turned into something like:
% \begin{center}
%   |(0,0,A)(1,0,B)(0,2,C)(0,0,D)(2,1,E)|
% \end{center}
% except that we use the integer constants |\c_zero|, |\c_one| and
% |\c_two| instead.
%
% \begin{macro}{\glet_gtmp:N}
% \begin{macro}{\gdef_gtmp:n}
% \begin{macro}{\gdef_gtmp:x}
%    Very convenient abbreviations for defining a continuation in
%    |\gtmp:w| for immediate use.
%    \begin{macrocode}
\cs_new:Npn\glet_gtmp:N{\cs_gset_eq:NN \gtmp:w}
\cs_new:Npn\gdef_gtmp:n{\cs_gset:Npn   \gtmp:w}
\cs_new:Npn\gdef_gtmp:x{\cs_gset:Npx   \gtmp:w}
%    \end{macrocode}
% \end{macro}
% \end{macro}
% \end{macro}
%
% \begin{macro}{\g_ldb_spec_tl}
% \begin{macro}{\g_ldb_triples_tl}
% \begin{macro}{\g_ldb_tag_tl}
%    We save the original context specification in |\g_ldb_spec_tl|
%    and use it when we print an error message.  After parsing this
%    context specification, the corresponding sequence of triples is
%    available in |\g_ldb_triples_tl|.  While parsing, whenever we
%    encounter a tag's identifier, we save it temporarily in
%    |\g_ldb_tag_tl|.
%    \begin{macrocode}
\tl_new:Nn\g_ldb_spec_tl   {}
\tl_new:Nn\g_ldb_triples_tl{}
\tl_new:Nn\g_ldb_tag_tl    {}
%    \end{macrocode}
% \end{macro}
% \end{macro}
% \end{macro}
%
% \begin{macro}{\g_ldb_mode}
% \begin{macro}{\g_ldb_type}
% \begin{macro}{\g_ldb_bdg}
% \begin{macro}{\g_ldb_segment}
%    We use these integer variables to keep track of what we have just
%    parsed and of what part of the context we are in.  |\g_ldb_mode|
%    encodes the local parsing state, e.g.\ have we just seen a
%    binding, a |<|, an indentifier\ldots\  |\g_ldb_type| represents
%    the type of the tag spec.  |\g_ldb_bdg| is the kind of binding
%    that precedes the tag spec.  |\g_ldb_segment| encodes a more
%    global parsing state: are we in the main segment of the context,
%    have we just seen a completed tag, are we now in the sequence of
%    end tags, or in the final sequence of begin tags?
%    \begin{center}
%    \DeleteShortVerb{\|}
%      \begin{tabular}{|l|l|l|l|l|}
%        \cline{2-5}
%          \multicolumn{1}{l}{}          &
%          \multicolumn{1}{|c}{\bf mode} &
%          \multicolumn{1}{|c}{\bf type} &
%          \multicolumn{1}{|c}{\bf bdg}  &
%          \multicolumn{1}{|c|}{\bf segment}\\
%        \hline
%  0 & before bdg       & begin     & sequenced & initial           \\
%  1 & after  bdg       & end       & tight     & main              \\
%  2 & after {\tt <}    & completed & loose     & after {\tt <foo>} \\
%  3 & after {\tt <foo} &           &           & end sequence      \\
%  4 & after {\tt foo}  &           &           & begin sequence    \\
%        \hline
%      \end{tabular}
%    \MakeShortVerb{\|}
%    \end{center}
%    \begin{macrocode}
\cs_new_eq:NN\g_ldb_mode   \c_zero
\cs_new_eq:NN\g_ldb_type   \c_zero
\cs_new_eq:NN\g_ldb_bdg    \c_zero
\cs_new_eq:NN\g_ldb_segment\c_zero
%    \end{macrocode}
% \end{macro}
% \end{macro}
% \end{macro}
% \end{macro}
%
% \begin{macro}{\g_ldb_abbrevs_tl}
%    A list of pairs.  Each pair represents an abbreviation and its
%    corresponding replacement text.  We also include pairs such as
%    |{*<-}{*}| for semantic normalization.
%    \begin{macrocode}
\tl_new:Nn\g_ldb_abbrevs_tl{
  {>...<}{>-><-<}   {>...}{>->}   {...<}{<-<}
  {<-<-} {<-}       {->->}{->}    {*<-*}{*}
  {*<-.} {*}        {.<-.}{*}     {*<-} {*}
  {<-*}  {*}        {<-.} {*}}
%    \end{macrocode}
% \end{macro}
%
% \begin{macro}{\ldb_parse:n}
%    This is the parser's top-level function.  It takes a context
%    specification as an argument, parses it, and places the
%    corresponding triples in |\g_ldb_triples_tl|.  It starts off
%    with a few initializations.
%    \begin{macrocode}
\cs_new:Npn\ldb_parse:n#1{
  \tl_gset:Nn\g_ldb_spec_tl   {#1}
  \tl_gclear:N  \g_ldb_triples_tl
  \tl_gclear:N  \g_ldb_tag_tl    {}
  \cs_gset_eq:NN\g_ldb_mode   \c_zero
  \cs_gset_eq:NN\g_ldb_type   \c_zero
  \cs_gset_eq:NN\g_ldb_bdg    \c_zero
  \cs_gset_eq:NN\g_ldb_segment\c_zero
%    \end{macrocode}
%    Then it proceeds to expand the abbreviations.  The empty pair of
%    abbreviations marks the end of the list of abbreviations.  The
%    reason for |\q_mark| is explained later.
%    \begin{macrocode}
  \gdef_gtmp:n{\ldb_expand_next:nnn{#1\q_mark}}
  \exp_after:wN\gtmp:w\g_ldb_abbrevs_tl{}{}
%    \end{macrocode}
%    When the abbreviation expansion process is done, it then invokes
%    the parser proper.  Finally, we arrive here.
%    |\ldb_parse_finish:| tidies up at the end of the parse, and
%    finally removes |\q_error| from the input.
%    \begin{macrocode}
  \ldb_parse_finish:\q_error}
%    \end{macrocode}
%    The purpose of |\q_error| is to make it possible to abort the
%    current |\ldb_parse:n| computation and proceed with the remainder
%    of the job.  The error handler uses \TeX's parameter scanning
%    mechanism to grab and discard everything up to, and including,
%    the |\q_error| token.  Naturally, in case the computation
%    completes successfully, it must take care of removing this token.
% \end{macro}
%
% \begin{macro}{\ldb_error:w}
%    Here is the error handler.  It aborts the current |\ldb_parse:n|
%    computation by grabbing and discarding everything up to and
%    including the |\q_error| token.
%    \begin{macrocode}
\cs_new:Npn\ldb_error:w#1\q_error{
  \msg_kernel_error:nn  { ldb } { illegal-context }
}

\msg_kernel_new:nnnn { ldb } { illegal-context }
  { Illegal~LDB~context:~\g_ldb_spec_tl.}
  { This~context~specification~is~illegal~and~
   will~be~ignored.~See~the~manual. }
%    \end{macrocode}
% \end{macro}
%
% \begin{macro}{\ldb_gtmp_error:}
%    Defines |\gtmp:w| as a continuation that invokes the error
%    handler.
%    \begin{macrocode}
\cs_new:Npn\ldb_gtmp_error:{\glet_gtmp:N\ldb_error:w}
%    \end{macrocode}
% \end{macro}
%
% \begin{macro}{\ldb_parse_finish:}
%    We must put off outputing a triple until we have seen what comes
%    after a tag identifier, in case it was followed by |>|.  As a
%    result, we may come to the end of the context and still have a
%    pending triple to output.  Here, we make sure to flush it out.
%    \begin{macrocode}
\cs_new:Npn\ldb_parse_finish:{
  \if_case:w\g_ldb_mode
%    \end{macrocode}
%    If mode=0, we are done, no triple is pending: we just make sure
%    to remove the |\q_error| token.
%    \begin{macrocode}
         \glet_gtmp:N\ldb_parse_done:w
%    \end{macrocode}
%    if we have just seen a binding or |<|, then the context is
%    incomplete and we signal an error.
%    \begin{macrocode}
  \or:   \ldb_gtmp_error:
  \or:   \ldb_gtmp_error:
%    \end{macrocode}
%    otherwise, we flush the pending triple and remove the |\q_error|
%    token.
%    \begin{macrocode}
  \else: \gdef_gtmp:n{\ldb_flush:NNN\c_zero\c_zero\c_zero
                      \ldb_parse_done:w}\fi:
  \gtmp:w}
%    \end{macrocode}
% \end{macro}
%
% \begin{macro}{\ldb_parse_done:w}
%    Removes the |\q_error| token when parsing successfully completes.
%    \begin{macrocode}
\cs_new:Npn\ldb_parse_done:w\q_error{}
%    \end{macrocode}
% \end{macro}
%
% \begin{macro}{\ldb_expand_it:w}
%    This macro is redefined during the process of expanding
%    abbreviations.  Its purpose is to scan the context for the
%    current abbreviation.  Let's say the current abbreviation is
%    |AA|.  Then |\ldb_expand_it:w| is redefined so that's its
%    parameter text is |#1AA#2\q_stop| and it is essentially invoked
%    on the context to which we append |AA\q_stop| to guarantee a
%    match.  We now come to the justification for |\q_mark|.  Suppose
%    the context has the form |...A|, i.e.\ it ends with |A| and
%    contains no occurrence of |AA|.  If |\ldb_expand_it:w| was simply
%    invoked on |...AAA\q_stop| we would find a match where the first
%    matching |A| is the one at the end of the context proper and the
%    second one is at the front of the |AA\q_stop| safeguard which we
%    appended to the context.  This is not a proper match.  In order
%    to prevent this problem, we insert the token |\q_mark| at the end
%    of the context.  Thus |\ldb_expand_it:w| is invoked on
%    |...A\q_mark AA\q_stop| and the spurious match can no longer
%    happen.  We must use a token that (1) cannot occur in a context,
%    and (2) is distinct from |\q_stop| (otherwise, if the context
%    ended with |AA|, we would match the first occurrence of
%    |AA\q_stop| and leave the second occurrence as trailing garbage).
%    \begin{macrocode}
\cs_new:Npn\ldb_expand_it:w{}
%    \end{macrocode}
% \end{macro}
%
% \begin{macro}{\ldb_expand_next:nnn}
%    Grabs the next abbreviation pair (args 2 and~3) and rewrites the
%    context (1st arg) accordingly.  At this point, we have already
%    put |\q_mark| at the end of the context.
%    \begin{macrocode}
\cs_new:Npn\ldb_expand_next:nnn#1#2#3{
  \tl_if_empty:nTF{#2}
%    \end{macrocode}
%    An empty abbreviation indicates that we have processed all
%    abbreviations.  We now proceed with the actual parsing task.
%    \begin{macrocode}
    {\ldb_parse_it:w#1}
%    \end{macrocode}
%    Otherwise, we redefine |\ldb_expand_it:w| to rewrite every
%    occurrence of the current abbreviation, as described earlier:
%    whenever we find a legitimate occurrence of |#2|, we replace it
%    by |#3| and we repeat the process.  When we are done, we proceed
%    with the next abbreviation pair.
%    \begin{macrocode}
    {\cs_gset:Npn\ldb_expand_it:w##1#2##2\q_stop
       {\tl_if_empty:nTF{##2}
          {\ldb_expand_next:nnn{##1}}
          {\ldb_expand_it:w##1#3##2\q_stop}}
     \ldb_expand_it:w#1#2\q_stop}}
%    \end{macrocode}
% \end{macro}
%
% \begin{macro}{\ldb_parse_it:w}
%    The entry point into the actual parser.  Note that we need to
%    remove the |\q_mark| token.  We then go on to scan for tight
%    bindings.
%    \begin{macrocode}
\cs_new:Npn\ldb_parse_it:w#1\q_mark{\ldb_parse_dot:w#1.\q_stop}
%    \end{macrocode}
% \end{macro}
%
% \begin{macro}{\ldb_parse_dot:w}
%    Scan for a tight binding.  The 1st argument is guaranteed not to
%    contain any tight binding since \TeX{} scan's for the first
%    occurrence; therefore we scan for other things starting with
%    loose bindings.  Then, if the 2nd arg is not empty, we have found
%    a legitimate occurrence of a tight binding and we call
%    |\ldb_push_dot:| to record that fact --- When the 2nd arg is
%    empty, the dot we just found is the one |\ldb_parse_it:w| itself
%    appended to catch the default case and guarantee a match.
%    Finally we proceed to look for more tight bindings in the 2nd
%    arg.
%    \begin{macrocode}
\cs_new:Npn\ldb_parse_dot:w#1.#2\q_stop{
  \tl_if_empty:nF{#1}{\ldb_parse_star:w#1*\q_stop}
  \tl_if_empty:nF{#2}{\ldb_push_dot:\ldb_parse_dot:w#2\q_stop}}
%    \end{macrocode}
% \end{macro}
%
% \begin{macro}{\ldb_parse_star:w}
%    Scan for a loose binding.  In the 1st arg we further look for
%    occurrences of |<|.  If the 2nd arg is not empty, we record
%    finding a legitimate loose binding, then proceed to scan the 2nd
%    arg for more loose bindings.
%    \begin{macrocode}
\cs_new:Npn\ldb_parse_star:w#1*#2\q_stop{
  \tl_if_empty:nF{#1}{\ldb_parse_begin:w#1<\q_stop}
  \tl_if_empty:nF{#2}{\ldb_push_star:\ldb_parse_star:w#2\q_stop}}
%    \end{macrocode}
% \end{macro}
%
% \begin{macro}{\ldb_parse_begin:w}
%    Scan for |<|.  In the 1st arg we further look for occurrences of
%    |>|.  If the 2nd arg is not empty, we record finding a legitimate
%    |<|, then proceed to scan the 2nd arg for more occurrences of
%    |<|.
%    \begin{macrocode}
\cs_new:Npn\ldb_parse_begin:w#1<#2\q_stop{
  \tl_if_empty:nF{#1}{\ldb_parse_end:w#1>\q_stop}
  \tl_if_empty:nF{#2}{\ldb_push_begin:\ldb_parse_begin:w#2\q_stop}}
%    \end{macrocode}
% \end{macro}
%
% \begin{macro}{\ldb_parse_end:w}
%    Scan for |>|.  If the 1st arg is not empty, it must be a tag
%    identifier, and we record it.  If the 2nd arg is not empty, we
%    record finding a legitimate |>|, then proceed to scan the 2nd arg
%    for more occurrences of |>|.
%    \begin{macrocode}
\cs_new:Npn\ldb_parse_end:w#1>#2\q_stop{
  \tl_if_empty:nF{#1}{\ldb_push_tag:n{#1}}
  \tl_if_empty:nF{#2}{\ldb_push_end:\ldb_parse_end:w#2\q_stop}}
%    \end{macrocode}
% \end{macro}
%
% \begin{macro}{\ldb_push_bdg:N}
%    Takes a constant integer as an argument.  |\c_one| for a tight
%    binding, |\c_two| for a loose binding.  The only places a binding
%    is allowed is after |<foo| or |foo|, i.e.\ in mode 3 or~4.  In
%    that case, we flush the pending triple and reset |\g_ldb_type|
%    to~0, |\g_ldb_bdg| to the given constant, and |\g_ldb_mode| to~1
%    to indicate that we have just seen a binding.
%    \begin{macrocode}
\cs_new:Npn\ldb_push_bdg:N#1{
  \if_num:w\g_ldb_mode<\c_three \ldb_gtmp_error:
  \else: \gdef_gtmp:n{\ldb_flush:NNN\c_zero#1\c_one} \fi:
  \gtmp:w}
%    \end{macrocode}
% \end{macro}
%
% \begin{macro}{\ldb_push_dot:}
% \begin{macro}{\ldb_push_star:}
%    Record a tight or a loose binding.
%    \begin{macrocode}
\cs_new:Npn\ldb_push_dot: {\ldb_push_bdg:N\c_one}
\cs_new:Npn\ldb_push_star:{\ldb_push_bdg:N\c_two}
%    \end{macrocode}
% \end{macro}
% \end{macro}
%
% \begin{macro}{\ldb_push_begin:}
%    Record finding an occurrence of |<|.  If we are in mode~2, we
%    have just found two consecutive occurrences of |<|.  This is
%    illegal and we signal an error.  If we were in mode~0 or~1 (i.e.\
%    expecting or having just found a binding), then we just make a
%    note that we found |<| by proceeding to mode~2.  Otherwise, we
%    must flush the pending triple and then reset |\g_ldb_type| to~0
%    (a begin tag), |\g_ldb_bdg| to~0 (sequencing: no intervening
%    binding was found), and |\g_ldb_mode| to~2 (we just found |<|).
%    \begin{macrocode}
\cs_new:Npn\ldb_push_begin:{
  \if_num:w\g_ldb_mode=\c_two \ldb_gtmp_error:
  \else:
  \if_num:w\g_ldb_mode<\c_two
         \gdef_gtmp:n{\cs_gset_eq:NN\g_ldb_mode\c_two}
  \else: \gdef_gtmp:n{\ldb_flush:NNN\c_zero\c_zero\c_two} \fi:\fi:
  \gtmp:w}
%    \end{macrocode}
% \end{macro}
%
% \begin{macro}{\ldb_push_tag:n}
%    Record finding a tag identifier (the argument).  If we are in
%    mode 3 or~4, we just found 2 identifiers in sequence.  This
%    should not be possible, but we signal an error anyway.
%    Otherwise, we record the identifier in |\g_ldb_tag_tl|.  If we
%    are in mode~2, then we proceed to mode~3 to indicate that we have
%    found something of the form |<foo|.  If we are in mode~0 or~1, we
%    proceed directly to mode~4 indicating that the identifier was not
%    preceded by |<|.
%    \begin{macrocode}
\cs_new:Npn\ldb_push_tag:n#1{
  \if_num:w\g_ldb_mode>\c_two \exp_after:wN\ldb_error:w
  \else: \tl_gset:Nn\g_ldb_tag_tl{#1}
         \if_num:w\g_ldb_mode=\c_two \cs_gset_eq:NN\g_ldb_mode\c_three
         \else:                      \cs_gset_eq:NN\g_ldb_mode\c_four
  \fi:\fi:}
%    \end{macrocode}
% \end{macro}
%
% \begin{macro}{\ldb_push_end:}
%    We have found an occurrence of |>| and we should flush the
%    pending triple.  If the mode is less than~3, we have not recently
%    seen an indentifier.  Therefore, we signal an error.  Otherwise,
%    if the current mode is~4, indicating that we have seen |<foo|,
%    then we now promote |\g_ldb_type| to~2 to indicate a completed
%    tag.  If the current mode is~4, then we promote |\g_ldb_type|
%    to~1 to indicate an end tag.  Finally we flush the pending triple
%    and reset |\g_ldb_type|, |\g_ldb_bdg| and |\g_ldb_mode| to~0.
%    \begin{macrocode}
\cs_new:Npn\ldb_push_end:{
  \if_num:w\g_ldb_mode<\c_three \ldb_gtmp_error:
  \else:
    \if_num:w\g_ldb_mode=\c_three \cs_gset_eq:NN\g_ldb_type\c_two
    \else:                        \cs_gset_eq:NN\g_ldb_type\c_one \fi:
    \gdef_gtmp:n{\ldb_flush:NNN\c_zero\c_zero\c_zero} \fi:
  \gtmp:w}
%    \end{macrocode}
% \end{macro}
%
% \begin{macro}{\ldb_flush:NNN}
%    Flush the pending triple.  Calls to this functions are always
%    followed by 3 integer constants which |\ldb_reset:NNN| uses to
%    update the values of |\g_ldb_type|, |\g_ldb_bdg| and
%    |\g_ldb_mode|.
%    \begin{macrocode}
\cs_new:Npn\ldb_flush:NNN{
%    \end{macrocode}
%    First, we select the appropriate handler according to whether we
%    are flushing a begin tag, an end tag, or a completed tag.
%    \begin{macrocode}
  \if_case:w\g_ldb_type \glet_gtmp:N\ldb_push_begin_tag:
  \or:                  \glet_gtmp:N\ldb_push_end_tag:
  \else:                \glet_gtmp:N\ldb_push_compl_tag:\fi:
%    \end{macrocode}
%    A binding, i.e.\ |\g_ldb_bdg|~$\neq$~0, is not allowed after a
%    completed tag, or in the final end/begin sequences, i.e.\
%    |\g_ldb_segment|~$>$~1.  If we find such a situation, we must
%    signal an error since only sequencing is permitted.
%    \begin{macrocode}
  \if_num:w\g_ldb_bdg=\c_zero \else:
  \if_num:w\g_ldb_segment>\c_one \ldb_gtmp_error: \fi:\fi:
%    \end{macrocode}
%    Push the pending triple onto |\g_ldb_triples_tl|, execute the
%    appropriate handler, and reset type, bdg and mode as explained
%    earlier.  Note that |\gtmp:w| may be the error handler and that
%    we first push the triple anyway; it doesn't hurt and makes the
%    code simpler.
%    \begin{macrocode}
  \ldb_push:\gtmp:w\ldb_reset:NNN}
%    \end{macrocode}
% \end{macro}
%
% \begin{macro}{\ldb_push:}
%    Push a new triple (bdg,type,tag) onto |\g_ldb_triples_tl|.  We
%    cannot simply use |\g_ldb_bdg| as the 1st element of the triple
%    because its value keeps being changed by the parsing algorithm.
%    We cannot try to expand it, because it is `let' to a non
%    expandable integer constant.  To get around that problem, we use
%    an |\if_case:w| to select the appropriate constant.  Another
%    solution would be to use |\g_ldb_bdg_tl| instead of
%    |\g_ldb_bdg|, and |\tl_gset:Nn| instead of |\cs_gset_eq:NN|, but that
%    would be less efficient since every numerical test would have to
%    expand |\g_ldb_bdg_tl|.  Note the use of |\g_tempa_toks| to
%    prevent expansion to apply to the tag identifier, just in case it
%    contains an active character with an expandable definition.
%    \begin{macrocode}
%%% FMi need toks for now:
%\newtoks \g_tmpa_toks

\cs_new:Npn\ldb_push:{
%  \global\g_tmpa_toks \exp_after:wN 
%      {
%        \cs:w  env_begin_code_
%              \g_ldb_tag_tl
%              :
%       \cs_end:
%      }
  \tl_gput_right:Nx\g_ldb_triples_tl
    {
     \if_case:w\g_ldb_bdg \c_zero\or:\c_one\else:\c_two\fi:
     \if_case:w\g_ldb_type\c_zero\or:\c_one\else:\c_two\fi:
     \exp_not:N\ldb_triple:nNN
%     \the\g_tmpa_toks
     {\g_ldb_tag_tl}
    }
%  \global\g_tmpa_toks{}
}
%    \end{macrocode}
% \end{macro}
%
% \begin{macro}{\ldb_reset:NNN}
%    Update type, bdg and mode with the given values.
%    |\ldb_reset:NNN| is called at the very end of |\ldb_flush:NNN|.
%    This is why every call to |\ldb_flush:NNN| must be followed by 3
%    integer constants.
%    \begin{macrocode}
\cs_new:Npn\ldb_reset:NNN#1#2#3{
  \cs_gset_eq:NN\g_ldb_type#1
  \cs_gset_eq:NN\g_ldb_bdg #2
  \cs_gset_eq:NN\g_ldb_mode#3}
%    \end{macrocode}
% \end{macro}
%
% \begin{macro}{\ldb_push_begin_tag:}
%    We have just flushed a triple involving a begin tag.  If we are
%    still in the main part of the context, that's where we remain, in
%    segment~1.  If we are in the end sequence part of the context,
%    i.e.\ segment~3, we move on to the begin sequence, i.e.\
%    segment~4.  If we are already in segment~4, that's where we stay.
%    \begin{macrocode}
\cs_new:Npn\ldb_push_begin_tag:{
  \if_num:w\g_ldb_segment<\c_three \cs_gset_eq:NN\g_ldb_segment\c_one
  \else:                           \cs_gset_eq:NN\g_ldb_segment\c_four\fi:}
%    \end{macrocode}
% \end{macro}
%
% \begin{macro}{\c_ldb_dash_tl}
%    We use this to compare with |\g_ldb_tag_tl| and determine
%    whether the tag identifier is |-|.
%    \begin{macrocode}
\tl_new:Nn\c_ldb_dash_tl{-}
%    \end{macrocode}
% \end{macro}
%
% \begin{macro}{\ldb_push_compl_tag:}
%    We have just flushed a triple involving a completed tag.  A
%    completed tag is only allowed in the main part of the context.
%    Therefore, if we are in segment 3 or~4, we signal an error.  Also
%    we check if the identifier is |-|.  If yes, we signal an error
%    because |<->| is illegal.  Otherwise, we simply update
%    |\g_ldb_segment| to~2 to indicate that we are now right after a
%    completed tag.
%    \begin{macrocode}
\cs_new:Npn\ldb_push_compl_tag:{
  \if_num:w\g_ldb_segment>\c_two              \ldb_gtmp_error: \else:
  \if_meaning:w \g_ldb_tag_tl\c_ldb_dash_tl \ldb_gtmp_error: \else:
    \gdef_gtmp:n{\cs_gset_eq:NN\g_ldb_segment\c_two}              \fi:\fi:
  \gtmp:w}
%    \end{macrocode}
% \end{macro}
%
% \begin{macro}{\ldb_push_end_tag:}
%    We have just flushed a triple involving an end tag.  Normally, we
%    would simply promote |\g_ldb_segment| to~3 to indicate that we
%    are now in the end sequence of the context.  However, if we are
%    at the beginning of the context, this end tag should not be
%    preceded by a binding.  Also, if we are in the main part of the
%    context, but not at the beginning, the end tag must be preceded
%    by a loose binding.  On the other hand, if the previous tag was a
%    completed tag, we must signal an error since a completed tag
%    cannot be followed by a binding.  If we are in the end sequence,
%    then only sequencing is allowed, and if we are in the begin
%    sequence, only begin tags are allowed.
%    \begin{macrocode}
\cs_new:Npn\ldb_push_end_tag:{
  \gdef_gtmp:n{\cs_gset_eq:NN\g_ldb_segment\c_three}
  \if_case:w\g_ldb_segment
         \if_num:w\g_ldb_bdg=\c_zero\else: \ldb_gtmp_error: \fi:
  \or:   \if_num:w\g_ldb_bdg=\c_two \else: \ldb_gtmp_error: \fi:
  \or:   \ldb_gtmp_error:
  \or:   \if_num:w\g_ldb_bdg=\c_zero\else: \ldb_gtmp_error: \fi:
  \else: \ldb_gtmp_error: \fi:
  \gtmp:w}
%    \end{macrocode}
% \end{macro}



%    \end{macrocode}
%




%
% \begin{macro}{\LDBenter}
%    This fixes a bug.
%    \begin{macrocode}
\cs_set:Npn\LDBenter{
  \group_begin:
  \ExplSyntaxOn
  \exp_after:wN
  \group_end:
  \ldb_enter:nnn
}

% FMi for testing without env integration
%
% This is not the way this should work!

\cs_set:Npn\LDBupdateBegin #1 {\ldb_update_begin:n 
                                         { #1 }}
\cs_set:Npn\LDBupdateEnd   #1 {\ldb_update_end:n
                                         { #1 }}

\cs_set:Npn\LDBdeclDefParm {\ldb_decl_defparm:N}
%    \end{macrocode}
% \end{macro}
%
%
% \subsection{Building the database}
%
%
% \begin{macro}{\g_ldb_begin_root_prop}
% \begin{macro}{\g_ldb_end_root_prop}
%    We allocate property lists for the two roots of the database tree.
%    \begin{macrocode}
\prop_new:N\g_ldb_begin_root_prop
\prop_new:N\g_ldb_end_root_prop
%<*precompile>
         \cs_record_name:N\g_ldb_begin_root_prop
         \cs_record_name:N\g_ldb_end_root_prop
%</precompile>
%    \end{macrocode}
% \end{macro}
% \end{macro}
%
% \begin{macro}{\l_ldb_curr_node_tl}
% \begin{macro}{\l_trace_tl}
%    The allocation of two token lists which will contain the current
%    node in the tree and some tracing information.
%    \begin{macrocode}
\tl_new:Nn\l_ldb_curr_node_tl{}
\tl_new:Nn\l_trace_tl{}
%    \end{macrocode}
% \end{macro}
% \end{macro}
%
%
%
% \begin{macro}{\ldb_enter:nnn}
%    This enters information in the database.
%    \begin{macrocode}
\cs_new:Npn\ldb_enter:nnn#1#2#3{
%    \end{macrocode}
%    \begin{macrocode}
   \ldb_parse:n{#1}
%<*trace>
    \trace_ldb_internal:n{{lex:~\tl_to_str:N\g_ldb_triples_tl}}
%</trace>
%    \end{macrocode}
%
%    \begin{macrocode}
   \seq_if_in:NnTF\l_ldb_def_parm_seq#2
     {\tl_set:Nn
         \l_ldb_parm_tl
         {
          \show_def_update:Nnn#2{#3}{#1}
         }
     }
     {\tl_set:Nn
        \l_ldb_parm_tl
        {
         \show_register_update:Nnn#2{#3}{#1}
        }
     }
%    \end{macrocode}
%    Now we move down the tree to put the new assignment into the
%    database thereby adding new nodes and pointers if necessary. 
%    this is done by simply calling |\g_ldb_triples_tl| which contains
%    a function in front of every triple so that we can safely proceed.
%    \begin{macrocode}
   \exp_after:wN\ldb_decide_root:nn\g_ldb_triples_tl\c_three\c_zero
%<*trace>
   \show_ldb_entry:nnn{#1}{#2}{#3}
%</trace>
}
%    \end{macrocode}
% \end{macro}

%
% \begin{macro}{\ldb_decide_root:nn}
%    \begin{macrocode}
\cs_new:Npn \ldb_decide_root:nn #1#2 {
   \if_meaning:w #1\c_zero
      \if_meaning:w #2\c_zero
        \tl_set:Nn\l_ldb_curr_node_tl \g_ldb_begin_root_prop
      \else:
        \tl_set:Nn\l_ldb_curr_node_tl \g_ldb_end_root_prop
      \fi:
   \else:
      \ERROR % should not happen I think
   \fi:
}
%    \end{macrocode}
% \end{macro}

%
% \begin{macro}{\l_ldb_parm_tl}
%    The |\l_ldb_parm_tl| holds the new assignment that will be added
%    to the data cell of one node in the tree.
%    \begin{macrocode}
\tl_new:Nn\l_ldb_parm_tl{}
%    \end{macrocode}
% \end{macro}
%
% \begin{macro}{\ldb_decl_defparm:N}
%    At the moment, we have to specify that a parameter belongs to the
%    class of parameters that are set using |\cs_set:Npn|. Of course, this
%    can be found out without human intervention if we keep a proper
%    record of all parameters that are set using |\int_new:N|,
%    |\skip_new:N|, etc., i.e.\ redefining those macros to save their
%    arguments on some lists which are then examined in |\ldb_enter:nnn|.
%    But for the moment I thought it is easier to specify the |\cs_set:Npn|
%    parameters using |\ldb_decl_defparm:N| in the document style.
%    \begin{macrocode}
\cs_new:Npn\ldb_decl_defparm:N#1{\seq_put_right:Nn\l_ldb_def_parm_seq#1}
%    \end{macrocode}
% \end{macro}
%
% \begin{macro}{\l_ldb_def_parm_seq}
%    |\ldb_decl_defparm:N| will write its argument onto a list (sorry sequence)
%    which is searched for parameter names in |\ldb_enter:nnn|.
%    \begin{macrocode}
\seq_new:N\l_ldb_def_parm_seq
%<*precompile>
         \cs_record_name:N\l_ldb_def_parm_seq
%</precompile>
%    \end{macrocode}
% \end{macro}
%
% \begin{macro}{\ldb_triple:nNN}
%    The |\ldb_triple:nNN| function is the heart of the database
%    generation. 
%
%    In |\l_ldb_curr_node_tl| we keep a pointer to the current
%    node of our data structure as we go along.
%
%    The token list pointers |\l_ldb_tight_tl|, |\l_ldb_loose_tl|,
%    etc., will contain pointers to subtrees, i.e., they are one step
%    ahead of |\l_ldb_curr_node_tl| (they are also pointers to
%    nodes not nodes!).  The idea is to
%    \begin{itemize}
%    \item
%      look for the class in |\l_ldb_curr_node_tl|. If it isn't there so
%      far it will be created in the following process.
%    \item
%      Then take a look at the second arg. If we have exhausted the
%      sequence of classes and bindings (i.e., if the second arg is
%      |\q_stop|) we update the |\l_ldb_assign_tl| write everything
%      back into the |\l_ldb_curr_node_tl| and terminate.
%    \item
%      Otherwise the second argument is a binding and go down one level
%      by changing |\l_ldb_curr_plist| to contain the corresponding
%      plist. However, we first have to allocate a new symbol for this
%      node in case it is still pointing to |\q_nil|.
%    \end{itemize}
%    \begin{macrocode}
\cs_set:Npn\ldb_triple:nNN#1#2#3{
%<*trace> 
   \trace_ldb_internal:n{\string\ldb_triple:nNN  \exp_not:n{{#1}{#2}{#3}}}
%</trace>
    \exp_after:wN
    \ldb_lookup:Nn
       \l_ldb_curr_node_tl
       {#1}
%    \end{macrocode}
%    Otherwise we check that the pointer to binding or sequencing
%    is already there; if not we generate it. We then save its name in
%    |\l_tmpa_tl| for later update of |\l_ldb_curr_node_tl|.
%    \begin{macrocode}
      \if_case:w #2
        \if_case:w #3 % begin
          \if_meaning:w\l_ldb_sb_tl\q_nil
              \cs_ggen_prop:N\l_ldb_sb_tl
          \fi:
          \tl_set_eq:NN\l_tmpa_tl\l_ldb_sb_tl
        \or: % end
          \if_meaning:w\l_ldb_se_tl\q_nil
              \cs_ggen_prop:N\l_ldb_se_tl
          \fi:
          \tl_set_eq:NN\l_tmpa_tl\l_ldb_se_tl
        \or: % complete
          \if_meaning:w\l_ldb_sc_tl\q_nil
              \cs_ggen_prop:N\l_ldb_sc_tl
          \fi:
          \tl_set_eq:NN\l_tmpa_tl\l_ldb_sc_tl
        \fi:
      \or: % tight binding
        \if_case:w #3 % begin
          \if_meaning:w\l_ldb_tb_tl\q_nil
              \cs_ggen_prop:N\l_ldb_tb_tl
          \fi:
          \tl_set_eq:NN\l_tmpa_tl\l_ldb_tb_tl
        \or: % end
          \err_latex_bug:n{Not~possible}
        \or: % complete
          \if_meaning:w\l_ldb_tc_tl\q_nil
              \cs_ggen_prop:N\l_ldb_tc_tl
          \fi:
          \tl_set_eq:NN\l_tmpa_tl\l_ldb_tc_tl
        \fi:
      \or: % loose binding
        \if_case:w #3 % begin
          \if_meaning:w\l_ldb_lb_tl\q_nil
              \cs_ggen_prop:N\l_ldb_lb_tl
          \fi:
          \tl_set_eq:NN\l_tmpa_tl\l_ldb_lb_tl
        \or: % end
          \if_meaning:w\l_ldb_le_tl\q_nil
              \cs_ggen_prop:N\l_ldb_le_tl
          \fi:
          \tl_set_eq:NN\l_tmpa_tl\l_ldb_le_tl
        \or: % complete
          \if_meaning:w\l_ldb_lc_tl\q_nil
              \cs_ggen_prop:N\l_ldb_lc_tl
          \fi:
          \tl_set_eq:NN\l_tmpa_tl\l_ldb_lc_tl
        \fi:
      \or: % parsing finished we now update assignments
%       \trace_ldb_internal:n{qstop~found:~update~parm}
        \if_meaning:w\l_ldb_assign_tl\q_nil
%FMi this one really holds tokens not a pointer (which is a prop)
           \cs_ggen_sym:N\l_ldb_assign_tl
        \fi:
        \exp_after:wN
        \tl_gput_right:No
            \l_ldb_assign_tl
            \l_ldb_parm_tl
      \fi:
%    \end{macrocode}
%    Put the node back (in case one have changed from |\q_nil| to a
%    symbol or the node wasn't present at all before).
%    \begin{macrocode}
      \ldb_put_class:n {#1}
%    \end{macrocode}
%    Move down the tree. The use of |\tl_set_eq:NN| results in
%    \begin{macrocode}
      \tl_set_eq:NN\l_ldb_curr_node_tl\l_tmpa_tl
}
%    \end{macrocode}
% \end{macro}
%
% \begin{macro}{\ldb_lookup:Nn}
%    The |\ldb_lookup:Nn| function gets a pointer to a class list, a
%    class, and will return the subtree pointers in the token list
%    pointers |\l_ldb_loose_tl|, |\l_ldb_tight_tl|, etc.  If the
%    class isn't present in |#1| |\q_nil| will be written into all
%    token list pointers.
%    \begin{macrocode}
\cs_new:Npn\ldb_lookup:Nn#1#2{
   \prop_get:NnN#1{#2}\l_tmpa_tl
%    \end{macrocode}
%    The |\prop_get:NnN| function will return |\q_no_value| if the
%    property wasn't found.
%    \begin{macrocode}
   \if_meaning:w\l_tmpa_tl\q_no_value
     \tl_set_eq:NN\l_ldb_sb_tl\q_nil
     \tl_set_eq:NN\l_ldb_se_tl\q_nil
     \tl_set_eq:NN\l_ldb_sc_tl\q_nil
     \tl_set_eq:NN\l_ldb_tb_tl\q_nil
     \tl_set_eq:NN\l_ldb_tc_tl\q_nil
     \tl_set_eq:NN\l_ldb_lb_tl\q_nil
     \tl_set_eq:NN\l_ldb_le_tl\q_nil
     \tl_set_eq:NN\l_ldb_lc_tl\q_nil
     \tl_set_eq:NN\l_ldb_assign_tl\q_nil
   \else:
     \exp_after:wN\ldb_get_pointers:w\l_tmpa_tl
        \q_nil\q_nil\q_nil\q_nil\q_nil\q_nil\q_nil\q_nil\q_stop
   \fi:}
%    \end{macrocode}
% \end{macro}
%
%  \begin{macro}{\ldb_get_pointers:w}
%    Routine that separates the pointers.
%    \begin{macrocode}
\cs_new:Npn\ldb_get_pointers:w#1#2#3#4#5#6#7#8#9{
  \tl_set:Nn\l_ldb_assign_tl{#1}
  \tl_set:Nn\l_ldb_lb_tl{#2}
  \tl_set:Nn\l_ldb_sb_tl{#3}
  \tl_set:Nn\l_ldb_tb_tl{#4}
  \tl_set:Nn\l_ldb_se_tl{#5}
  \tl_set:Nn\l_ldb_le_tl{#6}
  \tl_set:Nn\l_ldb_tc_tl{#7}
  \tl_set:Nn\l_ldb_lc_tl{#8}
  \tl_set:Nn\l_ldb_sc_tl{#9}
  \ldb_gobble:w
}
\cs_new:Npn\ldb_gobble:w#1\q_stop{}
%    \end{macrocode}
%  \end{macro}
%
%
%
%  \begin{macro}{\ldb_put_class:n}
%    Routine that writes a node back into the database using reverse
%    order.
%    \begin{macrocode}
\cs_new:Npn\ldb_put_class:n#1{
     \tl_clear:N\l_tmpb_tl
     \ldb_put_class_aux:N\l_ldb_sc_tl
     \ldb_put_class_aux:N\l_ldb_lc_tl
     \ldb_put_class_aux:N\l_ldb_tc_tl
     \ldb_put_class_aux:N\l_ldb_le_tl
     \ldb_put_class_aux:N\l_ldb_se_tl
     \ldb_put_class_aux:N\l_ldb_tb_tl
     \ldb_put_class_aux:N\l_ldb_sb_tl
     \ldb_put_class_aux:N\l_ldb_lb_tl
     \ldb_put_class_aux:N\l_ldb_assign_tl
     \exp_after:wN
     \prop_gput:Nno
       \l_ldb_curr_node_tl
        {#1}
        \l_tmpb_tl
}

\cs_new:Npn\ldb_put_class_aux:N#1{
    \if_meaning:w#1\q_nil
        \if_meaning:w\l_tmpb_tl\c_empty_tl
        \else:
          \tl_put_left:Nn\l_tmpb_tl\q_nil
        \fi:
    \else:
       \tl_put_left:No\l_tmpb_tl#1
    \fi:
}
%    \end{macrocode}
%  \end{macro}
%
%
%  \begin{macro}{\l_ldb_sb_tl}
%  \begin{macro}{\l_ldb_se_tl}
%  \begin{macro}{\l_ldb_sc_tl}
%  \begin{macro}{\l_ldb_tb_tl}
%  \begin{macro}{\l_ldb_tc_tl}
%  \begin{macro}{\l_ldb_lb_tl}
%  \begin{macro}{\l_ldb_le_tl}
%  \begin{macro}{\l_ldb_lc_tl}
%  \begin{macro}{\l_ldb_assign_tl}
%    Somewhere we should define the pointers from above.
%    \begin{macrocode}
\tl_new:Nn\l_ldb_sb_tl{}
\tl_new:Nn\l_ldb_se_tl{}
\tl_new:Nn\l_ldb_sc_tl{}
\tl_new:Nn\l_ldb_tb_tl{}
\tl_new:Nn\l_ldb_tc_tl{}
\tl_new:Nn\l_ldb_lb_tl{}
\tl_new:Nn\l_ldb_le_tl{}
\tl_new:Nn\l_ldb_lc_tl{}
\tl_new:Nn\l_ldb_assign_tl{}
%    \end{macrocode}
%  \end{macro}
%  \end{macro}
%  \end{macro}
%  \end{macro}
%  \end{macro}
%  \end{macro}
%  \end{macro}
%  \end{macro}
%  \end{macro}
%
%
%
%  \subsection{Updating parameters}
%
%
%  \begin{macro}{\l_ldb_classes_clist}
%    \begin{macrocode}
\clist_new:N \l_ldb_classes_clist
%    \end{macrocode}
%  \end{macro}
%    
%
%  \begin{macro}{\ldb_update_begin:n}
%    Routine called to update parameters at the `begin' of some
%    structure.
%    \begin{macrocode}
\cs_new:Npn\ldb_update_begin:n#1{
  \if_mode_vertical:
%<*trace>
  \trace_ldb:n{^^JLDB:~Begin~Processing~*****~
                                   #1
                                   \space *****~
                                   (sequenced)~
                                   \file_show_lineno:}
%</trace>
  \else:
%<*trace>
  \tl_set:Nn\l_tmpa_tl{#1}
  \trace_ldb:n{^^JLDB:~Begin~Processing~*****~
                                   #1
                                   \space *****~
                                   \file_show_lineno:}
%</trace>
    \tl_gclear:N\g_ldb_sbml_tl
    \tl_gclear:N\g_ldb_seml_tl
    \tl_gclear:N\g_ldb_next_scml_tl
  \fi:
%
  \tl_set_eq:NN\l_ldb_scml_tl\g_ldb_next_scml_tl
%  \tl_gclear:N \g_ldb_next_scml_tl
  \tl_clear:N \l_ldb_new_scml_tl
%
  \tl_set_eq:NN\l_ldb_lcml_tl\g_ldb_next_lcml_tl
  \tl_set_eq:NN\l_ldb_new_lcml_tl\g_ldb_next_lcml_tl
%
  \tl_set_eq:NN\l_ldb_tcml_tl\g_ldb_next_tcml_tl
  \tl_set_eq:NN\l_ldb_new_tcml_tl\g_ldb_next_tcml_tl
%
  \tl_set_eq:NN\l_ldb_new_leml_tl\l_ldb_leml_tl
  \tl_set_eq:NN\l_ldb_new_lbml_tl\l_ldb_lbml_tl

  \tl_clear:N \l_ldb_new_tbml_tl
  \tl_clear:N \l_ldb_new_sbml_tl
%    \end{macrocode}
%    Then we process the new classes in turn one by one.
%    |\ldb_update_aux:n| processes a single class and is mapped over
%    the clist.
%    \begin{macrocode}
  \clist_map_inline:nn{#1}{\ldb_update_aux:n {##1}}
%    \end{macrocode}
%    Finally we set the new match candidates. 
%    \begin{macrocode}
  \ldb_show_new_mls:
  \tl_set_eq:NN \l_ldb_lbml_tl \l_ldb_new_lbml_tl
  \tl_set_eq:NN \l_ldb_tbml_tl \l_ldb_new_tbml_tl
  \tl_set_eq:NN \l_ldb_leml_tl \l_ldb_new_leml_tl

  \tl_gset_eq:NN \g_ldb_sbml_tl \l_ldb_new_sbml_tl
%
  \tl_gset_eq:NN \g_ldb_next_scml_tl \l_ldb_new_scml_tl
  \tl_gset_eq:NN \g_ldb_next_lcml_tl \l_ldb_new_lcml_tl
  \tl_gset_eq:NN \g_ldb_next_tcml_tl \l_ldb_new_tcml_tl
}
%    \end{macrocode}
%  \end{macro}
%
%
%  \begin{macro}{\ldb_update_aux:n}
%    A helper function to |\ldb_update_begin:n|.
%    \begin{macrocode}
\cs_new:Npn\ldb_update_aux:n#1{
%<*trace> 
   \trace_ldb_internal:n{\string\ldb_update_aux:N \exp_not:n{{#1}}}
%</trace>
%    \end{macrocode}
%    Otherwise we first try to match the class in |#1| looking at the
%    loose candidates.
%    \begin{macrocode}
%<*trace>
       \show_new_ml:nNn {#1} l {lb}
       \tl_set:Nn\l_trace_tl{LBML}
%</trace>
       \tl_map_inline:Nn
          \l_ldb_lbml_tl
          { \ldb_update_single:nN {#1} ##1 }
%    \end{macrocode}
%    Then we are going to take a look at the tight candidates.
%    \begin{macrocode}
%<*speed>
     \if_meaning:w
        \l_ldb_tbml_tl
        \c_empty_tl
     \else:
%</speed>
%<*trace>
       \show_new_ml:nNn{#1} l {tb}
       \tl_set:Nn\l_trace_tl{TBML}
%</trace>
       \tl_map_inline:Nn
          \l_ldb_tbml_tl
          { \ldb_update_single:nN {#1} ##1 }
%<*speed>
     \fi:
%</speed>
%    \end{macrocode}
%    Next step is to look into begin-sequencing.
%    \begin{macrocode}
%<*speed>
     \if_meaning:w
        \g_ldb_sbml_tl
        \c_empty_tl
     \else:
%</speed>
%    \end{macrocode}
%    Now the non-empty begin-begin match list.
%    \begin{macrocode}
%<*trace>
       \show_new_ml:nNn {#1} g {sb}
       \tl_set:Nn\l_trace_tl{SBML}
%</trace>
       \tl_map_inline:Nn
          \g_ldb_sbml_tl
          { \ldb_update_single:nN {#1} ##1 }
%<*speed>
     \fi:
%</speed>
}
%    \end{macrocode}
%  \end{macro}
%
%
%  \begin{macro}{\ldb_update_single:nN}
%    Undoc
%    \begin{macrocode}
\cs_new:Npn\ldb_update_single:nN#1#2{
%<*trace> 
   \trace_ldb_internal:n{\string\ldb_update_single:nN \exp_not:n{{#1}{#2}}}
%</trace>
%<*trace>
      \trace_ldb_internal:n{..~looking~for~\token_to_str:N#1~in~
         \l_trace_tl\space
         \token_to_str:N#2~containing^^J^^J
         <<\tl_to_str:N#2>>^^J}
%</trace>
      \prop_get:NnN
        #2
        {#1}
        \l_tmpa_tl
      \if_meaning:w\l_tmpa_tl\q_no_value      % return if not found
%<*trace>
        \trace_ldb_internal:n{...~no~match~found}
%</trace>
      \else:
%<*trace>
        \trace_ldb_internal:n{...~found~\tl_to_str:N\l_tmpa_tl}
%</trace>
        \exp_after:wN\ldb_update_bmls:w\l_tmpa_tl
          \q_nil\q_nil\q_nil\q_nil\q_nil\q_nil\q_nil\q_nil\q_stop
      \fi:
}
%    \end{macrocode}
%  \end{macro}
%
%  \begin{macro}{\ldb_update_bmls:w}
%    Routine that separates the pointers.
%    \begin{macrocode}
\cs_new:Npn\ldb_update_bmls:w#1#2#3#4#5#6#7#8#9{
%<*trace> 
   \trace_ldb_internal:n{\string\ldb_update_bmls:w}
%</trace>
        \if_meaning:w#1\q_nil\else:
%<*trace>
          \trace_ldb_internal:n{...~execute~\string#1}
          \ldb_show_pointer:N #1
%</trace>
          #1                    % execute values
        \fi:
        \ldb_add_right:nn
           #2
           \l_ldb_new_lbml_tl
       \ldb_add_right:nn
           #3
           \l_ldb_new_sbml_tl
       \ldb_add_right:nn
           #4
           \l_ldb_new_tbml_tl
%    \end{macrocode}
%    We shouldn't find se matches with a begin tag since there must be
%    a |*| between the first and the second part of a spec.
%    \begin{macrocode}
       \ldb_check_nil:nN {se} #5 
       \ldb_add_right:nn
           #6
           \l_ldb_new_leml_tl
%       \ldb_check_nil:nN {tc} #7 
       \ldb_add_right:nn
           #7
           \l_ldb_new_tcml_tl
       \ldb_add_right:nn
           #8
           \l_ldb_new_lcml_tl
       \ldb_add_right:nn
           #9
           \l_ldb_new_scml_tl
  \ldb_gobble:w
}

\cs_new:Npn\ldb_check_nil:nN#1#2
 {
   \if_meaning:w\q_nil#2
   \else:
     \err_latex_bug:n{Not~ nil~ #1:~ \exp_not:N#2~ 
                      containing:~\tl_to_str:N#2}
   \fi:
 }
%    \end{macrocode}
%  \end{macro}
%
%
%
%  \begin{macro}{\ldb_update_end:n}
%    Similar beast for the `end'.
%    \begin{macrocode}
\cs_new:Npn\ldb_update_end:n#1{
  \if_mode_vertical:
%<*trace>
  \trace_ldb:n{^^JLDB:~End~Processing~*****~
                                 #1
                                 \space *****~
                                 (sequenced)~
                                 \file_show_lineno:}
%</trace>
  \else:
%<*trace>
  \trace_ldb:n{^^JLDB:~End~Processing~*****~
                                 #1
                                 \space *****~
                                 \file_show_lineno:}
%</trace>
    \tl_gset_eq:NN\g_ldb_sbml_tl\c_empty_tl
    \tl_gset_eq:NN\g_ldb_seml_tl\c_empty_tl
    \tl_set_eq:NN \l_ldb_scml_tl\c_empty_tl
  \fi:
  \ldb_show_mls:
%    \end{macrocode}
%    In the end tag matches in the current sbml list are no longer
%    valid so we clear them. As a result only new matches involving
%    this end tag are assigned to |\g_ldb_sbml_tl| below.
%    \begin{macrocode}
  \tl_clear:N \l_ldb_new_sbml_tl
%    \end{macrocode}
%    We also initialize |\l_ldb_new_seml_tl| to hold the new se
%    matches with the current end tag.
%    \begin{macrocode}
  \tl_clear:N \l_ldb_new_seml_tl
%    \end{macrocode}
%    New sc are possible so we clear the |\l_ldb_new_scml_tl|.
%    \begin{macrocode}
  \tl_clear:N \l_ldb_new_scml_tl
%    \end{macrocode}
%    
%    \begin{macrocode}
  \tl_set_eq:NN\l_ldb_new_lcml_tl \l_ldb_lcml_tl
%    \end{macrocode}
%    
%    \begin{macrocode}
  \tl_clear:N \l_ldb_new_tcml_tl
%    \end{macrocode}
%    |\ldb_update_end_aux:n| processes a single class and is mapped over
%    the clist.
%    \begin{macrocode}
  \clist_map_inline:nn{#1}{\ldb_update_end_aux:n {##1}}
%    \end{macrocode}
%    Finally we set the new match candidates.
%    \begin{macrocode}
  \ldb_show_new_mls:
  \tl_gset_eq:NN \g_ldb_sbml_tl \l_ldb_new_sbml_tl
  \tl_gset_eq:NN \g_ldb_seml_tl \l_ldb_new_seml_tl
%
  \tl_gset_eq:NN \g_ldb_next_scml_tl \l_ldb_new_scml_tl
  \tl_gset_eq:NN \g_ldb_next_lcml_tl \l_ldb_new_lcml_tl
  \tl_gset_eq:NN \g_ldb_next_tcml_tl \l_ldb_new_tcml_tl
}
%    \end{macrocode}
%  \end{macro}
%
%
%  \begin{macro}{\ldb_update_end_aux:n}
%    Undoc
%    \begin{macrocode}
\cs_new:Npn\ldb_update_end_aux:n#1{
%<*trace> 
   \trace_ldb_internal:n{\string\ldb_update_end_aux:n  \exp_not:n{{#1}}}
%</trace>
%    \end{macrocode}
%     We first try to match the class in |#1| in the LE match list.
%    \begin{macrocode}
%<*trace>
     \show_new_ml:nNn {#1} l{le}
     \tl_set:Nn\l_trace_tl{LEML}
%</trace>
%<*speed>
     \if_meaning:w
        \l_ldb_leml_tl
        \c_empty_tl
     \else:
%</speed>
       \tl_map_inline:Nn
          \l_ldb_leml_tl
          { \ldb_update_end_single:nN {#1} ##1 }
%<*speed>
     \fi:
%</speed>
%<*trace>
     \show_new_ml:nNn {#1} g{se}
     \tl_set:Nn\l_trace_tl{SEML}
%</trace>
%<*speed>
     \if_meaning:w
        \g_ldb_seml_tl
        \c_empty_tl
     \else:
%</speed>
       \tl_map_inline:Nn
          \g_ldb_seml_tl
          { \ldb_update_end_single:nN {#1} ##1 }
%<*speed>
     \fi:
%</speed>
%<*trace>
     \show_new_ml:nNn {#1} l{lc}
     \tl_set:Nn\l_trace_tl{LCML}
%</trace>
%<*speed>
     \if_meaning:w
        \l_ldb_lcml_tl
        \c_empty_tl
     \else:
%</speed>
       \tl_map_inline:Nn
          \l_ldb_lcml_tl
          { \ldb_update_end_single:nN {#1} ##1 }
%<*speed>
     \fi:
%</speed>
%<*trace>
     \show_new_ml:nNn {#1} l{tc}
     \tl_set:Nn\l_trace_tl{TCML}
%</trace>
%<*speed>
     \if_meaning:w
        \l_ldb_tcml_tl
        \c_empty_tl
     \else:
%</speed>
       \tl_map_inline:Nn
          \l_ldb_tcml_tl
          { \ldb_update_end_single:nN {#1} ##1 }
%<*speed>
     \fi:
%</speed>
%<*trace>
     \show_new_ml:nNn {#1} l{sc}
     \tl_set:Nn\l_trace_tl{SCML}
%</trace>
%<*speed>
     \if_meaning:w
        \l_ldb_scml_tl
        \c_empty_tl
     \else:
%</speed>
       \tl_map_inline:Nn
          \l_ldb_scml_tl
          { \ldb_update_end_single:nN {#1} ##1 }
%<*speed>
     \fi:
%</speed>
}
%    \end{macrocode}
%  \end{macro}
%
%
%  \begin{macro}{\ldb_update_end_single:nN}
%    Undoc
%    \begin{macrocode}
\cs_new:Npn\ldb_update_end_single:nN#1#2{
%<*trace>
      \trace_ldb_internal:n{..~looking~for~\token_to_str:N#1~in~
         \l_trace_tl\space
         \token_to_str:N#2~containing^^J^^J
         <<\tl_to_str:N#2>>^^J}
%</trace>
      \prop_get:NnN
        #2
        {#1}
        \l_tmpa_tl
      \if_meaning:w\l_tmpa_tl\q_no_value      % return if not found
%<*trace>
        \trace_ldb_internal:n{...~no~match~found}
%</trace>
      \else:
%<*trace>
        \trace_ldb_internal:n{...~found~\tl_to_str:N\l_tmpa_tl}
%</trace>
        \exp_after:wN\ldb_update_emls:w\l_tmpa_tl
          \q_nil\q_nil\q_nil\q_nil\q_nil\q_nil\q_nil\q_nil\q_stop
      \fi:
}
%    \end{macrocode}
%  \end{macro}
%
%
%  \begin{macro}{\ldb_update_emls:w}
%    Routine that separates the pointers.
%    \begin{macrocode}
\cs_new:Npn\ldb_update_emls:w#1#2#3#4#5#6#7#8#9{
        \if_meaning:w#1\q_nil\else:
%<*trace>
          \trace_ldb:n{...~execute~\string#1}
          \ldb_show_pointer:N #1
%</trace>
          #1                    % execute values
        \fi:
%    \end{macrocode}
%    Some of the pointers have to be |\q_nil| according to our
%    restrictions so we could make everything faster and more compact
%    by providing different data structures both types.
%    \begin{macrocode}
       \ldb_check_nil:nN{lb}#2
       \ldb_add_right:nn
           #3
           \l_ldb_new_sbml_tl
       \ldb_check_nil:nN{tb}#4
       \ldb_add_right:nn
           #5
           \l_ldb_new_seml_tl
       \ldb_check_nil:nN{le}#6
       \ldb_check_nil:nN{tc}#7
       \ldb_check_nil:nN{lc}#8
       \ldb_add_right:nn
           #9
           \l_ldb_new_scml_tl
  \ldb_gobble:w
}
%    \end{macrocode}
%  \end{macro}
%
%
%
%  \begin{macro}{\ldb_add_right:nn}
%    Undoc.
%    \begin{macrocode}
\cs_new:Npn\ldb_add_right:nn#1#2{
%<*trace> 
   \trace_ldb_internal:n{\string\ldb_add_right:nn \exp_not:n{{#1}{#2}}}
%</trace>
  \quark_if_nil:nF{#1}
    {
    \cs_set:Npn\tmp:w
        ##1#1##2\tmp:w
        {\tex_if:D !\exp_not:N##2!
%<*trace>
           \trace_ldb_internal:n{... will~add~\token_to_str:N#1~
                        to~\token_to_str:N#2~containing
                        ^^J^^J<<\tl_to_str:N#2>>^^J}
%</trace>
           \cs_set:Npn#2{##1#1}
         \else:
%<*trace>
           \trace_ldb_internal:n{... will~move~
                         \token_to_str:N#1~to~the~right~in~
                         \token_to_str:N#2~containing
                          ^^J^^J<<\tl_to_str:N#2>>^^J}
%</trace>
           \cs_set:Npn#2{##1##2}
         \fi:
        }
    \exp_after:wN \tmp:w#2#1\tmp:w
   }
}

%    \end{macrocode}
%  \end{macro}
%
%
%  \begin{macro}{\ldb_set_classes:nn}
%    Undoc
%    \begin{macrocode}
\cs_new:Npn\ldb_set_classes:nn#1#2{
  \tl_clear_new:c{l_ldb_cl_#1}
  \for_clist:Nnn\l_tmpa_tl{#2}
    {\tl_put_right:cc{l_ldb_cl_#1}{env_begin_code_ 
                                    \l_tmpa_tl 
                                    :}
    }
}
%    \end{macrocode}
%  \end{macro}
%
%
% \begin{macro}{\ldb_new_context:n}
%    The |\ldb_new_context:n| command will flush the current rdb
%    history and will start fresh with classes given in the argument.
%    \begin{macrocode}
%\cs_new:Npn\ldb_new_context:n{
%  \tl_set_eq:NN\l_ldb_lbml_tl\g_ldb_begin_root_prop
%  \tl_set_eq:NN\l_ldb_tbml_tl\c_empty_tl
%  \tl_gset_eq:NN\g_ldb_sbml_tl\c_empty_tl
%  \ldb_update_begin:n
%}
%    \end{macrocode}
%  \end{macro}
%
%
% \begin{macro}{\l_ldb_lbml_tl}
% \begin{macro}{\l_ldb_leml_tl}
% \begin{macro}{\l_ldb_tbml_tl}
%    Tight and loose match lists are local, they get restored by the
%    grouping mechanism. The initial value for the loose match list is
%    the pointer to the root.
%    \begin{macrocode}
\tl_new:Nn\l_ldb_lbml_tl{\g_ldb_begin_root_prop}
\tl_new:Nn\l_ldb_leml_tl{\g_ldb_end_root_prop}
\tl_new:Nn\l_ldb_tbml_tl{}
%    \end{macrocode}
% \end{macro}
% \end{macro}
% \end{macro}
%
%
% \begin{macro}{\g_ldb_sbml_tl}
% \begin{macro}{\g_ldb_seml_tl}
%    `begin-begin' and `end-begin' match lists are global since their
%    contents is needed between environments. The restoring for them
%    is therefore done explicitly.
%    grouping mechanism.
%    \begin{macrocode}
\tl_new:Nn\g_ldb_sbml_tl{}
\tl_new:Nn\g_ldb_seml_tl{}
%    \end{macrocode}
% \end{macro}
% \end{macro}
%
% \begin{macro}{\g_ldb_next_scml_tl}
% \begin{macro}{\g_ldb_next_lcml_tl}
% \begin{macro}{\g_ldb_next_tcml_tl}
%    More match lists.
%    \begin{macrocode}
\tl_new:Nn\g_ldb_next_scml_tl{}
\tl_new:Nn\g_ldb_next_lcml_tl{}
\tl_new:Nn\g_ldb_next_tcml_tl{}
%    \end{macrocode}
% \end{macro}
% \end{macro}
% \end{macro}
%
%
%  \begin{macro}{\l_ldb_new_lbml_tl}
%  \begin{macro}{\l_ldb_new_leml_tl}
%  \begin{macro}{\l_ldb_new_lceml_tl}
%  \begin{macro}{\l_ldb_new_tbml_tl}
%  \begin{macro}{\l_ldb_new_tcml_tl}
%  \begin{macro}{\l_ldb_new_seml_tl}
%  \begin{macro}{\l_ldb_new_scml_tl}
%    Undoc
%    \begin{macrocode}
\tl_new:Nn\l_ldb_new_lbml_tl{}
\tl_new:Nn\l_ldb_new_leml_tl{}
\tl_new:Nn\l_ldb_new_lcml_tl{}
\tl_new:Nn\l_ldb_new_tbml_tl{}
\tl_new:Nn\l_ldb_new_tcml_tl{}
\tl_new:Nn\l_ldb_new_seml_tl{}
\tl_new:Nn\l_ldb_new_scml_tl{}
%    \end{macrocode}
%  \end{macro}
%  \end{macro}
%  \end{macro}
%  \end{macro}
%  \end{macro}
%  \end{macro}
%  \end{macro}
%
%  \begin{macro}{\l_ldb_scml_tl}
%  \begin{macro}{\l_ldb_lcml_tl}
%  \begin{macro}{\l_ldb_tcml_tl}
%    undoc
%    \begin{macrocode}
\tl_new:Nn\l_ldb_scml_tl{}
\tl_new:Nn\l_ldb_lcml_tl{}
\tl_new:Nn\l_ldb_tcml_tl{}
%    \end{macrocode}
%  \end{macro}
%  \end{macro}
%  \end{macro}
%
%
% \subsection{Tracing the database operations}
%
% The code in this section is included only, if we are tracing the
% mechanism. To keep the other code compact most tracing stuff is packed
% in separate macros.
%
%    \begin{macrocode}
%<*trace>
%    \end{macrocode}
%
% \begin{macro}{\trace_ldb_internal:n}
% \begin{macro}{\trace_ldb_internal:n}
% \begin{macro}{\g_trace_ldb_status}
%    |\trace_ldb_internal:n| shows some internal infos if requested.
%    \begin{macrocode}
\cs_new:Npn\trace_ldb:n#1{
  \if_num:w\g_trace_ldb_status>\c_zero
    \iow_term:x{LDB:~ #1}
  \fi:}
\cs_new:Npn\trace_ldb_internal:n#1{
  \if_num:w\g_trace_ldb_status=\c_two
    \iow_term:x{LDB:~#1}
  \fi:}
\int_new:N\g_trace_ldb_status
%    \end{macrocode}
%    Default is to show a bit.
%    \begin{macrocode}
\int_gincr:N\g_trace_ldb_status
%    \end{macrocode}
% \end{macro}
% \end{macro}
% \end{macro}
%
% \begin{macro}{\show_ldb_entry:nnn}
%    |\show_ldb_entry:nnn| will take the arguments of |\ldb_enter:nnn| and
%    displays them on the terminal.
%    \begin{macrocode}
\cs_new:Npn\show_ldb_entry:nnn#1#2#3{
   \tl_gset:Nn\g_tmpa_tl{#3}
   \trace_ldb:n{Set~\token_to_str:N#2\space~with~value~
            \tl_to_str:N\g_tmpa_tl
            \space in~#1}
}
%    \end{macrocode}
% \end{macro}
%
% \begin{macro}{\show_new_ml:nNn}
%    Undoc
%    \begin{macrocode}
\cs_new:Npn\show_new_ml:nNn#1#2#3{
  \trace_ldb_internal:n{Updating~parms~with~class~`#1'~
             using~#3~list:~
             `\tl_to_str:c{#2_ldb_#3ml_tl}'
         }
}
%</trace>
%    \end{macrocode}
% \end{macro}
%
% \begin{macro}{\show_def_update:Nn}
%    |\show_def_update:Nn| will show the update of a parameter via
%    |\cs_set:Npn| by displaying first the old then the new value. To
%    keep the number of tokens produced in precompiling low, the
%    actual assignment is also carried out in this macro.
%    \begin{macrocode}
\cs_new:Npn\show_def_update:Nnn#1#2#3{
%<*trace>
   \tl_gset:Nn\g_tmpa_tl{#2}
   \trace_ldb:n{ .....~ Change~ \token_to_str:N#1  \space because~ of~ `#3':}
   \trace_ldb:n{ ..........~
       \exp_not:o{#1}~ ->~ \tl_to_str:N\g_tmpa_tl
               }
%</trace>
   \cs_set:Npn#1{#2}
}
%    \end{macrocode}
% \end{macro}
%
% \begin{macro}{\show_register_update:Nn}
%    Same game for normal parameters (counts, skips, etc.).
%    \begin{macrocode}
\cs_new:Npn\show_register_update:Nnn#1#2#3{
%<*trace>
   \tl_gset:Nn\g_tmpa_tl{#2}
   \trace_ldb:n{.....~ Change~ \token_to_str:N#1 \space because~ of~ `#3':}
   \trace_ldb:n{ ..........~
      \tex_the:D#1~ ->~
      \tl_to_str:N\g_tmpa_tl 
      }
%</trace>
   #1#2\scan_stop:
}
%    \end{macrocode}
% \end{macro}
%
% \begin{macro}{\ldb_show_pointer:N}
%    Show the contents of what a pointer is pointing to.
%    \begin{macrocode}
\cs_new:Npn\ldb_show_pointer:N#1
  {
   \trace_ldb:n{Value~ of~
         \token_to_str:N#1~containing^^J
         <<\tl_to_str:N#1>>}
  }
%    \end{macrocode}
% \end{macro}
%
% \begin{macro}{\ldb_show_mls:}
%    Show the matching lists.
%    \begin{macrocode}
\cs_new:Npn\ldb_show_mls:
  {
   \ldb_show_pointer:N\l_ldb_lcml_tl
   \ldb_show_pointer:N\l_ldb_tcml_tl
   \ldb_show_pointer:N\l_ldb_scml_tl
   \ldb_show_pointer:N\l_ldb_leml_tl
   \ldb_show_pointer:N\l_ldb_lbml_tl
   \ldb_show_pointer:N\l_ldb_tbml_tl
   \ldb_show_pointer:N\g_ldb_sbml_tl
   \ldb_show_pointer:N\g_ldb_seml_tl
  }
%    \end{macrocode}
% \end{macro}
%
% \begin{macro}{\ldb_show_new_mls:}
%    Show the new matching lists.
%    \begin{macrocode}
\cs_new:Npn\ldb_show_new_mls:
  {
   \ldb_show_pointer:N\l_ldb_new_lcml_tl
   \ldb_show_pointer:N\l_ldb_new_tcml_tl
   \ldb_show_pointer:N\l_ldb_new_scml_tl
   \ldb_show_pointer:N\l_ldb_new_leml_tl
   \ldb_show_pointer:N\l_ldb_new_lbml_tl
   \ldb_show_pointer:N\l_ldb_new_tbml_tl
   \ldb_show_pointer:N\l_ldb_new_sbml_tl
   \ldb_show_pointer:N\l_ldb_new_seml_tl
  }
%    \end{macrocode}
% \end{macro}
%
% \begin{macro}{\nopointer}
%    A function to switch off the showing of pointers.
%    \begin{macrocode}
\cs_new:Npn\nopointer{\cs_set:Npn\ldb_show_pointer:N##1{}}
%    \end{macrocode}
% \end{macro}
%



% \subsection{Tracing modules}
%
%  \begin{macro}{\traceon}
%  \begin{macro}{\traceoff}
%    Turning the tracing of modules on or off. (primitive version).
%    \begin{macrocode}
%<*trace>
\cs_new:Npn\traceon#1{\clist_map_inline:nn{#1}
         {
          \cs_if_free:cF{g_trace_ ##1 _status}
            {\int_gincr:c{g_trace_ ##1 _status}}
         }}
\cs_new:Npn\traceoff#1{\clist_map_inline:nn{#1}
         {
          \cs_if_free:cF{g_trace_ ##1 _status}
            {\int_gdecr:c{g_trace_ ##1 _status}}
         }}
%</trace>
%<-trace>\let_new:NN\traceon\use_none:n
%<-trace>\let_new:NN\traceoff\use_none:n
%    \end{macrocode}
%  \end{macro}
%  \end{macro}
%






%    \begin{macrocode}
%</package>
%    \end{macrocode}
%
%

\endinput

%
% \subsection{Functions}
% 
% \begin{function}{\ldb_enter:nnn}
% \begin{syntax}
%   "\ldb_enter:nnn" "{"<class list>"}" "{"<parm>"}" "{"<value>"}"
% \end{syntax}
% This function is used to enter values for parameters in a given
% context. The context is described by <class list> which is a list of
% class names (character strings) separated by either "." (tight
% binding), "*" (loose binding), "!" (direct nested "\begin"), or
% {\tt\string<} (sequence of "\end" followed by "\begin". From left to
% right the classes represent the order of nesting to which the
% parameter change should apply, that is the left-most class is the one
% that was entered last (Denys syntax). To the left and the right there
% is an implicit "*" assumed which means that the parameter change will
% apply until overwritten by a more specific entry (left star) and also
% will apply the first time a specified <class list> matches independent
% of the fact where in the stack of classes this happens.
% 
% <parm> can be either a \TeX{} parameter like a <counter> or a <tlp>,
% e.g.\ a function with no arguments. This isn't a real restriction
% since such a token list pointer can of course then held a function
% with arbitrary arguments as its <value>.
% At the moment <parm>s that are <tlp> have to be declared but this can
% be avoided later.
% 
% <value> might be any value which is sensible for the corresponding
% <parm>.
% 
%  All assignments done by this function are local so that they restore
% themselves when a group (e.g.\ an environment) ends.
% \end{function}
% 
% \begin{function}{\ldb_decl_defparm:N}
% \begin{syntax}
%  "\ldb_decl_defparm:N" <parm>
% \end{syntax}
% <parm> is declared as a parameter that gets its value assigned via a
% definition routine of \TeX{}, i.e., is not one of \TeX{}'s internal
% registers. This is a temporary funtion since the type of parameter can
% be determined by \LaTeX3 itself.
% \end{function}
% 
% \begin{function}{\ldb_set_classes:nn}
% \begin{syntax}
% 
% \end{syntax}
% \end{function}
% 
% \begin{function}{}
% \begin{syntax}
% 
% \end{syntax}
% \end{function}
% 
% \subsection{Internal functions}
% 
% Not described, probably of no interest to the implementor of higher
% level modules.
% 
% \subsection{Variables}
% 
% \begin{variable}{\l_ldb_lml_tl}
% Current loose match list holding possible candiates for loose matches
% with further nested environments.  When a new environment <env> starts
% the LDB manager looks for loose matches in this list. If a match is
% found the corresponding pointers are handled in the following way:
% \begin{itemize}
% \item
%   The loose pointer if not nil is added (or moved if already present)
% to the right in "\l_ldb_curr_lml_tl".
% \item
%   The tight pointer if not nil is added (or moved if already present)
% to the right in "\l_ldb_curr_tml_tl".
% \item
%   The begin-begin pointer if not nil is added (or moved if already
% present) to the right in "\l_ldb_curr_bbml_tl".
% \item
%   The end-begin pointer if not nil is added (or moved if already
% present) to the right in "\l_ldb_curr_ebml_tl".
% \item
%   The assign-begin pointer if not nil is evaluated resulting in local
% assignments to a set of parameters.
% \end{itemize}
% When the LDB manager has finished looking for matches (which involves
% also looking for tight, etc. matches) "\l_ldb_lml_tl" gets assigned
% the contents of "\l_ldb_curr_lml_tl". This means that the contents of
% the loose match list will be restored when the environment ends.
% \end{variable}
% 
% \begin{variable}{\l_ldb_tml_tl}
% Current tight match list holding possible candidates for tight
% matches. When a new environment starts this list is checked after
% checking the loose match list. Found matches are handled in the same
% way as above.  When the LDB manager has finished looking for matches
% "\l_ldb_tml_tl" gets assigned the contents of "\l_ldb_curr_tml_tl".
% This means that the contents of the tight match list will be restored
% when the environment ends.
% \end{variable}
% 
% \begin{variable}{\g_ldb_bbml_tl}
% Current begin-begin match list holding possible candidates for
% begin-begin matches. When a new environment starts this list is
% checked after checking the tight match list. Found matches are handled
% in the same way as above.  When the LDB manager has finished looking
% for matches "\g_ldb_bbml_tl" gets globally assigned the contents of
% "\l_ldb_curr_bbml_tl".  When some environment ends this list is
% globally cleared.
% \end{variable}
% 
% \begin{variable}{\g_ldb_ebml_tl}
% Current begin-end match list holding possible candidates for begin-end
% matches. When a new environment starts this list is checked if the
% begin-begin match list is empty (only one of both can possibly contain
% candidates). Found matches are handled in the same way as above.  When
% the LDB manager has finished looking for matches, "\g_ldb_ebml_tl"
% gets globally cleared to avoid spurious matches within this
% environment. When the end tag of this environment is reached the the
% "\g_ldb_ebml_tl" is checked again for match but this time only
% pointers to begin-end are processed (i.e. added or moved to the right
% in "\l_ldb_curr_ebml_tl" (which already holds the matches found at
% the beginning of the environment)). Then "\g_ldb_ebml_tl" is assigned
% globally the contents of "\l_ldb_curr_ebml_tl". The reason for
% checking twice is that the matches found at the beginning are those
% that come from LDB entries like ``"env1"\string<"env2*env3"'' where
% "env2" is actually a full environment (begin and end tag). The check
% at the end then captures entries like
% ``"env1"\string<"env2"\string<"env3"'' where "env2" stands for an end
% tag that is directly preceded by the end tag of "env3".
% \begin{quote}
%    (Deep breath.) You got that? [\TeX book p.~125]
% \end{quote}
% \end{variable}
% 
% \begin{variable}{\l_ldb_curr_lml_tl}
% Temp variable holding found loose matches (candiates for the next
% environment). Since older loose match candidates are also candidates
% for further nested environments this variable gets assigned the
% contents of "\l_ldb_lml_tl" when the LDB starts processing a begin tag.
% \end{variable}
% 
% \begin{variable}{\l_ldb_curr_tml_tl}
% Temp variable holding found tight matches (candiates for the next
% environment). Since older tight match candidates are no longer
% candidates for further nested environments this variable is cleared
% when the LDB starts processing a begin tag.
% \end{variable}
% 
% \begin{variable}{\l_ldb_curr_bbml_tl}
% Temp variable holding found begin-begin matches (candiates for the
% next environment). Since older match candidates are no longer
% candidates for further nested environments this variable is cleared
% when the LDB starts processing a begin tag.
% \end{variable}
% 
% \begin{variable}{\l_ldb_curr_ebml_tl}
% Temp variable holding found begin-end matches (candiates for the next
% environment). Since older match candidates are no longer candidates
% for further nested environments this variable is cleared when the LDB
% starts processing a begin tag. The values of this variable is actually
% used in the end tag where it is further updated (see
% "\g_ldb_ebml_tl"). Then its contents is assigned globally to
% "\g_ldb_ebml_tl" so that it survives the closing group of the end
% tag.
% \end{variable}
% 
% To interrupt sequencing information the begin of a paragraph
% (normally) clears "\g_ldb_bbml_tl"  the end (normally) clears
% "\g_ldb_ebml_tl"  and "\l_ldb_curr_ebml_tl".
% 
% \subsection{Tracing infos}
% 
% The LDB mechanism is  traced via "rdb".
% This time two levels of tracing information are provided.
% 
% Turning on "rdb" once will trace:
% \begin{itemize}
% \item  LDB lookup for begin and end tags,
% \item the changing of parameter values due to LDB queries.
% \end{itemize}
% 
% Turning on "rdb" twice will additionally trace:
% \begin{itemize}
% \item adding new entries to the LDB data structure,
% \item the internal mechanism of the LDB, e.g., changes of match lists
% (moving or adding items to a match list) , the search for matches, etc.
% \end{itemize}
% 
% \subsubsection{Functions}
% 
% \begin{function}{\trace_rdb:n}
% \begin{syntax}
%   "\trace_rdb:n" "{" <text> "}"
% \end{syntax}
% \end{function}
% 
% \begin{function}{\trace_ldb_internal:n}
% \begin{syntax}
%   "\trace_ldb_internal:n" "{" <text> "}"
% \end{syntax}
% \end{function}
% 
% 
