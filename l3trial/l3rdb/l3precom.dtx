

%  \section{Precompiling some functions}
%
%    \begin{macrocode}
\tex_message:D{LaTeX:~ precompiling~ code,}
%    \end{macrocode}
%
%    It might speed up the processing of documents when certain parts
%    of the document style file are `precompiled' and stored in a
%    separate file.
%
%  \begin{macro}{\c_cs_dump_stream}
%    We need to allocate an output stream in order to be able to write
%    the precompiled code out.
%    Stream number for the dump.
%
%    \begin{macrocode}
%<*precompile>
%\io_write_new:N\c_cs_dump_stream
%    \end{macrocode}
%  \end{macro}
%
%
%  \begin{macro}{\g_cs_dump_name_tl}
%    This \m{tl} is used to store the name of the file.
%
%    \begin{macrocode}
\tl_new:Nn\g_cs_dump_name_tl{}
%    \end{macrocode}
%  \end{macro}
%
%  \begin{macro}{\g_cs_dump_seq}
%    While processing the documentstyle we build up a list of
%    control sequence names to be dumped. For this purpose
%    we use the |\g_cs_dump_seq| sequence.
%
%    \begin{macrocode}
\seq_new:N\g_cs_dump_seq
%    \end{macrocode}
%  \end{macro}
%
%  \begin{macro}{\cs_record_name:N}
%  \begin{macro}{\cs_record_name:c}
%    These functions mark a control sequence for dumping into a
%    precompiled style. 
%
%    When the {\sf trace} `module' is included in the code we also write
%    information about the control sequence into a |.dmp| file.
%    \begin{macrocode}
\cs_new:Npn\cs_record_name:N#1{
%<*trace>
      \seq_gput_left:Nn
          \g_cs_trace_seq#1
%</trace>
      \seq_gput_left:Nn
          \g_cs_dump_seq#1}
\cs_new:Npn\cs_record_name:c{\exp_args:Nc\cs_record_name:N}
%    \end{macrocode}
%  \end{macro}
%  \end{macro}
%
%
% \begin{macro}{\cs_dump:}
%    When a document style calls |\cs_dump:| it triggers this code
%    to write all the precompilation information out to a file.
%
%    Before dumping, we write a message to the terminal informing the
%    `user' of this fact.
%    \begin{macrocode}
\cs_new:Npn\cs_dump:{
  \typeout{Precompiling~style~into~(\g_cs_dump_name_tl)}
  \iow_open:Nn\c_cs_dump_stream{\g_cs_dump_name_tl}
%    \end{macrocode}
%    The first thing we write on a `dump' file is a command that
%    allows us to use |*| in control sequences. We also need to be 
%    able to write to (and read from) the file internal control
%    sequences, containing |_| and |:|.
%    \begin{macrocode}
  \iow_now:Nx\c_cs_dump_stream
    {\group_begin:
     \char_catcode_array:D`\token_to_str:N\*=11\scan_stop:
     \token_to_str:N\style_names_start:
    }
  \seq_map_inline:Nn
       \g_cs_dump_seq
       {\tex_message:D{.}
        \iow_now:Nx\c_cs_dump_stream
%    \end{macrocode}
%    We use a direct |\gdef:Npn| to disable any type of local/global
%    check on the pointers.
%    \begin{macrocode}
             {\token_to_str:N\gdef:Npn
                 \token_to_str:N##1
                 {\tl_to_str:N##1}}
       }
%    \end{macrocode}
  \iow_now:Nx
     \c_cs_dump_stream
     {\group_end:}
  \iow_close:N\c_cs_dump_stream
  \tex_message:D{~finished}
}
%</precompile>
%    \end{macrocode}
% \end{macro}
%
%  \begin{macro}{\cs_load_dump:n}
%    A function to read a precompiled file into memory and skip until a
%    |\cs_dump:| command is found. If no such file is found, processing
%    continues and a subsequent |\cs_dump:| command will then create the
%    dump file.
%    \begin{macrocode}
\cs_new:Npn\cs_load_dump:n#1{
  \file_not_found:nTF{#1.cmp}
%<*precompile>
      {\tl_gset:Nn\g_cs_dump_name_tl{#1.cmp}}
%</precompile>
%<-precompile> {\err_tex_message:D{Cannot~ dump~ with~ this~ format}}
      {\input{#1.cmp}
       \cs_set_eq:NN\cs_dump:\fi:
       \if_false:}}
%    \end{macrocode}
%  \end{macro}
%
% \begin{macro}{\g_gen_sym_int}
% \begin{macro}{\g_ggen_sym_int}
%    Two counters to make up new local or global {\sl short\/} names
%    in pointer structures like the {\sc ldb}.
%    We use a fake counters since operations with them are seldom.
%    \begin{macrocode}
\int_new:N   \g_gen_sym_int
\int_gset:Nn \g_gen_sym_int{0}
\int_new:N   \g_ggen_sym_int
\int_gset:Nn \g_ggen_sym_int{0}
%    \end{macrocode}
% \end{macro}
% \end{macro}
%
% \begin{macro}{\cs_gen_sym:N}
% \begin{macro}{\cs_ggen_sym:N}
%    We need to be able to generate control sequences on the fly.
%    They will exist of a prefix, either |l*| or |g*|, followed
%    by the value of the counter |\g_gen_sym_int| (|\g_ggen_sym_int|)
%    in roman numeral representation. The generated control sequence
%    is locally stored in the token that was passed in |#1|.
%
%    \begin{macrocode}
\cs_new:Npn\cs_gen_sym:N#1{
  \int_gincr:N\g_gen_sym_int
  \tl_set:No#1{\cs:w l*\int_to_roman:n \g_gen_sym_int \cs_end:}
%<*precompile>
  \exp_after:wN\cs_record_name:N#1
%</precompile>
%    \end{macrocode}
%    We still want to define the initial value for the new symbol
%    globally to make sure that during compilation something is
%    written to the output file.
%    \begin{macrocode}
  \exp_after:wN\tl_clear_new:N#1}
%    \end{macrocode}
%    The global variant
%    \begin{macrocode}
\cs_new:Npn\cs_ggen_sym:N#1{
%<*trace> 
   \trace_ldb_internal:n{\string\cs_ggen_sym:N  \exp_not:n{{#1}}}
%</trace>
  \int_gincr:N\g_ggen_sym_int
  \tl_set:No#1{\cs:w g*\int_to_roman:w \g_ggen_sym_int \cs_end:}
%<*trace>
   \trace_ldb_internal:n{\token_to_meaning:N #1 ~ ...~ allocated}
%</trace>
%<*precompile>
  \exp_after:wN\cs_record_name:N#1
%</precompile>
  \exp_after:wN\tl_clear_new:N#1}


% FMi we really need to allocate property lists these days for nodes

\cs_new:Npn\cs_ggen_prop:N#1{
  \cs_ggen_sym:N #1
  \exp_after:wN \prop_clear:N #1
}
%    \end{macrocode}
% \end{macro}
% \end{macro}
%
%
%  \begin{macro}{\g_cs_trace_seq}
%    A sequence which holds the control sequence names that are to 
%    be dumped. They are stored together with their meaning.
%
%    \begin{macrocode}
%<*trace>
\seq_new:N\g_cs_trace_seq
%    \end{macrocode}
%  \end{macro}
%
%  \begin{macro}{\g_register_trace_seq}
%    Sequence holding the register names to be dumped with their
%    corresponding values.
%    \begin{macrocode}
\seq_new:N\g_register_trace_seq
%    \end{macrocode}
%  \end{macro}
%
%  \begin{macro}{\cs_record_meaning:N}
%    Function marking a control sequence for dumping with meaning.
%    \begin{macrocode}
\cs_set:Npn\cs_record_meaning:N#1{
      \seq_gput_left:Nn
          \g_cs_trace_seq#1}
%    \end{macrocode}
%  \end{macro}
%
%  \begin{macro}{\register_record_name:N}
%    Function marking a register for dumping with value.
%    \begin{macrocode}
\cs_set:Npn\register_record_name:N#1{
      \seq_gput_left:Nn
          \g_register_trace_seq#1}
%    \end{macrocode}
%  \end{macro}
%
%
% \begin{macro}{\dumpLaTeXstate}
%    The function |\dumpLaTeXstate| is used to write control sequences
%    and registers, together with their meaning or value in the |.dmp|
%    file. We write informational messages to the terminal during the 
%    dump.
%    \begin{macrocode}
\cs_new:Npn\dumpLaTeXstate#1{
  \typeout{Dumping~commands~into~(#1.dmp)}
  \iow_open:Nn\c_cs_dump_stream{#1.dmp}
  \tex_message:D{~commands}
  \seq_map_inline:Nn
       \g_cs_trace_seq
       {\tex_message:D{.}
        \iow_now:Nx\c_cs_dump_stream
             {\token_to_str:N##1~
              \token_to_meaning:N##1}
       }
  \tex_message:D{~registers}
  \seq_map_inline:Nn
       \g_register_trace_seq
       {\tex_message:D{.}
        \iow_now:Nx\c_cs_dump_stream
             {\token_to_str:N##1
              \tex_the:D##1}
       }
  \tex_message:D{~finished}
  \iow_close:N\c_cs_dump_stream
}
%</trace>
%    \end{macrocode}
%  \end{macro}


