% \iffalse
%
%% File l3fp-extras.dtx (C) Copyright 2012 The LaTeX3 Project
%%
%% It may be distributed and/or modified under the conditions of the
%% LaTeX Project Public License (LPPL), either version 1.3c of this
%% license or (at your option) any later version.  The latest version
%% of this license is in the file
%%
%%    http://www.latex-project.org/lppl.txt
%%
%% This file is part of the "l3trial bundle" (The Work in LPPL)
%% and all files in that bundle must be distributed together.
%%
%% The released version of this bundle is available from CTAN.
%%
%% -----------------------------------------------------------------------
%%
%% The development version of the bundle can be found at
%%
%%    http://www.latex-project.org/svnroot/experimental/trunk/
%%
%% for those people who are interested.
%%
%%%%%%%%%%%
%% NOTE: %%
%%%%%%%%%%%
%%
%%   Snapshots taken from the repository represent work in progress and may
%%   not work or may contain conflicting material!  We therefore ask
%%   people _not_ to put them into distributions, archives, etc. without
%%   prior consultation with the LaTeX Project Team.
%%
%% -----------------------------------------------------------------------
%%
%
%<*driver|package>
\RequirePackage{expl3}
\GetIdInfo$Id$
  {L3 Experimental additions to l3fp}
%</driver|package>
%<*driver>
\documentclass[full]{l3doc}
\usepackage{amsmath}
\begin{document}
  \DocInput{\jobname.dtx}
\end{document}
%</driver>
% \fi
%
% \title{^^A
%   The \pkg{l3fp-extras} package\\ Add ons to \pkg{l3fp}^^A
%   \thanks{This file describes v\ExplFileVersion,
%      last revised \ExplFileDate.}^^A
% }
%
% \author{^^A
%  The \LaTeX3 Project\thanks
%    {^^A
%      E-mail:
%        \href{mailto:latex-team@latex-project.org}
%          {latex-team@latex-project.org}^^A
%    }^^A
% }
%
% \date{Released \ExplFileDate}
%
% \maketitle
%
% \begin{documentation}
%
% \section{\pkg{l3fp-extras} documentation}
%
% Additions to \pkg{l3fp}.
%
% \subsection{Symbolic expressions}
%
% At the moment, \pkg{l3fp-symbolic} redefines a few \pkg{l3fp}
% internals to introduce the ability to use unbound variables in
% floating point expressions.
% \begin{verbatim}
%   \fp_set:Nn \l_test_fp { 1 + A * sin(B) }
%   \fp_show:N \l_test_fp
% \end{verbatim}
% gives |(1+(A*sin(B)))|.  One can assign a value to a variable with
% \cs{fp_var_set:nn}.  For example,
% \begin{verbatim}
%   \fp_var_set:nn { B } { pi / 2 }
%   \fp_show:N \l_test_fp
%   \fp_show:n { \l_test_fp }
% \end{verbatim}
% shows |(1+(A*sin(B)))|, since |\l_test_fp| has not changed, then
% |(1+(A*1))| because \cs{fp_show:n} forces the re-evaluation of
% \cs{l_test_fp}, and \pkg{l3fp} finds out that the sine function can be
% evaluated now that its argument has a definite value.  Note that |A*1|
% is not simplified to |A|: \cs{fp_eval:n} only attempts to evaluate
% computations whose operands are all known.
%
% \end{documentation}
%
% \begin{implementation}
%
% \section{\pkg{l3fp-extras} implementation}
%
%    \begin{macrocode}
%<*package>
%    \end{macrocode}
%
%    \begin{macrocode}
\ProvidesExplPackage
  {\ExplFileName}{\ExplFileDate}{\ExplFileVersion}{\ExplFileDescription}
%    \end{macrocode}
%
%    \begin{macrocode}
%</package>
%    \end{macrocode}
%
% \end{implementation}
%
% \PrintIndex
