% \iffalse
%
%% File l3check.dtx (C) Copyright 2012,2014-2017 The LaTeX3 Project
%%
%% It may be distributed and/or modified under the conditions of the
%% LaTeX Project Public License (LPPL), either version 1.3c of this
%% license or (at your option) any later version.  The latest version
%% of this license is in the file
%%
%%    http://www.latex-project.org/lppl.txt
%%
%% This file is part of the "l3trial bundle" (The Work in LPPL)
%% and all files in that bundle must be distributed together.
%%
%% The released version of this bundle is available from CTAN.
%%
%% -----------------------------------------------------------------------
%%
%% The development version of the bundle can be found at
%%
%%    http://www.latex-project.org/svnroot/experimental/trunk/
%%
%% for those people who are interested.
%%
%%%%%%%%%%%
%% NOTE: %%
%%%%%%%%%%%
%%
%%   Snapshots taken from the repository represent work in progress and may
%%   not work or may contain conflicting material!  We therefore ask
%%   people _not_ to put them into distributions, archives, etc. without
%%   prior consultation with the LaTeX Project Team.
%%
%% -----------------------------------------------------------------------
%%
%
%<*driver|package>
\RequirePackage{expl3}
%</driver|package>
%<*driver>
\documentclass[full]{l3doc}
\usepackage{amsmath}
\begin{document}
  \DocInput{\jobname.dtx}
\end{document}
%</driver>
% \fi
%
% \title{^^A
%   The \pkg{l3check} package\\ Checking and debugging \LaTeX3 code^^A
% }
%
% \author{^^A
%  The \LaTeX3 Project\thanks
%    {^^A
%      E-mail:
%        \href{mailto:latex-team@latex-project.org}
%          {latex-team@latex-project.org}^^A
%    }^^A
% }
%
% \date{Released 2017/04/01}
%
% \maketitle
%
% \begin{documentation}
%
% \section{Checks made in \texttt{l3kernel}}
%
% The \pkg{l3check} package redefines some of the \pkg{expl3} internals to
% look for potential errors in the code, such as calling a sequence function
% with a token list argument.  Those checks come with a
% performance penalty and are thus disabled by default; they are
% activated using \cs{check_on:n}.
%
% The (incomplete) list of checks that are performed is as follows:
% \begin{itemize}
%   \item \pkg{l3expan} |N|-type and |V|-type arguments must be single
%     tokens; |o| and |f|-type expansion should not be called on
%     arguments that do not expand.
%   \item \pkg{l3tl} when a \enquote{tl~var} argument is expected, it
%     is checked to be a non-long parameterless macro.
% \end{itemize}
%
% \section{Turning checks on and off}
%
% \begin{function}{\check_on:n}
%   \begin{syntax}
%     \cs{check_on:n} \Arg{modules}
%   \end{syntax}
%   Turns checking on globally for functions in each of the
%   \meta{modules}, given as a comma list.  This comes with a large
%   performance penalty, thus should only be used for testing purposes.
%   At present, the following modules are supported: |exp|, |tl|.  The
%   special module \texttt{l3kernel} is recognized and turns on
%   checking for all modules in the \pkg{l3kernel} bundle.  Checks are
%   turned on locally.
% \end{function}
%
% \begin{function}{\check_off:n}
%   \begin{syntax}
%     \cs{check_off:n} \Arg{module}
%   \end{syntax}
%   Turns checking off globally for functions in each of the
%   \meta{modules}, given as a comma list.  Note that this function
%   simply silences error messages, hence the performance penalty of
%   \cs{check_on:n} remains.  Checks are turned off locally.
% \end{function}
%
% \section{Defining checks}
%
% \begin{function}{\check_new:nn}
%   \begin{syntax}
%     \cs{check_new:nn} \Arg{module} \Arg{code}
%   \end{syntax}
%   Defines the action of \cs{check_on:n} \Arg{module} to be the
%   \meta{code}.  Note that |#| does not need to be doubled within
%   the \meta{code}.
% \end{function}
%
% \begin{function}{\check_patch:Nn, \check_patch:cn}
%   \begin{syntax}
%     \cs{check_patch:Nn} \meta{function} \Arg{replacement text}
%   \end{syntax}
%   Redefine the \meta{function}.  The \meta{function} will take
%   parameters according to its signature, which must consist
%   exclusively of |N| or |n|, and expand to the \meta{replacement
%   text}, followed by a copy of the original function.  For instance,
%   \begin{quote}
%     \cs{check_patch:Nn} \cs{tl_set:Nn} |{| \cs{check_defined:N} |#1| |}|
%   \end{quote}
%   redefines \cs{tl_set:Nn} to check that its first argument exists
%   before performing its initial duty.  Since the letter in the
%   signature of \cs{tl_set:Nn} is |N|, the code automatically checks
%   that indeed the first argument given is a single token.  When
%   patching an expandable \meta{function}, the \meta{replacement text}
%   must be expandable and expand to nothing, as it otherwise
%   interferes with the expandable \meta{function}.
% \end{function}
%
% \begin{function}{\check_patch:Npnn, \check_patch:cpnn}
%   \begin{syntax}
%     \cs{check_patch:Npnn} \meta{function} \meta{parameters_1} \Arg{parameters_2} \Arg{replacement text}
%   \end{syntax}
%   Redefine the \meta{function}.  The \meta{function} will take the
%   \meta{parameters} and expand to the \meta{replacement text},
%   followed by a copy of the original function followed by the
%   \meta{parameters_2}, which should normally be identical to the
%   \meta{parameters_1}, but with parameters |#1|, |#2|, \ldots{}
%   wrapped in braces (except for single-token parameters |N| and |V|).
%   For instance,
%   \begin{quote}
%     \cs{check_patch:Npnn} \cs{tl_set:Nn} |#1| |#2| |{ #1 {#2} }| \\
%       |{| \\
%         \cs{check_is_N:n} |{| |#1| |}| \\
%         \cs{check_defined:N} |#1| \\
%       |}|
%   \end{quote}
%   redefines \cs{tl_set:Nn} to check that its first argument is a
%   single token and that it exists, before performing its initial
%   duty.  Contrarily to \cs{check_patch:Nn}, the check that |#1| is a
%   single token must be put here by hand.  When patching an expandable
%   \meta{function}, the \meta{replacement text} must be expandable and
%   expand to nothing, as it otherwise interferes with the expandable
%   \meta{function}.
% \end{function}
%
% \section{Helpers for defining checks}
%
% \begin{function}{\check_is_defined:N}
%   ^^A todo: complete this section
% \end{function}
%
% \section{Internal functions}
%
% \begin{function}[EXP, pTF]{\@@_if_on:n}
%   \begin{syntax}
%     \cs{@@_if_on:nTF} \Arg{module} \Arg{true code} \Arg{false code}
%     \cs{@@_if_on_p:n} \Arg{module}
%   \end{syntax}
%   Returns \texttt{true} if the \meta{module}'s checks are currently
%   on, \texttt{false} otherwise, including if the \meta{module} does
%   not exist or has no checks.  This is used by \cs{check_patch:Nn}
%   and other functions in the \meta{module}'s checking code (defined
%   with \cs{check_new:nn}), to determine whether to raise errors or
%   not.
% \end{function}
%
% ^^A todo: add more errors to \cs{@@_split_name:Nnnn}.
% \begin{macro}[int, EXP]{\@@_split_name:Nnnn}
%   \begin{syntax}
%     \cs{@@_split_name:Nnnn} \meta{control sequence}
%       \Arg{code for function}
%       \Arg{code for variable}
%       \Arg{code for other}
%   \end{syntax}
%   Expands to one of the three \meta{code}, followed by brace groups
%   containing different informations depending on the case.  There are
%   many cases, checked in the following order until one matches.
%   \begin{itemize}
%     \item If the name contains a colon and is not |\:|, it is an
%       \pkg{expl3} function, and the \meta{signature} is what follows
%       the last colon.  Then \meta{code for function} \Arg{signature}
%       is left in the input stream.
%     \item If the name contains an underscore, and is not |\_|, it is
%       an \pkg{expl3} variable.  Its first character must be a letter
%       among |clgqs|, and the next |_|.  If the first character is |q|
%       or |s| (quark or scan mark), the \meta{type} is empty.
%       Otherwise the \meta{type} is the part after the last |_|.  Then
%       \meta{code for variable} \Arg{scope} \Arg{type} is left in the
%       input stream.
%     \item Otherwise the control sequence is not known to \pkg{expl3},
%       and \meta{code for other} is left in the input stream.
%   \end{itemize}
% \end{macro}
%
% \end{documentation}
%
% \begin{implementation}
%
% \section{\pkg{l3check} implementation}
%
%    \begin{macrocode}
%<*initex|package>
%    \end{macrocode}
%
%    \begin{macrocode}
%<@@=check>
%    \end{macrocode}
%
%    \begin{macrocode}
%<*package>
\ProvidesExplPackage{l3check}{2017/04/01}{}
  {L3 Experimental checking and debugging}
%</package>
%    \end{macrocode}
%
% \subsection{Infrastructure}
%
% \subsubsection{Variables, variants and copied primitives}
%
% \begin{macro}[aux]{\@@_tmp:w}
%   Used for defining other commands.
%    \begin{macrocode}
\cs_new_eq:NN \@@_tmp:w ?
%    \end{macrocode}
% \end{macro}
%
% \begin{macro}[aux]
%   {
%     \@@_old_exp_args:Nc, \@@_old_exp_args:NNc, \@@_old_exp_args:Ncc,
%     \@@_old_cs_if_exist:NTF, \@@_old_cs_if_exist:NT, \@@_old_cs_if_exist:NF,
%     \@@_old_cs_if_free:NTF, \@@_old_cs_if_free:NT, \@@_old_cs_if_free:NF,
%   }
%   Copy \cs{exp_args:Nc} and a few others with no check, to use in situations where the
%   check should be omitted.
%    \begin{macrocode}
\cs_new_eq:NN \@@_old_exp_args:Nc  \exp_args:Nc
\cs_new_eq:NN \@@_old_exp_args:NNc \exp_args:NNc
\cs_new_eq:NN \@@_old_exp_args:Ncc \exp_args:Ncc
\cs_new_eq:NN \@@_old_cs_if_exist:NTF \cs_if_exist:NTF
\cs_new_eq:NN \@@_old_cs_if_exist:NT \cs_if_exist:NT
\cs_new_eq:NN \@@_old_cs_if_exist:NF \cs_if_exist:NF
\cs_new_eq:NN \@@_old_cs_if_free:NTF \cs_if_free:NTF
\cs_new_eq:NN \@@_old_cs_if_free:NT \cs_if_free:NT
\cs_new_eq:NN \@@_old_cs_if_free:NF \cs_if_free:NF
%    \end{macrocode}
% \end{macro}
%
% \begin{macro}{\use:nf, \tl_if_in:ffTF, \str_case:fnF, \tl_if_empty:fTF}
%   Just a variant we need.
%    \begin{macrocode}
\cs_generate_variant:Nn \use:nn { nf }
\cs_generate_variant:Nn \tl_if_in:nnTF { ff }
\cs_generate_variant:Nn \str_case:nnF { f }
\cs_generate_variant:Nn \tl_if_empty:nTF { f }
\cs_generate_variant:Nn \tl_if_empty:nT { f }
\cs_generate_variant:Nn \tl_if_empty:nF { f }
%    \end{macrocode}
% \end{macro}
%
% \begin{macro}[int, EXP]{\@@_exp:w, \@@_exp_stop:}
%   Trigger and stop one-step expansion.
%
%  This is now provided from \pkg{l3expan}.
%    \begin{macrocode}
%\cs_new_eq:NN \@@_exp:w \tex_romannumeral:D
%\cs_new_eq:NN \@@_exp_stop: \c_zero
%    \end{macrocode}
% \end{macro}
%
% \begin{variable}[int]{\l_@@_module_tl}
%   Holds the name of the module for which we are currently defining
%   the checking code.  This is used by \cs{check_patch:Nn} to make
%   commands aware of what module the patch belongs to, so that
%   \cs{check_off:n} will correctly turn that check off.
%    \begin{macrocode}
\tl_new:N \l_@@_module_tl
%    \end{macrocode}
% \end{variable}
%
% \begin{variable}[int]{\l_@@_parm_int, \l_@@_parm_tl, \l_@@_parm_braced_tl}
% \begin{variable}[int]{\l_@@_extra_code_tl}
%   Used in \cs{check_patch:Nn} to construct the parameter text for use
%   in \cs{check_patch:Npnn}.  The \cs{l_@@_extra_code_tl} token list
%   contains code that checks that |N|-type arguments really are single
%   tokens.
%    \begin{macrocode}
\int_new:N \l_@@_parm_int
\tl_new:N \l_@@_parm_tl
\tl_new:N \l_@@_parm_braced_tl
\tl_new:N \l_@@_extra_code_tl
%    \end{macrocode}
% \end{variable}
% \end{variable}
%
% \begin{macro}{\@@_error:nn, \@@_error:nnn, \@@_error:nnnn}
% \begin{macro}[EXP]{\@@_error_exp:nn, \@@_error_exp:nnn, \@@_error_exp:nnnn}
%   Wrappers around errors, expandable or not.
%    \begin{macrocode}
\cs_new:Npn \@@_error:nn
  { \msg_error:nnn { check } }
\cs_new:Npn \@@_error:nnn
  { \msg_error:nnnn { check } }
\cs_new:Npn \@@_error:nnnn
  { \msg_error:nnnnn { check } }
\cs_new:Npn \@@_error_exp:nn
  { \msg_expandable_error:nnn { check } }
\cs_new:Npn \@@_error_exp:nnn
  { \msg_expandable_error:nnnn { check } }
\cs_new:Npn \@@_error_exp:nnnn
  { \msg_expandable_error:nnnnn { check } }
%    \end{macrocode}
% \end{macro}
% \end{macro}
%
% \begin{macro}{\@@_warning:nn, \@@_warning:nnn}
% \begin{macro}[EXP]{\@@_warning_exp:nn, \@@_warning_exp:nnn}
%   A wrapper around warnings, expandable or not.
%    \begin{macrocode}
\cs_new:Npn \@@_warning:nn
  { \msg_warning:nnn { check } }
\cs_new:Npn \@@_warning:nnn
  { \msg_warning:nnnn { check } }
\cs_new:Npn \@@_warning_exp:nn
  {
    \@@_old_cs_if_exist:NTF \msg_expandable_warning:nnn
      { \msg_expandable_warning:nnn }
      { \msg_expandable_error:nnn }
        { check }
  }
\cs_new:Npn \@@_warning_exp:nnn
  {
    \@@_old_cs_if_exist:NTF \msg_expandable_warning:nnnn
      { \msg_expandable_warning:nnnn }
      { \msg_expandable_error:nnnn }
        { check }
  }
%    \end{macrocode}
% \end{macro}
% \end{macro}
%
% ^^A todo: deal with active chars!
%
% \begin{macro}[int]{\@@_split_name:Nnnn}
% \begin{macro}[int, EXP]{\@@_split_name_exp:Nnnn}
% \begin{macro}[aux, EXP]
%   {
%     \@@_split_name_aux:w , \@@_split_name_colon:w ,
%     \@@_split_name_underscore:w , \@@_split_name_space:w ,
%     \@@_split_name_func:w , \@@_split_name_func_end:w ,
%     \@@_split_name_var:nNw , \@@_split_name_var_end:nNw ,
%   }
%   Two variants are provided so that the correct type of errors
%   (expandable or not) is used.  Expands to |#2| \Arg{signature} if
%   the name contains a colon and is not |\:|.  Expands to |#3|
%   \Arg{scope} \Arg{type} if the name contains an underscore and is
%   not |\_|.  Expands to |#4| otherwise.
%    \begin{macrocode}
\cs_new_protected:Npn \@@_split_name:Nnnn #1
  {
    \exp_after:wN \exp_after:wN
    \exp_after:wN \@@_split_name_aux:w
    \cs_to_str:N #1 \q_stop
    \@@_warning:nnn
    \@@_error:nnn
  }
\cs_new:Npn \@@_split_name_exp:Nnnn #1
  {
    \exp_after:wN \exp_after:wN
    \exp_after:wN \@@_split_name_aux:w
    \cs_to_str:N #1 \q_stop
    \@@_warning_exp:nnn
    \@@_error_exp:nnn
  }
\cs_gset:Npn \@@_tmp:w #1#2
  {
    \cs_new:Npn \@@_split_name_aux:w ##1 \q_stop ##2 ##3
      {
        \tl_if_single:nTF {##1}
          { \use_iii:nnn }
          {
            \tl_if_empty:oTF { \@@_split_name_colon:w ##1 #1 }
              {
                \tl_if_empty:oTF { \@@_split_name_underscore:w ##1 #2 }
                  { \use_iii:nnn }
                  {
                    \tl_if_empty:oF { \@@_split_name_space:w ##1 ~ }
                      { ##2 { bad-var-name } {##1} { } }
                    \tl_if_head_eq_charcode:fNF
                      { \str_tail:n {##1} } #2
                      { ##2 { bad-var-name } {##1} { } }
                    \str_case:fnF { \str_head:n {##1} }
                      { l { } g { } c { } q { } s { } }
                      { ##2 { bad-var-name } {##1} { } }
                    \@@_split_name_var:Nw { } ##1
                      \q_mark \@@_split_name_var:Nw #2
                      \q_mark \@@_split_name_var_end:Nw \q_stop
                  }
              }
              {
                \tl_if_empty:oF { \@@_split_name_space:w ##1 ~ }
                  { ##2 { bad-func-name } {##1} { } }
                \@@_split_name_func:w \q_mark ##1
                  \q_mark \@@_split_name_func:w #1
                  \q_mark \@@_split_name_func_end:w \q_stop
              }
          }
      }
    \cs_new:Npn \@@_split_name_colon:w ##1 #1 { }
    \cs_new:Npn \@@_split_name_underscore:w ##1 #2 { }
    \cs_new:Npn \@@_split_name_space:w ##1 ~ { }
    \cs_new:Npn \@@_split_name_func:w ##1 \q_mark ##2 #1 ##3 \q_mark ##4
      { ##4 ##2 \q_mark ##3 \q_mark ##4 }
    \cs_new:Npn \@@_split_name_func_end:w ##1 \q_mark ##2 \q_stop ##3##4##5
      { ##3 {##1} }
    \cs_new:Npn \@@_split_name_var:Nw ##1##2 #2 ##3 #2 ##4 \q_mark ##5
      { ##5 ##1 ##3 #2 ##4 \q_mark ##5 }
    \cs_new:Npn \@@_split_name_var_end:Nw ##1##2 \q_mark ##3 \q_stop ##4##5##6
      { ##5 {##1} {##2} }
  }
\exp_after:wN \@@_tmp:w \tl_to_str:n { : _ }
%    \end{macrocode}
% \end{macro}
% \end{macro}
% \end{macro}
%
% \begin{macro}[int, EXP, pTF]{\@@_if_tl_macro:N}
%   Test if the variable is a non-long, non-protected macro with no argument.
%    \begin{macrocode}
\prg_new_conditional:Npnn \@@_if_tl_macro:N #1 { p , T , F , TF }
  {
    \tl_if_empty:fTF
      {
        \token_get_prefix_spec:N #1
        \token_get_arg_spec:N #1
      }
      { \prg_return_true: }
      { \prg_return_false: }
  }
%    \end{macrocode}
% \end{macro}
%
% \subsubsection{Turning on/off checks}
%
% \begin{macro}[int, EXP, pTF]{\@@_if_on:n}
%   This test is to be used within the second argument of
%   \cs{check_new:nn}.  It is for instance inserted by
%   \cs{check_patch:Npnn}.  It would be cleaner to use
%   \cs{cs_if_exist_use:cF}, but this function uses expansion, which
%   leads to an infinite recursion when \pkg{l3expan} functions are
%   checking and call this test.
%    \begin{macrocode}
\prg_new_conditional:Npnn \@@_if_on:n #1 { p , T , F , TF }
  {
    \exp_after:wN \if_meaning:w \cs:w @@_return_#1: \cs_end: \scan_stop:
      \prg_return_false:
    \else:
      \cs:w @@_return_#1: \cs_end:
    \fi:
  }
%    \end{macrocode}
% \end{macro}
%
% \begin{macro}{\check_on:n, \check_off:n}
%   The argument is turned to a string (paranoia), then we loop over
%   \meta{modules} in the argument.  If
%   |\__check_module_|\meta{module}|:N| is defined, use it with
%   \cs{prg_return_true:}|/false:| as its argument, otherwise it is an
%   error.  This function is initially defined to patch commands of the
%   given \meta{module} with the appropriate checking code and to
%   redefine itself to do much less.  In subsequent calls, it just sets
%   |\__check_return_|\meta{module}|:| equal to its argument
%   \cs{prg_return_true:}|/false:|.
%    \begin{macrocode}
\cs_new_protected:Npn \check_on:n #1
  {
    \exp_args:No \clist_map_inline:nn { \tl_to_str:n {#1} }
      {
        \cs_if_exist_use:cTF { @@_module_##1 :N }
          { \prg_return_true: }
          { \msg_error:nnn { check } { module-unknown } {##1} }
      }
  }
\cs_new_protected:Npn \check_off:n #1
  {
    \exp_args:No \clist_map_inline:nn { \tl_to_str:n {#1} }
      {
        \cs_if_exist_use:cTF { @@_module_##1 :N }
          { \prg_return_false: }
          { \msg_error:nnn { check } { module-unknown } {##1} }
      }
  }
%    \end{macrocode}
% \end{macro}
%
% \begin{macro}{\check_new:nn}
% \begin{macro}[int]{\@@_new:nn, \@@_module_first_use:n}
%   This function does not accept a comma-delimited list of modules,
%   only single modules, so we check and refuse that.  If the module
%   |#1| already has a checking code defined, raise an error and stop.
%   Store the \meta{code} |#2| in a control sequence (using an
%   \texttt{x}-expanding assignment and \cs{exp_not:n} to avoid issues
%   with |#|), with some code before and after it.  In its original
%   definition given here, |\__check_module_|\meta{module}|:N| will do
%   nothing if its argument is \cs{prg_return_false:} because turning
%   off checking that has never been turne on is a no-op.  When given
%   \cs{prg_return_true:} for the first time, this function is
%   redefined by the auxiliary \cs{@@_module_first_use:n} to do very
%   little (simply set the \texttt{return} function).  The same
%   auxiliary defines the \texttt{return} function globally to
%   \texttt{false} so that there is no checking outside the current
%   group, but locally to \texttt{true} to have checking in the current
%   group.  Finally, \cs{l_@@_module_tl} is defined so that it can be
%   used in the patch code, such as for \cs{check_patch:Nn}.
%    \begin{macrocode}
\cs_new_protected:Npn \check_new:nn #1
  { \exp_args:No \@@_new:nn { \tl_to_str:n {#1} } }
\cs_new_protected:Npn \@@_new:nn #1#2
  {
    \tl_if_in:nnTF {#1} { , }
      {
        \msg_error:nnxx { check } { multiple-modules }
          { \token_to_str:N \check_new:nn } {#1}
      }
      {
        \cs_if_exist:cTF { @@_module_#1:N }
          { \msg_error:nnn { check } { module-exists } {#1} }
          {
            \cs_new_protected:cpx { @@_module_#1:N } ##1
              {
                \exp_not:N \token_if_eq_meaning:NNF
                  \exp_not:N \prg_return_false:
                  ##1
                  {
                    \exp_not:N \@@_module_first_use:n {#1}
                    \exp_not:n {#2}
                  }
              }
          }
      }
  }
\cs_new_protected:Npn \@@_module_first_use:n #1
  {
    \cs_gset_protected:cpx { @@_module_#1:N }
      { \exp_not:N \cs_set_eq:NN \exp_not:c { @@_return_#1: } }
    \cs_new_eq:cN { @@_return_#1: } \prg_return_false:
    \cs_set_eq:cN { @@_return_#1: } \prg_return_true:
    \tl_set:Nn \l_@@_module_tl {#1}
  }
%    \end{macrocode}
% \end{macro}
% \end{macro}
%
% ^^A todo: document, keep list up to date.
% \begin{macro}[int]{\@@_module_l3kernel:N}
%   A special case that cannot be covered by other means.
%    \begin{macrocode}
\cs_new_protected:cpn { @@_module_l3kernel:N } #1
  {
    \clist_map_inline:Nn \c_@@_kernel_modules_clist
      { \use:c { @@_module_##1:N } #1 }
  }
\clist_const:Nn \c_@@_kernel_modules_clist { cs , expan } % , tl , seq }
%    \end{macrocode}
% \end{macro}
%
% \subsubsection{Patching commands}
%
% \begin{macro}[int]{\check_patch:Npnn, \check_patch:cpnn}
% \begin{macro}[aux]{\@@_patch:NnNnnn, \@@_patch:NoNnnn}
% \begin{macro}[aux]{\@@_patch_exp:NnNnnn, \@@_patch_exp:NoNnnn}
%   The current code of |#1| is copied into a private control sequence,
%   whose name we build and pass to the auxiliary.  Distinguish the
%   case of protected and of expandable commands, by looking for
%   |protected| in the macro's prefix spec.  This distinction is used
%   to redefine |#1| to be a protected or an expandable macro, but also
%   to make sure that the redefinition of expandable macros does not
%   alter their expansion properties.  Namely, it is important for some
%   macros that |o|-expanding them twice fully expands them, so the
%   redefined macro uses \cs{exp:w} and \cs{exp_end:} to make
%   sure that its second expansion is the same as the second expansion
%   of the original macro.
%
%   We use \cs{@@_old_exp_args:Nc}, a copy of \cs{exp_args:Nc}, because we
%   are building a previously not-defined control sequence.
%    \begin{macrocode}
\cs_new_protected:Npn \check_patch:Npnn #1#2#
  {
    \token_if_macro:NTF #1
      {
        \tl_if_in:ffTF
          { \token_get_prefix_spec:N #1 }
          { \tl_to_str:n { protected } }
          { \@@_old_exp_args:Nc \@@_patch:NoNnnn }
          { \@@_old_exp_args:Nc \@@_patch_exp:NoNnnn }
              { @@_old_ \cs_to_str:N #1 }
              { \l_@@_module_tl }
              #1
              {#2}
      }
      { \msg_error:nnnnnn { check } { patch-non-macro } {#1} {#2} }
  }
\cs_generate_variant:Nn \check_patch:Npnn { c }
\cs_new_protected:Npn \@@_patch:NnNnnn #1#2#3#4#5#6
  {
    \if_meaning:w #1 #3 \else: \cs_new_eq:NN #1 #3 \fi:
    \cs_gset_protected:Npn #3 #4
      {
        \@@_if_on:nT {#2} { \if_false: { \fi: #6 \if_false: } \fi: }
        #1 #5
      }
  }
\cs_generate_variant:Nn \@@_patch:NnNnnn { No }
\cs_new_protected:Npn \@@_patch_exp:NnNnnn #1#2#3#4#5#6
  {
    \if_meaning:w #1 #3 \else: \cs_new_eq:NN #1 #3 \fi:
    \cs_gset:Npn #3 #4
      {
        \exp:w
        \@@_if_on:nT {#2} { \if_false: { \fi: #6 \if_false: } \fi: }
        \exp_after:wN \exp_after:wN \exp_after:wN \exp_end:
        #1 #5
      }
  }
\cs_generate_variant:Nn \@@_patch_exp:NnNnnn { No }
%    \end{macrocode}
% \end{macro}
% \end{macro}
% \end{macro}
%
% ^^A todo: instead of \@@_if_on:nT {#2} use \cs:w @@_if_on_#2:T \cs_end: already as a single token.
%
% \begin{macro}[int]{\check_patch:Nn, \check_patch:cn}
% \begin{macro}[aux]{\@@_patch_parm:Nnn, \@@_patch_parm:nNn, \@@_patch_parm:VNn}
%   Find the signature (with appropriate errors if that is not found),
%   then construct the parameter text for \cs{check_patch:Npnn}, both
%   braced and not braced.  At the same time, fill the
%   \cs{l_@@_extra_code_tl} variable with code that checks |N|-type
%   arguments.  Also accept |T| and |F| arguments.
%    \begin{macrocode}
\cs_new_protected:Npn \check_patch:Nn #1#2
  {
    \token_if_macro:NTF #1
      {
        \tl_if_in:ffTF
          { \token_get_prefix_spec:N #1 }
          { \tl_to_str:n { protected } }
          { \@@_patch_aux:NNn \check_is_N:n #1 }
          { \@@_patch_aux:NNn \check_is_N_exp:n #1 }
      }
      { \msg_error:nnnnnn { check } { patch-non-macro } {#1} { } { } }
    {#2}
  }
\cs_new_protected:Npn \@@_patch_aux:NNn #1#2#3
  {
    \@@_split_name:Nnnn #2
      { \@@_patch_parm:NNnn #1 #2 {#3} }
      { \use_i:nnn { \@@_error:nn { patch-var } {#2} } }
      { \@@_error:nn { patch-non-expl } {#2} }
  }
\cs_generate_variant:Nn \check_patch:Nn { c }
\cs_new_protected:Npn \@@_patch_parm:NNnn #1#2#3#4
  {
    \int_zero:N \l_@@_parm_int
    \tl_clear:N \l_@@_parm_tl
    \tl_clear:N \l_@@_parm_braced_tl
    \tl_clear:N \l_@@_extra_code_tl
    \tl_map_inline:nn {#4}
      {
        \int_incr:N \l_@@_parm_int
        \@@_patch_parm:VNNn \l_@@_parm_int ##1 #1 {#4}
      }
    \use:x
      {
        \exp_not:N \check_patch:Npnn
          \exp_not:N #2
          \exp_not:V \l_@@_parm_tl
          { \exp_not:V \l_@@_parm_braced_tl }
          {
            \exp_not:V \l_@@_extra_code_tl
            \exp_not:n {#3}
          }
      }
  }
\cs_new_protected:Npn \@@_patch_parm:nNNn #1#2#3#4
  {
    \str_case:nnTF {#2}
      {
        N
        {
          \tl_put_right:Nn \l_@@_parm_braced_tl { ## #1 }
          \tl_put_right:Nn \l_@@_extra_code_tl { #3 { ## #1 } }
        }
        n { \tl_put_right:Nn \l_@@_parm_braced_tl { { ## #1 } } }
        T { \tl_put_right:Nn \l_@@_parm_braced_tl { { ## #1 } } }
        F { \tl_put_right:Nn \l_@@_parm_braced_tl { { ## #1 } } }
      }
      { \tl_put_right:Nn \l_@@_parm_tl { ## #1 } }
      {
        \tl_map_break:n
          { \@@_error:nn { patch-non-base } {#4} }
      }
  }
\cs_generate_variant:Nn \@@_patch_parm:nNNn { V }
%    \end{macrocode}
% \end{macro}
% \end{macro}
%
% \subsubsection{Checks that can be made}
%
% ^^A todo: document
%
% \begin{macro}{\check_omit:NN}
% \begin{macro}[aux]{\@@_omit_aux:NNN}
%   This cannot be nested!
%    \begin{macrocode}
\cs_new_protected:Npn \check_omit:NN #1#2
  { \@@_old_exp_args:Nc \@@_omit_aux:NNN { @@_saved_ \cs_to_str:N #1 } #1 #2 }
\cs_new_protected:Npn \@@_omit_aux:NNN #1#2#3
  {
    \cs_gset_eq:NN #1 #2
    \cs_gset_protected:Npn #2
      { \cs_gset_eq:NN #2 #1 #3 }
  }
%    \end{macrocode}
% \end{macro}
% \end{macro}
%
% \begin{macro}{\check_is_N:n}
% \begin{macro}[EXP]{\check_is_N_exp:n}
% \begin{macro}[aux, EXP]{\@@_if_N_type:nF}
%   Those functions check if a token list is a single |N|-type token.
%   We cannot safely use some of the built-in \LaTeX{} functions since
%   the token list may contain \cs{q_stop}.
%    \begin{macrocode}
\cs_new_protected:Npn \check_is_N:n #1
  {
    \@@_if_N_type:nF {#1}
      { \@@_error:nn { is-not-N-type } {#1} }
  }
\cs_new:Npn \check_is_N_exp:n #1
  {
    \@@_if_N_type:nF {#1}
      { \@@_error_exp:nn { is-not-N-type } {#1} }
  }
\prg_new_conditional:Npnn \@@_if_N_type:n #1 { F , TF }
  {
    \tl_if_blank:nTF {#1}
      { \prg_return_false: }
      {
        \tl_if_single_token:nTF {#1}
          { \prg_return_true: }
          { \prg_return_false: }
      }
  }
%    \end{macrocode}
% \end{macro}
% \end{macro}
% \end{macro}
%
% \begin{macro}{\check_is_function:N}
% \begin{macro}[EXP]{\check_is_function_exp:N}
%    \begin{macrocode}
\cs_new_protected:Npn \check_is_function:N #1
  {
    \@@_split_name:Nnnn #1
      { \use_none:n }
      { \use_i:nnn { \@@_error:nnn { is-not } { function } {#1} } }
      { }
  }
\cs_new:Npn \check_is_function_exp:N #1
  {
    \@@_split_name_exp:Nnnn #1
      { \use_none:n }
      { \use_i:nnn { \@@_error_exp:nnn { is-not } { function } {#1} } }
      { }
  }
%    \end{macrocode}
% \end{macro}
% \end{macro}
%
% \begin{macro}{\check_is_variable:N}
% \begin{macro}[EXP]{\check_is_variable_exp:N}
%    \begin{macrocode}
\cs_new_protected:Npn \check_is_variable:N #1
  {
    \@@_split_name:Nnnn #1
      { \use_i:nn { \@@_error:nnn { is-not } { variable } {#1} } }
      { \use_none:nn }
      { }
  }
\cs_new:Npn \check_is_variable_exp:N #1
  {
    \@@_split_name_exp:Nnnn #1
      { \use_i:nn { \@@_error_exp:nnn { is-not } { variable } {#1} } }
      { \use_none:nn }
      { }
  }
%    \end{macrocode}
% \end{macro}
% \end{macro}
%
% \begin{macro}{\check_is_local_variable:N}
% \begin{macro}[EXP]{\check_is_local_variable_exp:N}
%    \begin{macrocode}
\cs_new_protected:Npn \check_is_local_variable:N
  { \@@_is_scope_variable:NNN l \@@_error:nnn }
\cs_new_protected:Npn \check_is_global_variable:N
  { \@@_is_scope_variable:NNN g \@@_error:nnn }
\cs_new:Npn \check_is_local_variable_exp:N
  { \@@_is_scope_variable:NNN l \@@_error_exp:nnn }
\cs_new:Npn \check_is_global_variable_exp:N
  { \@@_is_scope_variable:NNN g \@@_error_exp:nnn }
\cs_new:Npn \@@_is_scope_variable:NNN #1#2#3
  {
    \@@_split_name:Nnnn #3
      { \use_i:nn { #2 { is-not } { variable } {#3} } }
      { \@@_is_scope_variable:NNNnn #1#2#3 }
      { }
  }
\cs_new:Npn \@@_is_scope_variable:NNNnn #1#2#3#4#5
  {
    \str_if_eq:nnF {#1} {#4}
      { #2 { is-not-scope } {#1} {#3} }
  }
%    \end{macrocode}
% \end{macro}
% \end{macro}
%
% \begin{macro}[EXP]{\check_signature:N}
%   Expands to \cs{scan_stop:} or to the signature of a function.
%    \begin{macrocode}
\cs_new:Npn \check_signature:N #1
  {
    \@@_split_name_exp:Nnnn #1
      { \use:n }
      { \use_i:nnn \scan_stop: }
      { \scan_stop: }
  }
%    \end{macrocode}
% \end{macro}
%
% \begin{macro}
%   {
%     \check_signature_is_base:N,
%     \check_signature_should_be_empty:N,
%     \check_signature_should_be_nonempty:N
%   }
% \begin{macro}[aux]{\@@_signature_is_base:N}
%    \begin{macrocode}
\cs_new_protected:Npn \check_signature_is_base:N #1
  {
    \exp_last_unbraced:Nf \@@_signature_is_base:N
      { \check_signature:N #1 } \q_recursion_tail \q_recursion_stop
  }
\cs_new_protected:Npn \@@_signature_is_base:N #1
  {
    \token_if_eq_charcode:NNTF N #1
      { \@@_signature_is_base:N }
      {
        \token_if_eq_charcode:NNTF n #1
          { \@@_signature_is_base:N }
          {
            \quark_if_recursion_tail_stop_do:Nn #1 { \prg_return_true: }
            \use_i_delimit_by_q_recursion_stop:nw { \prg_return_false: }
          }
      }
  }
\cs_new_protected:Npn \check_signature_should_be_empty:N #1
  {
    \tl_if_empty:fF { \check_signature:N #1 }
      { \@@_warning:nn { arg-nopar } {#1} }
  }
\cs_new_protected:Npn \check_signature_should_be_nonempty:N #1
  {
    \tl_if_empty:fT { \check_signature:N #1 }
      { \@@_warning:nn { par-noarg } {#1} }
  }
%    \end{macrocode}
% \end{macro}
% \end{macro}
%
% \begin{macro}[EXP, aux]{\@@_is_name_type:NnN, \@@_is_name_type_aux:NnNnn}
%    \begin{macrocode}
\cs_new:Npn \@@_is_name_type:NnN #1#2#3
  {
    \@@_split_name:Nnnn #1
      { \use_i:nn { #3 { is-not } {#2} {#1} } }
      { \@@_is_name_type_aux:NnNnn #1#2#3 }
      { }
  }
\cs_new:Npn \@@_is_name_type_aux:NnNnn #1#2#3#4#5
  { \str_if_eq:nnF {#5} {#2} { #3 { is-not } {#2} {#1} } }
%    \end{macrocode}
% \end{macro}
%
% \begin{macro}{\check_is_defined:N}
% \begin{macro}[EXP]{\check_is_defined_exp:N, \check_is_defined_exp:c}
%   Do not use \cs{exp_args:Nc} (nor \cs{cs_generate_variant:Nn} for
%   the |c| variant in order to avoid infinite loop since the redefined
%   |c|-expansion calls this function.
%    \begin{macrocode}
\cs_new_protected:Npn \check_is_defined:N #1
  {
    \@@_old_cs_if_exist:NF #1
      { \@@_error:nn { is-not-defined } {#1} }
  }
\cs_new:Npn \check_is_defined_exp:N #1
  {
    \@@_old_cs_if_exist:NF #1
      { \@@_error_exp:nn { is-not-defined } {#1} }
  }
\cs_new:Npn \check_is_defined_exp:c #1
  { \exp_after:wN \check_is_defined_exp:N \cs:w #1 \cs_end: }
%    \end{macrocode}
% \end{macro}
% \end{macro}
%
% \begin{macro}{\check_is_tl:N, \check_is_defined_tl:N}
% \begin{macro}[EXP]{\check_is_tl_exp:N, \check_is_defined_tl_exp:N}
% \begin{macro}[EXP, aux]{\@@_is_tl_aux:NN}
%    \begin{macrocode}
\cs_new_protected:Npn \check_is_defined_tl:N #1
  {
    \@@_old_cs_if_exist:NTF #1
      { \@@_is_tl_aux:NN #1 \@@_error:nnn }
      { \@@_error:nn { is-not-defined } {#1} }
  }
\cs_new:Npn \check_is_defined_tl_exp:N #1
  {
    \@@_old_cs_if_exist:NTF #1
      { \@@_is_tl_aux:NN #1 \@@_error_exp:nnn }
      { \@@_error_exp:nn { is-not-defined } {#1} }
  }
\cs_new_protected:Npn \check_is_tl:N #1
  {
    \@@_old_cs_if_exist:NTF #1
      { \@@_is_tl_aux:NN #1 \@@_error:nnn }
      { \@@_is_name_type:NnN #1 { tl } \@@_error:nnn }
  }
\cs_new:Npn \check_is_tl_exp:N #1
  {
    \@@_old_cs_if_exist:NTF #1
      { \@@_is_tl_aux:NN #1 \@@_error_exp:nnn }
      { \@@_is_name_type:NnN #1 { tl } \@@_error_exp:nnn }
  }
\cs_new:Npn \@@_is_tl_aux:NN #1#2
  { \@@_if_tl_macro:NF #1 { #2 { is-not } { tl } {#1} } }
%    \end{macrocode}
% \end{macro}
% \end{macro}
% \end{macro}
%
% ^^A todo: consolidate into \check_is_type:Nn #1 { seq }, combine internals with tl too, building them with \cs:w \cs_end:
% \begin{macro}{\check_is_seq:N, \check_is_defined_seq:N}
% \begin{macro}[EXP]{\check_is_seq_exp:N, \check_is_defined_seq_exp:N}
%   ^^A todo: doc auxiliaries
%    \begin{macrocode}
\cs_new_protected:Npn \check_is_defined_seq:N #1
  {
    \@@_old_cs_if_exist:NTF #1
      { \@@_is_seq_aux:NN #1 \@@_error:nn }
      { \@@_error:nn { is-not-defined } {#1} }
  }
\cs_new:Npn \check_is_defined_seq_exp:N #1
  {
    \@@_old_cs_if_exist:NTF #1
      { \@@_is_seq_aux:NN #1 \@@_error_exp:nnn }
      { \@@_error_exp:nn { is-not-defined } {#1} }
  }
\cs_new_protected:Npn \check_is_seq:N #1
  {
    \@@_old_cs_if_exist:NTF #1
      { \@@_is_seq_aux:NN #1 \@@_error:nnn }
      { \@@_is_name_type:NnN #1 { seq } \@@_error:nnn }
  }
\cs_new:Npn \check_is_seq_exp:N #1
  {
    \@@_old_cs_if_exist:NTF #1
      { \@@_is_seq_aux:NN #1 \@@_error_exp:nnn }
      { \@@_is_name_type:NnN #1 { seq } \@@_error_exp:nnn }
  }
\cs_new:Npn \@@_is_seq_aux:NN #1#2
  { \@@_if_seq:NF #1 { #2 { is-not } { seq } {#1} } }
\prg_new_conditional:Npnn \@@_if_seq:N #1 { F }
  {
    \@@_if_tl_macro:NTF #1
      {
        \exp_args:No \tl_if_empty:oTF
          { \exp_after:wN \@@_if_seq_aux:w #1 \q_recursion_tail }
          {
            \exp_after:wN \@@_if_seq_aux:N
            #1 \q_recursion_tail \q_recursion_tail
            \q_recursion_tail \q_recursion_stop
          }
          { \prg_return_false: }
      }
      { \prg_return_false: }
  }
\cs_new:Npn \@@_if_seq_aux:w #1 \q_recursion_tail { }
\cs_new:Npn \@@_if_seq_aux:N #1
  {
    \str_if_eq:nnTF {#1} { \s__seq }
      { \@@_if_seq_loop:nn }
      { \@@_if_seq_break:Nw \prg_return_false: }
  }
\cs_new:Npn \@@_if_seq_break:Nw
    #1 #2 \q_recursion_tail \q_recursion_stop {#1}
\cs_new:Npn \@@_if_seq_loop:nn #1#2
  {
    \quark_if_recursion_tail_stop_do:nn {#1} { \prg_return_true: }
    \quark_if_recursion_tail_stop_do:nn {#2} { \prg_return_false: }
    \tl_if_single:nF {#1}
      { \@@_if_seq_break:Nw \prg_return_false: }
    \str_if_eq:nnF {#1} { \__seq_item:n }
      { \@@_if_seq_break:Nw \prg_return_false: }
    \@@_if_seq_loop:nn
  }
%    \end{macrocode}
% \end{macro}
% \end{macro}
%
% \subsubsection{General messages}
%
% ^^A todo: more informative error saying what the variable is if it has the wrong type.
%    \begin{macrocode}
\msg_new:nnn { check } { is-not-defined }
  { '#1'~is~not~defined. }
\msg_new:nnn { check } { is-not-N-type }
  { '#1'~is~not~a~single~N-type~token. }
\msg_new:nnn { check } { is-not-scope }
  {
    \use:nf { '#2'~is~not~a~ }
      {
        \str_case:nnF {#1}
          {
            { l } { local }
            { g } { global }
          } {#1} ~
        variable~(starting~with~'#1_').
      }
  }
\msg_new:nnn { check } { is-not }
  {
    \use:nf { '#2'~is~not~a~ }
      {
        \str_case:nnF {#1}
          {
            { function } { function }
            { variable } { variable }
            { tl } { token~list }
            { seq } { sequence }
          }
          {#1} .
      }
  }
\msg_new:nnn { check } { patch-var }
  { Only~functions~can~be~patched.~'#1'~is~a~variable }
\msg_new:nnn { check } { patch-non-expl }
  { Use~'\iow_char:N\\check_patch:Npnn',~not~'\iow_char:N\\check_patch:Nn'~to~patch~'#1' }
\msg_new:nnn { check } { patch-non-base }
  { Cannot~patch~'#1'~with~'\iow_char:N\\check_patch:Nn':~it~is~not~a~base~function! }
\msg_new:nnn { check } { patch-non-macro }
  { Cannot~patch~'#1':~it~is~not~a~macro! }
\msg_new:nnn { check } { bad-var-name }
  { Variable~name~'#1'~does~not~follow~expl3~conventions }
\msg_new:nnn { check } { bad-func-name }
  { Function~name~'#1'~does~not~follow~expl3~conventions }
\msg_new:nnn { check } { multiple-modules }
  { The~function~#1~expects~a~single~module,~not~'#2'. }
\msg_new:nnn { check } { module-exists }
  { The~checking~code~for~module~'#1'~is~already~defined. }
\msg_new:nnn { check } { module-unknown }
  { No~checking~code~defined~for~module~'#1'. }
%    \end{macrocode}
%
% ^^A todo: consolidate below with above.
%    \begin{macrocode}
\msg_new:nnnn { check } { non-declared-variable }
  {
    The~variable~\tl_trim_spaces:n{#1}~has~not~been~declared~
    \msg_line_context:.
  }
  {
    Checking~is~active,~and~you~have~tried~do~so~something~like: \\
    \ \ \tl_set:Nn ~ \tl_trim_spaces:n {#1} ~  \{ ~ ... ~ \} \\
    without~first~having: \\
    \ \ \tl_new:N ~ \tl_trim_spaces:n {#1} \\
    \\
    LaTeX~will~create~the~variable~and~continue.
  }
\msg_new:nnn { check } { par-noarg }
  { The~function~'#1'~takes~no~argument~but~is~not~defined~to~be~"_nopar".~This~is~unusual. }
\msg_new:nnn { check } { arg-nopar }
  { The~function~'#1'~is~"_nopar"~but~takes~arguments.~This~is~risky. }
\msg_new:nnn { check } { set-eq-undef }
  { The~function~'#1'~was~set~equal~to~'#2',~but~that~is~not~defined. }
\msg_new:nnn { check } { no-expansion }
  { Redundant~#1-expansion~of~'#2' }
\msg_new:nnn { check } { internal }
  { Internal~error:~#1!! }
\msg_new:nnn { check } { bad-signature }
  { Signature~'#2'~expected~for~an~argument~of~'#1'.~Got~'#3'. }
%    \end{macrocode}
%
% \subsection{\texttt{l3kernel}}
%
% ^^A todo: worry about drivers?
%
% The following is based on file \texttt{expl3-code} at revision 5859 or so.
%
% Difficulty with \cs{int_gdecr:N} being defined as \cs{tex_global:D}
% \cs{int_decr:N}.  Same for \cs{int_gadd:Nn},
% \cs{int_gsub:Nn}, \cs{int_gset:Nn},
% |(dim/skip/muskip)_(gzero:N/gset:Nn/gset_eq:NN/gadd:Nn/gsub:Nn)|,
% \cs{box_gset_eq:NN}, \cs{box_gset_eq_clear:NN},
% \cs{box_get_to_last:N}, \cs{hbox_gset:Nn}, \cs{hbox_gset_to_wd:Nnn},
% \cs{hbox_gset:Nw}, \cs{vbox_gset:Nn}, \cs{vbox_gset_top:Nn},
% \cs{vbox_gset_to_ht:Nnn}, \cs{vbox_gset:Nw}.
%
% \subsubsection{\pkg{l3basics}}
%
%    \begin{macrocode}
\check_new:nn { cs }
  {
%    \end{macrocode}
%
% First standardize some definitions: the $16$ assignment functions
% |\cs_(set/gset/new)(/_protected)(/_nopar):Np(n/x)| are normally
% defined in terms of each other, but that complicates checking, since
% some are defined as \cs{tex_long:D} or \cs{etex_protected:D} followed
% by others.
%    \begin{macrocode}
    \cs_gset_protected:Npn \cs_set_nopar:Npn { \tex_def:D }
    \cs_gset_protected:Npn \cs_set_nopar:Npx { \tex_edef:D }
    \cs_gset_protected:Npn \cs_gset_nopar:Npn { \tex_gdef:D }
    \cs_gset_protected:Npn \cs_gset_nopar:Npx { \tex_xdef:D }
    \cs_gset_protected:Npn \cs_set:Npn { \tex_long:D \tex_def:D }
    \cs_gset_protected:Npn \cs_set:Npx { \tex_long:D \tex_edef:D }
    \cs_gset_protected:Npn \cs_gset:Npn { \tex_long:D \tex_gdef:D }
    \cs_gset_protected:Npn \cs_gset:Npx { \tex_long:D \tex_xdef:D }
    \cs_gset_protected:Npn \cs_set_protected_nopar:Npn { \etex_protected:D \tex_def:D }
    \cs_gset_protected:Npn \cs_set_protected_nopar:Npx { \etex_protected:D \tex_edef:D }
    \cs_gset_protected:Npn \cs_gset_protected_nopar:Npn { \etex_protected:D \tex_gdef:D }
    \cs_gset_protected:Npn \cs_gset_protected_nopar:Npx { \etex_protected:D \tex_xdef:D }
    \cs_gset_protected:Npn \cs_set_protected:Npn { \etex_protected:D \tex_long:D \tex_def:D }
    \cs_gset_protected:Npn \cs_set_protected:Npx { \etex_protected:D \tex_long:D \tex_edef:D }
    \cs_gset_protected:Npn \cs_gset_protected:Npn { \etex_protected:D \tex_long:D \tex_gdef:D }
    \cs_gset_protected:Npn \cs_gset_protected:Npx { \etex_protected:D \tex_long:D \tex_xdef:D }
%    \end{macrocode}
% Simplify \cs{cs_gset_eq:NN} and \cs{cs_new_eq:NN} similarly rather
% than defining them in terms of \cs{cs_set_eq:NN}
%    \begin{macrocode}
    \cs_gset_protected:Npn \cs_gset_eq:NN #1 { \tex_global:D \tex_let:D #1 = ~ }
    \cs_gset_protected:Npn \cs_new_eq:NN #1  { \__chk_if_free_cs:N #1 \tex_global:D \tex_let:D #1 = ~ }
%    \end{macrocode}
%
% We could change \cs{group_begin:}, \cs{group_end:},
% \cs{group_insert_after:N}, but do not.
%
% Let us do |\cs_(set/gset/new)(/_protected)(/_nopar):N(/p)(n/x)|, $48$
% assignment functions.  For now, just check that the first argument is
% a single token, is a function (not yet), and for the |:Nn|/|:Nx|
% defining functions, check that the signature is right.  Additionally,
% check that \enquote{nopar} is used properly: have a warning when
% trying to define a long macro with no parameter or a non-long macro
% with parameters.  The plan is eventually to
% \begin{itemize}
%   \item list any undefined token appearing in any cs definition, in
%     particular variants of existing tokens;
%   \item detect what parts are arguments of what and detect e.g.,
%     brace groups used as N-type arguments, etc.
%   \item try to detect the case of expandable commands which use
%     protected commands.
%   \item perhaps detect functions from the \enquote{msg} module, and
%     check that the corresponding message exists.
% \end{itemize}
% Note that we do not touch the \enquote{new} functions as they call
% the \enquote{gset} functions internally (otherwise we would get two
% warnings/errors).
% % ^^A todo: reinstate \check_is_function:N when more is done
%    \begin{macrocode}
    \group_begin:
      \cs_set:Npn \check_tmp:w #1#2#3
        {
          \check_patch:cpnn { cs_ #1 #2 :Np #3 }
            ##1 ##2 ## { ##1 ##2 }
            {
              \check_is_N:n {##1}
              % \check_is_function:N ##1
              \tl_if_empty:nT {##2} { \@@_warning:nn { par-noarg } {##1} }
            }
          \check_patch:cpnn { cs_ #1 #2 _nopar:Np #3 }
            ##1 ##2 ## { ##1 ##2 }
            {
              \check_is_N:n {##1}
              % \check_is_function:N ##1
              \tl_if_empty:nF {##2} { \@@_warning:nn { arg-nopar } {##1} }
            }
          \check_patch:cpnn { cs_ #1 #2 :N #3 }
            ##1 { ##1 }
            {
              \check_is_N:n {##1}
              \check_is_function:N ##1
              \check_signature_is_base:N ##1
              \check_signature_should_be_nonempty:N ##1
            }
          \check_patch:cpnn { cs_ #1 #2 _nopar:N #3 }
            ##1 ##2 ## { ##1 ##2 }
            {
              \check_is_N:n {##1}
              \check_is_function:N ##1
              \check_signature_is_base:N ##1
              \check_signature_should_be_empty:N ##1
            }
        }
      \tl_map_inline:nn { {set} {gset} }
        {
          \tl_map_inline:nn { { } {_protected} }
            {
              \tl_map_inline:nn { n x }
                { \check_tmp:w {#1} {##1} {####1} }
            }
        }
    \group_end:
%    \end{macrocode}
%
% The second argument of |\cs_(set/gset/new)_eq:NN| should be defined,
% except if it is \cs{scan_stop:}, \cs{tex_relax:D} or
% \cs{tex_undefined:D}, in which case that was probably the point.
%    \begin{macrocode}
    \tl_map_inline:nn { {set} {gset} {new} }
      {
        \check_patch:cn { cs_#1_eq:NN }
          {
            \@@_old_cs_if_exist:NF ##2
              {
                \str_case:nnF {##2}
                  {
                    \tex_undefined:D { }
                    \tex_relax:D { }
                    \scan_stop: { }
                  }
                  { \@@_warning:nnn { set-eq-undef } {##1} {##2} }
              }
          }
      }
%    \end{macrocode}
%
% \begin{macro}{\__cs_check_exists:N}
%   Since \pkg{l3check} adds a check that the commands exist, no need
%   to keep the \pkg{l3kernel} check.
%    \begin{macrocode}
    \cs_gset_eq:NN \__cs_check_exists:N \use_none:n
%    \end{macrocode}
% \end{macro}
%
% Next the $8$ |\prg_(set/new)(/_protected)_conditional:N(/p)nn|, for
% which we just check that the first argument is N-type and is a
% function (the |:| is crucial to place |_p|), and for the |:Nnn|-type
% case, that the first argument is a function with an appropriate
% signature.  Also patch the |\prg_(set/new)_eq_conditional:NNn| to
% check that the N-type arguments are functions.
%    \begin{macrocode}
    \group_begin:
      \cs_set:Npn \check_tmp:w #1#2#3
        {
          \check_patch:cpnn { prg_ #1 #2 _conditional :Npnn }
            ##1 ##2 ## { ##1 ##2 }
            {
              \check_is_N:n {##1}
              \check_is_function:N ##1
            }
          \check_patch:cpnn { prg_ #1 #2 _conditional :Nnn }
            ##1 { ##1 }
            {
              \check_is_N:n {##1}
              \check_is_function:N ##1
              \check_signature_is_base:N ##1
            }
        }
      \tl_map_inline:nn { {set} {new} }
        {
          \tl_map_inline:nn { { } {_protected} }
            { \check_tmp:w {#1} {##1} }
          \check_patch:cn { prg_#1_eq_conditional:NNn }
            {
              \check_is_function:N ##1
              \check_is_function:N ##2
            }
        }
    \group_end:
%    \end{macrocode}
%
% A few functions simply need checking of their |N|-type arguments,
% or just that their |N|-type argument is a function.
%    \begin{macrocode}
    \check_patch:Nn \cs_to_str:N { }
    \check_patch:Nn \cs_undefine:N { }
    \check_patch:Nn \cs_show:N { }
    \check_patch:Nn \cs_log:N { }
    \check_patch:Nn \cs_if_exist_p:N { \check_is_function_exp:N #1 }
    \check_patch:Nn \cs_if_exist:NF  { \check_is_function_exp:N #1 }
    \check_patch:Nn \cs_if_exist:NT  { \check_is_function_exp:N #1 }
    \check_patch:Nn \cs_if_exist:NTF { \check_is_function_exp:N #1 }
    \check_patch:Nn \cs_if_free_p:N  { \check_is_function_exp:N #1 }
    \check_patch:Nn \cs_if_free:NF   { \check_is_function_exp:N #1 }
    \check_patch:Nn \cs_if_free:NT   { \check_is_function_exp:N #1 }
    \check_patch:Nn \cs_if_free:NTF  { \check_is_function_exp:N #1 }
    \check_patch:Nn \cs_if_exist_use:N   { \check_is_function_exp:N #1 }
    \check_patch:Nn \cs_if_exist_use:NF  { \check_is_function_exp:N #1 }
    \check_patch:Nn \cs_if_exist_use:NT  { \check_is_function_exp:N #1 }
    \check_patch:Nn \cs_if_exist_use:NTF { \check_is_function_exp:N #1 }
%    \end{macrocode}
%
%    \begin{macrocode}
    \check_patch:Nn \cs_generate_from_arg_count:NNnn
      {
        \check_is_function:N #1
        \check_is_function:N #2
        \str_if_eq_x:nnF { \check_signature:N #2 } { Npn }
          {
            \@@_error:nnnn { bad-signature }
              { \cs_generate_from_arg_count:NNnn } { Npn } {#1}
          }
      }
%    \end{macrocode}
%
% Maybe the following is a bit to restrictive.
%    \begin{macrocode}
    \tl_map_inline:nn
      { \cs_if_eq_p:NN \cs_if_eq:NNT \cs_if_eq:NNF \cs_if_eq:NNTF }
      {
        \check_patch:Nn #1
          {
            \check_is_function_exp:N ##1
            \check_is_defined_exp:N ##1
            \check_is_function_exp:N ##2
            \check_is_defined_exp:N ##2
          }
      }
%    \end{macrocode}
%
% When outputting a message, a function definition is made using
% begin-group and end-group tokens which are spaces (character code
% $32$), and that confuses the new \cs{cs_set_protected_nopar:Npn}.  To
% avoid this, go back temporarily to the old definition of that
% function.
%    \begin{macrocode}
    \cs_new_eq:NN \@@_new_cs_set_protected_nopar:Npn \cs_set_protected_nopar:Npn
    \use:x
      {
        \exp_not:n { \cs_gset_protected:Npn \__msg_interrupt_text:n #1 }
          {
            \exp_not:n { \cs_set_eq:NN \cs_set_protected_nopar:Npn \@@_old_cs_set_protected_nopar:Npn }
            \exp_not:o { \__msg_interrupt_text:n {#1} }
            \exp_not:n { \cs_set_eq:NN \cs_set_protected_nopar:Npn \@@_new_cs_set_protected_nopar:Npn }
          }
      }
%    \end{macrocode}
%
% In \cs{int_new:N} etc., there appears \cs{__chk_if_free_cs:N}, which
% calls \cs{cs_if_free:NF}.  Replace that by \cs{@@_old_cs_if_free:NF}
% to avoid balking at variables.
%    \begin{macrocode}
    \use:x
      {
        \exp_not:n { \cs_gset_protected:Npn \__chk_if_free_cs:N #1 }
          {
            \exp_not:f
              {
                \exp_after:wN \use_i_ii:nnn
                \exp_after:wN \exp_stop_f:
                \exp_after:wN \@@_old_cs_if_free:NF
                \__chk_if_free_cs:N {#1}
              }
          }
      }
%    \end{macrocode}
%
%    \begin{macrocode}
  }
%    \end{macrocode}
%
% \subsubsection{\pkg{l3expan}}
%
%    \begin{macrocode}
\check_new:nn { expan }
  {
%    \end{macrocode}
% ^^A todo: we may need \group_align_safe_begin:/end: in case of alignment primitive
% ^^A todo: check that first arg of \exp_args:N... is indeed N
%
% \begin{macro}
%   {
%     \cs_set_eq:cN, \cs_set_eq:Nc, \cs_set_eq:cc,
%     \cs_gset_eq:cN, \cs_gset_eq:Nc, \cs_gset_eq:cc,
%     \cs_new_eq:cN, \cs_new_eq:Nc, \cs_new_eq:cc,
%     \cs_set:cpn, \cs_set_protected:cpn, \cs_set_nopar:cpn, \cs_set_protected_nopar:cpn,
%     \cs_set:cpx, \cs_set_protected:cpx, \cs_set_nopar:cpx, \cs_set_protected_nopar:cpx,
%     \cs_gset:cpn, \cs_gset_protected:cpn, \cs_gset_nopar:cpn, \cs_gset_protected_nopar:cpn,
%     \cs_gset:cpx, \cs_gset_protected:cpx, \cs_gset_nopar:cpx, \cs_gset_protected_nopar:cpx,
%     \cs_new:cpn, \cs_new_protected:cpn, \cs_new_nopar:cpn, \cs_new_protected_nopar:cpn,
%     \cs_new:cpx, \cs_new_protected:cpx, \cs_new_nopar:cpx, \cs_new_protected_nopar:cpx,
%     \tl_new:c, \tl_show:c, \tl_log:c, \tl_clear_new:c, \tl_gclear_new:c, \tl_const:cn, \tl_const:cx,
%     \clist_new:c, \clist_show:c, \clist_log:c, \clist_clear_new:c, \clist_gclear_new:c, \clist_const:cn,
%     \seq_new:c, \seq_show:c, \seq_log:c, \seq_clear_new:c, \seq_gclear_new:c,
%     \prop_new:c, \prop_show:c, \prop_log:c, \prop_clear_new:c, \prop_gclear_new:c,
%     \int_new:c, \int_show:c, \int_log:c, \int_zero_new:c, \int_gzero_new:c, \int_const:cn,
%     \dim_new:c, \dim_show:c, \dim_log:c, \dim_zero_new:c, \dim_gzero_new:c, \dim_const:cn,
%     \skip_new:c, \skip_show:c, \skip_log:c, \skip_zero_new:c, \skip_gzero_new:c, \skip_const:cn,
%     \muskip_new:c, \muskip_show:c, \muskip_log:c, \muskip_zero_new:c, \muskip_gzero_new:c, \muskip_const:cn,
%   }
% Redefine some |:Nc|, |:cN| and |:cc| variants to use
% \cs{@@_old_exp_args:Nc} and similar functions which will omit
% existence checks, since here the resulting control sequences are
% often undefined.
%    \begin{macrocode}
    \tl_map_inline:nn { {set} {gset} {new} }
      {
        \cs_gset_protected:cpx { cs_#1_eq:cN }
          {
            \exp_not:N \@@_old_exp_args:Nc
            \exp_not:c { cs_#1_eq:NN }
          }
        \cs_gset_protected:cpx { cs_#1_eq:Nc }
          {
            \exp_not:N \@@_old_exp_args:NNc
            \exp_not:c { cs_#1_eq:NN }
          }
        \cs_gset_protected:cpx { cs_#1_eq:cc }
          {
            \exp_not:N \@@_old_exp_args:Ncc
            \exp_not:c { cs_#1_eq:NN }
          }
      }
    \group_begin:
      \cs_set_protected:Npn \check_tmp:w #1#2
        {
          \exp_after:wN \cs_gset_protected:Npx \cs:w #1 : c #2 \cs_end:
            {
              \exp_not:N \@@_old_exp_args:Nc
              \exp_not:c { #1 : N#2 }
            }
        }
      \tl_map_inline:nn { {set} {gset} {new} }
        {
          \tl_map_inline:nn { { } { _protected } { _nopar } { _protected_nopar } }
            {
              \tl_map_inline:nn { n x }
                { \check_tmp:w { cs_#1##1 } { p ####1 } }
            }
        }
      \tl_map_inline:nn { {tl} {clist} {seq} {prop} {int} {dim} {skip} {muskip} }
        {
          \check_tmp:w { #1_new } { }
          \check_tmp:w { #1_show } { }
          \check_tmp:w { #1_log } { }
        }
      \tl_map_inline:nn { {tl} {clist} {seq} {prop} }
        {
          \check_tmp:w { #1_clear_new } { }
          \check_tmp:w { #1_gclear_new } { }
        }
      \tl_map_inline:nn { {int} {dim} {skip} {muskip} }
        {
          \check_tmp:w { #1_zero_new } { }
          \check_tmp:w { #1_gzero_new } { }
          \check_tmp:w { #1_const } { n }
        }
      \check_tmp:w { tl_const } { n }
      \check_tmp:w { tl_const } { x }
      \check_tmp:w { clist_const } { n }
    \group_end:
%    \end{macrocode}
% \end{macro}
%
% \begin{macro}[EXP]{\use:c}
%   Remove fine-tuning.  Here \cs{use:N} does not exist so use
%   \cs{use:n}.
%    \begin{macrocode}
    \cs_gset:Npn \use:c { \exp_args:Nc \use:n }
%    \end{macrocode}
% \end{macro}
%
% \begin{macro}[EXP]
%   {
%     \exp_not:o, \exp_not:c, \exp_not:f, \exp_not:V, \exp_not:v,
%     \exp_args:No, \exp_args:NNo, \exp_args:NNNo, \exp_args:Nc,
%     \exp_args:Ncc, \exp_args:Nccc, \exp_args:NNc, \exp_args:Nf,
%     \exp_args:NV, \exp_args:Nv, \exp_args:NNV, \exp_args:NNv,
%     \exp_args:NNf, \exp_args:NVV, \exp_args:Ncf, \exp_args:Nco,
%     \exp_args:Ncco, \exp_args:NcNc, \exp_args:NcNo, \exp_args:NNNV,
%     \exp_last_unbraced:NV, \exp_last_unbraced:Nv,
%     \exp_last_unbraced:Nf, \exp_last_unbraced:No,
%     \exp_last_unbraced:Nco, \exp_last_unbraced:NcV,
%     \exp_last_unbraced:NNV, \exp_last_unbraced:NNo,
%     \exp_last_unbraced:NNNV, \exp_last_unbraced:NNNo,
%   }
%   Remove hand-tuning, to make sure that our modified \cs{::N},
%   \cs{::V}, \cs{::o}, and so on are called.  We need to base
%   \cs{exp_not:c} on \cs{exp_not:N} and not \cs{exp_not:n}
%   because the |n| variant is a primitive, which does not
%   accept unbraced arguments.
%    \begin{macrocode}
    \cs_gset:Npn \exp_not:o { \exp_args:No \exp_not:n }
    \cs_gset:Npn \exp_not:c { \exp_args:Nc \exp_not:N }
    \cs_gset:Npn \exp_not:f { \exp_args:Nf \exp_not:n }
    \cs_gset:Npn \exp_not:V { \exp_args:NV \exp_not:n }
    \cs_gset:Npn \exp_not:v { \exp_args:Nv \exp_not:n }
    \cs_gset:Npn \exp_args:No { \::o \::: }
    \cs_gset:Npn \exp_args:NNo { \::N \::o \::: }
    \cs_gset:Npn \exp_args:NNNo { \::N \::N \::o \::: }
    \cs_gset:Npn \exp_args:Nc { \::c \::: }
    \cs_gset:Npn \exp_args:Ncc { \::c \::c \::: }
    \cs_gset:Npn \exp_args:Nccc { \::c \::c \::c \::: }
    \cs_gset:Npn \exp_args:NNc { \::N \::c \::: }
    \cs_gset:Npn \exp_args:Nf { \::f \::: }
    \cs_gset:Npn \exp_args:Nv { \::v \::: }
    \cs_gset:Npn \exp_args:NV { \::V \::: }
    \cs_gset:Npn \exp_args:NNf { \::N \::f \::: }
    \cs_gset:Npn \exp_args:NNv { \::N \::v \::: }
    \cs_gset:Npn \exp_args:NNV { \::N \::V \::: }
    \cs_gset:Npn \exp_args:Nco { \::c \::o \::: }
    \cs_gset:Npn \exp_args:Ncf { \::c \::f \::: }
    \cs_gset:Npn \exp_args:NVV { \::V \::V \::: }
    \cs_gset:Npn \exp_args:NNNV { \::N \::N \::V \::: }
    \cs_gset:Npn \exp_args:NcNc { \::c \::N \::c \::: }
    \cs_gset:Npn \exp_args:NcNo { \::c \::N \::o \::: }
    \cs_gset:Npn \exp_args:Ncco { \::c \::c \::o \::: }
    \cs_gset:Npn \exp_last_unbraced:NV { \::V_unbraced \::: }
    \cs_gset:Npn \exp_last_unbraced:Nv { \::v_unbraced \::: }
    \cs_gset:Npn \exp_last_unbraced:No { \::o_unbraced \::: }
    \cs_gset:Npn \exp_last_unbraced:Nf { \::f_unbraced \::: }
    \cs_gset:Npn \exp_last_unbraced:Nco { \::c \::o_unbraced \::: }
    \cs_gset:Npn \exp_last_unbraced:NcV { \::c \::V_unbraced \::: }
    \cs_gset:Npn \exp_last_unbraced:NNV { \::N \::V_unbraced \::: }
    \cs_gset:Npn \exp_last_unbraced:NNo { \::N \::o_unbraced \::: }
    \cs_gset:Npn \exp_last_unbraced:NNNV { \::N \::N \::V_unbraced \::: }
    \cs_gset:Npn \exp_last_unbraced:NNNo { \::N \::N \::o_unbraced \::: }
%    \end{macrocode}
% \end{macro}
%
% \begin{macro}[EXP]{\::N, \::V, \::V_unbraced}
%   Check that the argument is a single token.  The |V|-type expansion
%   already takes care of checking for existence.
%    \begin{macrocode}
    \check_patch:Npnn \::N #1 \::: #2#3 { {#1} \::: {#2} {#3} }
      { \check_is_N_exp:n {#3} }
    \check_patch:Npnn \::V #1 \::: #2#3 { {#1} \::: {#2} {#3} }
      { \check_is_N_exp:n {#3} }
    \check_patch:Npnn \::V_unbraced \::: #1#2 { \::: {#1} {#2} }
      { \check_is_N_exp:n {#2} }
%    \end{macrocode}
% \end{macro}
%
% \begin{macro}[EXP]{\::c}
%   Check existence of the function constructed here.
%    \begin{macrocode}
    \check_patch:Npnn \::c #1 \::: #2#3 { {#1} \::: {#2} {#3} }
      { \check_is_defined_exp:c {#3} }
%    \end{macrocode}
% \end{macro}
%
% \begin{macro}[EXP]
%   {\::o, \::f, \::o_unbraced, \::f_unbraced, \exp_last_two_unbraced:Noo}
% \begin{macro}[EXP]{\check_should_o_expand:n, \check_should_f_expand:n}
% \begin{macro}[aux, EXP]{\@@_if_head_is_expandable:nF, \@@_if_head_is_expandable_aux:n}
%   These functions expand an argument, once or fully.  If the argument
%   does not expand at all under the specified expansion type, warn the
%   user.
%    \begin{macrocode}
    \prg_new_conditional:Npnn \@@_if_head_is_expandable:n #1 { F }
      {
        \tl_if_head_is_N_type:nTF {#1}
          {
            \exp_after:wN \tl_head:n
              \exp_after:wN { \@@_if_head_is_expandable_aux:n #1 }
          }
          { \prg_return_false: }
      }
    \cs_new:Npn \@@_if_head_is_expandable_aux:n #1
      {
        {
          \token_if_expandable:NTF #1
            { \prg_return_true: }
            { \prg_return_false: }
        }
      }
    \cs_new:Npn \check_should_o_expand:n #1
      {
        \@@_if_head_is_expandable:nF {#1}
          { \@@_warning_exp:nnn { no-expansion } { o } {#1} }
      }
    \cs_new:Npn \check_should_f_expand:n #1
      {
        \tl_if_head_eq_catcode:nNF {#1} \c_space_token
          {
            \@@_if_head_is_expandable:nF {#1}
              { \@@_warning_exp:nnn { no-expansion } { f } {#1} }
          }
      }
    \check_patch:Npnn \::o #1 \::: #2#3 { {#1} \::: {#2} {#3} }
      { \check_should_o_expand:n {#3} }
    \check_patch:Npnn \::f #1 \::: #2#3 { {#1} \::: {#2} {#3} }
      { \check_should_f_expand:n {#3} }
    \check_patch:Npnn \::o_unbraced \::: #1#2 { \::: {#1} {#2} }
      { \check_should_o_expand:n {#2} }
    \check_patch:Npnn \::f_unbraced \::: #1#2 { \::: {#1} {#2} }
      { \check_should_f_expand:n {#2} }
    \check_patch:Npnn \exp_last_two_unbraced:Noo #1#2#3 { {#1} {#2} {#3} }
      {
        \check_is_N:n {#1}
        \check_should_o_expand:n {#2}
        \check_should_o_expand:n {#3}
      }
%    \end{macrocode}
% \end{macro}
% \end{macro}
% \end{macro}
%
% ^^A todo: extra checks for \cs{cs_generate_variant:Nn}.
%    \begin{macrocode}
    \check_patch:Nn \cs_generate_variant:Nn
      {
        \check_is_function:N #1
        \check_is_defined:N #1
      }
%    \end{macrocode}
%
% ^^A todo: move somewhere sensible, find better solution
%
% Fix back a few commands, double checking that their original
% definition is still what it was when this code was written.  Well, we
% do not check \cs{__char_set_catcode:Nn}'s definition, because it
% includes |^^@|, which is a can of worms.
%    \begin{macrocode}
    \group_begin:
      \cs_set:Npn \check_tmp:w #1#2 \q_mark #3 ~ #4 \q_stop #5#6#7#8
        {
          \scan_stop: \scan_stop: \fi:
          \exp_not:N \q_mark
          \exp_not:N \q_stop
          \exp_not:N #6
          \exp_not:c { #7 : #8 #1 #3 }
        }
      \cs_if_eq:NNTF \check_tmp:w \__cs_generate_variant_loop_end:nwwwNNnn
        {
          \cs_gset:Npn \__cs_generate_variant_loop_end:nwwwNNnn #1#2 \q_mark #3 ~ #4 \q_stop #5#6#7#8
            {
              \scan_stop: \scan_stop: \fi:
              \exp_not:N \q_mark
              \exp_not:N \q_stop
              \exp_not:N #6
              \exp_after:wN \exp_not:N \cs:w #7 : #8 #1 #3 \cs_end:
            }
        }
        { \@@_error:nn { internal } { Cannot~patch~\__cs_generate_variant_loop_end:nwwwNNnn } }
      \cs_set_protected:Npn \check_tmp:w #1#2#3#4#5
        {
          \tex_catcode:D `#1 = \__int_eval:w #2 \__int_eval_end:
          #3 { \tex_catcode:D `#1 / 2 } = { 6 }
            {
              \group_begin: \exp_args:NNc \group_end:
              \__char_set_catcode:NNN {#1} #4 #5
            }
        }
      \cs_if_eq:NNTF \check_tmp:w \__char_set_catcode:NnNNN
        {
          \cs_gset_protected:Npn \__char_set_catcode:NnNNN #1#2#3#4#5
            {
              \tex_catcode:D `#1 = \__int_eval:w #2 \__int_eval_end:
              #3 { \tex_catcode:D `#1 / 2 } = { 6 }
                {
                  \group_begin: \exp_after:wN \group_end:
                  \exp_after:wN \__char_set_catcode:NNN \cs:w #1 \cs_end: #4 #5
                }
            }
        }
        { \@@_error:nn { internal } { Cannot~patch~\__char_set_catcode:NnNNN } }
      \tl_if_exist:cT { @sanitize }
        {
          \use:x
            {
              \exp_not:n { \cs_gset_protected:Npn \__char_set_catcode:Nn #1#2 }
                {
                  \exp_not:n { \cs_set_eq:NN \exp_args:Nc \@@_old_exp_args:Nc }
                  \exp_not:o { \__char_set_catcode:Nn {#1} {#2} }
                  \exp_not:n { \cs_set:Npn \exp_args:Nc { \::c \::: } }
                }
            }
        }
    \group_end:
%    \end{macrocode}
% We do not redefine \cs{exp_args:cc} because
% \cs{prg_new_conditional:Npnn} and its friends use that function to
% generate new control sequences.
%
%    \begin{macrocode}
  }
%    \end{macrocode}
%
% \subsubsection{\pkg{l3tl}}
%
% ^^A todo: everything below
% \cs{tl_new:N}, \cs{tl_const:Nn}/Nx, \cs{tl_clear:N}/gclear,
% \cs{tl_clear_new:N}/gclear, |\tl_(set/gset)_eq:(N/c)(N/c)|,
% \cs{tl_concat:NNN}/gconcat, |\tl_if_exist:(N/c)(T/F/TF//p)|,
% |\tl_(set/gset):(Nn/No/Nx)|,
% |\tl_(put/gput)_(left/right):N(n/V/o/x)|,
% |\tl_(set/gset)_rescan:Nnn|, |\tl_(/g)replace_(once/all):Nnn|
% |\tl_if_(empty/single):N|(TF), \cs{tl_if_eq:NN}(TF),
% \cs{tl_if_in:Nn}(TF), |\tl_map_function:NN|, |\tl_map_inline:Nn|,
% |\tl_map_variable:NNn|, |\tl_(to_str/use/count/head/tail):N|,
% |\tl_(g/)trim_spaces:N|, |\tl_(g/)reverse:N|,
% |\tl_if_head_eq_(charcode/catcode/meaning):nN|(TF), |\tl_item:Nn|,
% |\tl_(show/log):N|, |\tl_(g/)set_from_file(/_x):Nnn|
%
% Remove the \cs{l@expl@check@declarations@bool} loop business since it
% would be covered by \pkg{l3check}.
%
% |(tl/str/int/dim)_case:(N/n)n|(TF) count brace groups.
%
% \subsubsection{\pkg{l3seq}}
%
% \cs{seq_new:N} should also check name of var.
%
% |\seq_(g/)clear(/_new):N|, |\seq_(g/)set_eq:(N/c)(N/c)|,
% |\seq_if_(exist/empty):(N/c)|(TF), |\seq_(g/)remove_duplicates:N|,
% |\seq_if_in:Nn|(TF), |\seq_item:Nn|, |\seq_count:N|,
% |\seq_use:Nn(/nn)|, |\seq_map_function:NN|,
% |\seq_mapthread_function:NNN|, |\seq_map_inline:Nn|,
% |\seq_map_variable:NNn|, |\seq_(show/log):N|
%
% The following abuse |\tl_(g/)set:N(n/x/f)|:
% |\seq_(g/)set_from_clist:N(N/n)|,
% |\seq_(g/)set_split:Nnn|,
% |\seq_(g/)concat:NNN|,
% |\seq_(g/)(put_left/put_right/push):Nn|,
% |\seq_(g/)reverse:N|,
% |\seq_(get/pop/gpop)_(left/right):NN|(TF),
% |\seq_(get/pop/gpop):NN|(TF),
% |\seq_(g/)set_(filter/map):NNn|
%
% The following abuses it more: |\seq_(g/)remove_all:Nn| (interrupting
% assignment and starting it again).
%
% \subsubsection{\pkg{l3int}}
%
% In principle we could make integer expressions a bit more robust by
% adding a bunch of parentheses.
%
% |\int_(show/log):N|, \cs{int_new:N}, |\int_(g/)(zero/incr/decr):N|, |\int_(g/)zero_new:N|, \cs{int_const:Nn},
% |\int_(g/)set_eq:NN|, |\int_if_exist:N|(TF), |\int_(g/)(add/sub/set):Nn|,
% |\int_compare:nNn|(TF), |\int_(while/until)_do:nNnn|, |\int_do_(while/until):nNnn|,
% |\int_step_function:nnnN|, |\int_step_variable:nnnNn|
%
% Would be nice to change but cannot because of expansion: |\int_use:N|.
%
% Deprecated: |\int_(to/from)_(binary/hexadecimal/octal):n|.
%
% \subsubsection{\pkg{l3quark}}
%
% \cs{quark_new:N} abuses \cs{tl_const:Nn}.
%
% \cs{quark_if_recursion_tail_stop:N}, \cs{quark_if_recursion_tail_stop_do:Nn},
% |\quark_if_(nil/no_value):N|(TF)
%
% \subsubsection{\pkg{l3prg}}
%
% \cs{bool_new:N}, |\bool_(g/)set_(true/false):N|,
% |\bool_(g/)set_eq:(N/c)(N/c)|, |\bool_(g/)set:Nn|, |\bool_if:N|(TF),
% |\bool_(show/log):N|, |\bool_if_exist:(N/c)(TF)|, |\bool_(while_do/until_do/do_while/do_until):Nn|
%
% Deprecated: |\scan_align_safe_stop:|
%
% \subsubsection{\pkg{l3clist}}
%
% |\clist_new:(N/c)|, |\clist_const:Nn| abuses |\tl_const:Nx|, |\clist_(g/)clear:(N/c)|, |\clist_(g/)clear_new:(N/c)|,
% |\clist_(g/)set_eq:(N/c)(N/c)|, |\clist_(g/)set_from_seq:NN| abuses |\tl_(g/)set:N?|, |\clist_(g/)concat:NNN| abuses |\tl_(g/)set:Nx|,
% |\clist_if_exist:(N/c)|(TF), |\clist_(g/)set:Nn| abuses |\tl_(g/)set:Nx|, |\clist_(g/)put_(left/right):Nn| just N-type, |\clist_(get/pop/gpop):NN|(TF) abuse |\tl_(g/)set:N(n/x)|, |\clist_(g/)push:Nn|, |\clist_(g/)remove_duplicates:N|, |\clist_(g/)remove_all:Nn|, |\clist_(g/)reverse:N|, |\clist_if_in:Nn|(TF), |\clist_map_function:(N/n)N|, |\clist_map_inline:Nn|, |\clist_map_variable:(N/n)Nn|, |\clist_count:N|, |\clist_use:Nn(/nn)|, |\clist_item:Nn|, |\clist_(show/log):N|
%
% \subsubsection{\pkg{l3token}}
%
% ^^A todo: the internal \__peek_get_prefix_arg_replacement:wN is misnamed: __peek should be __token
%
% |\char_set_catcode_(escape/group_begin/group_end/math_toggle/alignment/end_line/parameter/math_superscript/math_subscript/ignore/space/letter/other/active/comment/invalid):N|,
% |\token_if_(group_begin/group_end/math_toggle/alignment/parameter/math_superscript/math_subscript/space/letter/other/active/macro/cs/expandable/primitive/chardef/mathchardef/(dim/int/muskip/skip/toks)_register/(protected/long/protected_long)_macro):N|(TF),
% |\peek_(catcode/charcode/meaning)(/_remove)(/_ignore_spaces):N|(TF),
% |\token_new:Nn|, |\token_if_eq_(meaning/catcode/charcode):NN|(TF),
% |\peek_(g/)after:Nw|,
% |\token_get_(prefix/arg/replacement)_spec:N|,
% |\char_(g/)set_active:Np(n/x)|, |\char_(g/)set_active_eq:NN|,
%
% \subsubsection{\pkg{l3prop}}
%
% |\prop_new:N|, |\prop_(g/)clear:N|, |\prop_(g/)clear_new:N|,
% |\prop_(g/)set_eq:(N/c)(N/c)|, |\prop_item:Nn|,
% |\prop_if_exist:(N/c)|(TF), |\prop_if_empty:N|(TF),
% |\prop_if_in:Nn|(TF), |\prop_map_function:NN|, |\prop_map_tokens:Nn|,
% |\prop_map_inline:Nn|, |\prop_(show/log):N|
%
% The following abuse |\tl_set:Nn| et al: |\prop_(g/)remove:Nn|, |\prop_(get/pop/gpop):NnN|(TF), |\prop_(g/)put(/_if_new):Nnn|
%
% |\prop_get:(N/c)n| (deprecated!)
%
% \subsubsection{\pkg{l3msg}}
%
% |\msg_(new/set/gset):nnn(/n)|?
%
% \subsubsection{\pkg{l3file}}
%
% ^^A todo: should \file_add_path:nN be renamed to \file_get_path:nN ?
%
% |\file_add_path:nN|, |\io(r/w)_new:N|, |\io(r/w)_open:Nn|,
% |\ior_open:Nn|(TF), |\io(r/w)_close:N|, |\ior_get(/_str):NN|,
% |\iow_(now/shipout/shipout_x):Nn| (special case |\c_(log/term)_iow|?),
% |\iow_char:N|, |\iow_wrap:nnnN|,
% |\ior_(str_/)map_inline:Nn|,
%
% \subsubsection{\pkg{l3skip}}
%
% |(dim/skip/muskip)_new:N|
% |(dim/skip/muskip)_(g/)zero:N|
% |(dim/skip/muskip)_(g/)zero_new:N|
% |(dim/skip/muskip)_if_exist:(N/c)|
% |(dim/skip/muskip)_const:Nn| abuses new and gset
% |(dim/skip/muskip)_(g/)(set/add/sub):Nn|
% |(dim/skip/muskip)_(g/)set_eq:NN|
% |dim_abs:N|
% |dim_compare:nNn|(TF)
% |dim_(while_do/until_do/do_while/do_until):nNnn|
% |(dim/skip/muskip)_(show/log):(N/c)|
% |skip_(horizontal/vertical):N|
% |\skip_split_finite_else_action:nnNN| % ^^A todo: rename?
%
% Would be nice to change but cannot because of expansion: |(dim/skip/muskip)_use:N|.
%
% \subsubsection{\pkg{l3keys}}
% ^^A todo: add checking of args of key properties such as |.tl_set:N|
%
% |\keyval_parse:NNn|, |\keys_set_known:nnN|, |\keys_set_filter:nnnN|,
%
% |\keys_(show/log):nn|, |\keys_set:nn|, |\keys_set_known:nn(/N)|,
% |\keys_set_filter:nnnN|, |\keys_set_groups:nnn|, could check the
% module is known due to |\keys_define:nn|, \emph{i.e.}, that at least
% one key is known for this module.
%
% \subsubsection{\pkg{l3fp}}
%
% |\fp_function:Nn|, |\fp_new_function:Npn|, |\fp_if_exist:(N/c)|(TF),
% |\fp_compare:nNn|(TF), |\fp_(do_until/do_while/until_do/while_do):nNnn|
% |\fp_to_(scientific/decimal/tl/dim/int):N|, |\fp_use:N|,
%
% |\fp_new:N|, |\fp_(set/gset/const):Nn| abuses |\tl_(set/gset/const):Nx|
% |\fp_(g/)set_eq:NN|, |\fp_(g/)zero(/_new):N|, |\fp_(show/log):N|,
%
% |\fp_(g/)set_from_dim:Nn| (deprecated!)
%
% \subsubsection{\pkg{l3box}}
%
% |\box_new:N|, |\box_(g/)clear(/_new):N|,
% |\box_(g/)set_eq(/_clear):NN|, |\box_if_exist:(N/c)|(TF)
% |\box_(ht/dp/wd):N|, |\box_set_(dp/ht/wd):Nn|, |\box_use(/_clear):N|,
% |\box_if_(horizontal/vertical/empty):N|(TF),
% |\box_(g/)set_to_last:N|, |\box_(show/log):N(/nn)|,
% |\hbox_(g/)set(:Nw/:Nn/_to_wd:Nnn)|, |\hbox_unpack(/_clear):N|,
% |\vbox_set(:Nw/:Nn/_top:Nn/_to_ht:Nnn)|, |\vbox_unpack(/_clear):N|,
% |\vbox_set_split_to_ht:NNn|,
% |\box_rotate:Nn|, |\box_resize(/_to_wd_and_ht):Nnn|,
% |\box_resize_to_(ht/ht_plus_dp/wd):Nn|, |\box_scale:Nnn|,
% |\box_clip:N|, |\box_(trim/viewport):Nnnnn|,
%
% \subsubsection{\pkg{l3coffins}}
%
% |\coffin_if_exist:N|(TF), |\coffin_clear:N|, |\coffin_new:N|,
% |\hcoffin_set(:Nn/:Nw)|, |\vcoffin_set(:Nnn/:Nnw)|,
% |\coffin_set_eq:NN|, |\coffin_(dp/ht/wd):(N/c)|,
% |\coffin_set_(horizontal/vertical)_pole:Nnn|,
% |\coffin_(join/attach/attach_mark):NnnNnnnn|,
% |\coffin_typeset:Nnnnn|, |\coffin_mark_handle:Nnnn|,
% |\coffin_display_handles:Nn|, |\coffin_(show/log)_structure:N|,
% |\coffin_rotate:Nn|, |\coffin_(resize/scale):Nnn|
%
% \subsubsection{pkg{l3color}}
%
% Alter |\color_group_begin:| if we decide to do something with |\group_begin:|.
%
% ^^A todo: also provide \tl_set_from_file:Nnn as \file_contents_get:nnN ?
%
% \subsubsection{More stuff we could do}
%
% \begin{itemize}
%   \item We could check for a scope of \texttt{l} or \texttt{g} in
%     \cs{tl_new:N}.
%   \item The \cs{tl_if_exist:NTF} conditionals should also have a type
%     check, and their |c| variant should be a true variant.
%   \item The \texttt{rescan} functions could test that their argument
%     is rescan-safe perhaps.
%   \item \cs{tl_map_function:NN} could check that the second argument
%     has signature |:n|.
%   \item \cs{tl_map_break:} and other break functions could check if
%     they appear in a mapping, rather than break havoc otherwise.
% \end{itemize}
%
% \begin{macro}
%   {
%     \tl_set_eq:cN, \tl_set_eq:Nc, \tl_set_eq:cc,
%     \tl_gset_eq:cN, \tl_gset_eq:Nc, \tl_gset_eq:cc,
%   }
%   Redefine some functions as proper variants of their base functions.
%    \begin{macrocode}
    \cs_gset_protected:Npn \tl_set_eq:cN
      { \exp_args:Nc \tl_set_eq:NN }
    \cs_gset_protected:Npn \tl_set_eq:Nc
      { \exp_args:NNc \tl_set_eq:NN }
    \cs_gset_protected:Npn \tl_set_eq:cc
      { \exp_args:Ncc \tl_set_eq:NN }
    \cs_gset_protected:Npn \tl_gset_eq:cN
      { \exp_args:Nc \tl_gset_eq:NN }
    \cs_gset_protected:Npn \tl_gset_eq:Nc
      { \exp_args:NNc \tl_gset_eq:NN }
    \cs_gset_protected:Npn \tl_gset_eq:cc
      { \exp_args:Ncc \tl_gset_eq:NN }
%    \end{macrocode}
% \end{macro}
%
% \begin{macro}{\tl_new:N, \tl_show:N, \tl_const:Nn, \tl_const:Nx}
%   Protected functions whose first argument is a token list which may
%   be undefined.
%    \begin{macrocode}
    % \@@_patch_protected:Npnn \tl_new:N #1 { {#1} }
    %   { \@@_variable_type_only:nn {#1} { tl } }
    % \@@_patch_protected:Npnn \tl_show:N #1 { {#1} }
    %   { \@@_variable_type_only:nn {#1} { tl } }
    % \@@_patch_protected:Npnn \tl_const:Nn #1 { {#1} }
    %   { \@@_variable_scope_type_only:nnn {#1} { c } { tl } }
    % \@@_patch_protected:Npnn \tl_const:Nx #1 { {#1} }
    %   { \@@_variable_scope_type_only:nnn {#1} { c } { tl } }
%    \end{macrocode}
% \end{macro}
%
% \begin{macro}[EXP, pTF]{\tl_if_exist:N}
%   Expandable test whose first argument is a token list which may be
%   undefined.
%    \begin{macrocode}
    % \tl_map_inline:nn
    %   { \tl_if_exist:NTF \tl_if_exist:NT \tl_if_exist:NF \tl_if_exist_p:N }
    %   {
    %     \@@_patch:Npnn #1 ##1 { {##1} }
    %       { \@@_variable_type_only_exp:nn {##1} { tl } }
    %   }
%    \end{macrocode}
% \end{macro}
%
% \begin{macro}
%   {
%     \tl_set:Nn, \tl_set:No, \tl_set:Nx,
%     \tl_put_left:Nn, \tl_put_left:NV,
%     \tl_put_left:No, \tl_put_left:Nx,
%     \tl_put_right:Nn, \tl_put_right:NV,
%     \tl_put_right:No, \tl_put_right:Nx,
%     \tl_set_rescan:Nnn,
%     \tl_replace_once:Nnn, \tl_replace_all:Nnn,
%     \tl_trim_spaces:N, \tl_reverse:N,
%   }
%   Protected functions whose first argument is a defined, local token list.
%    \begin{macrocode}
    % \tl_map_inline:nn
    %   {
    %     \tl_set:Nn \tl_set:No \tl_set:Nx
    %     \tl_put_left:Nn \tl_put_left:NV \tl_put_left:No \tl_put_left:Nx
    %     \tl_put_right:Nn \tl_put_right:NV \tl_put_right:No \tl_put_right:Nx
    %     \tl_set_rescan:Nnn
    %     \tl_replace_once:Nnn \tl_replace_all:Nnn
    %     \tl_trim_spaces:N \tl_reverse:N
    %   }
    %   {
    %     \@@_patch_protected:Npnn #1 ##1 { {##1} }
    %       { \@@_variable_scope_type:nnn {##1} { l } { tl } }
    %   }
%    \end{macrocode}
% \end{macro}
%
% \begin{macro}
%   {
%     \tl_gset:Nn, \tl_gset:No, \tl_gset:Nx,
%     \tl_gput_left:Nn, \tl_gput_left:NV,
%     \tl_gput_left:No, \tl_gput_left:Nx,
%     \tl_gput_right:Nn, \tl_gput_right:NV,
%     \tl_gput_right:No, \tl_gput_right:Nx,
%     \tl_gset_rescan:Nnn,
%     \tl_greplace_once:Nnn, \tl_greplace_all:Nnn,
%     \tl_gtrim_spaces:N, \tl_greverse:N,
%   }
%   Protected functions whose first argument is a defined, global token list.
%    \begin{macrocode}
    % \tl_map_inline:nn
    %   {
    %     \tl_gset:Nn \tl_gset:No \tl_gset:Nx
    %     \tl_gput_left:Nn \tl_gput_left:NV
    %     \tl_gput_left:No \tl_gput_left:Nx
    %     \tl_gput_right:Nn \tl_gput_right:NV
    %     \tl_gput_right:No \tl_gput_right:Nx
    %     \tl_gset_rescan:Nnn
    %     \tl_greplace_once:Nnn \tl_greplace_all:Nnn
    %     \tl_gtrim_spaces:N \tl_greverse:N
    %   }
    %   {
    %     \@@_patch_protected:Npnn #1 ##1 { {##1} }
    %       { \@@_variable_scope_type:nnn {##1} { g } { tl } }
    %   }
%    \end{macrocode}
% \end{macro}
%
% \begin{macro}{\tl_set_eq:NN, \tl_gset_eq:NN}
%   Protected functions whose first argument is a defined, local or
%   global token list, and whose second argument is a defined token
%   list.  Importantly, |#1| and |#2| are not braced in the first |n|
%   argument of \cs{@@_patch_protected:Npnn}.
%    \begin{macrocode}
    % \@@_patch_protected:Npnn \tl_set_eq:NN #1#2 { #1 #2 }
    %   {
    %     \@@_variable_scope_type:nnn {#1} { l } { tl }
    %     \@@_variable_type:nn {#2} { tl }
    %   }
    % \@@_patch_protected:Npnn \tl_gset_eq:NN #1#2 { #1 #2 }
    %   {
    %     \@@_variable_scope_type:nnn {#1} { g } { tl }
    %     \@@_variable_type:nn {#2} { tl }
    %   }
%    \end{macrocode}
% \end{macro}
%
% \begin{macro}{\tl_concat:NNN, \tl_gconcat:NNN}
%   Protected functions whose three arguments are defined token lists,
%   with a first argument local or global depending on the function.
%    \begin{macrocode}
    % \@@_patch_protected:Npnn \tl_concat:NNN #1#2#3 { {#1} {#2} {#3} }
    %   {
    %     \@@_variable_scope_type:nnn {#1} { l } { tl }
    %     \@@_variable_type:nn {#2} { tl }
    %     \@@_variable_type:nn {#3} { tl }
    %   }
    % \@@_patch_protected:Npnn \tl_gconcat:NNN #1#2#3 { {#1} {#2} {#3} }
    %   {
    %     \@@_variable_scope_type:nnn {#1} { g } { tl }
    %     \@@_variable_type:nn {#2} { tl }
    %     \@@_variable_type:nn {#3} { tl }
    %   }
%    \end{macrocode}
% \end{macro}
%
% \begin{macro}
%   {
%     \tl_map_inline:Nn, \tl_map_variable:NNn,
%     \tl_if_in:NnTF, \tl_if_in:NnT, \tl_if_in:NnF,
%   }
%   Protected functions whose first argument is a defined token list.
%    \begin{macrocode}
    % \tl_map_inline:nn
    %   {
    %     \tl_map_inline:Nn \tl_map_variable:NNn
    %     \tl_if_in:NnTF \tl_if_in:NnT \tl_if_in:NnF
    %   }
    %   {
    %     \@@_patch_protected:Npnn #1 ##1 { {##1} }
    %       { \@@_variable_type:nn {##1} { tl } }
    %   }
%    \end{macrocode}
% \end{macro}
%
% \begin{macro}[EXP, pTF]{\tl_if_empty:N, \tl_if_single:N}
% \begin{macro}[EXP]{\tl_map_function:NN, \tl_head:N, \tl_tail:N}
%   Expandable functions whose first argument is a defined token list.
%    \begin{macrocode}
    % \tl_map_inline:nn
    %   {
    %     \tl_if_empty_p:N  \tl_if_empty:NTF
    %     \tl_if_empty:NT   \tl_if_empty:NF
    %     \tl_if_single_p:N \tl_if_single:NTF
    %     \tl_if_single:NT  \tl_if_single:NF
    %     \tl_map_function:NN \tl_head:N \tl_tail:N
    %   }
    %   {
    %     \@@_patch:Npnn #1 ##1 { {##1} }
    %       { \@@_variable_type_exp:nn {##1} { tl } }
    %   }
%    \end{macrocode}
% \end{macro}
% \end{macro}
%
% \begin{macro}[EXP, pTF]{\tl_if_eq:NN}
%   Expandable functions whose two arguments are defined token lists.
%    \begin{macrocode}
    % \tl_map_inline:nn
    %   { \tl_if_eq_p:NN \tl_if_eq:NNTF \tl_if_eq:NNT \tl_if_eq:NNF }
    %   {
    %     \@@_patch:Npnn #1 ##1##2 { {##1} {##2} }
    %       {
    %         \@@_variable_type_exp:nn {##1} { tl }
    %         \@@_variable_type_exp:nn {##2} { tl }
    %       }
    %   }
%    \end{macrocode}
% \end{macro}
%
% \begin{macro}[EXP]{\tl_map_function:nN}
% \begin{macro}{\tl_map_variable:nNn}
%   An expandable and a protected functions whose second argument is defined.
%    \begin{macrocode}
    % \@@_patch:Npnn \tl_map_function:nN #1#2 { {#1} {#2} }
    %   { \@@_cs_exist_exp:n {#2} }
    % \@@_patch_protected:Npnn \tl_map_variable:nNn #1#2 { {#1} {#2} }
    %   { \@@_cs_exist:n {#2} }
%    \end{macrocode}
% \end{macro}
% \end{macro}
%
% \begin{macro}[EXP]{\tl_to_str:N}
%   We need to redefine that one from scratch, since it may be expected
%   to expand in two steps.
%    \begin{macrocode}
    % \cs_gset:Npn \tl_to_str:N #1
    %   {
    %     \etex_detokenize:D
    %       \@@_if_on:nT { l3tl }
    %         { \@@_variable_type_exp:nn {#1} { tl } }
    %       \exp_after:wN {#1}
    %   }
%    \end{macrocode}
% \end{macro}
%
% \begin{itemize}
%   \item The \cs{seq_if_exist:NTF} conditionals should also have a type
%     check, and their |c| variant should be a true variant.
% \end{itemize}
%
% \begin{macro}{\seq_new:N, \seq_show:N}
%   Protected functions whose first argument is a sequence which may be
%   undefined.
%    \begin{macrocode}
    % \@@_patch_protected:Npnn \seq_new:N #1 { {#1} }
    %   { \@@_variable_type_only:nn {#1} { seq } }
    % \@@_patch_protected:Npnn \seq_show:N #1 { {#1} }
    %   { \@@_variable_type_only:nn {#1} { seq } }
%    \end{macrocode}
% \end{macro}
%
% \begin{macro}[EXP, pTF]{\seq_if_exist:N}
%   Expandable test whose first argument is a sequence which may be
%   undefined.
%    \begin{macrocode}
    % \tl_map_inline:nn
    %   { \seq_if_exist:NTF \seq_if_exist:NT \seq_if_exist:NF \seq_if_exist_p:N }
    %   {
    %     \@@_patch:Npnn #1 ##1 { {##1} }
    %       { \@@_variable_type_only_exp:nn {##1} { seq } }
    %   }
%    \end{macrocode}
% \end{macro}
%
% \begin{macro}
%   {
%     \seq_set_split:Nnn,
%     \seq_put_left:Nn, \seq_put_right:Nn, \seq_push:Nn,
%     \seq_remove_duplicates:N, \seq_remove_all:Nn
%   }
%   Protected functions whose first argument is a defined, local sequence.
%    \begin{macrocode}
    % \tl_map_inline:nn
    %   {
    %     \seq_set_split:Nnn
    %     \seq_put_left:Nn \seq_put_right:Nn \seq_push:Nn
    %     \seq_remove_duplicates:N \seq_remove_all:Nn
    %   }
    %   {
    %     \@@_patch_protected:Npnn #1 ##1 { {##1} }
    %       { \@@_variable_scope_type:nnn {##1} { l } { seq } }
    %   }
%    \end{macrocode}
% \end{macro}
%
% \begin{macro}
%   {
%     \seq_gset_split:Nnn,
%     \seq_gput_left:Nn, \seq_gput_right:Nn, \seq_gpush:Nn,
%     \seq_gremove_duplicates:N, \seq_gremove_all:Nn,
%   }
%   Check that arguments are defined, have the right type and scope.
%    \begin{macrocode}
    % \tl_map_inline:nn
    %   {
    %     \seq_gset_split:Nnn
    %     \seq_gput_left:Nn \seq_gput_right:Nn \seq_gpush:Nn
    %     \seq_gremove_duplicates:N \seq_gremove_all:Nn
    %   }
    %   {
    %     \@@_patch_protected:Npnn #1 ##1 { {##1} }
    %       { \@@_variable_scope_type:nnn {##1} { g } { seq } }
    %   }
%    \end{macrocode}
% \end{macro}
%
% \begin{macro}{\seq_set_eq:NN, \seq_gset_eq:NN}
%    \begin{macrocode}
    % \@@_patch_protected:Npnn \seq_set_eq:NN #1#2 { #1 #2 }
    %   {
    %     \@@_variable_scope_type:nnn {#1} { l } { seq }
    %     \@@_variable_type:nn {#2} { seq }
    %   }
    % \@@_patch_protected:Npnn \seq_gset_eq:NN #1#2 { #1 #2 }
    %   {
    %     \@@_variable_scope_type:nnn {#1} { g } { seq }
    %     \@@_variable_type:nn {#2} { seq }
    %   }
%    \end{macrocode}
% \end{macro}
%
% \begin{macro}{\seq_concat:NNN, \seq_gconcat:NNN}
%    \begin{macrocode}
    % \@@_patch_protected:Npnn \seq_concat:NNN #1#2#3 { {#1} {#2} {#3} }
    %   {
    %     \@@_variable_scope_type:nnn {#1} { l } { seq }
    %     \@@_variable_type:nn {#2} { seq }
    %     \@@_variable_type:nn {#3} { seq }
    %   }
    % \@@_patch_protected:Npnn \seq_gconcat:NNN #1#2#3 { {#1} {#2} {#3} }
    %   {
    %     \@@_variable_scope_type:nnn {#1} { g } { seq }
    %     \@@_variable_type:nn {#2} { seq }
    %     \@@_variable_type:nn {#3} { seq }
    %   }
%    \end{macrocode}
% \end{macro}
%
% \begin{macro}{\seq_new:c}
%   Some functions should be redefined as true variants of their base
%   function.
%    \begin{macrocode}
    \cs_gset_protected:Npn \seq_new:c
      { \exp_args:Nc \seq_new:N }
    \cs_gset_protected:Npn \seq_clear:c
      { \exp_args:Nc \seq_clear:N }
    \cs_gset_protected:Npn \seq_gclear:c
      { \exp_args:Nc \seq_gclear:N }
    \cs_gset_protected:Npn \seq_clear_new:c
      { \exp_args:Nc \seq_clear_new:N }
    \cs_gset_protected:Npn \seq_gclear_new:c
      { \exp_args:Nc \seq_gclear_new:N }
    \cs_gset_protected:Npn \seq_set_eq:cN
      { \exp_args:Nc \seq_set_eq:NN }
    \cs_gset_protected:Npn \seq_set_eq:Nc
      { \exp_args:NNc \seq_set_eq:NN }
    \cs_gset_protected:Npn \seq_set_eq:cc
      { \exp_args:Ncc \seq_set_eq:NN }
    \cs_gset_protected:Npn \seq_gset_eq:cN
      { \exp_args:Nc \seq_gset_eq:NN }
    \cs_gset_protected:Npn \seq_gset_eq:Nc
      { \exp_args:NNc \seq_gset_eq:NN }
    \cs_gset_protected:Npn \seq_gset_eq:cc
      { \exp_args:Ncc \seq_gset_eq:NN }
%    \end{macrocode}
% \end{macro}
%
% \begin{macro}[pTF]{\seq_if_empty:c, \seq_if_exist:c}
%   Expandable functions which should be redefined as true variants of
%   their base function.
%    \begin{macrocode}
    \cs_gset:Npn \seq_if_empty:cTF
      { \exp_args:Nc \seq_if_empty:NTF }
    \cs_gset:Npn \seq_if_empty:cT
      { \exp_args:Nc \seq_if_empty:NT }
    \cs_gset:Npn \seq_if_empty:cF
      { \exp_args:Nc \seq_if_empty:NF }
    \cs_gset:Npn \seq_if_empty_p:c
      { \exp_args:Nc \seq_if_empty_p:N }
    \cs_gset:Npn \seq_if_exist:cTF
      { \exp_args:Nc \seq_if_exist:NTF }
    \cs_gset:Npn \seq_if_exist:cT
      { \exp_args:Nc \seq_if_exist:NT }
    \cs_gset:Npn \seq_if_exist:cF
      { \exp_args:Nc \seq_if_exist:NF }
    \cs_gset:Npn \seq_if_exist_p:c
      { \exp_args:Nc \seq_if_exist_p:N }
%    \end{macrocode}
% \end{macro}
%
% \begin{macro}
%   {
%     \seq_map_inline:Nn,
%     \seq_if_in:NnTF, \seq_if_in:NnT, \seq_if_in:NnF
%   }
%   Protected functions whose first argument is a defined sequence.
%    \begin{macrocode}
    % \tl_map_inline:nn
    %   { \seq_map_inline:Nn \seq_if_in:NnTF \seq_if_in:NnT \seq_if_in:NnF }
    %   {
    %     \@@_patch_protected:Npnn #1 ##1 { {##1} }
    %       { \@@_variable_type:nn {##1} { seq } }
    %   }
%    \end{macrocode}
% \end{macro}
%
% \begin{macro}[EXP, pTF]{\seq_if_empty:N, \seq_if_eq:NN}
%   For these two tests, we check the type, expandably.
%    \begin{macrocode}
    % \tl_map_inline:nn
    %   {
    %     \seq_if_empty_p:N \seq_if_empty:NTF
    %     \seq_if_empty:NT \seq_if_empty:NF
    %   }
    %   {
    %     \@@_patch:Npnn #1 ##1 { {##1} }
    %       { \@@_variable_type_exp:nn {##1} { seq } }
    %   }
    % \tl_map_inline:nn
    %   { \seq_if_eq_p:NN \seq_if_eq:NNTF \seq_if_eq:NNT \seq_if_eq:NNF }
    %   {
    %     \@@_patch:Npnn #1 ##1##2 { {##1} {##2} }
    %       {
    %         \@@_variable_type_exp:nn {##1} { seq }
    %         \@@_variable_type_exp:nn {##2} { seq }
    %       }
    %   }
%    \end{macrocode}
% \end{macro}
%
% \begin{macro}
%   {
%     \seq_set_split:Nnn, \seq_gset_split:Nnn,
%     \seq_pop_left:NN, \seq_gpop_left:NN, \seq_pop:NN, \seq_gpop:NN,
%   }
%   We redefine those to use |\cs_| functions rather than |\tl_|
%   functions.
%    \begin{macrocode}
    % \cs_gset_protected:Npn \seq_set_split:Nnn
    %   { \__seq_set_split:NNnn \cs_set_nopar:Npx }
    % \cs_gset_protected:Npn \seq_gset_split:Nnn
    %   { \__seq_set_split:NNnn \cs_gset_nopar:Npx }
    % \cs_gset_protected:Npn \seq_pop_left:NN
    %   { \__seq_pop:NNNN \__seq_pop_left:NNN \@@_tl_set:Nn }
    % \cs_gset_protected:Npn \seq_gpop_left:NN
    %   { \__seq_pop:NNNN \__seq_pop_left:NNN \@@_tl_gset:Nn }
    % \cs_gset_eq:NN \seq_pop:NN \seq_pop_left:NN
    % \cs_gset_eq:NN \seq_gpop:NN \seq_gpop_left:NN
%    \end{macrocode}
% \end{macro}
%
% \begin{macro}
%   {
%     \seq_map_variable:NNn ,
%     \seq_get_left:NN    , \seq_get_left:NNT    ,
%     \seq_get_left:NNF   , \seq_get_left:NNTF   ,
%     \seq_get_right:NN   , \seq_get_right:NNT   ,
%     \seq_get_right:NNF  , \seq_get_right:NNTF  ,
%     \seq_get:NN, \seq_get:NNT, \seq_get:NNF, \seq_get:NNTF,
%     \seq_pop_left:NN    , \seq_pop_left:NNT    ,
%     \seq_pop_left:NNF   , \seq_pop_left:NNTF   ,
%     \seq_pop_right:NN   , \seq_pop_right:NNT   ,
%     \seq_pop_right:NNF  , \seq_pop_right:NNTF  ,
%     \seq_pop:NN, \seq_pop:NNT, \seq_pop:NNF, \seq_pop:NNTF,
%     \seq_gpop_left:NN   , \seq_gpop_left:NNT   ,
%     \seq_gpop_left:NNF  , \seq_gpop_left:NNTF  ,
%     \seq_gpop_right:NN  , \seq_gpop_right:NNT  ,
%     \seq_gpop_right:NNF , \seq_gpop_right:NNTF ,
%     \seq_gpop:NN, \seq_gpop:NNT, \seq_gpop:NNF, \seq_gpop:NNTF,
%   }
%   One |seq| and one |tl| arguments.  The |pop| functions expect a
%   local sequence.  The |gpop| functions expect a global sequence.
%   Since the \cs{seq_map_variable:NNn} needs the same checks as the
%   |get| functions, we patch it here too.
%    \begin{macrocode}
    % \tl_map_inline:nn
    %   {
    %     \seq_map_variable:NNn
    %     \seq_get_left:NN \seq_get_left:NNT
    %     \seq_get_left:NNF \seq_get_left:NNTF
    %     \seq_get_right:NN \seq_get_right:NNT
    %     \seq_get_right:NNF \seq_get_right:NNTF
    %     \seq_get:NN \seq_get:NNT \seq_get:NNF \seq_get:NNTF
    %   }
    %   {
    %     \@@_patch_protected:Npnn #1 ##1##2 { {##1} {##2} }
    %       {
    %         \@@_variable_type:nn {##1} { seq }
    %         \@@_variable_type:nn {##2} { tl }
    %       }
    %   }
    % \tl_map_inline:nn
    %   {
    %     \seq_pop_left:NN \seq_pop_left:NNT
    %     \seq_pop_left:NNF \seq_pop_left:NNTF
    %     \seq_pop_right:NN \seq_pop_right:NNT
    %     \seq_pop_right:NNF \seq_pop_right:NNTF
    %     \seq_pop:NN \seq_pop:NNT \seq_pop:NNF \seq_pop:NNTF
    %   }
    %   {
    %     \@@_patch_protected:Npnn #1 ##1##2 { {##1} {##2} }
    %       {
    %         \@@_variable_scope_type:nnn {##1} { l } { seq }
    %         \@@_variable_type:nn {##2} { tl }
    %       }
    %   }
    % \tl_map_inline:nn
    %   {
    %     \seq_gpop_left:NN \seq_gpop_left:NNT
    %     \seq_gpop_left:NNF \seq_gpop_left:NNTF
    %     \seq_gpop_right:NN \seq_gpop_right:NNT
    %     \seq_gpop_right:NNF \seq_gpop_right:NNTF
    %     \seq_gpop:NN \seq_gpop:NNT \seq_gpop:NNF \seq_gpop:NNTF
    %   }
    %   {
    %     \@@_patch_protected:Npnn #1 ##1##2 { {##1} {##2} }
    %       {
    %         \@@_variable_scope_type:nnn {##1} { g } { seq }
    %         \@@_variable_type:nn {##2} { tl }
    %       }
    %   }
%    \end{macrocode}
% \end{macro}
%
% \begin{macro}[EXP]{\seq_map_function:NN}
%   Check that |#1| is a |seq| and |#2| is defined.
%    \begin{macrocode}
    % \@@_patch:Npnn \seq_map_function:NN #1#2 { {#1} {#2} }
    %   {
    %     \@@_variable_type_exp:nn {#1} { seq }
    %     \@@_cs_exist_exp:n {#2}
    %   }
%    \end{macrocode}
% \end{macro}
%
% \subsection{Let us go!}
%
%    \begin{macrocode}
\check_on:n { l3kernel }
%    \end{macrocode}
%
% \subsection{\texttt{l3kernel} test files failures}
%
% Note: this section has not been updated recently.
% \begin{itemize}
%   \item \file{m3basics002}: \cs{prg_new_conditional:Npnn} with an
%     unknown or empty condition leads to \cs{use:c} with undefined
%     argument.
%   \item \file{m3compare001}: an unknown comparison operator leads to
%     \cs{use:c} with undefined argument.
%   \item some \file{m3fp...}: raising an exception leads to a
%     \cs{use:c} warning.  The implementation should use \cs{cs:w}
%     \ldots{} \cs{cs_end:}.
%   \item \file{m3int001}, \file{m3int002}: \cs{int_from_base:nn} should
%     not |f|-expand its first argument!
%   \item A bunch of testfiles expand their arg with no good reason.
% \end{itemize}
%
%    \begin{macrocode}
%</initex|package>
%    \end{macrocode}
%
% \end{implementation}
%
% \PrintIndex
