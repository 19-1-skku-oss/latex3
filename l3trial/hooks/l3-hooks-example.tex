
\immediate\write18{tex l3hooks.dtx}

\documentclass{article}
\errorcontextlines=999
\makeatletter
\@namedef{ver@l3toks.sty}{}
\makeatother
\usepackage{expl3,l3hooks}
\begin{document}

\ExplSyntaxOn
\hook_post_push:nn {math} {@}
\hook_pre_push:nn {math} {?}
\everymath={!}
$a+b=c$\par

\hook_pre_pop:n {math}
\hook_post_pop:n {math}
$a+b=c$
\ExplSyntaxOff

\[xy=z\]

\ExplSyntaxOn
\hook_post_push:nn {par} {\makebox[0pt][r]{\P[\space]}}
\ExplSyntaxOff

\section{Test par hook}
There is no just ground, therefore, for the charge brought against me by
certain ignoramuses---that I have never written a moral tale, or, in more
precise words, a tale with a moral. They are not the critics predestined
to bring me out, and \emph{develop} my morals:---that is the secret. By and by
the ``North American Quarterly Humdrum'' will make them ashamed of their
stupidity. In the meantime, by way of staying execution---by way of
mitigating the accusations against me---I offer the sad history appended,---
a history about whose obvious moral there can be no question whatever,
since he who runs may read it in the large capitals which form the title
of the tale. I should have credit for this arrangement---a far wiser one
than that of La Fontaine and others, who reserve the impression to be
conveyed until the last moment, and thus sneak it in at the fag end of
their fables.

\ExplSyntaxOn
\hook_post_pop:n {par}
\ExplSyntaxOff

\section{Boxes}

\par
\begingroup
\ExplSyntaxOn
\fbox{\vbox{
  \hook_post_push:nn {hbox} {!}
  \hbox{a}
  \hbox{b}
  \hook_post_pop:n {hbox}
}}
\fbox{\vbox{
  \hook_post_once:nn {hbox} {!}
  \hbox{a}
  \hbox{b}
}}
\ExplSyntaxOff
\endgroup


\end{document}