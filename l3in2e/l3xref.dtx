% \iffalse
%% File: l3xref.dtx Copyright (C) 2006,2009 LaTeX3 project
%%
%% It may be distributed and/or modified under the conditions of the
%% LaTeX Project Public License (LPPL), either version 1.3c of this
%% license or (at your option) any later version.  The latest version
%% of this license is in the file
%%
%%    http://www.latex-project.org/lppl.txt
%%
%% This file is part of the ``expl3 bundle'' (The Work in LPPL)
%% and all files in that bundle must be distributed together.
%%
%% The released version of this bundle is available from CTAN.
%%
%% -----------------------------------------------------------------------
%%
%% The development version of the bundle can be found at
%%
%%    http://www.latex-project.org/cgi-bin/cvsweb.cgi/
%%
%% for those people who are interested.
%%
%%%%%%%%%%%
%% NOTE: %%
%%%%%%%%%%%
%%
%%   Snapshots taken from the repository represent work in progress and may
%%   not work or may contain conflicting material!  We therefore ask
%%   people _not_ to put them into distributions, archives, etc. without
%%   prior consultation with the LaTeX Project Team.
%%
%% -----------------------------------------------------------------------
%
%<*driver|package|testfile>
\RequirePackage{l3names}
%</driver|package|testfile>
%\fi
\GetIdInfo$Id$
       {L3 Experimental cross referencing}
%\iffalse
%<*testfile>
\ProvidesFile{\jobname.tex}
  [\filedate\space v\fileversion\space\filedescription]
%</testfile>
%<*driver>
%\fi
\ProvidesFile{\filename.\filenameext}
  [\filedate\space v\fileversion\space\filedescription]
%\iffalse
\documentclass[full]{l3doc}
\begin{document}
\DocInput{\filename.\filenameext}
\end{document}
%</driver>
% \fi
%
%
% \title{The \textsf{l3xref} package\thanks{This file
%         has version number \fileversion, last
%         revised \filedate.}\\
% Cross references}
% \author{\Team}
% \date{\filedate}
% \maketitle
%
%
% \section{Cross references}
%
% \begin{function}{\xref_set_label:n}
%   \begin{syntax}
%     "\xref_set_label:n" \Arg{name}
%   \end{syntax}
%   Sets a label in the text. Note that this function does not do
%   anything else than setting the correct labels. In particular, it
%   does not try to fix any spacing around the write node; this is a
%   task for the \textsf{galley2} module.
% \end{function}
%
% \begin{function}{\xref_new:nn}
%   \begin{syntax}
%     "\xref_new:nn" \Arg{type} \Arg{value}
%   \end{syntax}
%   Defines a new cross reference type <type>. This defines the token
%   list pointer |\l_xref_curr_|<type>|_tlp| with default value
%   <value> which gets written fully expanded when |\xref_set_label:n|
%   is called.
% \end{function}
%
%
% \begin{function}{\xref_deferred_new:nn}
%   \begin{syntax}
%     "\xref_deferred_new:nn" \Arg{type} \Arg{value}
%   \end{syntax}
%   Same as |\xref_new:n| except for this one, the value written
%   happens when \TeX\ ships out the page. Page numbers use this one
%   obviously.
% \end{function}
%
% \begin{function}{\xref_get_value:nn / (EXP)}
%   \begin{syntax}
%     "\xref_get_value:nn" \Arg{type} \Arg{name}
%   \end{syntax}
%   Extracts the cross reference information of type <type> for the
%   label <name>. This operation is expandable.
% \end{function}
%
%
% \StopEventually{}
%
% \subsection{Internal functions and variables}
%
% \begin{variable}{ \g_xref_all_curr_immediate_fields_prop |
%                   \g_xref_all_curr_deferred_fields_prop  }
% What they say they are ":)"
% \end{variable}
%
% \begin{variable}{ \xref_write  }
% A stream for writing cross references, although they are not required
% to be in a separate file.
% \end{variable}
%
% \begin{function}{\xref_define_label:nn}
%   \begin{syntax}
%     "\xref_define_label:nn" \Arg{name} \Arg{plist contents}
%   \end{syntax}
% Define the property list for each label; used internally by 
% "\xref_set_label:n".
% \end{function}
%
% \subsection{Implementation}
%
%
%    We start by ensuring that the required packages are loaded.
%    \begin{macrocode}
%<*package>
\ProvidesExplPackage
  {\filename}{\filedate}{\fileversion}{\filedescription}
\RequirePackage{l3quark}
\RequirePackage{l3toks}
\RequirePackage{l3io}
\RequirePackage{l3prop}
\RequirePackage{l3int}
\RequirePackage{l3token}
%</package>
%<*initex|package>
%    \end{macrocode}
%
%
%
% There are two kinds of information, namely information which is
% \emph{immediate} like a section title and then there's \emph{deferred}
% information like page numbers. Each reference type belong to one of
% these categories, which we save internally as the property lists
% |\g_xref_all_curr_immediate_fields_prop| and
% |\g_xref_all_curr_deferred_fields_prop| and the reference type
% \meta{xyz} exists as the key-info pair |\xref_|\meta{xyz}|_key|
% |{\l_xref_curr_|\meta{xyz}|_tlp}| on one of these lists. This way each
% new entry type is just added as another key-info pair. 
%
% When the cross references are generated at the beginning of the
% document each will turn into a control sequence. Thus |\label{mylab}|
% will internally refer to the property list |\g_xref_mylab_prop|.
%
% The extraction of values from this property list can be done in
% several different ways but we want to keep the operation
% expandable. Therefore we use a dedicated function for each type of
% cross reference, which looks like this:
% \begin{verbatim}
% \xref_get_value_xyz_aux:w -> #1 \xref_xyz_key #2#3\q_nil{#2}
% \end{verbatim}
% This will throw away all the bits we don't need. In case |xyz| is
% the first on the |mylab| property list |#1| is empty, if it's the
% last key-info pair |#3| is empty. The value of the field can be
% extracted with the function |\xref_get_value:nn| where the first
% argument is the type and the second the label name so here it would
% be |\xref_get_value:nn| |{xyz}| |{mylab}|.
%
%  \begin{macro}{\g_xref_all_curr_immediate_fields_prop}
%  \begin{macro}{\g_xref_all_curr_deferred_fields_prop}
%    The two main property lists for storing information. They contain
%    key-info pairs for all known types.
%    \begin{macrocode}
\prop_new:N \g_xref_all_curr_immediate_fields_prop
\prop_new:N \g_xref_all_curr_deferred_fields_prop
%    \end{macrocode}
%  \end{macro}
%  \end{macro}
%
%  \begin{macro}{\xref_new:nn}
%  \begin{macro}{\xref_deferred_new:nn}
%  \begin{macro}[aux]{\xref_new_aux:nnn}
%  Setting up a new cross reference type is fairly straight forward
%  when we follow the game plan mentioned earlier.
%    \begin{macrocode}
\cs_new_nopar:Npn \xref_new:nn {\xref_new_aux:nnn{immediate}}
\cs_new_nopar:Npn \xref_deferred_new:nn {\xref_new_aux:nnn{deferred}}
\cs_new_nopar:Npn \xref_new_aux:nnn #1#2#3{
%    \end{macrocode}
%  First put the new type in the relevant property list.
%    \begin{macrocode}
  \prop_gput:ccx {g_xref_all_curr_ #1 _fields_prop}
  { xref_ #2 _key }
  { \exp_not:c {l_xref_curr_#2_tlp }}
%    \end{macrocode}
% Then define the key to be a protected macro.\footnote{We could also
%   set it equal to \cs{scan_stop:} but this just feels ``cleaner''.}
%    \begin{macrocode}
  \cs_set_protected_nopar:cpn { xref_#2_key }{}
  \tlp_new:cn{l_xref_curr_#2_tlp}{#3}
%    \end{macrocode}
% Now for the function extracting the value of a reference. We could
% do this with a simple |\prop_if_in| thing put since we want to do
% things in an expandable way we make a separate grabber for each
% type---this is also faster. The grabber function can be defined by
% using an intricate construction of |\exp_after:wN| and other goodies
% but I prefer readable code. The end result for the input |xyz| is
% \begin{verbatim}
% \cs_set_nopar:Npn\xref_get_value_xyz_aux:w #1\xref_xyz_key #2#3\q_nil{#2}
% \end{verbatim}
%    \begin{macrocode}
  \toks_set:Nx \l_tmpa_toks {
    \exp_not:n { \cs_set_nopar:cpn {xref_get_value_#2_aux:w} ##1 }
    \exp_not:N \q_prop
    \exp_not:c { xref_#2_key }
    \exp_not:N \q_prop
  }
  \toks_use:N \l_tmpa_toks ##2 ##3\q_nil {##2}
}
%    \end{macrocode}
%  \end{macro}
%  \end{macro}
%  \end{macro}
%
%
% \begin{macro}{\xref_get_value:nn}
%   Getting the correct value for a given label-type pair is a matter
%   of connecting the correct grabber functions and property list.
%    \begin{macrocode}
\cs_new_nopar:Npn \xref_get_value:nn #1#2 {
  \cs_if_really_free:cTF{g_xref_#2_prop}
  {??}
  {
%    \end{macrocode}
% This next expansion may look a little weird but it isn't if you
% think about it!
%    \begin{macrocode}
    \exp_args:NcNc \exp_after:wN {xref_get_value_#1_aux:w}
    \toks_use:N {g_xref_#2_prop}
%    \end{macrocode}
%  Better put in the stop marker.
%    \begin{macrocode}
    \q_nil
  }
}
%    \end{macrocode}
%  Temporary! We expand the property list and so we can't have the
%  |\q_prop| marker just expand!
%    \begin{macrocode}
\cs_set_nopar:NNn \exp_after:cc 2 {
  \exp_after:wN \exp_after:wN
  \cs:w #1\exp_after:wN\cs_end: \cs:w #2\cs_end:
}
\cs_set_protected:Npn \q_prop {\q_prop}
%    \end{macrocode}
%  \end{macro}
%
%  \begin{macro}{\xref_define_label:nn}
%  \begin{macro}[aux]{\xref_define_label_aux:nn}
%    Define the property list for each label. We better do this in two
%    steps because the special catcode regime is in effect and since
%    some of the info fields are very likely to contain actual text,
%    we better make sure spaces aren't ignored! As for the meaning of
%    other characters then it is a possibility to also have a field
%    containing catcode instructions which can then be activated with
%    |\etex_scantokens:D|. 
%    \begin{macrocode}
\cs_new_protected_nopar:Npn \xref_define_label:nn {
  \group_begin:
    \char_set_catcode:nn {`\ }\c_ten
    \xref_define_label_aux:nn
}
%    \end{macrocode}
% If the label is already taken we have a multiply defined label and
% we should do something about it. For now we don't do anything
% spectacular.
%    \begin{macrocode}
\cs_new_nopar:Npn \xref_define_label_aux:nn #1#2 {
    \cs_if_really_free:cTF{g_xref_#1_prop}
    {\prop_new:c{g_xref_#1_prop}}{\WARNING}
    \toks_gset:cn{g_xref_#1_prop}{#2}
  \group_end:
}
%    \end{macrocode}
%  \end{macro}
%  \end{macro}
%
%
% \begin{macro}{\xref_set_label:n}
%   Then the generic command for setting a label. We expand the
%   immediate labels fully before calling the write function but make
%   sure the deferred fields aren't expanded just yet. Due to property
%   lists being implemented as token list registers we must expand the
%   `immediate' fields twice.
%    \begin{macrocode}
\cs_set_nopar:Npn \xref_set_label:n #1{
  \cs_set_nopar:Npx \xref_tmp:w{\toks_use:N\g_xref_all_curr_immediate_fields_prop}
  \exp_args:NNx\iow_deferred_expanded:Nn \xref_write{
    \xref_define_label:nn {#1} { 
      \xref_tmp:w 
      \toks_use:N \g_xref_all_curr_deferred_fields_prop
    }
  }
}
%    \end{macrocode}
%  \end{macro}
%
% \begin{macro}{\xref_write}
%   A stream for writing cross references although they do not require
%   to be in a separate file.
%    \begin{macrocode}
\iow_new:N \xref_write
%    \end{macrocode}
%  \end{macro}
%
%  That's it (for now).
%    \begin{macrocode}
%</initex|package>
%    \end{macrocode}
% 
%    \begin{macrocode}
%<*showmemory>
\showMemUsage
%</showmemory>
%    \end{macrocode}
%
% \subsection{Test file}
%
%    \begin{macrocode}
%<*testfile>
\documentclass{article}
\usepackage{l3xref}
\ExplSyntaxOn
\cs_set_nopar:Npn \startrecording {\iow_open:Nn \xref_write {\jobname.xref}}
\cs_set_nopar:Npn \DefineCrossReferences {
  \group_begin:
    \ExplSyntaxNamesOn
    \InputIfFileExists{\jobname.xref}{}{}
  \group_end:
}
\AtBeginDocument{\DefineCrossReferences\startrecording}

\xref_new:nn {name}{}
\cs_set_nopar:Npn \setname{\tlp_set:Nn\l_xref_curr_name_tlp}
\cs_set_nopar:Npn \getname{\xref_get_value:nn{name}}

\xref_deferred_new:nn {page}{\thepage}
\cs_set_nopar:Npn \getpage{\xref_get_value:nn{page}}

\xref_deferred_new:nn {valuepage}{\number\value{page}}
\cs_set_nopar:Npn \getvaluepage{\xref_get_value:nn{valuepage}}

\cs_set_eq:NN \setlabel \xref_set_label:n

\ExplSyntaxOff
\begin{document}
\pagenumbering{roman}

Text\setname{This is a name}\setlabel{testlabel1}.  More
text\setname{This is another name}\setlabel{testlabel2}. \clearpage

Text\setname{This is a third name}\setlabel{testlabel3}. More
text\setname{Hello World!}\setlabel{testlabel4}. \clearpage

\pagenumbering{arabic}

Text\setname{Name 5}\setlabel{testlabel5}. More text\setname{Name
  6}\setlabel{testlabel6}. \clearpage

Text\setname{Name 7}\setlabel{testlabel 7}. More text\setname{Name
  8}\setlabel{testlabel8}. \clearpage

Now let's extract some values. \getname{testlabel1} on page
\getpage{testlabel1} with value \getvaluepage{testlabel1}.

Now let's extract some values. \getname{testlabel 7} on page
\getpage{testlabel 7} with value \getvaluepage{testlabel 7}.
\end{document}
%</testfile>
%    \end{macrocode}
%
% \Finale
% \PrintIndex
%
% \endinput
