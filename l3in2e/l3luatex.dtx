% \iffalse
%% File: l3fp.dtx Copyright (C) 2010 LaTeX3 project
%%
%% It may be distributed and/or modified under the conditions of the
%% LaTeX Project Public License (LPPL), either version 1.3c of this
%% license or (at your option) any later version.  The latest version
%% of this license is in the file
%%
%%    http://www.latex-project.org/lppl.txt
%%
%% This file is part of the ``expl3 bundle'' (The Work in LPPL)
%% and all files in that bundle must be distributed together.
%%
%% The released version of this bundle is available from CTAN.
%%
%% -----------------------------------------------------------------------
%%
%% The development version of the bundle can be found at
%%
%%    http://www.latex-project.org/svnroot/experimental/trunk/
%%
%% for those people who are interested.
%%
%%%%%%%%%%%
%% NOTE: %%
%%%%%%%%%%%
%%
%%   Snapshots taken from the repository represent work in progress and may
%%   not work or may contain conflicting material!  We therefore ask
%%   people _not_ to put them into distributions, archives, etc. without
%%   prior consultation with the LaTeX Project Team.
%%
%% -----------------------------------------------------------------------
%<*driver|package>
\RequirePackage{l3names}
%</driver|package>
%\fi
\GetIdInfo$Id: l3luatex.dtx 1985 2010-07-18 09:05:56Z joseph $
  {L3 Experimental LuaTeX functions}
%\iffalse
%<*driver>
%\fi
\ProvidesFile{\filename.\filenameext}
  [\filedate\space v\fileversion\space\filedescription]
%\iffalse
\documentclass[full]{l3doc}
\begin{document}
  \DocInput{l3fp.dtx}
\end{document}
%</driver>
% \fi
%
% \title{The \textsf{l3luatex} package\thanks{This file
%         has version number \fileversion, last
%         revised \filedate.}\\
% \LuaTeX-specific functions}
% \author{\Team}
% \date{\filedate}
% \maketitle
%
%\begin{documentation}
%
%\section{Breaking out to \Lua}
%
% The \LuaTeX\ engine provides access to the \Lua\ programming language, 
% and with it access to the 'internals' of \TeX. In order to use 
% this within the framework provided here, two functions are available.
% When used with \pdfTeX\ or \XeTeX these will raise an error: use 
% \cs{engine_if_luatex:T} to avoid this. Details of coding the \LuaTeX\
% engine are detailed in the \LuaTeX\ manual.
% 
%\begin{function}{ \lua_now:x / (EXP) }
%  \begin{syntax}
%    \cs{lua_now:x} \Arg{token list}
%  \end{syntax}
%  The \meta{token list} is fully expandable using the current 
%  applicable \TeX\ category codes: this will include converting line
%  ends to spaces in the usual \TeX\ manner. The resulting 
%  \meta{Lua input} is passed to the Lua interpreter for processing.
%  Each \cs{lua_now:x} block is treated by Lua as a separate chunk.
%  The Lua interpreter will execute the \meta{Lua input} immediately, 
%  and in an expandable manner.
%\end{function}
%
%\begin{function}{ \lua_shipout:x / (EXP) }
%  \begin{syntax}
%    \cs{lua_shipout:x} \Arg{token list}
%  \end{syntax}
%  The \meta{token list} is fully expandable using the current 
%  applicable \TeX\ category codes: this will include converting line
%  ends to spaces in the usual \TeX\ manner. The resulting 
%  \meta{Lua input} is passed to the Lua interpreter for processing.
%  Each \cs{lua_now:x} block is treated by Lua as a separate chunk.
%  The Lua interpreter will execute the \meta{Lua input} during the
%  page-building routine. (At a \TeX\ level, the \meta{Lua input} is
%  stored as a 'whatsit'.)
%\end{function}
%
%\section{Category code tables}
%
% As well as providing methods to break out into \Lua, there are
% places where additional \LaTeX3 functions are provided by the 
% \LuaTeX\ engine. In particular, \LuaTeX\ provides category code 
% tables. These can be used to ensure that a set of category codes
% are in force in a more robust way than is possible with other 
% engines. These are therefore used by \cs{ExplSyntaxOn} and
% \pkg{ExplSyntaxOff} when using the \LuaTeX\ engine.
% 
%\begin{function}{ \cctab_new:N }
%  \begin{syntax}
%    \cs{cctab_new:N} \meta{category code table}
%  \end{syntax}
%  Creates a new category code table, initially with the codes as
%  used by \IniTeX.
%\end{function}
%
%\begin{function}{ \cctab_gset:Nn }
%  \begin{syntax}
%    \cs{cctab_gset:Nn} \meta{category code table}
%      \Arg{category code set up}
%  \end{syntax}
%  Sets the \meta{category code table} to apply the category codes
%  which apply when the prevailing regime is modified by the
%  \meta{category code set up}. Thus within a standard code block
%  the starting point will be the code applied by \cs{c_code_cctab}.
%  The assignment of the table is global: the underlying primitive does
%  not respect grouping.
%\end{function}
% 
%\begin{function}{ \cctab_begin:N }
%  \begin{syntax}
%    \cs{cctab_begin:N} \meta{category code table}
%  \end{syntax}
%  Switches the category codes in force to those stored in the
%  \meta{category code table}.  The prevailing codes before the 
%  function is called are added to a stack, for use with
%  \cs{cctab_end:}.
%\end{function}
%
%\begin{function}{ \cctab_end: }
%  \begin{syntax}
%    \cs{cctab_end:}
%  \end{syntax}
%  Ends the scope of a \meta{category code table} started using
%  \cs{cctab_begin:N}, retuning the codes to those in force before the
%  matching \cs{cctab_begin:N} was used.
%\end{function}
%
%\begin{variable}{ \c_code_cctab }
%  Category code table for the code environment. This does not include
%  setting the behaviour of the line-end character, which is only
%  altered by \cs{ExplSyntaxOn}.
%\end{variable}
%
%\begin{variable}{ \c_document_cctab }
%  Category code table for a standard \LaTeX\ document. This does not 
%  include setting the behaviour of the line-end character, which is 
%  only altered by \cs{ExplSyntaxOff}.
%\end{variable}
%
%\begin{variable}{ \c_initex_cctab }
%  Category code table as set up by \IniTeX.
%\end{variable}
%
%\begin{variable}{ \c_other_cctab }
%  Category code table where all characters have category code \( 12 \)
%  (other).
%\end{variable}
%
%\begin{variable}{ \c_string_cctab }
%  Category code table where all characters have category code \( 12 \)
%  (other) with the exception of spaces, which have category code
%  \( 10 \) (space).
%\end{variable}
%
%\end{documentation}
%
%\begin{implementation}
%
%\section{Implementation}
% 
%    Announce and ensure that the required packages are loaded.
%    \begin{macrocode}
%<*package>
\ProvidesExplPackage
  {\filename}{\filedate}{\fileversion}{\filedescription}
\package_check_loaded_expl:
%</package>
%<*initex|package>
%    \end{macrocode}
%    
%\begin{macro}{\lua_now:x}
%\begin{macro}{\lua_shipout:x}
%\begin{macro}{\lua_wrong_engine:}
% When \LuaTeX\ is in use, this is all a question of primitives with new
% names. On the other hand, for \pdfTeX\ and \XeTeX\ the argument should
% be removed from the input stream before issuing an error. This needs 
% to be expandable, so the same idea is used as for \texttt{V}-type
% expansion, with an appropriately-named but undefined function.
%    \begin{macrocode}
\luatex_if_engine:TF
  {
    \cs_new_eq:NN \lua_now:x     \luatex_directlua:D
    \cs_new_eq:NN \lua_shipout:x \luatex_latelua:D
  }
  {
    \cs_new:Npn \lua_now:x #1     { \lua_wrong_engine: }
    \cs_new:Npn \lua_shipout:x #1 { \lua_wrong_engine: }
  }
\group_begin:
\char_make_letter:N\!
\char_make_letter:N\ %
\cs_gset:Npn\lua_wrong_engine:{%
\LuaTeX engine not in use!%
}%
\group_end:%
%    \end{macrocode}
%\end{macro}
%\end{macro}
%\end{macro}
%
%\subsection{Category code tables}
%
%\begin{macro}{\g_cctab_allocate_int}
%\begin{macro}{\g_cctab_stack_int}
%\begin{macro}{\g_cctab_stack_seq}
% To allocate category code tables, both the read-only and stack 
% tables need to be followed. There is also a sequence stack for the
% dynamic tables themselves.
%    \begin{macrocode}
\int_new:N  \g_cctab_allocate_int
\int_set:Nn \g_cctab_allocate_int { -1 }
\int_new:N \g_cctab_stack_int
\seq_new:N \g_cctab_stack_seq
%    \end{macrocode}
%\end{macro}
%\end{macro} 
%\end{macro}  
%
%\begin{macro}{\cctab_new:N}
% Creating a new category code table is done slightly differently
% from other registers. Low-numbered tables are more efficiently-stored
% than high-numbered ones. There is also a need to have a stack of 
% flexible tables as well as the set of read-only ones. To satisfy both
% of these requirements, odd numbered tables are used for read-only
% tables, and even ones for the stack. Here, therefore, the odd numbers
% are allocated.
%    \begin{macrocode}
\cs_new_protected_nopar:Npn \cctab_new:N #1 {
  \cs_if_free:NTF #1
    { 
      \int_gadd:Nn \g_cctab_allocate_int { 2 }
       \int_compare:nNnTF 
         { \g_cctab_allocate_int } <  { \c_allocate_max_tl + 1 }
         { 
           \tex_global:D \tex_mathchardef:D #1 \g_cctab_allocate_int 
           \luatex_initcatcodetable:D #1
         }
         { 
           \msg_kernel_error:nnx { code } { out-of-registers } { cctab }
         }  
     }
     {
       \msg_kernel_error:nnx { code } { variable-already-defined } 
         { \token_to_str:N #1 } 
     }
} 
\luatex_if_engine:F {
  \cs_set_protected_nopar:Npn \cctab_new:N #1 { \lua_wrong_engine: }
}
%<*package>
\luatex_if_engine:T {
  \cs_set_protected_nopar:Npn \cctab_new:N #1 
    { 
      \newcatcodetable #1
      \luatex_initcatcodetable:D #1
    }
}
%</package>
%    \end{macrocode}
%\end{macro}
%
%\begin{macro}{\cctab_begin:N}
%\begin{macro}{\cctab_end:}
%\begin{macro}{\l_cctab_tmp_tl}
% The aim here is to ensure that the saved tables are read-only. This is
% done by using a stack of tables which are not read only, and actually
% having them as 'in use' copies.
%    \begin{macrocode}
\cs_new_protected_nopar:Npn \cctab_begin:N #1 {
  \seq_gpush:Nx \g_cctab_stack_seq { \tex_the:D \luatex_catcodetable:D }
  \luatex_catcodetable:D #1
  \int_gadd:Nn \g_cctab_stack_int { 2 }
  \int_compare:nNnT { \g_cctab_stack_int } > { 268435453 }
    { \msg_kernel_error:nn { code } { cctab-stack-full } }
  \luatex_savecatcodetable:D \g_cctab_stack_int 
  \luatex_catcodetable:D \g_cctab_stack_int 
}
\cs_new_protected_nopar:Npn \cctab_end: {
  \int_gsub:Nn \g_cctab_stack_int { 2 }
  \seq_gpop:NN \g_cctab_stack_seq \l_cctab_tmp_tl
  \quark_if_no_value:NT \l_cctab_tmp_tl
    { \tl_set:Nn \l_cctab_tmp_tl { 0 } }
  \luatex_catcodetable:D \l_cctab_tmp_tl \scan_stop:
}
\luatex_if_engine:F {
  \cs_set_protected_nopar:Npn \cctab_begin:N #1 { \lua_wrong_engine: }
  \cs_set_protected_nopar:Npn \cctab_end: { \lua_wrong_engine: }
}
%<*package>
\luatex_if_engine:T {
  \cs_set_protected_nopar:Npn \cctab_begin:N #1 
    { \BeginCatcodeRegime #1 }
  \cs_set_protected_nopar:Npn \cctab_end: 
    { \EndCatcodeRegime }
}
%</package>
\tl_new:N \l_cctab_tmp_tl
%    \end{macrocode}
%\end{macro}
%\end{macro}
%\end{macro}
%
%\begin{macro}{\cctab_gset:Nn}
% Category code tables are always global, so only one version is needed.
% The set up here is simple, and means that at the point of use there is
% no need to worry about escaping category codes.
%    \begin{macrocode}
\cs_new_protected:Npn \cctab_gset:Nn #1#2 {
  \group_begin:
    #2
    \luatex_savecatcodetable:D #1
  \group_end:
}
\luatex_if_engine:F {
  \cs_set_protected_nopar:Npn \cctab_gset:Nn #1#2 { \lua_wrong_engine: }
}
%    \end{macrocode}
%\end{macro}
%
%\begin{macro}{\c_code_cctab}
%\begin{macro}{\c_document_cctab}
%\begin{macro}{\c_initex_cctab}
%\begin{macro}{\c_other_cctab}
%\begin{macro}{\c_string_cctab}
% Creating category code tables is easy using the function above.
% The \texttt{other} and \texttt{string} ones are done by completely
% ignoring the existing codes as this makes life a lot less complex. The
% table for \pkg{expl3} category codes is always needed, whereas when in
% package mode the rest can be copied from the existing \LaTeXe\ package 
% \pkg{luatex}.
%    \begin{macrocode}
\luatex_if_engine:T {
  \cctab_new:N \c_code_cctab
  \cctab_gset:Nn \c_code_cctab { }
}
%<*package>  
\luatex_if_engine:T {
  \cs_new_eq:NN \c_document_cctab \CatcodeTableLaTeX
  \cs_new_eq:NN \c_initex_cctab   \CatcodeTableIniTeX
  \cs_new_eq:NN \c_other_cctab    \CatcodeTableOther
  \cs_new_eq:NN \c_string_cctab   \CatcodeTableString
}
%</package>   
%<*!package>    
\luatex_if_engine:T {
  \cctab_new:N \c_document_cctab
  \cctab_new:N \c_other_cctab
  \cctab_new:N \c_string_cctab
  \cctab_gset:Nn \c_document_cctab
    {
      \char_make_space:n       { 9 }
      \char_make_space:n       { 32 }
      \char_make_other:n       { 58 }
      \char_make_subscript:n   { 95 }
      \char_make_active:n      { 126 }
    }
  \cctab_gset:Nn \c_other_cctab
    {
      \prg_stepwise_inline:nnnn { 0 } { 1 } { 127 }
        { \char_make_other:n {#1} }
    } 
  \cctab_gset:Nn \c_string_cctab
    {
      \prg_stepwise_inline:nnnn { 0 } { 1 } { 127 }
        { \char_make_other:n {#1} }
      \char_make_space:n { 32 }  
    }  
}
%</!package>
%    \end{macrocode}
%\end{macro}
%\end{macro}
%\end{macro}
%\end{macro}
%\end{macro}
% 
%    \begin{macrocode}
%</initex|package>
%    \end{macrocode}
%
%\end{implementation}
%
%\PrintChanges
%
%\PrintIndex