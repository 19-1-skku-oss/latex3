% \iffalse
%% File: l3num.dtx Copyright (C) 2005-2008 Frank Mittelbach, LaTeX3 project
%%
%% It may be distributed and/or modified under the conditions of the
%% LaTeX Project Public License (LPPL), either version 1.3c of this
%% license or (at your option) any later version.  The latest version
%% of this license is in the file
%%
%%    http://www.latex-project.org/lppl.txt
%%
%% This file is part of the ``expl3 bundle'' (The Work in LPPL)
%% and all files in that bundle must be distributed together.
%%
%% The released version of this bundle is available from CTAN.
%%
%% -----------------------------------------------------------------------
%%
%% The development version of the bundle can be found at
%%
%%    http://www.latex-project.org/cgi-bin/cvsweb.cgi/
%%
%% for those people who are interested.
%%
%%%%%%%%%%%
%% NOTE: %%
%%%%%%%%%%%
%%
%%   Snapshots taken from the repository represent work in progress and may
%%   not work or may contain conflicting material!  We therefore ask
%%   people _not_ to put them into distributions, archives, etc. without
%%   prior consultation with the LaTeX Project Team.
%%
%% -----------------------------------------------------------------------
%
%<*driver|package>
\RequirePackage{l3names}
%</driver|package>
%\fi
\GetIdInfo$Id$
       {L3 Experimental token numbers}
%\iffalse
%<*driver>
%\fi
\ProvidesFile{\filename.\filenameext}
  [\filedate\space v\fileversion\space\filedescription]
%\iffalse
\documentclass[full]{l3doc}
\begin{document}
\DocInput{\filename.\filenameext}
\end{document}
%</driver>
% \fi
%
%
% \title{The \textsf{l3num} package\thanks{This file
%         has version number \fileversion, last
%         revised \filedate.}\\
% Integer in macros}
% \author{\Team}
% \date{\filedate}
% \maketitle
%
% \section{Macro Counters}
%
% Instead of using counter registers for manipulation of
% integer values it is sometimes useful to keep such values in
% macros. For this \LaTeX3 offers the  type ``num''.
%
% One reason is the limited number of registers inside
% \TeX{}. However, when using \eTeX{} this is no longer an issue. It
% remains to be seen if there are other compelling reasons to keep
% this module.
%
% It turns out there might be as with a \meta{num} data type, the
% allocation module can do its bookkeeping without the aid of
% \meta{int} registers.
%
% \subsection{Functions}
%
% \begin{function}{%
%                  \num_new:N |
%                  \num_new:c |
% }
% \begin{syntax}
%    "\num_new:N"   <num>
% \end{syntax}
% Defines <num> to be a new variable of type "num" (initialized to
% zero). There is no way to
% define constant counters with these functions.
% \end{function}
%
% \begin{function}{%
%                  \num_incr:N |
%                  \num_incr:c |
%                  \num_gincr:N |
%                  \num_gincr:c
% }
% \begin{syntax}
%   "\num_incr:N"   <num>
% \end{syntax}
% Increments <num> by one. For global variables the global versions
% should be used.
% \end{function}
%
% \begin{function}{%
%                  \num_decr:N |
%                  \num_decr:c |
%                  \num_gdecr:N |
%                  \num_gdecr:c |
% }
% \begin{syntax}
%   "\num_decr:N"   <num>
% \end{syntax}
% Decrements <num> by one. For global variables the global versions
% should be used.
% \end{function}
%
% \begin{function}{%
%                  \num_zero:N |
%                  \num_zero:c |
%                  \num_gzero:N |
%                  \num_gzero:c |
% }
% \begin{syntax}
%   "\num_zero:N"   <num>
% \end{syntax}
% Resets <num> to zero. For global variables the global versions
% should be used.
% \end{function}
%
% \begin{function}{%
%                  \num_set:Nn |
%                  \num_set:cn |
%                  \num_gset:Nn |
%                  \num_gset:cn |
% }
% \begin{syntax}
%   "\num_set:Nn"   <num> \Arg{integer}
% \end{syntax}
% These functions will set the <num> register to the <integer> value.
% \end{function}
%
% \begin{function}{%
%                  \num_gset_eq:NN |
%                  \num_gset_eq:cN |
%                  \num_gset_eq:Nc |
%                  \num_gset_eq:cc |
% }
% \begin{syntax}
%   "\num_gset_eq:NN"   <num1> <num2>
% \end{syntax}
% These functions will set the <num1> register equal to <num2>.
% \end{function}
%
% \begin{function}{%
%                  \num_add:Nn |
%                  \num_add:cn |
%                  \num_gadd:Nn |
%                  \num_gadd:cn |
% }
% \begin{syntax}
%   "\num_add:Nn"   <num> \Arg{integer}
% \end{syntax}
% These functions will add to the <num> register the value <integer>.  If
% the second argument is a <num> register too, the surrounding braces
% can be left out.
%
% \end{function}
%
%
% \begin{function}{%
%                  \num_use:N |
%                  \num_use:c |
% }
% \begin{syntax}
%   "\num_use:N"   <num>
% \end{syntax}
% This function returns the integer value kept in <num> in a way
% suitable for further processing.
% \begin{texnote}
%   Since these <num>s are implemented as macros, the function
%   "\num_use:N" is effectively a noop and mainly there for
%   consistency with similar functions in other modules.
% \end{texnote}
% \end{function}
%
% \begin{function}{ \num_show:N |
%                   \num_show:c }
% \begin{syntax}
%   "\num_show:N" <num>
% \end{syntax}
% This function pauses the compilation and displays the integer value kept 
% in <num> on the console output.
% \end{function}
%
%
% \begin{function}{%
%                  \num_eval:n |
% }
% \begin{syntax}
%   "\num_eval:n"   \Arg{integer-expr}
% \end{syntax}
% Evaluates the integer expression allowing normal mathematical
% operators like |+-/*|.
% \end{function}
%
%
% \begin{function}{%
%                  \num_compare:nNnTF |
%                  \num_compare:cNcTF |
%                  \num_compare:nNnT |
%                  \num_compare:nNnF |
% }
% \begin{syntax}
%   "\num_compare:nNnTF"   \Arg{num~expr} <rel> \Arg{num~expr}
%                          \Arg{true} \Arg{false}
% \end{syntax}
% These functions test two \m{num} expressions against each other. They
% are both evaluated by "\num_eval:n".
% \end{function}
%
% \begin{function}{%
%                  \num_compare_p:nNn |
% }
% \begin{syntax}
%   "\num_compare_p:nNn"   \Arg{num~expr} <rel> \Arg{num~expr}
% \end{syntax}
% A predicate version of the above functions.
% \end{function}
%
% \begin{function}{%
%                  \num_max_of:nn |
%                  \num_min_of:nn |
% }
% \begin{syntax}
%   "\num_max_of:nn"   \Arg{num~expr} \Arg{num~expr}
% \end{syntax}
% Return the largest or smallest of two \m{num} expressions.
% \end{function}
%
% \begin{function}{%
%                  \num_abs:n |
% }
% \begin{syntax}
%   "\num_abs:n"   \Arg{num~expr}
% \end{syntax}
% Return the numerical value of a \m{num} expression.
% \end{function}
%
% \begin{function}{%
%                  \num_elt_count:n |
% }
% \begin{syntax}
%   "\num_elt_count:n"   \Arg{balanced text} 
% \end{syntax}
% Discards \m{balanced text} and puts a "+1" in the input stream. Used
% to count elements in a token list. 
% \end{function}
%
% \subsection{Formatting a counter value}
%
% See the \textsf{l3int} module for ways of doing this.
%
% \subsection{Variable and constants}
%
% \begin{function}{%
%                  \const_new:Nn |
% }
% \begin{syntax}
%   "\const_new:Nn"  "\c_"<value> \Arg{value}
% \end{syntax}
% Defines a constant with <value>. If the constant is negative
% or very large it requires an <int> register.
% \end{function}
%
% \begin{variable}{%
%                  \c_minus_one |
%                  \c_zero |
%                  \c_one |
%                  \c_two |
%                  \c_three |
%                  \c_four |
%                  \c_six |
%                  \c_seven |
%                  \c_nine |
%                  \c_ten |
%                  \c_eleven |
%                  \c_sixteen |
%                  \c_hundred_one |
%                  \c_twohundred_fifty_five |
%                  \c_twohundred_fifty_six |
%                  \c_thousand |
%                  \c_ten_thousand |
%                  \c_twenty_thousand 
% }
% \\
% Set of constants denoting useful values.
% \begin{texnote}
% Most of these constants have been available under \LaTeX2 under names
% like \tn{tw@}, \tn{thr@@} etc.
% \end{texnote}
% \end{variable}
%
% \begin{variable}{%
%                  \l_tmpa_num |
%                  \l_tmpb_num |
%                  \l_tmpc_num |
%                  \g_tmpa_num |
%                  \g_tmpb_num |
% }
% \\
% Scratch register for immediate use. They are not used by conditionals
% or predicate functions.
% \end{variable}
%
% \subsection{Primitive functions}
%
%
% \begin{function}{%
%                  \num_value:w |
% }
% \begin{syntax}
%   "\num_value:w" <integer>  \\ 
%   "\num_value:w" <tokens>  <optional space>
% \end{syntax}
% Expands <tokens> until an <integer> is formed. One space may be
% gobbled in the process. Preferably use with "\num_eval:n".
% \begin{texnote}
% This is the \TeX{} primitive \tn{number}.
% \end{texnote}
% \end{function}
%
% \begin{function}{%
%                  \num_eval:w |
% }
% \begin{syntax}
%   "\num_eval:w" <integer expression> "\num_eval_end:"
% \end{syntax}
% Evaluates <integer expression>. The evaluation stops when an
% unexpandable token of catcode other than 12 is reached or
% "\num_eval_end:" is read. The latter is gobbled by the scanner
% mechanism.
% \begin{texnote}
% This is the \eTeX{} primitive \tn{numexpr}.
% \end{texnote}
% \end{function}
%
% \begin{function}{%
%                  \if_num:w |
% }
% \begin{syntax}
%   "\if_num:w" <number1> <rel> <number2> <true> "\else:" <false> "\fi:"
% \end{syntax}
% Compare two numbers. It is recommended to use "\num_eval:n" to
% correctly evaluate and terminate these numbers. <rel> is one of
% "<", "=" or ">" with catcode 12.
% \begin{texnote}
% This is the \TeX{} primitive \tn{ifnum}.
% \end{texnote}
% \end{function}
%
% \begin{function}{%
%                  \if_num_odd:w |
% }
% \begin{syntax}
%   "\if_num_odd:w" <number>  <true> "\else:" <false> "\fi:"
% \end{syntax}
% Execute <true> if <number> is odd, <false> otherwise.
% \begin{texnote}
% This is the \TeX{} primitive \tn{ifodd}.
% \end{texnote}
% \end{function}
%
% \begin{function}{%
%                  \if_case:w |
%                  \or: |
% }
% \begin{syntax}
%   "\if_case:w" <number> <case0> "\or:" <case1> "\or:" "..." "\else:"
%   <default> "\fi:"
% \end{syntax}
% Chooses case<number>. If you wish to use negative numbers as well,
% you can offset them with "\num_eval:n".
% \begin{texnote}
% These are the \TeX{} primitives \tn{ifcase} and \tn{or}.
% \end{texnote}
% \end{function}
%
% \StopEventually{}
%
% \subsection{The Implementation}
%
%
%    We start by ensuring that the required packages are loaded.
%    \begin{macrocode}
%<package>\ProvidesExplPackage
%<package>  {\filename}{\filedate}{\fileversion}{\filedescription}
%<package&!check>\RequirePackage{l3expan}\par
%<package&check>\RequirePackage{l3chk}\par
%<*initex|package>
%    \end{macrocode}
%
% \begin{macro}{\num_value:w}
% \begin{macro}{\num_eval:w}
% \begin{macro}{\num_eval_end:}
% \begin{macro}{\if_num:w}
% \begin{macro}{\if_num_odd:w}
% \begin{macro}{\if_case:w}
% \begin{macro}{\or:}
%   Here are the remaining primitives for number comparisons and
%   expressions.
%    \begin{macrocode}
\let_new:NN \num_value:w        \tex_number:D
\let_new:NN \num_eval:w         \etex_numexpr:D
\let_new:NN \num_eval_end:      \scan_stop:
\let_new:NN \if_num:w           \tex_ifnum:D
\let_new:NN \if_num_odd:w       \tex_ifodd:D
\let_new:NN \if_case:w          \tex_ifcase:D
\let_new:NN \or:                \tex_or:D
%    \end{macrocode}
% \end{macro}
% \end{macro}
% \end{macro}
% \end{macro}
% \end{macro}
% \end{macro}
% \end{macro}
%
% Functions that support \LaTeX's user accessible counters should be
% added here, too. But first the internal counters.
%
% \begin{macro}{\num_incr:N}
% \begin{macro}{\num_decr:N}
% \begin{macro}{\num_gincr:N}
% \begin{macro}{\num_gdecr:N}
%    Incrementing and decrementing of integer registers is done with
%    the following functions.
%    \begin{macrocode}
\def:Npn \num_incr:N #1{\num_add:Nn#1 1}
\def:Npn \num_decr:N #1{\num_add:Nn#1 \c_minus_one}
\def:Npn \num_gincr:N #1{\num_gadd:Nn#1 1}
\def:Npn \num_gdecr:N #1{\num_gadd:Nn#1 \c_minus_one}
%    \end{macrocode}
% \end{macro}
% \end{macro}
% \end{macro}
% \end{macro}
%
% \begin{macro}{\num_incr:c}
% \begin{macro}{\num_decr:c}
% \begin{macro}{\num_gincr:c}
% \begin{macro}{\num_gdecr:c}
%    We also need \ldots
%    \begin{macrocode}
\def_new:Npn \num_incr:c {\exp_args:Nc \num_incr:N}
\def_new:Npn \num_decr:c {\exp_args:Nc \num_decr:N}
\def_new:Npn \num_gincr:c {\exp_args:Nc \num_gincr:N}
\def_new:Npn \num_gdecr:c {\exp_args:Nc \num_gdecr:N}
%    \end{macrocode}
% \end{macro}
% \end{macro}
% \end{macro}
% \end{macro}
%
% \begin{macro}{\num_zero:N}
% \begin{macro}{\num_zero:c}
% \begin{macro}{\num_gzero:N}
% \begin{macro}{\num_gzero:c}
%    We also need \ldots
%    \begin{macrocode}
\def_new:Npn \num_zero:N #1 {\num_set:Nn #1 0}
\def_new:Npn \num_gzero:N #1 {\num_gset:Nn #1 0}
\def_new:Npn \num_zero:c {\exp_args:Nc \num_zero:N}
\def_new:Npn \num_gzero:c {\exp_args:Nc \num_gzero:N}
%    \end{macrocode}
% \end{macro}
% \end{macro}
% \end{macro}
% \end{macro}
%
%
%
% \begin{macro}{\num_new:N}
% \begin{macro}{\num_new:c}
%    Allocate a new \m{num} variable and initialize it with zero.
%    \begin{macrocode}
\def_new:Npn \num_new:N #1{\tlp_new:Nn #1{0}}
\def_new:Npn \num_new:c {\exp_args:Nc \num_new:N}
%    \end{macrocode}
% \end{macro}
% \end{macro}
%
%
% \begin{macro}{\num_eval:n}
% This function enables us to do all the operations without the aid of
% an \m{int} register.
%    \begin{macrocode}
\def_new:Npn \num_eval:n #1{\num_eval:w #1\num_eval_end:}
%    \end{macrocode}
% \end{macro}
%
% \begin{macro}{\num_set:Nn}
% \begin{macro}{\num_set:cn}
% \begin{macro}{\num_gset:Nn}
% \begin{macro}{\num_gset:cn}
% Assigning values to \m{num} registers.
%    \begin{macrocode}
\def_new:Npn \num_set:Nn #1#2{
  \tlp_set:No #1{ \tex_number:D \num_eval:n {#2} }
}
\def_new:Npn \num_gset:Nn {\pref_global:D \num_set:Nn}
\def_new:Npn \num_set:cn {\exp_args:Nc \num_set:Nn }
\def_new:Npn \num_gset:cn {\exp_args:Nc \num_gset:Nn }
%    \end{macrocode}
% \end{macro}
% \end{macro}
% \end{macro}
% \end{macro}
%
% \begin{macro}{\num_set_eq:NN}
% \begin{macro}{\num_set_eq:cN}
% \begin{macro}{\num_set_eq:Nc}
% \begin{macro}{\num_set_eq:cc}
% Setting \m{num} registers equal to each other.
%    \begin{macrocode}
\let_new:NN \num_set_eq:NN \tlp_set_eq:NN
\def_new:Npn \num_set_eq:cN {\exp_args:Nc \num_set_eq:NN}
\def_new:Npn \num_set_eq:Nc {\exp_args:NNc \num_set_eq:NN}
\def_new:Npn \num_set_eq:cc {\exp_args:Ncc \num_set_eq:NN}
%    \end{macrocode}
% \end{macro}
% \end{macro}
% \end{macro}
% \end{macro}
%
% \begin{macro}{\num_gset_eq:NN}
% \begin{macro}{\num_gset_eq:cN}
% \begin{macro}{\num_gset_eq:Nc}
% \begin{macro}{\num_gset_eq:cc}
% Setting \m{num} registers equal to each other.
%    \begin{macrocode}
\let_new:NN \num_gset_eq:NN \tlp_gset_eq:NN
\def_new:Npn \num_gset_eq:cN {\exp_args:Nc \num_gset_eq:NN}
\def_new:Npn \num_gset_eq:Nc {\exp_args:NNc \num_gset_eq:NN}
\def_new:Npn \num_gset_eq:cc {\exp_args:Ncc \num_gset_eq:NN}
%    \end{macrocode}
% \end{macro}
% \end{macro}
% \end{macro}
% \end{macro}
%
% \begin{macro}{\num_add:Nn}
% \begin{macro}{\num_add:cn}
% \begin{macro}{\num_gadd:Nn}
% \begin{macro}{\num_gadd:cn}
% Adding is easily done as the second argument goes through
% |\num_eval:n|.
%    \begin{macrocode}
\def_new:Npn \num_add:Nn #1#2 {\num_set:Nn #1{#1+#2}}
\def_new:Npn \num_add:cn {\exp_args:Nc\num_add:Nn}
\def_new:Npn \num_gadd:Nn {\pref_global:D \num_add:Nn}
\def_new:Npn \num_gadd:cn {\exp_args:Nc\num_gadd:Nn}
%    \end{macrocode}
% \end{macro}
% \end{macro}
% \end{macro}
% \end{macro}
%
%
% \begin{macro}{\num_use:N}
% \begin{macro}{\num_use:c}
%    Here is how num macros are accessed:
%    \begin{macrocode}
\let_new:NN\num_use:N \use:n
\let_new:NN\num_use:c \use:c
%    \end{macrocode}
% \end{macro}
% \end{macro}
%
% \begin{macro}{\num_show:N}
% \begin{macro}{\num_show:c}
%    Here is how num macros are diagnosed:
%    \begin{macrocode}
\let_new:NN\num_show:N \cs_show:N
\let_new:NN\num_show:c \cs_show:c
%    \end{macrocode}
% \end{macro}
% \end{macro}
%
%
% \begin{macro}{\num_compare:nNnTF}
% \begin{macro}{\num_compare:nNnT}
% \begin{macro}{\num_compare:nNnF}
% Simple comparison tests.
%    \begin{macrocode}
\def_test_function_new:npn {num_compare:nNn}#1#2#3{
  \if_num:w \num_eval:n {#1}#2\num_eval:n {#3}
}
\def_new:Npn \num_compare:cNcTF { \exp_args:NcNc\num_compare:nNnTF }
%    \end{macrocode}
% \end{macro}
% \end{macro}
% \end{macro}
%
% \begin{macro}{\num_compare_p:nNn}
% A predicate function.
%    \begin{macrocode}
\def_new:Npn \num_compare_p:nNn #1#2#3{
  \if_num:w \num_eval:n {#1}#2\num_eval:n {#3}
    \c_true
  \else:
    \c_false
  \fi:
}
%    \end{macrocode}
% \end{macro}
%
% \begin{macro}{\num_max_of:nn}
% \begin{macro}{\num_min_of:nn}
% \begin{macro}{\num_abs:n}
% Functions for $\min$, $\max$, and absolute value.
%    \begin{macrocode}
\def_new:Npn \num_abs:n#1{
  \if_num:w \num_eval:n{#1}<\c_zero \exp_after:NN -\fi: #1
}
\def_new:Npn \num_max_of:nn#1#2{\num_compare:nNnTF {#1}>{#2}{#1}{#2}}
\def_new:Npn \num_min_of:nn#1#2{\num_compare:nNnTF {#1}<{#2}{#1}{#2}}
%    \end{macrocode}
% \end{macro}
% \end{macro}
% \end{macro}
%
% \begin{macro}{\num_elt_count:n}
% \begin{macro}{\num_elt_count_prop:Nn}
% Helper function for counting elements in a list.
%    \begin{macrocode}
\def_long_new:Npn \num_elt_count:n #1 { + 1 }
\def_long_new:Npn \num_elt_count_prop:Nn #1#2 { + 1 }
%    \end{macrocode}
% \end{macro}
% \end{macro}
%
% \begin{macro}{\l_tmpa_num}
% \begin{macro}{\l_tmpb_num}
% \begin{macro}{\l_tmpc_num}
% \begin{macro}{\g_tmpa_num}
% \begin{macro}{\g_tmpb_num}
%    We provide an number local and two global \m{num}s, maybe we
%    need more or less.
%    \begin{macrocode}
\num_new:N \l_tmpa_num
\num_new:N \l_tmpb_num
\num_new:N \l_tmpc_num
\num_new:N \g_tmpa_num
\num_new:N \g_tmpb_num
%    \end{macrocode}
% \end{macro}
% \end{macro}
% \end{macro}
% \end{macro}
% \end{macro}
%
%  \subsubsection{Defining constants}
%  As stated, most constants can be defined as |\tex_chardef:D| or
%  |\tex_mathchardef:D| but that's engine dependent. Omega/Aleph
%  allows |\tex_chardef:D|s up to 65535 which is also the maximum
%  number of registers of all types.
%  \begin{macro}{\const_new:Nn}
%  \begin{macro}{\c_max_register_num}
%  \begin{macro}{\const_new_aux:Nw}
%    \begin{macrocode}
%    \begin{macrocode}
\engine_if_aleph:TF
{
  \let_new:NN \const_new_aux:Nw \tex_chardef:D
  \const_new_aux:Nw \c_max_register_num = 65535 \scan_stop:
}
{
  \let_new:NN \const_new_aux:Nw \tex_mathchardef:D
  \const_new_aux:Nw \c_max_register_num = 32767 \scan_stop:
}
\def_new:Npn \const_new:Nn #1#2 {
  \num_compare:nNnTF  {#2} > \c_minus_one
  {
    \num_compare:nNnTF {#2} > \c_max_register_num 
    {\int_new:N #1 \int_set:Nn #1{#2}}
    {\chk_if_new_cs:N #1 \const_new_aux:Nw #1 = #2 \scan_stop: }
  }
  {\int_new:N #1 \int_set:Nn #1{#2}}
}
%    \end{macrocode}
%  \end{macro}
%  \end{macro}
%  \end{macro}
%
% \begin{macro}{\c_minus_one}
% \begin{macro}{\c_zero}
% \begin{macro}{\c_one}
% \begin{macro}{\c_two}
% \begin{macro}{\c_three}
% \begin{macro}{\c_four}
% \begin{macro}{\c_six}
% \begin{macro}{\c_seven}
% \begin{macro}{\c_nine}
% \begin{macro}{\c_ten}
% \begin{macro}{\c_eleven}
% \begin{macro}{\c_sixteen}
% \begin{macro}{\c_thirty_two}
% \begin{macro}{\c_hundred_one}
% \begin{macro}{\c_twohundred_fifty_five}
% \begin{macro}{\c_twohundred_fifty_six}
% \begin{macro}{\c_thousand}
% \begin{macro}{\c_ten_thousand}
% \begin{macro}{\c_ten_thousand_one}
% \begin{macro}{\c_ten_thousand_two}
% \begin{macro}{\c_ten_thousand_three}
% \begin{macro}{\c_ten_thousand_four}
% \begin{macro}{\c_twenty_thousand}
%    And the usual constants, others are still missing. Please, make
%    every constant a real constant at least for the moment. We can
%    easily convert things in the end when we have found what
%    constants are used in critical places and what not.
%    \begin{macrocode}
 %% \tex_countdef:D \c_minus_one = 10 \scan_stop:
 %% \c_minus_one = -1 \scan_stop:        %% in l3basics
 %% \tex_chardef:D \c_sixteen    = 16\scan_stop: %% in l3basics
\const_new:Nn \c_zero   {0}
\const_new:Nn \c_one    {1}
\const_new:Nn \c_two    {2}
\const_new:Nn \c_three  {3}
\const_new:Nn \c_four   {4}
\const_new:Nn \c_six    {6}
\const_new:Nn \c_seven  {7}
\const_new:Nn \c_nine   {9}
\const_new:Nn \c_ten    {10}
\const_new:Nn \c_eleven {11}
\const_new:Nn \c_thirty_two {32}
%    \end{macrocode}
% The next one may seem a little odd (obviously!) but is useful when
% dealing with logical operators.
%    \begin{macrocode}
\const_new:Nn \c_hundred_one          {101}
\const_new:Nn \c_twohundred_fifty_five {255}
\const_new:Nn \c_twohundred_fifty_six {256}
\const_new:Nn \c_thousand             {1000}
\const_new:Nn \c_ten_thousand         {10000}
\const_new:Nn \c_ten_thousand_one     {10001}
\const_new:Nn \c_ten_thousand_two     {10002}
\const_new:Nn \c_ten_thousand_three   {10003}
\const_new:Nn \c_ten_thousand_four    {10004}
\const_new:Nn \c_twenty_thousand      {20000}
%    \end{macrocode}
% \end{macro}
% \end{macro}
% \end{macro}
% \end{macro}
% \end{macro}
% \end{macro}
% \end{macro}
% \end{macro}
% \end{macro}
% \end{macro}
% \end{macro}
% \end{macro}
% \end{macro}
% \end{macro}
% \end{macro}
% \end{macro}
% \end{macro}
% \end{macro}
% \end{macro}
% \end{macro}
% \end{macro}
% \end{macro}
% \end{macro}
%
%
%    \begin{macrocode}
%</initex|package>
%    \end{macrocode}
%
% \endinput
