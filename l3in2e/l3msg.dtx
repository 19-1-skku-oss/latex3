% \iffalse
% 
% !TEX program  = pdflatex
% !TEX encoding = ISO-8859-1
% 
%% File: l3msg.dtx Copyright (C) 2009 LaTeX3 project
%%
%% It may be distributed and/or modified under the conditions of the
%% LaTeX Project Public License (LPPL), either version 1.3c of this
%% license or (at your option) any later version.  The latest version
%% of this license is in the file
%%
%%    http://www.latex-project.org/lppl.txt
%%
%% This file is part of the ``expl3 bundle'' (The Work in LPPL)
%% and all files in that bundle must be distributed together.
%%
%% The released version of this bundle is available from CTAN.
%%
%% -----------------------------------------------------------------------
%%
%% The development version of the bundle can be found at
%%
%%    http://www.latex-project.org/cgi-bin/cvsweb.cgi/
%%
%% for those people who are interested.
%%
%%%%%%%%%%%
%% NOTE: %%
%%%%%%%%%%%
%%
%%   Snapshots taken from the repository represent work in progress and may
%%   not work or may contain conflicting material!  We therefore ask
%%   people _not_ to put them into distributions, archives, etc. without
%%   prior consultation with the LaTeX Project Team.
%%
%% -----------------------------------------------------------------------
%% 
%
%<*driver|package>
\RequirePackage{l3names}
%</driver|package>
%\fi
\GetIdInfo$Id: l3msg.dtx 1190 2009-04-19 10:44:07Z joseph $
  {L3 Experimental LaTeX Messages module}
%\iffalse
%<*driver>
%\fi
\ProvidesFile{\filename.\filenameext}
  [\filedate\space v\fileversion\space\filedescription]
%\iffalse
\documentclass[full]{l3doc}
\begin{document}
\DocInput{l3msg.dtx}
\end{document}
%</driver>
% \fi
%
% \title{The \textsf{l3msg} package\thanks{This file
%         has version number \fileversion, last
%         revised \filedate.}\\
% Messages to the user}
% \author{\Team}
% \date{\filedate}
% \maketitle
%  
% \tableofcontents
%
% \begin{documentation}
%
% \section{Communicating with the user}
% 
% Messages need to be passed to the user by modules, either when errors
% occur or to indicate how the code is proceeding. The \pkg{l3msg} 
% module provides a consistent method for doing this (as opposed to
% writing directly to the terminal or log).
% 
% The system used by \pkg{l3msg} to create messages divides the process
% into two distinct parts. Named messages are created in the first part
% of the process; at this stage, no decision is made about the type 
% output that the message will produce. The second part of the process
% is actually producing a message. At this stage a choice of message
% \emph{class} has to be made, for example \texttt{error}, 
% \texttt{warning} or \texttt{info}.
% 
% By separating out the creation and use of messages, several benefits
% are available. First, the messages can be altered later without 
% needing details of where they are used in the code. This makes it
% possible to alter the language used, the detail level and so on. 
% Secondly, the output which results from a given message can be 
% altered. This can be done on a message class, module or message name
% basis. In this way, message behaviour can be altered and messages can
% be entirely suppressed.
% 
% \subsection{Creating new messages}
% 
% All messages have to be created before they can be used.
% 
%\begin{function}{
%  \msg_new:nnnnn|
%  \msg_new:nnnn|
%  \msg_new:nnn|
%  \msg_set:nnnnn|
%  \msg_set:nnnn|
%  \msg_set:nnn
%}
%  \begin{syntax}
%    "\msg_new:nnnn" <module> <name> <text> <more text> <code>
%  \end{syntax}
%  Creates new message <name> for <module> to produce <text> initially, 
%  <more text> if requested by the user and to insert <code> after the 
%  message when used as an error. <text> and <more text> can use up to 
%  two macro parameters (|#1| and |#2|), which are supplied by the 
%  message system. 
%\end{function}
%
% \subsection{Message classes}
% 
% Creating message output requires the message to be given a class.
%
%\begin{function}{
%  \msg_class_new:nn|
%  \msg_class_set:nn
%}
%  \begin{syntax}
%    "\msg_class_new:nn" <class> <code>
%  \end{syntax}
%  Creates new <class> to output a message, using <code> to process
%  the message text.
%\end{function}
%
% The module defines several common message classes. The following 
% describes the standard behaviour of each class if no redirection of 
% the class or message is active.
%
%\begin{function}{
%  \msg_fatal:nnnn|
%  \msg_fatal:nnn|
%  \msg_fatal:nn
%}
%  \begin{syntax}
%    "\msg_fatal:nnnn" <module> <name> <arg one> <arg two>
%  \end{syntax}
%  Issues <module> error message <name>, passing <arg one> and 
%  <arg two> to the text-creating functions. The \TeX\ run then halts.
%\end{function}
%
%\begin{function}{
%  \msg_error:nnnn|
%  \msg_error:nnn|
%  \msg_error:nn
%}
%  \begin{syntax}
%    "\msg_error:nnnn" <module> <name> <arg one> <arg two>
%  \end{syntax}
%  Issues <module> error message <name>, passing <arg one> and 
%  <arg two> to the text-creating functions. 
%  \begin{texnote}
%    The standard output here is similar to \cs{PackageError}.
%  \end{texnote}
%\end{function}
%
%\begin{function}{
%  \msg_warning:nnnn|
%  \msg_warning:nnn|
%  \msg_warning:nn
%}
%  \begin{syntax}
%    "\msg_warning:nnnn" <module> <name> <arg one> <arg two>
%  \end{syntax}
%  Prints <module> message <name> to the terminal, passing <arg one> and
%  <arg two> to the text-creating functions. 
%  \begin{texnote}
%    The standard output here is similar to \cs{PackageWarningNoLine}.
%  \end{texnote}
%\end{function}
%
%\begin{function}{
%  \msg_info:nnnn|
%  \msg_info:nnn|
%  \msg_info:nn
%}
%  \begin{syntax}
%    "\msg_info:nnnn" <module> <name> <arg one> <arg two>
%  \end{syntax}
%  Prints <module> message <name> to the log, passing <arg one> and
%  <arg two> to the text-creating functions. 
%  \begin{texnote}
%    The standard output here is similar to \cs{PackageInfoNoLine}.
%  \end{texnote}
%\end{function}
%
%\begin{function}{
%  \msg_log:nnnn|
%  \msg_log:nnn|
%  \msg_log:nn
%}
%  \begin{syntax}
%    "\msg_info:nnnn" <module> <name> <arg one> <arg two>
%  \end{syntax}
%  Prints <module> message <name> to the log, passing <arg one> and
%  <arg two> to the text-creating functions. No continuation text is
%  added.
%\end{function}
%
%\begin{function}{
%  \msg_trace:nnnn|
%  \msg_trace:nnn|
%  \msg_trace:nn
%}
%  \begin{syntax}
%    "\msg_info:nnnn" <module> <name> <arg one> <arg two>
%  \end{syntax}
%  Prints <module> message <name> to the log, passing <arg one> and
%  <arg two> to the text-creating functions. No continuation text is
%  added.
%\end{function}
%
%\begin{function}{
%  \msg_none:nnnn|
%  \msg_none:nnn|
%  \msg_none:nn
%}
%  \begin{syntax}
%    "\msg_none:nnnn" <module> <name> <arg one> <arg two>
%  \end{syntax}
%  Does nothing: used for redirecting other message classes. 
%\end{function}
% 
% \subsection{Redirecting messages}
%
%\begin{function}{\msg_redirect_class:nn}
%  \begin{syntax}
%    "\msg_redirect_class:nn" <class one> <class two>
%  \end{syntax}
%  Redirect all messages of <class one> to appear as those of 
%  <class two>.
%\end{function}
%
%\begin{function}{\msg_redirect_module:nnn}
%  \begin{syntax}
%    "\msg_redirect_module:nnn" <class one> <module> <class two>
%  \end{syntax}
%  Redirect <module> messages of <class one> to appear in <class two>.
%  \begin{texnote}
%    This function can be used to make some messages ``silent'' by
%    default. For example, all of the \texttt{trace} messages of
%    <module> could be turned off with: 
%    \begin{verbatim}
%    \msg_redirect_module:nnn { trace } { module } { none }
%    \end{verbatim}
%  \end{texnote}
%\end{function}
%
%\begin{function}{\msg_redirect_name:nnn}
%  \begin{syntax}
%    "\msg_redirect_name:nnn" <module> <message> <class>
%  \end{syntax}
%  Redirect <module> <message> to appear using <class>.
%\end{function}
%
% \subsection{Support functions for output}
% 
%\begin{function}{\msg_line_context:}
%  \begin{syntax}
%    "\msg_line_context:" 
%  \end{syntax}
%  Prints the current line number preceded by \cs{c_msg_on_line_tl}.
%\end{function}
% 
%\begin{function}{\msg_line_number:}
%  \begin{syntax}
%    "\msg_line_number:" 
%  \end{syntax}
%  Prints the current line number.
%\end{function}
% 
%\begin{function}{
%  \msg_newline:|
%  \msg_two_newlines:
%}
%  \begin{syntax}
%    "\msg_newline:" 
%  \end{syntax}
%  Print one or two newlines with no continuation information.  
%\end{function}
%
%\begin{function}{
%  \msg_space:|
%  \msg_two_spaces:|
%  \msg_four_spaces:
%}
%  \begin{syntax}
%    "\msg_space:" 
%  \end{syntax}
%  Print one, two or four spaces: needed where a literal space would
%  otherwise be gobbled by \TeX.  
%\end{function}
%
% \subsection{Low-level functions}
% 
% The low-level functions do not make assumptions about module names.
% The output functions here produce messages directly, and do not 
% respond to redirection.
% 
%\begin{function}{
%  \msg_generic_new:nnnn|
%  \msg_generic_set:nnnn
%}
%  \begin{syntax}
%    "\msg_generic_new:nnnn"  <name> <text> <more text> <code>
%  \end{syntax}
%  Creates new message <name> to produce <text> initially, <more text>
%  if requested by the user and to insert <code> after the message when
%  used as an error. <text> and <more text> can use up to two macro
%  parameters (|#1| and |#2|), which are supplied by the message system.
%\end{function}
%
%\begin{function}{\msg_direct_interrupt:nnnxn}
%  \begin{syntax}
%    "\msg_direct_interrupt:nnnxn"  <first line> <text> 
%    ~~~~<continuation> <more text> <code>
%  \end{syntax}
%  Executes a \TeX\ error, interrupting compilation. The <first line> 
%  is displayed followed by <text> and the input prompt. <more text> is
%  displays if requested by the user, and <code> is inserted after the
%  message. If <more text> is blank a default is supplied. Each line of
%  <text> (broken with |\\|) begins with <continuation>.
%\end{function}
%
%\begin{function}{
%  \msg_direct_log:xn|
%  \msg_direct_term:xn
%}
%  \begin{syntax}
%    "\msg_direct_log:xn"  <text> <continuation>
%  \end{syntax}
%  Prints <text> to either the log or terminal. New lines (broken 
%  with |\\|) start with <continuation>. 
%  \begin{texnote}
%    \cs{msg_direct_log:xn} is equivalent to \LaTeXe's \cs{wlog}.
%  \end{texnote}
%\end{function}
%
% \subsection{Kernel-specific functions}
% 
%\begin{function}{
%  \msg_kernel_new:nnnn|
%  \msg_kernel_new:nnn|
%  \msg_kernel_new:nn|
%  \msg_kernel_set:nnnn|
%  \msg_kernel_set:nnn|
%  \msg_kernel_set:nn
%}
%  \begin{syntax}
%    "\msg_kernel_new:nnn" <name> <text> <more text> <code>
%  \end{syntax}
%  Creates new kernel message <name> to produce <text> initially, 
%  <more text> if requested by the user and to insert <code> after the 
%  message when used as an error. <text> and <more text> can use up to 
%  two macro parameters (|#1| and |#2|), which are supplied by the 
%  message system. 
%\end{function}
%
%\begin{function}{\msg_kernel_classes_new:n}
%  \begin{syntax}
%    "\msg_kernel_classes_new:n" <class>
%  \end{syntax}
%  Creates the \texttt{:n} and \texttt{:nn} variants for a kernel
%  message class.
%\end{function}
%
%\begin{function}{
%  \msg_kernel_fatal:nnn|
%  \msg_kernel_fatal:nn|
%  \msg_kernel_fatal:n
%}
%  \begin{syntax}
%    "\msg_kernel_fatal:nnn" <name> <arg one> <arg two>
%  \end{syntax}
%  Issues kernel error message <name>, passing <arg one> and 
%  <arg two> to the text-creating functions. The \TeX\ run then halts.
%  Cannot be redirected.
%\end{function}
%
%\begin{function}{
%  \msg_kernel_error:nnn|
%  \msg_kernel_error:nn|
%  \msg_kernel_error:n
%}
%  \begin{syntax}
%    "\msg_kernel_error:nnn" <name> <arg one> <arg two>
%  \end{syntax}
%  Issues kernel error message <name>, passing <arg one> and 
%  <arg two> to the text-creating functions. Cannot be redirected.
%\end{function}
%
%\begin{function}{
%  \msg_kernel_warning:nnn|
%  \msg_kernel_warning:nn|
%  \msg_kernel_warning:n
%}
%  \begin{syntax}
%    "\msg_kernel_warning:nnn" <name> <arg one> <arg two>
%  \end{syntax}
%  Prints kernel message <name> to the terminal, passing <arg one> and
%  <arg two> to the text-creating functions. 
%\end{function}
%
%\begin{function}{
%  \msg_kernel_info:nnn|
%  \msg_kernel_info:nn|
%  \msg_kernel_info:n
%}
%  \begin{syntax}
%    "\msg_kernel_info:nnn" <name> <arg one> <arg two>
%  \end{syntax}
%  Prints kernel message <name> to the log, passing <arg one> and
%  <arg two> to the text-creating functions. 
%\end{function}
% 
% \subsection{Variables and constants}
% 
%\begin{variable}{
%  \c_msg_fatal_tl|
%  \c_msg_error_tl|
%  \c_msg_warning_tl|
%  \c_msg_info_tl
%}
%  Simple headers for errors.
%\end{variable}
% 
%\begin{variable}{
%  \c_msg_fatal_text_tl|
%  \c_msg_help_text_tl|
%  \c_msg_no_info_text_tl|
%  \c_msg_return_text_tl
%}
%  Various pieces of text for use in messages, which are not changed by
%  the code here although they could be to alter the language.
%\end{variable}
%
%\begin{variable}{\c_msg_hide_tl}
%  An empty variable with a number of (category code 11) spaces at the
%  end of its name. This is used to push the \TeX\ part of an error 
%  message ``off the screen''.
%\end{variable}
%
%\begin{variable}{\c_msg_on_line_tl}
%  The ``on line'' phrase for line numbers.
%\end{variable}
%
%\begin{variable}{
%  \c_msg_text_prefix_tl|
%  \c_msg_more_text_prefix_tl|
%  \c_msg_code_prefix_tl
%}
%  Header information for storing the ``paths'' to parts of a message.
%\end{variable}
%
%\begin{variable}{
%  \l_msg_class_tl|
%  \l_msg_current_class_tl
%}
%  Information about message method, used for filtering.
%\end{variable}
%
%\begin{variable}{\l_msg_names_clist}
%  List of all of the message names defined.
%\end{variable}
%
%\begin{variable}{
%  \l_msg_redirect_classes_prop|
%  \l_msg_redirect_names_prop
%}
%  Re-direction lists containing the class of message to convert an 
%  different one.
%\end{variable}
%
%\begin{variable}{\l_msg_redirect_classes_clist}
%  List so that filtering does not loop.
%\end{variable}
% 
% \end{documentation}
% 
% \begin{implementation}
%
% \subsection{The implementation}
%
% The usual lead-off.
%    \begin{macrocode}
%<*package>    
\ProvidesExplPackage
  {\filename}{\filedate}{\fileversion}{\filedescription}
\RequirePackage{
  l3basics,
  l3tl,
  l3clist, 
  l3io,
  l3token,
  l3prop
}
%</package>
%<*initex|package>
%    \end{macrocode}
%    
% \LaTeX\ is handling context, so the \TeX\ ``noise'' is turned down.
%    \begin{macrocode}
\int_set:Nn \tex_errorcontextlines:D { \c_minus_one }
%    \end{macrocode}
%    
% \subsubsection{Variables and constants}
%    
%\begin{macro}{\c_msg_fatal_tl}
%\begin{macro}{\c_msg_error_tl}
%\begin{macro}{\c_msg_warning_tl}
%\begin{macro}{\c_msg_info_tl}
% Header information.
%    \begin{macrocode}
\tl_new:Nn \c_msg_fatal_tl   { Fatal Error }
\tl_new:Nn \c_msg_error_tl   { Error }
\tl_new:Nn \c_msg_warning_tl { Warning }
\tl_new:Nn \c_msg_info_tl    { Info }
%    \end{macrocode}
%\end{macro}    
%\end{macro} 
%\end{macro} 
%\end{macro} 
%    
%\begin{macro}{\c_msg_fatal_text_tl}
%\begin{macro}{\c_msg_help_text_tl}
%\begin{macro}{\c_msg_no_info_text_tl}
%\begin{macro}{\c_msg_return_text_tl}
% Simple pieces of text for messages.
%    \begin{macrocode}
\tl_new:Nn \c_msg_fatal_text_tl {
  This~is~a~fatal~error:~LaTeX~will~abort.
}
\tlp_new:Nn \c_msg_help_text_tl {
  For~immediate~help~type~H~<return>.
}
\tl_new:Nn \c_msg_no_info_text_tl {
  LaTeX~does~not~know~anything~more~about~this~error,~sorry.
  \c_msg_return_text_tl
}
\tl_new:Nn \c_msg_return_text_tl {
  \msg_two_newlines: 
  Try~typing~<return>~to~proceed.
  \msg_newline:
  If~that~doesn't~work,~type~X~<return>~to~quit.
}
%    \end{macrocode}
%\end{macro}    
%\end{macro} 
%\end{macro} 
%\end{macro} 
%
%\begin{macro}{\c_msg_hide_tl}
% This needs lots of spaces in the name, as it is used to effectively
% hide \TeX's error information and only leave \LaTeX's on the screen.
% No indentation here as \verb*| | is a letter!
%    \begin{macrocode}
\group_begin:
\char_make_letter:N\ %
\tlist_to_lowercase:n{%
\group_end:%
\tl_new:Nn%
\c_msg_hide_tl                                                         %
{}%
}%
%    \end{macrocode}
%\end{macro}
%
%\begin{macro}{\c_msg_on_line_tl}
% ``On line''.
%    \begin{macrocode}
\tlp_new:Nn \c_msg_on_line_tl { on~line }
%    \end{macrocode}
%\end{macro}
%
%\begin{macro}{\c_msg_text_prefix_tl}
%\begin{macro}{\c_msg_more_text_prefix_tl}
%\begin{macro}{\c_msg_code_prefix_tl}
% Prefixes for storage areas.
%    \begin{macrocode}
\tl_new:Nn \c_msg_text_prefix_tl      { msg_text // }
\tl_new:Nn \c_msg_more_text_prefix_tl { msg_text_more // }
\tl_new:Nn \c_msg_code_prefix_tl      { msg_code // }
%    \end{macrocode}
%\end{macro}
%\end{macro}
%\end{macro}
%
%\begin{macro}{\l_msg_class_tl}
%\begin{macro}{\l_msg_current_class_tl}
% For holding the current message method and that for redirection.
%    \begin{macrocode}
\tl_new:N \l_msg_class_tl
\tl_new:N \l_msg_current_class_tl
%    \end{macrocode}
%\end{macro}
%\end{macro}
%
%\begin{macro}{\l_msg_names_clist}
% A list of all of the messages defined.
%    \begin{macrocode}
\clist_new:N \l_msg_names_clist
%    \end{macrocode}
%\end{macro}
%
%\begin{macro}{\l_msg_redirect_classes_prop}
%\begin{macro}{\l_msg_redirect_names_prop}
% For filtering messages, a list of all messages and of those which have
% to be modified is required.
%    \begin{macrocode}
\prop_new:N \l_msg_redirect_classes_prop
\prop_new:N \l_msg_redirect_names_prop
%    \end{macrocode}
%\end{macro}
%\end{macro}
%\begin{macro}{\l_msg_redirect_classes_clist}
% To prevent an infinite loop.
%    \begin{macrocode}
\clist_new:N \l_msg_redirect_classes_clist
%    \end{macrocode}
%\end{macro}
%
% \subsubsection{Output helper functions}
%
%\begin{macro}{\msg_line_number:}
%\begin{macro}{\msg_line_context:}
% For writing the line number nicely.
%    \begin{macrocode}
\cs_new_nopar:Nn { \msg_line_number: } {
  \toks_use:N \tex_inputlineno:D
}
\cs_new_nopar:Nn { \msg_line_context: } {
  \msg_space:
  \c_msg_on_line_tl
  \msg_space:
  \msg_line_number:
}
%    \end{macrocode}
%\end{macro}
%\end{macro}
%
%\begin{macro}{\msg_newline:}
%\begin{macro}{\msg_two_newlines:}
% Always forces a new line.
%    \begin{macrocode}
\cs_new_nopar:Nn \msg_newline:      { ^^J }
\cs_new_nopar:Nn \msg_two_newlines: { ^^J ^^J }
%    \end{macrocode}
%\end{macro}
%\end{macro}
%    
%\begin{macro}{\msg_space:}
%\begin{macro}{\msg_two_spaces:}
%\begin{macro}{\msg_four_spaces:}
% For printing spaces, some very simple functions.
%    \begin{macrocode}
\cs_new_nopar:Nn \msg_space:       { ~ }
\cs_new_nopar:Nn \msg_two_spaces:  { \msg_space: \msg_space: }
\cs_new_nopar:Nn \msg_four_spaces: { \msg_two_spaces: \msg_two_spaces: }
%    \end{macrocode}
%\end{macro}
%\end{macro}
%\end{macro}
%
%\subsubsection{Generic functions}
%
% The lowest level functions make no assumptions about modules, 
% \emph{etc.}
%
%\begin{macro}{\msg_generic_new:nnnn}
% Creating a new message is basically the same as the non-checking
% version, and so after a check everything hands over.
%    \begin{macrocode}
\cs_new_nopar:Npn \msg_generic_new:nnnn #1 {  
  \exp_args:Nc \chk_if_new_cs:N { \c_msg_text_prefix_tl #1 :nn }
  \msg_generic_set:nnnn {#1}
}
%    \end{macrocode}
%\end{macro}
%
%\begin{macro}{\msg_generic_set:nnnn}
% Creating a message is quite simple. There must be a short text part,
% while the other parts may not exist. To avoid filling up the hash 
% table with empty functions,only non-empty arguments are stored.
%    \begin{macrocode}
\cs_new:Nn \msg_generic_set:nnnn {  
  \cs_set:cn { \c_msg_text_prefix_tl #1 :nn } {#2}
  \clist_if_in:NnF \l_msg_names_clist { // #1 / } {
    \clist_put_right:Nn \l_msg_names_clist { // #1 / }
  }
  \tl_if_empty:nTF {#3} {
    \cs_set_eq:cN { \c_msg_more_text_prefix_tl #1 } \c_undefined
  }{
    \cs_set:cn { \c_msg_more_text_prefix_tl #1 :nn } {#3}
  }
  \tl_if_empty:nTF {#4} {
    \cs_set_eq:cN { \c_msg_code_prefix_tl #1 : } \c_undefined
  }{
    \cs_set:cn { \c_msg_code_prefix_tl #1 : } {#4}
  }
}
%    \end{macrocode}
%\end{macro}
%
%\begin{macro}{\msg_direct_interrupt:nnnxn}
%\begin{macro}[aux]{\msg_direct_interrupt:n}
% The low-level interruption macro is rather opaque, unfortunately. The 
% idea here is to create a a message which hides all of \TeX's own 
% information by filling the output up with spaces. To achieve this, 
% spaces have to be letters: hence no indentation. The odd 
% \cs{c_msg_hide_tl} actually does the hiding: it is the large run of 
% spaces in the name that is important here. The meaning of |\\|
% is altered so that the explanation text is a simple run whilst the 
% initial error has line-continuation shown.
%    \begin{macrocode}
\group_begin:
  \char_set_lccode:nn {`\&} {`\ } % {
  \char_set_lccode:w `\} = `\ \scan_stop:
  \char_make_active:N \&
  \char_make_letter:N\ %
\tlist_to_lowercase:n{%
\group_end:%
\cs_new_protected:Nn\msg_direct_interrupt:nnnxn{%
\group_begin:%
\cs_set_eq:NN\\\msg_newline:%
\msg_direct_interrupt_aux:n{#4}%
\cs_set_nopar:Npn\\{\msg_newline:#3}%
\tex_errhelp:D\l_msg_tmp_tl%
\cs_set:Npn&{%
\tex_errmessage:D{%
#1\msg_newline:%
#2\msg_two_newlines:%
\c_msg_help_text_tl%
\c_msg_hide_tl                                                         %
}%
}%
&%
\group_end:%
#5%
}%
}%
\cs_new:Nn \msg_direct_interrupt_aux:n {
  \tl_if_empty:nTF {#1} {
    \tl_set:Nx \l_msg_tmp_tl { { \c_msg_no_info_text_tl } }
  }{
    \tl_set:Nx \l_msg_tmp_tl { {#1 } }
  }
}
%    \end{macrocode}
%\end{macro}
%\end{macro}
%
%\begin{macro}{\msg_direct_log:xn}
%\begin{macro}{\msg_direct_term:xn}
% Printing to the log or terminal without a stop is rather easier.
%    \begin{macrocode}
\cs_new_protected:Nn \msg_direct_log:xn {
  \group_begin:
    \cs_set:Npn \\ { \msg_newline: #2 }
    \iow_log:x { #1 \msg_newline: }
  \group_end:
}
\cs_new_protected:Nn \msg_direct_term:xn {
  \group_begin:
    \cs_set:Npn \\ { \msg_newline: #2 }
    \iow_term:x { #1 \msg_newline: }
  \group_end:
}
%    \end{macrocode}
%\end{macro}
%\end{macro}
%
%\subsubsection{General functions}
%
% The main functions for messaging are built around the separation of
% module from the message name. These have short names as they will be
% widely used.
% 
%\begin{macro}{\msg_new:nnnnn}
%\begin{macro}{\msg_new:nnnn}
%\begin{macro}{\msg_new:nnn}
%\begin{macro}{\msg_set:nnnnn}
%\begin{macro}{\msg_set:nnnn}
%\begin{macro}{\msg_set:nnn}
% For making messages.
%    \begin{macrocode}
\cs_new_nopar:Npn \msg_new:nnnnn #1#2 {
  \msg_generic_new:nnnn { #1 / #2 }
}
\cs_new:Nn \msg_new:nnnn {
  \msg_generic_new:nnnn { #1 / #2 } {#3}  {#4} { }
}
\cs_new:Nn \msg_new:nnn {
  \msg_generic_new:nnnn { #1 / #2 } {#3}  { } { }
}
\cs_new_nopar:Npn \msg_set:nnnnn #1#2 {
  \msg_generic_set:nnnn { #1 / #2 }
}
\cs_new:Nn \msg_set:nnnn {
  \msg_generic_set:nnnn { #1 / #2 } {#3} {#4} { }
}
\cs_new:Nn \msg_set:nnn {
  \msg_generic_set:nnnn { #1 / #2 } {#3} { } { }
}
%    \end{macrocode}
%\end{macro}
%\end{macro}
%\end{macro}
%\end{macro} 
%\end{macro}
%\end{macro}
%
%\begin{macro}{\msg_class_new:nn}
%\begin{macro}{\msg_class_set:nn}
% Creating a new class produces three new functions, with varying 
% numbers of arguments. The \cs{msg_class_loop:n} function is set up
% so that redirection will work as desired.
%    \begin{macrocode}
\cs_new_nopar:Npn \msg_class_new:nn #1 {
  \exp_args:Nc \chk_if_new_cs:N { msg_ #1 :nnnn }
  \prop_new:c { l_msg_ #1 _redirect_prop }
  \msg_class_set:nn {#1}
}
\cs_new_nopar:Nn \msg_class_set:nn {
  \prop_clear:c { l_msg_ #1 _redirect_prop }
  \cs_set_protected:cn { msg_ #1 :nnnn } {
    \cs_set:Nn \msg_class_loop:n {
      \clist_if_in:NnTF \l_msg_redirect_classes_clist {#1} {
        \msg_kernel_error:n { message~loop }
      }{
        \clist_put_right:Nn \l_msg_redirect_classes_clist {#1}
        \cs_if_exist:cTF { msg_ ####1 :nnnn } {
          \use:c { msg_ ####1 :nnnn } {##1} {##2} {##3} {##4}
        }{
          \msg_kernel_error:nn { message~class~unknown } { ####1 }
        }
      }
    }
    \cs_if_exist:cTF { \c_msg_text_prefix_tl ##1 / ##2 :nn } {
      \msg_class_aux_i:nnnn {#1} {#2} {##1} {##2} 
    }{
      \msg_kernel_error:nnn { message~unknown } { ##1 } { ##2 }
    }
  }
  \cs_set:cn { msg_ #1 :nnn } {
    \use:c { msg_ #1 :nnnn } {##1} {##2} {##3} { }
  }
  \cs_set:cn { msg_ #1 :nn } {
    \use:c { msg_ #1 :nnnn } {##1} {##2} { } { }
  }
}
%    \end{macrocode}
%\end{macro}
%\end{macro}
%\begin{macro}[aux]{\msg_class_loop:n}
% An auxiliary function for looping when filtering.
%    \begin{macrocode}
\cs_new:Nn \msg_class_loop:n { }
%    \end{macrocode}
%\end{macro}
%\begin{macro}[aux]{\msg_class_aux_i:nnnn}
% The first auxiliary function checks if the current class of message
% (|#1|) is to converted into a different class.
%    \begin{macrocode}
\cs_new_nopar:Nn \msg_class_aux_i:nnnn {
  \prop_if_in:NnTF \l_msg_redirect_classes_prop {#1} {
    \prop_get:NnN \l_msg_redirect_classes_prop {#1} \l_msg_class_tl
      \msg_class_loop:n { \l_msg_class_tl }
  }{
    \msg_class_aux_ii:nnnn {#1} {#2} {#3} {#4}
  }
}
%    \end{macrocode}
%\end{macro}
%\begin{macro}[aux]{\msg_class_aux_ii:nnnn}
% The second step of the redirection is to see if the current module is
% listed as one to alter for the current message class.
%    \begin{macrocode}
\cs_new:Nn \msg_class_aux_ii:nnnn {
  \exp_args:Nc \prop_if_in:NnTF { l_msg_ #1 _redirect_prop } {#3} {
    \tl_set:Nn \l_msg_current_type_tl {#1}
    \prop_get:cnN { l_msg_ #1 _redirect_prop } {#3}
      \l_msg_class_tl
    \tl_if_eq:NNTF \l_msg_current_type_tl \l_msg_class_tl {
      \msg_class_aux_iii:nnnn {#1} {#2} {#3} {#4}
    }{
      \msg_class_loop:n { \l_msg_class_tl }
    }
  }{
    \msg_class_aux_iii:nnnn {#1} {#2} {#3} {#4}
  }
}
%    \end{macrocode}
%\end{macro}
%\begin{macro}[aux]{\msg_class_aux_iii:nnnn}
% The final stage is to check if the current message name is on the
% list to be filtered. If it is, the appropriate filter is applied 
% unless the filter points to the current message method. If the message
% is not on the filter list at all, or if the filter points to the 
% current method, the code for the message (|#2|) is finally used.
%    \begin{macrocode}
\cs_new:Nn \msg_class_aux_iii:nnnn {
  \prop_if_in:NnTF \l_msg_redirect_names_prop { // #3 / #4 / } {
    \tl_set:Nn \l_msg_current_type_tl {#1}
    \prop_get:NnN \l_msg_redirect_names_prop { // #3 / #4 / } 
      \l_msg_class_tl
    \tl_if_eq:NNTF \l_msg_current_type_tl \l_msg_class_tl {
      \clist_clear:N \l_msg_redirect_classes_clist
      #2
    }{
      \msg_class_loop:n { \l_msg_class_tl }
    }
  }{
    \clist_clear:N \l_msg_redirect_classes_clist
    #2 
  }
}
%    \end{macrocode}
%\end{macro}
%
%\begin{macro}{\msg_fatal:nnnn}
%\begin{macro}{\msg_fatal:nnn}
%\begin{macro}{\msg_fatal:nn}
% For fatal errors, after the error message \TeX\ bails out.
%    \begin{macrocode}
\msg_class_new:nn { fatal } {
  \msg_direct_interrupt:nnnxn
    { \c_msg_fatal_tl \msg_two_newlines: } 
    { 
      ( \c_msg_fatal_tl ) \msg_space:
      \use:c { \c_msg_text_prefix_tl // #1 / #2 :nn } {#3} {#4} 
    }
    { ( \c_msg_fatal_tl ) \msg_space: }
    { \c_msg_fatal_text_tl }
    { \tex_end:D }
}
%    \end{macrocode}
%\end{macro}
%\end{macro}
%\end{macro}
%
%\begin{macro}{\msg_error:nnnn}
%\begin{macro}{\msg_error:nnn}
%\begin{macro}{\msg_error:nn}
% For an error, the interrupt routine is called, then any recovery code
% is tried.
%    \begin{macrocode}
\msg_class_new:nn { error } {
  \msg_direct_interrupt:nnnxn
    { #1~\c_msg_error_tl \msg_newline: } 
    { 
      ( #1 ) \msg_space:
      \use:c { \c_msg_text_prefix_tl #1 / #2 :nn } {#3} {#4} 
    }
    { ( #1 ) \msg_space: }
    { 
      \cs_if_exist:cTF { \c_msg_more_text_prefix_tl #1 / #2 :nn } {
        \use:c { \c_msg_more_text_prefix_tl #1 / #2 :nn } {#3} {#4}
      }{ 
        \c_msg_no_info_text_tl
      } 
    }
    { 
      \cs_if_exist:cT { \c_msg_code_prefix_tl #1 /#2 :nn } {
        \use:c { \c_msg_code_prefix_tl #1 / #2 :nn} {#3} {#4}
      } 
    }
}
%    \end{macrocode}
%\end{macro}
%\end{macro}
%\end{macro}
%
%\begin{macro}{\msg_warning:nnnn}
%\begin{macro}{\msg_warning:nnn}
%\begin{macro}{\msg_warning:nn}
% Warnings are printed to the terminal.
%    \begin{macrocode}
\msg_class_new:nn { warning } {
  \msg_direct_term:xn { 
    \msg_space: #1~\c_msg_warning_tl :~  
    \use:c { \c_msg_text_prefix_tl #1 / #2 :nn } {#3} {#4} 
  }
  { ( #1 ) \msg_two_spaces: }
}
%    \end{macrocode}
%\end{macro}
%\end{macro}
%\end{macro}
%
%\begin{macro}{\msg_info:nnnn}
%\begin{macro}{\msg_info:nnn}
%\begin{macro}{\msg_info:nn}
% Information only goes into the log.
%    \begin{macrocode}
\msg_class_new:nn { info } {
  \msg_direct_log:xn { 
    \msg_space: #1~\c_msg_info_tl :~  
    \use:c { \c_msg_text_prefix_tl #1 / #2 :nn } {#3} {#4} 
  }
  { ( #1 ) \msg_two_spaces: }
}
%    \end{macrocode}
%\end{macro}
%\end{macro}
%\end{macro}
%
%\begin{macro}{\msg_log:nnnn}
%\begin{macro}{\msg_log:nnn}
%\begin{macro}{\msg_log:nn}
%  ``Log'' data is very similar to information, but with no extras 
%  added.
%    \begin{macrocode}
\msg_class_new:nn { log } {
  \msg_direct_log:xn { 
    \use:c { \c_msg_text_prefix_tl #1 / #2 :nn } {#3} {#4} 
  }
  { }
}
%    \end{macrocode}
%\end{macro}
%\end{macro}
%\end{macro}
%
%\begin{macro}{\msg_trace:nnnn}
%\begin{macro}{\msg_trace:nnn}
%\begin{macro}{\msg_trace:nn}
%  Trace data is the same as log data, more or less
%    \begin{macrocode}
\msg_class_new:nn { trace } {
  \msg_direct_log:xn { 
    \use:c { \c_msg_text_prefix_tl #1 / #2 :nn } {#3} {#4} 
  }
  { }
}
%    \end{macrocode}
%\end{macro}
%\end{macro}
%\end{macro}
%
%\begin{macro}{\msg_none:nnnn}
%\begin{macro}{\msg_none:nnn}
%\begin{macro}{\msg_none:nn}
% The \texttt{none} message type is needed so that input can be gobbled.
%    \begin{macrocode}
\msg_class_new:nn { none } { }
%    \end{macrocode}
%\end{macro}
%\end{macro}
%\end{macro}
%
%\subsubsection{Redirection functions}
%
%\begin{macro}{\msg_redirect_class:nn}
% Converts class one into class two.
%    \begin{macrocode}
\cs_new_nopar:Nn \msg_redirect_class:nn {
  \prop_put:Nnn \l_msg_redirect_classes_prop {#1} {#2}
}
%    \end{macrocode}
%\end{macro}
%
%\begin{macro}{\msg_redirect_name:nnn}
% Named message will always use the given class.
%    \begin{macrocode}
\cs_new_nopar:Nn \msg_redirect_name:nnn {
  \prop_put:Nnn \l_msg_redirect_names_prop { // #1 / #2 / } {#3}
}
%    \end{macrocode}
%\end{macro}
%
%\begin{macro}{\msg_redirect_module:nnn}
% For when all messages of a class should be altered for a given module.
%    \begin{macrocode}
\cs_new_nopar:Nn \msg_redirect_module:nnn {
  \prop_put:cnn { l_msg_ #1 _redirect_prop } {#2} {#3}
}
%    \end{macrocode}
%\end{macro}
%
%\subsubsection{Kernel-specific functions}
%
%\begin{macro}{\msg_kernel_new:nnnn}
%\begin{macro}{\msg_kernel_new:nnn}
%\begin{macro}{\msg_kernel_new:nn}
%\begin{macro}{\msg_kernel_set:nnnn}
%\begin{macro}{\msg_kernel_set:nnn}
%\begin{macro}{\msg_kernel_set:nn}
% The kernel needs some messages of its own. These are created using
% pre-built functions. Two functions are provided: one more general
% and one which only has the short text part.
%    \begin{macrocode}
\cs_new_nopar:Npn \msg_kernel_new:nnnn #1 {
  \msg_new:nnnnn { LaTeX } {#1}
}
\cs_new_nopar:Npn \msg_kernel_new:nnn #1 {
  \msg_new:nnnn { LaTeX } {#1} 
}
\cs_new_nopar:Npn \msg_kernel_new:nn #1 {
  \msg_new:nnn { LaTeX } {#1} 
}
\cs_new_nopar:Npn \msg_kernel_set:nnnn #1 {
  \msg_set:nnnnn { LaTeX } {#1}
}
\cs_new_nopar:Npn \msg_kernel_set:nnn #1 {
  \msg_set:nnnn { LaTeX } {#1} 
}
\cs_new_nopar:Npn \msg_kernel_set:nn #1 {
  \msg_set:nnn { LaTeX } {#1} 
}
%    \end{macrocode}
%\end{macro}
%\end{macro}
%\end{macro}
%\end{macro}
%\end{macro}
%\end{macro}
%
%\begin{macro}{\msg_kernel_classes_new:n}
%    \begin{macrocode}
\cs_new_nopar:Nn \msg_kernel_classes_new:n {
  \cs_new_protected:cn { msg_kernel_ #1 :nn } {
    \use:c { msg_kernel_ #1 :nnn } {##1} {##2} { }
  }
  \cs_new_protected:cn { msg_kernel_ #1 :n } {
    \use:c { msg_kernel_ #1 :nnn } {##1} { } { }
  }
}
%    \end{macrocode}
%\end{macro}
%
%\begin{macro}{\msg_kernel_fatal:nnn}
%\begin{macro}{\msg_kernel_fatal:nn}
%\begin{macro}{\msg_kernel_fatal:n}
% Fatal kernel errors cannot be re-defined.
%    \begin{macrocode}
\cs_new_protected:Nn \msg_kernel_fatal:nnn {
  \msg_direct_interrupt:nnnxn
    { \c_msg_fatal_tl \msg_two_newlines: } 
    { 
      ( LaTeX ) \msg_space:
      \use:c { \c_msg_text_prefix_tl // LaTeX / #1 :nn } {#2} {#3} 
    }
    { ( LaTeX ) \msg_space: }
    { \c_msg_fatal_text_tl }
    { \tex_end:D }
}
\msg_kernel_classes_new:n { fatal }
%    \end{macrocode}
%\end{macro}
%\end{macro}
%\end{macro}
%
%\begin{macro}{\msg_kernel_error:nnn}
%\begin{macro}{\msg_kernel_error:nn}
%\begin{macro}{\msg_kernel_error:n}
% Neither can kernel errors.
%    \begin{macrocode}
\cs_new_protected:Nn \msg_kernel_error:nnn {
  \msg_direct_interrupt:nnnxn
    { LaTeX~\c_msg_error_tl \msg_newline: } 
    { 
      ( LaTeX ) \msg_space:
      \use:c { \c_msg_text_prefix_tl LaTeX / #1 :nn } {#2} {#3} 
    }
    { ( LaTeX ) \msg_space: }
    { 
      \cs_if_exist:cTF { \c_msg_more_text_prefix_tl LaTeX / #1 :nn } {
        \use:c { \c_msg_more_text_prefix_tl LaTeX / #1 :nn } {#2} {#3}
      }{ 
        \c_msg_no_info_text_tl
      } 
    }
    { 
      \cs_if_exist:cT { \c_msg_code_prefix_tl LaTeX /#1 :nn } {
        \use:c { \c_msg_code_prefix_tl LaTeX / #1 :nn} {#2} {#3}
      } 
    }
}
\msg_kernel_classes_new:n { error }
%    \end{macrocode}
%\end{macro}
%\end{macro}
%\end{macro}
%
%\begin{macro}{\msg_kernel_warning:nnn}
%\begin{macro}{\msg_kernel_warning:nn}
%\begin{macro}{\msg_kernel_warning:n}
%\begin{macro}{\msg_kernel_info:nnn}
%\begin{macro}{\msg_kernel_info:nn}
%\begin{macro}{\msg_kernel_info:n}
% Life is much more simple for warnings and information messages, as
% these are just short-cuts to the standard classes.
%    \begin{macrocode}
\cs_new_protected_nopar:Npn \msg_kernel_warning:nnn {
  \msg_warning:nnnn { LaTeX }
}
\msg_kernel_classes_new:n { warning }
\cs_new_protected_nopar:Npn \msg_kernel_info:nnn {
  \msg_info:nnnn { LaTeX }
}
\msg_kernel_classes_new:n { info }
%    \end{macrocode}
%\end{macro}
%\end{macro}
%\end{macro}
%\end{macro}
%\end{macro}
%\end{macro}
%
% Some very basic error messages.
%    \begin{macrocode}
\msg_kernel_new:nnn 
  { coding~bug } 
  { This~is~a~LaTeX~bug:~check~coding! \\ #1 }
  {#2}
\msg_kernel_new:nnn 
  { message~unknown } 
  { Unknown~message~`#2'~for~module~`#1'. }
  { LaTeX~was~asked~to~display~a~message~by~the~`#1'~module.\\
    The~message~was~supposed~to~be~called~`#2',~but~I~can't\\
    find~a~message~with~that~name.
    \c_msg_return_text_tl }
\msg_kernel_new:nnn 
  { message~class~unknown } 
  { Unknown~message~class~`#1'. }
  { You~have~asked~for~a~message~to~be~redirected~to~class~`#1' \\
    but~this~class~is~unknown. 
    \c_msg_return_text_tl }
\msg_kernel_new:nnn 
  { message~loop } 
  { Message~redirection~loop. }
  { You~have~asked~for~a~message~to~be~redirected,\\
    but~the~redirection~instructions~form~a~loop:\\
    you've~lost~the~message. 
    \c_msg_return_text_tl }
%    \end{macrocode}
%    
%    \begin{macrocode}
%</initex|package>
%    \end{macrocode}
%
% \end{implementation}
% 
% \PrintIndex
