% \iffalse
%% File: l3token.dtx Copyright (C) 2005-2009 LaTeX3 project
%%
%% It may be distributed and/or modified under the conditions of the
%% LaTeX Project Public License (LPPL), either version 1.3c of this
%% license or (at your option) any later version.  The latest version
%% of this license is in the file
%%
%%    http://www.latex-project.org/lppl.txt
%%
%% This file is part of the ``expl3 bundle'' (The Work in LPPL)
%% and all files in that bundle must be distributed together.
%%
%% The released version of this bundle is available from CTAN.
%%
%% -----------------------------------------------------------------------
%%
%% The development version of the bundle can be found at
%%
%%    http://www.latex-project.org/cgi-bin/cvsweb.cgi/
%%
%% for those people who are interested.
%%
%%%%%%%%%%%
%% NOTE: %%
%%%%%%%%%%%
%%
%%   Snapshots taken from the repository represent work in progress and may
%%   not work or may contain conflicting material!  We therefore ask
%%   people _not_ to put them into distributions, archives, etc. without
%%   prior consultation with the LaTeX Project Team.
%%
%% -----------------------------------------------------------------------
%<*driver|package>
\RequirePackage{l3names}
%</driver|package>
%\fi
\GetIdInfo$Id$
       {L3 Experimental token investigation and manipulation}
%\iffalse
%<*driver>
%\fi
\ProvidesFile{\filename.\filenameext}
  [\filedate\space v\fileversion\space\filedescription]
%\iffalse
\documentclass[full]{l3doc}
\begin{document}
\DocInput{\filename.\filenameext}
\end{document}
%</driver>
% \fi
%
% \title{The \textsf{l3token} package\thanks{This file
%         has version number \fileversion, last
%         revised \filedate.}\\
% Basic Definitions}
% \author{\Team}
% \date{\filedate}
% \maketitle
%
%
% \section{A token of my appreciation\dots}
%
% This module deals with tokens. Now this is perhaps not the most
% precise description so let's try with a better description: When
% programming in \TeX, it is often desirable to know just what a
% certain token is: is it a control sequence or something
% else. Similarly one often needs to know if a control sequence is
% expandable or not, a macro or a primitive, how many arguments it
% takes etc. Another thing of great importance (especially when it
% comes to document commands) is looking ahead in the token stream to
% see if a certain character is present and maybe even remove it or
% disregard other tokens while scanning. This module provides
% functions for both and as such will have two primary function
% categories: |\token| for anything that deals with tokens and
% |\peek| for looking ahead in the token stream.
%
% Most of the time we will be using the term `token' but most of the
% time the function we're describing can equally well by used on a
% control sequence as such one is one token as well.
%
% We shall refer to list of tokens as |tlist|s and such lists
% represented by a single control sequence is a `token list pointer'
% |tlp|. Functions for these two types are found in the \textsf{l3tlp}
% module.
%
% \subsection{Character tokens}
%
% \begin{function}{
%     \char_set_catcode:nn |
%     \char_set_catcode:w  |
%     \char_value_catcode:n |
%     \char_value_catcode:w |
%     \char_show_value_catcode:n |
%     \char_show_value_catcode:w 
% }
%   \begin{syntax}
%     "\char_set_catcode:nn" \Arg{char number} \Arg{number}\\
%     "\char_set_catcode:w" <char> = <number> \\
%     "\char_value_catcode:n" \Arg{char number} \\
%     "\char_show_value_catcode:n" \Arg{char number}
%   \end{syntax}
%   "\char_set_catcode:nn" sets the category code of a character,
%   "\char_value_catcode:n" returns its value for use in integer tests
%   and "\char_show_value_catcode:n" pausing the typesetting and prints 
%   the value on the terminal and in the log file. 
%   The ":w" form should be avoided. 
%   (Will: should we then just not mention it?)
%
%   "\char_set_catcode" is more usefully abstracted below.
% \begin{texnote}
% "\char_set_catcode:w" is the \TeX\ primitive \tn{catcode} renamed.
% \end{texnote}
% \end{function}
%
% \begin{function}{
%              \char_make_escape:n  |
%         \char_make_begin_group:n  |
%           \char_make_end_group:n  |
%          \char_make_math_shift:n  |
%           \char_make_alignment:n  |
%            \char_make_end_line:n  |
%           \char_make_parameter:n  |
%    \char_make_math_superscript:n  |
%      \char_make_math_subscript:n  |
%              \char_make_ignore:n  |
%               \char_make_space:n  |
%              \char_make_letter:n  |
%               \char_make_other:n  |
%              \char_make_active:n  |
%             \char_make_comment:n  |
%             \char_make_invalid:n  }
% \begin{syntax}
% "\char_make_letter:n" \Arg{character number}
% "\char_make_letter:n {64}"
% "\char_make_letter:n {`\@}"
% \end{syntax}
% Sets the catcode of the character referred to by its <character number>.
% \end{function}
%
% \begin{function}{
%              \char_make_escape:N  |
%         \char_make_begin_group:N  |
%           \char_make_end_group:N  |
%          \char_make_math_shift:N  |
%           \char_make_alignment:N  |
%            \char_make_end_line:N  |
%           \char_make_parameter:N  |
%    \char_make_math_superscript:N  |
%      \char_make_math_subscript:N  |
%              \char_make_ignore:N  |
%               \char_make_space:N  |
%              \char_make_letter:N  |
%               \char_make_other:N  |
%              \char_make_active:N  |
%             \char_make_comment:N  |
%             \char_make_invalid:N  }
% \begin{syntax}
% "\char_make_letter:N" \Arg{character}
% "\char_make_letter:N" "@"
% "\char_make_letter:N" "\%"
% \end{syntax}
% Sets the catcode of the <character>, which may have to be escaped.
% \begin{texnote}
% "\char_make_other:N" is \LaTeXe's "\@makeother".
% \end{texnote}
% \end{function}
%
% \begin{function}{
%     \char_set_lccode:nn |
%     \char_set_lccode:w  |
%     \char_value_lccode:n |
%     \char_value_lccode:w |
%     \char_show_value_lccode:n |
%     \char_show_value_lccode:w 
% }
%   \begin{syntax}
%     "\char_set_lccode:nn" \Arg{char} \Arg{number} \\
%     "\char_set_lccode:w" <char> = <number> \\
%     "\char_value_lccode:n" \Arg{char}\\
%     "\char_show_value_lccode:n" \Arg{char}
%   \end{syntax}
%   Set the lower caser representation of <char> for when <char> is
%   being converted in "\tlist_to_lowercase:n". As above, the
%   ":w" form is only for people who really, really know what they are
%   doing.
% \begin{texnote}
% "\char_set_lccode:w" is the \TeX\ primitive \tn{lccode} renamed.
% \end{texnote}
% \end{function}
%
% \begin{function}{
%     \char_set_uccode:nn |
%     \char_set_uccode:w  |
%     \char_value_uccode:n |
%     \char_value_uccode:w |
%     \char_show_value_uccode:n |
%     \char_show_value_uccode:w 
% }
%   \begin{syntax}
%     "\char_set_uccode:nn" \Arg{char} \Arg{number}\\
%     "\char_set_uccode:w" <char> = <number> \\
%     "\char_value_uccode:n" \Arg{char}\\
%     "\char_show_value_uccode:n" \Arg{char}
%   \end{syntax} 
%   Set the uppercase representation of <char> for when <char> is being
%   converted in "\tlist_to_uppercase:n". As above, the ":w"
%   form is only for people who really, really know what they are
%   doing.
% \begin{texnote}
% "\char_set_uccode:w" is the \TeX\ primitive \tn{uccode} renamed.
% \end{texnote}
% \end{function}
%
% \begin{function}{
%     \char_set_sfcode:nn |
%     \char_set_sfcode:w  |
%     \char_value_sfcode:n |
%     \char_value_sfcode:w |
%     \char_show_value_sfcode:n |
%     \char_show_value_sfcode:w 
% }
%   \begin{syntax}
%     "\char_set_sfcode:nn" \Arg{char} \Arg{number}\\
%     "\char_set_sfcode:w" <char> = <number> \\
%     "\char_value_sfcode:n" \Arg{char}\\
%     "\char_show_value_sfcode:n" \Arg{char}
%   \end{syntax} 
%   Set the space factor for <char>.
% \begin{texnote}
% "\char_set_sfcode:w" is the \TeX\ primitive \tn{sfcode} renamed.
% \end{texnote}
% \end{function}
%
% \begin{function}{
%     \char_set_mathcode:nn |
%     \char_set_mathcode:w  |
%     \char_gset_mathcode:nn |
%     \char_gset_mathcode:w  |
%     \char_value_mathcode:n |
%     \char_value_mathcode:w |
%     \char_show_value_mathcode:n |
%     \char_show_value_mathcode:w 
% }
%   \begin{syntax}
%     "\char_set_mathcode:nn" \Arg{char} \Arg{number}\\
%     "\char_set_mathcode:w" <char> = <number> \\
%     "\char_value_mathcode:n" \Arg{char}\\
%     "\char_show_value_mathcode:n" \Arg{char}
%   \end{syntax} 
%   Set the math code for <char>.
% \begin{texnote}
% "\char_set_mathcode:w" is the \TeX\ primitive \tn{mathcode} renamed.
% \end{texnote}
% \end{function}
%
%
% \subsection{Generic tokens}
%
%  \begin{function}{%
%                   \token_new:Nn |
%  }
%  \begin{syntax}
%     "\token_new:Nn" <token 1> \Arg{token 2}
%  \end{syntax}
%  Defines <token 1> to globally be a snapshot of <token 2>. This will
%  be an implicit representation of <token 2>.
%  \end{function}
%
%  \begin{variable}{%
%                   \c_group_begin_token |
%                   \c_group_end_token |
%                   \c_math_shift_token |
%                   \c_alignment_tab_token |
%                   \c_parameter_token |
%                   \c_math_superscript_token |
%                   \c_math_subscript_token |
%                   \c_space_token |
%                   \c_letter_token |
%                   \c_other_char_token |
%                   \c_active_char_token |
%  }
%  \begin{syntax}
%  \end{syntax}
%  Some useful constants. They have category codes~1, 2, 3, 4, 6, 7,
%  8, 10, 11, 12, and 13 respectively. They are all implicit tokens.
%  \end{variable}
%
% \begin{function}{
%     \token_if_group_begin_p:N / (EXP) |
%     \token_if_group_begin:N / (TF) (EXP) |
% }
%   \begin{syntax}
%     "\token_if_group_begin:NTF" <token> \Arg{true} \Arg{false}
%   \end{syntax}
%   Check if <token> is a begin group token.
% \end{function}
%
% \begin{function}{
%     \token_if_group_end_p:N / (EXP) |
%     \token_if_group_end:N / (TF) (EXP) |
% }
%   \begin{syntax}
%     "\token_if_group_end:NTF" <token> \Arg{true} \Arg{false}
%   \end{syntax}
%   Check if <token> is an end group token.
% \end{function}
%
% \begin{function}{
%     \token_if_math_shift_p:N / (EXP) |
%     \token_if_math_shift:N / (TF) (EXP) |
% }
%   \begin{syntax}
%     "\token_if_math_shift:NTF" <token> \Arg{true} \Arg{false}
%   \end{syntax}
%   Check if <token> is a math shift token.
% \end{function}
%
%
% \begin{function}{
%     \token_if_alignment_tab_p:N / (EXP) |
%     \token_if_alignment_tab:N / (TF) (EXP) |
% }
%   \begin{syntax}
%     "\token_if_aligment_tab:NTF" <token> \Arg{true} \Arg{false}
%   \end{syntax}
%   Check if <token> is an aligment tab token.
% \end{function}
%
% \begin{function}{
%     \token_if_parameter_p:N / (EXP) |
%     \token_if_parameter:N / (TF) (EXP) |
% }
%   \begin{syntax}
%     "\token_if_parameter:NTF" <token> \Arg{true} \Arg{false}
%   \end{syntax}
%   Check if <token> is a parameter token.
% \end{function}
%
% \begin{function}{
%     \token_if_math_superscript_p:N / (EXP) |
%     \token_if_math_superscript:N / (TF) (EXP) |
% }
%   \begin{syntax}
%     "\token_if_math_superscript:NTF" <token> \Arg{true} \Arg{false}
%   \end{syntax}
%   Check if <token> is a math superscript token.
% \end{function}
%
% \begin{function}{
%     \token_if_math_subscript_p:N / (EXP) |
%     \token_if_math_subscript:N / (TF) (EXP) |
% }
%   \begin{syntax}
%     "\token_if_math_subscript:NTF" <token> \Arg{true} \Arg{false}
%   \end{syntax}
%   Check if <token> is a math subscript token.
% \end{function}
%
% \begin{function}{
%     \token_if_space_p:N / (EXP) |
%     \token_if_space:N / (TF) (EXP) |
% }
%   \begin{syntax}
%     "\token_if_space:NTF" <token> \Arg{true} \Arg{false}
%   \end{syntax}
%   Check if <token> is a space token.
% \end{function}
%
%
% \begin{function}{
%     \token_if_letter_p:N / (EXP) |
%     \token_if_letter:N / (TF) (EXP) |
% }
%   \begin{syntax}
%     "\token_if_letter:NTF" <token> \Arg{true} \Arg{false}
%   \end{syntax}
%   Check if <token> is a letter token.
% \end{function}
%
%
% \begin{function}{
%     \token_if_other_char_p:N / (EXP) |
%     \token_if_other_char:N / (TF) (EXP) |
% }
%   \begin{syntax}
%     "\token_if_other_char:NTF" <token> \Arg{true} \Arg{false}
%   \end{syntax}
%   Check if <token> is an other char token.
% \end{function}
%
% \begin{function}{
%     \token_if_active_char_p:N / (EXP) |
%     \token_if_active_char:N / (TF) (EXP) |
% }
%   \begin{syntax}
%     "\token_if_active_char:NTF" <token> \Arg{true} \Arg{false}
%   \end{syntax}
%   Check if <token> is an active char token.
% \end{function}
%
%  \begin{function}{%
%                   \token_if_eq_meaning_p:NN / (EXP) |
%                   \token_if_eq_meaning:NN / (TF) (EXP)
%  }
%  \begin{syntax}
%   "\token_if_eq_meaning:NNTF" <token1> <token2>\Arg{true} \Arg{false}
%  \end{syntax}
%  Check if the meaning of two tokens are identical.
%  \end{function}
%
%  \begin{function}{%
%                   \token_if_eq_catcode_p:NN / (EXP) |
%                   \token_if_eq_catcode:NN / (TF) (EXP)
%  }
%  \begin{syntax}
%   "\token_if_eq_catcode:NNTF" <token1> <token2>\Arg{true} \Arg{false}
%  \end{syntax}
%  Check if the category codes of two tokens are equal. If both tokens
%  are control sequences the test will be true.
%  \end{function}
%
%  \begin{function}{%
%                   \token_if_eq_charcode_p:NN / (EXP) |
%                   \token_if_eq_charcode:NN / (TF) (EXP)
%  }
%  \begin{syntax}
%   "\token_if_eq_catcode:NNTF" <token1> <token2>\Arg{true} \Arg{false}
%  \end{syntax}
%  Check if the character codes of two tokens are equal. If both tokens
%  are control sequences the test will be true.
%  \end{function}
%
% \begin{function}{
%     \token_if_macro_p:N / (EXP) |
%     \token_if_macro:N / (TF) (EXP) |
% }
%   \begin{syntax}
%     "\token_if_macro:NTF" <token> \Arg{true} \Arg{false}
%   \end{syntax}
%   Check if <token> is a macro.
% \end{function}
%
% \begin{function}{
%     \token_if_cs_p:N / (EXP) |
%     \token_if_cs:N / (TF) (EXP) |
% }
%   \begin{syntax}
%     "\token_if_cs:NTF" <token> \Arg{true} \Arg{false}
%   \end{syntax}
%   Check if <token> is a control sequence or not. This can be useful
%   for situations where the next token in the input stream is being
%   looked at and you want to determine what should be done to it.
% \end{function}
%
% \begin{function}{
%     \token_if_expandable_p:N / (EXP) |
%     \token_if_expandable:N / (TF) (EXP)
% }
%   \begin{syntax}
%     "\token_if_expandable:NTF" <token> \Arg{true} \Arg{false}
%   \end{syntax}
%   Check if <token> is expandable or not. Note that <token> can very
%   well be an active character.
% \end{function}
%
% The next set of functions here are for picking apart control
% sequences. Sometimes it is useful to know if a control sequence has
% arguments and if so, how many. Similarly its status with respect to
% "\long" or "\protected" is good to have. Finally it can be very
% useful to know if a control sequence is of a certain type: Is this
% \m{toks} register we're trying to to something with really a
% \m{toks} register at all? 
%
% \begin{function}{
%     \token_if_long_macro_p:N / (EXP) |
%     \token_if_long_macro:N / (TF) (EXP) |
% }
%   \begin{syntax}
%     "\token_if_long_macro:NTF" <token> \Arg{true} \Arg{false}
%   \end{syntax}
%   Check if <token> is a ``long'' macro.
% \end{function}
%
%
% \begin{function}{
%     \token_if_protected_macro_p:N / (EXP) |
%     \token_if_protected_macro:N / (TF) (EXP) |
% }
%   \begin{syntax}
%     "\token_if_long_macro:NTF" <token> \Arg{true} \Arg{false}
%   \end{syntax}
%   Check if <token> is a ``protected'' macro. This test does \emph{not}
%   return <true> if the macro is also ``long'', see below.
% \end{function}
%
% \begin{function}{
%     \token_if_protected_long_macro_p:N / (EXP) |
%     \token_if_protected_long_macro:N / (TF) (EXP) |
% }
%   \begin{syntax}
%     "\token_if_protected_long_macro:NTF" <token> \Arg{true} \Arg{false}
%   \end{syntax}
%   Check if <token> is a ``protected long'' macro.
% \end{function}
%
% \begin{function}{
%     \token_if_chardef_p:N / (EXP) |
%     \token_if_chardef:N / (TF) (EXP) |
% }
%   \begin{syntax}
%     "\token_if_chardef:NTF" <token> \Arg{true} \Arg{false}
%   \end{syntax}
%   Check if <token> is defined to be a chardef.
% \end{function}
%
% \begin{function}{
%     \token_if_mathchardef_p:N / (EXP) |
%     \token_if_mathchardef:N / (TF) (EXP) |
% }
%   \begin{syntax}
%     "\token_if_mathchardef:NTF" <token> \Arg{true} \Arg{false}
%   \end{syntax}
%   Check if <token> is defined to be a mathchardef.
% \end{function}
%
% \begin{function}{
%     \token_if_int_register_p:N / (EXP) |
%     \token_if_int_register:N / (TF) (EXP) |
% }
%   \begin{syntax}
%     "\token_if_int_register:NTF" <token> \Arg{true} \Arg{false}
%   \end{syntax}
%   Check if <token> is defined to be an integer register.
% \end{function}
%
% \begin{function}{
%     \token_if_dim_register_p:N / (EXP) |
%     \token_if_dim_register:N / (TF) (EXP) |
% }
%   \begin{syntax}
%     "\token_if_dim_register:NTF" <token> \Arg{true} \Arg{false}
%   \end{syntax}
%   Check if <token> is defined to be a dimension register.
% \end{function}
%
% \begin{function}{
%     \token_if_skip_register_p:N / (EXP) |
%     \token_if_skip_register:N / (TF) (EXP) |
% }
%   \begin{syntax}
%     "\token_if_skip_register:NTF" <token> \Arg{true} \Arg{false}
%   \end{syntax}
%   Check if <token> is defined to be a skip register.
% \end{function}
%
% \begin{function}{
%     \token_if_toks_register_p:N / (EXP) |
%     \token_if_toks_register:N / (TF) (EXP) |
% }
%   \begin{syntax}
%     "\token_if_toks_register:NTF" <token> \Arg{true} \Arg{false}
%   \end{syntax}
%   Check if <token> is defined to be a toks register.
% \end{function}
%
%
% \begin{function}{
%     \token_get_prefix_spec:N / (EXP) |
%     \token_get_arg_spec:N    / (EXP) |
%     \token_get_replacement_spec:N / (EXP)
% }
%   \begin{syntax}
%     "\token_get_arg_spec:N" <token> 
%   \end{syntax}
%   If token is a macro with definition |\cs_set:Npn||\next|
%   |#1#2{x`#1--#2'y}|, the |prefix| function will return the string
%   |\long|, the |arg| function returns the string |#1#2| and the
%   |replacement| function returns the string |x`#1--#2'y|. If <token>
%   isn't a macro, these functions return the |\scan_stop:| token.
%
%   If the |arg_spec| contains the string |->|, then the |spec| function
%   will produce incorrect results.
% \end{function}
% 
% \subsubsection{Useless code: because we can!}
%
% \begin{function}{
%     \token_if_primitive_p:N / (EXP) |
%     \token_if_primitive:N / (TF) (EXP) |
% }
%   \begin{syntax}
%     "\token_if_primitive:NTF" <token> \Arg{true} \Arg{false}
%   \end{syntax}
%   Check if <token> is a primitive. Probably not a very useful function.
% \end{function}
%
%
%  \subsection{Peeking ahead at the next token}
%
%  \begin{variable}{%
%                   \l_peek_token |
%                   \g_peek_token |
%                   \l_peek_search_token |
%  }
%  \begin{syntax}
%  \end{syntax}
%  Some useful variables. Initially they are set to "?".
%  \end{variable}
%
%
%  \begin{function}{%
%                   \peek_after:NN |
%                   \peek_gafter:NN |
%  }
%  \begin{syntax}
%     "\peek_after:NN" <function><token>
%  \end{syntax}
%  Assign <token> to "\l_peek_token" and then run <function> which
%  should perform some sort of test on this token. Leaves <token> in
%  the input stream. "\peek_gafter:NN" does this globally to the token
%  "\g_peek_token".
%  \begin{texnote}
%  This is the primitive \tn{futurelet} turned into a function.
%  \end{texnote}
%  \end{function}
%
%  \begin{function}{%
%                   \peek_meaning:N / (TF)  |
%                   \peek_meaning_ignore_spaces:N / (TF) |
%                   \peek_meaning_remove:N / (TF) |
%                   \peek_meaning_remove_ignore_spaces:N / (TF)
%  }
%  \begin{syntax}
%    "\peek_meaning:NTF" <token> \Arg{true} \Arg{false}
%  \end{syntax}
%  "\peek_meaning:NTF" checks (by using "\if_meaning:w") if <token>
%  equals the next token in the input stream and executes either <true
%  code> or <false code> accordingly. "\peek_meaning_remove:NTF" does
%  the same but additionally removes the token if found. The
%  "ignore_spaces" versions skips blank spaces before making the
%  decision.
%  \begin{texnote}
%  This is equivalent to \LaTeXe's \tn{@ifnextchar}.
%  \end{texnote}
%  \end{function}
%
%  \begin{function}{%
%                   \peek_charcode:N / (TF) |
%                   \peek_charcode_ignore_spaces:N / (TF) |
%                   \peek_charcode_remove:N / (TF) |
%                   \peek_charcode_remove_ignore_spaces:N / (TF)
%  }
%  \begin{syntax}
%    "\peek_charcode:NTF" <token> \Arg{true} \Arg{false}
%  \end{syntax}
%  Same as for the "\peek_meaning:NTF" functions above but these use
%  "\if_charcode:w" to compare the tokens.
%  \end{function}
%
%  \begin{function}{%
%                   \peek_catcode:N / (TF) |
%                   \peek_catcode_ignore_spaces:N / (TF) |
%                   \peek_catcode_remove:N / (TF) |
%                   \peek_catcode_remove_ignore_spaces:N / (TF)
%  }
%  \begin{syntax}
%    "\peek_catcode:NTF" <token> \Arg{true} \Arg{false}
%  \end{syntax}
%  Same as for the "\peek_meaning:NTF" functions above but these use
%  "\if_catcode:w" to compare the tokens.
%  \end{function}
%
%
%  \begin{function}{%
%                   \peek_token_generic:NN / (TF) |
%                   \peek_token_remove_generic:NN / (TF)
%  }
%  \begin{syntax}
%     "\peek_token_generic:NNTF" <token><function> \Arg{true} \Arg{false}
%  \end{syntax}
%  "\peek_token_generic:NNTF" looks ahead and checks if the next token
%  in the input stream is equal to <token>. It uses <function> to make
%  that decision. "\peek_token_remove_generic:NNTF" does the same
%  thing but additionally removes <token> from the input stream if it
%  is found. This also works if <token> is either
%  "\c_group_begin_token" or "\c_group_end_token".
%  \end{function}
%
%  \begin{function}{%
%                   \peek_execute_branches_meaning: |
%                   \peek_execute_branches_charcode: |
%                   \peek_execute_branches_catcode: |
%  }
%  \begin{syntax}
%     "\peek_execute_branches_meaning:" 
%  \end{syntax}
%  These functions compare the token we are searching for with the
%  token found (after optional ignoring of specific tokens). They come
%  in the usual three versions when \TeX{} is comparing tokens:
%  meaning, character code, and category code.
%  \end{function}
%
% \StopEventually{}
%
% \subsection{Documentation of internal functions}
%
% \begin{variable}{%
%                  \l_peek_true_tlp |
%                  \l_peek_false_tlp |
% }
% \begin{syntax}
% \end{syntax}
% These token list pointers are used internally when choosing either
% the true or false branches of a test.
% \end{variable}
% 
% \begin{variable}{ \l_peek_search_tlp }
% \begin{syntax}
% \end{syntax}
% Used to store "\l_peek_search_token".
% \end{variable}
% 
% \begin{function}{%
%                  \peek_tmp:w |
% }
% \begin{syntax}
% \end{syntax}
% Scratch function used to gobble tokens from the input stream.
% \end{function}
% 
% \begin{variable}{%
%                  \l_peek_true_aux_tlp |
%                  \c_peek_true_remove_next_tlp |
% }
% \begin{syntax}
% \end{syntax}
% These token list pointers are used internally when choosing either
% the true or false branches of a test.
% \end{variable}
% 
% \begin{function}{%
%                  \peek_ignore_spaces_execute_branches: |
%                  \peek_ignore_spaces_aux: |
% }
% \begin{syntax}
% \end{syntax}
% Functions used to ignore space tokens in the input stream.
% \end{function}
%
%
% \subsection{Implementation}
%
% First a few required packages to get this going.
%    \begin{macrocode}
%<package>\ProvidesExplPackage
%<package>  {\filename}{\filedate}{\fileversion}{\filedescription}
%<package>\RequirePackage{l3prg}
%<package>\RequirePackage{l3int}
%<*initex|package>
%    \end{macrocode}
%
% \subsubsection{Character tokens}
%
%    \begin{macro}{ \char_set_catcode:w        ,
%                   \char_set_catcode:nn       ,
%                   \char_value_catcode:w      ,
%                   \char_value_catcode:n      ,
%                   \char_show_value_catcode:w ,
%                   \char_show_value_catcode:n }  
%    \begin{macrocode}
\cs_new_eq:NN \char_set_catcode:w \tex_catcode:D
\cs_new_nopar:Npn \char_set_catcode:nn #1#2 {
  \char_set_catcode:w #1 = \int_eval:n {#2}
}
\cs_new_nopar:Npn \char_value_catcode:w { \int_use:N \tex_catcode:D }
\cs_new_nopar:Npn \char_value_catcode:n #1 {
  \char_value_catcode:w \int_eval:n {#1}
}
\cs_new_nopar:Npn \char_show_value_catcode:w {
  \tex_showthe:D \tex_catcode:D
}
\cs_new_nopar:Npn \char_show_value_catcode:n #1 {
  \char_show_value_catcode:w \int_eval:n {#1}
}
%    \end{macrocode}
%    \end{macro}
%
% \begin{macro}{ \char_make_escape:N         , \char_make_begin_group:N      ,
%                \char_make_end_group:N      , \char_make_math_shift:N       ,
%                \char_make_alignment:N      , \char_make_end_line:N         ,
%                \char_make_parameter:N      , \char_make_math_superscript:N ,
%                \char_make_math_subscript:N , \char_make_ignore:N           ,
%                \char_make_space:N          , \char_make_letter:N           ,
%                \char_make_other:N          , \char_make_active:N           ,
%                \char_make_comment:N        , \char_make_invalid:N          }
%    \begin{macrocode}
\cs_new_nopar:Npn \char_make_escape:N           #1 { \char_set_catcode:nn {`#1} {\c_zero}   }
\cs_new_nopar:Npn \char_make_begin_group:N      #1 { \char_set_catcode:nn {`#1} {\c_one}    }
\cs_new_nopar:Npn \char_make_end_group:N        #1 { \char_set_catcode:nn {`#1} {\c_two}    }
\cs_new_nopar:Npn \char_make_math_shift:N       #1 { \char_set_catcode:nn {`#1} {\c_three}  }
\cs_new_nopar:Npn \char_make_alignment:N        #1 { \char_set_catcode:nn {`#1} {\c_four}   }
\cs_new_nopar:Npn \char_make_end_line:N         #1 { \char_set_catcode:nn {`#1} {\c_five}   }
\cs_new_nopar:Npn \char_make_parameter:N        #1 { \char_set_catcode:nn {`#1} {\c_six}    }
\cs_new_nopar:Npn \char_make_math_superscript:N #1 { \char_set_catcode:nn {`#1} {\c_seven}  }
\cs_new_nopar:Npn \char_make_math_subscript:N   #1 { \char_set_catcode:nn {`#1} {\c_eight}  }
\cs_new_nopar:Npn \char_make_ignore:N           #1 { \char_set_catcode:nn {`#1} {\c_nine}   }
\cs_new_nopar:Npn \char_make_space:N            #1 { \char_set_catcode:nn {`#1} {\c_ten}    }
\cs_new_nopar:Npn \char_make_letter:N           #1 { \char_set_catcode:nn {`#1} {\c_eleven}   }
\cs_new_nopar:Npn \char_make_other:N            #1 { \char_set_catcode:nn {`#1} {\c_twelve}   }
\cs_new_nopar:Npn \char_make_active:N           #1 { \char_set_catcode:nn {`#1} {\c_thirteen} }
\cs_new_nopar:Npn \char_make_comment:N          #1 { \char_set_catcode:nn {`#1} {\c_fourteen} }
\cs_new_nopar:Npn \char_make_invalid:N          #1 { \char_set_catcode:nn {`#1} {\c_fifteen}  }
%    \end{macrocode}
% \end{macro}                                                 
%
% \begin{macro}{ \char_make_escape:n         , \char_make_begin_group:n      ,
%                \char_make_end_group:n      , \char_make_math_shift:n       ,
%                \char_make_alignment:n      , \char_make_end_line:n         ,
%                \char_make_parameter:n      , \char_make_math_superscript:n ,
%                \char_make_math_subscript:n , \char_make_ignore:n           ,
%                \char_make_space:n          , \char_make_letter:n           ,
%                \char_make_other:n          , \char_make_active:n           ,
%                \char_make_comment:n        , \char_make_invalid:n          }
%    \begin{macrocode}
\cs_new_nopar:Npn \char_make_escape:n           #1 { \char_set_catcode:nn {#1} {\c_zero}   }
\cs_new_nopar:Npn \char_make_begin_group:n      #1 { \char_set_catcode:nn {#1} {\c_one}    }
\cs_new_nopar:Npn \char_make_end_group:n        #1 { \char_set_catcode:nn {#1} {\c_two}    }
\cs_new_nopar:Npn \char_make_math_shift:n       #1 { \char_set_catcode:nn {#1} {\c_three}  }
\cs_new_nopar:Npn \char_make_alignment:n        #1 { \char_set_catcode:nn {#1} {\c_four}   }
\cs_new_nopar:Npn \char_make_end_line:n         #1 { \char_set_catcode:nn {#1} {\c_five}   }
\cs_new_nopar:Npn \char_make_parameter:n        #1 { \char_set_catcode:nn {#1} {\c_six}    }
\cs_new_nopar:Npn \char_make_math_superscript:n #1 { \char_set_catcode:nn {#1} {\c_seven}  }
\cs_new_nopar:Npn \char_make_math_subscript:n   #1 { \char_set_catcode:nn {#1} {\c_eight}  }
\cs_new_nopar:Npn \char_make_ignore:n           #1 { \char_set_catcode:nn {#1} {\c_nine}   }
\cs_new_nopar:Npn \char_make_space:n            #1 { \char_set_catcode:nn {#1} {\c_ten}    }
\cs_new_nopar:Npn \char_make_letter:n           #1 { \char_set_catcode:nn {#1} {\c_eleven}   }
\cs_new_nopar:Npn \char_make_other:n            #1 { \char_set_catcode:nn {#1} {\c_twelve}   }
\cs_new_nopar:Npn \char_make_active:n           #1 { \char_set_catcode:nn {#1} {\c_thirteen} }
\cs_new_nopar:Npn \char_make_comment:n          #1 { \char_set_catcode:nn {#1} {\c_fourteen} }
\cs_new_nopar:Npn \char_make_invalid:n          #1 { \char_set_catcode:nn {#1} {\c_fifteen}  }
%    \end{macrocode}
% \end{macro}
%
%    \begin{macro}{\char_set_mathcode:w,
%                  \char_set_mathcode:nn,
%                  \char_gset_mathcode:w,
%                  \char_gset_mathcode:nn,
%                  \char_value_mathcode:w,
%                  \char_value_mathcode:n,
%                  \char_show_value_mathcode:w,
%                  \char_show_value_mathcode:n}  
% Math codes.
%    \begin{macrocode}
\cs_new_eq:NN \char_set_mathcode:w \tex_mathcode:D
\cs_new_nopar:Npn \char_set_mathcode:nn #1#2 {
  \char_set_mathcode:w #1 = \int_eval:n {#2}
}
\cs_new_protected_nopar:Npn \char_gset_mathcode:w { \pref_global:D \tex_mathcode:D }
\cs_new_nopar:Npn \char_gset_mathcode:nn #1#2 {
  \char_gset_mathcode:w #1 = \int_eval:n {#2}
}
\cs_new_nopar:Npn \char_value_mathcode:w { \int_use:N \tex_mathcode:D }
\cs_new_nopar:Npn \char_value_mathcode:n #1 {
  \char_value_mathcode:w \int_eval:n {#1}
}
\cs_new_nopar:Npn \char_show_value_mathcode:w { \tex_showthe:D \tex_mathcode:D }
\cs_new_nopar:Npn \char_show_value_mathcode:n #1 {
  \char_show_value_mathcode:w \int_eval:n {#1}
}
%    \end{macrocode}
%    \end{macro}
%
%    \begin{macro}{\char_set_lccode:w,        \char_set_lccode:nn,
%                  \char_value_lccode:w,      \char_value_lccode:n,
%                  \char_show_value_lccode:w, \char_show_value_lccode:n}  
%    \begin{macrocode}
\cs_new_eq:NN \char_set_lccode:w \tex_lccode:D
\cs_new_nopar:Npn \char_set_lccode:nn #1#2{
  \char_set_lccode:w #1 = \int_eval:n{#2}
}
\cs_new_nopar:Npn \char_value_lccode:w {\int_use:N\tex_lccode:D}
\cs_new_nopar:Npn \char_value_lccode:n #1{\char_value_lccode:w \int_eval:n{#1}}
\cs_new_nopar:Npn \char_show_value_lccode:w {\tex_showthe:D\tex_lccode:D}
\cs_new_nopar:Npn \char_show_value_lccode:n #1{
  \char_show_value_lccode:w \int_eval:n{#1}}
%    \end{macrocode}
%    \end{macro}
%
%    \begin{macro}{\char_set_uccode:w,        \char_set_uccode:nn,
%                  \char_value_uccode:w,      \char_value_uccode:n,
%                  \char_show_value_uccode:w, \char_show_value_uccode:n }  
%    \begin{macrocode}
\cs_new_eq:NN \char_set_uccode:w \tex_uccode:D
\cs_new_nopar:Npn \char_set_uccode:nn #1#2{
  \char_set_uccode:w #1 = \int_eval:n{#2}
}
\cs_new_nopar:Npn \char_value_uccode:w {\int_use:N\tex_uccode:D}
\cs_new_nopar:Npn \char_value_uccode:n #1{\char_value_uccode:w \int_eval:n{#1}}
\cs_new_nopar:Npn \char_show_value_uccode:w {\tex_showthe:D\tex_uccode:D}
\cs_new_nopar:Npn \char_show_value_uccode:n #1{
  \char_show_value_uccode:w \int_eval:n{#1}}
%    \end{macrocode}
%    \end{macro}
%
%    \begin{macro}{\char_set_sfcode:w,        \char_set_sfcode:nn,
%                  \char_value_sfcode:w,      \char_value_sfcode:n,
%                  \char_show_value_sfcode:w, \char_show_value_sfcode:n}  
%    \begin{macrocode}
\cs_new_eq:NN \char_set_sfcode:w \tex_sfcode:D
\cs_new_nopar:Npn \char_set_sfcode:nn #1#2 {
  \char_set_sfcode:w #1 = \int_eval:n {#2}
}
\cs_new_nopar:Npn \char_value_sfcode:w { \int_use:N \tex_sfcode:D }
\cs_new_nopar:Npn \char_value_sfcode:n #1 {\char_value_sfcode:w \int_eval:n {#1} }
\cs_new_nopar:Npn \char_show_value_sfcode:w { \tex_showthe:D \tex_sfcode:D }
\cs_new_nopar:Npn \char_show_value_sfcode:n #1 {
  \char_show_value_sfcode:w \int_eval:n{#1}
}
%    \end{macrocode}
%    \end{macro}
%
% \subsubsection{Generic tokens}
%
%  \begin{macro}{\token_new:Nn}
%  Creates a new token. (Will: why can't this just be 
%  "\cs_new_eq:NN \token_new:Nn \cs_gnew_eq:NN"? Seriously, that doesn't work!)
%    \begin{macrocode}
\cs_new_nopar:Npn \token_new:Nn #1#2 {\cs_gnew_eq:NN #1#2}
%    \end{macrocode}
%  \end{macro}
%
%  \begin{macro}{\c_group_begin_token,
%                \c_group_end_token,
%                \c_math_shift_token,
%                \c_alignment_tab_token,
%                \c_parameter_token,
%                \c_math_superscript_token,
%                \c_math_subscript_token,
%                \c_space_token,
%                \c_letter_token,
%                \c_other_char_token,
%                \c_active_char_token }
%  We define these useful tokens. We have to do it by hand with the
%  brace tokens for obvious reasons.
%    \begin{macrocode}
\cs_new_eq:NN \c_group_begin_token {
\cs_new_eq:NN \c_group_end_token }
\group_begin:
\char_set_catcode:nn{`\*}{3}
\token_new:Nn \c_math_shift_token {*}
\char_set_catcode:nn{`\*}{4}
\token_new:Nn \c_alignment_tab_token {*}
\token_new:Nn \c_parameter_token {#}
\token_new:Nn \c_math_superscript_token {^}
\char_set_catcode:nn{`\*}{8}
\token_new:Nn \c_math_subscript_token {*}
\token_new:Nn \c_space_token {~}
\token_new:Nn \c_letter_token {a}
\token_new:Nn \c_other_char_token {1}
\char_set_catcode:nn{`\*}{13}
\cs_gset_nopar:Npn \c_active_char_token {\exp_not:N*}
\group_end:
%    \end{macrocode}
%  \end{macro}
%
% \begin{macro}{\token_if_group_begin_p:N}
% \begin{macro}[TF]{\token_if_group_begin:N}
%   Check if token is a begin group token. We use the constant
%   "\c_group_begin_token" for this.
%    \begin{macrocode}
\cs_new:Npn \token_if_group_begin_p:N #1{
  \if_catcode:w \exp_not:N #1\c_group_begin_token
    \c_true_bool
  \else:
    \c_false_bool
  \fi:
}
\def_long_test_function_new:npn {token_if_group_begin:N} #1{
  \if_bool:N\token_if_group_begin_p:N #1}
%    \end{macrocode}
% \end{macro}
% \end{macro}
%
% \begin{macro}{ \token_if_group_end_p:N }
% \begin{macro}[TF]{ \token_if_group_end:N }
%   Check if token is a end group token. We use the constant
%   "\c_group_end_token" for this.
%    \begin{macrocode}
\cs_new:Npn \token_if_group_end_p:N #1{
  \if_catcode:w \exp_not:N #1\c_group_end_token
    \c_true_bool
  \else:
    \c_false_bool
  \fi:
}
\def_long_test_function_new:npn {token_if_group_end:N} #1{
  \if_bool:N\token_if_group_end_p:N #1}
%    \end{macrocode}
% \end{macro}
% \end{macro}
%
% \begin{macro}{\token_if_math_shift_p:N}
% \begin{macro}[TF]{\token_if_math_shift:N}
%   Check if token is a math shift token. We use the constant
%   "\c_math_shift_token" for this.
%    \begin{macrocode}
\cs_new:Npn \token_if_math_shift_p:N #1{
  \if_catcode:w \exp_not:N #1\c_math_shift_token
    \c_true_bool
  \else:
    \c_false_bool
  \fi:
}
\def_long_test_function_new:npn {token_if_math_shift:N} #1{
  \if_bool:N\token_if_math_shift_p:N#1}
%    \end{macrocode}
% \end{macro}
% \end{macro}
%
% \begin{macro}{\token_if_alignment_tab_p:N}
% \begin{macro}[TF]{\token_if_alignment_tab:N}
%   Check if token is an alignment tab token. We use the constant
%   "\c_alignment_tab_token" for this.
%    \begin{macrocode}
\cs_new:Npn \token_if_alignment_tab_p:N #1{
  \if_catcode:w \exp_not:N #1\c_alignment_tab_token
    \c_true_bool
  \else:
    \c_false_bool
  \fi:
}
\def_long_test_function_new:npn {token_if_alignment_tab:N} #1{
  \if_bool:N\token_if_alignment_tab_p:N#1}
%    \end{macrocode}
% \end{macro}
% \end{macro}
%
% \begin{macro}{\token_if_parameter_p:N}
% \begin{macro}[TF]{\token_if_parameter:N}
%   Check if token is a parameter token. We use the constant
%   "\c_parameter_token" for this. We have to trick \TeX{} a bit to
%   avoid an error message.
%    \begin{macrocode}
\cs_new:Npn \token_if_parameter_p:N #1{
  \exp_after:wN\if_catcode:w \cs:w c_parameter_token\cs_end:\exp_not:N #1
    \c_true_bool
  \else:
    \c_false_bool
  \fi:
}
\def_long_test_function_new:npn {token_if_parameter:N} #1{
  \if_bool:N\token_if_parameter_p:N#1}
%    \end{macrocode}
% \end{macro}
% \end{macro}
%
% \begin{macro}{\token_if_math_superscript_p:N}
% \begin{macro}[TF]{\token_if_math_superscript:N}
%   Check if token is a math superscript token. We use the constant
%   "\c_math_superscript_token" for this.
%    \begin{macrocode}
\cs_new:Npn \token_if_math_superscript_p:N #1{
  \if_catcode:w \exp_not:N #1\c_math_superscript_token
    \c_true_bool
  \else:
    \c_false_bool
  \fi:
}
\def_long_test_function_new:npn {token_if_math_superscript:N} #1{
  \if_bool:N\token_if_math_superscript_p:N #1}
%    \end{macrocode}
% \end{macro}
% \end{macro}
%
% \begin{macro}{\token_if_math_subscript_p:N}
% \begin{macro}[TF]{\token_if_math_subscript:N}
%   Check if token is a math subscript token. We use the constant
%   "\c_math_subscript_token" for this.
%    \begin{macrocode}
\cs_new:Npn \token_if_math_subscript_p:N #1{
  \if_catcode:w \exp_not:N #1\c_math_subscript_token
    \c_true_bool
  \else:
    \c_false_bool
  \fi:
}
\def_long_test_function_new:npn {token_if_math_subscript:N} #1{
  \if_bool:N\token_if_math_subscript_p:N #1}
%    \end{macrocode}
% \end{macro}
% \end{macro}
%
% \begin{macro}{\token_if_space_p:N}
% \begin{macro}[TF]{\token_if_space:N}
%   Check if token is a space token. We use the constant
%   "\c_space_token" for this.
%    \begin{macrocode}
\cs_new:Npn \token_if_space_p:N #1{
  \if_catcode:w \exp_not:N #1\c_space_token
    \c_true_bool
  \else:
    \c_false_bool
  \fi:
}
\def_long_test_function_new:npn {token_if_space:N} #1{
  \if_bool:N\token_if_space_p:N #1}
%    \end{macrocode}
% \end{macro}
% \end{macro}
%
% \begin{macro}{\token_if_letter_p:N}
% \begin{macro}[TF]{\token_if_letter:N}
%   Check if token is a letter token. We use the constant
%   "\c_letter_token" for this.
%    \begin{macrocode}
\cs_new:Npn \token_if_letter_p:N #1{
  \if_catcode:w \exp_not:N #1\c_letter_token
    \c_true_bool
  \else:
    \c_false_bool
  \fi:
}
\def_long_test_function_new:npn {token_if_letter:N} #1{
  \if_bool:N\token_if_letter_p:N #1}
%    \end{macrocode}
% \end{macro}
% \end{macro}
%
% \begin{macro}{\token_if_other_char_p:N}
% \begin{macro}[TF]{\token_if_other_char:N}
%   Check if token is an other char token. We use the constant
%   "\c_other_char_token" for this.
%    \begin{macrocode}
\cs_new:Npn \token_if_other_char_p:N #1{
  \if_catcode:w \exp_not:N #1\c_other_char_token
    \c_true_bool
  \else:
    \c_false_bool
  \fi:
}
\def_long_test_function_new:npn {token_if_other_char:N} #1{
  \if_bool:N\token_if_other_char_p:N #1}
%    \end{macrocode}
% \end{macro}
% \end{macro}
%
% \begin{macro}{\token_if_active_char_p:N}
% \begin{macro}[TF]{\token_if_active_char:N}
%   Check if token is an active char token. We use the constant
%   "\c_active_char_token" for this.
%    \begin{macrocode}
\cs_new:Npn \token_if_active_char_p:N #1{
  \if_catcode:w \exp_not:N #1\c_active_char_token
    \c_true_bool
  \else:
    \c_false_bool
  \fi:
}
\def_long_test_function_new:npn {token_if_active_char:N} #1{
  \if_bool:N\token_if_active_char_p:N #1}
%    \end{macrocode}
% \end{macro}
% \end{macro}
%
%  \begin{macro}{\token_if_eq_meaning_p:NN}
%  \begin{macro}[TF]{\token_if_eq_meaning:NN}
%  Check if the tokens |#1| and |#2| have same meaning.
%    \begin{macrocode}
\cs_new_nopar:Npn \token_if_eq_meaning_p:NN #1#2 {
  \if_meaning:w  #1  #2
    \c_true_bool
  \else:
    \c_false_bool
  \fi:
}
\def_test_function_new:npn {token_if_eq_meaning:NN}#1#2{
  \if_meaning:w  #1  #2}
%    \end{macrocode}
% \end{macro}
% \end{macro}
%
% \begin{macro}{\token_if_eq_catcode_p:NN}
% \begin{macro}[TF]{\token_if_eq_catcode:NN}
%  Check if the tokens |#1| and |#2| have same category code.
%    \begin{macrocode}
\cs_new_nopar:Npn \token_if_eq_catcode_p:NN #1#2 {
  \if_catcode:w \exp_not:N #1 \exp_not:N #2
    \c_true_bool
  \else:
    \c_false_bool
  \fi:
}
\def_test_function_new:npn {token_if_eq_catcode:NN}#1#2{
  \if_bool:N\token_if_eq_catcode_p:NN#1#2}
%    \end{macrocode}
% \end{macro}
% \end{macro}
%
%
%
% \begin{macro}{\token_if_eq_charcode_p:NN}
% \begin{macro}[TF]{\token_if_eq_charcode:NN}
%  Check if the tokens |#1| and |#2| have same character code.
%    \begin{macrocode}
\cs_new_nopar:Npn \token_if_eq_charcode_p:NN #1#2 {
  \if_charcode:w \exp_not:N #1 \exp_not:N #2
    \c_true_bool
  \else:
    \c_false_bool
  \fi:
}
\def_test_function_new:npn {token_if_eq_charcode:NN}#1#2{
  \if_bool:N\token_if_eq_charcode_p:NN#1#2}
%    \end{macrocode}
% \end{macro}
% \end{macro}
%
% \begin{macro}{\token_if_macro_p:N}
% \begin{macro}[TF]{\token_if_macro:N}
% \begin{macro}[aux]{\token_if_macro_p_aux:w}
%   When a token is a macro, "\token_to_meaning:N" will always output
%   something like "\long macro:#1->#1" so we simply check to see if
%   the meaning contains "->". Argument "#2" in the code below will be
%   empty if the string "->" isn't present, proof that the token was
%   not a macro (which is why we reverse the emptiness test). However
%   this function will fail on its own auxiliary function (and a few
%   other private functions as well) but that should certainly never
%   be a problem!
%    \begin{macrocode}
\cs_new_nopar:Npn \token_if_macro_p:N #1 {
  \exp_after:wN \token_if_macro_p_aux:w \token_to_meaning:N #1 -> \q_nil
}
\cs_new_nopar:Npn \token_if_macro_p_aux:w #1 -> #2 \q_nil{
  \if_bool:N \tlist_if_empty_p:n{#2} \c_false_bool \else: \c_true_bool \fi:
}
\def_test_function_new:npn {token_if_macro:N} #1{\if_bool:N\token_if_macro_p:N#1}
%    \end{macrocode}
%  \end{macro}
%  \end{macro}
%  \end{macro}
%
% \begin{macro}{\token_if_cs_p:N}
% \begin{macro}[TF]{\token_if_cs:N}
%   Check if token has same catcode as a control sequence. We use
%   "\scan_stop:" for this.
%    \begin{macrocode}
\cs_new_nopar:Npn \token_if_cs_p:N {\token_if_eq_catcode_p:NN \scan_stop:}
\def_test_function_new:npn {token_if_cs:N} #1{
  \if_bool:N \token_if_eq_catcode_p:NN \scan_stop: #1}
%    \end{macrocode}
% \end{macro}
% \end{macro}
%
%
% \begin{macro}{\token_if_expandable_p:N}
% \begin{macro}[TF]{\token_if_expandable:N}
%   Check if token is expandable. We use the fact that \TeX\ will
%   temporarily convert "\exp_not:N" \m{token} into "\scan_stop:" if
%   \m{token} is expandable.
%    \begin{macrocode}
\cs_set_nopar:Npn \token_if_expandable_p:N #1{
  \cs_if_exist:NTF #1 {
    \exp_after:wN \if_meaning:w \exp_not:N #1 #1
      \c_false_bool 
    \else: 
      \c_true_bool 
    \fi:
  } {
    \c_false_bool
  }
}
\def_test_function:npn {token_if_expandable:N} #1 {
  \if_bool:N \token_if_expandable_p:N #1
}
%    \end{macrocode}
% \end{macro}
% \end{macro}
%
% \begin{macro}{\token_if_chardef_p:N,
%               \token_if_mathchardef_p:N,
%               \token_if_int_register_p:N,
%               \token_if_skip_register_p:N,
%               \token_if_dim_register_p:N,
%               \token_if_toks_register_p:N,
%               \token_if_protected_macro_p:N,
%               \token_if_long_macro_p:N,
%               \token_if_protected_long_macro_p:N}
% \begin{macro}[aux]{
%               \token_if_chardef_p_aux:w,
%               \token_if_mathchardef_p_aux:w,
%               \token_if_int_register_p_aux:w,
%               \token_if_skip_register_p_aux:w,
%               \token_if_dim_register_p_aux:w,
%               \token_if_toks_register_p_aux:w,
%               \token_if_protected_macro_p_aux:w,
%               \token_if_long_macro_p_aux:w,
%               \token_if_protected_long_macro_p_aux:w}
%   Most of these functions have to check the meaning of the token in
%   question so we need to do some checkups on which characters are
%   output by |\token_to_meaning:N|. As usual, these characters have
%   catcode 12 so we must do some serious substitutions in the code
%   below\dots
%    \begin{macrocode}
\group_begin:
  \char_set_lccode:nn {`\X}{`\n} 
  \char_set_lccode:nn {`\Y}{`\t}
  \char_set_lccode:nn {`\Z}{`\d}
  \char_set_lccode:nn {`\?}{`\\}
  \tlist_map_inline:nn{\X\Y\Z\M\T\C\H\A\R\O\U\S\K\I\P\L\G\P\E}
    {\char_set_catcode:nn {`#1}{12}}
%    \end{macrocode}
% We convert the token list to lowercase and restore the catcode and
% lowercase code changes.
%    \begin{macrocode}
\tlist_to_lowercase:n{
  \group_end:
%    \end{macrocode}
% First up is checking if something has been defined with
% |\tex_chardef:D| or |\tex_mathchardef:D|. This is easy since \TeX\
% thinks of such tokens as hexadecimal so it stores them as
% |\char"|\meta{hex~number} or |\mathchar"|\meta{hex~number}.
%    \begin{macrocode}
\cs_new_nopar:Npn \token_if_chardef_p:N #1 {
  \exp_after:wN \token_if_chardef_p_aux:w
  \token_to_meaning:N #1?CHAR"\q_nil
}
\cs_new_nopar:Npn \token_if_chardef_p_aux:w #1?CHAR"#2\q_nil{
  \tlist_if_empty_p:n{#1}
}
\cs_new_nopar:Npn \token_if_mathchardef_p:N #1 {
  \exp_after:wN \token_if_mathchardef_p_aux:w
  \token_to_meaning:N #1?MAYHCHAR"\q_nil
}
\cs_new_nopar:Npn \token_if_mathchardef_p_aux:w #1?MAYHCHAR"#2\q_nil{
  \tlist_if_empty_p:n{#1}
}
%    \end{macrocode}
% Integer registers are a little more difficult since they expand to
% |\count|\meta{number} and there is also a primitive |\countdef|. So
% we have to check for that primitive as well.
%    \begin{macrocode}
\cs_set_nopar:Npn \token_if_int_register_p:N #1{
  \if_meaning:w \tex_countdef:D #1 
    \c_false_bool
  \else:
    \exp_after:wN \token_if_int_register_p_aux:w
      \token_to_meaning:N #1?COUXY\q_nil
  \fi:
}
\cs_new_nopar:Npn \token_if_int_register_p_aux:w #1?COUXY#2\q_nil{
  \tlist_if_empty_p:n{#1}
}
%    \end{macrocode}
% Skip registers are done the same way as the integer registers.
%    \begin{macrocode}
\cs_set_nopar:Npn \token_if_skip_register_p:N #1{
  \if_meaning:w \tex_skipdef:D #1 
  \c_false_bool
  \else:
  \exp_after:wN \token_if_skip_register_p_aux:w
  \token_to_meaning:N #1?SKIP\q_nil
  \fi:
}
\cs_new_nopar:Npn \token_if_skip_register_p_aux:w #1?SKIP#2\q_nil{
  \tlist_if_empty_p:n{#1}
}
%    \end{macrocode}
% Dim registers. No news here
%    \begin{macrocode}
\cs_set_nopar:Npn \token_if_dim_register_p:N #1{
  \if_meaning:w \tex_dimendef:D #1 
    \c_false_bool
  \else:
    \exp_after:wN \token_if_dim_register_p_aux:w
    \token_to_meaning:N #1?ZIMEX\q_nil 
  \fi:
}
\cs_new_nopar:Npn \token_if_dim_register_p_aux:w #1?ZIMEX#2\q_nil{
  \tlist_if_empty_p:n{#1}
}
%    \end{macrocode}
% Toks registers. Ho-hum.
%    \begin{macrocode}
\cs_set_nopar:Npn \token_if_toks_register_p:N #1{
  \if_meaning:w \tex_toksdef:D #1 
    \c_false_bool
  \else:
    \exp_after:wN \token_if_toks_register_p_aux:w
    \token_to_meaning:N #1?YOKS\q_nil
  \fi:
}
\cs_new_nopar:Npn \token_if_toks_register_p_aux:w #1?YOKS#2\q_nil{
  \tlist_if_empty_p:n{#1}
}
%    \end{macrocode}
% Protected macros.
%    \begin{macrocode}
\cs_new_nopar:Npn \token_if_protected_macro_p:N #1 {
  \exp_after:wN \token_if_protected_macro_p_aux:w
  \token_to_meaning:N #1?PROYECYEZ~MACRO\q_nil
}
\cs_new_nopar:Npn \token_if_protected_macro_p_aux:w #1?PROYECYEZ~MACRO#2\q_nil{
  \tlist_if_empty_p:n{#1}
}
%    \end{macrocode}
% Long macros.
%    \begin{macrocode}
\cs_new_nopar:Npn \token_if_long_macro_p:N #1 {
  \exp_after:wN \token_if_long_macro_p_aux:w
  \token_to_meaning:N #1?LOXG~MACRO\q_nil
}
\cs_new_nopar:Npn \token_if_long_macro_p_aux:w #1?LOXG~MACRO#2\q_nil{
  \tlist_if_empty_p:n{#1}
}
%    \end{macrocode}
% Finally protected long macros where we for once don't have to add an
% extra test since there is no primitive for the combined prefixes.
%    \begin{macrocode}
\cs_new_nopar:Npn \token_if_protected_long_macro_p:N #1 {
  \exp_after:wN \token_if_protected_long_macro_p_aux:w 
  \token_to_meaning:N #1?PROYECYEZ?LOXG~MACRO\q_nil
}
\cs_new_nopar:Npn \token_if_protected_long_macro_p_aux:w #1
  ?PROYECYEZ?LOXG~MACRO#2\q_nil{
  \tlist_if_empty_p:n{#1}
}
%    \end{macrocode}
% Finally the |\tlist_to_lowercase:n| ends!
%    \begin{macrocode}
}
%    \end{macrocode}
% \end{macro}
% \end{macro}
%
% \begin{macro}[TF]{\token_if_chardef:N,\token_if_mathchardef:N,
%   \token_if_long_macro:N,             \token_if_protected_macro:N,
%   \token_if_protected_long_macro:N,   \token_if_dim_register:N,
%   \token_if_skip_register:N,          \token_if_int_register:N,
%   \token_if_toks_register:N}
%    \begin{macrocode}
\def_test_function_new:npn {token_if_chardef:N} #1 {
  \if_bool:N \token_if_chardef_p:N #1
}
\def_test_function_new:npn {token_if_mathchardef:N} #1 {
  \if_bool:N \token_if_mathchardef_p:N #1
}
\def_test_function_new:npn {token_if_long_macro:N} #1 {
  \if_bool:N \token_if_long_macro_p:N #1
}
\def_test_function_new:npn {token_if_protected_macro:N} #1 {
  \if_bool:N \token_if_protected_macro_p:N #1
}
\def_test_function_new:npn {token_if_protected_long_macro:N} #1 {
  \if_bool:N \token_if_protected_long_macro_p:N #1
}
\def_test_function_new:npn {token_if_dim_register:N} #1 {
  \if_bool:N \token_if_dim_register_p:N #1
}
\def_test_function_new:npn {token_if_skip_register:N} #1 {
  \if_bool:N \token_if_skip_register_p:N #1
}
\def_test_function_new:npn {token_if_int_register:N} #1 {
  \if_bool:N \token_if_int_register_p:N #1
}
\def_test_function_new:npn {token_if_toks_register:N} #1 {
  \if_bool:N \token_if_toks_register_p:N #1
}
%    \end{macrocode}
% \end{macro}
%
% We do not provide a function for testing if a control sequence is
% ``outer'' since we don't use that in \LaTeX3.
%
% \begin{macro}[aux]{\token_get_prefix_arg_replacement_aux:w}
% \begin{macro}{\token_get_prefix_spec:N}
% \begin{macro}{\token_get_arg_spec:N}
% \begin{macro}{\token_get_replacement_spec:N}
%   In the \textsf{xparse} package we sometimes want to test if a
%   control sequence can be expanded to reveal a hidden
%   value. However, we cannot just expand the macro blindly as it may
%   have arguments and none might be present. Therefore we define
%   these functions to pick either the prefix(es), the argument
%   specification, or the replacement text from a macro. All of this
%   information is returned as characters with catcode~12. If the
%   token in question isn't a macro, the token |\scan_stop:| is
%   returned instead.
%    \begin{macrocode}
\group_begin:
\char_set_lccode:nn {`\?}{`\:}
\char_set_catcode:nn{`\M}{12}
\char_set_catcode:nn{`\A}{12}
\char_set_catcode:nn{`\C}{12}
\char_set_catcode:nn{`\R}{12}
\char_set_catcode:nn{`\O}{12}
\tlist_to_lowercase:n{
  \group_end:
  \cs_new_nopar:Npn \token_get_prefix_arg_replacement_aux:w #1MACRO?#2->#3\q_nil#4{
    #4{#1}{#2}{#3}
  }
  \cs_new_nopar:Npn\token_get_prefix_spec:N #1{
    \token_if_macro:NTF #1{
      \exp_after:wN \token_get_prefix_arg_replacement_aux:w
      \token_to_meaning:N #1\q_nil\use_i:nnn
    }{\scan_stop:}
  }
  \cs_new_nopar:Npn\token_get_arg_spec:N #1{
    \token_if_macro:NTF #1{
      \exp_after:wN \token_get_prefix_arg_replacement_aux:w
      \token_to_meaning:N #1\q_nil\use_ii:nnn
    }{\scan_stop:}
  }
  \cs_new_nopar:Npn\token_get_replacement_spec:N #1{
    \token_if_macro:NTF #1{
      \exp_after:wN \token_get_prefix_arg_replacement_aux:w
      \token_to_meaning:N #1\q_nil\use_iii:nnn
    }{\scan_stop:}
  }
}
%    \end{macrocode}
% \end{macro}
% \end{macro}
% \end{macro}
% \end{macro}
% 
% \paragraph{Useless code: because we can!}
%
% \begin{macro}{\token_if_primitive_p:N}
% \begin{macro}[aux]{\token_if_primitive_p_aux:N}
% \begin{macro}[TF]{\token_if_primitive:N}
%   It is rather hard to determine if a token is a primitive. First we
%   can check if it is a control sequence or active character. If
%   either, we check if it is a macro. Then we can go through a
%   tedious process of testing for different register types\dots{} I
%   don't actually think this function is useful but you never know.
%    \begin{macrocode}
\cs_new_nopar:Npn \token_if_primitive_p:N #1{
  \if_bool:N \token_if_cs_p:N #1\scan_stop:
    \if_bool:N \token_if_macro_p:N #1 
      \c_false_bool 
    \else: 
      \token_if_primitive_p_aux:N #1 
    \fi:
  \else:
    \if_bool:N \token_if_active_char_p:N #1
      \if_bool:N \token_if_macro_p:N #1 
        \c_false_bool 
      \else: 
        \token_if_primitive_p_aux:N #1
      \fi:
    \else:
      \c_false_bool
    \fi:
  \fi:
}
\cs_new_nopar:Npn \token_if_primitive_p_aux:N #1{
  \if_bool:N \token_if_chardef_p:N #1 \c_false_bool
  \else:
    \if_bool:N \token_if_mathchardef_p:N #1 \c_false_bool
    \else: 
      \if_bool:N \token_if_int_register_p:N #1 \c_false_bool
      \else: 
        \if_bool:N \token_if_skip_register_p:N #1 \c_false_bool
        \else: 
          \if_bool:N \token_if_dim_register_p:N #1 \c_false_bool
          \else: 
            \if_bool:N   \token_if_toks_register_p:N #1 \c_false_bool
            \else: 
%    \end{macrocode}
% We made it!
%    \begin{macrocode}
              \c_true_bool
            \fi:
          \fi:
        \fi:
      \fi:
    \fi:
  \fi:
}
\def_test_function_new:npn {token_if_primitive:N} #1 {
  \if_bool:N\token_if_primitive_p:N #1
  }
%    \end{macrocode}
% \end{macro}
% \end{macro}
% \end{macro}
%
%
%  \subsubsection{Peeking ahead at the next token}
%
%  \begin{macro}{\l_peek_token}
%  \begin{macro}{\g_peek_token}
%  \begin{macro}{\l_peek_search_token}
%  We define some other tokens which will initially be the character
%  |?|.
%    \begin{macrocode}
\token_new:Nn \l_peek_token {?}
\token_new:Nn \g_peek_token {?}
\token_new:Nn \l_peek_search_token {?}
%    \end{macrocode}
%  \end{macro}
%  \end{macro}
%  \end{macro}
%
%
%  \begin{macro}{\peek_after:NN}
%  \begin{macro}{\peek_gafter:NN}
%  |\peek_after:NN| takes two argument where the first is a function
%  acting on |\l_peek_token| and the second is the next token in the
%  input stream which |\l_peek_token| is set equal to.
%  |\peek_gafter:NN| does the same globally to |\g_peek_token|.
%    \begin{macrocode}
\cs_new_nopar:Npn \peek_after:NN {\tex_futurelet:D \l_peek_token }
\cs_new_nopar:Npn \peek_gafter:NN {
  \pref_global:D \tex_futurelet:D \g_peek_token
}
%    \end{macrocode}
%  \end{macro}
%  \end{macro}
%
%  For normal purposes there are four main cases:
%  \begin{enumerate}
%     \item peek at the next token.
%     \item peek at the next non-space token.
%     \item peek at the next token and remove it.
%     \item peek at the next non-space token and remove it.
%  \end{enumerate}
%
%
%  The generic functions will take four arguments: The token to search
%  for, the test function to run on it and the true/false cases.
%  The general algorithm is this:
%  \begin{enumerate}
%    \item
%    Store the token to search for in |\l_peek_search_token|.
%    \item
%    In order to avoid doubling of hash marks where it seems unnatural
%    we put the \meta{true} and \meta{false} cases through an |x| type
%    expansion but using |\exp_not:n| to avoid any expansion. This has
%    the same effect as putting it through a \meta{toks} register but
%    is faster. Also put in a special alignment safe group end.
%    \item
%    Put in an alignment safe group begin.
%    \item
%    Peek ahead and call the function which will act on the next token
%    in the input stream.
%  \end{enumerate}
%
%  \begin{macro}{\l_peek_true_tlp}
%  \begin{macro}{\l_peek_false_tlp}
%  Two dedicated token list pointers that store the true and false
%  cases.
%    \begin{macrocode}
\tlp_new:Nn \l_peek_true_tlp {}
\tlp_new:Nn \l_peek_false_tlp {}
%    \end{macrocode}
%  \end{macro}
%  \end{macro}
%
%  \begin{macro}{\peek_tmp:w}
%    Scratch function used for storing the token to be removed if
%    found.
%    \begin{macrocode}
\cs_new_nopar:Npn \peek_tmp:w {}
%    \end{macrocode}
%  \end{macro}
%
%  \begin{macro}{\l_peek_search_tlp}
%    We also use this token list pointer for storing the token we want
%    to compare. This turns out to be useful.
%    \begin{macrocode}
\tlp_new:Nn \l_peek_search_tlp {}
%    \end{macrocode}
%  \end{macro}
%
%  \begin{macro}[TF]{\peek_token_generic:NN}
%  \begin{arguments}
%    \item the function to execute (obey or ignore spaces, etc.),
%    \item the special token we're looking for.
%  \end{arguments}
%    \begin{macrocode}
\cs_new:Npn \peek_token_generic:NNTF #1#2#3#4 {
  \cs_set_eq:NN \l_peek_search_token #2
  \tlp_set:Nn \l_peek_search_tlp {#2}
  \tlp_set:Nx \l_peek_true_tlp  {\exp_not:n{\group_align_safe_end: #3}}
  \tlp_set:Nx \l_peek_false_tlp {\exp_not:n{\group_align_safe_end: #4}}
  \group_align_safe_begin:
    \peek_after:NN #1
}
\cs_new:Npn \peek_token_generic:NNT #1#2#3 {
  \peek_token_generic:NNTF #1#2 {#3} {}
}
\cs_new:Npn \peek_token_generic:NNF #1#2#3 {
  \peek_token_generic:NNTF #1#2 {} {#3}
}
%    \end{macrocode}
%  \end{macro}
%
%  \begin{macro}[TF]{\peek_token_remove_generic:NN}
%    If we want to be able to remove any character from the input
%    stream we might as well do it the same way for all characters so
%    we define this as little differently from above.
%    \begin{macrocode}
\cs_new:Npn \peek_token_remove_generic:NNTF #1#2#3#4 {
  \cs_set_eq:NN \l_peek_search_token #2
  \tlp_set:Nn \l_peek_search_tlp {#2}
  \tlp_set:Nx \l_peek_true_aux_tlp { \exp_not:n{ #3 } }
  \tlp_set_eq:NN \l_peek_true_tlp \c_peek_true_remove_next_tlp
  \tlp_set:Nx \l_peek_false_tlp {\exp_not:n{\group_align_safe_end: #4}}
  \group_align_safe_begin:
    \peek_after:NN #1
}
\cs_new:Npn \peek_token_remove_generic:NNT #1#2#3 {
  \peek_token_remove_generic:NNTF #1#2 {#3} {}
}
\cs_new:Npn \peek_token_remove_generic:NNF #1#2#3 {
  \peek_token_remove_generic:NNTF #1#2 {} {#3}
}
%    \end{macrocode}
%  \end{macro}
%
%  \begin{macro}{\l_peek_true_aux_tlp}
%  \begin{macro}{\c_peek_true_remove_next_tlp}
%    Two token list pointers to help with removing the character from
%    the input stream.
%    \begin{macrocode}
\tlp_new:Nn \l_peek_true_aux_tlp {}
\tlp_new:Nn \c_peek_true_remove_next_tlp {\group_align_safe_end: 
  \tex_afterassignment:D \l_peek_true_aux_tlp \cs_set_eq:NN \peek_tmp:w 
}
%    \end{macrocode}
%  \end{macro}
%  \end{macro}
%
%  \begin{macro}{\peek_execute_branches_meaning:}
%  \begin{macro}{\peek_execute_branches_catcode:}
%  \begin{macro}{\peek_execute_branches_charcode:}
%  \begin{macro}[aux]{\peek_execute_branches_charcode_aux:NN}
%    There are three major tests between tokens in \TeX: meaning,
%    catcode and charcode. Hence we define three basic test functions
%    that set in after the ignoring phase is over and done with.
%    \begin{macrocode}
\cs_new_nopar:Npn \peek_execute_branches_meaning: {
  \if_meaning:w \l_peek_token \l_peek_search_token
    \exp_after:wN \l_peek_true_tlp
  \else:
    \exp_after:wN \l_peek_false_tlp
  \fi:
}
\cs_new_nopar:Npn \peek_execute_branches_catcode: {
  \if_catcode:w \exp_not:N\l_peek_token \exp_not:N\l_peek_search_token
    \exp_after:wN \l_peek_true_tlp
  \else:
    \exp_after:wN \l_peek_false_tlp
  \fi:
}
%    \end{macrocode}
% For the charcode version we do things a little differently. We want
% to check the token directly but if we do this we face problems if
% the next thing in the input stream is a braced group or a space
% token. The braced group would be read as a complete argument and the
% space would be gobbled by \TeX's argument reading routines. Hence we
% test for both of these and if one of them is found we just execute
% the false result directly since no one should ever try to use the
% |charcode| function for searching for |\c_group_begin_token| or
% |\c_space_token|.
%    \begin{macrocode}
\cs_new_nopar:Npn \peek_execute_branches_charcode: {
  \bool_if:nTF {
    \token_if_eq_catcode_p:NN \l_peek_token \c_group_begin_token ||
    \token_if_eq_meaning_p:NN \l_peek_token \c_space_token
  }
  { \l_peek_false_tlp  }
%    \end{macrocode}
% Otherwise we call a small auxiliary function that just grabs the
% next token. We can do that because it really is a single token; we
% just have insert it again afterwards. Also we stored the token we
% were looking for in the token list pointer |\l_peek_search_tlp| so
% we unpack it again for this function.
%    \begin{macrocode}
  { \exp_after:wN \peek_execute_branches_charcode_aux:NN \l_peek_search_tlp }
}
%    \end{macrocode}
% Then we just do the usual |\if_charcode:w| comparison. We also
% remember to insert |#2| again after executing the true or false
% branches.
%    \begin{macrocode}
\cs_new:Npn \peek_execute_branches_charcode_aux:NN #1#2{
  \if_charcode:w \exp_not:N #1\exp_not:N#2
    \exp_after:wN \l_peek_true_tlp
  \else:
    \exp_after:wN \l_peek_false_tlp
  \fi:
  #2
}
%    \end{macrocode}
%  \end{macro}
%  \end{macro}
%  \end{macro}
%  \end{macro}
%
% \begin{macro}[aux]{\peek_def_aux:nnnn,\peek_def_aux_ii:nnnnn}
% This function aids defining conditional variants without too much
% repeated code. I hope that it doesn't detract too much from the readability.
%    \begin{macrocode}
\cs_new_nopar:Npn \peek_def_aux:nnnn #1#2#3#4 {
  \peek_def_aux_ii:nnnnn {#1} {#2} {#3} {#4} { TF }
  \peek_def_aux_ii:nnnnn {#1} {#2} {#3} {#4} { T  }
  \peek_def_aux_ii:nnnnn {#1} {#2} {#3} {#4} { F  }
}
\cs_new_nopar:Npn \peek_def_aux_ii:nnnnn #1#2#3#4#5 {
  \cs_new_nopar:cpx { #1 #5 } {
    \tlist_if_empty:nF {#2} {
      \exp_not:n { \cs_set_eq:NN \peek_execute_branches: #2 }
    }
    \exp_not:o { \cs:w #3 #5 \cs_end: }
    \exp_not:n { #4 }
  }
}
%    \end{macrocode}
% \end{macro}
%
% \begin{macro}[TF]{\peek_meaning:N}
% Here we use meaning comparison with |\if_meaning:w|.
%    \begin{macrocode}
\peek_def_aux:nnnn
  { peek_meaning:N }
  {}
  { peek_token_generic:NN }
  { \peek_execute_branches_meaning: }
%    \end{macrocode}
% \end{macro}
% 
% \begin{macro}[TF]{\peek_meaning_ignore_spaces:N}
%    \begin{macrocode}
\peek_def_aux:nnnn
  { peek_meaning_ignore_spaces:N }
  { \peek_execute_branches_meaning: }
  { peek_token_generic:NN }
  { \peek_ignore_spaces_execute_branches: }
%    \end{macrocode}
% \end{macro}
% 
% \begin{macro}[TF]{\peek_meaning_remove:N}
%    \begin{macrocode}
\peek_def_aux:nnnn
  { peek_meaning_remove:N }
  {}
  { peek_token_remove_generic:NN }
  { \peek_execute_branches_meaning: }
%    \end{macrocode}
% \end{macro}
% 
% \begin{macro}[TF]{\peek_meaning_remove_ignore_spaces:N}
%    \begin{macrocode}
\peek_def_aux:nnnn
  { peek_meaning_remove_ignore_spaces:N }
  { \peek_execute_branches_meaning: }
  { peek_token_remove_generic:NN }
  { \peek_ignore_spaces_execute_branches: }
%    \end{macrocode}
% \end{macro}
%
% \begin{macro}[TF]{\peek_catcode:N}
% Here we use catcode comparison with |\if_catcode:w|.
%    \begin{macrocode}
\peek_def_aux:nnnn
  { peek_catcode:N }
  {}
  { peek_token_generic:NN }
  { \peek_execute_branches_catcode: }
%    \end{macrocode}
% \end{macro}
% 
% \begin{macro}[TF]{\peek_catcode_ignore_spaces:N}
%    \begin{macrocode}
\peek_def_aux:nnnn
  { peek_catcode_ignore_spaces:N }
  { \peek_execute_branches_catcode: }
  { peek_token_generic:NN }
  { \peek_ignore_spaces_execute_branches: }
%    \end{macrocode}
% \end{macro}
% 
% \begin{macro}[TF]{\peek_catcode_remove:N}
%    \begin{macrocode}
\peek_def_aux:nnnn
  { peek_catcode_remove:N }
  {}
  { peek_token_remove_generic:NN }
  { \peek_execute_branches_catcode: }
%    \end{macrocode}
% \end{macro}
% 
% \begin{macro}[TF]{\peek_catcode_remove_ignore_spaces:N}
%    \begin{macrocode}
\peek_def_aux:nnnn
  { peek_catcode_remove_ignore_spaces:N }
  { \peek_execute_branches_catcode: }
  { peek_token_remove_generic:NN }
  { \peek_ignore_spaces_execute_branches: }
%    \end{macrocode}
% \end{macro}
%
% \begin{macro}[TF]{\peek_charcode:N}
% Here we use charcode comparison with |\if_charcode:w|.
%    \begin{macrocode}
\peek_def_aux:nnnn
  { peek_charcode:N }
  {}
  { peek_token_generic:NN }
  { \peek_execute_branches_charcode: }
%    \end{macrocode}
% \end{macro}
% 
% \begin{macro}[TF]{\peek_charcode_ignore_spaces:N}
%    \begin{macrocode}
\peek_def_aux:nnnn
  { peek_charcode_ignore_spaces:N }
  { \peek_execute_branches_charcode: }
  { peek_token_generic:NN }
  { \peek_ignore_spaces_execute_branches: }
%    \end{macrocode}
% \end{macro}
% 
% \begin{macro}[TF]{\peek_charcode_remove:N}
%    \begin{macrocode}
\peek_def_aux:nnnn
  { peek_charcode_remove:N }
  {}
  { peek_token_remove_generic:NN }
  { \peek_execute_branches_charcode: }
%    \end{macrocode}
% \end{macro}
% 
% \begin{macro}[TF]{\peek_charcode_remove_ignore_spaces:N}
%    \begin{macrocode}
\peek_def_aux:nnnn
  { peek_charcode_remove_ignore_spaces:N }
  { \peek_execute_branches_charcode: }
  { peek_token_remove_generic:NN }
  { \peek_ignore_spaces_execute_branches:} 
%    \end{macrocode}
% \end{macro}
%
% \begin{macro}{\peek_ignore_spaces_aux:,
%               \peek_ignore_spaces_execute_branches:}
%   Throw away a space token and search again. We could define this in
%   a more devious way where the auxiliary function gobbles the space
%   token but then what do we do if we decide that a certain function
%   should ignore more than one specific token? For example someone
%   might find it interesting to define a |\peek_| function that
%   ignores |a|'s and |b|'s! Or maybe different kinds of ``funny
%   spaces''\dots{} Therefore I have decided to use this version which
%   uses |\tex_afterassignment:D| to call the auxiliary function after
%   the next token has been removed by |\cs_set_eq:NN|. That way it is
%   easily extensible.
% \begin{macrocode}
\cs_new_nopar:Npn \peek_ignore_spaces_aux: {
  \peek_after:NN \peek_ignore_spaces_execute_branches:
}
\cs_new_nopar:Npn \peek_ignore_spaces_execute_branches: {
  \token_if_eq_meaning:NNTF \l_peek_token \c_space_token
  {  \tex_afterassignment:D \peek_ignore_spaces_aux: 
     \cs_set_eq:NN \peek_tmp:w
  } 
  \peek_execute_branches:
}
%    \end{macrocode}
% \end{macro}
%
%    \begin{macrocode}
%</initex|package>
%    \end{macrocode}
% 
%    \begin{macrocode}
%<*showmemory>
\showMemUsage
%</showmemory>
%    \end{macrocode}
%
% \Finale
% \PrintIndex
%
% \endinput
