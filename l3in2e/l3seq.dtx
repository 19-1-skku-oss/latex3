% \iffalse
%% File: l3seq.dtx Copyright (C) 1990-1998 LaTeX3 project
%
%<*dtx>
          \ProvidesFile{l3seq.dtx}
%</dtx>
%<package>\NeedsTeXFormat{LaTeX2e}
%<package>\ProvidesPackage{l3seq}
%<driver> \ProvidesFile{l3seq.drv}
% \fi
%         \ProvidesFile{l3seq.dtx}
          [1998/04/12 v1.0b L3 Experimental Token List Pointers]
%
% \iffalse
%<*driver>
\documentclass{l3doc}

\begin{document}
\DocInput{l3seq.dtx}
\end{document}
%</driver>
% \fi
%
%<package>\RequirePackage{l3expan}
%
% \GetFileInfo{l3seq.dtx}
% \title{The \textsf{l3seq} package\thanks{This file
%         has version number \fileversion, last
%         revised \filedate.}\\
% Sequences}
% \author{\Team}
% \date{\filedate}
% \maketitle
%
%
%
% \section{Sequence}
% 
% \LaTeX3 implements a data type called `sequences'. These are special
% token lists that can be accessed via special function on the `left'.
% Appending tokens is possible at both ends. Appended token lists can be
% accessed only as a union.  The token lists that form the individual
% items of a sequence might contain any tokens except two internal
% functions that are used to structure sequences (see section internal
% functions below).  It is also possible to map functions on such
% sequences so that they are executed for every item on the sequence.
% 
% All functions that return items from a sequence in some <tlp> assume
% that the <tlp> is local. See remarks below if you need a global
% returned value.
% 
% The defined functions are not orthogonal in the sense that every
% possible variation possible is actually available. If you need a new
% variant use the expansion functions described in the package
% \texttt{l3expan} to build it.
% 
% Adding items to the left of a sequence can currently be done with
% either something like "\seq_put_left:Nn" or with a ``stack'' function
% like "\seq_push:Nn" which has the same effect. Maybe one should
% therefore remove the ``left'' functions totally.
% 
% \subsection{Functions}
% 
% \begin{function}{%
%                  \seq_new:N |
%                  \seq_new:O |
%                  \seq_new:c |
% }
% \begin{syntax}
%    "\seq_new:N" <sequence>
% \end{syntax}
% Defines <sequence> to be a variable of type sequences.
% \end{function}
% 
% \begin{function}{%
%                  \seq_clear:N |
%                  \seq_clear:c |
%                  \seq_gclear:N |
%                  \seq_gclear:c |
% }
% \begin{syntax}
%   "\seq_clear:N"  <sequence>
% \end{syntax}
% These functions locally or globally clear <sequence>.
% \end{function}
% 
% \begin{function}{%
%                  \seq_put_left:Nn |
%                  \seq_put_left:No |
%                  \seq_put_left:Nx |
%                  \seq_put_left:cn |
%                  \seq_put_right:Nn |
%                  \seq_put_right:No |
%                  \seq_put_right:Nx |
% }
% \begin{syntax}
%   "\seq_put_left:Nn" <sequence> <token list>
% \end{syntax}
% Locally  appends <token list> as a single item to the left
% or right of <sequence>. <token list> might get expanded before
% appending.
% \end{function}
% 
% \begin{function}{%
%                  \seq_gput_left:Nn |
%                  \seq_gput_right:Nn |
%                  \seq_gput_right:No |
%                  \seq_gput_right:cn |
%                  \seq_gput_right:co |
%                  \seq_gput_right:cc |
% }
% \begin{syntax}
%   "\seq_gput_left:Nn" <sequence> <token list>
% \end{syntax}
% Globally appends <token list> as a single item to the left
% or right of <sequence>.
% \end{function}
% 
% \begin{function}{%
%                  \seq_get:NN |
%                  \seq_get:cN |
% }
% \begin{syntax}
%    "\seq_get:NN" <sequence> <tlp>
% \end{syntax}
% Functions that locally assign the left-most item of <sequence> to the
% token list pointer <tlp>. Item is not removed from <sequence>! If you
% need a global return value you need to code something like this:
% \begin{quote}
%   "\seq_get:NN" <sequence> "\l_tmpa_tlp" \\
%   "\tlp_gset_eq:NN" <global tlp> "\l_tmpa_tlp"
% \end{quote}
% But if this kind of construction is used often enough a separate
% function should be provided.
% \end{function}
% 
% \begin{function}{%
%                  \seq_gconcat:NNN |
%                  \seq_gconcat:ccc |
% }
% \begin{syntax}
%   "\seq_gconcat:NNN" <seq1> <seq2> <seq3>
% \end{syntax}
% Function that conatenates <seq2> and <seq3> and globally assigns the
% result to <seq1>.
% \end{function}
% 
% \begin{function}{\seq_map:NN}
% \begin{syntax}
%    "\seq_map:NN" <sequences> <function>
% \end{syntax}
% This function applies <function> (which must be a function with one
% argument) to every item of <sequence>. <function> is not executed
% within a sub-group so that side effects can be achieved locally. The
% operation is not expandable which means that it can't be used within
% write operations etc.
% 
% In the current implementation the next functions are more efficient and
% should be preferred.
% \end{function}
% 
% \begin{function}{\seq_map_inline:Nn |
%                  \seq_map_inline:cn
% }
% \begin{syntax}
%   "\seq_map_inline:Nn" <sequence> "{" <inline function> "}"
% \end{syntax}
% Applies <inline function> (which should be the direct coding for a
% function with one argument (i.e.\ use "##1" as the place holder for
% this argument)) to every item of <sequence>.  <inline function> is not
% executed within a sub-group so that side effects can be achieved locally.
% The operation is not expandable which means that it can't be used
% within write operations etc.
% \end{function}
% 
% \subsection{Predicates and conditionals}
% 
% \begin{function}{\seq_empty_p:N}
% \begin{syntax}
%   "\seq_empty_p:N" <sequence>
% \end{syntax}
% This predicate returns `true' if <sequence> is `empty' i.e., doesn't
% contain any tokens.
% \end{function}
% 
% \begin{function}{%
%                  \seq_empty:NTF |
%                  \seq_empty:cTF |
%                  \seq_empty:NF |
%                  \seq_empty:cF |
% }
% \begin{syntax}
%   "\seq_empty:NTF" <sequence> "{" <true code> "}{" <false code> "}"
% \end{syntax}
% Set of conditionals that test whether or not a particular <sequence>
% is empty and if so executes either <true code> or <false code>.
% \end{function}
% 
% \begin{function}{%
%                  \seq_if_in:NnTF |
%                  \seq_if_in:cnTF |
%                  \seq_if_in:coTF |
% }
% \begin{syntax}
%   "\seq_if_in:NnTF" <sequ> "{" <item> "}{" <true code> "}{" <false code> "}"
% \end{syntax}
% Function that tests if <item> is in <sequ>. Depending on the result
% either <true code> or <false code> is executed.
% \end{function}
% 
% \subsection{Internal functions}
% 
% \begin{function}{\seq_empty_err:N}
% \begin{syntax}
%   "\seq_empty_err:N" <sequence>
% \end{syntax}
% Signals an \LaTeX3 error if <sequence> is empty.
% \end{function}
% 
% \begin{function}{\seq_pop_aux:nnNN}
% \begin{syntax}
%   "\seq_pop_aux:nnNN" <assign1> <assign2> <sequence> <tlp>
% \end{syntax}
% Function that assigns the left-most item of <sequence> to <tlp> using
% <assign1> and assigns the tail to <sequence> using <assign2>. This
% function could be used to implement a global return function.
% \end{function}
% 
% \begin{function}{%
%                  \seq_get_aux:w |
%                  \seq_pop_aux:w |
%                  \seq_put_aux:Nnn |
%                  \seq_put_aux:w |
% }
% Functions used to implement put and get operations. They  are not for
% meant for direct use.
% \end{function}
% 
% \begin{function}{%
%                  \seq_elt:w |
%                   \seq_elt_end: }
% Functions (usually used as constants) that separates items within a
% sequence. They might get special meaning during mapping operations and
% are not supposed to show up as tokens within an item appended to a
% sequence.
% \end{function}
% 
%
% \subsection {Sequences Implementation}
%
% A sequence is a control sequence whose top-level expansion is of the
% form `|\seq_elt:w| \m{text$\sb1$} |\seq_elt_end:| \ldots{}
% |\seq_elt:w| \zv{text$\sb{n}$} \ldots'. We use explicit delimiters
% instead of braces around \m{text} to allow efficient searching for
% an item in the sequence.
%
% \begin{macro}{\seq_elt:w}
% \begin{macro}{\seq_elt_end:}
%    We allocate the delimiters and make them noops as default.
%    \begin{macrocode}
\let_new:NN \seq_elt:w \use_noop:
\let_new:NN \seq_elt_end: \use_noop:
%    \end{macrocode}
% \end{macro}
% \end{macro}
%
%
% \begin{macro}{\seq_new:N}
% \begin{macro}{\seq_new:O}
% \begin{macro}{\seq_new:c}
%    Sequences are implemented using token lists.
%    \begin{macrocode}
\def_new:Npn \seq_new:N #1{\tlp_new:Nn #1{}}
\def_new:Npn \seq_new:c {\exp_args:Nc \seq_new:N}
\def_new:Npn \seq_new:O {\exp_args:No \seq_new:N}
%    \end{macrocode}
% \end{macro}
% \end{macro}
% \end{macro}
%
% \begin{macro}{\seq_clear:N}
% \begin{macro}{\seq_clear:c}
% \begin{macro}{\seq_gclear:N}
% \begin{macro}{\seq_gclear:c}
%    Clearing a sequence is the same as clearing a token list.
%    \begin{macrocode}
\let_new:NN \seq_clear:N \tlp_clear:N
\let_new:NN \seq_clear:c \tlp_clear:c
\let_new:NN \seq_gclear:N \tlp_gclear:N
\let_new:NN \seq_gclear:c \tlp_gclear:c
%    \end{macrocode}
% \end{macro}
% \end{macro}
% \end{macro}
% \end{macro}
%
% \begin{macro}{\seq_clear_new:N}
% \begin{macro}{\seq_clear_new:c}
% \begin{macro}{\seq_gclear_new:N}
% \begin{macro}{\seq_gclear_new:c}
%    Clearing a sequence is the same as clearing a token list.
%    \begin{macrocode}
\let_new:NN \seq_clear_new:N \tlp_clear_new:N
\let_new:NN \seq_clear_new:c \tlp_clear_new:c
\let_new:NN \seq_gclear_new:N \tlp_gclear_new:N
\let_new:NN \seq_gclear_new:c \tlp_gclear_new:c
%    \end{macrocode}
% \end{macro}
% \end{macro}
% \end{macro}
% \end{macro}
%
% \begin{macro}{\seq_empty_p:N}
%    A predicate which evaluates to |\c_true| iff the sequence is empty.
%    \begin{macrocode}
\let_new:NN \seq_empty_p:N \tlp_empty_p:N
%    \end{macrocode}
% \end{macro}
%
% \begin{macro}{\seq_empty:NTF}
% \begin{macro}{\seq_empty:cTF}
% \begin{macro}{\seq_empty:NF}
% \begin{macro}{\seq_empty:cF}
%    |\seq_empty:NTF|\m{seq}\m{true~case}\m{false~case} will check
%    whether the \m{seq} is empty and then select one of the other
%    arguments. |seq_empty:cTF| turns its first argument into a 
%    control sequence to get the name of the sequence.
%    \begin{macrocode}
\def_new:Npn \seq_empty:NTF #1{
   \if_meaning:NN#1\c_empty_tlp
     \exp_after:NN\use_choice_i:nn
   \else: \exp_after:NN\use_choice_ii:nn  \fi:}
\def_new:Npn \seq_empty:cTF {\exp_args:Nc\seq_empty:NTF}
%    \end{macrocode}
%    A variant of this, is only to do something if the sequence is
%    {\em not} empty.
%    \begin{macrocode}
\def_new:Npn \seq_empty:NF #1{
   \if_meaning:NN#1\c_empty_tlp \exp_after:NN\use_none:n
   \else: \exp_after:NN\use:n \fi:}
\def_new:Npn \seq_empty:cF {\exp_args:Nc\seq_empty:NF}
%    \end{macrocode}
% \end{macro}
% \end{macro}
% \end{macro}
% \end{macro}
%
%
% \begin{macro}{\seq_empty_err:N}
%   Signals an error if the sequence is empty.
%    \begin{macrocode}
\def_new:Npn \seq_empty_err:N #1{\if_meaning:NN#1\c_empty_tlp
%    \end{macrocode}
%    As I said before, I don't think we need to provide checks for this
%    kind of error, since it is a severe internal macro package error
%    that can not be produced by the user directly. Can it? So the
%    next line of code should be probably removed.
%    \begin{macrocode}
  \tlp_clear:N \l_testa_tlp % catch prefixes
  \err_latex_bug:n{Empty~sequence~`\token_to_string:N#1'}\fi:}
%    \end{macrocode}
% \end{macro}
%
% \begin{macro}{\seq_get:NN}
% \begin{macro}{\seq_get:cN}
%    |\seq_get:NN |\zz{sequence}\zz{cmd} defines \zz{cmd} to be the
%    left_most element of \zz{sequence}.
%    \begin{macrocode}
\def_new:Npn \seq_get:NN #1{
  \seq_empty_err:N #1
  \exp_after:NN\seq_get_aux:w #1\q_stop}
\def_new:Npn \seq_get_aux:w \seq_elt:w #1\seq_elt_end:
                #2\q_stop #3{\tlp_set:Nn #3{#1}}
\def_new:Npn \seq_get:cN {\exp_args:Nc \seq_get:NN}
%    \end{macrocode}
% \end{macro}
% \end{macro}
%
% \begin{macro}{\seq_pop_aux:nnNN}
%    |\seq_pop_aux:nnNN| \zz{def$\sb1$} \zz{def$\sb2$} \zz{sequence}
%    \zz{cmd} assigns the left_most element of \zz{sequence} to
%    \zz{cmd} using \zz{def$\sb2$}, and assigns the tail of
%    \zz{sequence} to \zz{sequence} using \zz{def$\sb1$}.
%    \begin{macrocode}
\def_new:Npn \seq_pop_aux:nnNN #1#2#3{
  \seq_empty_err:N #3
  \exp_after:NN\seq_pop_aux:w #3\q_stop #1#2#3}
\def_new:Npn \seq_pop_aux:w \seq_elt:w #1\seq_elt_end:
                #2\q_stop #3#4#5#6{#3#5{#2}#4#6{#1}}
%    \end{macrocode}
% \end{macro}
%
% \begin{macro}{\seq_put_aux:Nnn}
%    |\seq_put_aux:Nnn| \zz{sequence} \zv{left} \zv{right} adds the
%    elements specified by \zv{left} to the left of \zz{sequence}, and
%    those specified by \zv{right} to the right.
%    \begin{macrocode}
\def_new:Npn \seq_put_aux:Nnn #1{
  \exp_after:NN\seq_put_aux:w #1\q_stop #1}
\def_new:Npn \seq_put_aux:w #1\q_stop #2#3#4{\tlp_set:Nn #2{#3#1#4}}
%    \end{macrocode}
% \end{macro}
%
% \begin{macro}{\seq_put_left:Nn}
% \begin{macro}{\seq_put_left:No}
% \begin{macro}{\seq_put_left:Nx}
% \begin{macro}{\seq_put_left:cn}
% \begin{macro}{\seq_put_right:Nn}
% \begin{macro}{\seq_put_right:No}
% \begin{macro}{\seq_put_right:Nx}
%    Here are the usual operations for adding to the left and right.
%    \begin{macrocode}
\def_new:Npn \seq_put_left:Nn #1#2{
%    \end{macrocode}
%    We can't put in a |\use_noop:| instead of |{}| since this argument is
%    passed literally (and we would end up with many |\use_noop:|s inside
%    the sequences.
%    \begin{macrocode}
        \seq_put_aux:Nnn #1{\seq_elt:w #2\seq_elt_end:}{}}
\def_new:Npn \seq_put_left:cn {\exp_args:Nc\seq_put_left:Nn}
\def_new:Npn \seq_put_left:No {\exp_args:NNo\seq_put_left:Nn}
\def_new:Npn \seq_put_left:Nx {\exp_args:Nnx\seq_put_left:Nn}
\def_new:Npn \seq_put_right:Nn #1#2{
        \seq_put_aux:Nnn #1{}{\seq_elt:w #2\seq_elt_end:}}
\def_new:Npn \seq_put_right:No {\exp_args:Nno\seq_put_right:Nn}
\def_new:Npn \seq_put_right:Nx {\exp_args:Nnx\seq_put_right:Nn}
%    \end{macrocode}
% \end{macro}
% \end{macro}
% \end{macro}
% \end{macro}
% \end{macro}
% \end{macro}
% \end{macro}
%
% \begin{macro}{\seq_gput_left:Nn}
% \begin{macro}{\seq_gput_right:Nn}
% \begin{macro}{\seq_gput_right:No}
% \begin{macro}{\seq_gput_right:cn}
% \begin{macro}{\seq_gput_right:co}
% \begin{macro}{\seq_gput_right:cc}
%    An here the global variants.
%    \begin{macrocode}
\def_new:Npn \seq_gput_left:Nn {
%<*check>
   \pref_global_chk:
%</check>
%<_check> \pref_global:D
   \seq_put_left:Nn}
\def_new:Npn \seq_gput_right:Nn {
%<*check>
   \pref_global_chk:
%</check>
%<_check> \pref_global:D
   \seq_put_right:Nn}
\def_new:Npn \seq_gput_right:No {\exp_args:NNo \seq_gput_right:Nn}
\def_new:Npn \seq_gput_right:cn {\exp_args:Nc  \seq_gput_right:Nn}
\def_new:Npn \seq_gput_right:co {\exp_args:Nco \seq_gput_right:Nn}
\def_new:Npn \seq_gput_right:cc {\exp_args:Ncc \seq_gput_right:Nn}
%    \end{macrocode}
% \end{macro}
% \end{macro}
% \end{macro}
% \end{macro}
% \end{macro}
% \end{macro}
%
% \begin{macro}{\seq_map:NN}
%    |\seq_map:NN| \zz{sequence} \zz{cmd} applies \zz{cmd} to each
%    element of \zz{sequence}, from left to right. Since we don't have
%    braces, this implementation is not very efficient. It might be
%    better to say that \zz{cmd} must be a function with one argument
%    that is delimited by |\seq_elt_end:|.
%    \begin{macrocode}
\def_new:Npn \seq_map:NN #1#2{
    \def:Npn \seq_elt:w ##1\seq_elt_end: {#2{##1}}#1}
%    \end{macrocode}
% \end{macro}
%
% \begin{macro}{\seq_map_inline:Nn}
% \begin{macro}{\seq_map_inline:cn}
%    When no braces are used, this version of mapping seems more
%    natural.
%    \begin{macrocode}
\def_new:Npn \seq_map_inline:Nn #1#2{
   \gdef:Npn \seq_elt:w ##1\seq_elt_end: {#2}#1}
\def_new:Npn \seq_map_inline:cn{\exp_args:Nc\seq_map_inline:Nn}
%    \end{macrocode}
% \end{macro}
% \end{macro}
%
% \begin{macro}{\seq_gconcat:NNN}
% \begin{macro}{\seq_gconcat:ccc}
%    |\seq_gconcat:NNN| \m{seq~1} \m{seq~2} \m{seq~3} will globally
%    assign \m{seq~1} the concatenation of \m{seq~2} and \m{seq~3}.
%    \begin{macrocode}
\def_new:Npn \seq_gconcat:NNN #1#2#3{
   \l_tmpa_toks \exp_after:NN{#2}
   \l_tmpb_toks \exp_after:NN{#3}
   \gdef:Npx #1{\toks_use:N \l_tmpa_toks \toks_use:N \l_tmpb_toks}}
\def_new:Npn \seq_gconcat:ccc{\exp_args:Nccc\seq_gconcat:NNN}
%    \end{macrocode}
% \end{macro}
% \end{macro}
%
% \begin{macro}{\seq_if_in:NnTF}
% \begin{macro}{\seq_if_in:cnTF}
% \begin{macro}{\seq_if_in:coTF}
%    |\seq_if_in:NnTF| \m{seq}\m{item} \m{true~case} \m{false~case}
%    will check whether \m{item} is in \m{seq} and then either execute
%    the \m{true~case} or the \m{false~case}. \m{true~case} and
%    \m{false~case} may contain incomplete |\if_char_code:w|
%    statements.
%    \begin{macrocode}
\def_new:Npn \seq_if_in:NnTF #1#2{
  \def:Npn\tmp:w
      ##1\seq_elt:w #2\seq_elt_end: ##2##3\q_stop{
%    \end{macrocode}
%    Note that |##2| contains exactly one token which we can compare
%    with |\q_no_value|.
%    \begin{macrocode}
        \if_meaning:NN\q_no_value##2
          \exp_after:NN\use_choice_ii:nn
        \else:
          \exp_after:NN\use_choice_i:nn
        \fi:
      }
  \exp_after:NN
  \tmp:w #1\seq_elt:w
  #2\seq_elt_end: \q_no_value \q_stop}
\def_new:Npn \seq_if_in:coTF {\exp_args:Nco \seq_if_in:NnTF}
\def_new:Npn \seq_if_in:cnTF {\exp_args:Nc \seq_if_in:NnTF}
%    \end{macrocode}
% \end{macro}
% \end{macro}
% \end{macro}
%
%
%
% \subsection{Stack operations}
%
% We build stacks from sequences,  but here we put the
% specific functions together.
%
%
% \begin{macro}{\seq_push:Nn}
% \begin{macro}{\seq_push:No}
% \begin{macro}{\seq_push:cn}
% \begin{macro}{\seq_pop:NN}
% \begin{macro}{\seq_pop:cN}
%    Since sequences can be used as stacks, we ought to have both
%    `push' and `pop'. In most cases they are nothing more then new
%    names for old functions.
%    \begin{macrocode}
\let_new:NN \seq_push:Nn \seq_put_left:Nn
\let_new:NN \seq_push:No \seq_put_left:No
\let_new:NN \seq_push:cn \seq_put_left:cn
\def_new:Npn \seq_pop:NN {\seq_pop_aux:nnNN \tlp_set:Nn \tlp_set:Nn}
\def_new:Npn \seq_pop:cN {\exp_args:Nc \seq_pop:NN}
%    \end{macrocode}
% \end{macro}
% \end{macro}
% \end{macro}
% \end{macro}
% \end{macro}
%
%
%
% \begin{macro}{\seq_gpush:Nn}
% \begin{macro}{\seq_gpush:No}
% \begin{macro}{\seq_gpush:cn}
% \begin{macro}{\seq_gpop:NN}
% \begin{macro}{\seq_gpop:cN}
%    I don't agree with Denys that one needs only local stacks,
%    actually I believe that one will probably need the functions
%    here more often. In case of |\seq_gpop:NN| the value is
%    nevertheless returned locally.
%    \begin{macrocode}
\let_new:NN \seq_gpush:Nn \seq_gput_left:Nn
\def_new:Npn \seq_gpush:No {\exp_args:NNo \seq_gpush:Nn}
\def_new:Npn \seq_gpush:cn {\exp_args:Nc \seq_gpush:Nn}
\def_new:Npn \seq_gpop:NN {\seq_pop_aux:nnNN \tlp_gset:Nn \tlp_set:Nn}
\def_new:Npn \seq_gpop:cN {\exp_args:Nc \seq_gpop:NN}
%    \end{macrocode}
% \end{macro}
% \end{macro}
% \end{macro}
% \end{macro}
% \end{macro}
%
%
% \begin{macro}{\seq_top:NN}
% \begin{macro}{\seq_top:cN}
%    Looking at the top element of the stack without removing it is
%    done with this operation.
%    \begin{macrocode}
\let_new:NN \seq_top:NN \seq_get:NN
\let_new:NN \seq_top:cN \seq_get:cN
%    \end{macrocode}
% \end{macro}
% \end{macro}
%
%
%    Show token usage:
%    \begin{macrocode}
%<*showmemory>
%\showMemUsage
%</showmemory>
%    \end{macrocode}
%
%
%

