% \iffalse
%% File: l3basics.dtx Copyright (C) 1990-2006 LaTeX3 project
%%
%% It may be distributed and/or modified under the conditions of the
%% LaTeX Project Public License (LPPL), either version 1.3a of this
%% license or (at your option) any later version.  The latest version
%% of this license is in the file
%%
%%    http://www.latex-project.org/lppl.txt
%%
%% This file is part of the ``expl3 bundle'' (The Work in LPPL)
%% and all files in that bundle must be distributed together.
%%
%% The released version of this bundle is available from CTAN.
%%
%% -----------------------------------------------------------------------
%%
%% The development version of the bundle can be found at
%%
%%    http://www.latex-project.org/cgi-bin/cvsweb.cgi/
%%
%% for those people who are interested.
%%
%%%%%%%%%%%
%% NOTE: %%
%%%%%%%%%%%
%%
%%   Snapshots taken from the repository represent work in progress and may
%%   not work or may contain conflicting material!  We therefore ask
%%   people _not_ to put them into distributions, archives, etc. without
%%   prior consultation with the LaTeX Project Team.
%%
%% -----------------------------------------------------------------------
%<package>\RequirePackage{l3names}
%<*dtx>
%\fi
\def\GetIdInfo$#1: #2.dtx,v #3 #4 #5 #6 #7$#8{%
  \def\fileversion{#3}%
  \def\filedate{#4}%
  \ProvidesFile{#2.dtx}[#4 v#3 #8]%
}
%\iffalse
%</dtx>
%\fi
\GetIdInfo$Id$
       {L3 Experimental basic definitions}
% \iffalse
%<*driver>
\documentclass{l3doc}

\begin{document}
\DocInput{l3basics.dtx}
\end{document}
%</driver>
% \fi
%
% \title{The \textsf{l3basics} package\thanks{This file
%         has version number \fileversion, last
%         revised \filedate.}\\
% Basic Definitions}
% \author{\Team}
% \date{\filedate}
% \maketitle
%
%
% As the name suggest this package holds some basic definitions which
% are needed by most or all other packages in this set.
%
% \section{Basics}
%
% Here we describe those functions that used all over the place. With
% that we mean functions dealing with the construction and testing of
% control sequences. Furthermore the basic parts of conditional
% processing are covered; conditional processing dealing with specific
% data types is described in the modules specific for the respective
% data types.
%
% \subsection{Predicates and conditionals}
% 
% \subsubsection{Primitive conditionals} 
%
% The \eTeX\ engine itself provides many different conditionals. Some
% expand whatever comes after them and others don't. Hence the names
% for these underlying functions will often contain a |:w| part but
% higher level functions are often available. See for instance
% |\int_compare_p:nNn| which is a wrapper for |\if_num:w|.
%
% Certain conditionals deal with specific data types like boxes and
% fonts and are described there. The ones described below are either
% the universal conditionals or deal with control sequences. We will
% prefix primitive conditionals with |\if_|.
% 
% \begin{function}{%
%                  \if_true: |
%                  \if_false: |
%                  \else: |
%                  \fi: |
%                  \reverse_if:N |
% }
% \begin{syntax}
%   "\if_true:" <true code> "\else:" <false code> "\fi:" \\
%   "\if_false:" <true code> "\else:" <false code> "\fi:" \\
%   "\reverse_if:N" <primitive conditional>
% \end{syntax}
% "\if_true:" always executes <true code>, while "\if_false:" always
% executes <false code>. "\reverse_if:N" reverses any two-way primitive
% conditional. "\else:" and "\fi:" delimit the branches of the conditional.
% \end{function}
% 
% \begin{function}{%
%                  \if_meaning:NN |
%                  \if_cs_meaning_eq:NN |
%                  \if_token_eq:NN |
% }
% \begin{syntax}
%   "\if_meaning:NN" <cs1> <cs2> <true code> "\else:" <false code>
%   "\fi:" \\
%   "\if_cs_meaning_eq:NN" <cs1> <cs2> <true code> "\else:"
%   <false code> "\fi:" \\
%   "\if_token_eq:NN" <token1> <token2> <true code> "\else:" <false
%   code> "\fi:"
% \end{syntax}
% "\if_meaning:NN" executes <true code> when the replacement text,
% i.e., the expansion of <cs1> and <cs2> are the same, otherwise it
% executes <false code>. However this name isn't really that
% good. What the \TeX\ primitive does is compare two tokens to see if
% they are equal. Hence this is actually a "token" functions. A
% similar argument applies to the situation where it is used to
% compare control sequences, where it is the meaning being compared.
% Something to be cleaned up at some point.
% \end{function}
%
% \begin{function}{%
%                  \if:w          |
%                  \if_charcode:w |
%                  \if_catcode:w  
% }
% \begin{syntax}
%   "\if:w" <token1> <token2> <true code> "\else:" <false code> "\fi:" \\
%   "\if_catcode:w" <token1> <token2> <true code> "\else:" <false
%   code> "\fi:"
% \end{syntax}
% These conditionals will expand any following tokens until two
% unexpandable tokens are left. If you wish to prevent this expansion,
% prefix the token in question with "\exp_not:N". "\if_catcode:w"
% tests if the category codes of the two tokens are the same whereas
% "\if:w" tests if the character codes are
% identical. "\if_charcode:w" is an alternative name for "\if:w".
% \end{function}
%
% \begin{function}{%
%                  \if_cs_exist:N  |
%                  \if_cs_exist:w   
% }
% \begin{syntax}
%   "\if_cs_exist:N" <cs> <true code> "\else:" <false code> "\fi:" \\
%   "\if_cs_exist:w" <tokens> "\cs_end:" <true code> "\else:" <false
%   code> "\fi:"
% \end{syntax}
% Check if <cs> appears in the hash table or if the control sequence
% that can be formed from <tokens> appears in the hash table. The
% latter function does not turn the control sequence in question into
% "\scan_stop:"! This can be useful when dealing with control
% sequences which cannot be entered as a single token.
% \end{function}
%
% \begin{function}{
%     \if_mode_horizontal: |
%     \if_mode_vertical: |
%     \if_mode_math: |
%     \if_mode_inner: 
% }
% \begin{syntax}
%   "\if_horizontal_mode:" <true code> "\else:" <false code> "\fi:"
% \end{syntax}
% Execute <true code> if currently in horizontal mode, otherwise
% execute <false code>. Similar for the other functions.
% \end{function}
%
% \subsubsection{Non-primitive conditionals}
%
% \begin{function}{%
%                  \cs_if_eq_p:NN 
% }
% \begin{syntax}
%   "\cs_if_eq_p:NN" <cs1> <cs2>
% \end{syntax}
% Returns `true' if <cs1> and <cs2> are textually the same, i.e.\ have
% the same name, otherwise it returns `false'.
% \end{function}
%
% \begin{function}{%
%                  \cs_if_eq:NNTF |
%                  \cs_if_eq:NNT |
%                  \cs_if_eq:NNF |
%                  \cs_if_eq:cNTF |
%                  \cs_if_eq:cNT |
%                  \cs_if_eq:cNF |
%                  \cs_if_eq:NcTF |
%                  \cs_if_eq:NcT |
%                  \cs_if_eq:NcF |
%                  \cs_if_eq:ccTF |
%                  \cs_if_eq:ccT |
%                  \cs_if_eq:ccF 
% }
% \begin{syntax}
%    "\cs_if_eq:NNTF" <cs1> <cs2> "{"<true code>"}{"<false code>"}"
% \end{syntax}
% These functions check if <cs1> and <cs2> have same meaning and then
% execute either <true code> or <false code>.
% \end{function}
%
%
% \begin{function}{%
%                  \cs_if_free_p:N 
% }
% \begin{syntax}
%   "\cs_if_free_p:N" <cs>
% \end{syntax}
% Returns `true' if <cs> is either undefined or equal to "\scan_stop:".
% However, it returns `false' if <cs> is textually "\c_undefined" (the
% constantly undefined function), or  textually "\scan_stop:".
% \end{function}
%
% \begin{function}{%
%                  \cs_if_free:NTF |
%                  \cs_if_free:NT |
%                  \cs_if_free:NF |
%                  \cs_if_free:cTF |
%                  \cs_if_free:cT |
%                  \cs_if_free:cF 
% }
% \begin{syntax}
%    "\cs_if_free:NTF" <cs> "{"<true code>"}{"<false code>"}"
% \end{syntax}
% These functions check if <cs> is free and then execute either <true
% code> or <false code>.
% \begin{texnote}
%   The conditional "\cs_if_free:cTF" is the \LaTeX3 implementation of
%   the \LaTeX2 function \tn{@ifundefined}. The other functions
%   haven't been around before.
% \end{texnote}
% \end{function}
%
% \begin{function}{%
%                  \cs_if_really_free:cTF |
%                  \cs_if_really_free:cF |
%                  \cs_if_really_free:cT 
% }
% \begin{syntax}
%   "\cs_if_really_free:cTF" "{"<tokens>"}" "{"<true code>"} {"<false
%     code>"}"
% \end{syntax}
% Similar to "\cs_if_free:cTF" but does not put anything previously
% undefined into the hash table. Useful for special control sequences
% like "\foo/bar" which cannot be entered as one token.
% \end{function}
%
% \begin{function}{%
%                  \cs_if_exist_p:N 
% }
% \begin{syntax}
%   "\cs_if_exist_p:N" <cs>
% \end{syntax}
% This function does the opposite of "\cs_if_free_p:N".
% \end{function}
%
% \begin{function}{%
%                  \cs_if_exist:NTF |
%                  \cs_if_exist:NT |
%                  \cs_if_exist:NF |
%                  \cs_if_exist:cTF |
%                  \cs_if_exist:cT |
%                  \cs_if_exist:cF 
% }
% \begin{syntax}
%    "\cs_if_exist:NTF" <cs> "{"<true code>"}{"<false code>"}"
% \end{syntax}
% These functions check if <cs> exists and then execute either <true
% code> or <false code>. Exactly the opposite of "\cs_if_free:NTF".
% \end{function}
%
% \begin{function}{%
%                  \cs_if_really_exist:cTF |
%                  \cs_if_really_exist:cF |
%                  \cs_if_really_exist:cT 
% }
% \begin{syntax}
%   "\cs_if_really_exist:cTF" "{"<tokens>"}" "{"<true code>"} {"<false
%     code>"}"
% \end{syntax}
% The opposite of "\cs_if_really_free:cTF".
% \end{function}
%
%
% \begin{function}{%
%                  \chk_new_cs:N |
% }
% \begin{syntax}
%   "\chk_new_cs:N" <cs>
% \end{syntax}
% This function checks that <cs> is so far either undefined or equals
% "\scan_stop:" (the function that is assigned to newly created
% control sequences by \TeX{} when "\cs:w" "..." "\cs_end:" is
% used).
% \end{function}
%
% \begin{function}{%
%                  \chk_exist_cs:N |
%                  \chk_exist_cs:c 
% }
% \begin{syntax}
%   "\chk_exist_cs:N" <cs>
% \end{syntax}
% This function checks that <cs> is defined. If it is not an error
% is generated.
% \end{function}
%
%
% \begin{variable}{%
%                  \c_true | 
%                  \c_false 
% }
% \begin{syntax}
% \end{syntax}
% Constants that represent `true' or `false', respectively. Used to
% implement predicates.
% \end{variable}
%
%
%
% \subsection{Selecting and discarding tokens from the input stream}
%
%  The conditional processing could not have been implemented without
%  being able to gobble and select which tokens to use from the input
%  stream.
%
% \begin{function}{%
%                  \use_none:n |
%                  \use_none:nn |
%                  \use_none:nnn |
%                  \use_none:nnnn |
%                  \use_none:nnnnn |
%                  \use_none:nnnnnn |
%                  \use_none:nnnnnnn |
%                  \use_none:nnnnnnnn |
%                  \use_none:nnnnnnnnn |
% }
% \begin{syntax}
%    "\use_none:n"  "{"<arg1>"}"\\
%    "\use_none:nn" "{"<arg1>"}{"<arg2>"}"
% \end{syntax}
%  These functions gobble the tokens or brace groups from the input
%  stream.
% \end{function}
%
%
% \begin{function}{\use_arg_i:n}
% \begin{syntax}
%   "\use_arg_i:n"  "{" <code1> "}"
% \end{syntax}
% Function that executes the next argument after removing the
% surrounding braces. Used to implement conditionals.
% \end{function}
%
% \begin{function}{%
%                  \use_arg_i:nn |
%                  \use_arg_ii:nn |
% }
% \begin{syntax}
%   "\use_arg_i:nn"  "{" <code1> "}{" <code2> "}"
% \end{syntax}
% Functions that execute the first or second argument respectively,
% after removing the surrounding braces. Primarily used to implement
% conditionals.
% \end{function}
%
% \begin{function}{%
%                  \use_arg_i:nnn |
%                  \use_arg_ii:nnn |
%                  \use_arg_iii:nnn |
% }
% \begin{syntax}
%   "\use_arg_i:nnn"  "{" <arg1> "}{" <arg2> "}{" <arg3> "}"
% \end{syntax}
% Functions that pick up one of three arguments and execute them after
% removing the surrounding braces. Should be described somewhere else.
% \end{function}
%
% \begin{function}{%
%                  \use_arg_i:nnnn |
%                  \use_arg_ii:nnnn |
%                  \use_arg_iii:nnnn |
%                  \use_arg_iv:nnnn |
% }
% \begin{syntax}
%   "\use_arg_i:nnnn"  "{" <arg1> "}{" <arg2> "}{" <arg3> "}{" <arg4> "}"
% \end{syntax}
% Functions that pick up one of four arguments and execute them after
% removing the surrounding braces.
% \end{function}
%
% A different kind of functions for selecting tokens from the token
% stream are those that use delimited arguments. 
%
% \begin{function}{%
%                  \use_none_delimit_by_q_nil:w |
%                  \use_none_delimit_by_q_stop:w |
% }
% \begin{syntax}
%    "\use_none_delimit_by_q_nil:w" <balanced text> "\q_nil"
% \end{syntax}
% Gobbles <balanced text>. Useful in gobbling the remainder in a list
% structure.
% \end{function}
%
% \begin{function}{%
%                  \use_arg_i_delimit_by_q_nil:nw |
%                  \use_arg_i_delimit_by_q_stop:nw |
% }
% \begin{syntax}
%    "\use_arg_i_delimit_by_q_nil:nw" "{"<arg>"}" <balanced text> "\q_nil"
% \end{syntax}
% Gobbles <balanced text> and executes <arg> afterwards. This can also
% be used to get the first item in a token list.
% \end{function}
%
%
% \begin{function}{%
%                  \use_arg_i_after_fi:nw |
%                  \use_arg_i_after_else:nw |
%                  \use_arg_i_after_or:nw
% }
% \begin{syntax}
%    "\use_arg_i_after_fi:nw" "{"<arg>"}" "\fi:"
%    "\use_arg_i_after_else:nw" "{"<arg>"}" "\else:" <balanced text> "\fi:"
%    "\use_arg_i_after_or:nw" "{"<arg>"}" "\or:" <balanced text> "\fi:"
% \end{syntax}
% Executes <arg> after executing closing out "\fi:".
% \end{function}
%
% \subsection{Internal functions}
%
% \begin{function}{%
%                  \cs:w |
%                  \cs_end: 
% }
% \begin{syntax}
%   "\cs:w" <tokens> "\cs_end:"
% \end{syntax}
% This is the \TeX{} internal way of generating a  control sequence from
% some token list. <tokens> get expanded and must ultimately result in a
% sequence of characters.
% \begin{texnote}
% These functions are the primitives \tn{csname} and \tn{endcsname}.
% "\cs:w" is considered weird because it expands tokens until it reaches
% "\cs_end:".
% \end{texnote}
% \end{function}
%
% \begin{function}{\pref_global:D |
%                  \pref_long:D |
%                  \pref_protected:D |
% }
% \begin{syntax}
%   "\pref_global:D" "\def:Npn"
% \end{syntax}
% Prefix functions that can be used in front of some definition
% functions (namely \ldots). The result of prefixing a function
% definition with "\pref_global:D" makes the definition global,
% "\pref_long:D" change the argument scanning mechanism so that it
% allows "\par" tokens in the argument of the prefixed function,
% and "\pref_protected:D" makes the definition robust in "\write"s etc.
%
%
% None of these internal functions should be used by a programmer since
% the necessary combinations are all available as separate function,
% e.g., "\def_long:Npn" is internally implemented as "\pref_long:D"
% "\def:Npn".
% \begin{texnote}
%   These prefixes are the primitives \tn{global}, \tn{long}, and
%   \tn{protected}.  The \tn{outer} isn't used at all within \LaTeX3
%   because \ldots
% \end{texnote}
% \end{function}
%
%
%
% \begin{function}{\io_put_log:x |
%                  \io_put_term:x |
%                  \io_put_deferred:Nx |
% }
% \begin{syntax}
% "\io_put_log:x" "{"<message>"}"
% "\io_put_deferred:Nx" <write_stream> "{"<message>"}"
% \end{syntax}
%  Writes <message> to either to log or the terminal.
% \end{function}
%
% \subsection{Defining functions}
%
%
% There are two types of function definitions in \LaTeX3:  versions
% that check if the function name is still unused, and versions that
% simply make the definition. The later are used for internal scratch
% functions that get new meanings all over the place.
%
% For each type there is an additional choice to be made: Does the
% function to be defined contain delimited arguments? The answer in
% 99\% of the cases is no, so in most cases the programmer just want
% to input the number of arguments, which is basically how
% \cs{newcommand} in \LaTeXe{} works. Therefore we provide functions
% that expect a number as the primary type and later on in this module
% you can find the ones with the more primitive syntax.
%
% A definition of a new function can be done locally and globally. Currently
% nearly all function definitions are done locally on top level, in
% other words they are global but don't show it. Therefore I think it may
% be better to remove the local variants in the future and declare all
% checked function definitions global.
%
% \begin{texnote}
% While \TeX{} makes all definition functions directly available to the
% user \LaTeX3 hides them very carefully to avoid the problems with
% definitions that are overwritten accidentally. Many functions that are in
% \TeX{} a combination of prefixes and definition functions are provided
% as individual functions.
% \end{texnote}
%
%
% \subsubsection{Defining new functions}
%
% Firstly comes to variants most used namely those taking a number to
% denote the number of arguments.
%
% \begin{function}{\def_new:NNn |
%                  \def_new:NNx |
%                  \def_new:cNn |
%                  \def_new:cNx
% }
% \begin{syntax}
%   "\def_new:NNn" <cs> <num> "{" <code> "}"
% \end{syntax}
% Defines a new function, making sure that <cs> is unused so far.
% <num> is the number of arguments which is in the interval $[0,9]$
% otherwise an error is raised. It is under the responsibility of the
% programmer to name the new function according to the rules laid out
% in the previous section.  <code> is either passed literally or may
% be subject to expansion (under the "x" variants).
% \end{function}
%
% \begin{function}{\gdef_new:NNn |
%                  \gdef_new:cNn |
%                  \gdef_new:NNx |
%                  \gdef_new:cNx 
% }
% \begin{syntax}
%   "\gdef_new:NNn" <cs> <num> "{" <code> "}"
% \end{syntax}
% Like "\def_new:NNn" but defines the new function globally.
% \end{function}
%
% \begin{function}{\def_long_new:NNn |
%                  \def_long_new:NNx |
%                  \def_long_new:cNn |
%                  \def_long_new:cNx
% }
% \begin{syntax}
%   "\def_long_new:NNn" <cs> <num> "{" <code> "}"
% \end{syntax}
% Defines a function that may contain "\par" tokens in the argument(s)
% when called. This is not allowed for normal functions.
% \end{function}
%
% \begin{function}{\gdef_long_new:NNn |
%                  \gdef_long_new:NNx |
%                  \gdef_long_new:cNn |
%                  \gdef_long_new:cNx
% }
% \begin{syntax}
%   "\gdef_long_new:NNn" <cs> <num> "{" <code> "}"
% \end{syntax}
% Global versions of the above functions.
% \end{function}
%
% \begin{function}{\def_protected_new:NNn |
%                  \def_protected_new:NNx |
%                  \def_protected_new:cNn |
%                  \def_protected_new:cNx
% }
% \begin{syntax}
%   "\def_protected_new:NNn" <cs> <num> "{" <code> "}"
% \end{syntax}
% Defines a function that does not expand when inside an |x| type
% expansion.
% \end{function}
%
% \begin{function}{\gdef_protected_new:NNn |
%                  \gdef_protected_new:NNx |
%                  \gdef_protected_new:cNn |
%                  \gdef_protected_new:cNx
% }
% \begin{syntax}
%   "\gdef_protected_new:NNn" <cs> <num> "{" <code> "}"
% \end{syntax}
% Global versions of the above functions.
% \end{function}
%
% \begin{function}{\def_protected_long_new:NNn |
%                  \def_protected_long_new:NNx |
%                  \def_protected_long_new:cNn |
%                  \def_protected_long_new:cNx
% }
% \begin{syntax}
%   "\def_protected_long_new:NNn" <cs> <num> "{" <code> "}"
% \end{syntax}
% Defines a function that is both robust and may contain "\par" tokens
% in the argument(s) when called.
% \end{function}
%
% \begin{function}{\gdef_protected_long_new:NNn |
%                  \gdef_protected_long_new:NNx |
%                  \gdef_protected_long_new:cNn |
%                  \gdef_protected_long_new:cNx
% }
% \begin{syntax}
%   "\gdef_protected_long_new:NNn" <cs> <num> "{" <code> "}"
% \end{syntax}
% Global versions of the above functions.
% \end{function}
%
%
%
%
%
% Secondly comes the ones where the programmer can use delimited
% arguments. Rarely needed outside the kernel.
%
%
% \begin{function}{\def_new:Npn |
%                  \def_new:Npx |
%                  \def_new:cpn |
%                  \def_new:cpx
% }
% \begin{syntax}
%   "\def_new:Npn" <cs> <parms> "{" <code> "}"
% \end{syntax}
% Defines a new function, making sure that <cs> is unused so far.
% <parms> may consist of arbitrary parameter specification in \TeX{}
% syntax. It is under the responsibility of the programmer to name the
% new function according to the rules laid out in the previous section.
% <code> is either passed literally or may be subject to expansion
% (under the "x" variants).
% \end{function}
%
% \begin{function}{\gdef_new:Npn |
%                  \gdef_new:cpn |
%                  \gdef_new:Npx |
%                  \gdef_new:cpx 
% }
% \begin{syntax}
%   "\gdef_new:Npn" <cs> <parms> "{" <code> "}"
% \end{syntax}
% Like "\def_new:Npn" but defines the new function globally. See
% comments above.
% \end{function}
%
% \begin{function}{\def_long_new:Npn |
%                  \def_long_new:Npx |
%                  \def_long_new:cpn |
%                  \def_long_new:cpx
% }
% \begin{syntax}
%   "\def_long_new:Npn" <cs> <parms> "{" <code> "}"
% \end{syntax}
% Defines a function that may contain "\par" tokens in the argument(s)
% when called. This is not allowed for normal functions.
% \end{function}
%
% \begin{function}{\gdef_long_new:Npn |
%                  \gdef_long_new:Npx |
%                  \gdef_long_new:cpn |
%                  \gdef_long_new:cpx
% }
% \begin{syntax}
%   "\gdef_long_new:Npn" <cs> <parms> "{" <code> "}"
% \end{syntax}
% Global versions of the above functions.
% \end{function}
%
% \begin{function}{\def_protected_new:Npn |
%                  \def_protected_new:Npx |
%                  \def_protected_new:cpn |
%                  \def_protected_new:cpx
% }
% \begin{syntax}
%   "\def_protected_new:Npn" <cs> <parms> "{" <code> "}"
% \end{syntax}
% Defines a function that does not expand when inside an |x| type
% expansion.
% \end{function}
%
% \begin{function}{\gdef_protected_new:Npn |
%                  \gdef_protected_new:Npx |
%                  \gdef_protected_new:cpn |
%                  \gdef_protected_new:cpx
% }
% \begin{syntax}
%   "\gdef_protected_new:Npn" <cs> <parms> "{" <code> "}"
% \end{syntax}
% Global versions of the above functions.
% \end{function}
%
% \begin{function}{\def_protected_long_new:Npn |
%                  \def_protected_long_new:Npx |
%                  \def_protected_long_new:cpn |
%                  \def_protected_long_new:cpx
% }
% \begin{syntax}
%   "\def_protected_long_new:Npn" <cs> <parms> "{" <code> "}"
% \end{syntax}
% Defines a function that is both robust and may contain "\par" tokens
% in the argument(s) when called.
% \end{function}
%
% \begin{function}{\gdef_protected_long_new:Npn |
%                  \gdef_protected_long_new:Npx |
%                  \gdef_protected_long_new:cpn |
%                  \gdef_protected_long_new:cpx
% }
% \begin{syntax}
%   "\gdef_protected_long_new:Npn" <cs> <parms> "{" <code> "}"
% \end{syntax}
% Global versions of the above functions.
% \end{function}
%
% \begin{function}{\let_new:NN |
%                  \let_new:cN |
%                  \let_new:Nc |
%                  \let_new:cc |
%                  \glet_new:NN|
%                  \glet_new:cN|
%                  \glet_new:Nc|
%                  \glet_new:cc }
% \begin{syntax}
%   "\let_new:NN" <cs1> <cs2>
% \end{syntax}
% Gives the function <cs1> the current meaning of <cs2>. Again, we may
% do this always globally.
% \end{function}
%
%
%
% \subsubsection{Undefining functions}
%
% \begin{function}{ \cs_gundefine:N
% }
% \begin{syntax}
%   "\cs_gundefine:N" <cs>
% \end{syntax}
% Undefines the control sequence.
% \end{function}
%
% \subsubsection{Defining internal functions (no checks)}
%
% Besides the function definitions that check whether or not their
% argument is an unused function we need function definitions that
% overwrite currently used definitions. The following functions are
% provided for this purpose.
%
% First comes the versions expecting a number to denote the number of
% arguments.
%
% \begin{function}{\def:NNn |
%                  \def:NNx |
%                  \def:cNn |
%                  \def:cNx |
% }
% \begin{syntax}
%   "\def:NNn" <cs> <num> "{" <code> "}"
% \end{syntax}
% Like "\def_new:NNn" etc.\ but does not check the <cs> name.
% \end{function}
%
% \begin{function}{\gdef:NNn |
%                  \gdef:NNx |
%                  \gdef:cNn |
%                  \gdef:cNx |
% }
% \begin{syntax}
%   "\gdef:NNn" <cs> <num> "{" <code> "}"
% \end{syntax}
% Like "\def:NNn" but defines the <cs> globally.
% \end{function}
%
%
% \begin{function}{\def_long:NNn |
%                  \def_long:NNx |
%                  \def_long:cNn |
%                  \def_long:cNx |
% }
% \begin{syntax}
%   "\def_long:NNn" <cs> <num> "{" <code> "}"
% \end{syntax}
% Like "\def:NNn" but allows "\par" tokens in the arguments of the
% function being defined.
% \end{function}
%
% \begin{function}{\gdef_long:NNn |
%                  \gdef_long:NNx |
%                  \gdef_long:cNn |
%                  \gdef_long:cNx |
% }
% \begin{syntax}
%   "\gdef_long:NNn" <cs> <num> "{" <code> "}"
% \end{syntax}
% Global variant of "\def_long:NNn".
% \end{function}
%
% \begin{function}{\def_protected:NNn |
%                  \def_protected:cNn |
%                  \def_protected:NNx |
%                  \def_protected:cNx |
% }
% \begin{syntax}
%   "\def_protected:NNn" <cs> <num> "{" <code> "}"
% \end{syntax}
% Naturally robust macro that won't expand in an |x| type argument.
% This also comes as a |long| version. If you for some reason want to
% expand it inside an |x| type expansion, prefix it with
% |\exp_after:NN \use_noop:|.
% \end{function}
%
% \begin{function}{\gdef_protected:NNn |
%                  \gdef_protected:cNn |
%                  \gdef_protected:NNx |
%                  \gdef_protected:cNx |
% }
% \begin{syntax}
%   "\gdef_protected:NNn" <cs> <num> "{" <code> "}"
% \end{syntax}
% Global versions of the above functions.
% \end{function}
%
% \begin{function}{\def_protected_long:NNn |
%                  \def_protected_long:cNn |
%                  \def_protected_long:NNx |
%                  \def_protected_long:cNx |
% }
% \begin{syntax}
%   "\def_protected_long:NNn" <cs> <num> "{" <code> "}"
% \end{syntax}
% Naturally robust macro that won't expand in an |x| type argument.
% These varieties also allow |\par| tokens in the arguments of the
% function being defined.
% \end{function}
%
% \begin{function}{\gdef_protected_long:NNn |
%                  \gdef_protected_long:cNn |
%                  \gdef_protected_long:NNx |
%                  \gdef_protected_long:cNx |
% }
% \begin{syntax}
%   "\gdef_protected_long:NNn" <cs> <num> "{" <code> "}"
% \end{syntax}
% Global versions of the above functions.
% \end{function}
%
% Secondly the ones that use the primitive parameter build-up:
%
% \begin{function}{\def:Npn |
%                  \def:Npx |
%                  \def:cpn |
%                  \def:cpx |
% }
% \begin{syntax}
%   "\def:Npn" <cs> <parms> "{" <code> "}"
% \end{syntax}
% Like "\def_new:Npn" etc.\ but does not check the <cs> name.
% \begin{texnote}
% "\def:Npn" is the \LaTeX3 name for \TeX{}'s \tn{def} and "\def:Npx"
% corresponds to the primitive \tn{edef}. The "\def:cpn" function was
% known in \LaTeX2 as \tn{@namedef}. "\def:cpx" has no equivalent.
% \end{texnote}
% \end{function}
%
% \begin{function}{\gdef:Npn |
%                  \gdef:Npx |
%                  \gdef:cpn |
%                  \gdef:cpx |
% }
% \begin{syntax}
%   "\gdef:Npn" <cs> <parms> "{" <code> "}"
% \end{syntax}
% Like "\def:Npn" but defines the <cs> globally.
% \begin{texnote}
% "\gdef:Npn" and "\gdef:Npx" are known to \TeX{}hackers as \tn{gdef}
% and \tn{xdef}.
% \end{texnote}
% \end{function}
%
%
% \begin{function}{\def_long:Npn |
%                  \def_long:Npx |
%                  \def_long:cpn |
%                  \def_long:cpx |
% }
% \begin{syntax}
%   "\def_long:Npn" <cs> <parms> "{" <code> "}"
% \end{syntax}
% Like "\def:Npn" but allows "\par" tokens in the arguments of the
% function being defined.
% \end{function}
%
% \begin{function}{\gdef_long:Npn |
%                  \gdef_long:Npx |
%                  \gdef_long:cpn |
%                  \gdef_long:cpx |
% }
% \begin{syntax}
%   "\gdef_long:Npn" <cs> <parms> "{" <code> "}"
% \end{syntax}
% Global variant of "\def_long:Npn".
% \end{function}
%
% \begin{function}{\def_protected:Npn |
%                  \def_protected:cpn |
%                  \def_protected:Npx |
%                  \def_protected:cpx |
% }
% \begin{syntax}
%   "\def_protected:Npn" <cs> <parms> "{" <code> "}"
% \end{syntax}
% Naturally robust macro that won't expand in an |x| type argument.
% This also comes as a |long| version. If you for some reason want to
% expand it inside an |x| type expansion, prefix it with
% |\exp_after:NN \use_noop:|.
% \end{function}
%
% \begin{function}{\gdef_protected:Npn |
%                  \gdef_protected:cpn |
%                  \gdef_protected:Npx |
%                  \gdef_protected:cpx |
% }
% \begin{syntax}
%   "\gdef_protected:Npn" <cs> <parms> "{" <code> "}"
% \end{syntax}
% Global versions of the above functions.
% \end{function}
%
% \begin{function}{\def_protected_long:Npn |
%                  \def_protected_long:cpn |
%                  \def_protected_long:Npx |
%                  \def_protected_long:cpx |
% }
% \begin{syntax}
%   "\def_protected_long:Npn" <cs> <parms> "{" <code> "}"
% \end{syntax}
% Naturally robust macro that won't expand in an |x| type argument.
% These varieties also allow |\par| tokens in the arguments of the
% function being defined.
% \end{function}
%
% \begin{function}{\gdef_protected_long:Npn |
%                  \gdef_protected_long:cpn |
%                  \gdef_protected_long:Npx |
%                  \gdef_protected_long:cpx |
% }
% \begin{syntax}
%   "\gdef_protected_long:Npn" <cs> <parms> "{" <code> "}"
% \end{syntax}
% Global versions of the above functions.
% \end{function}
%
%
%
% \begin{function}{\let:NN |
%                  \let:cN  |
%                  \let:Nc  |
%                  \let:cc  |
%                  \glet:NN |
%                  \glet:cN |
%                  \glet:Nc |
%                  \glet:cc
%                 }
% \begin{syntax}
%   "\let:cN" <cs1> <cs2>
% \end{syntax}
% Gives the function <cs1> the current meaning of <cs2>. Again, we may
% always do this globally.
% \end{function}
%
% \begin{function}{\let:NwN}
% \begin{syntax}
%   "\let:NwN"  <cs1> <cs2>
%   "\let:NwN"  <cs1> "=" <cs2>
% \end{syntax}
% These functions assign the meaning of <cs2> locally or globally to the
% function <cs1>. Because the \TeX{} primitive operation is being used
% which may have an equal sign and (a certain number of) spaces between
% <cs1> and <cs2> the name contains a "w". (Not happy about this
% convention!).
% \begin{texnote}
% "\let:NwN" is the \LaTeX3 name for \TeX{}'s \tn{let}.
% \end{texnote}
% \end{function}
%
%
% \subsection{Defining test functions}
%
%
% \begin{function}{
%     \def_test_function:npn |
%     \def_long_test_function:npn |
%     \def_test_function_new:npn |
%     \def_long_test_function_new:npn |
% }
%   \begin{syntax}
%     "\def_test_function_new:npn" "{"name"}" <parms> "{"<test>"}"
%   \end{syntax}
%   Define all the common test cases for a simple test to reduce the
%   risk of typos. As an example here's how we defined the functions
%   "\cs_free:cTF", "\cs_free:cT" and "\cs_free:cF". You just have to
%   fill in the test.
% \begin{verbatim}
% \def_test_function:npn{cs_free:c} #1 {
%   \exp_after:NN \if_meaning:NN \cs:w#1\cs_end: \scan_stop:}
% \end{verbatim}
%   Be careful not to use this function inside some primitive
%   conditional as \TeX\ will most likely get confused because of the
%   unmatched conditionals.
% \end{function}
%
%
% \subsection{The innards of a function}
%
% \begin{function}{\cs_to_str:N}
% \begin{syntax}
%   "\cs_to_str:N" <cs>
% \end{syntax}
% This function return the name of <cs> as a sequence of letters with
% the escape character removed.
% \end{function}
%
% \begin{function}{\token_to_string:N}
% \begin{syntax}
%   "\token_to_string:N" <arg>
% \end{syntax}
% This function return the name of <arg> as a sequence of letters
% including the escape character.
% \end{function}
%
% \begin{function}{\token_to_meaning:N}
% \begin{syntax}
%   "\token_to_meaning:N" <arg>
% \end{syntax}
% This function returns the type and definition of <arg> as a sequence
% of letters.
% \end{function}
%
% Other functions regarding arbitrary tokens can be found in the
% \textsf{l3token} module.
%
%  \subsection{Grouping and scanning}
%
% \begin{function}{\scan_stop:}
% \begin{syntax}
%   "\scan_stop:"
% \end{syntax}
% This function stops \TeX's scanning ahead when ending a number.
% \begin{texnote}
% This is the \TeX{} primitive \tn{relax} renamed.
% \end{texnote}
% \end{function}
%
%
% \begin{function}{\group_begin:|
%                  \group_end:}
% \begin{syntax}
%   "\group_begin:" <...> "\group_end:"
% \end{syntax}
% Encloses <...> inside a group.
% \begin{texnote}
% These are the \TeX{} primitives \tn{begingroup} and \tn{endgroup}
% renamed.
% \end{texnote}
% \end{function}
%
%
% \subsection{Engine specific definitions}
%
% \begin{function}{\engine_aleph:TF}
% \begin{syntax}
%   "\engine_aleph:TF" "{"<true code>"}" "{"<false code>"}"
% \end{syntax}
% This function detects if we're running an Aleph based format. This
% is particularly useful when allocating registers.
% \end{function}
%
%
%
% \StopEventually{}
%
%
% \subsection{The Implementation}
% We start by ensuring that the required packages are loaded.  We need
% \textsf{l3names} to get things going but we actually need it very
% early on, so it is loaded at the very top of this file. Also, most
% of the code below won't run until \textsf{l3expan} has been
% loaded.
%
% \subsubsection{Renaming some \TeX{} primitives (again)}
%
% \begin{macro}{\let:NwN}
% Having given all the tex primitives a consistent name, we need to
% give sensible names to the ones we actually want to use.
% These will be defined as needed in the appropriate modules, but
% do a few now, just to get started.\footnote{This renaming gets expensive
% in terms of csname usage, an alternative scheme would be to just use
% the ``tex\ldots D'' name in the cases where no good alternative exists.}
%    \begin{macrocode}
%<*initex|package>
\tex_let:D \let:NwN            \tex_let:D
%    \end{macrocode}
% \end{macro}
% \begin{macro}{\if_true:}
% \begin{macro}{\if_false:}
% \begin{macro}{\else:}
% \begin{macro}{\fi:}
% \begin{macro}{\reverse_if:N}
% \begin{macro}{\if:w}
% \begin{macro}{\if_charcode:w}
% \begin{macro}{\if_catcode:w}
% Then some conditionals.
%    \begin{macrocode}
\let:NwN   \if_true:           \tex_iftrue:D
\let:NwN   \if_false:          \tex_iffalse:D
\let:NwN   \else:              \tex_else:D
\let:NwN   \fi:                \tex_fi:D
\let:NwN   \reverse_if:N       \etex_unless:D
\let:NwN   \if:w               \tex_if:D
\let:NwN   \if_charcode:w     \tex_if:D
\let:NwN   \if_catcode:w      \tex_ifcat:D
%    \end{macrocode}
% \end{macro}
% \end{macro}
% \end{macro}
% \end{macro}
% \end{macro}
% \end{macro}
% \end{macro}
% \end{macro}
% \begin{macro}{\if_meaning:NN}
% \begin{macro}{\if_token_eq:NN}
% \begin{macro}{\if_cs_meaning_eq:NN}
% Some different names for |\ifx|.\footnote{MH: Clean up at some point}
%    \begin{macrocode}
\let:NwN   \if_meaning:NN      \tex_ifx:D
\let:NwN   \if_token_eq:NN     \tex_ifx:D
\let:NwN   \if_cs_meaning_eq:NN\tex_ifx:D
%    \end{macrocode}
% \end{macro}
% \end{macro}
% \end{macro}
% \begin{macro}{\if_mode_math:}
% \begin{macro}{\if_mode_horizontal:}
% \begin{macro}{\if_mode_vertical:}
% \begin{macro}{\if_mode_inner:}
% \TeX{} lets us detect some if its modes.
%    \begin{macrocode}
\let:NwN   \if_mode_math:      \tex_ifmmode:D
\let:NwN   \if_mode_horizontal:\tex_ifhmode:D
\let:NwN   \if_mode_vertical:  \tex_ifvmode:D
\let:NwN   \if_mode_inner:     \tex_ifinner:D
%    \end{macrocode}
% \end{macro}
% \end{macro}
% \end{macro}
% \end{macro}
% \begin{macro}{\if_cs_exist:N}
% \begin{macro}{\if_cs_exist:w}
%    \begin{macrocode}
\let:NwN   \if_cs_exist:N      \etex_ifdefined:D
\let:NwN   \if_cs_exist:w      \etex_ifcsname:D
%    \end{macrocode}
% \end{macro}
% \end{macro}
% 
% \begin{macro}{\io_put_deferred:Nx}
% \begin{macro}{\token_to_meaning:N}
% \begin{macro}{\token_to_string:N}
% \begin{macro}{\cs:w}
% \begin{macro}{\cs_end:}
%    \begin{macrocode}
\let:NwN   \io_put_deferred:Nx \tex_write:D
\let:NwN   \token_to_meaning:N \tex_meaning:D
\let:NwN   \token_to_string:N  \tex_string:D
\let:NwN   \cs:w               \tex_csname:D
\let:NwN   \cs_end:            \tex_endcsname:D
%    \end{macrocode}
% \end{macro}
% \end{macro}
% \end{macro}
% \end{macro}
% \end{macro}
% \begin{macro}{\exp_after:NN}
% \begin{macro}{\exp_not:N}
% \begin{macro}{\exp_not:n}
%  The three |\exp_| functions are used in the \textsf{l3expan} module
%  where they are described.
%    \begin{macrocode}
\let:NwN   \exp_after:NN       \tex_expandafter:D
\let:NwN   \exp_not:N          \tex_noexpand:D
\let:NwN   \exp_not:n          \etex_unexpanded:D
%    \end{macrocode}
% \end{macro}
% \end{macro}
% \end{macro}
% \begin{macro}{\scan_stop:}
% \begin{macro}{\group_begin:}
% \begin{macro}{\group_end:}
%  The next three are basic functions for which there also exist
%  versions that are safe inside alignments. These safe versions are
%  defined in the \textsf{l3prg} module.
%    \begin{macrocode}
\let:NwN   \scan_stop:         \tex_relax:D
\let:NwN   \group_begin:       \tex_begingroup:D
\let:NwN   \group_end:         \tex_endgroup:D
%    \end{macrocode}
% \end{macro}
% \end{macro}
% \end{macro}
% \begin{macro}{\group_execute_after:N}
%    \begin{macrocode}
\let:NwN \group_execute_after:N \tex_aftergroup:D      
%    \end{macrocode}
% \end{macro}
% \begin{macro}{\the_internal:D}
% \begin{macro}{\pref_global:D}
% \begin{macro}{\pref_long:D}
% \begin{macro}{\pref_protected:D}
%    These following names are temporary and should be removed as soon
%    as possible (April 1998).
%    \begin{macrocode}
\let:NwN   \the_internal:D     \tex_the:D
\let:NwN   \pref_global:D      \tex_global:D
%    \end{macrocode}
% "\pref_long:D" has been documented for years but didn't exist\dots{}
% Added it and the robustness prefix.
%    \begin{macrocode}
\let:NwN   \pref_long:D        \tex_long:D
\let:NwN   \pref_protected:D   \etex_protected:D
%    \end{macrocode}
% \end{macro}
% \end{macro}
% \end{macro}
% \end{macro}
%
%
%
% \subsubsection {Defining functions}
%
%
% We start by providing functions for the typical definition
% functions. First the local ones.
%
% \begin{macro}{\def:Npn}
% \begin{macro}{\def:Npx}
% \begin{macro}{\def_long:Npn}
% \begin{macro}{\def_long:Npx}
% \begin{macro}{\def_protected:Npn}
% \begin{macro}{\def_protected:Npx}
% \begin{macro}{\def_protected_long:Npn}
% \begin{macro}{\def_protected_long:Npx}
%   All assignment functions in \LaTeX3 should be naturally robust;
%   after all, the \TeX\ primitives for assignments are and it can be
%   a cause of problems if others aren't.
%    \begin{macrocode}
\let:NwN   \def:Npn            \tex_def:D
\let:NwN   \def:Npx            \tex_edef:D
\pref_protected:D \def:Npn \def_long:Npn {\pref_long:D \def:Npn}
\pref_protected:D \def:Npn \def_long:Npx {\pref_long:D \def:Npx}
\pref_protected:D \def:Npn \def_protected:Npn {\pref_protected:D \def:Npn}
\pref_protected:D \def:Npn \def_protected:Npx {\pref_protected:D \def:Npx}
\def_protected:Npn \def_protected_long:Npn {
  \pref_protected:D \pref_long:D \def:Npn
}
\def_protected:Npn \def_protected_long:Npx {
  \pref_protected:D \pref_long:D \def:Npx
}
%    \end{macrocode}
% \end{macro}
% \end{macro}
% \end{macro}
% \end{macro}
% \end{macro}
% \end{macro}
% \end{macro}
% \end{macro}
%
% \begin{macro}{\gdef:Npn}
% \begin{macro}{\gdef:Npx}
% \begin{macro}{\gdef_long:Npn}
% \begin{macro}{\gdef_long:Npx}
% \begin{macro}{\gdef_protected:Npn}
% \begin{macro}{\gdef_protected:Npx}
% \begin{macro}{\gdef_protected_long:Npn}
% \begin{macro}{\gdef_protected_long:Npx}
%   Global versions of the above functions.
%    \begin{macrocode}
\let:NwN   \gdef:Npn           \tex_gdef:D
\let:NwN   \gdef:Npx           \tex_xdef:D
\def_protected:Npn \gdef_long:Npn {\pref_long:D \gdef:Npn}
\def_protected:Npn \gdef_long:Npx {\pref_long:D \gdef:Npx}
\def_protected:Npn \gdef_protected:Npn {\pref_protected:D \gdef:Npn}
\def_protected:Npn \gdef_protected:Npx {\pref_protected:D \gdef:Npx}
\def_protected:Npn \gdef_protected_long:Npn {
  \pref_protected:D \pref_long:D \gdef:Npn
}
\def_protected:Npn \gdef_protected_long:Npx {
  \pref_protected:D \pref_long:D \gdef:Npx
}
%    \end{macrocode}
% \end{macro}
% \end{macro}
% \end{macro}
% \end{macro}
% \end{macro}
% \end{macro}
% \end{macro}
% \end{macro}
%
% \subsubsection{Predicate implementation}
%
% I think Michael originated the idea of expandable boolean tests.  At
% first these were supposed to expand into either \texttt{TT} or
% \texttt{TF} to be tested using |\if:w| but this was later changed to
% \texttt{00} and \texttt{01}, so they can be used in logical
% operations (see the \textsf{l3prg} module). We need this from the
% get-go.
%
% \begin{macro}{\c_true}
% \begin{macro}{\c_false}
%    Here are the canonical boolean values.
%    \begin{macrocode}
\def:Npn \c_true  {00}
\def:Npn \c_false {01}
%    \end{macrocode}
% \end{macro}
% \end{macro}
%
% \subsubsection {Defining and checking (new) functions}
%
% \begin{macro}{\c_minus_one}
% \begin{macro}{\c_sixteen}
%    We need the constants |\c_minus_one| and |\c_sixteen| now for
%    writing information to the log and the terminal but the
%    rest are defined in the \textsf{l3num} module -- at least for the
%    ones that can be defined with |\tex_chardef:D| or
%    |\tex_mathchardef:D|. Otherwise the \textsf{l3int} module is
%    required but it can't be used until the allocation has been set
%    up properly! The actual allocation mechanism is in
%    \textsf{l3alloc} and as \TeX{} wants to reserve count registers
%    0--9, the first available one is~10 so we use that for
%    |\c_minus_one|.
%    \begin{macrocode}
%<*!initex>
\let:NwN \c_minus_one\m@ne
%</!initex>
%<*!package>
\tex_countdef:D \c_minus_one = 10 \scan_stop:
\c_minus_one = -1 \scan_stop:
%</!package>
\tex_chardef:D \c_sixteen = 16\scan_stop:
%    \end{macrocode}
% \end{macro}
% \end{macro}
%
%    We provide two kinds of functions that can be used to define
%    control sequences. On the one hand we have functions that check
%    if their argument doesn't already exist, they are called
%    |\..._new|. The second type of defining functions doesn't check
%    if the argument is already defined.
%
%    Before we can define them, we need some auxiliary macros that
%    allow us to generate error messages. The definitions here are
%    only temporary, they will be redefined later on.
%
% \begin{macro}{\io_put_log:x}
% \begin{macro}{\io_put_term:x}
%    We define a routine to write only to the log file. And a
%    similar one for writing to both the log file and the terminal.
%
%    \begin{macrocode}
\def:Npn \io_put_log:x{
      \tex_immediate:D\io_put_deferred:Nx \c_minus_one }
\def:Npn \io_put_term:x{
      \tex_immediate:D\io_put_deferred:Nx \c_sixteen }
%    \end{macrocode}
% \end{macro}
% \end{macro}
%
% \begin{macro}{\err_latex_bug:x}
%    This will show internal errors.
%    \begin{macrocode}
\def:Npn\err_latex_bug:x#1{
   \io_put_term:x{This~is~a~LaTeX~bug!~Check~coding!}\tex_errmessage:D{#1}}
%    \end{macrocode}
% \end{macro}
%
% \begin{macro}{\cs_record_meaning:N}
%    This macro will be used later on for tracing purposes. But we
%    need some more modules to define it, so we just give some dummy
%    definition here.
%    \begin{macrocode}
%<*trace>
\def:Npn \cs_record_meaning:N#1{}
%</trace>
%    \end{macrocode}
% \end{macro}
%
% We need these two to make |\chk_new_cs:N| bulletproof.
%    \begin{macrocode}
\def_long:Npn \use_none:n  #1{}
\def_long:Npn \use_arg_i:n #1{#1}
%    \end{macrocode}
% \begin{macro}{\chk_new_cs:N}
%   This command is called by |\def_new:Npn| and |\let_new:NN| etc.\
%   to make sure that the argument sequence is not already in use. If
%   it is, an error is signalled.  It checks if \m{csname} is
%   undefined or |\scan_stop:|. Otherwise an error message is
%   issued. We have to make sure we don't put the argument into the
%   conditional processing since it may be an |\if...| type function!
%    \begin {macrocode}
\def:Npn \chk_new_cs:N #1{
  \if:w \cs_if_free_p:N #1
    \exp_after:NN \use_none:n
  \else:
    \exp_after:NN \use_arg_i:n
  \fi:
  {    
    \err_latex_bug:x {Command~name~`\token_to_string:N #1'~
                      already~defined!~
                      Current~meaning:~\token_to_meaning:N #1
                    }
  }
%<*trace>
  \cs_record_meaning:N#1
%     \io_put_term:x{Defining~\token_to_string:N #1~on~%}
  \io_put_log:x{Defining~\token_to_string:N #1~on~
                line~\tex_the:D \tex_inputlineno:D}
%</trace>
}
%    \end{macrocode}
% \end{macro}
%
%    On 2005/11/20 Morten said: I think names for testing if a certain
%    condition is true or false should always contain |if| to avoid
%    confusion. This is what we do for lots of other types of test
%    functions. For now I have defined both names for the functions
%    checking names of control sequences.
%
% \begin{macro}{\cs_if_exist_p:N}
%    Expands into |\c_true| if the control sequence given as its
%    argument \emph{is} in use.
%    \begin{macrocode}
\def:Npn \cs_if_exist_p:N #1{
  \if:w \cs_if_free_p:N #1
    \c_false
  \else:
    \c_true \fi:}
%    \end{macrocode}
% \end{macro}
%
% \begin{macro}{\chk_if_exist_cs:N }
% \begin{macro}{\chk_if_exist_cs:c }
%    This function issues a warning message when the control sequence
%    in its argument does not exist.
%    \begin{macrocode}
\def:Npn \chk_if_exist_cs:N #1 {
  \if:w \cs_if_exist_p:N #1
  \else:
    \err_latex_bug:x{Command~ `\token_to_string:N #1'~
                     not~ yet~ defined!}
  \fi:}
\def:Npn \chk_if_exist_cs:c #1 {
  \exp_after:NN \chk_if_exist_cs:N \cs:w #1\cs_end: }
%    \end{macrocode}
% \end{macro}
% \end{macro}
%
% \begin{macro}{\cs_if_free_p:N}
%    Expands into |\c_true| if the control sequence given as its
%    argument is not yet in use.  Note that we make sure to expand
%    into |\c_false| if the control sequence is textually
%    |\c_undefined| or |\scan_stop:|, so that we don't end up
%    (re)defining them.
%    \begin{macrocode}
\def:Npn \cs_if_free_p:N #1{
  \if_cs_exist:N #1
    \if_meaning:NN#1\scan_stop:
      \if:w\cs_if_eq_name_p:NN #1\scan_stop:
        \c_false \else: \c_true \fi:
    \else:
       \c_false
    \fi:
  \else:
    \if:w \cs_if_eq_name_p:NN #1\c_undefined
      \c_false \else: \c_true \fi:
  \fi:
}
\let:NwN \cs_free_p:N \cs_if_free_p:N
%    \end{macrocode}
% \end{macro}
%
%
% \begin{macro}{\str_if_eq_p:nn}
% \begin{macro}{\str_if_eq_p_aux:w}
%   Takes 2 lists of characters as arguments and expands into
%   |\c_true| if they are equal, and |\c_false| otherwise. Note that
%   in the current implementation spaces in these strings are
%   ignored.\footnote{This is a function which could use
%     \cs{tlist_compare:xx}.}
%    \begin{macrocode}
\def:Npn \str_if_eq_p:nn #1#2{
  \str_if_eq_p_aux:w #1\scan_stop:\\#2\scan_stop:\\
}
\def:Npn \str_if_eq_p_aux:w #1#2\\#3#4\\{
  \if_meaning:NN#1#3
    \if_meaning:NN#1\scan_stop:\c_true \else:
    \if_meaning:NN#3\scan_stop:\c_false \else:
    \str_if_eq_p_aux:w #2\\#4\\\fi:\fi:
  \else:\c_false \fi:}
%    \end{macrocode}
% \end{macro}
% \end{macro}
%
% \begin{macro}{\cs_if_eq_name_p:NN}
%   An application of the above function, already streamlined for
%   speed, so I put it in here.  It takes two control sequences as
%   arguments and expands into true iff they have the same name.
%   We make it long in case one of them is |\par|!
%    \begin{macrocode}
\def_long:Npn \cs_if_eq_name_p:NN #1#2{
  \exp_after:NN\exp_after:NN
  \exp_after:NN\str_if_eq_p_aux:w
  \exp_after:NN\token_to_string:N
  \exp_after:NN#1
  \exp_after:NN\scan_stop:
  \exp_after:NN\\
  \token_to_string:N#2\scan_stop:\\}
%    \end{macrocode}
% \end{macro}
%
%
% \begin{macro}{\str_if_eq_var_p:nf}
% \begin{macro}{\str_if_eq_var_start:nnN}
% \begin{macro}{\str_if_eq_var_stop:w}
%   A variant of |\str_if_eq_p:nn| which has the advantage of obeying
%   spaces in at least the second argument. See \textsf{l3quark} for
%   an application. From the hand of David Kastrup with slight
%   modifications to make it fit with the remainder of the expl3
%   language.
%
%   The macro builds a string of |\if:w \fi:| pairs from the first
%   argument. The idea is to turn the comparison of |ab| and |cde|
%   into
% \begin{verbatim}
% \tex_number:D
%   \if:w \scan_stop: \if:w b\if:w a cde\scan_stop: '\fi: \fi: \fi:
%  13
% \end{verbatim}
%   The |'| is important here. If all tests are true, the |'| is read
%   as part of the number in which case the returned number is |13| in
%   octal notation so |\tex_number:D| returns |11|. If one test
%   returns false the |'| is never seen and then we get just |13|.  We
%   wrap the whole process in an external |\if:w| in order to make it
%   return either |\c_true| or |\c_false| since some parts of
%   \textsf{l3prg} expect a predicate to return one of these two
%   tokens.
%    \begin{macrocode}
\def:Npn \str_if_eq_var_p:nf#1{
  \if:w \tex_number:D\str_if_eq_var_start:nnN{}{}#1\scan_stop:
}
\def:Npn\str_if_eq_var_start:nnN#1#2#3{
  \if:w#3\scan_stop:\exp_after:NN\str_if_eq_var_stop:w\fi:
  \str_if_eq_var_start:nnN{\if:w#3#1}{#2\fi:}
}
\def:Npn\str_if_eq_var_stop:w\str_if_eq_var_start:nnN#1#2#3{
  #1#3\scan_stop:'#213~\c_true\else:\c_false\fi:
}
%    \end{macrocode}
% \end{macro}
% \end{macro}
% \end{macro}
%
%
%
%
% \subsubsection{More new definitions}
%
%
%
% \begin{macro}{\def_new:Npn}
% \begin{macro}{\def_new:Npx}
% \begin{macro}{\def_long_new:Npn}
% \begin{macro}{\def_long_new:Npx}
% \begin{macro}{\def_protected_new:Npn}
% \begin{macro}{\def_protected_new:Npx}
% \begin{macro}{\def_protected_long_new:Npn}
% \begin{macro}{\def_protected_long_new:Npx}
%   These are like |\def:Npn| and |\let:NN|, but they first check that
%   the argument command is not already in use. You may use
%   |\pref_global:D|, |\pref_long:D|, |\pref_protected:D|, and
%   |\tex_outer:D| as prefixes.
%     \begin {macrocode}
\def_protected:Npn \def_new:Npn #1{\chk_new_cs:N #1
                         \def:Npn #1}
\def_protected:Npn \def_new:Npx #1{\chk_new_cs:N #1
                         \def:Npx #1}
\def_protected:Npn \def_long_new:Npn #1{\chk_new_cs:N #1
                                 \def_long:Npn #1}
\def_protected:Npn \def_long_new:Npx #1{\chk_new_cs:N #1
                                 \def_long:Npx #1}
\def_protected:Npn \def_protected_new:Npn #1{\chk_new_cs:N #1
                                 \def_protected:Npn #1}
\def_protected:Npn \def_protected_new:Npx #1{\chk_new_cs:N #1
                                 \def_protected:Npx #1}
\def_protected:Npn \def_protected_long_new:Npn #1{\chk_new_cs:N #1
                                 \def_protected_long:Npn #1}
\def_protected:Npn \def_protected_long_new:Npx #1{\chk_new_cs:N #1
                                 \def_protected_long:Npx #1}
%    \end{macrocode}
% \end{macro}
% \end{macro}
% \end{macro}
% \end{macro}
% \end{macro}
% \end{macro}
% \end{macro}
% \end{macro}
%
% \begin{macro}{\gdef_new:Npn}
% \begin{macro}{\gdef_new:Npx}
% \begin{macro}{\gdef_long_new:Npn}
% \begin{macro}{\gdef_long_new:Npx}
% \begin{macro}{\gdef_protected_new:Npn}
% \begin{macro}{\gdef_protected_new:Npx}
% \begin{macro}{\gdef_protected_long_new:Npn}
% \begin{macro}{\gdef_protected_long_new:Npx}
%   Global versions of the above functions.
%     \begin {macrocode}
\def_protected_new:Npn \gdef_new:Npn #1{\chk_new_cs:N #1
                         \gdef:Npn #1}
\def_protected_new:Npn \gdef_new:Npx #1{\chk_new_cs:N #1
                         \gdef:Npx #1}
\def_protected_new:Npn \gdef_long_new:Npn #1{\chk_new_cs:N #1
                                 \gdef_long:Npn #1}
\def_protected_new:Npn \gdef_long_new:Npx #1{\chk_new_cs:N #1
                                 \gdef_long:Npx #1}
\def_protected_new:Npn \gdef_protected_new:Npn #1{\chk_new_cs:N #1
                                 \gdef_protected:Npn #1}
\def_protected_new:Npn \gdef_protected_new:Npx #1{\chk_new_cs:N #1
                                 \gdef_protected:Npx #1}
\def_protected_new:Npn \gdef_protected_long_new:Npn #1{\chk_new_cs:N #1
                                 \gdef_protected_long:Npn #1}
\def_protected_new:Npn \gdef_protected_long_new:Npx #1{\chk_new_cs:N #1
                                 \gdef_protected_long:Npx #1}
%    \end{macrocode}
% \end{macro}
% \end{macro}
% \end{macro}
% \end{macro}
% \end{macro}
% \end{macro}
% \end{macro}
% \end{macro}
%
%
% \begin{macro}{\def:cpn}
% \begin{macro}{\def:cpx}
% \begin{macro}{\gdef:cpn}
% \begin{macro}{\gdef:cpx}
% \begin{macro}{\def_new:cpn}
% \begin{macro}{\def_new:cpx}
% \begin{macro}{\gdef_new:cpn}
% \begin{macro}{\gdef_new:cpx}
%   Like |\def:Npn| and |\def_new:Npn|, except that the first argument
%   consists of the sequence of characters that should be used to form
%   the name of the desired control sequence (the |c| stands for
%   csname argument, see the expansion module.). Global versions are
%   also provided.
%
%    |\def:cpn|\m{string}\m{rep-text} will turn \m{string} into a
%    csname and then assign \m {rep-text} to it by using |\def:Npn|.
%    This means that there might be a parameter string between the two
%    arguments.
%    \begin{macrocode}
\def_new:Npn \def:cpn #1{\exp_after:NN \def:Npn \cs:w #1\cs_end:}
\def_new:Npn \def:cpx #1{\exp_after:NN \def:Npx \cs:w #1\cs_end:}
\def_new:Npn \gdef:cpn #1{\exp_after:NN \gdef:Npn \cs:w #1\cs_end:}
\def_new:Npn \gdef:cpx #1{\exp_after:NN \gdef:Npx \cs:w #1\cs_end:}
\def_new:Npn \def_new:cpn #1{\exp_after:NN \def_new:Npn \cs:w #1\cs_end:}
\def_new:Npn \def_new:cpx #1{\exp_after:NN \def_new:Npx \cs:w #1\cs_end:}
\def_new:Npn \gdef_new:cpn #1{\exp_after:NN \gdef_new:Npn \cs:w #1\cs_end:}
\def_new:Npn \gdef_new:cpx #1{\exp_after:NN \gdef_new:Npx \cs:w #1\cs_end:}
%    \end{macrocode}
% \end{macro}
% \end{macro}
% \end{macro}
% \end{macro}
% \end{macro}
% \end{macro}
% \end{macro}
% \end{macro}
%
%
%
% \begin{macro}{\def_long:cpn}
% \begin{macro}{\def_long:cpx}
% \begin{macro}{\gdef_long:cpn}
% \begin{macro}{\gdef_long:cpx}
% \begin{macro}{\def_long_new:cpn}
% \begin{macro}{\def_long_new:cpx}
% \begin{macro}{\gdef_long_new:cpn}
% \begin{macro}{\gdef_long_new:cpx}
%   Variants of the |\def_long:Npn| versions which make a csname out
%   of the first arguments. We may also do this globally.
%    \begin{macrocode}
\def_new:Npn \def_long:cpn #1{\exp_after:NN \def_long:Npn \cs:w #1\cs_end:}
\def_new:Npn \def_long:cpx #1{
  \exp_after:NN\def_long:Npx\cs:w #1\cs_end:}
\def_new:Npn \gdef_long:cpn #1{
  \exp_after:NN \gdef_long:Npn \cs:w #1\cs_end:}
\def_new:Npn \gdef_long:cpx #1{
  \exp_after:NN\gdef_long:Npx\cs:w #1\cs_end:}
\def_new:Npn \def_long_new:cpn #1{
  \exp_after:NN \def_long_new:Npn \cs:w #1\cs_end:}
\def_new:Npn \def_long_new:cpx #1{
  \exp_after:NN \def_long_new:Npx \cs:w #1\cs_end:}
\def_new:Npn \gdef_long_new:cpn #1{
  \exp_after:NN \gdef_long_new:Npn \cs:w #1\cs_end:}
\def_new:Npn \gdef_long_new:cpx #1{
  \exp_after:NN \gdef_long_new:Npx \cs:w #1\cs_end:}
%    \end{macrocode}
% \end{macro}
% \end{macro}
% \end{macro}
% \end{macro}
% \end{macro}
% \end{macro}
% \end{macro}
% \end{macro}
%
% \begin{macro}{\def_protected:cpn}
% \begin{macro}{\def_protected:cpx}
% \begin{macro}{\gdef_protected:cpn}
% \begin{macro}{\gdef_protected:cpx}
% \begin{macro}{\def_protected_new:cpn}
% \begin{macro}{\def_protected_new:cpx}
% \begin{macro}{\gdef_protected_new:cpn}
% \begin{macro}{\gdef_protected_new:cpx}
%   Variants of the |\def_protected:Npn| versions which make a csname
%   out of the first arguments. We may also do this globally.
%    \begin{macrocode}
\def_new:Npn \def_protected:cpn #1{
  \exp_after:NN \def_protected:Npn \cs:w #1\cs_end:}
\def_new:Npn \def_protected:cpx #1{
  \exp_after:NN\def_protected:Npx\cs:w #1\cs_end:}
\def_new:Npn \gdef_protected:cpn #1{
  \exp_after:NN \gdef_protected:Npn \cs:w #1\cs_end:}
\def_new:Npn \gdef_protected:cpx #1{
  \exp_after:NN\gdef_protected:Npx\cs:w #1\cs_end:}
\def_new:Npn \def_protected_new:cpn #1{
  \exp_after:NN \def_protected_new:Npn \cs:w #1\cs_end:}
\def_new:Npn \def_protected_new:cpx #1{
  \exp_after:NN \def_protected_new:Npx \cs:w #1\cs_end:}
\def_new:Npn \gdef_protected_new:cpn #1{
  \exp_after:NN \gdef_protected_new:Npn \cs:w #1\cs_end:}
\def_new:Npn \gdef_protected_new:cpx #1{
  \exp_after:NN \gdef_protected_new:Npx \cs:w #1\cs_end:}
%    \end{macrocode}
% \end{macro}
% \end{macro}
% \end{macro}
% \end{macro}
% \end{macro}
% \end{macro}
% \end{macro}
% \end{macro}
%
% \begin{macro}{\def_protected_long:cpn}
% \begin{macro}{\def_protected_long:cpx}
% \begin{macro}{\gdef_protected_long:cpn}
% \begin{macro}{\gdef_protected_long:cpx}
% \begin{macro}{\def_protected_long_new:cpn}
% \begin{macro}{\def_protected_long_new:cpx}
% \begin{macro}{\gdef_protected_long_new:cpn}
% \begin{macro}{\gdef_protected_long_new:cpx}
%   Variants of the |\def_protected_long:Npn| versions which make a csname
%   out of the first arguments. We may also do this globally.
%    \begin{macrocode}
\def_new:Npn \def_protected_long:cpn #1{
  \exp_after:NN \def_protected_long:Npn \cs:w #1\cs_end:}
\def_new:Npn \def_protected_long:cpx #1{
  \exp_after:NN\def_protected_long:Npx\cs:w #1\cs_end:}
\def_new:Npn \gdef_protected_long:cpn #1{
  \exp_after:NN \gdef_protected_long:Npn \cs:w #1\cs_end:}
\def_new:Npn \gdef_protected_long:cpx #1{
  \exp_after:NN\gdef_protected_long:Npx\cs:w #1\cs_end:}
\def_new:Npn \def_protected_long_new:cpn #1{
  \exp_after:NN \def_protected_long_new:Npn \cs:w #1\cs_end:}
\def_new:Npn \def_protected_long_new:cpx #1{
  \exp_after:NN \def_protected_long_new:Npx \cs:w #1\cs_end:}
\def_new:Npn \gdef_protected_long_new:cpn #1{
  \exp_after:NN \gdef_protected_long_new:Npn \cs:w #1\cs_end:}
\def_new:Npn \gdef_protected_long_new:cpx #1{
  \exp_after:NN \gdef_protected_long_new:Npx \cs:w #1\cs_end:}
%    \end{macrocode}
% \end{macro}
% \end{macro}
% \end{macro}
% \end{macro}
% \end{macro}
% \end{macro}
% \end{macro}
% \end{macro}
%
%
% \begin{macro}{\def_aux_0:NNn}
% \begin{macro}{\def_aux_1:NNn}
% \begin{macro}{\def_aux_2:NNn}
% \begin{macro}{\def_aux_3:NNn}
% \begin{macro}{\def_aux_4:NNn}
% \begin{macro}{\def_aux_5:NNn}
% \begin{macro}{\def_aux_6:NNn}
% \begin{macro}{\def_aux_7:NNn}
% \begin{macro}{\def_aux_8:NNn}
% \begin{macro}{\def_aux_9:NNn}
% \begin{macro}{\def_aux:NNnn}
% \begin{macro}{\def_aux:Ncnn}
% \begin{macro}{\def_arg_number_error_msg:Nn}
%   Defining a function with $n$ arguments. First some helper functions.
%    \begin{macrocode}
\def_new:cpn {def_aux_0:NNn} #1#2 {#1 #2 }
\def_new:cpn {def_aux_1:NNn} #1#2 {#1 #2 ##1 }
\def_new:cpn {def_aux_2:NNn} #1#2 {#1 #2 ##1##2 }
\def_new:cpn {def_aux_3:NNn} #1#2 {#1 #2 ##1##2##3 }
\def_new:cpn {def_aux_4:NNn} #1#2 {#1 #2 ##1##2##3##4 }
\def_new:cpn {def_aux_5:NNn} #1#2 {#1 #2 ##1##2##3##4##5 }
\def_new:cpn {def_aux_6:NNn} #1#2 {#1 #2 ##1##2##3##4##5##6 }
\def_new:cpn {def_aux_7:NNn} #1#2 {#1 #2 ##1##2##3##4##5##6##7 }
\def_new:cpn {def_aux_8:NNn} #1#2 {#1 #2 ##1##2##3##4##5##6##7##8 }
\def_new:cpn {def_aux_9:NNn} #1#2 {#1 #2 ##1##2##3##4##5##6##7##8##9 }
%    \end{macrocode}
% Then the function itself which checks for the existance of such a
% helper function. If it doesn't exist, return an error. Otherwise
% call it to define \verb|#2| with the correct number of arguments.
%    \begin{macrocode}
\def_protected_long_new:Npn \def_aux:NNnn #1#2#3#4 {
  \cs_if_really_exist:cTF {def_aux_\tex_the:D\etex_numexpr:D #3 :NNn}
  {
    \cs_use:c {def_aux_\tex_the:D\etex_numexpr:D #3 :NNn} #1 #2 {#4}
  }
  { \def_arg_number_error_msg:Nn #2{#3} }
}
\def_new:Npn \def_aux:Ncnn #1#2{
  \exp_after:NN \def_aux:NNnn \exp_after:NN #1 \cs:w #2\cs_end:}
%    \end{macrocode}
% The error message.
%    \begin{macrocode}
\def_new:Npn \def_arg_number_error_msg:Nn #1#2 {
  \err_latex_bug:x{
    You're~ trying~ to~ define~ the~ command~ `\token_to_string:N #1'~
    with~ \use_arg_i:n{\tex_the:D\etex_numexpr:D #2\scan_stop:} ~
    arguments~ but~ I~ only~ allow~ 0-9~ arguments.~ I~ can~ probably~
    not~ help~ you~ here 
  } 
}
%    \end{macrocode}
% \end{macro}
% \end{macro}
% \end{macro}
% \end{macro}
% \end{macro}
% \end{macro}
% \end{macro}
% \end{macro}
% \end{macro}
% \end{macro}
% \end{macro}
% \end{macro}
% \end{macro}
%
% \begin{macro}{\def_aux_use_0_parameter:}
% \begin{macro}{\def_aux_use_1_parameter:}
% \begin{macro}{\def_aux_use_2_parameter:}
% \begin{macro}{\def_aux_use_3_parameter:}
% \begin{macro}{\def_aux_use_4_parameter:}
% \begin{macro}{\def_aux_use_5_parameter:}
% \begin{macro}{\def_aux_use_6_parameter:}
% \begin{macro}{\def_aux_use_7_parameter:}
% \begin{macro}{\def_aux_use_8_parameter:}
% \begin{macro}{\def_aux_use_9_parameter:}
%   Something similar to |\def_aux_9:NNn| but for using the
%   parameters.
%    \begin{macrocode}
\def:cpn{def_aux_use_0_parameter:}{}
\def:cpn{def_aux_use_1_parameter:}{{##1}}
\def:cpn{def_aux_use_2_parameter:}{{##1}{##2}}
\def:cpn{def_aux_use_3_parameter:}{{##1}{##2}{##3}}
\def:cpn{def_aux_use_4_parameter:}{{##1}{##2}{##3}{##4}}
\def:cpn{def_aux_use_5_parameter:}{{##1}{##2}{##3}{##4}{##5}}
\def:cpn{def_aux_use_6_parameter:}{{##1}{##2}{##3}{##4}{##5}{##6}}
\def:cpn{def_aux_use_7_parameter:}{{##1}{##2}{##3}{##4}{##5}{##6}{##7}}
\def:cpn{def_aux_use_8_parameter:}{
  {##1}{##2}{##3}{##4}{##5}{##6}{##7}{##8}}
\def:cpn{def_aux_use_9_parameter:}{
  {##1}{##2}{##3}{##4}{##5}{##6}{##7}{##8}{##9}}
%    \end{macrocode}
% \end{macro}
% \end{macro}
% \end{macro}
% \end{macro}
% \end{macro}
% \end{macro}
% \end{macro}
% \end{macro}
% \end{macro}
% \end{macro}
%
% \begin{macro}{\def:NNn}
% \begin{macro}{\def:NNx}
% \begin{macro}{\def:cNn}
% \begin{macro}{\def:cNx}
% \begin{macro}{\gdef:NNn}
% \begin{macro}{\gdef:NNx}
% \begin{macro}{\gdef:cNn}
% \begin{macro}{\gdef:cNx}
% \begin{macro}{\def_new:NNn}
% \begin{macro}{\def_new:NNx}
% \begin{macro}{\def_new:cNn}
% \begin{macro}{\def_new:cNx}
% \begin{macro}{\gdef_new:NNn}
% \begin{macro}{\gdef_new:NNx}
% \begin{macro}{\gdef_new:cNn}
% \begin{macro}{\gdef_new:cNx}
%   Defining macros without delimited arguments is now relatively
%   easy. First local and global versions of the usual |\def:Npn|
%   operation.
%    \begin{macrocode}
\def_new:Npn \def:NNn { \def_aux:NNnn \def:Npn }
\def_new:Npn \def:NNx { \def_aux:NNnn \def:Npx }
\def_new:Npn \def:cNn { \def_aux:Ncnn \def:Npn }
\def_new:Npn \def:cNx { \def_aux:Ncnn \def:Npx }
\def_new:Npn \gdef:NNn { \def_aux:NNnn \gdef:Npn }
\def_new:Npn \gdef:NNx { \def_aux:NNnn \gdef:Npx }
\def_new:Npn \gdef:cNn { \def_aux:Ncnn \gdef:Npn }
\def_new:Npn \gdef:cNx { \def_aux:Ncnn \gdef:Npx }
\def_new:Npn \def_new:NNn { \def_aux:NNnn \def_new:Npn }
\def_new:Npn \def_new:NNx { \def_aux:NNnn \def_new:Npx }
\def_new:Npn \def_new:cNn { \def_aux:Ncnn \def_new:Npn }
\def_new:Npn \def_new:cNx { \def_aux:Ncnn \def_new:Npx }
\def_new:Npn \gdef_new:NNn { \def_aux:NNnn \gdef_new:Npn }
\def_new:Npn \gdef_new:NNx { \def_aux:NNnn \gdef_new:Npx }
\def_new:Npn \gdef_new:cNn { \def_aux:Ncnn \gdef_new:Npn }
\def_new:Npn \gdef_new:cNx { \def_aux:Ncnn \gdef_new:Npx }
%    \end{macrocode}
% \end{macro}
% \end{macro}
% \end{macro}
% \end{macro}
% \end{macro}
% \end{macro}
% \end{macro}
% \end{macro}
% \end{macro}
% \end{macro}
% \end{macro}
% \end{macro}
% \end{macro}
% \end{macro}
% \end{macro}
% \end{macro}
%
% \begin{macro}{\def_long:NNn}
% \begin{macro}{\def_long:NNx}
% \begin{macro}{\def_long:cNn}
% \begin{macro}{\def_long:cNx}
% \begin{macro}{\gdef_long:NNn}
% \begin{macro}{\gdef_long:NNx}
% \begin{macro}{\gdef_long:cNn}
% \begin{macro}{\gdef_long:cNx}
% \begin{macro}{\def_long_new:NNn}
% \begin{macro}{\def_long_new:NNx}
% \begin{macro}{\def_long_new:cNn}
% \begin{macro}{\def_long_new:cNx}
% \begin{macro}{\gdef_long_new:NNn}
% \begin{macro}{\gdef_long_new:NNx}
% \begin{macro}{\gdef_long_new:cNn}
% \begin{macro}{\gdef_long_new:cNx}
% Long versions of the above.
%    \begin{macrocode}
\def_new:Npn \def_long:NNn { \def_aux:NNnn \def_long:Npn }
\def_new:Npn \def_long:NNx { \def_aux:NNnn \def_long:Npx }
\def_new:Npn \def_long:cNn { \def_aux:Ncnn \def_long:Npn }
\def_new:Npn \def_long:cNx { \def_aux:Ncnn \def_long:Npx }
\def_new:Npn \gdef_long:NNn { \def_aux:NNnn \gdef_long:Npn }
\def_new:Npn \gdef_long:NNx { \def_aux:NNnn \gdef_long:Npx }
\def_new:Npn \gdef_long:cNn { \def_aux:Ncnn \gdef_long:Npn }
\def_new:Npn \gdef_long:cNx { \def_aux:Ncnn \gdef_long:Npx }
\def_new:Npn \def_long_new:NNn { \def_aux:NNnn \def_long_new:Npn }
\def_new:Npn \def_long_new:NNx { \def_aux:NNnn \def_long_new:Npx }
\def_new:Npn \def_long_new:cNn { \def_aux:Ncnn \def_long_new:Npn }
\def_new:Npn \def_long_new:cNx { \def_aux:Ncnn \def_long_new:Npx }
\def_new:Npn \gdef_long_new:NNn { \def_aux:NNnn \gdef_long_new:Npn }
\def_new:Npn \gdef_long_new:NNx { \def_aux:NNnn \gdef_long_new:Npx }
\def_new:Npn \gdef_long_new:cNn { \def_aux:Ncnn \gdef_long_new:Npn }
\def_new:Npn \gdef_long_new:cNx { \def_aux:Ncnn \gdef_long_new:Npx }
%    \end{macrocode}
% \end{macro}
% \end{macro}
% \end{macro}
% \end{macro}
% \end{macro}
% \end{macro}
% \end{macro}
% \end{macro}
% \end{macro}
% \end{macro}
% \end{macro}
% \end{macro}
% \end{macro}
% \end{macro}
% \end{macro}
% \end{macro}
%
%
% \begin{macro}{\def_protected:NNn}
% \begin{macro}{\def_protected:NNx}
% \begin{macro}{\def_protected:cNn}
% \begin{macro}{\def_protected:cNx}
% \begin{macro}{\gdef_protected:NNn}
% \begin{macro}{\gdef_protected:NNx}
% \begin{macro}{\gdef_protected:cNn}
% \begin{macro}{\gdef_protected:cNx}
% \begin{macro}{\def_protected_new:NNn}
% \begin{macro}{\def_protected_new:NNx}
% \begin{macro}{\def_protected_new:cNn}
% \begin{macro}{\def_protected_new:cNx}
% \begin{macro}{\gdef_protected_new:NNn}
% \begin{macro}{\gdef_protected_new:NNx}
% \begin{macro}{\gdef_protected_new:cNn}
% \begin{macro}{\gdef_protected_new:cNx}
% Protected versions of the above.
%    \begin{macrocode}
\def_new:Npn \def_protected:NNn { \def_aux:NNnn \def_protected:Npn }
\def_new:Npn \def_protected:NNx { \def_aux:NNnn \def_protected:Npx }
\def_new:Npn \def_protected:cNn { \def_aux:Ncnn \def_protected:Npn }
\def_new:Npn \def_protected:cNx { \def_aux:Ncnn \def_protected:Npx }
\def_new:Npn \gdef_protected:NNn { \def_aux:NNnn \gdef_protected:Npn }
\def_new:Npn \gdef_protected:NNx { \def_aux:NNnn \gdef_protected:Npx }
\def_new:Npn \gdef_protected:cNn { \def_aux:Ncnn \gdef_protected:Npn }
\def_new:Npn \gdef_protected:cNx { \def_aux:Ncnn \gdef_protected:Npx }
\def_new:Npn \def_protected_new:NNn { \def_aux:NNnn \def_protected_new:Npn }
\def_new:Npn \def_protected_new:NNx { \def_aux:NNnn \def_protected_new:Npx }
\def_new:Npn \def_protected_new:cNn { \def_aux:Ncnn \def_protected_new:Npn }
\def_new:Npn \def_protected_new:cNx { \def_aux:Ncnn \def_protected_new:Npx }
\def_new:Npn \gdef_protected_new:NNn { \def_aux:NNnn \gdef_protected_new:Npn }
\def_new:Npn \gdef_protected_new:NNx { \def_aux:NNnn \gdef_protected_new:Npx }
\def_new:Npn \gdef_protected_new:cNn { \def_aux:Ncnn \gdef_protected_new:Npn }
\def_new:Npn \gdef_protected_new:cNx { \def_aux:Ncnn \gdef_protected_new:Npx }
%    \end{macrocode}
% \end{macro}
% \end{macro}
% \end{macro}
% \end{macro}
% \end{macro}
% \end{macro}
% \end{macro}
% \end{macro}
% \end{macro}
% \end{macro}
% \end{macro}
% \end{macro}
% \end{macro}
% \end{macro}
% \end{macro}
% \end{macro}
%
% \begin{macro}{\def_protected_long:NNn}
% \begin{macro}{\def_protected_long:NNx}
% \begin{macro}{\def_protected_long:cNn}
% \begin{macro}{\def_protected_long:cNx}
% \begin{macro}{\gdef_protected_long:NNn}
% \begin{macro}{\gdef_protected_long:NNx}
% \begin{macro}{\gdef_protected_long:cNn}
% \begin{macro}{\gdef_protected_long:cNx}
% \begin{macro}{\def_protected_long_new:NNn}
% \begin{macro}{\def_protected_long_new:NNx}
% \begin{macro}{\def_protected_long_new:cNn}
% \begin{macro}{\def_protected_long_new:cNx}
% \begin{macro}{\gdef_protected_long_new:NNn}
% \begin{macro}{\gdef_protected_long_new:NNx}
% \begin{macro}{\gdef_protected_long_new:cNn}
% \begin{macro}{\gdef_protected_long_new:cNx}
% And finally both long and protected.
%    \begin{macrocode}
\def_new:Npn \def_protected_long:NNn { \def_aux:NNnn \def_protected_long:Npn }
\def_new:Npn \def_protected_long:NNx { \def_aux:NNnn \def_protected_long:Npx }
\def_new:Npn \def_protected_long:cNn { \def_aux:Ncnn \def_protected_long:Npn }
\def_new:Npn \def_protected_long:cNx { \def_aux:Ncnn \def_protected_long:Npx }
\def_new:Npn \gdef_protected_long:NNn { \def_aux:NNnn \gdef_protected_long:Npn }
\def_new:Npn \gdef_protected_long:NNx { \def_aux:NNnn \gdef_protected_long:Npx }
\def_new:Npn \gdef_protected_long:cNn { \def_aux:Ncnn \gdef_protected_long:Npn }
\def_new:Npn \gdef_protected_long:cNx { \def_aux:Ncnn \gdef_protected_long:Npx }
\def_new:Npn \def_protected_long_new:NNn {
  \def_aux:NNnn \def_protected_long_new:Npn }
\def_new:Npn \def_protected_long_new:NNx {
  \def_aux:NNnn \def_protected_long_new:Npx }
\def_new:Npn \def_protected_long_new:cNn {
  \def_aux:Ncnn \def_protected_long_new:Npn }
\def_new:Npn \def_protected_long_new:cNx {
  \def_aux:Ncnn \def_protected_long_new:Npx }
\def_new:Npn \gdef_protected_long_new:NNn {
  \def_aux:NNnn \gdef_protected_long_new:Npn }
\def_new:Npn \gdef_protected_long_new:NNx {
  \def_aux:NNnn \gdef_protected_long_new:Npx }
\def_new:Npn \gdef_protected_long_new:cNn {
  \def_aux:Ncnn \gdef_protected_long_new:Npn }
\def_new:Npn \gdef_protected_long_new:cNx {
  \def_aux:Ncnn \gdef_protected_long_new:Npx }
%    \end{macrocode}
% \end{macro}
% \end{macro}
% \end{macro}
% \end{macro}
% \end{macro}
% \end{macro}
% \end{macro}
% \end{macro}
% \end{macro}
% \end{macro}
% \end{macro}
% \end{macro}
% \end{macro}
% \end{macro}
% \end{macro}
% \end{macro}
%
%
% \begin{macro}{\let:NN}
% \begin{macro}{\let:cN}
% \begin{macro}{\let:Nc}
% \begin{macro}{\let:cc}
% \begin{macro}{\let_new:NN}
% \begin{macro}{\let_new:cN}
% \begin{macro}{\let_new:Nc}
% \begin{macro}{\let_new:cc}
%    These macros allow us to copy the definition of a control sequence
%    to another control sequence.
%    \begin{macrocode}
\def_protected_long_new:Npn \let:NN #1{
%    \end{macrocode}
%    The |=| sign allows us to define funny char tokens like .|=|
%    itself or \verb*= = with this function. For the definition of
%    |\c_space_chartok{~}| to work we need the |~| after the |=|
%    \begin{macrocode}
                              \let:NwN #1=~}
\def_new:Npn\let:cN #1 {\exp_after:NN\let:NN\cs:w#1\cs_end:}
\def_new:Npn\let:Nc{\exp_args:NNc\let:NN}
\def_new:Npn\let:cc{\exp_args:Ncc\let:NN}
\def_new:Npn \let_new:NN #1{\chk_new_cs:N #1
                              \let:NN #1}
\def_new:Npn \let_new:cN {\exp_args:Nc \let_new:NN}
\def_new:Npn \let_new:Nc {\exp_args:NNc \let_new:NN}
\def_new:Npn \let_new:cc {\exp_args:Ncc \let_new:NN}
%    \end{macrocode}
% \end{macro}
% \end{macro}
% \end{macro}
% \end{macro}
% \end{macro}
% \end{macro}
% \end{macro}
% \end{macro}
%
% \begin{macro}{\glet:NN}
% \begin{macro}{\glet:cN}
% \begin{macro}{\glet:Nc}
% \begin{macro}{\glet:cc}
% \begin{macro}{\glet_new:NN}
% \begin{macro}{\glet_new:cN}
% \begin{macro}{\glet_new:Nc}
% \begin{macro}{\glet_new:cc}
%  These are global versions of some of the previously defined functions.
%    \begin{macrocode}
\def_protected_new:Npn \glet:NN {\pref_global:D \let:NN}
\def_protected_new:Npn \glet:Nc {\exp_args:NNc \glet:NN}
\def_protected_new:Npn \glet:cN {\exp_args:Nc \glet:NN}
\def_new:Npn \glet:cc {\exp_args:Ncc \glet:NN}
\def_new:Npn \glet_new:NN #1{\chk_new_cs:N #1
                               \tex_global:D\let:NN #1}
\def_new:Npn \glet_new:cN {\exp_args:Nc \glet_new:NN}
\def_new:Npn \glet_new:Nc {\exp_args:NNc \glet_new:NN}
\def_new:Npn \glet_new:cc {\exp_args:Ncc \glet_new:NN}
%    \end{macrocode}
% \end{macro}
% \end{macro}
% \end{macro}
% \end{macro}
% \end{macro}
% \end{macro}
% \end{macro}
% \end{macro}
%
%
% \begin{macro}{\def:No}
% \begin{macro}{\gdef:No}
%    |\def:No| expands its second argument one time before making
%    the definition.
%    \begin{macrocode}
\def_new:Npn \def:No{\exp_args:NNo\def:Npn}
\def_new:Npn \gdef:No  {\exp_args:NNo\gdef:Npn}
%    \end{macrocode}
% \end{macro}
% \end{macro}
%
% \begin{macro}{\def_test_function_aux:Nnnn}
% \begin{macro}{\def_test_function_aux:Nnnx}
% \begin{macro}{\def_test_function:npn}
% \begin{macro}{\def_test_function:npx}
% \begin{macro}{\def_long_test_function:npn}
% \begin{macro}{\def_long_test_function:npx}
% \begin{macro}{\def_test_function_new:npn}
% \begin{macro}{\def_test_function_new:npx}
% \begin{macro}{\def_long_test_function_new:npn}
% \begin{macro}{\def_long_test_function_new:npx}
%   We will often be defining several almost identical TF, T and F
%   type functions so it makes sense for us to define a small function
%   that will do this for us so that we are less likely to introduce
%   typos (it does tend to happen). By doing it in two steps as below
%   we can still retain a simple interface where you write the \TeX\
%   parameters as usual. Just don't do it when you're already within a
%   conditional!
%
%   I think the ways of exiting conditionals below are as fast as they
%   get. Using |\reverse_if:N| instead of |\else:| didn't give any
%   difference I could measure.
%    \begin{macrocode}
\def_long_new:Npn \def_test_function_aux:Nnnn #1#2#3#4{
  #1 {#2TF} #3 {#4
    \exp_after:NN\use_arg_i:nn\else:\exp_after:NN\use_arg_ii:nn\fi:}
  #1 {#2T} #3 {#4
    \else:\exp_after:NN\use_none:nn\fi:\use_arg_i:n}
  #1 {#2F} #3 {#4
    \exp_after:NN\use_none:nn\fi:\use_arg_i:n}}
\def_long_new:Npn \def_test_function_aux:Nnnx #1#2#3#4{
  #1 {#2TF} #3 {#4
    \exp_not:n{\exp_after:NN\use_arg_i:nn\else:\exp_after:NN\use_arg_ii:nn\fi:}}
  #1 {#2T} #3 {#4
    \exp_not:n{\else:\exp_after:NN\use_none:nn\fi:\use_arg_i:n}}
  #1 {#2F} #3 {#4
    \exp_not:n{\exp_after:NN\use_none:nn\fi:\use_arg_i:n}}}
\def_long_new:Npn \def_test_function:npn #1#2#{
  \def_test_function_aux:Nnnn \def:cpn {#1}{#2}
}
\def_long_new:Npn \def_test_function:npx #1#2#{
  \def_test_function_aux:Nnnx \def:cpx {#1}{#2}
}
\def_long_new:Npn \def_long_test_function:npn #1#2#{
  \def_test_function_aux:Nnnn \def_long:cpn {#1}{#2}
}
\def_long_new:Npn \def_long_test_function:npx #1#2#{
  \def_test_function_aux:Nnnx \def_long:cpx {#1}{#2}
}
\def_long_new:Npn \def_test_function_new:npn #1#2#{
  \def_test_function_aux:Nnnn \def_new:cpn {#1}{#2}
}
\def_long_new:Npn \def_long_test_function_new:npn #1#2#{
  \def_test_function_aux:Nnnn \def_long_new:cpn {#1}{#2}
}
\def_long_new:Npn \def_test_function_new:npx #1#2#{
  \def_test_function_aux:Nnnx \def_new:cpx {#1}{#2}
}
\def_long_new:Npn \def_long_test_function_new:npx #1#2#{
  \def_test_function_aux:Nnnx \def_long_new:cpx {#1}{#2}
}
%    \end{macrocode}
% \end{macro}
% \end{macro}
% \end{macro}
% \end{macro}
% \end{macro}
% \end{macro}
% \end{macro}
% \end{macro}
% \end{macro}
% \end{macro}
%
%
% \subsubsection{Further checking}
%
% \begin{macro}{\cs_if_free:NTF}
% \begin{macro}{\cs_if_free:NT}
% \begin{macro}{\cs_if_free:NF}
%    The old |\@ifundefined| of \LaTeX{} 2.09 is re-implemented in the
%    function |\cs_free:cTF|, again in a way that |\else:| and |\fi:|
%    are removed. In this implementation this is absolutely
%    necessary because functions inside the conditional parts expect
%    to read further input from outside the conditional.  Actually,
%    the first part of the code below is more general, since it checks
%    \m{csnames} directly and therefore allows both |\scan_stop:| and
%    |\c_undefined|.
%    \begin{macrocode}
\def_long_test_function_new:npn {cs_if_free:N}#1{\if:w\cs_if_free_p:N #1}
\let:NN \cs_free:NTF \cs_if_free:NTF
\let:NN \cs_free:NT \cs_if_free:NT
\let:NN \cs_free:NF \cs_if_free:NF
%    \end{macrocode}
% \end{macro}
% \end{macro}
% \end{macro}
%
% \begin{macro}{\cs_if_free:cTF}
% \begin{macro}{\cs_if_free:cT}
% \begin{macro}{\cs_if_free:cF}
%    We have to implement the |c| variants `by hand' because a different
%    test is necessary and I don't want the overhead for the test with
%    |\if:w|.  What a mistake Don made by making this a
%    feature of |\cs:w|. If I'm not totally mistaken this
%    feature alone has cost him more than 600\$ for bug-checks.
%    \begin{macrocode}
\def_long_test_function_new:npn {cs_if_free:c}#1{
  \exp_after:NN \if_meaning:NN \cs:w#1\cs_end: \scan_stop:}
\let:NN \cs_free:cTF \cs_if_free:cTF
\let:NN \cs_free:cT \cs_if_free:cT
\let:NN \cs_free:cF \cs_if_free:cF
%    \end{macrocode}
% \end{macro}
% \end{macro}
% \end{macro}
%
% \begin{macro}{\cs_if_really_free:cTF}
% \begin{macro}{\cs_if_really_free:cT}
% \begin{macro}{\cs_if_really_free:cF}
%   These versions are for special control sequences that can only be
%   formed through |\cs:w ... \cs_end:|. They do not turn the control
%   sequence formed into |\scan_stop:|.
%    \begin{macrocode}
\def_long_test_function_new:npn {cs_if_really_free:c}#1{
  \reverse_if:N\if_cs_exist:w #1\cs_end:}
\let:NN \cs_really_free:cTF \cs_if_really_free:cTF
\let:NN \cs_really_free:cT \cs_if_really_free:cT
\let:NN \cs_really_free:cF \cs_if_really_free:cF
%    \end{macrocode}
% \end{macro}
% \end{macro}
% \end{macro}
%
% \begin{macro}{\cs_if_exist:NTF}
% \begin{macro}{\cs_if_exist:NT}
% \begin{macro}{\cs_if_exist:NF}
% \begin{macro}{\cs_if_exist:cTF}
% \begin{macro}{\cs_if_exist:cT}
% \begin{macro}{\cs_if_exist:cF}
% \begin{macro}{\cs_if_really_exist:cTF}
% \begin{macro}{\cs_if_really_exist:cT}
% \begin{macro}{\cs_if_really_exist:cF}
%   Now the same functions but with reverse logic: test if the control
%   sequence exists.
%    \begin{macrocode}
\def_long_test_function_new:npn {cs_if_exist:N}#1{\if:w\cs_if_exist_p:N #1}
\def_long_test_function_new:npn {cs_if_exist:c}#1{
  \exp_after:NN\reverse_if:N
  \exp_after:NN \if_meaning:NN \cs:w#1\cs_end: \scan_stop:}
\def_long_test_function_new:npn {cs_if_really_exist:c}#1{
  \if_cs_exist:w #1\cs_end:}
%    \end{macrocode}
% \end{macro}
% \end{macro}
% \end{macro}
% \end{macro}
% \end{macro}
% \end{macro}
% \end{macro}
% \end{macro}
% \end{macro}
%
%
% \subsubsection{Freeing memory}
%
% \begin{macro}{\cs_gundefine:N }
%    The following function is used to free the main memory from the
%    definition of some function that isn't in use any longer.
%    \begin{macrocode}
\def_new:Npn \cs_gundefine:N #1{\glet:NN #1\c_undefined}
%    \end{macrocode}
% \end{macro}
%
%
%  \subsubsection{Engine specific definitions}
%
% \begin{macro}{\engine_aleph:TF}
%  In some cases it will be useful to know which engine we're running.
%    \begin{macrocode}
\def_test_function_new:npn {engine_aleph:}{\if_cs_exist:N \aleph_textdir:D}
%    \end{macrocode}
% \end{macro}
%
%
%
% \subsubsection{Selecting tokens}
%
% \begin{macro}{\use_arg_i:n}
%    This macro grabs its argument and returns it back to the input
%    (with outer braces removed).
%    \begin{macrocode}
%\def_long_new:Npn \use_arg_i:n #1{#1}% moved earlier
%    \end{macrocode}
% \end{macro}
%
% \begin{macro}{\use:c}
% \begin{macro}{\cs_use:c}
% \begin{macro}{\use:cc}
%    This macro grabs its argument and returns a csname from it.
%    \begin{macrocode}
\def_new:Npn \use:c #1{\cs:w #1\cs_end:}
\def_new:Npn \cs_use:c #1 { \cs:w#1\cs_end: }
%    \end{macrocode}
%    THE NAME IS COMPLETELY WRONG!!!!!  
% Morten says: Perhaps this is really |\exp_args:cc| instead?
%    \begin{macrocode}
\def_new:Npn \use:cc #1#2
  {\cs:w #1\exp_after:NN\cs_end:\cs:w #2\cs_end:}
%    \end{macrocode}
% \end{macro}
% \end{macro}
% \end{macro}
%
% \begin{macro}{\use_arg_i:nn}
% \begin{macro}{\use_arg_ii:nn}
%    These macros are needed to provide functions with true and false
%    cases, as introduced by Michael some time ago. By using
%    |\exp_after:NN| |\use_arg_i:nn | |\else:| constructions it
%    is possible to write code where the true or false case is able to
%    access the following tokens from the input stream, which is not
%    possible if the |\c_true| syntax is used.
%    \begin{macrocode}
\def_long_new:Npn \use_arg_i:nn #1#2{#1}
\def_long_new:Npn \use_arg_ii:nn #1#2{#2}
%    \end{macrocode}
% \end{macro}
% \end{macro}
%
%
%
% \begin{macro}{\use_arg_i:nnn}
% \begin{macro}{\use_arg_ii:nnn}
% \begin{macro}{\use_arg_iii:nnn}
% \begin{macro}{\use_arg_i:nnnn}
% \begin{macro}{\use_arg_ii:nnnn}
% \begin{macro}{\use_arg_iii:nnnn}
% \begin{macro}{\use_arg_iv:nnnn}
%    We also need something for picking up arguments from a longer
%    list.
%    \begin{macrocode}
\def_long_new:NNn \use_arg_i:nnn    3{#1}
\def_long_new:NNn \use_arg_ii:nnn   3{#2}
\def_long_new:NNn \use_arg_iii:nnn  3{#3}
\def_long_new:NNn \use_arg_i:nnnn   4{#1}
\def_long_new:NNn \use_arg_ii:nnnn  4{#2}
\def_long_new:NNn \use_arg_iii:nnnn 4{#3}
\def_long_new:NNn \use_arg_iv:nnnn  4{#4}
%    \end{macrocode}
% \end{macro}
% \end{macro}
% \end{macro}
% \end{macro}
% \end{macro}
% \end{macro}
% \end{macro}
%
% \begin{macro}{\use_arg_i_ii:nn}
%    And a function for grabbing two arguments and returning them
%    again.
%    \begin{macrocode}
\def_long_new:NNn\use_arg_i_ii:nn 2{#1#2}
%    \end{macrocode}
% \end{macro}
%
% \begin{macro}{\use_none_delimit_by_q_nil:w}
% \begin{macro}{\use_none_delimit_by_q_stop:w}
%   Functions that gobble everything until they see either |\q_nil| or
%   |\q_stop| resp.
%    \begin{macrocode}
\def_long_new:Npn \use_none_delimit_by_q_nil:w #1\q_nil{}
\def_long_new:Npn \use_none_delimit_by_q_stop:w #1\q_stop{}
%    \end{macrocode}
% \end{macro}
% \end{macro}
%
% \begin{macro}{\use_arg_i_delimit_by_q_nil:nw}
% \begin{macro}{\use_arg_i_delimit_by_q_stop:nw}
%   Same as above but execute first argument after gobbling. Very
%   useful when you need to skip the rest of a mapping sequence but
%   want an easy way to control what should be expanded next.
%    \begin{macrocode}
\def_long_new:Npn \use_arg_i_delimit_by_q_nil:nw #1#2\q_nil{#1}
\def_long_new:Npn \use_arg_i_delimit_by_q_stop:nw #1#2\q_stop{#1}
%    \end{macrocode}
% \end{macro}
% \end{macro}
%
%
% \begin{macro}{\use_arg_i_after_fi:nw}
% \begin{macro}{\use_arg_i_after_else:nw}
% \begin{macro}{\use_arg_i_after_or:nw}
%   Returns the first argument after ending the conditional.
%    \begin{macrocode}
\def_long_new:Npn \use_arg_i_after_fi:nw #1\fi:{\fi: #1}
\def_long_new:Npn \use_arg_i_after_else:nw #1\else:#2\fi:{\fi: #1}
\def_long_new:Npn \use_arg_i_after_or:nw #1\or: #2\fi: {\or:\fi:#1}
%    \end{macrocode}
% \end{macro}
% \end{macro}
% \end{macro}
%
% \subsubsection{Gobbling tokens from input}
%
% \begin{macro}{\use_none:n}
% \begin{macro}{\use_none:nn}
% \begin{macro}{\use_none:nnn}
% \begin{macro}{\use_none:nnnn}
% \begin{macro}{\use_none:nnnnn}
% \begin{macro}{\use_none:nnnnnn}
% \begin{macro}{\use_none:nnnnnnn}
% \begin{macro}{\use_none:nnnnnnnn}
% \begin{macro}{\use_none:nnnnnnnnn}
%   To gobble tokens from the input we use a standard naming
%   convention: the number of tokens gobbled is given by the number of
%   |n|'s following the |:| in the name. Although defining
%   |\use_none:nnn| and above as separate calls of |\use_none:n| and
%   |\use_none:nn| is slightly faster, this is very non-intuitive to
%   the programmer who will assume that expanding such a function once
%   will take care of gobbling all the tokens in one go.
%    \begin{macrocode}
%\def_long_new:NNn \use_none:n 1{}% moved earlier
\def_long_new:NNn \use_none:nn 2{}
\def_long_new:NNn \use_none:nnn 3{}
\def_long_new:NNn \use_none:nnnn 4{}
\def_long_new:NNn \use_none:nnnnn 5{}
\def_long_new:NNn \use_none:nnnnnn 6{}
\def_long_new:NNn \use_none:nnnnnnn 7{}
\def_long_new:NNn \use_none:nnnnnnnn 8{}
\def_long_new:NNn \use_none:nnnnnnnnn 9{}
%    \end{macrocode}
% \end{macro}
% \end{macro}
% \end{macro}
% \end{macro}
% \end{macro}
% \end{macro}
% \end{macro}
% \end{macro}
% \end{macro}
%
%
% \subsubsection{Scratch functions}
%
% \begin{macro}{\gtmp:w}
%    This function is for global scratch definitions that are used
%    immediately afterwards. It should be used when we need a function
%    that operates on input, i.e.\ has arguments. If we want to save
%    only some tokens for later use, token-list scratch variables
%    should be used.
%    \begin{macrocode}
\def_new:Npn \gtmp:w {}
%    \end{macrocode}
% \end{macro}
%
% \begin{macro}{\tmp:w}
%    This is a local version of the previous function.
%    \begin{macrocode}
\def_new:Npn \tmp:w {}
%    \end{macrocode}
% \end{macro}
%
% \begin{macro}{\use_noop:}
%    I don't think this function belongs here, but one place is as
%    good as any other. I want to use this function when I want to
%    express `no operation'.
%    \begin{macrocode}
\def_new:Npn \use_noop: {}
%    \end{macrocode}
% \end{macro}
%
% \subsubsection{Strings and input stream token lists}
%
% \begin{macro}{\cs_to_str:N}
%   This converts a control sequence into the character string of its
%   name, removing the leading escape character.
%    \begin{macrocode}
\def_new:Npn \cs_to_str:N {\exp_after:NN\use_none:n \token_to_string:N}
%    \end{macrocode}
% \end{macro}
%
%
%
%
%
% \begin{macro}{\cs_if_eq:NNTF}
% \begin{macro}{\cs_if_eq:NNT}
% \begin{macro}{\cs_if_eq:NNF}
% \begin{macro}{\cs_if_eq:cNTF}
% \begin{macro}{\cs_if_eq:cNT}
% \begin{macro}{\cs_if_eq:cNF}
% \begin{macro}{\cs_if_eq:NcTF}
% \begin{macro}{\cs_if_eq:NcT}
% \begin{macro}{\cs_if_eq:NcF}
% \begin{macro}{\cs_if_eq:ccTF}
% \begin{macro}{\cs_if_eq:ccT}
% \begin{macro}{\cs_if_eq:ccF}
% Check if two control sequences are identical.
%    \begin{macrocode}
\def_test_function_new:npn {cs_if_eq:NN} #1#2{\if_meaning:NN #1#2}
\def_new:Npn \cs_if_eq:cNTF {\exp_args:Nc \cs_if_eq:NNTF}
\def_new:Npn \cs_if_eq:cNT {\exp_args:Nc \cs_if_eq:NNT}
\def_new:Npn \cs_if_eq:cNF {\exp_args:Nc \cs_if_eq:NNF}
\def_new:Npn \cs_if_eq:NcTF {\exp_args:NNc \cs_if_eq:NNTF}
\def_new:Npn \cs_if_eq:NcT {\exp_args:NNc \cs_if_eq:NNT}
\def_new:Npn \cs_if_eq:NcF {\exp_args:NNc \cs_if_eq:NNF}
\def_new:Npn \cs_if_eq:ccTF {\exp_args:Ncc \cs_if_eq:NNTF}
\def_new:Npn \cs_if_eq:ccT {\exp_args:Ncc \cs_if_eq:NNT}
\def_new:Npn \cs_if_eq:ccF {\exp_args:Ncc \cs_if_eq:NNF}
%    \end{macrocode}
% \end{macro}
% \end{macro}
% \end{macro}
% \end{macro}
% \end{macro}
% \end{macro}
% \end{macro}
% \end{macro}
% \end{macro}
% \end{macro}
% \end{macro}
% \end{macro}
%
% Finally some code that is needed as we do not distribute the file
% module at the moment (so we simply define the needed function via an
% existing \LaTeX{} command) and some other stuff which was set up
% elsewhere, in undistributed modules.
%    \begin{macrocode}
\def_new:Npn\file_not_found:nTF #1#2#3{\IfFileExists{#1}{#3}{#2}}
%    \end{macrocode}
%
%
% \subsubsection{Predicates and conditionals}
% \label{sec:predicates}
%
% \LaTeX3 has three concepts for conditional flow processing:
% \begin{enumerate}
%
% \item
%   Functions that carry out a test an then execute, depending on its
% result, either the code supplied in the \m{true arg} or the \m{false
% arg}. These arguments are denoted with "T" and "F" repectively. An
% example would be
% \begin{quote}
%  "\cs_free:cTF{abc}{...}{...}"
% \end{quote}
% a function that will turn the first argument into a control sequence
% (since its marked as "c") then checks whether this control sequence is
% still free and then depending on the result carry out the code in the
% second argument (true case) or in the third argument (false case).
%
% \item
%   Functions that return a special type of boolean value which can be
% tested by the function "\if:w". All functions of this type
% have names that end with "_p" in the description part. For example
% \begin{quote}
%  "\cs_free_p:N"
% \end{quote}
% would be a predicate function for the same type of test as the
% function above. It would return `true' if its argument (a single token
% denoted by "N") is still free for definition. It would be used in
% constructions like
% \begin{quote}
%  "\if:w \cs_free_p:N \l_foo_bar ... \else: ... \fi:"
% \end{quote}
%
% \item
%   Actually there is a third one, namely the original concept used in
% plain \TeX{}. This belongs to the second form but needs further
% thoughts.
% \end{enumerate}
%
% Important to note is that conditionals with \m{true code} and/or
% \m{false code} are always defined in a way that the code of the chosen
% alternative can operate on following tokens in the input stream while
% the predicate implementations always have an "\else:" or "\fi:"
% interfering. This can be important in scanner implementations.
%
%
%    \begin{macrocode}
%</initex|package>
%<*showmemory>
\showMemUsage
%</showmemory>
%    \end{macrocode}
%
% \endinput
%
%  $Log$
%  Revision 1.38  2006/08/06 17:05:25  morten
%  Make \chk_new_cs:N bulletproof
%
%  Revision 1.37  2006/07/30 14:37:47  morten
%  Added \str_if_eq_var_p:nf
%
%  Revision 1.36  2006/07/23 13:43:21  morten
%  Removed extra occurrence of \str_if_eq_p:nn
%
%  Revision 1.35  2006/07/17 22:25:31  morten
%  Fix logic of \chk_new_cs:N and rename \cs_if_eq_p:NN to
%  \cs_if_eq_name_p:NN because it is not their meaning but their names
%  which are compared.
%
%  Revision 1.34  2006/07/08 22:46:17  morten
%  Rearranged some code and added complete set of \use_none:<args>
%  functions.
%
%  Revision 1.33  2006/07/06 14:53:56  morten
%  Added \def:NNn functions, reorganized code and much much more.
%
%  Revision 1.32  2006/06/03 17:14:43  morten
%  Added more \use_arg_... functions.
%
%  Revision 1.31  2006/03/20 18:26:33  braams
%  Updated the copyright notice (2006) and demoted all implementation
%  sections to subsections and so on to clean up the toc for source3.tex
%
%  Revision 1.30  2006/01/22 19:35:46  morten
%  Changed \use_none:nnn to gobble all tokens in one go and added
%  \use_arg_i_ii:nn for l3keyval.
%
%  Revision 1.29  2006/01/16 09:47:49  morten
%  Fixed deleted brace
%
%  Revision 1.28  2006/01/15 19:08:28  morten
%  Fixed typos in some \if names
%
%  Revision 1.27  2006/01/04 01:29:36  morten
%  Changed "robust" to "protected".
%
%  Revision 1.26  2005/12/27 09:53:55  morten
%  Fixed some names, added a few more \use_... functions, changed RCS
%  information retrieval
%
%  Revision 1.25  2005/12/23 11:16:15  morten
%  *** empty log message ***
%
%  Revision 1.24  2005/12/02 15:21:55  morten
%  Fixed minor typo
%
%  Revision 1.23  2005/11/20 07:10:03  morten
%  Added a few missing functions. New function type
%  \def_test_function:npn to help define TF-type functions.
%
%  Revision 1.22  2005/11/08 04:16:17  morten
%  Fixed typo.
%
%  Revision 1.21  2005/10/27 22:35:13  morten
%  Cleaned up a bit and moved some code to l3num.
%
%  Revision 1.20  2005/07/12 16:10:24  morten
%  Allow \cs_if_eq_p:NN to accept \par. (Yes, real life experience)
%
%  Revision 1.19  2005/04/06 21:34:55  morten
%  Added more functions. Moved \engine_aleph:TF from l3prg
%
%  Revision 1.18  2005/03/25 23:52:24  braams
%  The definition of \next needs to be seen at initex time,
%  not at typesetting time
%
%  Revision 1.17  2005/03/22 23:28:46  morten
%  Moved some code to l3prg, removed obsolete code, cleaned up documentation.
%
%  Revision 1.16  2005/03/16 22:31:44  braams
%  Changed definition of \c_zero and \c_minus_one
%
%  Revision 1.15  2005/03/15 23:16:27  braams
%  Some tweaking with the docstrip guards to get the documentation to
%  come out right.
%
%  Revision 1.14  2005/03/15 22:50:00  braams
%  Made loadable by initex as the second file
%
%  Revision 1.13  2005/03/11 21:32:21  braams
%  Fixed the use of RCS information;
%  moved \RequirePackage after \StopEventually
%
