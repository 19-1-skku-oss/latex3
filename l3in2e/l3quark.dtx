% \iffalse
%% File: l3quark.dtx Copyright (C) 1990-1998 LaTeX3 project
%
%<*dtx>
          \ProvidesFile{l3quark.dtx}
%</dtx>
%<package>\NeedsTeXFormat{LaTeX2e}
%<package>\ProvidesPackage{l3quark}
%<driver> \ProvidesFile{l3quark.drv}
% \fi
%         \ProvidesFile{l3quark.dtx}
          [1998/04/20 v1.0d L3 Experimental Quark Commands]
%
% \iffalse
%<*driver>
\documentclass{l3doc}

\begin{document}
\DocInput{l3quark.dtx}
\end{document}
%</driver>
% \fi
%
%
% \GetFileInfo{l3quark.dtx}
% \title{The \textsf{l3quark} package\thanks{This file
%         has version number \fileversion, last
%         revised \filedate.}\\
% ``Quarks''}
% \author{\Team}
% \date{\filedate}
% \maketitle
%
%
% \section{Quarks}
%^^A \label{sec:quarks}
% 
%    
% A special type of constants in \LaTeX3 are `quarks'. These are control
% sequences that expand to themselves and should therefore NEVER be
% executed directly in the code. This would result in an endless loop!
% 
% They are meant to be used as delimiter is weird functions (for example
% as the stop token (i.e., "\q_stop"). They also permit the following
% ingenious trick: when you pick up a token in a temporary, and you want
% to know whether you have picked up a particular quark, all you have to
% do is compare the temporary to the quark using "\if_meaning:NN". A
% set of special quark testing functions is set up below.
% 
% By convention all constants of type quark start out with "\q_".
% 
% \subsection{Functions}
% 
% \begin{function}{%
%                  \quark_new:N |
% }
% \begin{syntax}
%    "\quark_new:N"   <quark>
% \end{syntax}
% Defines <quark> to be a new constant of type "quark".
% \end{function}
% 
% \begin{function}{%
%                  \quark_if_no_value:nTF |
%                  \quark_if_no_value:nF  |
%                  \quark_if_no_value:nT  |
%                  \quark_if_no_value:NTF |
%                  \quark_if_no_value:NF  |
% }
% \begin{syntax}
%    "\quark_if_no_value:nTF" "{" <token list> "}"
%                          "   {"<true code>"}{"<false code>"}"
%    "\quark_if_no_value:NTF" <tlp>
%                          "   {"<true code>"}{"<false code>"}"
% \end{syntax}
% This tests whether or not <token list> contains only the quark
% "\q_no_value".
% 
% If <token list> to be tested is stored in a token list pointer use
% "\quark_if_no_value:NTF", or "\quark_if_no_value:NF" or check the
% value directly with "\if_meaning:NN". All those cases are faster then
% "\quark_if_no_value:nTF" so should be preferred.
% 
% \begin{texnote}
% But  be aware of the fact that "\if_meaning:NN" can result in an
% overflow of \TeX{}'s parameter stack since it leaves the corresponding
% "\fi:" on the input until the whole replacement text is processed. It
% is therefore better in recursions to use "\quark_if_no_value:NTF" as
% it will remove the conditional prior to processing the "T" or "F" case
% and so allows tail-recursion.
% \end{texnote}
% \end{function}
% 
% \subsection{Constants}
% 
% \begin{variable}{\q_no_value} The canonical `missing value quark'
% that is returned by certain functions to denote that a requested value
% is not found in the data structure.
% \end{variable}
% 
% \begin{variable}{\q_stop}
% This constant is used as a a marker in parameter text. This allows a
% scanning function to find the end of some input string.
% \end{variable}
% 
% \begin{variable}{\q_nil}
% This constant represent the nil pointer in pointer structures.
% \end{variable}
% 
% \section{Implementation}
%
%    We start by ensuring that the required packages are loaded.
%    \begin{macrocode}
%<package>\RequirePackage{l3tlp}
%<*package>
%    \end{macrocode}
%    
% \begin{macro}{\quark_new:N}
%    Allocate a new quark.
%    \begin{macrocode}
\def_new:Npn \quark_new:N #1{\tlp_new:Nn #1{#1}}
%    \end{macrocode}
% \end{macro}
%
% \begin{macro}{\q_stop}
% \begin{macro}{\q_no_value}
% \begin{macro}{\q_nil}
%    |\q_stop| is often used as a marker in parameter text,
%    |\q_no_value| is the canonical missing value, and |\q_nil|
%    represents a nil pointer in some data structures.
%    \begin{macrocode}
\quark_new:N \q_stop
\quark_new:N \q_no_value
\quark_new:N \q_nil
%    \end{macrocode}
% \end{macro}
% \end{macro}
% \end{macro}
%
% \begin{macro}{\q_error}
% \begin{macro}{\q_mark}
%    We need two additional quarks.  |\q_error| delimits the end of
%    the computation for purposes of error recovery.  |\q_mark| is
%    used in parameter text when we need a scanning boundary that is
%    distinct from |\q_stop|.
%    \begin{macrocode}
\quark_new:N\q_error
\quark_new:N\q_mark
%    \end{macrocode}
% \end{macro}
% \end{macro}
%
%
% \begin{macro}{\quark_if_no_value:NTF}
% \begin{macro}{\quark_if_no_value:NF}
% \begin{macro}{\quark_if_no_value:nTF}
% \begin{macro}{\quark_if_no_value:nT}
% \begin{macro}{\quark_if_no_value:nF}
%    Here we test if we found a special quark as the first argument.
%    The argument might contain an arbitrary list of tokens, therefore
%    we have to wrap it up in a token list pointer.
%    \begin{macrocode}
\def_new:Npn \quark_if_no_value:NTF #1{
%    \end{macrocode}
%    We better start with |\q_no_value| as the first argument since
%    the whole thing may otherwise loop if |#1| is wrongly given
%    a string like |aabc| instead of a single token.\footnote{It may
%    still loop in special circumstances however!}
%    \begin{macrocode}
     \if_meaning:NN\q_no_value#1
          \exp_after:NN\use_choice_i:nn
     \else: \exp_after:NN\use_choice_ii:nn \fi:}
%    \end{macrocode}
%    It would be possible to speed up the following commands by
%    providing individual implementations similar to the one above.
%    Should perhaps be done if they are used often!
%    \begin{macrocode}
\def_new:Npn \quark_if_no_value:NF #1{\quark_if_no_value:NTF {#1}\use_noop:}
\def_new:Npn \quark_if_no_value:nTF #1{\tlp_gset:Nn \g_testa_tlp {#1}
    \quark_if_no_value:NTF\g_testa_tlp}
\def_new:Npn \quark_if_no_value:nF #1{\quark_if_no_value:nTF {#1}\use_noop:}
\def_new:Npn \quark_if_no_value:nT #1#2{\quark_if_no_value:nTF {#1}
        {#2}\use_noop:}
%    \end{macrocode}
% \end{macro}
% \end{macro}
% \end{macro}
% \end{macro}
% \end{macro}
%
% \begin{macro}{\quark_if_nil:NTF}
%    A function to check for the presence of |\q_nil|.
%    \begin{macrocode}
\def_new:Npn\quark_if_nil:NTF#1{
  \if_meaning:NN#1\q_nil
    \exp_after:NN\use_choice_i:nn
  \else:
    \exp_after:NN\use_choice_ii:nn\fi:}
%    \end{macrocode}
% \end{macro}
%
%    Show token usage:
%    \begin{macrocode}
%</package>
%<*showmemory>
\showMemUsage
%</showmemory>
%    \end{macrocode}
%
%
