% \iffalse
%% File: l3expan.dtx Copyright (C) 1990-1998 LaTeX3 project
%
%<*dtx>
          \ProvidesFile{l3expan.dtx}
%</dtx>
%<package>\NeedsTeXFormat{LaTeX2e}
%<package>\ProvidesPackage{l3expan}
%<driver> \ProvidesFile{l3expan.drv}
% \fi
%         \ProvidesFile{l3expan.dtx}
          [1998/04/12 v1.0c L3 Experimental Argument Expansion module]
%
% \iffalse
%<*driver>
\documentclass{l3doc}

\begin{document}
\DocInput{l3expan.dtx}
\end{document}
%</driver>
% \fi
%
%<package>\RequirePackage{l3chk,l3tlp}\CodeStart
%
% \GetFileInfo{l3expan.dtx}
% \title{The \textsf{l3expan} package\thanks{This file
%         has version number \fileversion, last
%         revised \filedate.}\\
% Controlling Expansion of Command Arguments}
% \author{\Team}
% \date{\filedate}
% \maketitle
%
%
%
% \section{\LaTeX3 functions}
% 
% All \LaTeX3 functions contain a colon in their
% name. Characters following the colon are used to denote the number and
% the ``type'' of arguments that the function takes. An uppercase "N" is
% used to denote an argument that consists of a single token and a
% lowercase "n" is used when the argument can consist of several tokens
% surrounded by braces. In case of "n" arguments that consist of a
% single token the surrounding braces can be omitted in nearly all
% situations---functions that force the use of braces even for single
% token arguments are explicitly mentioned. For example, "\seq_gpush:Nn"
% is a function that takes two arguments, the first is a single token
% (the sequence) and the second may consist of several tokens surrounded
% by braces.
% 
% This concept of argument specification makes it easy to read the code
% and should be followed when defining new functions.
% 
% \subsection{Expanding arguments of functions}
% 
% Within code it is often necessary to expand or partially expand
% arguments before passing it on to some function. For example, if the
% token list pointer "\l_tmpa_tlp" contains the current file
% that should be pushed onto some stack, we can not write
% \begin{quote}
%   "\seq_gpush:Nn"                           \\
%   "   \g_file_name_stack"                   \\
%   "   \l_tmpa_tlp"
% \end{quote}
% since this would put the token "\l_tmpa_tlp" and not its
% contents on the stack. Instead a suitable number of "\exp_after:NN"
% would be necessary (together with extra braces) to change the order of
% execution, i.e.
% \begin{quote}
%   "\exp_after:NN"                              \\
%   "   \seq_gpush:Nn"                           \\
%   "\exp_after:NN"                              \\
%   "      \g_file_name_stack"                   \\
%   "\exp_after:NN"                              \\
%   "      {\l_tmpa_tlp}"
% \end{quote}
% 
% The above example is probably the simplest case but is already shows
% how the code changes to something difficult to understand. Therefore
% \LaTeX3 provides the programmer with a general scheme that keeps the
% code compact and easy to understand.
% To denote that some argument to a function needs special treatment one
% just uses different letters in the argument part of the function to
% mark the desired behavior. In the above example one would write
% \begin{quote}
%   "\seq_gpush:No"                           \\
%   "   \g_file_name_stack"                   \\
%   "   \l_tmpa_tlp"
% \end{quote}
% to achieve the desired effect. Here the "o" stands for expand this
% (the second) argument once before passing it to the function.
% 
% The following letters can be used to denote special treatment of
% arguments before passing it to the basic function:
% \begin{description}
%  \item[o] One time expanded token or token-list. In the latter case,
%  effectively only the first token in the list gets expanded. Since
%  the expansion might result in more than one token, the result is
%  surrounded for further processing with braces.
% 
%  \item[x] Fully expanded token or token-list. Like "o" but the
%  argument is expanded using "\def:Npx" before it is passed on. This means
% that expansion takes place until only unexpandable tokens are left.
% 
%  \item[N,O,X] Like "n", "o", "x" but the argument must be a single
%  token without any braces around it.
% 
%  \item[c] A character string or a token-list that ultimately expands
% to characters. This string (after expansion) is used to construct a
% command name that is eventually passed on.
% 
%  \item[C] A character string or a token-list that ultimately expands
% to characters. From this string (after expansion) a command name is
% constructed and then this command name is expanded once (like "o").
% The result of this is eventually passed on. In other words
% \begin{quote}
%   "\seq_gpush:NC"                           \\
%   "   \g_file_name_stack"                   \\
%   "   {l_tmpa_tlp}"
% \end{quote}
% Has the same effect as the example above.
% 
% 
% 
% \end{description}
% 
% Due to memory constraints not all possible variations are implemented
% for every base function. Instead only those that are used within the
% \LaTeX3 kernel or otherwise seem of general interest are implemented.
% Consult the module description to find out which functions are
% actually defined. The next section explains how to define missing
% variants.
% 
% \subsection{Defining new variants}
% 
% The definition of variant forms for base functions may be necessary
% when writing new functions or when applying a kernel function in a
% situation that we haven't thought of before.
% 
% Internally preprocessing of arguments is done with functions from the
% "\exp_" module.  They all look alike, an example would be
% "\exp_args:NNo". This function has three arguments, the first and the
% second are a single tokens  the third argument gets
% expanded once. If "\seq_gpush:No" wouldn't be defined the example
% above could be coded in the following way:
% \begin{quote}
%   "\exp_args:NNo\seq_gpush:Nn"              \\
%   "   \g_file_name_stack"                   \\
%   "   \l_tmpa_tlp"
% \end{quote}
% In other words, the first argument to "\exp_args:NNo" is the base
% function and the other arguments are preprocessed and then passed to
% this base function. In the example the first argument to the base
% function should be a single token which is left unchanged while the
% second argument is expanded once. From this example we can also see
% how the variants are defined. They just expand into the appropriate
% "\exp_" function followed by the desired base function, e.g.
% \begin{quote}
%   "\def_new:Npn\seq_gpush:No{\exp_args:NNo\seq_gpush:Nn}"
% \end{quote}
% Providing variants in this way in style files is uncritical as the
% "\def_new:Npn" function will silently accept definitions whenever the
% new definition is identical to an already given one. Therefore adding
% such definition to later releases of the kernel will not make such
% style files obsolete.
% 
% The available internal functions for argument expansion come in to
% flavours, some of them are faster then others. Therefore it is usually
% best to follow the following guidelines when defining new functions
% that are supposed to come with variant forms:
% \begin{itemize}
% \item
%   Arguments that might need expansion should come first in the list of
% arguments to make processing faster.
% \item
%   Arguments that should consist of single tokens should come first.
% \item
%   Arguments that need full expansion (i.e., are denoted with "x")
% should be avoided if possible as they can not be processed very fast.
% \item
%   In general "n",  "x", and "o" (if not in the last position) will
% need special processing which is not fast and not expandable, i.e.,
% functions of this type may not work correctly in arguments that are
% itself subject to "x" expansion. Therefore it is best to use the
% ``expandable'' functions (i.e., those that contain only "c", "N", "O"
% or "o" in the last position) whenever possible.
% \end{itemize}
% 
% \subsection{Manipulating the first argument}
% 
% \begin{function}{%
%                  \exp_args:No
% }
% \begin{syntax}
%   " \exp_args:No" <funct> <arg1> <arg2> "..."
% \end{syntax}
% The first argument of <funct> (i.e., <arg1>) is expanded once, the
% result is surrounded by braces and passed to <funct>. <funct> may have
% more than one argument---all others are passed unchanged.
% \end{function}
% 
% \begin{function}{%
%                  \exp_args:Nc
% }
% \begin{syntax}
%   " \exp_args:Nc" <funct> <arg1> <arg2> "..."
% \end{syntax}
% The first argument of <funct> (i.e., <arg1>) is expanded until only
% characters remain. (An internal error occurs if something else is the
% result of this expansion.) Then the result is turned into a control
% sequence and passed to <funct> as the first argument. <funct> may have
% more than one argument---all others are passed unchanged.
% \end{function}
% 
% \begin{function}{%
%                  \exp_args:NC
% }
% \begin{syntax}
%   " \exp_args:Nc" <funct> <arg1> <arg2> "..."
% \end{syntax}
% The first argument of <funct> (i.e., <arg1>) is expanded until only
% characters remain. (An internal error occurs if something else is the
% result of this expansion.) Then the result is turned into a control
% sequence which is then expanded once more. The result of this is then
% passed to <funct> as the first argument. <funct> may have more than
% one argument---all others are passed unchanged.
% \end{function}
% 
% \begin{function}{%
%                  \exp_args:Nx
% }
% \begin{syntax}
%   " \exp_args:Nx" <funct> <arg1> <arg2> "..."
% \end{syntax}
% The first argument of <funct> (i.e., <arg1>) is fully expanded until
% only unexpandable tokens remain, the result is surrounded by braces
% and passed to <funct>. <funct> may have more than one argument---all
% others are passed unchanged.
% As mentioned before, this type of function is relatively slow.
% \end{function}
% 
% \subsection{Manipulating two arguments}
% 
% \begin{function}{%
%                  \exp_args:Nno |
%                  \exp_args:NNx |
%                  \exp_args:Nnx |
%                  \exp_args:Noo |
%                  \exp_args:Nox |
%                  \exp_args:Nxo |
%                  \exp_args:Nxx |
% }
% \begin{syntax}
%   "\exp_args:Nno" <funct> <arg1> <arg2> "..."
% \end{syntax}
% The above functions all manipulate the first two arguments of <funct>.
% They are all slow and non-expandable.
% \end{function}
% 
% \begin{function}{%
%                  \exp_args:NNo |
%                  \exp_args:NNc |
%                  \exp_args:NOo |
%                  \exp_args:NOc |
%                  \exp_args:Nco |
%                  \exp_args:Ncc |
% }
% \begin{syntax}
%   "\exp_args:NNo" <funct> <arg1> <arg2> "..."
% \end{syntax}
% These are the fast and expandable functions for the first two arguments.
% \end{function}
% 
% \subsection{Manipulating three arguments}
% 
% So far not all possible functions are provided and even the selection
% below may be reduced in the future as far as the non-expandable
% functions are concerned.
% 
% \begin{function}{%
%                  \exp_args:Nnno |
%                  \exp_args:Nnnx |
%                  \exp_args:Noox |
%                  \exp_args:Nnox |
%                  \exp_args:Ncnx |
% }
% \begin{syntax}
%   "\exp_args:Nnno" <funct> <arg1> <arg2> <arg3> "..."
% \end{syntax}
% All the above functions are non-expandable.
% \end{function}
% 
% \begin{function}{%
%                  \exp_args:NNOo |
%                  \exp_args:NOOo |
%                  \exp_args:Nccc |
% }
% \begin{syntax}
%   "\exp_args:NNOo" <funct> <arg1> <arg2> <arg3> "..."
% \end{syntax}
% These are the fast and expandable functions for the first three
% arguments.
% \end{function}
% 
% \subsection{Internal functions and variables}
% 
% \begin{function}{\exp_after:NN}
% \begin{syntax}
%   "\exp_after:NN" <token1> <token2>
% \end{syntax}
% This will expand <token2> once before processing <token1>. This is
% similar to "\exp_args:No" except that no braces are put around the
% result of expanding <token2>.
% \begin{texnote}
% This is the primitive \tn{expandafter} which was renamed to fit into
% the naming conventions of \LaTeX3.
% \end{texnote}
% \end{function}
% 
% \begin{function}{\exp_not:N |
%                  \exp_not:c }
% \begin{syntax}
%   "\exp_not:N" <token>
% \end{syntax}
% This function will prohibit the expansion of <token> in situation
% where <token> would otherwise be replaced by it definition, e.g.,
% inside an argument that is handled by the "x" convention.
% \begin{texnote}
% "\exp_not:N" is the primitive \tn{noexpand} renamed.
% \end{texnote}
% \end{function}
% 
% 
% \begin{function}{\exp_not:o}
% \begin{syntax}
%   "\exp_not:o" <token>
% \end{syntax}
% 
% The name of this command is a lie. Perhaps it should be called
% ``"exp_perhaps_once"''. What it actually does is, it expands <token> and
% then issues an "\exp_not:N" to prohibit further expansion of the
% first token in the replacement text of <token>. This means that if the
% replacement text of <token> consists of more than one token all
% further tokens are still subject to full expansion. The command is, for
% example, useful when generating a new cs name inside a "x" situation
% that shouldn't be expanded further, e.g., "\exp_not:o" "\cs:w" "foo"
% "\cs_end:".
% \begin{texnote}
% This command has no equivalent.
% \end{texnote}
% \end{function}
% 
% \begin{variable}{\l_exp_tlp |  \l_exp_toks}
% The "\exp_" module has its private variables to temporarily store
% results of the argument expansion. This is done to avoid interference
% with other functions using temporary variables.
% \end{variable}
%
% \StopEventually{}
%
% \section {Argument Expansion}
%
%  \subsection{General expansion}
%
% In this section a general mechanism for defining functions to handle
% argument handling is defined.
% These general expansion functions are expandable unless |x| is used.
% (Any version of |x| is going to have to use
% one of the \LaTeX3\ names for |\def:Npx| at some point, and so is never
% going to be expandable.)
%
% In a later section some common cases are coded by a more direct
% method, typically using calls to |\exp_after:NN|.
%
%    \begin{macrocode}
%<*package>
%    \end{macrocode}
%
% \begin{macro}{\l_exp_tlp}
%    We need a scratch token list pointer.
%    \begin{macrocode}
\tlp_new:Nn\l_exp_tlp{}
%    \end{macrocode}
% \end{macro}
%
%  This code uses internal functions with names that start with |\::|
%  to perform the expansions.
%  \begin{macro}{\exp_arg_next:nnn}
%    This is basically the same function as |\Dexp_arg_next:nnn|.
%    \begin{macrocode}
\def_new:Npn\exp_arg_next:nnn#1#2#3{%
      #2\:::{#3#1}}
%    \end{macrocode}
% \end{macro}
%
%  \begin{macro}{\::n}
%    This function is used to skip an argument that doesn't need to
%    be expanded.
%    \begin{macrocode}
\def_new:Npn\::n#1\:::#2#3{%
                  #1\:::{#2{#3}}}
%    \end{macrocode}
% \end{macro}
%
%  \begin{macro}{\::N}
%    This function is used to skip an argument that consists of a
%    single token and doesn't need to be expanded.
%^^A was \let_new:NN\::N\::n (changed to match \;N above.
%    \begin{macrocode}
\def_new:Npn\::N#1\:::#2#3{%
                  #1\:::{#2#3}}
%    \end{macrocode}
% \end{macro}
%
%  \begin{macro}{\::c}
%    This function is used to skip an argument that is turned into
%    as control sequence without expansion.
%    \begin{macrocode}
\def_new:Npn\::c#1\:::#2#3{%
  \exp_after:NN\exp_arg_next:nnn\cs:w #3\cs_end:{#1}{#2}}
%    \end{macrocode}
% \end{macro}
%
%  \begin{macro}{\::o}
%    This function is used to expand an argument once.
%    \begin{macrocode}
\def_new:Npn\::o#1\:::#2#3{%
  \exp_after:NN\exp_arg_next:nnn\exp_after:NN{\exp_after:NN{#3}}{#1}{#2}}
%    \end{macrocode}
% \end{macro}
%
%  \begin{macro}{\::x}
%    This function is used to expand an argument fully.
%    
%    \begin{macrocode}
\def_new:Npn\::x#1\:::#2#3{%
  \tlp_set:Nx\l_exp_tlp{{{#3}}}%
  \exp_after:NN\exp_arg_next:nnn\l_exp_tlp{#1}{#2}}
%    \end{macrocode}
% \end{macro}
%
%  \begin{macro}{\:::}
%    Just another name for the identity function.
%    \begin{macrocode}
\def_new:Npn\:::#1{#1}
%    \end{macrocode}
% \end{macro}
%
%  \begin{macro}{\::C}
%    This function creates a control sequence out of |#3| and expands
%    that once before passing it on to |\exp_arg_next:nnn|.
%    \begin{macrocode}
\def_new:Npn\::C#1\:::#2#3{%
  \exp_after:NN\exp_C_aux:nnn\cs:w #3\cs_end:{#1}{#2}}
%    \end{macrocode}
% \end{macro}
%
%  \begin{macro}{\exp_C_aux:nnn}
%    A helper function for |\::C| wich expands its argument before
%    passing it on to |\exp_arg_next:nnn|.
%    \begin{macrocode}
\def_new:Npn\exp_C_aux:nnn #1
 {
  \exp_after:NN
  \exp_arg_next:nnn
  \exp_after:NN
     {
  \exp_after:NN
      {#1}
     }
 }
%    \end{macrocode}
% \end{macro}
%
%  \begin{macro}{\exp_args:NC}
%  \begin{macro}{\exp_args:Nccx}
%  \begin{macro}{\exp_args:Ncnx}
%  \begin{macro}{\exp_args:NNno}
%  \begin{macro}{\exp_args:Nnno}
%  \begin{macro}{\exp_args:Nnnx}
%  \begin{macro}{\exp_args:Nno}
%  \begin{macro}{\exp_args:Nnox}
%  \begin{macro}{\exp_args:NNx}
%  \begin{macro}{\exp_args:Nnx}
%  \begin{macro}{\exp_args:Noo}
%  \begin{macro}{\exp_args:Noox}
%  \begin{macro}{\exp_args:Nox}
%  \begin{macro}{\exp_args:Nx}
%  \begin{macro}{\exp_args:Nxx}
%  \begin{macro}{\exp_args:Nxx}
%    Here are the actual function definitions, using the helper functions
%    above.
%    \begin{macrocode}
\def:Npn \exp_args:NC {\::C\:::}
%\def:Npn \exp_args:Nc {\::c\:::}
%\def:Npn \exp_args:Ncc {\::c\::c\:::}
%\def:Npn \exp_args:Nccc {\::c\::c\::c\:::}
\def:Npn \exp_args:Nccx {\::c\::c\::x\:::}
\def:Npn \exp_args:Ncnx {\::c\::n\::x\:::}
%\def:Npn \exp_args:Nco {\::c\::o\:::}
%\def:Npn \exp_args:NNc {\::N\::c\:::}
%\def:Npn \exp_args:NNNo {\::N\::N\::o\:::}
\def:Npn \exp_args:NNno {\::N\::n\::o\:::}
\def:Npn \exp_args:Nnno {\::n\::n\::o\:::}
\def:Npn \exp_args:Nnnx {\::n\::n\::x\:::}
%\def:Npn \exp_args:NNo {\::N\::o\::\:::}
\def:Npn \exp_args:Nno {\::n\::o\:::}
%\def:Npn \exp_args:NNOo {\::N\::O\::o\:::}
\def:Npn \exp_args:Nnox {\::n\::o\::x\:::}
\def:Npn \exp_args:NNx {\::N\::x\:::}
\def:Npn \exp_args:Nnx {\::n\::x\:::}
%\def:Npn \exp_args:No {\::o\:::}
%\def:Npn \exp_args:NOc {\::O\::c\:::}
%\def:Npn \exp_args:NOo {\::O\::o\:::}
\def:Npn \exp_args:Noo {\::o\::o\:::}
%\def:Npn \exp_args:NOOo {\::O\::O\::o\:::}
\def:Npn \exp_args:Noox {\::o\::o\::x\:::}
\def:Npn \exp_args:Nox {\::o\::x\:::}
\def:Npn \exp_args:Nx {\::x\:::}
\def:Npn \exp_args:Nxo {\::x\::o\:::}
\def:Npn \exp_args:Nxx {\::x\::x\:::}
%    \end{macrocode}
%  \end{macro}
%  \end{macro}
%  \end{macro}
%  \end{macro}
%  \end{macro}
%  \end{macro}
%  \end{macro}
%  \end{macro}
%  \end{macro}
%  \end{macro}
%  \end{macro}
%  \end{macro}
%  \end{macro}
%  \end{macro}
%  \end{macro}
%  \end{macro}
%
%
% \subsection{Preventing expansion}
%
%  \begin{macro}{\exp_not:N}
%    This is the \TeX\ primitive that was defined earlier.
%  \end{macro}
%
%  \begin{macro}{\exp_not:o}
%  \begin{macro}{\exp_not:c}
%    Two helper functions, which we can probably live without it.
%    \begin{macrocode}
\def_new:Npn\exp_not:o{\exp_after:NN\exp_not:N}
\def_new:Npn\exp_not:c#1{\exp_after:NN\exp_not:N\cs:w#1\cs_end:}
%    \end{macrocode}
%  \end{macro}
%  \end{macro}
%
% \subsection{Single token expansion}
%
% Expansion for arguments that are single tokens is done with the
% functions below. I first thought of using a different module name
% but then I saw that this wouldn't do since I could then never
% determine for, say, |\seq_put:no| whether this means single, or
% general expansion. Therefore I decided to use uppercase `O' for
% single expansion.
%
% One of the most important features of these functions is that they
% are fully expandable and therefore allow to prefix them with
% |\pref_global:D| for example. This together with the fact that the
% above concept is much slower in general means that we should convert
% whenever possible and perhaps remove all remaining occurences by
% hand-encoding in the end.
%
% \begin{macro}{\exp_args:No}
% \begin{macro}{\exp_args:NOo}
% \begin{macro}{\exp_args:NNo}
% \begin{macro}{\exp_args:NOOo}
% \begin{macro}{\exp_args:NNOo}
% \begin{macro}{\exp_args:NNNo}
%    This looks somewhat horrible but it runs well with the other
%    syntax. It is important to see that these functions really need
%    single tokens as arguments whenever capital letters are used.
%    \begin{macrocode}
\def_new:Npn \exp_args:No #1#2{\exp_after:NN#1\exp_after:NN{#2}}
\def_new:Npn \exp_args:NOo #1#2#3{\exp_after:NN\exp_args:No \exp_after:NN#1
  \exp_after:NN#2\exp_after:NN{#3}}
\def_new:Npn \exp_args:NOOo #1#2#3#4{\exp_after:NN\exp_args:NOo
  \exp_after:NN#1\exp_after:NN#2\exp_after:NN#3\exp_after:NN{#4}}
\def_new:Npn \exp_args:NNo #1#2#3{\exp_after:NN#1\exp_after:NN#2
  \exp_after:NN{#3}}
\def_new:Npn \exp_args:NNOo #1#2#3#4{\exp_after:NN\exp_args:NNo
  \exp_after:NN#1\exp_after:NN#2\exp_after:NN#3\exp_after:NN{#4}}
\def_new:Npn \exp_args:NNNo #1#2#3#4{\exp_after:NN#1\exp_after:NN#2
  \exp_after:NN#3\exp_after:NN{#4}}
%    \end{macrocode}
% \end{macro}
% \end{macro}
% \end{macro}
% \end{macro}
% \end{macro}
% \end{macro}
%
%
% \begin{macro}{\exp_args:Nc}
% \begin{macro}{\exp_args:NNc}
% \begin{macro}{\exp_args:NOc}
% \begin{macro}{\exp_args:Ncc}
% \begin{macro}{\exp_args:Nccc}
%    Here are the functions that turn their argument into csnames but
%    are  expandable.
%    \begin{macrocode}
\def_new:Npn \exp_args:Nc #1#2{\exp_after:NN#1\cs:w#2\cs_end:}
\def_new:Npn \exp_args:NNc #1#2#3{\exp_after:NN#1\exp_after:NN#2
    \cs:w#3\cs_end:}
\def_new:Npn \exp_args:NOc#1#2#3{\exp_after:NN\exp_args:No\exp_after:NN
    #1\exp_after:NN#2\cs:w#3\cs_end:}
\def_new:Npn \exp_args:Ncc #1#2#3{\exp_after:NN#1
    \cs:w#2\exp_after:NN\cs_end:\cs:w#3\cs_end:}
\def_new:Npn \exp_args:Nccc #1#2#3#4{\exp_after:NN#1
    \cs:w#2\exp_after:NN\cs_end:\cs:w#3\exp_after:NN
      \cs_end:\cs:w #4\cs_end:}
%    \end{macrocode}
% \end{macro}
% \end{macro}
% \end{macro}
% \end{macro}
% \end{macro}
%
%
%  \begin{macro}{\exp_args:Nco}
%    If we force that the third argument
%    always has braces, we could implement this function
%    with less tokens and only two arguments.
%    \begin{macrocode}
\def_new:Npn \exp_args:Nco #1#2#3{\exp_after:NN#1\cs:w#2\exp_after:NN
     \cs_end:\exp_after:NN{#3}}
%    \end{macrocode}
%  \end{macro}
%
%
%  \begin{macro}{\exp_def_form:nnn}
%    This command is a recent addition which was actually added 
%    when we wrote the article for TUGboat (while most of the other
%    code goes way back to 1993).
%    \begin{macrocode}
\def:Npn\exp_def_form:nnn#1#2#3{
   \exp_after:NN
   \def:Npn
     \cs:w 
        #1:#3 
       \exp_after:NN
     \cs_end:
     \exp_after:NN
       {
        \cs:w 
           exp_args:N#3 
          \exp_after:NN
        \cs_end:
        \cs:w 
           #1:#2
        \cs_end:
       }
%    \end{macrocode}
%    We also have to test if |exp_arg:N#3| is already defined 
%    and if not define it via the
%    |\::| commands using the chars in |#3|
%    \begin{macrocode}
    \cs_free:cT
          {exp_args:N#3}
          {\def:cpx {exp_args:N#3} 
                    {\exp_args_form_x:w #3 :}
          }
}
%    \end{macrocode}
%  \end{macro}
%
%
%  \begin{macro}{\exp_args_form_x:w}
%    This command graps char by char outputting |\::#1| (not expanded
%    further) until we see a |:|. That colon is in fact also turned into
%    |\:::| so that the required structure for |\exp_args...| commands
%    is correctly terminated.
%    \begin{macrocode}
\def_new:Npn\exp_args_form_x:w #1 {
  \exp_after:NN \exp_not:N \cs:w ::#1 \cs_end:
  \if_meaning:NN #1 :
  \else:
    \exp_after:NN\exp_args_form_x:w
  \fi:}
%    \end{macrocode}
%  \end{macro}

%    Show token usage:
%    \begin{macrocode}
%</package>
%<*showmemory>
\showMemUsage
%</showmemory>
%    \end{macrocode}
%
%



