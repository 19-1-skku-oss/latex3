% \iffalse
%% File: l3expan.dtx Copyright (C) 1990-2009 LaTeX3 project
%%
%% It may be distributed and/or modified under the conditions of the
%% LaTeX Project Public License (LPPL), either version 1.3c of this
%% license or (at your option) any later version.  The latest version
%% of this license is in the file
%%
%%    http://www.latex-project.org/lppl.txt
%%
%% This file is part of the ``expl3 bundle'' (The Work in LPPL)
%% and all files in that bundle must be distributed together.
%%
%% The released version of this bundle is available from CTAN.
%%
%% -----------------------------------------------------------------------
%%
%% The development version of the bundle can be found at
%%
%%    http://www.latex-project.org/svnroot/experimental/trunk/
%%
%% for those people who are interested.
%%
%%%%%%%%%%%
%% NOTE: %%
%%%%%%%%%%%
%%
%%   Snapshots taken from the repository represent work in progress and may
%%   not work or may contain conflicting material!  We therefore ask
%%   people _not_ to put them into distributions, archives, etc. without
%%   prior consultation with the LaTeX Project Team.
%%
%% -----------------------------------------------------------------------
%
%<*driver|package>
\RequirePackage{l3names}
%</driver|package>
%\fi
\GetIdInfo$Id$
       {L3 Experimental Argument Expansion module}
%\iffalse
%<*driver>
%\fi
\ProvidesFile{\filename.\filenameext}
  [\filedate\space v\fileversion\space\filedescription]
%\iffalse
\documentclass[full]{l3doc}
\begin{document}
\DocInput{l3expan.dtx}
\end{document}
%</driver>
% \fi
%
%
% \title{The \pkg{l3expan} package\thanks{This file
%         has version number \fileversion, last
%         revised \filedate.}\\
% Controlling Expansion of Function Arguments}
% \author{\Team}
% \date{\filedate}
% \maketitle
%
% \begin{documentation}
%
% \section{Brief overview}
%
% The functions in this module all have prefix |exp|. 
%
% Not all possible variations are implemented for every base
% function. Instead only those that are used within the \LaTeX3 kernel
% or otherwise seem to be of general interest are implemented.
% Consult the module description to find out which functions are
% actually defined. The next section explains how to define missing
% variants.
% 
%
% \section{Defining new variants}
%
% The definition of variant forms for base functions may be necessary
% when writing new functions or when applying a kernel function in a
% situation that we haven't thought of before.
%
% Internally preprocessing of arguments is done with functions from the
% "\exp_" module.  They all look alike, an example would be
% "\exp_args:NNo". This function has three arguments, the first and the
% second are a single tokens  the third argument gets
% expanded once. If "\seq_gpush:No" wouldn't be defined the example
% above could be coded in the following way:
% \begin{verbatim}
%   \exp_args:NNo\seq_gpush:Nn
%      \g_file_name_stack
%      \l_tmpa_tl
% \end{verbatim}
% In other words, the first argument to "\exp_args:NNo" is the base
% function and the other arguments are preprocessed and then passed to
% this base function. In the example the first argument to the base
% function should be a single token which is left unchanged while the
% second argument is expanded once. From this example we can also see
% how the variants are defined. They just expand into the appropriate
% "\exp_" function followed by the desired base function, e.g.
% \begin{quote}
%   "\cs_new_nopar:Npn\seq_gpush:No{\exp_args:NNo\seq_gpush:Nn}"
% \end{quote}
% Providing variants in this way in style files is uncritical as the
% "\cs_new_nopar:Npn" function will silently accept definitions whenever the
% new definition is identical to an already given one. Therefore adding
% such definition to later releases of the kernel will not make such
% style files obsolete.
%
% The steps above may be automated by using the function
% "\cs_generate_variant:Nn", described next.
%
% \subsection{Methods for defining variants}
%
% \begin{function}{\cs_generate_variant:Nn}
% \begin{syntax}
% "\cs_generate_variant:Nn" <function> <variant argument>
% "\cs_generate_variant:Nn " "\foo:Nn" "{c}"
% "\cs_generate_variant:Nn" "\foo:Nn" "{c,Nx,cx}"
% \end{syntax}
% This function greatly simplify the task of defining
% variantions on a base function with differing expansion control.  
% The first example is equivalent to writing
% \begin{verbatim}
%   \cs_set_nopar:Npn \foo:cn { \exp_args:Nc \foo:Nn }
% \end{verbatim}
% If used with a comma-separated list of variants, it does so for each
% variant form, i.e., the second example is equivalent to
% \begin{verbatim}
%   \cs_set_nopar:Npn \foo:cn { \exp_args:Nc \foo:Nn }
%   \cs_set_nopar:Npn \foo:Nx { \exp_args:NNx \foo:Nn }
%   \cs_set_nopar:Npn \foo:cx { \exp_args:Ncx \foo:Nn }
% \end{verbatim}
% \end{function}
%
% \paragraph{Internal functions} \mbox{}
%
% \begin{function}{\cs_generate_internal_variant:n}
% \begin{syntax}
% "\cs_generate_internal_variant:n" \Arg{args}
% \end{syntax}
% Defines the appropriate "\exp_args:N"<args> function, if necessary,
% to perform the expansion control specified by <args>.
% \end{function}
%
% \section{Introducing the variants}
%
% The available internal functions for argument expansion come in two
% flavours, some of them are faster then others. Therefore it is usually
% best to follow the following guidelines when defining new functions
% that are supposed to come with variant forms:
% \begin{itemize}
% \item
%   Arguments that might need expansion should come first in the list of
%   arguments to make processing faster.
% \item
%   Arguments that should consist of single tokens should come first.
% \item
%   Arguments that need full expansion (i.e., are denoted with "x")
%   should be avoided if possible as they can not be processed very fast.
% \item 
%   In general "n", "x", and "o" (if not in the last position) will
%   need special processing which is not fast and not expandable,
%   i.e., functions of this type may not work correctly in arguments
%   that are itself subject to "x" expansion. Therefore it is best to
%   use the ``expandable'' functions (i.e., those that contain only
%   "c", "N", "o" or "f" in the last position) whenever possible.
% \end{itemize}
%
% The |V| type returns the value of a register, which can be one of
% |tl|, |num|, |int|, |skip|, |dim|, |toks|, or built-in \TeX\
% registers. The |v| type is the same except it first creates a
% control sequence out of its argument before returning the
% value. This recent addition to the argument specifiers may shake
% things up a bit as most places where |o| is used will be replaced by
% |V|. The documentation you are currently reading will therefore
% require a fair bit of re-writing.
%
% In general, the programmer should not need to be concerned with
% expansion control. When simply using the content of a variable,
% functions with a "V" specifier should be used. For those referred to by
% (cs)name, the "v" specifier is available for the same purpose. Only when
% specific expansion steps are needed, such as when using delimited
% arguments, should the lower-level functions with "o" specifiers be employed.
%
% The |f| type is so special that it deserves an example.
% Let's pretend we want to set "\aaa" equal to the control sequence
% stemming from turning "b \l_tmpa_tl b" into a control
% sequence. Furthermore we want to store the execution of it in a
% \m{toks} register. In this example we assume "\l_tmpa_tl" contains
% the text string "lur". The straight forward approach is
% \begin{quote}
% "\toks_set:No \l_tmpa_toks {\cs_set_eq:Nc \aaa {b \l_tmpa_tl b}}"
% \end{quote}
% Unfortunately this only puts
% "\exp_args:NNc \cs_set_eq:NN \aaa {b \l_tmpa_tl b}" into "\l_tmpa_toks"
% and not "\cs_set_eq:NwN \aaa = \blurb" as we probably wanted. Using
% "\toks_set:Nx" is not an option as that will die horribly. Instead
% we can do a
% \begin{quote}
% "\toks_set:Nf \l_tmpa_toks {\cs_set_eq:Nc \aaa {b \l_tmpa_tl b}}"
% \end{quote}
% which puts the desired result in "\l_tmpa_toks". It requires
% "\toks_set:Nf" to be defined as
% \begin{quote}
% "\cs_set_nopar:Npn \toks_set:Nf {\exp_args:NNf \toks_set:Nn}"
% \end{quote}
% If you use this type of expansion in conditional processing then
% you should stick to using "TF" type functions only as it does not
% try to finish any "\if... \fi:" itself!
%
%
% \begin{function}{\exp_arg:x}
% \begin{syntax}
%   "\exp_arg:x" \Arg{arg}
% \end{syntax}
% <arg> is expanded fully using an "x" expansion.
%
% When pdf\TeX\ 1.50 arrives, it will contain a primitive for
% performing the equivalent of an "x" expansion after only one
% expansion and most importantly: as an expandable operation.
%
% However, we will not be using the expandable command "\expanded"
% until it is broadly available; note that as of \TeX~Live~2009,
% only Lua\TeX\ supports this primitive. Until further notice,
% "\exp_arg:x" will remain \emph{unexpandable}.
% \end{function}
%
%
%
% \section{Manipulating the first argument}
%
% \begin{function}{ \exp_args:No / (EXP) }
% \begin{syntax}
%   " \exp_args:No" <funct> <arg1> <arg2> "..."
% \end{syntax}
% The first argument of <funct> (i.e., <arg1>) is expanded once, the
% result is surrounded by braces and passed to <funct>. <funct> may have
% more than one argument---all others are passed unchanged.
% \end{function}
%
% \begin{function}{ \exp_args:Nc / (EXP) | \exp_args:cc / (EXP) }
% \begin{syntax}
%   " \exp_args:Nc" <funct> <arg1> <arg2> "..."
% \end{syntax}
% The first argument of <funct> (i.e., <arg1>) is expanded until only
% characters remain. (An internal error occurs if something else is the
% result of this expansion.) Then the result is turned into a control
% sequence and passed to <funct> as the first argument. <funct> may have
% more than one argument---all others are passed unchanged.
%
% In the ":cc" variant, the <funct> control sequence itself is constructed 
% (with the same process as described above) before <arg1> is turned into a
% control sequence and passed as its argument.
% \end{function}
%
% \begin{function}{ \exp_args:NV / (EXP) }
% \begin{syntax}
%   " \exp_args:NV" <funct> <register>
% \end{syntax}
% The first argument of <funct> (i.e., <register>) is expanded to its
% value. By value we mean a number stored in an |int| or |num|
% register, the length value of a |dim|, |skip| or |muskip| register, the
% contents of a |toks| register or the unexpanded contents of a |tl var.|
% register. The value is passed onto <funct> in braces.
% \end{function}
%
% \begin{function}{ \exp_args:Nv / (EXP) }
% \begin{syntax}
%   " \exp_args:Nv" <funct> \Arg{register}
% \end{syntax}
% Like the |V| type except the register is given by a list of
% characters from which a control sequence name is generated.
% \end{function}
%
% \begin{function}{ \exp_args:Nx }
% \begin{syntax}
%   " \exp_args:Nx" <funct> <arg1> <arg2> "..."
% \end{syntax}
% The first argument of <funct> (i.e., <arg1>) is fully expanded until
% only unexpandable tokens remain, the result is surrounded by braces
% and passed to <funct>. <funct> may have more than one argument---all
% others are passed unchanged.
% As mentioned before, this type of function is relatively slow.
% \end{function}
%
% \begin{function}{ \exp_args:Nf / (EXP) }
% \begin{syntax}
%   " \exp_args:Nf" <funct> <arg1> <arg2> "..."
% \end{syntax}
% The first argument of <funct> (i.e., <arg1>) undergoes full
% expansion until the first unexpandable token is encountered, the
% result is surrounded by braces and passed to <funct>. <funct> may
% have more than one argument---all others are passed unchanged.
% Beware of its special behavior as explained above.
% \end{function}
%
% \section{Manipulating two arguments}
%
% \begin{function}{%
%                  \exp_args:NNx |
%                  \exp_args:Nnx |
%                  \exp_args:Ncx |
%                  \exp_args:Nox |
%                  \exp_args:Nxo |
%                  \exp_args:Nxx |
% }
% \begin{syntax}
%   "\exp_args:Nnx" <funct> <arg1> <arg2> "..."
% \end{syntax}
% The above functions all manipulate the first two arguments of <funct>.
% They are all slow and non-expandable.
% \end{function}
%
% \begin{function}{%
%                  \exp_args:NNo / (EXP) |
%                  \exp_args:NNc / (EXP) |
%                  \exp_args:NNv / (EXP) |
%                  \exp_args:NNV / (EXP) |
%                  \exp_args:NNf / (EXP) |
%                  \exp_args:Nno / (EXP) |
%                  \exp_args:NnV / (EXP) |
%                  \exp_args:Nnf / (EXP) |
%                  \exp_args:Noo / (EXP) |
%                  \exp_args:Noc / (EXP) |
%                  \exp_args:Nco / (EXP) |
%                  \exp_args:Ncf / (EXP) |
%                  \exp_args:Ncc / (EXP) |
%                  \exp_args:Nff / (EXP) |
%                  \exp_args:Nfo / (EXP) |
%                  \exp_args:NVV / (EXP) |
% }
% \begin{syntax}
%   "\exp_args:NNo" <funct> <arg1> <arg2> "..."
% \end{syntax}
% These are the fast and expandable functions for the first two arguments.
% \end{function}
%
% \section{Manipulating three arguments}
%
% So far not all possible functions are provided and even the selection
% below may be reduced in the future as far as the non-expandable
% functions are concerned.
%
% \begin{function}{%
%                  \exp_args:NNnx |
%                  \exp_args:NNox |
%                  \exp_args:Nnnx |
%                  \exp_args:Nnox |
%                  \exp_args:Noox |
%                  \exp_args:Ncnx |
%                  \exp_args:Nccx |
% }
% \begin{syntax}
%   "\exp_args:Nnnx" <funct> <arg1> <arg2> <arg3> "..."
% \end{syntax}
% All the above functions are non-expandable.
% \end{function}
%
% \begin{function}{%
%                  \exp_args:NNNo / (EXP) |
%                  \exp_args:NNNV / (EXP) |
%                  \exp_args:NNoo / (EXP) |
%                  \exp_args:NNno / (EXP) |
%                  \exp_args:Nnno / (EXP) |
%                  \exp_args:Nnnc / (EXP) |
%                  \exp_args:Nooo / (EXP) |
%                  \exp_args:Nccc / (EXP) |
%                  \exp_args:NcNc / (EXP) |
%                  \exp_args:NcNo / (EXP) |
%                  \exp_args:Ncco / (EXP) |
% }
% \begin{syntax}
%   "\exp_args:NNoo" <funct> <arg1> <arg2> <arg3> "..."
% \end{syntax}
% These are the fast and expandable functions for the first three
% arguments.
% \end{function}
%
% \section{Preventing expansion}
%
% \begin{function}{\exp_not:N |
%                  \exp_not:c |
%                  \exp_not:n }
% \begin{syntax}
%   "\exp_not:N" <token>
%   "\exp_not:n" \Arg{token list}
% \end{syntax}
% This function will prohibit the expansion of <token> in situation
% where <token> would otherwise be replaced by it definition, e.g.,
% inside an argument that is handled by the "x" convention.
% \begin{texnote}
% "\exp_not:N" is the primitive \tn{noexpand} renamed and "\exp_not:n"
% is the \eTeX{} primitive \tn{unexpanded}.
% \end{texnote}
% \end{function}
%
% \begin{function}{\exp_not:o |
%                  \exp_not:f }
% \begin{syntax}
%   "\exp_not:o" \Arg{token list}
% \end{syntax}
% Same as "\exp_not:n" except <token list> is expanded once for the
% "o" type and for the "f" type the token list is expanded until an
% unexpandable token is found, and the result of these expansions is then
% prohibited from being expanded further.
% \end{function}
%
% \begin{function}{\exp_not:V |
%                  \exp_not:v }
% \begin{syntax}
%   "\exp_not:V" <register>
%   "\exp_not:v" \Arg{token list}
% \end{syntax}
% The value of <register> is retrieved and then passed on to
% "\exp_not:n" which will prohibit further expansion. The |v| type
% first creates a control sequence from <token list> but is otherwise
% identical to |V|.
% \end{function}
%
% \begin{function}{\exp_stop_f:}
% \begin{syntax}
%   <f expansion> ... "\exp_stop_f:" 
% \end{syntax}
% This function stops an "f" type expansion. An example use is one such as
% \begin{verbatim}
% \tl_set:Nf \l_tmpa_tl {
%   \if_case:w \l_tmpa_int 
%     \or:   \use_i_after_orelse:nw {\exp_stop_f: \textbullet}
%     \or:   \use_i_after_orelse:nw {\exp_stop_f: \textendash}
%     \else: \use_i_after_fi:nw     {\exp_stop_f: else-item}
%   \fi:
% }
% \end{verbatim}
% This ensures the expansion in stopped right after finishing the
% conditional but without expanding "\textbullet" etc.
% \begin{texnote}
%   This function is a space token but it is better to distinguish
%   this expansion stopping token from a desired space token when
%   writing code.
% \end{texnote}
% \end{function}
% 
% \section{Unbraced expansion}
%
% \begin{function}{
%   \exp_last_unbraced:Nf|
%   \exp_last_unbraced:NV|
%   \exp_last_unbraced:Nv|
%   \exp_last_unbraced:NcV|
%   \exp_last_unbraced:NNo|
%   \exp_last_unbraced:NNV|
%   \exp_last_unbraced:NNNo|
%   }
% \begin{syntax}
%   "\exp_last_unbraced:NV" <token> <variable name>
% \end{syntax}
%   There are a small number of occasions where the last argument 
%   in an expansion run must be expanded unbraced. These functions 
%   should only be used inside functions, \emph{not} for creating
%   variants.
% \end{function}
%
% \end{documentation}
%
% \begin{implementation}
%
% \section{\pkg{l3expan} implementation}
%
% \subsection{Internal functions and variables}
%
% \begin{function}{\exp_after:wN}
% \begin{syntax}
%   "\exp_after:wN" <token1> <token2>
% \end{syntax}
% This will expand <token2> once before processing <token1>. This is
% similar to "\exp_args:No" except that no braces are put around the
% result of expanding <token2>.
% \begin{texnote}
% This is the primitive \tn{expandafter} which was renamed to fit into
% the naming conventions of \LaTeX3.
% \end{texnote}
% \end{function}
%
% \begin{variable}{\l_exp_tl}
% \begin{syntax}\end{syntax}
% The "\exp_" module has its private variables to temporarily store
% results of the argument expansion. This is done to avoid interference
% with other functions using temporary variables.
% \end{variable}
%
% \begin{function}{ \exp_eval_register:N / (EXP) | 
%                   \exp_eval_register:c / (EXP) }
% \begin{syntax}
% "\exp_eval_register:N" <register>
% \end{syntax}
% These functions evaluates a register as part of a "V" or "v" expansion 
% (respectively). A register might exist as
% one of two things: A parameter-less non-long, non-protected macro
% or a built-in \TeX\ register such as |\count|.
% \end{function}
% 
% \begin{function}{\exp_eval_error_msg:w}
% \begin{syntax}
% "\exp_eval_error_msg:w" <register>
% \end{syntax}
% Used to generate an error message if a variable called as part of a
% "v" or "V" expansion is defined as \cs{scan_stop:}. This typically
% indicates that an incorrect cs name has been used.
% \end{function}
%
% \begin{function}{\::n|\::N|\::c|\::o|\::f|\::x|\::v|\::V|\:::}
% \begin{syntax}
% "\cs_set_nopar:Npn \exp_args:Ncof {\::c\::o\::f\:::}"
% \end{syntax}
% Internal forms for the base expansion types.
% \end{function}
%
% \subsection{Module code}
%
%    We start by ensuring that the required packages are loaded.
%    \begin{macrocode}
%<*package>
\ProvidesExplPackage
  {\filename}{\filedate}{\fileversion}{\filedescription}
\package_check_loaded_expl:
%</package>
%<*initex|package>
%    \end{macrocode}
%
% \begin{macro}{\exp_after:wN}
% \begin{macro}{\exp_not:N}
% \begin{macro}{\exp_not:n}
% These are defined in \pkg{l3basics}.
%    \begin{macrocode}
%<*bootstrap>
\cs_set_eq:NwN   \exp_after:wN       \tex_expandafter:D
\cs_set_eq:NwN   \exp_not:N          \tex_noexpand:D
\cs_set_eq:NwN   \exp_not:n          \etex_unexpanded:D
%</bootstrap>
%    \end{macrocode}
% \end{macro}
% \end{macro}
% \end{macro}
%
% \subsection{General expansion}
%
% In this section a general mechanism for defining functions to handle
% argument handling is defined.  These general expansion functions are
% expandable unless |x| is used.  (Any version of |x| is going to have
% to use one of the \LaTeX3\ names for |\cs_set_nopar:Npx| at some point, and
% so is never going to be expandable.\footnote{However, some
%   primitives have certain characteristics that means that their
%   arguments undergo an \texttt{x} type expansion but the primitive
%   is in fact still expandable. We shall make it very clear when such
%   a function is expandable.})
%
% The definition of expansion functions with this technique happens
% in section~\ref{sec:gendef}.
% In section~\ref{sec:handtune} some common cases are coded by a more direct
% method for efficiency, typically using calls to |\exp_after:wN|.
%
% \begin{macro}{\l_exp_tl}
%    We need a scratch token list variable.
%    We don't use "tl" methods so that \pkg{l3expan} can be loaded earlier.
%    \begin{macrocode}
\cs_new_nopar:Npn \l_exp_tl {}
%    \end{macrocode}
% \end{macro}
%
% This code uses internal functions with names that start with |\::|
% to perform the expansions. All macros are |long| as this turned out
% to be desirable since the tokens undergoing expansion may be
% arbitrary user input.
%
% An argument manipulator |\::|\meta{Z} always has signature |#1\:::#2#3|
% where |#1| holds the remaining argument manipulations to be performed,
% |\:::| serves as an end marker for the list of manipulations, |#2|
% is the carried over result of the previous expansion steps and |#3| is
% the argument about to be processed.
%
%  \begin{macro}[aux]{\exp_arg_next:nnn}
%  \begin{macro}[aux]{\exp_arg_next_nobrace:nnn}
%    |#1| is the result of an expansion step, |#2| is the remaining
%    argument manipulations and |#3| is the current result of the
%    expansion chain.  This auxilliary function moves |#1| back after
%    |#3| in the input stream and checks if any expansion is left to
%    be done by calling |#2|. In by far the most cases we will require
%    to add a set of braces to the result of an argument manipulation
%    so it is more effective to do it directly here. Actually, so far
%    only the |c| of the final argument manipulation variants does not
%    require a set of braces.
%    \begin{macrocode}
\cs_new:Npn\exp_arg_next:nnn#1#2#3{
  #2\:::{#3{#1}}
}
\cs_new:Npn\exp_arg_next_nobrace:nnn#1#2#3{
  #2\:::{#3#1}
}
%    \end{macrocode}
% \end{macro}
% \end{macro}
%
%  \begin{macro}{\:::}
%    The end marker is just another name for the identity function.
%    \begin{macrocode}
\cs_new:Npn\:::#1{#1}
%    \end{macrocode}
% \end{macro}
%
%  \begin{macro}{\::n}
%    This function is used to skip an argument that doesn't need to
%    be expanded.
%    \begin{macrocode}
\cs_new:Npn\::n#1\:::#2#3{
  #1\:::{#2{#3}}
}
%    \end{macrocode}
% \end{macro}
%
%  \begin{macro}{\::N}
%    This function is used to skip an argument that consists of a
%    single token and doesn't need to be expanded.
%    \begin{macrocode}
\cs_new:Npn\::N#1\:::#2#3{
  #1\:::{#2#3}
}
%    \end{macrocode}
% \end{macro}
%
%  \begin{macro}{\::c}
%    This function is used to skip an argument that is turned into
%    as control sequence without expansion.
%    \begin{macrocode}
\cs_new:Npn\::c#1\:::#2#3{
  \exp_after:wN\exp_arg_next_nobrace:nnn\cs:w #3\cs_end:{#1}{#2}
}
%    \end{macrocode}
% \end{macro}
%
%  \begin{macro}{\::o}
%    This function is used to expand an argument once.
%    \begin{macrocode}
\cs_new:Npn\::o#1\:::#2#3{
  \exp_after:wN\exp_arg_next:nnn\exp_after:wN{#3}{#1}{#2}
}
%    \end{macrocode}
% \end{macro}
%
%
%  \begin{macro}{\::f}
%  \begin{macro}{\exp_stop_f:}
%    This function is used to expand a token list until the first
%    unexpandable token is found. The underlying "\tex_romannumeral:D -`0"
%    expands everything in its way to find something terminating the
%    number and thereby expands the function in front of it. This
%    scanning procedure is terminated once the expansion hits
%    something non-expandable or a space. We introduce "\exp_stop_f:"
%    to mark such an end of expansion marker; in case the scanner hits
%    a number, this number also terminates the scanning and is left
%    untouched. In the example shown earlier the scanning was stopped
%    once \TeX{} had fully expanded "\cs_set_eq:Nc \aaa {b \l_tmpa_tl b}"
%    into "\cs_set_eq:NwN \aaa = \blurb" which then turned out to contain
%    the non-expandable token "\cs_set_eq:NwN".  Since the expansion of
%    "\tex_romannumeral:D -`0" is \m{null}, we wind up with a fully
%    expanded list, only \TeX{} has not tried to execute any of the
%    non-expandable tokens. This is what differentiates this function
%    from the "x" argument type.
%    \begin{macrocode}
\cs_new:Npn\::f#1\:::#2#3{
  \exp_after:wN\exp_arg_next:nnn
  \exp_after:wN{\tex_romannumeral:D -`0 #3}
  {#1}{#2}
}
\cs_new_nopar:Npn \exp_stop_f: {~}
%    \end{macrocode}
% \end{macro}
% \end{macro}
%
%  \begin{macro}{\::x,\exp_arg:x}
%    This function is used to expand an argument fully.
%    We could use the new expandable primitive "\expanded" here, but we
%    don't want to create incompatibilities between engines.
%    \begin{macrocode}
%%\cs_new_eq:NN \exp_arg:x \expanded % Move eventually.
%%\cs_if_free:NTF\exp_arg:x{
  \cs_new:Npn \::x #1 \::: #2#3 {
    \cs_set_nopar:Npx \l_exp_tl {{#3}}
    \exp_after:wN \exp_arg_next:nnn \l_exp_tl {#1}{#2}
  }
%%}
%%{
%%  \cs_new:Npn\::x#1\:::#2#3{
%%    \exp_after:wN\exp_arg_next:nnn
%%    \exp_after:wN{\exp_arg:x{#3}}{#1}{#2}
%%  }
%%}
%    \end{macrocode}
% \end{macro}
%
%
%  \begin{macro}{\::v}
%  \begin{macro}{\::V}
%    These functions return the value of a register, i.e., one of
%    |tl|, |num|, |int|, |skip|, |dim| and |muskip|. The |V| version
%    expects a single token whereas |v| like |c| creates a csname from
%    its argument given in braces and then evaluates it as if it was a
%    |V|. The sequence |\tex_romannumeral:D -`0| sets off an |f| type
%    expansion. The argument is returned in braces.
%    \begin{macrocode}
\cs_new:Npn \::V#1\:::#2#3{
  \exp_after:wN\exp_arg_next:nnn
  \exp_after:wN{
    \tex_romannumeral:D -`0 
    \exp_eval_register:N #3
  }
  {#1}{#2}
}
\cs_new:Npn \::v#1\:::#2#3{
  \exp_after:wN\exp_arg_next:nnn
  \exp_after:wN{
    \tex_romannumeral:D -`0 
    \exp_eval_register:c {#3}
  }
  {#1}{#2}
}
%    \end{macrocode}
% \end{macro}
% \end{macro}
%
%  \begin{macro}{\exp_eval_register:N,\exp_eval_register:c}
%  \begin{macro}{\exp_eval_error_msg:w}
%    This function evaluates a register. Now a register might exist as
%    one of two things: A parameter-less macro or a built-in \TeX\
%    register such as |\count|. For the \TeX\ registers we have to
%    utilize a |\tex_the:D| whereas for the macros we merely have to
%    expand them once. The trick is to find out when to use
%    |\tex_the:D| and when not to. What we do here is try to find out
%    whether the token will expand to something else when hit with
%    |\exp_after:wN|. The technique is to compare the meaning of the
%    register in question when it has been prefixed with |\exp_not:N|
%    and the register itself. If it is a macro, the prefixed
%    |\exp_not:N| will temporarily turn it into the primitive
%    |\tex_relax:D|.
%    \begin{macrocode}
\cs_set_nopar:Npn \exp_eval_register:N #1{
  \exp_after:wN \if_meaning:w \exp_not:N #1#1
%    \end{macrocode}
% If the token was not a macro it may be a malformed variable from a
% |c| expansion in which case it is equal to the primitive
% |\tex_relax:D|. In that case we throw an error. We could let \TeX\
% do it for us but that would result in the rather obscure
% \begin{quote}
%   |! You can't use `\relax' after \the.|
% \end{quote}
% which while quite true doesn't give many hints as to what actually
% went wrong. We provide something more sensible.
%    \begin{macrocode}
    \if_meaning:w \tex_relax:D #1
      \exp_eval_error_msg:w 
    \fi:
%    \end{macrocode}
% The next bit requires some explanation. The function must be
% initiated by the sequence |\tex_romannumeral:D -`0| and we want to 
% terminate this expansion chain by inserting an |\exp_stop_f:|
% token. However, we have to expand the register |#1| before we do
% that. If it is a \TeX\ register, we need to execute the sequence
% |\exp_after:wN\exp_stop_f:\tex_the:D #1| and if it is a macro we
% need to execute |\exp_after:wN\exp_stop_f: #1|. We therefore issue
% the longer of the two sequences and if the register is a macro, we
% remove the |\tex_the:D|.
%    \begin{macrocode}
  \else:
    \exp_after:wN \use_i_ii:nnn
  \fi:
  \exp_after:wN \exp_stop_f: \tex_the:D #1
}
\cs_set_nopar:Npn \exp_eval_register:c #1{
  \exp_after:wN\exp_eval_register:N\cs:w #1\cs_end:
}
%    \end{macrocode}
% Clean up nicely, then call the undefined control sequence. The
% result is an error message looking like this:
% \begin{verbatim}
% ! Undefined control sequence.
% \exp_eval_error_msg:w ...erroneous variable used! 
%                                                  
% l.55 \tl_set:Nv \l_tmpa_tl {undefined_tl}
% \end{verbatim}
%    \begin{macrocode}
\group_begin:%
\tex_catcode:D`\!=11\tex_relax:D%
\tex_catcode:D`\ =11\tex_relax:D%
\cs_gset:Npn\exp_eval_error_msg:w#1\tex_the:D#2{%
\fi:\fi:\erroneous variable used!}%
\group_end:%
%    \end{macrocode}
%  \end{macro}
%  \end{macro}
%
% \subsection{Hand-tuned definitions}
% \label{sec:handtune}
%
% One of the most important features of these functions is that they
% are fully expandable and therefore allow to prefix them with
% |\pref_global:D| for example. This together with the fact that the
% `general' concept above is slower means that we should convert
% whenever possible and perhaps remove all remaining occurences by
% hand-encoding in the end.
%
% \begin{macro}{\exp_args:No}
% \begin{macro}{\exp_args:NNo}
% \begin{macro}{\exp_args:NNNo}
%    \begin{macrocode}
\cs_new:Npn \exp_args:No #1#2{\exp_after:wN#1\exp_after:wN{#2}}
\cs_new:Npn \exp_args:NNo #1#2#3{\exp_after:wN#1\exp_after:wN#2
  \exp_after:wN{#3}}
\cs_new:Npn \exp_args:NNNo #1#2#3#4{\exp_after:wN#1\exp_after:wN#2
  \exp_after:wN#3\exp_after:wN{#4}}
%    \end{macrocode}
% \end{macro}
% \end{macro}
% \end{macro}
%
%
% \begin{macro}{\exp_args:Nc,\exp_args:cc,\exp_args:NNc,\exp_args:Ncc,\exp_args:Nccc}
%    Here are the functions that turn their argument into csnames but
%    are  expandable.
%    \begin{macrocode}
\cs_set:Npn \exp_args:Nc #1#2{\exp_after:wN#1\cs:w#2\cs_end:}
\cs_new:Npn \exp_args:cc #1#2{\cs:w #1\exp_after:wN\cs_end:\cs:w #2\cs_end:}
\cs_new:Npn \exp_args:NNc #1#2#3{\exp_after:wN#1\exp_after:wN#2
    \cs:w#3\cs_end:}
\cs_new:Npn \exp_args:Ncc #1#2#3{\exp_after:wN#1
    \cs:w#2\exp_after:wN\cs_end:\cs:w#3\cs_end:}
\cs_new:Npn \exp_args:Nccc #1#2#3#4{\exp_after:wN#1
    \cs:w#2\exp_after:wN\cs_end:\cs:w#3\exp_after:wN
      \cs_end:\cs:w #4\cs_end:}
%    \end{macrocode}
% \end{macro}
%
%  \begin{macro}{\exp_args:Nco}
%    If we force that the third argument
%    always has braces, we could implement this function
%    with less tokens and only two arguments.
%    \begin{macrocode}
\cs_new:Npn \exp_args:Nco #1#2#3{\exp_after:wN#1\cs:w#2\exp_after:wN
     \cs_end:\exp_after:wN{#3}}
%    \end{macrocode}
%  \end{macro}
%
% \subsection{Definitions with the `general' technique}
% \label{sec:gendef}
%
%  \begin{macro}{\exp_args:Nf,\exp_args:NV,\exp_args:Nv,\exp_args:Nx}
%    \begin{macrocode}
\cs_set_nopar:Npn \exp_args:Nf {\::f\:::}
\cs_set_nopar:Npn \exp_args:Nv {\::v\:::}
\cs_set_nopar:Npn \exp_args:NV {\::V\:::}
\cs_set_nopar:Npn \exp_args:Nx {\::x\:::}
%    \end{macrocode}
%  \end{macro}
%
%  \begin{macro}{\exp_args:NNV,\exp_args:NNv,\exp_args:NNf,\exp_args:NNx,
%                \exp_args:NVV,
%                \exp_args:Ncx,
%                \exp_args:Nfo,\exp_args:Nff,
%                \exp_args:Ncf,\exp_args:Nco,
%                \exp_args:Nnf,\exp_args:Nno,\exp_args:NnV,\exp_args:Nnx,
%                \exp_args:Noo,\exp_args:Noc,\exp_args:Nox,
%                \exp_args:Nxo,\exp_args:Nxx}
%    Here are the actual function definitions, using the helper functions
%    above.
%    \begin{macrocode}
\cs_set_nopar:Npn \exp_args:NNf {\::N\::f\:::}
\cs_set_nopar:Npn \exp_args:NNv {\::N\::v\:::}
\cs_set_nopar:Npn \exp_args:NNV {\::N\::V\:::}
\cs_set_nopar:Npn \exp_args:NNx {\::N\::x\:::}

\cs_set_nopar:Npn \exp_args:Ncx {\::c\::x\:::}
\cs_set_nopar:Npn \exp_args:Nfo {\::f\::o\:::}
\cs_set_nopar:Npn \exp_args:Nff {\::f\::f\:::}
\cs_set_nopar:Npn \exp_args:Ncf {\::c\::f\:::}
\cs_set_nopar:Npn \exp_args:Nnf {\::n\::f\:::}
\cs_set_nopar:Npn \exp_args:Nno {\::n\::o\:::}
\cs_set_nopar:Npn \exp_args:NnV {\::n\::V\:::}
\cs_set_nopar:Npn \exp_args:Nnx {\::n\::x\:::}

\cs_set_nopar:Npn \exp_args:Noc {\::o\::c\:::}
\cs_set_nopar:Npn \exp_args:Noo {\::o\::o\:::}
\cs_set_nopar:Npn \exp_args:Nox {\::o\::x\:::}

\cs_set_nopar:Npn \exp_args:NVV {\::V\::V\:::}

\cs_set_nopar:Npn \exp_args:Nxo {\::x\::o\:::}
\cs_set_nopar:Npn \exp_args:Nxx {\::x\::x\:::}
%    \end{macrocode}
%  \end{macro}
% 
%  \begin{macro}{\exp_args:Ncco,
%                \exp_args:Nccx,
%                \exp_args:Ncnx,
%                \exp_args:NcNc,
%                \exp_args:NcNo,
%                \exp_args:NNno,
%                \exp_args:NNNV,
%                \exp_args:Nnno,
%                \exp_args:Nnnx,
%                \exp_args:Nnox,
%                \exp_args:Nooo,
%                \exp_args:Noox,
%                \exp_args:Nnnc,
%                \exp_args:NNnx,
%                \exp_args:NNoo,
%                \exp_args:NNox}
%    \begin{macrocode}
\cs_set_nopar:Npn \exp_args:NNNV {\::N\::N\::V\:::}

\cs_set_nopar:Npn \exp_args:NNno {\::N\::n\::o\:::}
\cs_set_nopar:Npn \exp_args:NNnx {\::N\::n\::x\:::}
\cs_set_nopar:Npn \exp_args:NNoo {\::N\::o\::o\:::}
\cs_set_nopar:Npn \exp_args:NNox {\::N\::o\::x\:::}

\cs_set_nopar:Npn \exp_args:Nnnc {\::n\::n\::c\:::}
\cs_set_nopar:Npn \exp_args:Nnno {\::n\::n\::o\:::}
\cs_set_nopar:Npn \exp_args:Nnnx {\::n\::n\::x\:::}
\cs_set_nopar:Npn \exp_args:Nnox {\::n\::o\::x\:::}

\cs_set_nopar:Npn \exp_args:NcNc {\::c\::N\::c\:::}
\cs_set_nopar:Npn \exp_args:NcNo {\::c\::N\::o\:::}
\cs_set_nopar:Npn \exp_args:Ncco {\::c\::c\::o\:::}
\cs_set_nopar:Npn \exp_args:Ncco {\::c\::c\::o\:::}
\cs_set_nopar:Npn \exp_args:Nccx {\::c\::c\::x\:::}
\cs_set_nopar:Npn \exp_args:Ncnx {\::c\::n\::x\:::}

\cs_set_nopar:Npn \exp_args:Noox {\::o\::o\::x\:::}
\cs_set_nopar:Npn \exp_args:Nooo {\::o\::o\::o\:::}
%    \end{macrocode}
%  \end{macro}
%
%
% \subsection{Preventing expansion}
%
%
%  \begin{macro}{\exp_not:o}
%  \begin{macro}{\exp_not:f}
%  \begin{macro}{\exp_not:v}
%  \begin{macro}{\exp_not:V}
%    \begin{macrocode}
\cs_new:Npn\exp_not:o#1{\exp_not:n\exp_after:wN{#1}}
\cs_new:Npn\exp_not:f#1{
  \exp_not:n\exp_after:wN{\tex_romannumeral:D -`0 #1}
}
\cs_new:Npn\exp_not:v#1{
  \exp_not:n\exp_after:wN{\tex_romannumeral:D -`0 \exp_eval_register:c {#1}}
}
\cs_new:Npn\exp_not:V#1{
  \exp_not:n\exp_after:wN{\tex_romannumeral:D -`0 \exp_eval_register:N #1}
}
%    \end{macrocode}
%  \end{macro}
%  \end{macro}
%  \end{macro}
%  \end{macro}
%  \end{macro}
% 
%  \begin{macro}{\exp_not:c}
%    A helper function.
%    \begin{macrocode}
\cs_new:Npn\exp_not:c#1{\exp_after:wN\exp_not:N\cs:w#1\cs_end:}
%    \end{macrocode}
%  \end{macro}
%
%
% \subsection{Defining function variants}
%
%
% \begin{macro}{\cs_generate_variant:Nn}
% \begin{macro}[aux]{\cs_generate_variant_aux:nnNn}
% \begin{macro}[aux]{\cs_generate_variant_aux:nnw}
% \begin{macro}[aux]{\cs_generate_variant_aux:N}
% \begin{arguments}
% \item Base form of a function; e.g., "\tl_set:Nn"
% \item One or more variant argument specifiers; e.g., "{Nx,c,cx}"
% \end{arguments}
% Split up the original base function to grab its name and signature
% consisting of $k$ letters. Then we wish to iterate through the list
% of variant argument specifiers, and for each one construct a new
% function name using the original base name, the variant signature
% consisting of $l$ letters and the last $k-l$ letters of the base
% signature. For example, for a base function "\tl_set:Nn" which
% needs a "c" variant form, we want the new signature to be "cn".
%    \begin{macrocode}
\cs_new:Npn \cs_generate_variant:Nn #1 {
  \chk_if_exist_cs:N #1
  \cs_split_function:NN #1 \cs_generate_variant_aux:nnNn
}
%    \end{macrocode}
% We discard the boolean and then set off a loop through the desired
% variant forms.
%    \begin{macrocode}
\cs_set:Npn \cs_generate_variant_aux:nnNn #1#2#3#4{
  \cs_generate_variant_aux:nnw {#1}{#2} #4,?,\q_recursion_stop
}
%    \end{macrocode}
% Next is the real work to be done. We now have 1: base name, 2: base
% signature, 3: beginning of variant signature. To construct the new
% csname and the "\exp_args:Ncc" form, we need the variant signature.
% In our example, we wanted to discard the first two letters of the
% base signature because the variant form started with "cc". This is
% the same as putting first "cc" in the signature and then
% "\use_none:nn" followed by the base signature "NNn". We therefore
% call a small loop that outputs an "n" for each letter in the variant
% signature and use this to call the correct "\use_none:" variant.
% Firstly though, we check whether to terminate the loop.
%    \begin{macrocode}
\cs_set:Npn \cs_generate_variant_aux:nnw #1 #2 #3, {
  \if:w ? #3
    \exp_after:wN \use_none_delimit_by_q_recursion_stop:w 
  \fi:
%    \end{macrocode}
% Then check if the variant form has already been defined.
%    \begin{macrocode}
  \cs_if_free:cTF {
    #1:#3\use:c {use_none:\cs_generate_variant_aux:N #3 ?}#2
  } 
  {
%    \end{macrocode}
% If not, then define it and then additionally check if
% the |\exp_args:N| form needed is defined.
%    \begin{macrocode}
    \cs_new_nopar:cpx { 
      #1:#3 \use:c{use_none:\cs_generate_variant_aux:N #3 ?}#2
    }
    {
      \exp_not:c { exp_args:N #3} \exp_not:c {#1:#2}
    }
    \cs_generate_internal_variant:n {#3}
  }
%    \end{macrocode}
% Otherwise tell that it was already defined.
%    \begin{macrocode}
  {
    \iow_log:x{ 
      Variant~\token_to_str:c { 
        #1:#3\use:c {use_none:\cs_generate_variant_aux:N #3 ?}#2
      }~already~defined;~ not~ changing~ it~on~line~
      \tex_the:D \tex_inputlineno:D
    }
  }
%    \end{macrocode}
% Recurse.
%    \begin{macrocode}
  \cs_generate_variant_aux:nnw{#1}{#2}
}
%    \end{macrocode}
% The small loop for defining the required number of "n"s. Break when
% seeing a "?".
%    \begin{macrocode}
\cs_set:Npn \cs_generate_variant_aux:N #1{
  \if:w ?#1 \exp_after:wN\use_none:nn \fi: n \cs_generate_variant_aux:N
}
%    \end{macrocode}
% \end{macro}
% \end{macro}
% \end{macro}
% \end{macro}
%
%
%  \begin{macro}{\cs_generate_internal_variant:n}
%    Test if |exp_args:N #1| is already defined
%    and if not define it via the
%    |\::| commands using the chars in |#1|
%    \begin{macrocode}
\cs_new:Npn \cs_generate_internal_variant:n #1 {
  \cs_if_free:cT { exp_args:N #1 }{
%    \end{macrocode}
%    We use "new" to log the definition if we have to make one.
%    \begin{macrocode}
    \cs_new:cpx { exp_args:N #1 } 
                { \cs_generate_internal_variant_aux:n #1 : }
  }
}
%    \end{macrocode}
%  \end{macro}
%
%  \begin{macro}[aux]{\cs_generate_internal_variant_aux:n}
%    This command grabs char by char outputting |\::#1| (not expanded
%    further) until we see a |:|. That colon is in fact also turned into
%    |\:::| so that the required structure for |\exp_args...| commands
%    is correctly terminated.
%    \begin{macrocode}
\cs_new:Npn \cs_generate_internal_variant_aux:n #1 {
  \exp_not:c{::#1}
  \if_meaning:w #1 :
    \exp_after:wN \use_none:n
  \fi:
  \cs_generate_internal_variant_aux:n
}
%    \end{macrocode}
%  \end{macro}
%
% \subsection{Last-unbraced versions}
%
%\begin{macro}[aux]{\exp_arg_last_unbraced:nn}
%\begin{macro}[aux]{\::f_unbraced}
%\begin{macro}[aux]{\::o_unbraced}
%\begin{macro}[aux]{\::V_unbraced}
%\begin{macro}[aux]{\::v_unbraced}
% There are a few places where the last argument needs to be available 
% unbraced. First some helper macros.
%    \begin{macrocode}
\cs_new:Npn \exp_arg_last_unbraced:nn #1#2 { #2#1 }
\cs_new:Npn \::f_unbraced \:::#1#2 {
  \exp_after:wN \exp_arg_last_unbraced:nn
  \exp_after:wN { \tex_romannumeral:D -`0 #2 } {#1}
}
\cs_new:Npn \::o_unbraced \:::#1#2 {
  \exp_after:wN \exp_arg_last_unbraced:nn \exp_after:wN {#2 }{#1}
}
\cs_new:Npn \::V_unbraced \:::#1#2 {
  \exp_after:wN \exp_arg_last_unbraced:nn
  \exp_after:wN { \tex_romannumeral:D -`0 \exp_eval_register:N #2 } {#1}
}
\cs_new:Npn \::v_unbraced \:::#1#2 {
  \exp_after:wN \exp_arg_last_unbraced:nn
  \exp_after:wN { 
    \tex_romannumeral:D -`0 \exp_eval_register:c {#2} 
  } {#1}
}
%    \end{macrocode}
%\end{macro}
%\end{macro}
%\end{macro}
%\end{macro}
%\end{macro}
%
%\begin{macro}{\exp_last_unbraced:NV}
%\begin{macro}{\exp_last_unbraced:Nv}
%\begin{macro}{\exp_last_unbraced:Nf}
%\begin{macro}{\exp_last_unbraced:NcV}
%\begin{macro}{\exp_last_unbraced:NNo}
%\begin{macro}{\exp_last_unbraced:NNV}
%\begin{macro}{\exp_last_unbraced:NNNo}
% Now the business end.
%    \begin{macrocode}
\cs_new_nopar:Npn \exp_last_unbraced:Nf { \::f_unbraced \::: }
\cs_new_nopar:Npn \exp_last_unbraced:NV { \::V_unbraced \::: }
\cs_new_nopar:Npn \exp_last_unbraced:Nv { \::v_unbraced \::: }
\cs_new_nopar:Npn \exp_last_unbraced:NcV { 
  \::c \::V_unbraced \::: 
}
\cs_new:Npn \exp_last_unbraced:NNo #1#2#3 { 
  \exp_after:wN #1 \exp_after:wN #2 #3
}
\cs_new_nopar:Npn \exp_last_unbraced:NNV { 
  \::N \::V_unbraced \::: 
}
\cs_new:Npn \exp_last_unbraced:NNNo #1#2#3#4 { 
  \exp_after:wN #1 \exp_after:wN #2 \exp_after:wN #3 #4
}
%    \end{macrocode}
%\end{macro}
%\end{macro}
%\end{macro}
%\end{macro}
%\end{macro}
%\end{macro}
%\end{macro}
%
%    \begin{macrocode}
%</initex|package>
%    \end{macrocode}
%
%    Show token usage:
%    \begin{macrocode}
%<*showmemory>
\showMemUsage
%</showmemory>
%    \end{macrocode}
%
% \end{implementation}
% \PrintIndex
%
% \endinput
