% \iffalse
%% File: l3expan.dtx Copyright (C) 1990-2009 LaTeX3 project
%%
%% It may be distributed and/or modified under the conditions of the
%% LaTeX Project Public License (LPPL), either version 1.3c of this
%% license or (at your option) any later version.  The latest version
%% of this license is in the file
%%
%%    http://www.latex-project.org/lppl.txt
%%
%% This file is part of the ``expl3 bundle'' (The Work in LPPL)
%% and all files in that bundle must be distributed together.
%%
%% The released version of this bundle is available from CTAN.
%%
%% -----------------------------------------------------------------------
%%
%% The development version of the bundle can be found at
%%
%%    http://www.latex-project.org/cgi-bin/cvsweb.cgi/
%%
%% for those people who are interested.
%%
%%%%%%%%%%%
%% NOTE: %%
%%%%%%%%%%%
%%
%%   Snapshots taken from the repository represent work in progress and may
%%   not work or may contain conflicting material!  We therefore ask
%%   people _not_ to put them into distributions, archives, etc. without
%%   prior consultation with the LaTeX Project Team.
%%
%% -----------------------------------------------------------------------
%
%<*driver|package>
\RequirePackage{l3names}
%</driver|package>
%\fi
\GetIdInfo$Id$
       {L3 Experimental Argument Expansion module}
%\iffalse
%<*driver>
%\fi
\ProvidesFile{\filename.\filenameext}
  [\filedate\space v\fileversion\space\filedescription]
%\iffalse
\documentclass[full]{l3doc}
\begin{document}
\DocInput{\filename.\filenameext}
\end{document}
%</driver>
% \fi
%
%
% \title{The \pkg{l3expan} package\thanks{This file
%         has version number \fileversion, last
%         revised \filedate.}\\
% Controlling Expansion of Function Arguments}
% \author{\Team}
% \date{\filedate}
% \maketitle
%
%
% \section{Expansion control}
%
% \subsection{Brief overview}
%
% The functions in this module all have prefix |exp|. 
%
% Not all possible variations are implemented for every base
% function. Instead only those that are used within the \LaTeX3 kernel
% or otherwise seem to be of general interest are implemented.
% Consult the module description to find out which functions are
% actually defined. The next section explains how to define missing
% variants.
% 
%
% \subsection{Defining new variants}
%
% The definition of variant forms for base functions may be necessary
% when writing new functions or when applying a kernel function in a
% situation that we haven't thought of before.
%
% Internally preprocessing of arguments is done with functions from the
% "\exp_" module.  They all look alike, an example would be
% "\exp_args:NNo". This function has three arguments, the first and the
% second are a single tokens  the third argument gets
% expanded once. If "\seq_gpush:No" wouldn't be defined the example
% above could be coded in the following way:
% \begin{verbatim}
%   \exp_args:NNo\seq_gpush:Nn
%      \g_file_name_stack
%      \l_tmpa_tlp
% \end{verbatim}
% In other words, the first argument to "\exp_args:NNo" is the base
% function and the other arguments are preprocessed and then passed to
% this base function. In the example the first argument to the base
% function should be a single token which is left unchanged while the
% second argument is expanded once. From this example we can also see
% how the variants are defined. They just expand into the appropriate
% "\exp_" function followed by the desired base function, e.g.
% \begin{quote}
%   "\cs_new_nopar:Npn\seq_gpush:No{\exp_args:NNo\seq_gpush:Nn}"
% \end{quote}
% Providing variants in this way in style files is uncritical as the
% "\cs_new_nopar:Npn" function will silently accept definitions whenever the
% new definition is identical to an already given one. Therefore adding
% such definition to later releases of the kernel will not make such
% style files obsolete.
%
% \begin{function}{\exp_def_form:nnn}
% \begin{syntax}
%   "\exp_def_form:nnn" <base name> <base arg.> <list of args>
% \end{syntax}
% This function provides an easy way to define new function variants.
% To continue the example above:
% \begin{verbatim}
%   \exp_def_form:nnn {seq_push} {Nn} {No,cn}
% \end{verbatim}
% defines the variants "\seq_gpush:No" and "\seq_gpush:cn".
% \end{function}
%
%
% The available internal functions for argument expansion come in two
% flavours, some of them are faster then others. Therefore it is usually
% best to follow the following guidelines when defining new functions
% that are supposed to come with variant forms:
% \begin{itemize}
% \item
%   Arguments that might need expansion should come first in the list of
%   arguments to make processing faster.
% \item
%   Arguments that should consist of single tokens should come first.
% \item
%   Arguments that need full expansion (i.e., are denoted with "x")
%   should be avoided if possible as they can not be processed very fast.
% \item 
%   In general "n", "x", and "o" (if not in the last position) will
%   need special processing which is not fast and not expandable,
%   i.e., functions of this type may not work correctly in arguments
%   that are itself subject to "x" expansion. Therefore it is best to
%   use the ``expandable'' functions (i.e., those that contain only
%   "c", "N", "o" or "f" in the last position) whenever possible.
% \end{itemize}
%
% The |V| type returns the value of a register, which can be one of
% |tlp|, |num|, |int|, |skip|, |dim| or |toks|. The |v| type is the
% same except it first creates a control sequence out of its argument
% before returning the value. This recent addition to the argument
% specifiers may shake things up a bit as most places where |o| is
% used will be replaced by |V|. The documentation you are currently
% reading will therefore require a fair bit of re-writing.
%
% The |f| type is so special that it deserves an example.
% Let's pretend we want to set "\aaa" equal to the control sequence
% stemming from turning "b \l_tmpa_tlp b" into a control
% sequence. Furthermore we want to store the execution of it in a
% \m{toks} register. In this example we assume "\l_tmpa_tlp" contains
% the text string "lur". The straight forward approach is
% \begin{quote}
% "\toks_set:No \l_tmpa_toks {\cs_set_eq:Nc \aaa {b \l_tmpa_tlp b}}"
% \end{quote}
% Unfortunately this only puts
% "\exp_args:NNc \cs_set_eq:NN \aaa {b \l_tmpa_tlp b}" into "\l_tmpa_toks"
% and not "\cs_set_eq:NwN \aaa = \blurb" as we probably wanted. Using
% "\toks_set:Nx" is not an option as that will die horribly. Instead
% we can do a
% \begin{quote}
% "\toks_set:Nf \l_tmpa_toks {\cs_set_eq:Nc \aaa {b \l_tmpa_tlp b}}"
% \end{quote}
% which puts the desired result in "\l_tmpa_toks". It requires
% "\toks_set:Nf" to be defined as
% \begin{quote}
% "\cs_set_nopar:Npn \toks_set:Nf {\exp_args:NNf \toks_set:Nn}"
% \end{quote}
% If you use this type of expansion in conditional processing then
% you should stick to using "TF" type functions only as it does not
% try to finish any "\if... \fi:" itself!
%
%
% When pdf\TeX\ 1.50 arrives, it will contain a primitive for
% performing the equivalent of an "x" expansion after only one
% expansion and most importantly: as an expandable operation.
% \begin{function}{\exp_arg:x}
% \begin{syntax}
%   "\exp_arg:x" \Arg{arg}
% \end{syntax}
% <arg> is expanded fully using an "x" expansion.
% \end{function}
%
%
% \subsection{Manipulating the first argument}
%
% \begin{function}{ \exp_args:No / (EXP) }
% \begin{syntax}
%   " \exp_args:No" <funct> <arg1> <arg2> "..."
% \end{syntax}
% The first argument of <funct> (i.e., <arg1>) is expanded once, the
% result is surrounded by braces and passed to <funct>. <funct> may have
% more than one argument---all others are passed unchanged.
% \end{function}
%
% \begin{function}{ \exp_args:Nd / (EXP) }
% \begin{syntax}
%   " \exp_args:Nd" <funct> <arg1> <arg2> "..."
% \end{syntax}
% The first argument of <funct> (i.e., <arg1>) is expanded twice, the
% result is surrounded by braces and passed to <funct>. <funct> may have
% more than one argument---all others are passed unchanged.
% \end{function}
%
% \begin{function}{ \exp_args:Nc / (EXP) }
% \begin{syntax}
%   " \exp_args:Nc" <funct> <arg1> <arg2> "..."
% \end{syntax}
% The first argument of <funct> (i.e., <arg1>) is expanded until only
% characters remain. (An internal error occurs if something else is the
% result of this expansion.) Then the result is turned into a control
% sequence and passed to <funct> as the first argument. <funct> may have
% more than one argument---all others are passed unchanged.
% \end{function}
%
% \begin{function}{ \exp_args:NV / (EXP) }
% \begin{syntax}
%   " \exp_args:NV" <funct> <register>
% \end{syntax}
% The first argument of <funct> (i.e., <register>) is expanded to its
% value. By value we mean a number stored in an |int| or |num|
% register, the length value of a |dim|, |skip| or |muskip| register, the
% contents of a |toks| register or the unexpanded contents of a |tlp|
% register. The value is passed onto <funct> in braces.
% \end{function}
%
% \begin{function}{ \exp_args:Nv / (EXP) }
% \begin{syntax}
%   " \exp_args:Nv" <funct> \Arg{register}
% \end{syntax}
% Like the |V| type except the register is given by a list of
% characters from which a control sequence name is generated.
% \end{function}
%
% \begin{function}{ \exp_args:Nx }
% \begin{syntax}
%   " \exp_args:Nx" <funct> <arg1> <arg2> "..."
% \end{syntax}
% The first argument of <funct> (i.e., <arg1>) is fully expanded until
% only unexpandable tokens remain, the result is surrounded by braces
% and passed to <funct>. <funct> may have more than one argument---all
% others are passed unchanged.
% As mentioned before, this type of function is relatively slow.
% \end{function}
%
% \begin{function}{ \exp_args:Nf / (EXP) }
% \begin{syntax}
%   " \exp_args:Nf" <funct> <arg1> <arg2> "..."
% \end{syntax}
% The first argument of <funct> (i.e., <arg1>) undergoes full
% expansion until the first unexpandable token is encountered, the
% result is surrounded by braces and passed to <funct>. <funct> may
% have more than one argument---all others are passed unchanged.
% Beware of its special behavior as explained above.
% \end{function}
%
% \begin{function}{ \exp_args:NE / (EXP) }
% \begin{syntax}
%   " \exp_args:NE" <funct> <arg1> <arg2> "..."
% \end{syntax}
% The first argument of <funct> (i.e., <arg1>) is expanded once, the
% result is \emph{not} surrounded by braces and passed to <funct>.
% \end{function}
%
% \subsection{Manipulating two arguments}
%
% \begin{function}{%
%                  \exp_args:NNx |
%                  \exp_args:Nnx |
%                  \exp_args:Ncx |
%                  \exp_args:Nox |
%                  \exp_args:Nxo |
%                  \exp_args:Nxx |
% }
% \begin{syntax}
%   "\exp_args:Nnx" <funct> <arg1> <arg2> "..."
% \end{syntax}
% The above functions all manipulate the first two arguments of <funct>.
% They are all slow and non-expandable.
% \end{function}
%
% \begin{function}{%
%                  \exp_args:NNo / (EXP) |
%                  \exp_args:NNc / (EXP) |
%                  \exp_args:NNv / (EXP) |
%                  \exp_args:NNV / (EXP) |
%                  \exp_args:NNE / (EXP) |
%                  \exp_args:NNd / (EXP) |
%                  \exp_args:NNf / (EXP) |
%                  \exp_args:Nno / (EXP) |
%                  \exp_args:Nnf / (EXP) |
%                  \exp_args:Noo / (EXP) |
%                  \exp_args:Noc / (EXP) |
%                  \exp_args:Nco / (EXP) |
%                  \exp_args:Ncc / (EXP) |
%                  \exp_args:NcE / (EXP) |
%                  \exp_args:Nfo / (EXP) |
%                  \exp_args:NEE / (EXP) |
%                  \exp_args:NVV / (EXP) |
% }
% \begin{syntax}
%   "\exp_args:NNo" <funct> <arg1> <arg2> "..."
% \end{syntax}
% These are the fast and expandable functions for the first two arguments.
% \end{function}
%
% \subsection{Manipulating three arguments}
%
% So far not all possible functions are provided and even the selection
% below may be reduced in the future as far as the non-expandable
% functions are concerned.
%
% \begin{function}{%
%                  \exp_args:NNnx |
%                  \exp_args:NNox |
%                  \exp_args:Nnnx |
%                  \exp_args:Nnox |
%                  \exp_args:Noox |
%                  \exp_args:Ncnx |
%                  \exp_args:Nccx |
% }
% \begin{syntax}
%   "\exp_args:Nnnx" <funct> <arg1> <arg2> <arg3> "..."
% \end{syntax}
% All the above functions are non-expandable.
% \end{function}
%
% \begin{function}{%
%                  \exp_args:NNNE / (EXP) |
%                  \exp_args:NNNo / (EXP) |
%                  \exp_args:NNoo / (EXP) |
%                  \exp_args:NNno / (EXP) |
%                  \exp_args:NnnN / (EXP) |
%                  \exp_args:Nnno / (EXP) |
%                  \exp_args:Nnnc / (EXP) |
%                  \exp_args:Nooo / (EXP) |
%                  \exp_args:Nccc / (EXP) |
%                  \exp_args:NcNc / (EXP) |
%                  \exp_args:NcNo / (EXP) |
%                  \exp_args:Ncco / (EXP) |
% }
% \begin{syntax}
%   "\exp_args:NNoo" <funct> <arg1> <arg2> <arg3> "..."
% \end{syntax}
% These are the fast and expandable functions for the first three
% arguments.
% \end{function}
%
% \subsection{Preventing expansion}
%
% \begin{function}{\exp_not:N |
%                  \exp_not:c |
%                  \exp_not:n }
% \begin{syntax}
%   "\exp_not:N" <token>
%   "\exp_not:n" \Arg{token list}
% \end{syntax}
% This function will prohibit the expansion of <token> in situation
% where <token> would otherwise be replaced by it definition, e.g.,
% inside an argument that is handled by the "x" convention.
% \begin{texnote}
% "\exp_not:N" is the primitive \tn{noexpand} renamed and "\exp_not:n"
% is the \eTeX{} primitive \tn{unexpanded}.
% \end{texnote}
% \end{function}
%
% \begin{function}{\exp_not:o |
%                  \exp_not:d |
%                  \exp_not:f }
% \begin{syntax}
%   "\exp_not:o" \Arg{token list}
% \end{syntax}
% Same as "\exp_not:n" except <token list> is expanded once for the
% "o" type and twice for the "d" type and the result of this expansion
% is then prohibited from being expanded further. 
% \end{function}
%
% \begin{function}{\exp_not:V |
%                  \exp_not:v }
% \begin{syntax}
%   "\exp_not:V" <register>
%   "\exp_not:v" \Arg{token list}
% \end{syntax}
% The value of <register> is retrieved and then passed on to
% "\exp_not:n" which will prohibit further expansion. The |v| type
% first creates a control sequence from <token list> but is otherwise
% identical to |V|.
% \end{function}
% 
% \begin{function}{\exp_not:E}
% \begin{syntax}
%   "\exp_not:E" <token>
% \end{syntax}
% The name of this command is a lie. Perhaps it should be called
% ``"exp_perhaps_once"''. What it actually does is, it expands <token>
% and then issues an "\exp_not:N" to prohibit further expansion of the
% first token in the replacement text of <token>. This means that if
% the replacement text of <token> consists of more than one token all
% further tokens are still subject to full expansion.
% \begin{texnote}
% This command has no equivalent.
% \end{texnote}
% \end{function}
% 
%
% \begin{function}{\exp_stop_f:}
% \begin{syntax}
%   <f expansion> ... "\exp_stop_f:" 
% \end{syntax}
% This function stops an "f" type expansion. An example use is one such as
% \begin{verbatim}
% \tlp_set:Nf \l_tmpa_tlp {
%   \if_case:w \l_tmpa_int 
%     \or:   \use_i_after_orelse:nw {\exp_stop_f: \textbullet}
%     \or:   \use_i_after_orelse:nw {\exp_stop_f: \textendash}
%     \else: \use_i_after_fi:nw     {\exp_stop_f: else-item}
%   \fi:
% }
% \end{verbatim}
% This ensures the expansion in stopped right after finishing the
% conditional but without expanding "\textbullet" etc.
% \begin{texnote}
%   This function is a space token but it is better to distinguish
%   this expansion stopping token from a desired space token when
%   writing code.
% \end{texnote}
% \end{function}
%
% \StopEventually{}
%
% \subsection{Internal functions and variables}
%
% \begin{function}{\exp_after:NN}
% \begin{syntax}
%   "\exp_after:NN" <token1> <token2>
% \end{syntax}
% This will expand <token2> once before processing <token1>. This is
% similar to "\exp_args:No" except that no braces are put around the
% result of expanding <token2>.
% \begin{texnote}
% This is the primitive \tn{expandafter} which was renamed to fit into
% the naming conventions of \LaTeX3.
% \end{texnote}
% \end{function}
%
% \begin{variable}{\l_exp_tlp}
% \begin{syntax}\end{syntax}
% The "\exp_" module has its private variables to temporarily store
% results of the argument expansion. This is done to avoid interference
% with other functions using temporary variables.
% \end{variable}
%
% \begin{function}{ \exp_eval_register:N / (EXP) | 
%                   \exp_eval_register:c / (EXP) }
% \begin{syntax}
% "\exp_eval_register:N" <register>
% \end{syntax}
% These functions evaluates a register as part of a "V" or "v" expansion 
% (respectively). A register might exist as
% one of two things: A parameter-less non-long, non-protected macro
% or a built-in \TeX\ register such as |\count|.
% \end{function}
%
% \begin{function}{\::n|\::N|\::c|\::o|\::f|\::x|\::v|\::V|\::E|\::e|\::d|\:::}
% \begin{syntax}
% "\cs_set_nopar:Npn \exp_args:Ncof {\::c\::o\::f\:::}"
% \end{syntax}
% Internal forms for the base expansion types.
% \end{function}
%
% \subsection{The Implementation}
%
%    We start by ensuring that the required packages are loaded.
%    \begin{macrocode}
%<package>\ProvidesExplPackage
%<package>  {\filename}{\filedate}{\fileversion}{\filedescription}
%<package>\RequirePackage{l3tlp}
%<*initex|package>
%    \end{macrocode}
%
% \begin{macro}{\exp_after:NN}
% \begin{macro}{\exp_not:N}
% \begin{macro}{\exp_not:n}
% These are defined in \pkg{l3basics}.
%    \begin{macrocode}
%<*bootstrap>
\cs_set_eq:NwN   \exp_after:NN       \tex_expandafter:D
\cs_set_eq:NwN   \exp_not:N          \tex_noexpand:D
\cs_set_eq:NwN   \exp_not:n          \etex_unexpanded:D
%</bootstrap>
%    \end{macrocode}
% \end{macro}
% \end{macro}
% \end{macro}
%
% \subsubsection{General expansion}
%
% In this section a general mechanism for defining functions to handle
% argument handling is defined.  These general expansion functions are
% expandable unless |x| is used.  (Any version of |x| is going to have
% to use one of the \LaTeX3\ names for |\cs_set_nopar:Npx| at some point, and
% so is never going to be expandable.\footnote{However, some
%   primitives have certain characteristics that means that their
%   arguments undergo an \texttt{x} type expansion but the primitive
%   is in fact still expandable. We shall make it very clear when such
%   a function is expandable.})
%
% The definition of expansion functions with this technique happens
% in section~\ref{sec:gendef}.
% In section~\ref{sec:handtune} some common cases are coded by a more direct
% method for efficiency, typically using calls to |\exp_after:NN|.
%
% \begin{macro}{\l_exp_tlp}
%    We need a scratch token list pointer.
%    \begin{macrocode}
\tlp_new:N \l_exp_tlp
%    \end{macrocode}
% \end{macro}
%
% This code uses internal functions with names that start with |\::|
% to perform the expansions. All macros are |long| as this turned out
% to be desirable since the tokens undergoing expansion may be
% arbitrary user input.
%
% An argument manipulator |\::|\meta{Z} always has signature |#1\:::#2#3|
% where |#1| holds the remaining argument manipulations to be performed,
% |\:::| serves as an end marker for the list of manipulations, |#2|
% is the carried over result of the previous expansion steps and |#3| is
% the argument about to be processed.
%
%  \begin{macro}[aux]{\exp_arg_next:nnn}
%  \begin{macro}[aux]{\exp_arg_next_nobrace:nnn}
%    |#1| is the result of an expansion step, |#2| is the remaining
%    argument manipulations and |#3| is the current result of the
%    expansion chain.  This auxilliary function moves |#1| back after
%    |#3| in the input stream and checks if any expansion is left to
%    be done by calling |#2|. In by far the most cases we will require
%    to add a set of braces to the result of an argument manipulation
%    so it is more effective to do it directly here. Actually, so far
%    only the |c| of the final argument manipulation variants does not
%    require a set of braces.
%    \begin{macrocode}
\cs_new:Npn\exp_arg_next:nnn#1#2#3{
  #2\:::{#3{#1}}
}
\cs_new:Npn\exp_arg_next_nobrace:nnn#1#2#3{
  #2\:::{#3#1}
}
%    \end{macrocode}
% \end{macro}
% \end{macro}
%
%  \begin{macro}{\:::}
%    The end marker is just another name for the identity function.
%    \begin{macrocode}
\cs_new:Npn\:::#1{#1}
%    \end{macrocode}
% \end{macro}
%
%  \begin{macro}{\::n}
%    This function is used to skip an argument that doesn't need to
%    be expanded.
%    \begin{macrocode}
\cs_new:Npn\::n#1\:::#2#3{
  #1\:::{#2{#3}}
}
%    \end{macrocode}
% \end{macro}
%
%  \begin{macro}{\::N}
%    This function is used to skip an argument that consists of a
%    single token and doesn't need to be expanded.
%    \begin{macrocode}
\cs_new:Npn\::N#1\:::#2#3{
  #1\:::{#2#3}
}
%    \end{macrocode}
% \end{macro}
%
%  \begin{macro}{\::c}
%    This function is used to skip an argument that is turned into
%    as control sequence without expansion.
%    \begin{macrocode}
\cs_new:Npn\::c#1\:::#2#3{
  \exp_after:NN\exp_arg_next_nobrace:nnn\cs:w #3\cs_end:{#1}{#2}
}
%    \end{macrocode}
% \end{macro}
%
%  \begin{macro}{\::o}
%    This function is used to expand an argument once.
%    \begin{macrocode}
\cs_new:Npn\::o#1\:::#2#3{
  \exp_after:NN\exp_arg_next:nnn\exp_after:NN{#3}{#1}{#2}
}
%    \end{macrocode}
% \end{macro}
%
%
%  \begin{macro}{\::f}
%  \begin{macro}{\exp_stop_f:}
%    This function is used to expand a token list until the first
%    unexpandable token is found. The underlying "\tex_romannumeral:D -`0"
%    expands everything in its way to find something terminating the
%    number and thereby expands the function in front of it. This
%    scanning procedure is terminated once the expansion hits
%    something non-expandable or a space. We introduce "\exp_stop_f:"
%    to mark such an end of expansion marker; in case the scanner hits
%    a number, this number also terminates the scanning and is left
%    untouched. In the example shown earlier the scanning was stopped
%    once \TeX{} had fully expanded "\cs_set_eq:Nc \aaa {b \l_tmpa_tlp b}"
%    into "\cs_set_eq:NwN \aaa = \blurb" which then turned out to contain
%    the non-expandable token "\cs_set_eq:NwN".  Since the expansion of
%    "\tex_romannumeral:D -`0" is \m{null}, we wind up with a fully
%    expanded list, only \TeX{} has not tried to execute any of the
%    non-expandable tokens. This is what differentiates this function
%    from the "x" argument type.
%    \begin{macrocode}
\cs_new:Npn\::f#1\:::#2#3{
  \exp_after:NN\exp_arg_next:nnn
  \exp_after:NN{\tex_romannumeral:D -`0 #3}
  {#1}{#2}
}
\cs_new_nopar:Npn \exp_stop_f: {~}
%    \end{macrocode}
% \end{macro}
% \end{macro}
%
%  \begin{macro}{\::x,\exp_arg:x}
%    This function is used to expand an argument fully. If the
%    pdf\TeX\ primitive "\expanded" is present, we use it.
%    \begin{macrocode}
\cs_new_eq:NN \exp_arg:x \expanded % Move eventually.
\cs_if_free:NTF\exp_arg:x{
  \cs_new:Npn\::x#1\:::#2#3{
    % \tlp_set:Nx\l_exp_tlp{{{#3}}}
    \cs_set_nopar:Npx \l_exp_tlp{{#3}}
    \exp_after:NN\exp_arg_next:nnn\l_exp_tlp{#1}{#2}}
}
{
  \cs_new:Npn\::x#1\:::#2#3{
    \exp_after:NN\exp_arg_next:nnn
    \exp_after:NN{\exp_arg:x{{#3}}}{#1}{#2}
  }
}
%    \end{macrocode}
% \end{macro}
%
%
%  \begin{macro}{\::v}
%  \begin{macro}{\::V}
%    These functions return the value of a register, i.e., one of
%    |tlp|, |num|, |int|, |skip|, |dim| and |muskip|. The |V| version
%    expects a single token whereas |v| like |c| creates a csname from
%    its argument given in braces and then evaluates it as if it was a
%    |V|. The sequence |\tex_romannumeral:D -`0| sets off an |f| type
%    expansion. The argument is returned in braces.
%    \begin{macrocode}
\cs_new:Npn \::V#1\:::#2#3{
  \exp_after:NN\exp_arg_next:nnn
  \exp_after:NN{
    \tex_romannumeral:D -`0 
    \exp_eval_register:N #3
  }
  {#1}{#2}
}
\cs_new:Npn \::v#1\:::#2#3{
  \exp_after:NN\exp_arg_next:nnn
  \exp_after:NN{
    \tex_romannumeral:D -`0 
    \exp_eval_register:c {#3}
  }
  {#1}{#2}
}
%    \end{macrocode}
% \end{macro}
% \end{macro}
%
%  \begin{macro}{\exp_eval_register:N}
%  \begin{macro}{\exp_eval_register:c}
%  \begin{macro}[aux]{\exp_eval_register_aux:w}
%    This function evaluates a register. Now a register might exist as
%    one of two things: A parameter-less non-long, non-protected macro
%    or a built-in \TeX\ register such as |\count|. For the \TeX\
%    registers we have to utilize a |\tex_the:D| whereas for the
%    macros we merely have to expand them once. The trick is to find
%    out when to use |\tex_the:D| and when not to. What we do here is
%    to take a look at the meaning of the register in question. A
%    macro register will have meaning |macro:->foo bar| whereas a
%    \TeX\ register will have meaning |\count240|. If the seventh
%    token of the meaning is a dash, then we can be fairly sure it was
%    a parameter-less macro.\footnote{It could also be a macro with
%    delimited arguments where the first token it expects to see is a
%    dash but checking for this unlikely case is not worth the
%    trouble.} The auxilliary macro therefore simply checks the
%    seventh token and then has to execute a |\tex_the:D| if it was
%    not a |-| and expand the macro once if it was.
%    \begin{macrocode}
\cs_set_nopar:Npn \exp_eval_register:N #1{
  \exp_after:NN \exp_eval_register_aux:w \token_to_meaning:N #1xxxxxx
  \q_nil
%    \end{macrocode}
% The next bit requires some explanation. The function must be
% initiated by the sequence |\tex_romannumeral:D -`0| and we want to 
% terminate this expansion chain by inserting an |\exp_stop:f|
% token. However, we have to expand the register |#1| before we do
% that. If it is a \TeX\ register, we need to execute the sequence
% |\exp_after:NN\exp_stop_f:\tex_the:D #1| and if it is a macro we
% need to execute |\exp_after:NN\exp_stop_f: #1|. We therefore issue
% the longer of the two sequences and if the register is a macro, we
% remove the |\tex_the:D|.
%    \begin{macrocode}
  \exp_after:NN \exp_stop_f: \tex_the:D #1
}
\cs_set_nopar:Npn \exp_eval_register_aux:w #1#2#3#4#5#6#7#8\q_nil{
 \if:w -#7  \exp_after:NN \use_i_ii:nnn \fi:
}
\cs_set:Npn \use_i_ii:nnn #1#2#3{#1#2}
\cs_set_nopar:Npn \exp_eval_register:c #1{
  \exp_after:NN\exp_eval_register:N\cs:w #1\cs_end:
}
%    \end{macrocode}
%  \end{macro}
%  \end{macro}
%  \end{macro}
%
%
%  Here are some that might not stay but let's see.
%  \begin{macro}{\::E}
%    This function is used to expand an argument once and return it
%    \emph{without} braces. Use this only when you feel pretty
%    comfortable about your input! Actually this is pretty much just
%    generic wrapper for |\exp_after:NN|.
%    \begin{macrocode}
\cs_new:Npn\::E#1\:::#2#3{
  \exp_after:NN\exp_arg_next_nobrace:nnn \exp_after:NN{#3}{#1}{#2}
}
%    \end{macrocode}
% \end{macro}
%
%  \begin{macro}{\::e}
%    Same as |\::E| really but conceptually they are different.
%    This isn't used and should be removed.
%    \begin{macrocode}
\cs_new:Npn\::e#1\:::#2#3{
  \exp_after:NN\exp_arg_next_nobrace:nnn \exp_after:NN{#3}{#1}{#2}
}
%    \end{macrocode}
% \end{macro}
%
%  \begin{macro}{\::d}
%    This function is used to expand an argument twice. Mostly useful
%    for |toks| type things. This is likely to vanish with |V| around.
%    \begin{macrocode}
\cs_new:Npn\::d#1\:::#2#3{
  \exp_after:NN\exp_after:NN\exp_after:NN\exp_arg_next:nnn
  \exp_after:NN\exp_after:NN\exp_after:NN{#3}{#1}{#2}
}
%    \end{macrocode}
% \end{macro}
%
% \subsubsection{Hand-tuned definitions}
% \label{sec:handtune}
%
% One of the most important features of these functions is that they
% are fully expandable and therefore allow to prefix them with
% |\pref_global:D| for example. This together with the fact that the
% `general' concept above is slower means that we should convert
% whenever possible and perhaps remove all remaining occurences by
% hand-encoding in the end.
%
% \begin{macro}{\exp_args:No}
% \begin{macro}{\exp_args:NNo}
% \begin{macro}{\exp_args:NNNo}
%    \begin{macrocode}
\cs_new:Npn \exp_args:No #1#2{\exp_after:NN#1\exp_after:NN{#2}}
\cs_new:Npn \exp_args:NNo #1#2#3{\exp_after:NN#1\exp_after:NN#2
  \exp_after:NN{#3}}
\cs_new:Npn \exp_args:NNNo #1#2#3#4{\exp_after:NN#1\exp_after:NN#2
  \exp_after:NN#3\exp_after:NN{#4}}
%    \end{macrocode}
% \end{macro}
% \end{macro}
% \end{macro}
%
%
% \begin{macro}{\exp_args:Nc,\exp_args:NNc,\exp_args:Ncc,\exp_args:Nccc}
%    Here are the functions that turn their argument into csnames but
%    are  expandable.
%    \begin{macrocode}
\cs_new:Npn \exp_args:Nc #1#2{\exp_after:NN#1\cs:w#2\cs_end:}
\cs_new:Npn \exp_args:NNc #1#2#3{\exp_after:NN#1\exp_after:NN#2
    \cs:w#3\cs_end:}
\cs_new:Npn \exp_args:Ncc #1#2#3{\exp_after:NN#1
    \cs:w#2\exp_after:NN\cs_end:\cs:w#3\cs_end:}
\cs_new:Npn \exp_args:Nccc #1#2#3#4{\exp_after:NN#1
    \cs:w#2\exp_after:NN\cs_end:\cs:w#3\exp_after:NN
      \cs_end:\cs:w #4\cs_end:}
%    \end{macrocode}
% \end{macro}
%
%  \begin{macro}{\exp_args:Nco}
%    If we force that the third argument
%    always has braces, we could implement this function
%    with less tokens and only two arguments.
%    \begin{macrocode}
\cs_new:Npn \exp_args:Nco #1#2#3{\exp_after:NN#1\cs:w#2\exp_after:NN
     \cs_end:\exp_after:NN{#3}}
%    \end{macrocode}
%  \end{macro}
%
%  \begin{macro}{\exp_args:NE,\exp_args:NNE,\exp_args:NNNE,
%    \exp_args:NEE,\exp_args:NcE}
% We do most of them by hand here. This also means that we get a name
% for |\exp_after:NN| that fits with the rest of the code.
%    \begin{macrocode}
\cs_set_eq:NN  \exp_args:NE \exp_after:NN
\cs_set_nopar:Npn \exp_args:NNE #1{\exp_args:NE#1\exp_args:NE}
\cs_set_nopar:Npn \exp_args:NNNE #1#2{\exp_args:NE#1\exp_args:NE#2\exp_args:NE}
\cs_set_nopar:Npn \exp_args:NEE #1{\exp_args:NE\exp_args:NE\exp_args:NE#1\exp_args:NE}
\cs_set_nopar:Npn \exp_args:NcE #1#2{\exp_after:NN #1\cs:w #2\exp_after:NN\cs_end:}
%    \end{macrocode}
% \end{macro}
%
% \subsubsection{Definitions with the `general' technique}
% \label{sec:gendef}
%
%  \begin{macro}{\exp_args:Nd,\exp_args:Nf,\exp_args:NV,\exp_args:Nv,\exp_args:Nx}
%    \begin{macrocode}
\cs_set_nopar:Npn \exp_args:Nd {\::d\:::}
\cs_set_nopar:Npn \exp_args:Nf {\::f\:::}
\cs_set_nopar:Npn \exp_args:Nv {\::v\:::}
\cs_set_nopar:Npn \exp_args:NV {\::V\:::}
\cs_set_nopar:Npn \exp_args:Nx {\::x\:::}
%    \end{macrocode}
%  \end{macro}
%
%  \begin{macro}{\exp_args:NNV,\exp_args:NNv,\exp_args:NNd,\exp_args:NNf,\exp_args:NNx,
%                \exp_args:NVV,
%                \exp_args:Ncx,
%                \exp_args:Nfo,
%                \exp_args:Nnf,\exp_args:Nno,\exp_args:Nnx,
%                \exp_args:Noo,\exp_args:Noc,\exp_args:Nox,
%                \exp_args:Nxo,\exp_args:Nxx}
%    Here are the actual function definitions, using the helper functions
%    above.
%    \begin{macrocode}
\cs_set_nopar:Npn \exp_args:NNd {\::N\::d\:::}
\cs_set_nopar:Npn \exp_args:NNf {\::N\::f\:::}
\cs_set_nopar:Npn \exp_args:NNv {\::N\::v\:::}
\cs_set_nopar:Npn \exp_args:NNV {\::N\::V\:::}
\cs_set_nopar:Npn \exp_args:NNx {\::N\::x\:::}

\cs_set_nopar:Npn \exp_args:Ncx {\::c\::x\:::}
\cs_set_nopar:Npn \exp_args:Nfo {\::f\::o\:::}

\cs_set_nopar:Npn \exp_args:Nnf {\::n\::f\:::}
\cs_set_nopar:Npn \exp_args:Nno {\::n\::o\:::}
\cs_set_nopar:Npn \exp_args:Nnx {\::n\::x\:::}

\cs_set_nopar:Npn \exp_args:Noc {\::o\::c\:::}
\cs_set_nopar:Npn \exp_args:Noo {\::o\::o\:::}
\cs_set_nopar:Npn \exp_args:Nox {\::o\::x\:::}

\cs_set_nopar:Npn \exp_args:NVV {\::V\::V\:::}

\cs_set_nopar:Npn \exp_args:Nxo {\::x\::o\:::}
\cs_set_nopar:Npn \exp_args:Nxx {\::x\::x\:::}
%    \end{macrocode}
%  \end{macro}
% 
%  \begin{macro}{\exp_args:Ncco,
%                \exp_args:Nccx,
%                \exp_args:Ncnx,
%                \exp_args:NcNc,
%                \exp_args:NcNo,
%                \exp_args:NNno,
%                \exp_args:NnnN,
%                \exp_args:Nnno,
%                \exp_args:Nnnx,
%                \exp_args:Nnox,
%                \exp_args:Nooo,
%                \exp_args:Noox,
%                \exp_args:Nnnc,
%                \exp_args:NNnx,
%                \exp_args:NNoo,
%                \exp_args:NNox}
%    \begin{macrocode}
\cs_set_nopar:Npn \exp_args:NNno {\::N\::n\::o\:::}
\cs_set_nopar:Npn \exp_args:NNnx {\::N\::n\::x\:::}
\cs_set_nopar:Npn \exp_args:NNoo {\::N\::o\::o\:::}
\cs_set_nopar:Npn \exp_args:NNox {\::N\::o\::x\:::}

\cs_set_nopar:Npn \exp_args:NnnN {\::n\::n\::N\:::}   %% Strange one this one...
\cs_set_nopar:Npn \exp_args:Nnnc {\::n\::n\::c\:::}
\cs_set_nopar:Npn \exp_args:Nnno {\::n\::n\::o\:::}
\cs_set_nopar:Npn \exp_args:Nnnx {\::n\::n\::x\:::}
\cs_set_nopar:Npn \exp_args:Nnox {\::n\::o\::x\:::}

\cs_set_nopar:Npn \exp_args:NcNc {\::c\::N\::c\:::}
\cs_set_nopar:Npn \exp_args:NcNo {\::c\::N\::o\:::}
\cs_set_nopar:Npn \exp_args:Ncco {\::c\::c\::o\:::}
\cs_set_nopar:Npn \exp_args:Ncco {\::c\::c\::o\:::}
\cs_set_nopar:Npn \exp_args:Nccx {\::c\::c\::x\:::}
\cs_set_nopar:Npn \exp_args:Ncnx {\::c\::n\::x\:::}

\cs_set_nopar:Npn \exp_args:Noox {\::o\::o\::x\:::}
\cs_set_nopar:Npn \exp_args:Nooo {\::o\::o\::o\:::}
%    \end{macrocode}
%  \end{macro}
%
%
% \subsubsection{Preventing expansion}
%
%
%  \begin{macro}{\exp_not:o}
%  \begin{macro}{\exp_not:d}
%  \begin{macro}{\exp_not:f}
%  \begin{macro}{\exp_not:v}
%  \begin{macro}{\exp_not:V}
%    \begin{macrocode}
\cs_new:Npn\exp_not:o#1{\exp_not:n\exp_after:NN{#1}}
\cs_new:Npn\exp_not:d#1{
  \exp_not:n\exp_after:NN\exp_after:NN\exp_after:NN{#1}
}
\cs_new:Npn\exp_not:f#1{
  \exp_not:n\exp_after:NN{\tex_romannumeral:D -`0 #1}
}
\cs_new:Npn\exp_not:v#1{
  \exp_not:n\exp_after:NN{\tex_romannumeral:D -`0 \exp_eval_register:c {#1}}
}
\cs_new:Npn\exp_not:V#1{
  \exp_not:n\exp_after:NN{\tex_romannumeral:D -`0 \exp_eval_register:N #1}
}
%    \end{macrocode}
%  \end{macro}
%  \end{macro}
%  \end{macro}
%  \end{macro}
%  \end{macro}
% 
%  \begin{macro}{\exp_not:E}
%  \begin{macro}{\exp_not:c}
%    Two helper functions, which we can probably live without it.
%    \begin{macrocode}
\cs_new_nopar:Npn\exp_not:E{\exp_after:NN\exp_not:N}
\cs_new:Npn\exp_not:c#1{\exp_after:NN\exp_not:N\cs:w#1\cs_end:}
%    \end{macrocode}
%  \end{macro}
%  \end{macro}
%
% \subsubsection{Higher-level functions}
%
%  \begin{macro}{\exp_def_form:nnn}
%    This command is a recent addition which was actually added
%    when we wrote the article for TUGboat (while most of the other
%    code goes way back to 1993).
%    \begin{macrocode}
\cs_set_nopar:Npn\exp_def_form:nnn#1#2#3{
   \exp_after:NN
   \cs_set_nopar:Npn
     \cs:w
        #1:#3
       \exp_after:NN
     \cs_end:
     \exp_after:NN
       {
        \cs:w
           exp_args:N#3
          \exp_after:NN
        \cs_end:
        \cs:w
           #1:#2
        \cs_end:
       }
%    \end{macrocode}
%    We also have to test if |exp_args:N#3| is already defined
%    and if not define it via the
%    |\::| commands using the chars in |#3|
%    \begin{macrocode}
    \cs_if_free:cT
          {exp_args:N#3}
          {\cs_set_nopar:cpx {exp_args:N#3}
                    {\exp_args_form_x:w #3 :}
          }
}
%    \end{macrocode}
%  \end{macro}
%
%
%  \begin{macro}[aux]{\exp_args_form_x:w}
%    This command grabs char by char outputting |\::#1| (not expanded
%    further) until we see a |:|. That colon is in fact also turned into
%    |\:::| so that the required structure for |\exp_args...| commands
%    is correctly terminated.
%    \begin{macrocode}
\cs_new_nopar:Npn\exp_args_form_x:w #1 {
  \exp_not:c{::#1}
  \if_meaning:NN #1 :
  \else:
    \exp_after:NN\exp_args_form_x:w
  \fi:}
%    \end{macrocode}
%  \end{macro}
%
%    \begin{macrocode}
%</initex|package>
%    \end{macrocode}
% 
%    Show token usage:
%    \begin{macrocode}
%<*showmemory>
\showMemUsage
%</showmemory>
%    \end{macrocode}
%
% \Finale
% \PrintIndex
%
% \endinput
