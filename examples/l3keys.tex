
%% NB: this file needs updating and does not work with the current version of expl3!

\documentclass{article}
\usepackage[T1]{fontenc}
\usepackage{expl3,lmodern,xparse}
\usepackage[parfill]{parskip}

% The document commands for using keys is very simple.

\ExplSyntaxOn
\DeclareDocumentCommand \SetModuleKeys { m } {
  \keys_set:nn { module } {#1}
}
\ExplSyntaxOff

\begin{document}

% Each test starts with a short piece of text explaining the property
% being used. Under this, the expected output is given, followed
% if relevant by the text of any error messages expected.

% Test one: Simply creating a key with one argument and then using it.
% Notice that the text length determines the limits of the argument:
% spaces at either end are ignored. Literal spaces between text are
% counted, as are those enclosed in braces.
% ----------------------------------------------------------------------
%   You said: `Hello World'
%   You said: ` Hello World '
%   You said: `'
% ----------------------------------------------------------------------
\ExplSyntaxOn

\keys_define:nn { module } {
  key .code:n = {You~said:~`#1'\\},
}
\ExplSyntaxOff

\SetModuleKeys{
  key = Hello World ,
  key = { Hello World },
  key
}

% Test two: Variants of .code:n and .code:Nn exist which carry out a
% full expansion on code definition. First the :x variant is tested.
% ----------------------------------------------------------------------
%   Temp holds: ABC
%   Temp holds: 123
% ----------------------------------------------------------------------
\newcommand*{\temp}{123}

\ExplSyntaxOn
\keys_define:nn { module } {
  key-one .code:n = Temp~holds:~\temp\\,
  key-two .code:x = Temp~holds:~\temp,
}
\ExplSyntaxOff

\renewcommand*{\temp}{ABC}

\SetModuleKeys{
  key-one,
  key-two
}

% Test three: Values can be required and forbidden with the
% .value_required: and .value_forbidden: properties.
% ----------------------------------------------------------------------
%   All okay
%   All okay
% ----------------------------------------------------------------------
% The key `module/key-one' cannot taken a value:
% the given input `Not allowed' is being ignored.
%
% The key `module/key-two' requires a value
% and is being ignored.
% ----------------------------------------------------------------------
\ExplSyntaxOn
\keys_define:nn { module } {
  key-one .code:n = All~okay\\,
  key-one .value_forbidden:,
  key-two .code:n = #1,
  key-two .value_required:
}
\ExplSyntaxOff

\SetModuleKeys{
  key-one,
  key-one = Not allowed,
  key-two = All okay,
  key-two
}

% Test four: Keys can be given a default value, to be used if nothing is
% specified by the user. This can be done by name or by value.
% ----------------------------------------------------------------------
%   Default
%   Real
%   Stuff
%   Real
%   Literal
%   Real
% ----------------------------------------------------------------------
\ExplSyntaxOn
\renewcommand*{\temp}{Stuff}
\keys_define:nn { module } {
  key-one .code:n      =  #1\\,
  key-one .default:n   = \temp,
  key-two .code:n      =  #1\\,
  key-two .default:V   = \temp,
  key-three .code:n    =  #1\\,
  key-three .default:n = Literal,
}
\ExplSyntaxOff
\renewcommand*{\temp}{Default}

\SetModuleKeys{
  key-one,
  key-one = Real,
  key-two,
  key-two = Real,
  key-three,
  key-three = Real
}

% Test five: Keys can be created to store data both locally and
% globally. This is illustrated using token list variables.
% ----------------------------------------------------------------------
%   Some content
%   Outside
%   Some content
%   Some content
% ----------------------------------------------------------------------
\ExplSyntaxOn
\tl_new:N \l_temp_tl    \tl_set:Nn  \l_temp_tl {Outside}
\tl_new:N \g_temp_tl    \tl_gset:Nn \g_temp_tl {Outside}
\keys_define:nn { module } {
  key-one .tl_set:N  = \l_temp_tl,
  key-two .tl_gset:N = \g_temp_tl,
}

\begingroup
\SetModuleKeys{
  key-one = Some~content
}
\l_temp_tl\\
\endgroup
\l_temp_tl\\

\begingroup
\SetModuleKeys{
  key-two = Some~content
}
\g_temp_tl\\
\endgroup
\g_temp_tl\\
\ExplSyntaxOff

% Test six: An expanded version for setting token lists.
% ----------------------------------------------------------------------
%   Unexpanded
%   Expanded
% ----------------------------------------------------------------------
\ExplSyntaxOn
\tl_set:Nn \l_temp_tl {Expanded}
\tl_new:N \l_tempa_tl
\tl_new:N \l_tempb_tl
\keys_define:nn { module } {
  key-one .tl_set:N   = \l_tempa_tl,
  key-two .tl_set_x:N = \l_tempb_tl,
}

\SetModuleKeys{
  key-one = \l_temp_tl,
  key-two = \l_temp_tl,
}
\tl_set:Nn \l_temp_tl {Unxpanded}
\l_tempa_tl\\
\l_tempb_tl\\

\ExplSyntaxOff

% Test seven: Choices can be created either from a list of options or
% one by one. Both are tested here by creating a list then adding to it
% with an extra key. Unknown choices lead to an error message.
% ----------------------------------------------------------------------
% Choice `a' is number 1
% Choice `b' is number 2
% Choice `c' is number 3
% An extra choice
% ----------------------------------------------------------------------
% Choice `e' unknown for key `module/key':
% the key is being ignored.
% ----------------------------------------------------------------------
\ExplSyntaxOn
\keys_define:nn { module } {
  key .choice_code:n = {
    Choice~`\l_keys_choice_tl'~is~number~\int_use:N \l_keys_choice_int\\
  },
  key .generate_choices:n = {a,b,c},
  key/d .code:n = An~extra~choice \\
}
\ExplSyntaxOff

\SetModuleKeys{
  key = a,
  key = b,
  key = c,
  key = d,
  key = e
}

% Test eight: Meta-keys for setting sub-choices.
% ----------------------------------------------------------------------
% `Value' and `Value'
% ----------------------------------------------------------------------
\ExplSyntaxOn
\keys_define:nn { module } {
  key-one   .meta:n = { key-two = #1, key-three = #1 },
  key-two   .code:n = `#1'~,
  key-three .code:n = and~`#1'
}
\ExplSyntaxOff
\SetModuleKeys{
  key-one = Value
}

% Test nine: A completely unknown key should cause an error.
% ----------------------------------------------------------------------
% The key `module/unknown key' is unknown and is being ignored.
% ----------------------------------------------------------------------
\SetModuleKeys{
  unknown key = some value
}

% Test ten: If the special `unknown' key has been set up, however,
% unknown input can be re-directed.
% ----------------------------------------------------------------------
% You tried to set key `unknown key' to `some value'
% ----------------------------------------------------------------------
\ExplSyntaxOn
\keys_define:nn { module } {
   unknown .code:n = You~tried~to~set~key~`\l_keys_key_tl'~to~`#1'
}
\ExplSyntaxOff

\SetModuleKeys{
  unknown key = some~value
}

\end{document}

