
\documentclass{article}

\usepackage[T1]{fontenc}
\usepackage{times,multicol}
\usepackage{xcoffins}

\usepackage{color}
\newcommand\cbox[2][.8]{{\setlength\fboxsep{0pt}\colorbox[gray]{#1}{#2}}}

\newcommand\hrulebox [2]{\setbox#1\hbox to#2{{\scriptsize\itshape
                         \color{blue}\hrulefill #2\hrulefill}}}
\newcommand\vrulebox [2]{\setbox#1\vbox to#2{{\hsize 1pt\centering\scriptsize
                         \itshape\color{blue}\leaders\vrule\vfill
                         \hbox to0pt{\hss #2\hss}\leaders\vrule\vfill\par}}}


\addtolength\textwidth{10pt}

\showboxdepth 9999
\showboxbreadth 9999
\tracingonline 1


\scrollmode

\newbox\zzz


\begin{document}


\title{Test file for coffins}
\author{FMi}
\maketitle

First we add a few test coffins:
\begin{verbatim}
  \newcoffin \aaa
  \newcoffin \bbb
  \newcoffin \ccc
  \newcoffin \ddd
\end{verbatim}
  \newcoffin \aaa
  \newcoffin \bbb
  \newcoffin \ccc
  \newcoffin \ddd
and some boxes not set up as ordinary boxes (without extra poles):
\begin{verbatim}
  \newbox \xxx
  \newbox \yyy
\end{verbatim}
  \newbox \xxx
  \newbox \yyy


If a coffin receives data a set of ``natural'' default poles are automatically added.
\begin{verbatim}
\sbox\aaa{\fbox{\begin{tabular}[b]{l}123\\4\\5\end{tabular}}}
\showcoffindata \aaa
\end{verbatim}
\sbox\aaa{\fbox{\begin{tabular}[b]{l}123\\4\\5\end{tabular}}}
\showcoffindata \aaa


In contrast, boxes not declared as coffins have no poles defined. However, if
used as coffins at least the natural default poles can be used without
further problems, i.e.,they are changed to coffins automatically on use.
\begin{verbatim}
\sbox\xxx{\fbox{Some very looooonggg Caption Text}}
\showpoles \xxx
\end{verbatim}
\sbox\xxx{\fbox{Some very looooonggg Caption Text}}
\showcoffindata \xxx


Using \verb|\setvcoffin| instead of \verb|\sbox| gives us the codes with
\texttt{T} and \texttt{B}. Similar to \verb|\parbox| this command requires to
specify a target width of the box:
\begin{verbatim}
\setvcoffin \aaa {1.5cm} {\centering
   \fbox{\begin{tabular}[b]{l}123\\4\\5\end{tabular}}}
\showcoffindata \aaa
\end{verbatim}
\setvcoffin \aaa {1.5cm} {\centering
   \fbox{\begin{tabular}[b]{l}123\\4\\5\end{tabular}}}
\showcoffindata \aaa

There are a dozen natural poles per box (some with multiple names); you can
move each of the poles to someplace else (either to an absolute position in
``box space'' or relative to its previous position).  The coordinates can
refer to box dimensions (using \verb|\height|, \verb|\depth|,
 and \verb|width|). First we move some pole and add one
additional one:
\begin{verbatim}
\setcoffinpole \aaa [h]{t}(\height -3pt)
\setcoffinpole \aaa [v]{l}(0pt)
\setcoffinpole \aaa [h]{mybottom}(\depth +2pt)
\showcoffindata \aaa
\end{verbatim}
\setcoffinpole \aaa [h]{t}(\height -3pt)
\setcoffinpole \aaa [v]{l}(0pt)
\setcoffinpole \aaa [h]{mybottom}(\depth +2pt)
\showcoffindata \aaa


More interesting in many cases is the relative move of poles. For this the
poles better exist or one gets an error and \texttt{0pt,0pt} is used:
\begin{verbatim}
\adjustcoffinpole \aaa [h]{H}(1pt)
\adjustcoffinpole \aaa [v]{l}(1pt)
\adjustcoffinpole \aaa [h]{undefined}(1pt)
\showcoffindata \aaa
\end{verbatim}
\adjustcoffinpole \aaa [h]{H}(1pt)
\adjustcoffinpole \aaa [v]{l}(1pt)
\adjustcoffinpole \aaa [h]{undefined}(1pt)
\showcoffindata \aaa



\sbox \aaa {\fontsize{14.4}{5.5pc minus .5pc}\sffamily C\,H\,A\,P\,T\,E\,R 
            \fontsize{36}{40pt}\selectfont 2}

\setvcoffin \bbb {11cm}
   {\raggedleft\normalfont\fontsize{36}{38pt}\bfseries 
    The Structure of a \LaTeX{} Document}

\setvcoffin \ccc {13cm}
   {Some sample blind text to produce paragraph data after a heading.
    Some sample blind text to produce paragraph data after a heading.
    Some sample blind text to produce paragraph data after a heading.
    Some sample blind text to produce paragraph data after a heading.
    \endgraf
    Some more sample blind text to produce paragraph data after a heading.
    Some more sample blind text to produce paragraph data after a heading.
}


Suppose you have the following blocks of text:

\bigskip

\noindent\cbox{\usebox\aaa}

\begin{multicols}{2}
\ttfamily\tiny
\ExplSyntaxOn
   \noindent \coffin_print_pole_values:Nn \aaa \\ 
\ExplSyntaxOff
\end{multicols}

\medskip

\noindent\cbox{\usebox\bbb}

\medskip

\noindent\cbox{\usebox\ccc}

\bigskip

produced by:
\begin{verbatim}
\sbox \aaa {\fontsize{14.4}{5.5pc minus .5pc}\sffamily 
             C\,H\,A\,P\,T\,E\,R 
            \fontsize{36}{40pt}\selectfont 2}
\setvcoffin \bbb {11cm}
   {\raggedleft\normalfont\fontsize{36}{38pt}\bfseries 
    The Structure of a \LaTeX{} Document}
\setvcoffin \ccc {13cm}
   {Some sample blind text to produce paragraph 
    data after a heading.                        ... }
\end{verbatim}

How do you turn this into a heading of TLC2?


\newbox\RBi
\newbox\RBii
\newbox\RBiii

\hrulebox\RBi{62pt}
\vrulebox\RBii{90pt}
\vrulebox\RBiii{60pt}

\aligncoffins  \bbb [T,r] \aaa [H,r](-62pt,60pt)
\aligncoffins  \ccc[T,r] \bbb[B,r](62pt,90pt)

  \aligncoffins * \ccc[\bbb-T,\bbb-hc] \RBiii[b,hc](28pt,0pt)
  \aligncoffins * \ccc[\bbb-b,\bbb-r] \RBi[H,r](0pt,-5pt)
  \aligncoffins * \ccc[\bbb-B,\bbb-hc] \RBii[t,r](14pt,0pt)

\newpage

How do you best define/describe the following design?

\medskip

\noindent\cbox{\usebox\ccc}






\newpage \pagestyle{empty}

\newcoffin \eee
\newcoffin \fff
\newcoffin \ggg

\sbox \aaa {\small\scshape les vases communicants}
\sbox \bbb {\scshape comunicating}
\sbox \ccc {\fontsize{70pt}{60pt} \bfseries Ve\S els}
\sbox \ddd {\scshape andr\'e breton}
\setvcoffin \eee {4.7cm}{\noindent Translated by Mary Ann Caws \&\break
                    Geoffrey T.\,Harris, with notes \&\break
                    introduction by Mary Ann Caws\parfillskip=0pt\relax
                    }
\sbox \fff {University of Nebraska Press: Lincoln \& London}

\setvcoffin \ggg {100mm}{\noindent\fbox{\parbox{97mm}{\leavevmode\vspace*{228mm}}}}


\aligncoffins   \ccc [H,r] \ddd [H,r](0pt,-12pt)
\aligncoffins   \ccc [t,l] \bbb      (0pt,6pt)
\aligncoffins   \ccc [H,r] \aaa [H,r](0pt,138pt)
\aligncoffins   \eee [T,l] \ccc      (0pt,22pc)
\aligncoffins   \fff       \eee [B,l](0pt,4pc)
\aligncoffins   \ggg [b,l] \fff [B,l](54pt,11pc)


\vspace*{-2cm}
\noindent\cbox{\usebox\ggg}


\newpage

\section{Rotation}

\sbox \aaa {\tabcolsep0pt\begin{tabular}[c]{|c|}\hline a\\b\\cccccccccccccc\\d
                \\e\\e\\e\\e\\\hline\end{tabular}}
\sbox \bbb {\fbox{A sample Text}}

% just for the sake of it:
\adjustcoffinpole \aaa {T}(24pt)
\displaycoffinpoles \aaa {black}
x\cbox{\usebox \aaa }x
%
\rotatecoffin \aaa {45}
\displaycoffinhandle \aaa {b}{hc} {blue}
\displaycoffinhandle \aaa {b}{l} {blue}
\displaycoffinhandle \aaa {t}{r} {blue}
\displaycoffinhandle \aaa {vc}{r} {blue}
\displaycoffinhandle \aaa {vc}{l} {blue}
\quad
x\cbox{\usebox \aaa }x
\rotatecoffin \aaa {45}
\displaycoffinhandle \aaa {b}{r} {red}
\displaycoffinhandle \aaa {vc}{l} {red}
\quad
x\cbox{\usebox \aaa }x

\vspace{1cm}

\rotatecoffin \aaa {45}
\displaycoffinhandle \aaa {b}{l} {yellow}
%
x\cbox{\usebox \aaa }x
%

\vspace{1cm}

x\cbox{\usebox \bbb }x
%
\rotatecoffin \bbb {45}
\displaycoffinhandle \bbb {B}{hc} {yellow}
\quad
x\cbox{\usebox \bbb }x


\section{Rotation + alignment}

\aligncoffins \aaa [b,l] \bbb[B,hc](30pt,0pt)
x\cbox{\usebox \aaa }x


After we have aligned a roated box with some other box we need to decide about
the bounding box of the new box. Right now this become the enclosing box and
we do not maintain information about the inner boxes. So when we rotate that
new box there seems to be unnecessary space in the enclosing bounding box,
even though that is correct if you think of the aligned box being unrotated.

There is at least the possibility to refer to the handles of the inner boxes
as one can see by two of the green handles

\medskip

\rotatecoffin \aaa {45}
\displaycoffinhandle \aaa {vc}{hc} {green}
\displaycoffinhandle \aaa {H}{l} {green}
\displaycoffinhandle \aaa {\aaa-T}{\aaa-r} {green}
\displaycoffinhandle \aaa {\bbb-H}{\bbb-r} {green}
%\quad
x\cbox{\usebox \aaa }x

We could do better, if we want to, by actually checking for max and
min of all inner bounding box corners and construct the result BB box from
that---but is it worth it?

Perhaps it is! After all, the current implementation shows different results
depending on when you align boxes and when you rotate, e.g., aligning first
gives totally different bounding box results. 

\newpage

Aligning first and then 135 + 45 rotation gives this:

\sbox \aaa {\tabcolsep0pt\begin{tabular}[c]{|c|}\hline a\\b\\cccccccccccccc\\d
                \\e\\e\\e\\e\\\hline\end{tabular}}
\sbox \bbb {\fbox{A sample Text}}
\rotatecoffin \bbb {90}

\aligncoffins \aaa [b,l] \bbb[B,hc](-30pt,0pt)
x\cbox{\usebox \aaa }x
\qquad
\rotatecoffin \aaa {135}
x\cbox{\usebox \aaa }x
\qquad
\rotatecoffin \aaa {45}
x\cbox{\usebox \aaa }x




\newpage

\section{Rotation by small amounts}

\sbox \aaa {\tabcolsep0pt\begin{tabular}[c]{|c|}\hline a\\b\\ccccccccc\\d
                \\e\\e\\e\\\hline\end{tabular}}

\subsection{30 + 30 +30 }
\rotatecoffin \aaa {30}
x\cbox{\usebox \aaa }x
\rotatecoffin \aaa {30}
x\cbox{\usebox \aaa }x
\rotatecoffin \aaa {30}
x\cbox{\usebox \aaa }x

\subsection{6 * 10 + 45 + 45 + 30}

\rotatecoffin \aaa {10}
x\cbox{\usebox \aaa }x
\rotatecoffin \aaa {10}
x\cbox{\usebox \aaa }x
\rotatecoffin \aaa {10}
x\cbox{\usebox \aaa }x

\vspace{1cm}
\rotatecoffin \aaa {10}
x\cbox{\usebox \aaa }x
\rotatecoffin \aaa {10}
x\cbox{\usebox \aaa }x
\rotatecoffin \aaa {10}
x\cbox{\usebox \aaa }x

\vspace{1cm}
\rotatecoffin \aaa {45}
x\cbox{\usebox \aaa }x
\rotatecoffin \aaa {45}
x\cbox{\usebox \aaa }x
\rotatecoffin \aaa {30}
x\cbox{\usebox \aaa }x


\end{document}


