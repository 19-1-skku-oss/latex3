% \iffalse meta-comment
%
%% File: l3tl-build.dtx Copyright (C) 2011-2016 The LaTeX3 Project
%%
%% It may be distributed and/or modified under the conditions of the
%% LaTeX Project Public License (LPPL), either version 1.3c of this
%% license or (at your option) any later version.  The latest version
%% of this license is in the file
%%
%%    http://www.latex-project.org/lppl.txt
%%
%% This file is part of the "l3experimental bundle" (The Work in LPPL)
%% and all files in that bundle must be distributed together.
%%
%% The released version of this bundle is available from CTAN.
%%
%% -----------------------------------------------------------------------
%%
%% The development version of the bundle can be found at
%%
%%    http://www.latex-project.org/svnroot/experimental/trunk/
%%
%% for those people who are interested.
%%
%%%%%%%%%%%
%% NOTE: %%
%%%%%%%%%%%
%%
%%   Snapshots taken from the repository represent work in progress and may
%%   not work or may contain conflicting material!  We therefore ask
%%   people _not_ to put them into distributions, archives, etc. without
%%   prior consultation with the LaTeX3 Project.
%%
%% -----------------------------------------------------------------------
%
%<*driver|package>
% The version of expl3 required is tested as early as possible, as
% some really old versions do not define \ProvidesExplPackage.
\RequirePackage{expl3}[2016/05/18]
%<package>\@ifpackagelater{expl3}{2016/05/18}
%<package>  {}
%<package>  {%
%<package>    \PackageError{l3tl-build}{Support package l3kernel too old}
%<package>      {%
%<package>        Please install an up to date version of l3kernel\MessageBreak
%<package>        using your TeX package manager or from CTAN.\MessageBreak
%<package>        \MessageBreak
%<package>        Loading l3tl-build will abort!%
%<package>      }%
%<package>    \endinput
%<package>  }
\GetIdInfo$Id$
  {L3 Experimental token list construction}
%</driver|package>
%<*driver>
\documentclass[full]{l3doc}
\usepackage{amsmath}
\begin{document}
  \DocInput{\jobname.dtx}
\end{document}
%</driver>
% \fi
%
%
% \title{^^A
%   The \textsf{l3tl-build} package: building token lists^^A
%   \thanks{This file describes v\ExplFileVersion,
%     last revised \ExplFileDate.}^^A
% }
%
% \author{^^A
%  The \LaTeX3 Project\thanks
%    {^^A
%      E-mail:
%        \href{mailto:latex-team@latex-project.org}
%          {latex-team@latex-project.org}^^A
%    }^^A
% }
%
% \date{Released \ExplFileDate}
%
% \maketitle
%
% \begin{documentation}
%
% \section{\pkg{l3tl-build} documentation}
%
% This module provides no user function: it is meant for kernel use
% only.
%
% There are two main ways of building token lists from individual
% tokens. Either in one go within an \texttt{x}-expanding assignment, or
% by repeatedly using \cs{tl_put_right:Nn}. The first method takes a
% linear time, but only allows expandable operations. The second method
% takes a time quadratic in the length of the token list, but allows
% expandable and non-expandable operations.
%
% The goal of this module is to provide functions to build a token list
% piece by piece in linear time, while allowing non-expandable
% operations. This is achieved by abusing \tn{toks}: adding some tokens
% to the token list is done by storing them in a free token register
% (time $O(1)$ for each such operation). Those token registers are only
% put together at the end, within an \texttt{x}-expanding assignment,
% which takes a linear time.\footnote{If we run out of token registers,
%   then the currently filled-up \tn{toks} are put together in a
%   temporary token list, and cleared, and we ultimately use
%   \cs{tl_put_right:Nx} to put those chunks together. Hence the true
%   asymptotic is quadratic, with a very small constant.}  Of course,
% all this must be done in a group: we can't go and clobber the values
% of legitimate \tn{toks} used by \LaTeXe{}.
%
% Since none of the current applications need the ability to insert
% material on the left of the token list, I have not implemented
% that. This could be done for instance by using odd-numbered \tn{toks}
% for the left part, and even-numbered \tn{toks} for the right part.
%
% \subsection{Internal functions}
%
% \begin{function}
%   {
%     \__tl_build:Nw, \__tl_gbuild:Nw,
%     \__tl_build_x:Nw, \__tl_gbuild_x:Nw
%   }
%   \begin{syntax}
%     \cs{__tl_build:Nw} \meta{tl~var} \texttt{\ldots{}}
%     \cs{__tl_build_one:n} \Arg{tokens_1} \texttt{\ldots{}}
%     \cs{__tl_build_one:n} \Arg{tokens_2} \texttt{\ldots{}}
%     \ldots{}
%     \cs{__tl_build_end:}
%   \end{syntax}
%   Defines the \meta{tl~var} to contain the contents of \meta{tokens1}
%   followed by \meta{tokens2}, \emph{etc.} This is built in such a way
%   to be more efficient than repeatedly using \cs{tl_put_right:Nn}. The
%   code in \enquote{\texttt{\ldots{}}} does not need to be
%   expandable. The commands \cs{__tl_build:Nw} and \cs{__tl_build_end:}
%   start and end a group. The assignment to the \meta{tl~var} occurs
%   just after the end of that group, using \cs{tl_set:Nn},
%   \cs{tl_gset:Nn}, \cs{tl_set:Nx}, or \cs{tl_gset:Nx}.
% \end{function}
%
% \begin{function}{\__tl_build_one:n, \__tl_build_one:o, \__tl_build_one:x}
%   \begin{syntax}
%     \cs{__tl_build_one:n} \Arg{tokens}
%   \end{syntax}
%   This function may only be used within the scope of a
%   \cs{__tl_build:Nw} function. It adds the \meta{tokens} on the
%   right of the current token list.
% \end{function}
%
% \begin{function}{\__tl_build_end:}
%   Ends the scope started by \cs{__tl_build:Nw}, and performs the
%   relevant assignment.
% \end{function}
%
% \end{documentation}
%
% \begin{implementation}
%
% \section{\pkg{l3tl-build} implementation}
%
%    \begin{macrocode}
%<*initex|package>
%    \end{macrocode}
%
%    \begin{macrocode}
%<@@=tl_build>
%    \end{macrocode}
%
%    \begin{macrocode}
\ProvidesExplPackage
  {\ExplFileName}{\ExplFileDate}{\ExplFileVersion}{\ExplFileDescription}
%    \end{macrocode}
%
% \subsection{Variables and helper functions}
%
% \begin{variable}{\l_@@_start_index_int, \l_@@_index_int}
%   Integers pointing to the starting index (currently always starts at
%   zero), and the current index. The corresponding \tn{toks} are
%   accessed directly by number.
%    \begin{macrocode}
\int_new:N \l_@@_start_index_int
\int_new:N \l_@@_index_int
%    \end{macrocode}
% \end{variable}
%
% \begin{variable}{\l_@@_result_tl}
%   The resulting token list is normally built in one go by unpacking
%   all \tn{toks} in some range. In the rare cases where there are too
%   many \cs{@@_one:n} commands, leading to the depletion of
%   registers, the contents of the current set of \tn{toks} is unpacked
%   into \cs{l_@@_result_tl}. This prevents overflow from
%   affecting the end-user (beyond an obvious performance hit).
%    \begin{macrocode}
\tl_new:N \l_@@_result_tl
%    \end{macrocode}
% \end{variable}
%
% \begin{macro}{\@@_unpack:}
% \begin{macro}[aux, EXP]{\@@_unpack_loop:w}
%   The various pieces of the token list are built in \tn{toks} from the
%   \texttt{start_index} (inclusive) to the (current) \texttt{index}
%   (excluded). Those \tn{toks} are unpacked and stored in order in the
%   \texttt{result} token list.  Optimizations would be possible here,
%   for instance, unpacking $10$ \tn{toks} at a time with a macro
%   expanding to |\the\toks#10...\the\toks#19|, but this should be kept
%   for much later.
%    \begin{macrocode}
\cs_new_protected_nopar:Npn \@@_unpack:
  {
    \tl_put_right:Nx \l_@@_result_tl
      {
        \exp_after:wN \@@_unpack_loop:w
          \int_use:N \l_@@_start_index_int ;
        \__prg_break_point:
      }
  }
\cs_new:Npn \@@_unpack_loop:w #1 ;
  {
    \if_int_compare:w #1 = \l_@@_index_int
      \exp_after:wN \__prg_break:
    \fi:
    \tex_the:D \tex_toks:D #1 \exp_stop_f:
    \exp_after:wN \@@_unpack_loop:w
      \int_use:N \__int_eval:w #1 + \c_one ;
  }
%    \end{macrocode}
% \end{macro}
% \end{macro}
%
% \subsection{Building the token list}
%
% \begin{macro}
%   {
%     \@@:Nw  , \@@_x:Nw  ,
%     \__tl_gbuild:Nw , \__tl_gbuild_x:Nw
%   }
% \begin{macro}[aux]{\@@_aux:NNw}
%   Similar to what is done for coffins: redefine some command, here
%   \cs{@@_end_aux:n} to hold the relevant assignment (see
%   \cs{@@_end:} for details). Then initialize the start index and
%   the current index at zero, and empty the \texttt{result} token list.
%    \begin{macrocode}
\cs_new_protected_nopar:Npn \@@:Nw
  { \@@_aux:NNw \tl_set:Nn }
\cs_new_protected_nopar:Npn \@@_x:Nw
  { \@@_aux:NNw \tl_set:Nx }
\cs_new_protected_nopar:Npn \__tl_gbuild:Nw
  { \@@_aux:NNw \tl_gset:Nn }
\cs_new_protected_nopar:Npn \__tl_gbuild_x:Nw
  { \@@_aux:NNw \tl_gset:Nx }
\cs_new_protected:Npn \@@_aux:NNw #1#2
  {
    \group_begin:
      \cs_set_nopar:Npn \@@_end_assignment:n
        { \group_end: #1 #2 }
      \int_zero:N \l_@@_start_index_int
      \int_zero:N \l_@@_index_int
      \tl_clear:N \l_@@_result_tl
  }
%    \end{macrocode}
% \end{macro}
% \end{macro}
%
% \begin{macro}{\@@_end:}
% \begin{macro}[aux]{\@@_end_assignment:n}
%   When we are done building a token list, unpack all \tn{toks} into
%   the \texttt{result} token list, and expand this list before closing
%   the group. The \cs{@@_end_assignment:n} function is defined by
%   \cs{@@_aux:NNw} to end the group and hold the relevant
%   assignment. Its value outside is irrelevant, but just in case, we
%   set it to a function which would clean up the contents of
%   \cs{l_@@_result_tl}.
%    \begin{macrocode}
\cs_new_protected_nopar:Npn \@@_end:
  {
      \@@_unpack:
      \exp_args:No
    \@@_end_assignment:n \l_@@_result_tl
  }
\cs_new_eq:NN \@@_end_assignment:n \use_none:n
%    \end{macrocode}
% \end{macro}
% \end{macro}
%
% \begin{macro}{\@@_one:n, \@@_one:o, \@@_one:x}
%   Store the tokens in a free \tn{toks}, then move the pointer to the
%   next one. If we overflow, unpack the current \tn{toks}, and reset
%   the current index, preparing to fill more \tn{toks}.  This could be
%   optimized by avoiding to read |#1|, using \tn{afterassignment}.
%    \begin{macrocode}
\cs_new_protected:Npn \@@_one:n #1
  {
    \tex_toks:D \l_@@_index_int {#1}
    \tex_advance:D \l_@@_index_int \c_one
    \if_int_compare:w \l_@@_index_int > \c_max_register_int
      \@@_unpack:
      \l_@@_index_int \l_@@_start_index_int
    \fi:
  }
\cs_new_protected:Npn \@@_one:o #1
  {
    \tex_toks:D \l_@@_index_int \exp_after:wN {#1}
    \tex_advance:D \l_@@_index_int \c_one
    \if_int_compare:w \l_@@_index_int > \c_max_register_int
      \@@_unpack:
      \l_@@_index_int \l_@@_start_index_int
    \fi:
  }
\cs_new_protected:Npn \@@_one:x #1
  { \use:x { \@@_one:n {#1} } }
%    \end{macrocode}
% \end{macro}
%
%    \begin{macrocode}
%</initex|package>
%    \end{macrocode}
%
% \end{implementation}
%
% \PrintIndex
