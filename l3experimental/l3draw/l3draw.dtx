% \iffalse meta-comment
%
%% File: l3draw.dtx Copyright(C) 2018 The LaTeX3 Project
%
% It may be distributed and/or modified under the conditions of the
% LaTeX Project Public License (LPPL), either version 1.3c of this
% license or (at your option) any later version.  The latest version
% of this license is in the file
%
%    http://www.latex-project.org/lppl.txt
%
% This file is part of the "l3experimental bundle" (The Work in LPPL)
% and all files in that bundle must be distributed together.
%
% -----------------------------------------------------------------------
%
% The development version of the bundle can be found at
%
%    https://github.com/latex3/latex3
%
% for those people who are interested.
%
%<*driver|package>
\RequirePackage{expl3}
%</driver|package>
%<*driver>
\documentclass[full]{l3doc}
\begin{document}
  \DocInput{\jobname.dtx}
\end{document}
%</driver>
% \fi
%
% \title{^^A
%   The \pkg{l3draw} package\\ Core drawing support^^A
% }
%
% \author{^^A
%  The \LaTeX3 Project\thanks
%    {^^A
%      E-mail:
%        \href{mailto:latex-team@latex-project.org}
%          {latex-team@latex-project.org}^^A
%    }^^A
% }
%
% \date{Released 2017/12/16}
%
% \maketitle
%
% \begin{documentation}
%
% \section{\pkg{l3draw} documentation}
%
% The \pkg{l3draw} package provides a set of tools for creating (vector)
% drawings in \pkg{expl3}. It is heavily inspired by the \pkg{pgf} layer of
% the Ti\textit{k}Z system, with many of the interfaces having the same form.
% However, the code provided here is build entirely on core \pkg{expl3} ideas
% and uses the \LaTeX3 FPU for numerical support.
%
% Numerical expressions in \pkg{l3draw} are handled as floating point
% expressions, unless otherwise noted. This means that they may contain or
% omit explicit units. Where units are omitted, they will automatically be
% taken as given in (\TeX{}) points.
%
% The code here is \emph{highly} experimental.
%
% \subsection{Drawings}
%
% \begin{function}{\draw_begin:, \draw_end:}
%   \begin{syntax}
%     \cs{draw_begin:}
%     ...
%     \cs{draw_end:}
%   \end{syntax}
%   Each drawing should be created within a \cs{draw_begin:}/\cs{draw_end:}
%   function pair. The \texttt{begin} function sets up a number of key
%   data structures for the rest of the functions here: unless otherwise
%   specified, use of |\draw_...| functions outside of this
%   \enquote{environment} is \emph{not supported}.
%
%   The drawing created within the environment will be inserted into
%   the typesetting stream by the \cs{draw_end:} function, which will
%   switch out of vertical mode if required.
% \end{function}
%
% \subsection{Graphics state}
%
% Within the drawing environment, a number of functions control how drawings
% will appear. Note that these all apply \emph{globally}, though some are
% rest at the start of each drawing (\cs{draw_begin:}).
%
% \begin{function}{\g_draw_linewidth_default_dim}
%   The default value of the linewidth for stokes, set at the start
%   of every drawing (\cs{draw_begin:}).
% \end{function}
%
% \begin{function}{\draw_linewidth:n, \draw_inner_linewidth:n}
%   \begin{syntax}
%     \cs{draw_linewidth:n} \Arg{width}
%   \end{syntax}
%   Sets the width to be used for stroking to the \meta{width} (an 
%   \meta{fp expr}).
% \end{function}
%
% \begin{function}{\draw_nonzero_rule:, \draw_evenodd_rule:}
%   \begin{syntax}
%     \cs{draw_nonzero_rule:}
%   \end{syntax}
%   Active either the non-zero winding number or the even-odd rule,
%   respectively, for determining what is inside a fill or clip area.
%   For technical reasons, these command are not influenced by scoping
%   and apply on an ongoing basis.
% \end{function}
%
% \begin{function}
%   {
%     \draw_cap_butt:      ,
%     \draw_cap_rectangle: ,
%     \draw_cap_round:
%   }
%   \begin{syntax}
%     \cs{draw_cap_butt:}
%   \end{syntax}
%   Sets the style of terminal stroke position to one of butt, rectangle or
%   round.
% \end{function}
%
% \begin{function}
%   {
%     \draw_join_bevel: ,
%     \draw_join_miter: ,
%     \draw_join_round:
%   }
%   \begin{syntax}
%     \cs{draw_cap_butt:}
%   \end{syntax}
%   Sets the style of stroke joins to one of bevel, miter or round.
% \end{function}
%
% \begin{function}{\draw_miterlimit:n}
%   \begin{syntax}
%     \cs{draw_miterlimit:n} \Arg{limit}
%   \end{syntax}
%   Sets the miter \meta{limit} of lines joined as a miter, as described in the
%   PDF and PostScript manuals. The \meta{limit} is an \meta{fp expr}.
% \end{function}
%
% \subsection{Points}
%
% Functions supporting the calculation of points (co-ordinates) are expandable
% and may be used outside of the drawing environment. When used in this
% way, they all yield a co-ordinate tuple, for example
% \begin{verbatim}
%   \tl_set:Nx \l_tmpa_tl { \draw_point:nn { 1 } { 2 } }
%   \tl_show:N \l_tmpa_tl
% \end{verbatim}
% gives
% \begin{verbatim}
%   > \l_tmpa_tl=1pt,2pt.
%   <recently read> }
% \end{verbatim}
%
% This output form is then suitable as \emph{input} for subsequent point
% calculations, \emph{i.e.}~where a \meta{point} is required it may be
% given as a tuple. This \emph{may} include units and surrounding
% parentheses, for example
% \begin{verbatim}
%   1,2
%   (1,2)
%   1cm,3pt
%   (1pt,2cm)
%   2 * sind(30), 2^4in
% \end{verbatim}
% are all valid input forms. Notice that each part of the tuple may itself
% be a float point expression.
%
% Notice that in contrast to \pkg{pgf} it is possible to give the positions
% of points \emph{directly}.
%
% \subsubsection{Basic point functions}
%
% \begin{function}[EXP]{\draw_point:nn}
%   \begin{syntax}
%     \cs{draw_point:nn} \Arg{x} \Arg{y}
%   \end{syntax}
%   Gives the co-ordinates of the point at \meta{x} and \meta{y}, both of
%   which are \meta{fp expr}.
% \end{function}
%
% \begin{function}[EXP]{\draw_point_polar:nn, \draw_point_polar:nnn}
%   \begin{syntax}
%     \cs{draw_point_polar:nn} \Arg{angle} \Arg{radius}
%     \cs{draw_point_polar:nnn} \Arg{angle} \Arg{radius-a} \Arg{radius-b}
%   \end{syntax}
%   Gives the co-ordinates of the point at \meta{angle} (an \meta{fp expr} in
%   \emph{degrees}) and \meta{radius}. The three-argument version accepts
%   two radii of different lengths.
%   
%   Note the interface here is somewhat different from that in \pkg{pgf}:
%   the one- and two-radii versions in \pkg{l3draw} use separate functions,
%   whilst in \pkg{pgf} they use the same function and a keyword.
% \end{function}
%
% \begin{function}[EXP]{\draw_point_add:nn}
%   \begin{syntax}
%     \cs{draw_point_add:nn} \Arg{point1} \Arg{point2}
%   \end{syntax}
%   Adds \meta{point1} to \meta{point2}.
% \end{function}
%
% \begin{function}[EXP]{\draw_point_diff:nn}
%   \begin{syntax}
%     \cs{draw_point_diff:nn} \Arg{point1} \Arg{point2}
%   \end{syntax}
%   Subtracts \meta{point1} from \meta{point2}.
% \end{function}
%
% \begin{function}[EXP]{\draw_point_scale:nn}
%   \begin{syntax}
%     \cs{draw_point_scale:nn} \Arg{scale} \Arg{point}
%   \end{syntax}
%   Scales the \meta{point} by the \meta{scale} (an \meta{fp expr}).
% \end{function}
%
% \begin{function}[EXP]{\draw_point_unit_vector:n}
%   \begin{syntax}
%     \cs{draw_point_unit_vector:n} \Arg{point}
%   \end{syntax}
%   Expands to the co-ordinates of a unit vector joining the \meta{point}
%   with the origin.
% \end{function}
%
% \begin{function}[EXP]{\draw_point_transform:n}
%   \begin{syntax}
%     \cs{draw_point_transform:n} \Arg{point}
%   \end{syntax}
%   Evaluates the position of the \meta{point} subject to the current
%   transformation matrix. This operation is applied automatically by
%   most higher-level functions (\emph{e.g.}~path manipulations).
% \end{function}
%
% \subsubsection{Points on a vector basis}
%
% As well as giving explicit values, it is possible to describe points
% in terms of underlying direction vectors. The latter are initially
% co-incident with the standard Cartesian axes, but may be altered by
% the user.
%
% \begin{function}{\draw_xvec_set:n, \draw_yvec_set:n, \draw_zvec_set:n}
%   \begin{syntax}
%     \cs{draw_xvec_set:n} \Arg{point}
%   \end{syntax}
%   Defines the appropriate base vector to point toward the \meta{point}
%   on the canvas. The standard settings for the $x$- and $y$-vectors are
%   $1\,\mathrm{cm}$ along the relevant canvas axis, whilst for the
%   $z$-vector an appropriate direction is taken.
% \end{function}
%
% \begin{function}[EXP]{\draw_point_vec:nn, \draw_point_vec:nnn}
%   \begin{syntax}
%     \cs{draw_point_vec:nn} \Arg{xscale} \Arg{yscale}
%     \cs{draw_point_vec:nnn} \Arg{xscale} \Arg{yscale} \Arg{zscale}
%   \end{syntax}
%   Expands to the co-ordinate of the point at \meta{xscale} times the
%   $x$-vector and \meta{yscale} times the $y$-vector. The three-argument
%   version extends this to include the $z$-vector.
% \end{function}
%
% \begin{function}[EXP]{\draw_point_vec_polar:nn, \draw_point_vec_polar:nnn}
%   \begin{syntax}
%     \cs{draw_point_vec_polar:nn} \Arg{angle} \Arg{radius}
%     \cs{draw_point_vec_polar:nnn} \Arg{angle} \Arg{radius-a} \Arg{radius-b}
%   \end{syntax}
%   Gives the co-ordinates of the point at \meta{angle} (an \meta{fp expr} in
%   \emph{degrees}) and \meta{radius}, relative to the prevailing
%   $x$- and $y$-vectors. The three-argument version accepts two radii of
%   different lengths.
%   
%   Note the interface here is somewhat different from that in \pkg{pgf}:
%   the one- and two-radii versions in \pkg{l3draw} use separate functions,
%   whilst in \pkg{pgf} they use the same function and a keyword.
% \end{function}
%
% \subsubsection{Intersections}
%
% \begin{function}[EXP]{\draw_point_intersect_lines:nnnn}
%   \begin{syntax}
%     \cs{draw_point_intersect_lines:nnnn} \Arg{point1} \Arg{point2} \Arg{point3} \Arg{point4}
%   \end{syntax}
%   Evaluates the point at the intersection of one line, joining
%   \meta{point1} and \meta{point2}, and a second line joining \meta{point3}
%   and \meta{point4}. If the lines do not intersect, or are coincident, and
%   error will occur.
% \end{function}
%
% \begin{function}[EXP]{\draw_point_intersect_circles:nnnn}
%   \begin{syntax}
%     \cs{draw_point_intersect_circles:nnnnn}
%       \Arg{center1} \Arg{radius1} \Arg{center2} \Arg{radius2} \Arg{root}
%   \end{syntax}
%   Evaluates the point at the intersection of one circle with
%   \meta{center1} and \meta{radius1}, and a second circle with \meta{center2}
%   and \meta{radius2}. If the circles do not intersect, or are coincident, and
%   error will occur.
%
%   Note the interface here has a different argument ordering from that in
%   \pkg{pgf}, which has the two centers then the two radii.
% \end{function}
%
% \subsubsection{Interpolations}
%
% \begin{function}[EXP]{\draw_point_interpolate_line:nnn}
%   \begin{syntax}
%     \cs{draw_point_interpolate_line:nnn} \Arg{part} \Arg{point1} \Arg{point2}
%   \end{syntax}
%   Expands to the point which is \meta{part} way along the line joining
%   \meta{point1} and \meta{point2}. The \meta{part} may be an interpolation or
%   an extrapolation, and is a floating point value expressing a percentage
%   along the line, \emph{e.g.}~a value of \texttt{0.5} would be half-way
%   between the two points.
% \end{function}
%
% \begin{function}[EXP]{\draw_point_interpolate_distance:nnn}
%   \begin{syntax}
%     \cs{draw_point_interpolate_distance:nnn} \Arg{distance} \Arg{point expr1} \Arg{point expr2}
%   \end{syntax}
%   Expands to the point which is \meta{distance} way along the line joining
%   \meta{point1} and \meta{point2}. The \meta{distance} may be an interpolation
%   or an extrapolation.
% \end{function}
%
% \begin{function}[EXP]{\draw_point_interpolate_curve:nnnnnn}
%   \begin{syntax}
%     \cs{draw_point_interpolate_curve:nnnnnn} \Arg{part}
%       \Arg{start} \Arg{control1} \Arg{control2} \Arg{end}
%   \end{syntax}
%   Expands to the point which is \meta{part} way along the curve between
%   \meta{start} and \meta{end} and defined by \meta{control1} and
%   \meta{control2}. The \meta{part} may be an interpolation or
%   an extrapolation, and is a floating point value expressing a percentage
%   along the curve, \emph{e.g.}~a value of \texttt{0.5} would be half-way
%   along the curve.
% \end{function}
%
% \subsection{Paths}
%
% Paths are constructed by combining one or more operations before applying
% one or more actions. Thus until a path is \enquote{used}, it may be
% manipulated or indeed discarded entirely. Only one path is active at
% any one time, and the path is \emph{not} affected by \TeX{} grouping.
%
% \begin{function}{\draw_path_corner_arc:n}
%   \begin{syntax}
%     \cs{draw_path_corner_arc:n} \Arg{length}
%   \end{syntax}
%   Sets the degree of rounding applied to corners in a path: if the
%   \meta{length} is \texttt{0pt} then no rounding applies. The value of the
%   \meta{length} is local to the current \TeX{} group. \emph{At present,
%   corner arcs are not activated in the code.}
% \end{function}
%
% \begin{function}{\draw_path_moveto:n}
%   \begin{syntax}
%     \cs{draw_path_moveto:n} \Arg{point}
%   \end{syntax}
%   Moves the reference point of the path to the \meta{point}, but will
%   not join this to any previous point.
% \end{function}
%
% \begin{function}{\draw_path_lineto:n}
%   \begin{syntax}
%     \cs{draw_path_lineto:n} \Arg{point}
%   \end{syntax}
%   Joins the current path to the \meta{point} with a straight line.
% \end{function}
%
% \begin{function}{\draw_path_curveto:nnn}
%   \begin{syntax}
%     \cs{draw_path_curveto:nnn} \Arg{control1} \Arg{control2} \Arg{end}
%   \end{syntax}
%   Joins the current path to the \meta{end} with a curved line defined by
%   cubic Bézier points \meta{control1} and \meta{control2}.
% \end{function}
%
% \begin{function}{\draw_path_curveto:nn}
%   \begin{syntax}
%     \cs{draw_path_curveto:nn} \Arg{control} \Arg{end}
%   \end{syntax}
%   Joins the current path to the \meta{end} with a curved line defined by
%   quadratic Bézier point \meta{control}.
% \end{function}
%
% \begin{function}{\draw_path_arc:nnn, \draw_path_arc:nnnn}
%   \begin{syntax}
%     \cs{draw_path_arc:nnn} \Arg{angle1} \Arg{angle2} \Arg{radius}
%     \cs{draw_path_arc:nnnn} \Arg{angle1} \Arg{angle2} \Arg{radius-a} \Arg{radius-b}
%   \end{syntax}
%   Joins the current path with an arc between \meta{angle1} and \meta{angle2}
%   and of \meta{radius}. The four-argument version accepts two radii of
%   different lengths.
%
%   Note the interface here has a different argument ordering from that in
%   \pkg{pgf}, which has the two centers then the two radii.
% \end{function}
%
% \begin{function}{\draw_path_arc_axes:nnnn}
%   \begin{syntax}
%     \cs{draw_path_arc_axes:nnn} \Arg{angle1} \Arg{angle2} \Arg{vector1} \Arg{vector2}
%   \end{syntax}
%   Appends the portion of an ellipse from \meta{angle1} to \meta{angle2} of an
%   ellipse with axes along \meta{vector1} and \meta{vector2} to the current path. 
% \end{function}
%
% \begin{function}{\draw_path_ellipse:nnnn}
%   \begin{syntax}
%     \cs{draw_path_ellipse:nnn} \Arg{center} \Arg{vector1} \Arg{vector2}
%   \end{syntax}
%   Appends an ellipse at \meta{center} with axes along \meta{vector1} and
%   \meta{vector2} to the current path. 
% \end{function}
%
% \begin{function}{\draw_path_circle:nn}
%   \begin{syntax}
%     \cs{draw_path_circle:nn} \Arg{center} \Arg{radius}
%   \end{syntax}
%   Appends a circle of \meta{radius} at \meta{center} to the current path. 
% \end{function}
%
% \begin{function}{\draw_path_rectangle:nn, \draw_path_rectangle_corners:nn}
%   \begin{syntax}
%     \cs{draw_path_rectangle:nn} \Arg{lower-left} \Arg{displacement}
%     \cs{draw_path_rectangle_corners:nn} \Arg{lower-left} \Arg{top-right}
%   \end{syntax}
%   Appends a rectangle starting at \meta{lower-left} to the current path,
%   with the size of the rectangle determined either by a \meta{displacement}
%   or the position of the \meta{top-right}.
% \end{function}
%
% \begin{function}{\draw_path_grid:nnnn}
%   \begin{syntax}
%     \cs{draw_path_grid:nnnn} \Arg{xspace} \Arg{yspace} \Arg{lower-left} \Arg{upper-right}
%   \end{syntax}
%   Constructs a grid of \meta{xspace} and \meta{yspace} from the
%   \meta{lower-left} to the \meta{upper-right}, and appends this to the
%   current path.
% \end{function}
%
% \begin{function}{\draw_path_close:}
%   \begin{syntax}
%     \cs{draw_path_close:}
%   \end{syntax}
%   Closes the current part of the path by appending a straight line from
%   the current point to the starting point of the path.
% \end{function}
%
% \begin{function}{\draw_path_use:n, \draw_path_use_clear:n}
%   \begin{syntax}
%     \cs{draw_path_use:n} \Arg{action(s)}
%   \end{syntax}
%   Inserts the current path, carrying out one ore more possible \meta{actions}
%   (a comma list):
%   \begin{itemize}
%     \item \texttt{clear} Resets the path to empty
%     \item \texttt{clip} Clips any content outside of the path
%     \item \texttt{draw}
%     \item \texttt{fill} Fills the interior of the path with the current
%       file color
%     \item \texttt{stroke} Draws a line along the current path
%   \end{itemize}
% \end{function}
%
% \subsection{Transformations}
%
% Points are normally used unchanged relative to the canvas axes. This can
% be modified by applying a transformation matrix.
%
% \begin{function}{\draw_transform_reset:}
%   \begin{syntax}
%     \cs{draw_transform_reset:}
%   \end{syntax}
%   Resets the matrix to the identity.
% \end{function}
%
% \begin{function}{\draw_transform_concat:nnnnn}
%   \begin{syntax}
%     \cs{draw_transform_concat:nnnnn}
%       \Arg{a} \Arg{b} \Arg{c} \Arg{d} \Arg{vector}
%   \end{syntax}
%   Appends the given transformation to the currently-active one. The
%   transformation is made up of a matrix \meta{a}, \meta{b}, \meta{c} and
%   \meta{d}, and a shift by the \meta{vector}.
% \end{function}
%
% \begin{function}{\draw_transform:nnnnn}
%   \begin{syntax}
%     \cs{draw_transform:nnnnn}
%       \Arg{a} \Arg{b} \Arg{c} \Arg{d} \Arg{vector}
%   \end{syntax}
%   Applies the transformation matrix specified, over-writing any existing
%   matrix. The transformation is made up of a matrix \meta{a}, \meta{b},
%   \meta{c} and \meta{d}, and a shift by the \meta{vector}.
% \end{function}
%
% \begin{function}{\draw_transform_triangle:nnn}
%   \begin{syntax}
%     \cs{draw_transform_triangle:nnn}
%       \Arg{origin} \Arg{point1} \Arg{point2}
%   \end{syntax}
%   Applies a transformation such that the co-ordinates $(0, 0)$, $(1, 0)$
%   and $(0, 1)$ are given by the \meta{origin}, \meta{point1} and
%   \meta{point2}, respectively.
% \end{function}
%
% \begin{function}{\draw_transform_invert:}
%   \begin{syntax}
%     \cs{draw_transform_invert:}
%   \end{syntax}
%   Inverts the current transformation matrix and reverses the current
%   shift vector.
% \end{function}
%
% \end{documentation}
%
% \begin{implementation}
%
% \section{\pkg{l3draw} implementation}
%
%    \begin{macrocode}
%<*initex|package>
%    \end{macrocode}
%
%    \begin{macrocode}
%<@@=draw>
%    \end{macrocode}
%
%    \begin{macrocode}
%<*package>
\ProvidesExplPackage{l3draw}{2018/02/06}{}
  {L3 Experimental core drawing support}
%</package>
%    \end{macrocode}
%
% Everything else is in the sub-files!
%
%    \begin{macrocode}
%</initex|package>
%    \end{macrocode}
%
% \end{implementation}
%
% \PrintIndex
