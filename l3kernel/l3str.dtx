% \iffalse meta-comment
%
%% File: l3str.dtx Copyright (C) 2011-2016 The LaTeX3 Project
%%
%% It may be distributed and/or modified under the conditions of the
%% LaTeX Project Public License (LPPL), either version 1.3c of this
%% license or (at your option) any later version.  The latest version
%% of this license is in the file
%%
%%    http://www.latex-project.org/lppl.txt
%%
%% This file is part of the "l3kernel bundle" (The Work in LPPL)
%% and all files in that bundle must be distributed together.
%%
%% The released version of this bundle is available from CTAN.
%%
%% -----------------------------------------------------------------------
%%
%% The development version of the bundle can be found at
%%
%%    http://www.latex-project.org/svnroot/experimental/trunk/
%%
%% for those people who are interested.
%%
%%%%%%%%%%%
%% NOTE: %%
%%%%%%%%%%%
%%
%%   Snapshots taken from the repository represent work in progress and may
%%   not work or may contain conflicting material!  We therefore ask
%%   people _not_ to put them into distributions, archives, etc. without
%%   prior consultation with the LaTeX3 Project.
%%
%% -----------------------------------------------------------------------
%
%<*driver>
\documentclass[full]{l3doc}
%</driver>
%<*driver|package>
\GetIdInfo$Id$
  {L3 Strings}
%</driver|package>
%<*driver>
\begin{document}
  \DocInput{\jobname.dtx}
\end{document}
%</driver>
% \fi
%
% \title{^^A
%   The \pkg{l3str} package\\Strings^^A
%   \thanks{This file describes v\ExplFileVersion,
%      last revised \ExplFileDate.}^^A
% }
%
% \author{^^A
%  The \LaTeX3 Project\thanks
%    {^^A
%      E-mail:
%        \href{mailto:latex-team@latex-project.org}
%          {latex-team@latex-project.org}^^A
%    }^^A
% }
%
% \date{Released \ExplFileDate}
%
% \maketitle
%
% \begin{documentation}
%
% \TeX{} associates each character with a category code: as such, there is no
% concept of a \enquote{string} as commonly understood in many other
% programming languages. However, there are places where we wish to manipulate
% token lists while in some sense \enquote{ignoring} category codes: this is
% done by treating token lists as strings in a \TeX{} sense.
%
% A \TeX{} string (and thus an \pkg{expl3} string) is a series of characters
% which have category code $12$ (\enquote{other}) with the exception of
% space characters which have category code $10$ (\enquote{space}). Thus
% at a technical level, a \TeX{} string is a token list with the appropriate
% category codes. In this documentation, these will simply be referred to as
% strings.
%
% String variables are simply specialised token lists, but by convention
% should be named with the suffix \texttt{\ldots{}str}.  Such variables
% should contain characters with category code $12$ (other), except
% spaces, which have category code $10$ (blank space).  All the
% functions in this module which accept a token list argument first
% convert it to a string using \cs{tl_to_str:n} for internal processing,
% and will not treat a token list or the corresponding string
% representation differently.
%
% Note that as string variables are a special case of token list variables
% the coverage of \cs{str_\ldots{}:N} functions is somewhat smaller than
% \cs{tl_\ldots{}:N}.
%
% The functions \cs{cs_to_str:N}, \cs{tl_to_str:n}, \cs{tl_to_str:N} and
% \cs{token_to_str:N} (and variants) will generate strings from the appropriate
% input: these are documented in \pkg{l3basics}, \pkg{l3tl} and \pkg{l3token},
% respectively.
%
% Most expandable functions in this module come in three flavours:
% \begin{itemize}
%   \item \cs{str_...:N}, which expect a token list or string
%     variable as their argument;
%   \item \cs{str_...:n}, taking any token list (or string) as an
%     argument;
%   \item \cs{str_..._ignore_spaces:n}, which ignores any space
%     encountered during the operation: these functions are typically
%     faster than those which take care of escaping spaces
%     appropriately.
% \end{itemize}
%
% \section{Building strings}
%
% \begin{function}[added = 2015-09-18]{\str_new:N, \str_new:c}
%   \begin{syntax}
%     \cs{str_new:N} \meta{str~var}
%   \end{syntax}
%   Creates a new \meta{str~var} or raises an error if the name is
%   already taken.  The declaration is global.  The \meta{str~var} will
%   initially be empty.
% \end{function}
%
% \begin{function}[added = 2015-09-18]
%   {\str_const:Nn, \str_const:Nx, \str_const:cn, \str_const:cx}
%   \begin{syntax}
%     \cs{str_const:Nn} \meta{str~var} \Arg{token list}
%   \end{syntax}
%   Creates a new constant \meta{str~var} or raises an error if the name
%   is already taken.  The value of the \meta{str~var} will be set
%   globally to the \meta{token list}, converted to a string.
% \end{function}
%
% \begin{function}[added = 2015-09-18]
%   {\str_clear:N, \str_clear:c, \str_gclear:N, \str_gclear:c}
%   \begin{syntax}
%     \cs{str_clear:N} \meta{str~var}
%   \end{syntax}
%   Clears the content of the \meta{str~var}.
% \end{function}
%
% \begin{function}[added = 2015-09-18]{\str_clear_new:N, \str_clear_new:c}
%   \begin{syntax}
%     \cs{str_clear_new:N} \meta{str~var}
%   \end{syntax}
%   Ensures that the \meta{str~var} exists globally by applying
%   \cs{str_new:N} if necessary, then applies \cs{str_(g)clear:N} to leave
%   the \meta{str~var} empty.
% \end{function}
%
% \begin{function}[added = 2015-09-18]
%   {
%     \str_set_eq:NN,  \str_set_eq:cN,  \str_set_eq:Nc,  \str_set_eq:cc,
%     \str_gset_eq:NN, \str_gset_eq:cN, \str_gset_eq:Nc, \str_gset_eq:cc
%   }
%   \begin{syntax}
%     \cs{str_set_eq:NN} \meta{str~var_1} \meta{str~var_2}
%   \end{syntax}
%   Sets the content of \meta{str~var_1} equal to that of
%   \meta{str~var_2}.
% \end{function}
%
% \section{Adding data to string variables}
%
% \begin{function}[added = 2015-09-18]
%   {
%     \str_set:Nn,  \str_set:Nx,  \str_set:cn,  \str_set:cx,
%     \str_gset:Nn, \str_gset:Nx, \str_gset:cn, \str_gset:cx
%   }
%   \begin{syntax}
%     \cs{str_set:Nn} \meta{str var} \Arg{token list}
%   \end{syntax}
%   Converts the \meta{token list} to a \meta{string}, and stores the
%   result in \meta{str var}.
% \end{function}
%
% \begin{function}[added = 2015-09-18]
%   {
%     \str_put_left:Nn,  \str_put_left:Nx,
%     \str_put_left:cn,  \str_put_left:cx,
%     \str_gput_left:Nn, \str_gput_left:Nx,
%     \str_gput_left:cn, \str_gput_left:cx
%   }
%   \begin{syntax}
%     \cs{str_put_left:Nn} \meta{str var} \Arg{token list}
%   \end{syntax}
%   Converts the \meta{token list} to a \meta{string}, and prepends the
%   result to \meta{str var}.  The current contents of the \meta{str
%     var} are not automatically converted to a string.
% \end{function}
%
% \begin{function}[added = 2015-09-18]
%   {
%     \str_put_right:Nn,  \str_put_right:Nx,
%     \str_put_right:cn,  \str_put_right:cx,
%     \str_gput_right:Nn, \str_gput_right:Nx,
%     \str_gput_right:cn, \str_gput_right:cx
%   }
%   \begin{syntax}
%     \cs{str_put_right:Nn} \meta{str var} \Arg{token list}
%   \end{syntax}
%   Converts the \meta{token list} to a \meta{string}, and appends the
%   result to \meta{str var}.  The current contents of the \meta{str
%     var} are not automatically converted to a string.
% \end{function}
%
% \subsection{String conditionals}
%
% \begin{function}[EXP, pTF, added = 2015-09-18]
%   {\str_if_exist:N, \str_if_exist:c}
%   \begin{syntax}
%     \cs{str_if_exist_p:N} \meta{str~var}
%     \cs{str_if_exist:NTF} \meta{str~var} \Arg{true code} \Arg{false code}
%   \end{syntax}
%   Tests whether the \meta{str~var} is currently defined.  This does not
%   check that the \meta{str~var} really is a string.
% \end{function}
%
% \begin{function}[EXP,pTF, added = 2015-09-18]
%   {\str_if_empty:N, \str_if_empty:c}
%   \begin{syntax}
%     \cs{sr_if_empty_p:N} \meta{str~var}
%     \cs{str_if_empty:NTF} \meta{str~var} \Arg{true code} \Arg{false code}
%   \end{syntax}
%   Tests if the \meta{string variable} is entirely empty
%   (\emph{i.e.}~contains no characters at all).
% \end{function}
%
% \begin{function}[EXP,pTF, added = 2015-09-18]
%   {\str_if_eq:NN, \str_if_eq:Nc, \str_if_eq:cN, \str_if_eq:cc}
%   \begin{syntax}
%     \cs{str_if_eq_p:NN} \meta{str~var_1} \meta{str~var_2}
%     \cs{str_if_eq:NNTF} \meta{str~var_1} \meta{str~var_2} \Arg{true code} \Arg{false code}
%   \end{syntax}
%   Compares the content of two \meta{str variables} and
%   is logically \texttt{true} if the two contain the same characters.
% \end{function}
%
% \begin{function}[EXP,pTF]
%   {
%     \str_if_eq:nn, \str_if_eq:Vn, \str_if_eq:on, \str_if_eq:no,
%     \str_if_eq:nV, \str_if_eq:VV
%   }
%   \begin{syntax}
%     \cs{str_if_eq_p:nn} \Arg{tl_1} \Arg{tl_2}
%     \cs{str_if_eq:nnTF} \Arg{tl_1} \Arg{tl_2} \Arg{true code} \Arg{false code}
%   \end{syntax}
%   Compares the two \meta{token lists} on a character by character
%   basis, and is \texttt{true} if the two lists contain the same
%   characters in the same order. Thus for example
%   \begin{verbatim}
%     \str_if_eq_p:no { abc } { \tl_to_str:n { abc } }
%   \end{verbatim}
%   is logically \texttt{true}.
% \end{function}
%
% \begin{function}[EXP,pTF, added = 2012-06-05]{\str_if_eq_x:nn}
%   \begin{syntax}
%     \cs{str_if_eq_x_p:nn} \Arg{tl_1} \Arg{tl_2}
%     \cs{str_if_eq_x:nnTF} \Arg{tl_1} \Arg{tl_2} \Arg{true code} \Arg{false code}
%   \end{syntax}
%   Compares the full expansion of two \meta{token lists} on a character by
%   character basis, and is \texttt{true} if the two lists contain the same
%   characters in the same order. Thus for example
%   \begin{verbatim}
%     \str_if_eq_x_p:nn { abc } { \tl_to_str:n { abc } }
%   \end{verbatim}
%   is logically \texttt{true}.
% \end{function}
%
% \begin{function}[added = 2013-07-24, updated = 2015-02-28, EXP, TF]
%   {\str_case:nn, \str_case:on, \str_case:nV, \str_case:nv}
%   \begin{syntax}
%     \cs{str_case:nnTF} \Arg{test string} \\
%     ~~|{| \\
%     ~~~~\Arg{string case_1} \Arg{code case_1} \\
%     ~~~~\Arg{string case_2} \Arg{code case_2} \\
%     ~~~~\ldots \\
%     ~~~~\Arg{string case_n} \Arg{code case_n} \\
%     ~~|}| \\
%     ~~\Arg{true code}
%     ~~\Arg{false code}
%   \end{syntax}
%   This function compares the \meta{test string} in turn with each
%   of the \meta{string cases}. If the two are equal (as described for
%   \cs{str_if_eq:nnTF} then the
%   associated \meta{code} is left in the input stream. If any of the
%   cases are matched, the \meta{true code} is also inserted into the
%   input stream (after the code for the appropriate case), while if none
%   match then the \meta{false code} is inserted. The function
%   \cs{str_case:nn}, which does nothing if there is no match, is also
%   available.
% \end{function}
%
% \begin{function}[added = 2013-07-24, EXP, TF]{\str_case_x:nn}
%   \begin{syntax}
%     \cs{str_case_x:nnF} \Arg{test string} \\
%     ~~|{| \\
%     ~~~~\Arg{string case_1} \Arg{code case_1} \\
%     ~~~~\Arg{string case_2} \Arg{code case_2} \\
%     ~~~~\ldots \\
%     ~~~~\Arg{string case_n} \Arg{code case_n} \\
%     ~~|}| \\
%     ~~\Arg{true code}
%     ~~\Arg{false code}
%   \end{syntax}
%   This function compares the full expansion of the \meta{test string}
%   in turn with the full expansion of the \meta{string cases}. If the two
%   full expansions are equal (as described for \cs{str_if_eq:nnTF} then the
%   associated \meta{code} is left in the input stream.  If any of the
%   cases are matched, the \meta{true code} is also inserted into the
%   input stream (after the code for the appropriate case), while if none
%   match then the \meta{false code} is inserted. The function
%   \cs{str_case_x:nn}, which does nothing if there is no match, is also
%   available.
%   The \meta{test string} is expanded in each comparison, and must
%   always yield the same result: for example, random numbers must
%   not be used within this string.
% \end{function}
%
% \section{Working with the content of strings}
%
% \begin{function}[EXP, added = 2015-09-18]{\str_use:N, \str_use:c}
%   \begin{syntax}
%     \cs{str_use:N} \meta{str~var}
%   \end{syntax}
%   Recovers the content of a \meta{str~var} and places it
%   directly in the input stream. An error will be raised if the variable
%   does not exist or if it is invalid. Note that it is possible to use
%   a \meta{str} directly without an accessor function.
% \end{function}
%
% \begin{function}[EXP, added = 2015-09-18]
%   {\str_count:N, \str_count:c, \str_count:n, \str_count_ignore_spaces:n}
%   \begin{syntax}
%     \cs{str_count:n} \Arg{token list}
%   \end{syntax}
%   Leaves in the input stream the number of characters in the string
%   representation of \meta{token list}, as an integer denotation.  The
%   functions differ in their treatment of spaces.  In the case of
%   \cs{str_count:N} and \cs{str_count:n}, all characters including
%   spaces are counted.  The \cs{str_count_ignore_spaces:n} function
%   leaves the number of non-space characters in the input stream.
% \end{function}
%
% \begin{function}[EXP, added = 2015-09-18]
%   {\str_count_spaces:N, \str_count_spaces:c, \str_count_spaces:n}
%   \begin{syntax}
%     \cs{str_count_spaces:n} \Arg{token list}
%   \end{syntax}
%   Leaves in the input stream the number of space characters in the
%   string representation of \meta{token list}, as an integer
%   denotation. Of course, this function has no \texttt{_ignore_spaces}
%   variant.
% \end{function}
%
% \begin{function}[EXP, added = 2015-09-18]
%   {\str_head:N, \str_head:c, \str_head:n, \str_head_ignore_spaces:n}
%   \begin{syntax}
%     \cs{str_head:n} \Arg{token list}
%   \end{syntax}
%   Converts the \meta{token list} into a \meta{string}.  The first
%   character in the \meta{string} is then left in the input stream,
%   with category code \enquote{other}.  The functions differ if the
%   first character is a space: \cs{str_head:N} and \cs{str_head:n}
%   return a space token with category code~$10$ (blank space), while
%   the \cs{str_head_ignore_spaces:n} function ignores this space
%   character and leaves the first non-space character in the input
%   stream.  If the \meta{string} is empty (or only contains spaces in
%   the case of the \texttt{_ignore_spaces} function), then nothing is
%   left on the input stream.
% \end{function}
%
% \begin{function}[EXP, added = 2015-09-18]
%   {\str_tail:N, \str_tail:c, \str_tail:n, \str_tail_ignore_spaces:n}
%   \begin{syntax}
%     \cs{str_tail:n} \Arg{token list}
%   \end{syntax}
%   Converts the \meta{token list} to a \meta{string}, removes the first
%   character, and leaves the remaining characters (if any) in the input
%   stream, with category codes $12$ and $10$ (for spaces).  The
%   functions differ in the case where the first character is a space:
%   \cs{str_tail:N} and \cs{str_tail:n} will trim only that space, while
%   \cs{str_tail_ignore_spaces:n} removes the first non-space character
%   and any space before it.  If the \meta{token list} is empty (or
%   blank in the case of the \texttt{_ignore_spaces} variant), then
%   nothing is left on the input stream.
% \end{function}
%
% \begin{function}[EXP, added = 2015-09-18]
%   {\str_item:Nn, \str_item:nn, \str_item_ignore_spaces:nn}
%   \begin{syntax}
%     \cs{str_item:nn} \Arg{token list} \Arg{integer expression}
%   \end{syntax}
%   Converts the \meta{token list} to a \meta{string}, and leaves in the
%   input stream the character in position \meta{integer expression} of
%   the \meta{string}, starting at $1$ for the first (left-most)
%   character.  In the case of \cs{str_item:Nn} and \cs{str_item:nn},
%   all characters including spaces are taken into account.  The
%   \cs{str_item_ignore_spaces:nn} function skips spaces when counting
%   characters.  If the \meta{integer expression} is negative,
%   characters are counted from the end of the \meta{string}. Hence,
%   $-1$ is the right-most character, \emph{etc.}
% \end{function}
%
% \begin{function}[EXP, added = 2015-09-18]
%   {
%     \str_range:Nnn, \str_range:cnn, \str_range:nnn,
%     \str_range_ignore_spaces:nnn
%   }
%   \begin{syntax}
%     \cs{str_range:nnn} \Arg{token list} \Arg{start index} \Arg{end index}
%   \end{syntax}
%   Converts the \meta{token list} to a \meta{string}, and leaves in the
%   input stream the characters from the \meta{start index} to the
%   \meta{end index} inclusive.  Positive \meta{indices} are counted
%   from the start of the string, $1$~being the first character, and
%   negative \meta{indices} are counted from the end of the string,
%   $-1$~being the last character.  If either of \meta{start index} or
%   \meta{end index} is~$0$, the result is empty.  For instance,
%   \begin{verbatim}
%     \iow_term:x { \str_range:nnn { abcdef } { 2 } { 5 } }
%     \iow_term:x { \str_range:nnn { abcdef } { -4 } { -1 } }
%     \iow_term:x { \str_range:nnn { abcdef } { -2 } { -1 } }
%     \iow_term:x { \str_range:nnn { abcdef } { 0 } { -1 } }
%   \end{verbatim}
%   will print \texttt{bcde}, \texttt{cdef}, \texttt{ef}, and an empty
%   line to the terminal. The \meta{start index} must always be smaller than
%   or equal to the \meta{end index}: if this is not the case then no output
%   is generated. Thus
%   \begin{verbatim}
%     \iow_term:x { \str_range:nnn { abcdef } { 5 } { 2 } }
%     \iow_term:x { \str_range:nnn { abcdef } { -1 } { -4 } }
%   \end{verbatim}
%   both yield empty strings.
% \end{function}
%
% \section{String manipulation}
%
% \begin{function}[rEXP, added = 2015-03-01]
%    {
%      \str_lower_case:n, \str_lower_case:f,
%      \str_upper_case:n, \str_upper_case:f
%   }
%   \begin{syntax}
%     \cs{str_lower_case:n} \Arg{tokens}
%     \cs{str_upper_case:n} \Arg{tokens}
%   \end{syntax}
%   Converts the input \meta{tokens} to their string representation, as
%   described for \cs{tl_to_str:n}, and then to the lower or upper
%   case representation using a one-to-one mapping as described by the
%   Unicode Consortium file |UnicodeData.txt|.
%
%   These functions are intended for case changing programmatic data in
%   places where upper/lower case distinctions are meaningful. One example
%   would be automatically generating a function name from user input where
%   some case changing is needed. In this situation the input is programmatic,
%   not textual, case does have meaning and a language-independent one-to-one
%   mapping is appropriate. For example
%   \begin{verbatim}
%     \cs_new_protected:Npn \myfunc:nn #1#2
%       {
%         \cs_set_protected:cpn
%           {
%             user
%             \str_upper_case:f { \tl_head:n {#1} }
%             \str_lower_case:f { \tl_tail:n {#1} }
%           }
%           { #2 }
%       }
%   \end{verbatim}
%   would be used to generate a function with an auto-generated name consisting
%   of the upper case equivalent of the supplied name followed by the lower
%   case equivalent of the rest of the input.
%
%   These functions should \emph{not} be used for
%   \begin{itemize}
%     \item Caseless comparisons: use \cs{str_fold_case:n} for this
%       situation (case folding is district from lower casing).
%     \item Case changing text for typesetting: see the \cs{tl_lower_case:n(n)},
%       \cs{tl_upper_case:n(n)} and \cs{tl_mixed_case:n(n)} functions which
%       correctly deal with context-dependence and other factors appropriate
%       to text case changing.
%   \end{itemize}
%
%   \begin{texnote}
%     As with all \pkg{expl3} functions, the input supported by
%     \cs{str_fold_case:n} is \emph{engine-native} characters which are or
%     interoperate with \textsc{utf-8}. As such, when used with \pdfTeX{}
%     \emph{only} the Latin alphabet characters A--Z will be case-folded
%     (\emph{i.e.}~the \textsc{ascii} range which coincides with
%     \textsc{utf-8}). Full \textsc{utf-8} support is available with both
%     \XeTeX{} and \LuaTeX{}, subject only to the fact that \XeTeX{} in
%     particular has issues with characters of code above hexadecimal
%     $0\mathrm{xFFFF}$ when interacting with \cs{tl_to_str:n}.
%   \end{texnote}
% \end{function}
%
% \begin{function}[EXP, added = 2014-06-19, updated = 2016-03-07]
%   {\str_fold_case:n, \str_fold_case:V}
%   \begin{syntax}
%     \cs{str_fold_case:n} \Arg{tokens}
%   \end{syntax}
%   Converts the input \meta{tokens} to their string representation, as
%   described for \cs{tl_to_str:n}, and then folds the case of the resulting
%   \meta{string} to remove case information. The result of this process is
%   left in the input stream.
%
%   String folding is a process used for material such as identifiers rather
%   than for \enquote{text}. The folding provided by \cs{str_fold_case:n}
%   follows the mappings provided by the \href{http://www.unicode.org}^^A
%   {Unicode Consortium}, who
%   \href{http://www.unicode.org/faq/casemap_charprop.html#2}{state}:
%   \begin{quote}
%     Case folding is primarily used for caseless comparison of text, such
%     as identifiers in a computer program, rather than actual text
%     transformation. Case folding in Unicode is based on the lowercase
%     mapping, but includes additional changes to the source text to help make
%     it language-insensitive and consistent. As a result, case-folded text
%     should be used solely for internal processing and generally should not be
%     stored or displayed to the end user.
%   \end{quote}
%   The folding approach implemented by \cs{str_fold_case:n} follows the
%   \enquote{full} scheme defined by the Unicode Consortium
%   (\emph{e.g.}~\SS folds to \texttt{SS}). As case-folding is
%   a language-insensitive process, there is no special treatment of
%   Turkic input (\emph{i.e.}~\texttt{I} always folds to \texttt{i} and
%   not to \texttt{\i}).
%
%   \begin{texnote}
%     As with all \pkg{expl3} functions, the input supported by
%     \cs{str_fold_case:n} is \emph{engine-native} characters which are or
%     interoperate with \textsc{utf-8}. As such, when used with \pdfTeX{}
%     \emph{only} the Latin alphabet characters A--Z will be case-folded
%     (\emph{i.e.}~the \textsc{ascii} range which coincides with
%     \textsc{utf-8}). Full \textsc{utf-8} support is available with both
%     \XeTeX{} and \LuaTeX{}, subject only to the fact that \XeTeX{} in
%     particular has issues with characters of code above hexadecimal
%     $0\mathrm{xFFFF}$ when interacting with \cs{tl_to_str:n}.
%   \end{texnote}
% \end{function}
%
% \section{Viewing strings}
%
% \begin{function}[added = 2015-09-18]
%   {\str_show:N, \str_show:c, \str_show:n}
%   \begin{syntax}
%     \cs{str_show:N} \meta{str~var}
%   \end{syntax}
%   Displays the content of the \meta{str~var} on the terminal.
% \end{function}
%
% \section{Constant token lists}
%
% \begin{variable}[added = 2015-09-19]
%   {
%     \c_ampersand_str,
%     \c_atsign_str,
%     \c_backslash_str,
%     \c_left_brace_str,
%     \c_right_brace_str,
%     \c_circumflex_str,
%     \c_colon_str,
%     \c_dollar_str,
%     \c_hash_str,
%     \c_percent_str,
%     \c_tilde_str,
%     \c_underscore_str
%   }
%   Constant strings, containing a single character token, with category
%   code $12$.
% \end{variable}
%
% \section{Scratch strings}
%
% \begin{variable}{\l_tmpa_str, \l_tmpb_str}
%   Scratch strings for local assignment. These are never used by
%   the kernel code, and so are safe for use with any \LaTeX3-defined
%   function. However, they may be overwritten by other non-kernel
%   code and so should only be used for short-term storage.
% \end{variable}
%
% \begin{variable}{\g_tmpa_str, \g_tmpb_str}
%   Scratch strings for global assignment. These are never used by
%   the kernel code, and so are safe for use with any \LaTeX3-defined
%   function. However, they may be overwritten by other non-kernel
%   code and so should only be used for short-term storage.
% \end{variable}
%
% \subsection{Internal string functions}
%
% \begin{function}[EXP]{\__str_if_eq_x:nn}
%   \begin{syntax}
%     \cs{__str_if_eq_x:nn} \Arg{tl_1} \Arg{tl_2}
%   \end{syntax}
%   Compares the full expansion of two \meta{token lists} on a character by
%   character basis, and is \texttt{true} if the two lists contain the same
%   characters in the same order. Leaves |0| in the input stream if the
%   condition is true, and |+1| or |-1| otherwise.
% \end{function}
%
% \begin{function}{\__str_if_eq_x_return:nn}
%   \begin{syntax}
%     \cs{__str_if_eq_x_return:nn} \Arg{tl_1} \Arg{tl_2}
%   \end{syntax}
%   Compares the full expansion of two \meta{token lists} on a character by
%   character basis, and is \texttt{true} if the two lists contain the same
%   characters in the same order. Either \cs{prg_return_true:} or
%   \cs{prg_return_false:} is then left in the input stream. This is a version
%   of \cs{str_if_eq_x:nn(TF)} coded for speed.
% \end{function}
%
% \begin{function}[EXP]{\__str_to_other:n}
%   \begin{syntax}
%     \cs{__str_to_other:n} \Arg{token list}
%   \end{syntax}
%   Converts the \meta{token list} to a \meta{other string}, where
%   spaces have category code \enquote{other}.  This function can be
%   \texttt{f}-expanded without fear of losing a leading space, since
%   spaces do not have category code $10$ in its result.  It takes a
%   time quadratic in the character count of the string.
% \end{function}
%
% \begin{function}[EXP]{\__str_count:n}
%   \begin{syntax}
%     \cs{__str_count:n} \Arg{other string}
%   \end{syntax}
%   This function expects an argument that is entirely made of
%   characters with category \enquote{other}, as produced by
%   \cs{__str_to_other:n}.  It leaves in the input stream the number of
%   character tokens in the \meta{other string}, faster than the
%   analogous \cs{str_count:n} function.
% \end{function}
%
% \begin{function}[EXP]{\__str_range:nnn}
%   \begin{syntax}
%     \cs{__str_range:nnn} \Arg{other string} \Arg{start index} \Arg{end index}
%   \end{syntax}
%   Identical to \cs{str_range:nnn} except that the first argument is
%   expected to be entirely made of characters with category
%   \enquote{other}, as produced by \cs{__str_to_other:n}, and the
%   result is also an \meta{other string}.
% \end{function}
%
% \end{documentation}
%
% \begin{implementation}
%
% \section{\pkg{l3str} implementation}
%
%    \begin{macrocode}
%<*initex|package>
%    \end{macrocode}
%
%    \begin{macrocode}
%<@@=str>
%    \end{macrocode}
%
% \subsection{Creating and setting string variables}
%
% \begin{macro}
%   {
%     \str_new:N, \str_new:c,
%     \str_use:N, \str_use:c,
%     \str_clear:N, \str_clear:c,
%     \str_gclear:N,\str_gclear:c,
%     \str_clear_new:N, \str_clear_new:c,
%     \str_gclear_new:N, \str_gclear_new:c
%   }
% \begin{macro}
%   {
%     \str_set_eq:NN,  \str_set_eq:cN,  \str_set_eq:Nc,  \str_set_eq:cc,
%     \str_gset_eq:NN, \str_gset_eq:cN, \str_gset_eq:Nc, \str_gset_eq:cc
%   }
%   A string is simply a token list. The full mapping system isn't set up
%   yet so do things by hand.
%    \begin{macrocode}
\group_begin:
  \cs_set_protected:Npn \@@_tmp:n #1
    {
      \tl_if_blank:nF {#1}
        {
          \cs_new_eq:cc { str_ #1 :N } { tl_ #1 :N }
          \exp_args:Nc \cs_generate_variant:Nn { str_ #1 :N } { c }
          \@@_tmp:n
        }
    }
  \@@_tmp:n
    { new }
    { use }
    { clear }
    { gclear }
    { clear_new }
    { gclear_new }
    { }
\group_end:
\cs_new_eq:NN \str_set_eq:NN \tl_set_eq:NN
\cs_new_eq:NN \str_gset_eq:NN \tl_gset_eq:NN
\cs_generate_variant:Nn \str_set_eq:NN  { c , Nc , cc }
\cs_generate_variant:Nn \str_gset_eq:NN { c , Nc , cc }
%    \end{macrocode}
% \end{macro}
% \end{macro}
%
% \begin{macro}
%   {
%     \str_set:Nn, \str_set:Nx,
%     \str_set:cn, \str_set:cx,
%     \str_gset:Nn, \str_gset:Nx,
%     \str_gset:cn, \str_gset:cx,
%     \str_const:Nn, \str_const:Nx,
%     \str_const:cn, \str_const:cx,
%     \str_put_left:Nn, \str_put_left:Nx,
%     \str_put_left:cn, \str_put_left:cx,
%     \str_gput_left:Nn, \str_gput_left:Nx,
%     \str_gput_left:cn, \str_gput_left:cx,
%     \str_put_right:Nn, \str_put_right:Nx,
%     \str_put_right:cn, \str_put_right:cx,
%     \str_gput_right:Nn, \str_gput_right:Nx,
%     \str_gput_right:cn, \str_gput_right:cx,
%   }
%   Simply convert the token list inputs to \meta{strings}.
%    \begin{macrocode}
\group_begin:
  \cs_set_protected:Npn \@@_tmp:n #1
    {
      \tl_if_blank:nF {#1}
        {
          \cs_new_protected:cpx { str_ #1 :Nn } ##1##2
            { \exp_not:c { tl_ #1 :Nx } ##1 { \exp_not:N \tl_to_str:n {##2} } }
          \exp_args:Nc \cs_generate_variant:Nn { str_ #1 :Nn } { Nx , cn , cx }
          \@@_tmp:n
        }
    }
  \@@_tmp:n
    { set }
    { gset }
    { const }
    { put_left }
    { gput_left }
    { put_right }
    { gput_right }
    { }
\group_end:
%    \end{macrocode}
% \end{macro}
%
% \subsection{String comparisons}
%
% \begin{macro}[pTF, EXP]
%   {
%     \str_if_empty:N, \str_if_empty:c,
%     \str_if_exist:N, \str_if_exist:c
%   }
%   More copy-paste!
%    \begin{macrocode}
\prg_new_eq_conditional:NNn \str_if_exist:N \tl_if_exist:N { p , T , F , TF }
\prg_new_eq_conditional:NNn \str_if_exist:c \tl_if_exist:c { p , T , F , TF }
\prg_new_eq_conditional:NNn \str_if_empty:N \tl_if_empty:N { p , T , F , TF }
\prg_new_eq_conditional:NNn \str_if_empty:c \tl_if_empty:c { p , T , F , TF }
%    \end{macrocode}
% \end{macro}
%
% \begin{macro}[int, EXP]{\@@_if_eq_x:nn}
% \begin{macro}[aux, EXP]{\@@_escape_x:n}
%   String comparisons rely on the primitive \cs{(pdf)strcmp} if available:
%   \LuaTeX{} does not have it, so emulation is required. As the net result
%   is that we do not \emph{always} use the primitive, the correct approach
%   is to wrap up in a function with defined behaviour. That's done by
%   providing a wrapper and then redefining in the \LuaTeX{} case. Note that
%   the necessary Lua code is covered in \pkg{l3boostrap}: long-term this may
%   need to go into a separate Lua file, but at present it's somewhere that
%   spaces are not skipped for ease-of-input. The need to detokenize and force
%   expansion of input arises from the case where a |#| token is used in the
%   input, \emph{e.g.}~|\__str_if_eq_x:nn {#} { \tl_to_str:n {#} }|, which
%   otherwise will fail as \cs{luatex_luaescapestring:D} does not double
%   such tokens.
%    \begin{macrocode}
\cs_new:Npn \@@_if_eq_x:nn #1#2 { \pdftex_strcmp:D {#1} {#2} }
\cs_if_exist:NT \luatex_luatexversion:D
   {
     \cs_set:Npn \@@_if_eq_x:nn #1#2
       {
          \luatex_directlua:D
            {
              l3kernel.strcmp
                (
                  " \@@_escape_x:n {#1} " ,
                  " \@@_escape_x:n {#2} "
                )
            }
       }
     \cs_new:Npn \@@_escape_x:n #1
       {
         \luatex_luaescapestring:D
           {
             \etex_detokenize:D \exp_after:wN { \luatex_expanded:D {#1} }
           }
       }
   }
%    \end{macrocode}
% \end{macro}
% \end{macro}
%
% \begin{macro}[int, EXP]{\@@_if_eq_x_return:nn}
%   It turns out that we often need to compare a token list
%   with the result of applying some function to it, and
%   return with \cs{prg_return_true/false:}. This test is
%   similar to \cs{str_if_eq:nnTF} (see \pkg{l3str}),
%   but is hard-coded for speed.
%    \begin{macrocode}
\cs_new:Npn \@@_if_eq_x_return:nn #1 #2
  {
    \if_int_compare:w \@@_if_eq_x:nn {#1} {#2} = \c_zero
      \prg_return_true:
    \else:
      \prg_return_false:
    \fi:
  }
%    \end{macrocode}
% \end{macro}
%
% \begin{macro}[pTF, EXP]
%   {
%     \str_if_eq:nn, \str_if_eq:Vn, \str_if_eq:on, \str_if_eq:nV,
%     \str_if_eq:no, \str_if_eq:VV,
%     \str_if_eq_x:nn
%   }
%   Modern engines provide a direct way of comparing two token lists,
%   but returning a number. This set of conditionals therefore make life
%   a bit clearer. The \texttt{nn} and \texttt{xx} versions are created
%   directly as this is most efficient.
%    \begin{macrocode}
\prg_new_conditional:Npnn \str_if_eq:nn #1#2 { p , T , F , TF }
  {
    \if_int_compare:w
      \@@_if_eq_x:nn { \exp_not:n {#1} } { \exp_not:n {#2} }
      = \c_zero
      \prg_return_true: \else: \prg_return_false: \fi:
  }
\cs_generate_variant:Nn \str_if_eq_p:nn {  V ,  o }
\cs_generate_variant:Nn \str_if_eq_p:nn { nV , no , VV }
\cs_generate_variant:Nn \str_if_eq:nnT  {  V ,  o }
\cs_generate_variant:Nn \str_if_eq:nnT  { nV , no , VV }
\cs_generate_variant:Nn \str_if_eq:nnF  {  V ,  o }
\cs_generate_variant:Nn \str_if_eq:nnF  { nV , no , VV }
\cs_generate_variant:Nn \str_if_eq:nnTF {  V ,  o }
\cs_generate_variant:Nn \str_if_eq:nnTF { nV , no , VV }
\prg_new_conditional:Npnn \str_if_eq_x:nn #1#2 { p , T , F , TF }
  {
    \if_int_compare:w \@@_if_eq_x:nn {#1} {#2} = \c_zero
      \prg_return_true: \else: \prg_return_false: \fi:
  }
%    \end{macrocode}
% \end{macro}
%
% \begin{macro}[EXP, pTF]
%   {\str_if_eq:NN, \str_if_eq:Nc, \str_if_eq:cN, \str_if_eq:cc}
%   Note that \cs{str_if_eq:NN} is different from
%   \cs{tl_if_eq:NN} because it needs to ignore category codes.
%    \begin{macrocode}
\prg_new_conditional:Npnn \str_if_eq:NN #1#2 { p , TF , T , F }
  {
    \if_int_compare:w \@@_if_eq_x:nn { \tl_to_str:N #1 } { \tl_to_str:N #2 }
      = \c_zero \prg_return_true: \else: \prg_return_false: \fi:
  }
\cs_generate_variant:Nn \str_if_eq:NNT  { c , Nc , cc }
\cs_generate_variant:Nn \str_if_eq:NNF  { c , Nc , cc }
\cs_generate_variant:Nn \str_if_eq:NNTF { c , Nc , cc }
\cs_generate_variant:Nn \str_if_eq_p:NN { c , Nc , cc }
%    \end{macrocode}
% \end{macro}
%
% \begin{macro}[EXP]
%   {\str_case:nn, \str_case:on, \str_case:nV, \str_case:nv, \str_case_x:nn}
% \begin{macro}[EXP, TF]
%   {\str_case:nn, \str_case:on, \str_case:nV, \str_case:nv, \str_case_x:nn}
% \begin{macro}[EXP, aux]{\@@_case:nnTF, \@@_case_x:nnTF}
% \begin{macro}[aux, EXP]
%   {\@@_case:nw, \@@_case_x:nw, \@@_case_end:nw}
%   Much the same as \cs{tl_case:nn(TF)} here: just a change in the internal
%   comparison.
%    \begin{macrocode}
\cs_new:Npn \str_case:nn #1#2
  {
    \exp:w
    \@@_case:nnTF {#1} {#2} { } { }
  }
\cs_new:Npn \str_case:nnT #1#2#3
  {
    \exp:w
    \@@_case:nnTF {#1} {#2} {#3} { }
  }
\cs_new:Npn \str_case:nnF #1#2
  {
    \exp:w
    \@@_case:nnTF {#1} {#2} { }
  }
\cs_new:Npn \str_case:nnTF #1#2
  {
    \exp:w
    \@@_case:nnTF {#1} {#2}
  }
\cs_new:Npn \@@_case:nnTF #1#2#3#4
  { \@@_case:nw {#1} #2 {#1} { } \q_mark {#3} \q_mark {#4} \q_stop }
\cs_generate_variant:Nn \str_case:nn   { o , nV , nv }
\cs_generate_variant:Nn \str_case:nnT  { o , nV , nv }
\cs_generate_variant:Nn \str_case:nnF  { o , nV , nv }
\cs_generate_variant:Nn \str_case:nnTF { o , nV , nv }
\cs_new:Npn \@@_case:nw #1#2#3
  {
    \str_if_eq:nnTF {#1} {#2}
      { \@@_case_end:nw {#3} }
      { \@@_case:nw {#1} }
  }
\cs_new:Npn \str_case_x:nn #1#2
  {
    \exp:w
    \@@_case_x:nnTF {#1} {#2} { } { }
  }
\cs_new:Npn \str_case_x:nnT #1#2#3
  {
    \exp:w
    \@@_case_x:nnTF {#1} {#2} {#3} { }
  }
\cs_new:Npn \str_case_x:nnF #1#2
  {
    \exp:w
    \@@_case_x:nnTF {#1} {#2} { }
  }
\cs_new:Npn \str_case_x:nnTF #1#2
  {
    \exp:w
    \@@_case_x:nnTF {#1} {#2}
  }
\cs_new:Npn \@@_case_x:nnTF #1#2#3#4
  { \@@_case_x:nw {#1} #2 {#1} { } \q_mark {#3} \q_mark {#4} \q_stop }
\cs_new:Npn \@@_case_x:nw #1#2#3
  {
    \str_if_eq_x:nnTF {#1} {#2}
      { \@@_case_end:nw {#3} }
      { \@@_case_x:nw {#1} }
  }
\cs_new_eq:NN \@@_case_end:nw \__prg_case_end:nw
%    \end{macrocode}
% \end{macro}
% \end{macro}
% \end{macro}
% \end{macro}
%
% \subsection{Accessing specific characters in a string}
%
% \begin{macro}[EXP, int]{\@@_to_other:n}
% \begin{macro}[EXP, aux]{\@@_to_other_loop:w, \@@_to_other_end:w}
%   First apply \cs{tl_to_str:n}, then replace all spaces by
%   \enquote{other} spaces, $8$ at a time, storing the converted part of
%   the string between the \cs{q_mark} and \cs{q_stop} markers.  The end
%   is detected when \cs{@@_to_other_loop:w} finds one of the trailing
%   |A|, distinguished from any contents of the initial token list by
%   their category.  Then \cs{@@_to_other_end:w} is called, and finds
%   the result between \cs{q_mark} and the first |A| (well, there is
%   also the need to remove a space).
%    \begin{macrocode}
\cs_new:Npn \@@_to_other:n #1
  {
    \exp_after:wN \@@_to_other_loop:w
      \tl_to_str:n {#1} ~ A ~ A ~ A ~ A ~ A ~ A ~ A ~ A ~ \q_mark \q_stop
  }
\group_begin:
\tex_lccode:D `\* = `\  %
\tex_lccode:D `\A = `\A 
\tex_lowercase:D
  {
    \group_end:
    \cs_new:Npn \@@_to_other_loop:w
      #1 ~ #2 ~ #3 ~ #4 ~ #5 ~ #6 ~ #7 ~ #8 ~ #9 \q_stop
      {
        \if_meaning:w A #8
          \@@_to_other_end:w
        \fi:
        \@@_to_other_loop:w
        #9 #1 * #2 * #3 * #4 * #5 * #6 * #7 * #8 * \q_stop
      }
    \cs_new:Npn \@@_to_other_end:w \fi: #1 \q_mark #2 * A #3 \q_stop
      { \fi: #2 }
  }
%    \end{macrocode}
% \end{macro}
% \end{macro}
%
% \begin{macro}[EXP]
%   {\str_item:Nn, \str_item:cn, \str_item:nn, \str_item_ignore_spaces:nn}
% \begin{macro}[EXP, aux]{\@@_item:nn, \@@_item:w}
%   The \cs{str_item:nn} hands its argument with spaces escaped to
%   \cs{@@_item:nn}, and makes sure to turn the result back into
%   a proper string (with category code~$10$ spaces) eventually.  The
%   \cs{str_item_ignore_spaces:nn} function does not escape spaces,
%   which are thus ignored by \cs{@@_item:nn} since
%   everything else is done with undelimited arguments.
%   Evaluate the \meta{index} argument~|#2| and count characters in
%   the string, passing those two numbers to \cs{@@_item:w} for
%   further analysis.  If the \meta{index} is negative, shift it by
%   the \meta{count} to know the how many character to discard, and if
%   that is still negative give an empty result.  If the \meta{index}
%   is larger than the \meta{count}, give an empty result, and
%   otherwise discard $\meta{index}-1$ characters before returning the
%   following one.  The shift by $-1$ is obtained by inserting an empty
%   brace group before the string in that case: that brace group also
%   covers the case where the \meta{index} is zero.
%    \begin{macrocode}
\cs_new_nopar:Npn \str_item:Nn { \exp_args:No \str_item:nn }
\cs_generate_variant:Nn \str_item:Nn { c }
\cs_new:Npn \str_item:nn #1#2
  {
    \exp_args:Nf \tl_to_str:n
      {
        \exp_args:Nf \@@_item:nn
          { \@@_to_other:n {#1} } {#2}
      }
  }
\cs_new:Npn \str_item_ignore_spaces:nn #1
  { \exp_args:No \@@_item:nn { \tl_to_str:n {#1} } }
\cs_new:Npn \@@_item:nn #1#2
  {
    \exp_after:wN \@@_item:w
    \__int_value:w \__int_eval:w #2 \exp_after:wN ;
    \__int_value:w \@@_count:n {#1} ;
    #1 \q_stop
  }
\cs_new:Npn \@@_item:w #1; #2;
  {
    \int_compare:nNnTF {#1} < \c_zero
      {
        \int_compare:nNnTF {#1} < {-#2}
          { \use_none_delimit_by_q_stop:w }
          {
            \exp_after:wN \use_i_delimit_by_q_stop:nw
            \exp:w \exp_after:wN \@@_skip_exp_end:w
              \__int_value:w \__int_eval:w #1 + #2 ;
          }
      }
      {
        \int_compare:nNnTF {#1} > {#2}
          { \use_none_delimit_by_q_stop:w }
          {
            \exp_after:wN \use_i_delimit_by_q_stop:nw
            \exp:w \@@_skip_exp_end:w #1 ; { }
          }
      }
  }
%    \end{macrocode}
% \end{macro}
% \end{macro}
%
% \begin{macro}[EXP, aux]{\@@_skip_exp_end:w}
% \begin{macro}[EXP, aux]
%   {\@@_skip_loop:wNNNNNNNN, \@@_skip_end:w, \@@_skip_end:NNNNNNNN}
%   Removes |max(#1,0)| characters from the input stream, and then
%   leaves \cs{exp_end:}.  This should be expanded using
%   \cs{exp:w}.  We remove characters $8$ at a time until
%   there are at most $8$ to remove.  Then we do a dirty trick: the
%   \cs{if_case:w} construction leaves between $0$ and $8$ times the
%   \cs{or:} control sequence, and those \cs{or:} become arguments of
%   \cs{@@_skip_end:NNNNNNNN}.  If the number of characters to remove
%   is $6$, say, then there are two \cs{or:} left, and the $8$ arguments
%   of \cs{@@_skip_end:NNNNNNNN} are the two \cs{or:}, and $6$
%   characters from the input stream, exactly what we wanted to
%   remove. Then close the \cs{if_case:w} conditional with \cs{fi:}, and
%   stop the initial expansion with \cs{exp_end:} (see places where
%   \cs{@@_skip_exp_end:w} is called).
%    \begin{macrocode}
\cs_new:Npn \@@_skip_exp_end:w #1;
  {
    \if_int_compare:w #1 > \c_eight
      \exp_after:wN \@@_skip_loop:wNNNNNNNN
    \else:
      \exp_after:wN \@@_skip_end:w
      \__int_value:w \__int_eval:w
    \fi:
    #1 ;
  }
\cs_new:Npn \@@_skip_loop:wNNNNNNNN #1; #2#3#4#5#6#7#8#9
  { \exp_after:wN \@@_skip_exp_end:w \__int_value:w \__int_eval:w #1 - \c_eight ; }
\cs_new:Npn \@@_skip_end:w #1 ;
  {
    \exp_after:wN \@@_skip_end:NNNNNNNN
    \if_case:w #1 \exp_stop_f: \or: \or: \or: \or: \or: \or: \or: \or:
  }
\cs_new:Npn \@@_skip_end:NNNNNNNN #1#2#3#4#5#6#7#8 { \fi: \exp_end: }
%    \end{macrocode}
% \end{macro}
% \end{macro}
%
% \begin{macro}[EXP]
%   {\str_range:Nnn, \str_range:nnn, \str_range_ignore_spaces:nnn}
% \begin{macro}[EXP, int]{\@@_range:nnn}
% \begin{macro}[EXP, aux]{\@@_range:w, \@@_range:nnw}
%   Sanitize the string.  Then evaluate the arguments.  At this stage we
%   also decrement the \meta{start index}, since our goal is to know how
%   many characters should be removed.  Then limit the range to be
%   non-negative and at most the length of the string (this avoids
%   needing to check for the end of the string when grabbing
%   characters), shifting negative numbers by the appropriate amount.
%   Afterwards, skip characters, then keep some more, and finally drop
%   the end of the string.
%    \begin{macrocode}
\cs_new_nopar:Npn \str_range:Nnn { \exp_args:No \str_range:nnn }
\cs_generate_variant:Nn \str_range:Nnn { c }
\cs_new:Npn \str_range:nnn #1#2#3
  {
    \exp_args:Nf \tl_to_str:n
      {
        \exp_args:Nf \@@_range:nnn
          { \@@_to_other:n {#1} } {#2} {#3}
      }
  }
\cs_new:Npn \str_range_ignore_spaces:nnn #1
  { \exp_args:No \@@_range:nnn { \tl_to_str:n {#1} } }
\cs_new:Npn \@@_range:nnn #1#2#3
  {
    \exp_after:wN \@@_range:w
    \__int_value:w \@@_count:n {#1} \exp_after:wN ;
    \__int_value:w \__int_eval:w #2 - \c_one \exp_after:wN ;
    \__int_value:w \__int_eval:w #3 ;
    #1 \q_stop
  }
\cs_new:Npn \@@_range:w #1; #2; #3;
  {
    \exp_args:Nf \@@_range:nnw
      { \@@_range_normalize:nn {#2} {#1} }
      { \@@_range_normalize:nn {#3} {#1} }
  }
\cs_new:Npn \@@_range:nnw #1#2
  {
    \exp_after:wN \@@_collect_delimit_by_q_stop:w
    \__int_value:w \__int_eval:w #2 - #1 \exp_after:wN ;
    \exp:w \@@_skip_exp_end:w #1 ;
  }
%    \end{macrocode}
% \end{macro}
% \end{macro}
% \end{macro}
% \begin{macro}[EXP, aux]{\@@_range_normalize:nn}
%   This function converts an \meta{index} argument into an explicit
%   position in the string (a result of $0$ denoting \enquote{out of
%     bounds}).  Expects two explicit integer arguments: the
%   \meta{index} |#1| and the string count~|#2|.  If |#1| is negative,
%   replace it by $|#1| + |#2| + 1$, then limit to the range $[0,
%   |#2|]$.
%    \begin{macrocode}
\cs_new:Npn \@@_range_normalize:nn #1#2
  {
    \int_eval:n
      {
        \if_int_compare:w #1 < \c_zero
          \if_int_compare:w #1 < -#2 \exp_stop_f:
            \c_zero
          \else:
            #1 + #2 + \c_one
          \fi:
        \else:
          \if_int_compare:w #1 < #2 \exp_stop_f:
            #1
          \else:
            #2
          \fi:
        \fi:
      }
  }
%    \end{macrocode}
% \end{macro}
% \begin{macro}[EXP, aux]{\@@_collect_delimit_by_q_stop:w}
% \begin{macro}[EXP, aux]
%   {
%     \@@_collect_loop:wn, \@@_collect_loop:wnNNNNNNN,
%     \@@_collect_end:wn, \@@_collect_end:nnnnnnnnw
%   }
%   Collects |max(#1,0)| characters, and removes everything else until
%   \cs{q_stop}. This is somewhat similar to \cs{@@_skip_exp_end:w}, but
%   accepts integer expression arguments.  This time we can only grab
%   $7$ characters at a time.  At the end, we use an \cs{if_case:w}
%   trick again, so that the $8$ first arguments of
%   \cs{@@_collect_end:nnnnnnnnw} are some \cs{or:}, followed by an
%   \cs{fi:}, followed by |#1| characters from the input stream. Simply
%   leaving this in the input stream will close the conditional properly
%   and the \cs{or:} disappear.
%    \begin{macrocode}
\cs_new:Npn \@@_collect_delimit_by_q_stop:w #1;
  { \@@_collect_loop:wn #1 ; { } }
\cs_new:Npn \@@_collect_loop:wn #1 ;
  {
    \if_int_compare:w #1 > \c_seven
      \exp_after:wN \@@_collect_loop:wnNNNNNNN
    \else:
      \exp_after:wN \@@_collect_end:wn
    \fi:
    #1 ;
  }
\cs_new:Npn \@@_collect_loop:wnNNNNNNN #1; #2 #3#4#5#6#7#8#9
  {
    \exp_after:wN \@@_collect_loop:wn
    \__int_value:w \__int_eval:w #1 - \c_seven ;
    { #2 #3#4#5#6#7#8#9 }
  }
\cs_new:Npn \@@_collect_end:wn #1 ;
  {
    \exp_after:wN \@@_collect_end:nnnnnnnnw
    \if_case:w \if_int_compare:w #1 > \c_zero #1 \else: 0 \fi: \exp_stop_f:
    \or: \or: \or: \or: \or: \or: \fi:
  }
\cs_new:Npn \@@_collect_end:nnnnnnnnw #1#2#3#4#5#6#7#8 #9 \q_stop
  { #1#2#3#4#5#6#7#8 }
%    \end{macrocode}
% \end{macro}
% \end{macro}
%
% \subsection{Counting characters}
%
% \begin{macro}[EXP]
%   {\str_count_spaces:N, \str_count_spaces:c, \str_count_spaces:n}
% \begin{macro}[EXP, aux]{\@@_count_spaces_loop:w}
%   To speed up this function, we grab and discard $9$ space-delimited
%   arguments in each iteration of the loop.  The loop stops when the
%   last argument is one of the trailing |X|\meta{number}, and that
%   \meta{number} is added to the sum of $9$ that precedes, to adjust
%   the result.
%    \begin{macrocode}
\cs_new_nopar:Npn \str_count_spaces:N
  { \exp_args:No \str_count_spaces:n }
\cs_generate_variant:Nn \str_count_spaces:N { c }
\cs_new:Npn \str_count_spaces:n #1
  {
    \int_eval:n
      {
        \exp_after:wN \@@_count_spaces_loop:w
        \tl_to_str:n {#1} ~
        X 7 ~ X 6 ~ X 5 ~ X 4 ~ X 3 ~ X 2 ~ X 1 ~ X 0 ~ X -1 ~
        \q_stop
      }
  }
\cs_new:Npn \@@_count_spaces_loop:w #1~#2~#3~#4~#5~#6~#7~#8~#9~
  {
    \if_meaning:w X #9
      \use_i_delimit_by_q_stop:nw
    \fi:
    \c_nine + \@@_count_spaces_loop:w
  }
%    \end{macrocode}
% \end{macro}
% \end{macro}
%
% \begin{macro}[EXP]
%   {\str_count:N, \str_count:c, \str_count:n, \str_count_ignore_spaces:n}
% \begin{macro}[EXP, int]{\@@_count:n}
% \begin{macro}[EXP, aux]{\@@_count_aux:n, \@@_count_loop:NNNNNNNNN}
%   To count characters in a string we could first escape all spaces
%   using \cs{@@_to_other:n}, then pass the result to \cs{tl_count:n}.
%   However, the escaping step would be quadratic in the number of
%   characters in the string, and we can do better.  Namely, sum the
%   number of spaces (\cs{str_count_spaces:n}) and the result of
%   \cs{tl_count:n}, which ignores spaces.  Since strings tend to be
%   longer than token lists, we use specialized functions to count
%   characters ignoring spaces.  Namely, loop, grabbing $9$ non-space
%   characters at each step, and end as soon as we reach one of the $9$
%   trailing items.  The internal function \cs{@@_count:n}, used in
%   \cs{str_item:nn} and \cs{str_range:nnn}, is similar to
%   \cs{str_count_ignore_spaces:n} but expects its argument to already
%   be a string or a string with spaces escaped.
%    \begin{macrocode}
\cs_new_nopar:Npn \str_count:N { \exp_args:No \str_count:n }
\cs_generate_variant:Nn \str_count:N { c }
\cs_new:Npn \str_count:n #1
  {
    \@@_count_aux:n
      {
        \str_count_spaces:n {#1}
        + \exp_after:wN \@@_count_loop:NNNNNNNNN \tl_to_str:n {#1}
      }
  }
\cs_new:Npn \@@_count:n #1
  {
    \@@_count_aux:n
      { \@@_count_loop:NNNNNNNNN #1 }
  }
\cs_new:Npn \str_count_ignore_spaces:n #1
  {
    \@@_count_aux:n
      { \exp_after:wN \@@_count_loop:NNNNNNNNN \tl_to_str:n {#1} }
  }
\cs_new:Npn \@@_count_aux:n #1
  {
    \int_eval:n
      {
        #1
        { X \c_eight } { X \c_seven } { X \c_six   }
        { X \c_five  } { X \c_four  } { X \c_three }
        { X \c_two   } { X \c_one   } { X \c_zero  }
        \q_stop
      }
  }
\cs_new:Npn \@@_count_loop:NNNNNNNNN #1#2#3#4#5#6#7#8#9
  {
    \if_meaning:w X #9
      \exp_after:wN \use_none_delimit_by_q_stop:w
    \fi:
    \c_nine + \@@_count_loop:NNNNNNNNN
  }
%    \end{macrocode}
% \end{macro}
% \end{macro}
% \end{macro}
%
% \subsection{The first character in a string}
%
% \begin{macro}[EXP]
%   {\str_head:N, \str_head:c, \str_head:n, \str_head_ignore_spaces:n}
% \begin{macro}[EXP, aux]{\@@_head:w}
%   The \texttt{_ignore_spaces} variant applies \cs{tl_to_str:n} then
%   grabs the first item, thus skipping spaces.
%   As usual, \cs{str_head:N} expands its argument and
%   hands it to \cs{str_head:n}.  To circumvent the fact that \TeX{}
%   skips spaces when grabbing undelimited macro parameters,
%   \cs{@@_head:w} takes an argument delimited by a space. If |#1|
%   starts with a non-space character, \cs{use_i_delimit_by_q_stop:nw}
%   leaves that in the input stream. On the other hand, if |#1| starts
%   with a space, the \cs{@@_head:w} takes an empty argument, and the
%   single (initially braced) space in the definition of \cs{@@_head:w}
%   makes its way to the output. Finally, for an empty argument, the
%   (braced) empty brace group in the definition of \cs{str_head:n}
%   gives an empty result after passing through
%   \cs{use_i_delimit_by_q_stop:nw}.
%    \begin{macrocode}
\cs_new_nopar:Npn \str_head:N { \exp_args:No \str_head:n }
\cs_generate_variant:Nn \str_head:N { c }
\cs_set:Npn \str_head:n #1
  {
    \exp_after:wN \@@_head:w
    \tl_to_str:n {#1}
    { { } } ~ \q_stop
  }
\cs_set:Npn \@@_head:w #1 ~ %
  { \use_i_delimit_by_q_stop:nw #1 { ~ } }
\cs_new:Npn \str_head_ignore_spaces:n #1
  {
    \exp_after:wN \use_i_delimit_by_q_stop:nw
    \tl_to_str:n {#1} { } \q_stop
  }
%    \end{macrocode}
% \end{macro}
% \end{macro}
%
% \begin{macro}[EXP]
%   {\str_tail:N, \str_tail:c, \str_tail:n, \str_tail_ignore_spaces:n}
% \begin{macro}[EXP, aux]{\@@_tail_auxi:w, \@@_tail_auxii:w}
%   Getting the tail is a little bit more convoluted than the head of a
%   string.  We hit the front of the string with \cs{reverse_if:N}
%   \cs{if_charcode:w} \cs{scan_stop:}.  This removes the first
%   character, and necessarily makes the test true, since the character
%   cannot match \cs{scan_stop:}. The auxiliary function then inserts
%   the required \cs{fi:} to close the conditional, and leaves the tail
%   of the string in the input stream.  The details are such that an
%   empty string has an empty tail (this requires in particular that the
%   end-marker |X| be unexpandable and not a control sequence).  The
%   \texttt{_ignore_spaces} is rather simpler: after converting the
%   input to a string, \cs{@@_tail_auxii:w} removes one undelimited
%   argument and leaves everything else until an end-marker \cs{q_mark}.
%   One can check that an empty (or blank) string yields an empty
%   tail.
%    \begin{macrocode}
\cs_new_nopar:Npn \str_tail:N { \exp_args:No \str_tail:n }
\cs_generate_variant:Nn \str_tail:N { c }
\cs_set:Npn \str_tail:n #1
  {
    \exp_after:wN \@@_tail_auxi:w
    \reverse_if:N \if_charcode:w
        \scan_stop: \tl_to_str:n {#1} X X \q_stop
  }
\cs_set:Npn \@@_tail_auxi:w #1 X #2 \q_stop { \fi: #1 }
\cs_new:Npn \str_tail_ignore_spaces:n #1
  {
    \exp_after:wN \@@_tail_auxii:w
    \tl_to_str:n {#1} \q_mark \q_mark \q_stop
  }
\cs_new:Npn \@@_tail_auxii:w #1 #2 \q_mark #3 \q_stop { #2 }
%    \end{macrocode}
% \end{macro}
% \end{macro}
%
% \subsection{String manipulation}
%
% \begin{macro}[EXP]
%   {
%     \str_fold_case:n, \str_fold_case:V,
%     \str_lower_case:n, \str_lower_case:f,
%     \str_upper_case:n, \str_upper_case:f
%   }
% \begin{macro}[aux, EXP]{\@@_change_case:nn}
% \begin{macro}[aux, EXP]{\@@_change_case_aux:nn}
% \begin{macro}[aux, EXP]{\@@_change_case_result:n}
% \begin{macro}[aux, EXP]{\@@_change_case_output:nw, \@@_change_case_output:fw}
% \begin{macro}[aux, EXP]{\@@_change_case_end:nw}
% \begin{macro}[aux, EXP]{\@@_change_case_loop:nw}
% \begin{macro}[aux, EXP]{\@@_change_case_space:n}
% \begin{macro}[aux, EXP]{\@@_change_case_char:nN}
% \begin{macro}[aux]
%   {\@@_lookup_lower:N, \@@_lookup_upper:N, \@@_lookup_fold:N}
%   Case changing for programmatic reasons is done by first detokenizing
%   input then doing a simple loop that only has to worry about spaces
%   and everything else. The output is detokenized to allow data sharing
%   with text-based case changing.
%    \begin{macrocode}
\cs_new:Npn \str_fold_case:n  #1 { \@@_change_case:nn {#1} { fold } }
\cs_new:Npn \str_lower_case:n #1 { \@@_change_case:nn {#1} { lower } }
\cs_new:Npn \str_upper_case:n #1 { \@@_change_case:nn {#1} { upper } }
\cs_generate_variant:Nn \str_fold_case:n  { V }
\cs_generate_variant:Nn \str_lower_case:n { f }
\cs_generate_variant:Nn \str_upper_case:n { f }
\cs_new:Npn \@@_change_case:nn #1
  {
    \exp_after:wN \@@_change_case_aux:nn \exp_after:wN
      { \tl_to_str:n {#1} }
  }
\cs_new:Npn \@@_change_case_aux:nn #1#2
  {
    \@@_change_case_loop:nw {#2} #1 \q_recursion_tail \q_recursion_stop
      \@@_change_case_result:n { }
  }
\cs_new:Npn \@@_change_case_output:nw #1#2 \@@_change_case_result:n #3
  { #2 \@@_change_case_result:n { #3 #1 } }
\cs_generate_variant:Nn  \@@_change_case_output:nw { f }
\cs_new:Npn \@@_change_case_end:wn #1 \@@_change_case_result:n #2 { #2 }
\cs_new:Npn \@@_change_case_loop:nw #1#2 \q_recursion_stop
  {
    \tl_if_head_is_space:nTF {#2}
      { \@@_change_case_space:n }
      { \@@_change_case_char:nN }
    {#1} #2 \q_recursion_stop
  }
\use:x
  { \cs_new:Npn \exp_not:N \@@_change_case_space:n ##1 \c_space_tl }
  {
    \@@_change_case_output:nw { ~ }
    \@@_change_case_loop:nw {#1}
  }
\cs_new:Npn \@@_change_case_char:nN #1#2
  {
    \quark_if_recursion_tail_stop_do:Nn #2
      { \@@_change_case_end:wn }
    \cs_if_exist:cTF { c__unicode_ #1 _ #2 _tl }
      {
        \@@_change_case_output:fw
          { \tl_to_str:c { c__unicode_ #1 _ #2 _tl } }
      }
      { \@@_change_case_char_aux:nN {#1} #2 }
    \@@_change_case_loop:nw {#1}
  }
%    \end{macrocode}
%   For Unicode engines there's a look up to see if the current character
%   has a valid one-to-one case change mapping. That's not needed for $8$-bit
%   engines: as they don't have \cs{utex_char:D} all of the changes they can
%   make are hard-coded and so already picked up above.
%    \begin{macrocode}
\cs_if_exist:NTF \utex_char:D
  {
    \cs_new:Npn \@@_change_case_char_aux:nN #1#2
      {
        \int_compare:nNnTF { \use:c { __str_lookup_ #1 :N } #2 } = { 0 }
          { \@@_change_case_output:nw {#2} }
          {
            \@@_change_case_output:fw
              { \utex_char:D \use:c { __str_lookup_ #1 :N } #2 ~ }
          }
      }
    \cs_set_protected:Npn \@@_lookup_lower:N #1 { \tex_lccode:D `#1 }
    \cs_set_protected:Npn \@@_lookup_upper:N #1 { \tex_uccode:D `#1 }
    \cs_set_eq:NN \@@_lookup_fold:N \@@_lookup_lower:N
  }
  {
    \cs_new:Npn \@@_change_case_char_aux:nN #1#2
      { \@@_change_case_output:nw {#2} }
  }
%    \end{macrocode}
% \end{macro}
% \end{macro}
% \end{macro}
% \end{macro}
% \end{macro}
% \end{macro}
% \end{macro}
% \end{macro}
% \end{macro}
% \end{macro}
%
% \begin{variable}
%   {
%     \c_ampersand_str,
%     \c_atsign_str,
%     \c_backslash_str,
%     \c_left_brace_str,
%     \c_right_brace_str,
%     \c_circumflex_str,
%     \c_colon_str,
%     \c_dollar_str,
%     \c_hash_str,
%     \c_percent_str,
%     \c_tilde_str,
%     \c_underscore_str
%   }
%   For all of those strings, use \cs{cs_to_str:N} to get characters with
%   the correct category code without worries
%    \begin{macrocode}
\str_const:Nx \c_ampersand_str   { \cs_to_str:N \& }
\str_const:Nx \c_atsign_str      { \cs_to_str:N \@ }
\str_const:Nx \c_backslash_str   { \cs_to_str:N \\ }
\str_const:Nx \c_left_brace_str  { \cs_to_str:N \{ }
\str_const:Nx \c_right_brace_str { \cs_to_str:N \} }
\str_const:Nx \c_circumflex_str  { \cs_to_str:N \^ }
\str_const:Nx \c_colon_str       { \cs_to_str:N \: }
\str_const:Nx \c_dollar_str      { \cs_to_str:N \$ }
\str_const:Nx \c_hash_str        { \cs_to_str:N \# }
\str_const:Nx \c_percent_str     { \cs_to_str:N \% }
\str_const:Nx \c_tilde_str       { \cs_to_str:N \~ }
\str_const:Nx \c_underscore_str  { \cs_to_str:N \_ }
%    \end{macrocode}
% \end{variable}
%
% \begin{variable}{\l_tmpa_str, \l_tmpb_str, \g_tmpa_str, \g_tmpb_str}
%   Scratch strings.
%    \begin{macrocode}
\str_new:N \l_tmpa_str
\str_new:N \l_tmpb_str
\str_new:N \g_tmpa_str
\str_new:N \g_tmpb_str
%    \end{macrocode}
% \end{variable}
%
% \subsection{Viewing strings}
%
% \begin{macro}{\str_show:n, \str_show:N, \str_show:c}
%   Displays a string on the terminal.
%    \begin{macrocode}
\cs_new_eq:NN \str_show:n \tl_show:n
\cs_new_eq:NN \str_show:N \tl_show:N
\cs_generate_variant:Nn \str_show:N { c }
%    \end{macrocode}
% \end{macro}
%
% \subsection{Unicode data for case changing}
%
%    \begin{macrocode}
%<@@=unicode>
%    \end{macrocode}
%
% Case changing both for strings and \enquote{text} requires data from
% the Unicode Consortium. Some of this is build in to the format (as
% \tn{lccode} and \tn{uccode} values) but this covers only the simple
% one-to-one situations and does not fully handle for example case folding.
%
% The data required for cross-module manipulations is loaded here: currently
% this means for |str| and |tl| functions. As such, the prefix used is not
% |str| but rather |unicode|. For performance (as the entire data set must
% be read during each run) and as this code comes somewhat early in the
% load process, there is quite a bit of low-level code here.
%
% As only the data needs to remain at the end of this process, everything
% is set up inside a group.
%    \begin{macrocode}
\group_begin:
%    \end{macrocode}
% A read stream is needed. The I/O module is not yet in place \emph{and}
% we do not want to use up a stream. We therefore use a known free one in
% format mode or look for the next free one in package mode (covers plain,
% \LaTeXe{} and Con\TeX{}t MkII and MkIV).
%    \begin{macrocode}
%<*initex>
  \tex_chardef:D \g_@@_data_ior \c_zero
%</initex>
%<*package>
  \tex_chardef:D \g_@@_data_ior
    \etex_numexpr:D
      \cs_if_exist:NTF \lastallocatedread
        { \lastallocatedread }
        {
          \cs_if_exist:NTF \c_syst_last_allocated_read
            { \c_syst_last_allocated_read }
            { \tex_count:D 16 ~ }
        }
        + 1
    \scan_stop:
%</package>
%    \end{macrocode}
% Set up to read each file. As they use C-style comments, there is a need to
% deal with |#|. At the same time, spaces are important so they need to be
% picked up as they are important. Beyond that, the current category code
% scheme works fine. With no I/O loop available, hard-code one that will work
% quickly.
%    \begin{macrocode}
  \cs_set_protected:Npn \@@_map_inline:n #1
    {
      \group_begin:
        \tex_catcode:D `\# = 12 \scan_stop:
        \tex_catcode:D `\  = 10 \scan_stop:
        \tex_openin:D \g_@@_data_ior = #1 \scan_stop:
        \cs_if_exist:NT \utex_char:D 
          { \@@_map_loop: }
        \tex_closein:D \g_@@_data_ior
      \group_end:
    }
  \cs_set_protected:Npn \@@_map_loop:
    {
      \tex_ifeof:D \g_@@_data_ior
        \exp_after:wN \use_none:n
      \else:
        \exp_after:wN \use:n
      \fi:
        {
          \tex_read:D \g_@@_data_ior to \l_@@_tmp_tl
          \if_meaning:w \c_empty_tl \l_@@_tmp_tl
          \else:
            \exp_after:wN \@@_parse:w \l_@@_tmp_tl \q_stop
          \fi:
          \@@_map_loop:
        }
    }
%    \end{macrocode}
% The lead-off parser for each line is common for all of the files. If
% the line starts with a |#| it's a comment. There's one special comment
% line to look out for in \texttt{SpecialCasing.txt} as we want to ignore
% everything after it. As this line does not appear in any other sources
% and the test is quite quick (there are relatively few comment lines), it
% can be present in all of the passes.
%    \begin{macrocode}
  \cs_set_protected:Npn \@@_parse:w #1#2 \q_stop
    {
      \reverse_if:N \if:w \c_hash_str #1
        \@@_parse_auxi:w #1#2 \q_stop
      \else:
        \if_int_compare:w \__str_if_eq_x:nn
          { \exp_not:n {#2} } { ~Conditional~Mappings~ } = \c_zero
          \cs_set_protected:Npn \@@_parse:w ##1 \q_stop { }
        \fi:
      \fi:
    }
%    \end{macrocode}
% Storing each exception is always done in the same way: create a constant
% token list which expands to exactly the mapping. These will have the
% category codes \enquote{now} (so should be letters) but will be detokenized
% for string use.
%    \begin{macrocode}
  \cs_set_protected:Npn \@@_store:nnnnn #1#2#3#4#5
    {
      \tl_const:cx { c_@@_ #2 _ \utex_char:D "#1 _tl }
        {
          \utex_char:D "#3 ~
          \utex_char:D "#4 ~
          \tl_if_blank:nF {#5}
            { \utex_char:D "#5 }
        }
    }
%    \end{macrocode}   
% Parse the main Unicode data file for title case exceptions (the one-to-one
% lower and upper case mappings it contains will all be covered by the \TeX{}
% data).
%    \begin{macrocode}
  \cs_set_protected:Npn \@@_parse_auxi:w
    #1 ; #2 ; #3 ; #4 ; #5 ; #6 ; #7 ; #8 ; #9 ;
    { \@@_parse_auxii:w #1 ; }
  \cs_set_protected:Npn \@@_parse_auxii:w
    #1 ; #2 ; #3 ; #4 ; #5 ; #6 ; #7 \q_stop
    {
      \tl_if_blank:nF {#7}
        {
          \if_int_compare:w \__str_if_eq_x:nn { #5 ~ } {#7} = \c_zero
          \else:
            \tl_const:cx
              { c_@@_title_ \utex_char:D "#1 _tl }
              { \utex_char:D "#7 }
          \fi:
        }
    }
  \@@_map_inline:n { UnicodeData.txt }
%    \end{macrocode}   
%  The set up for case folding is in two parts. For the basic (core) mappings,
%  folding is the same as lower casing in most positions so only store
%  the differences. For the more complex foldings, always store the result,
%  splitting up the two or three code points in the input as required.
%    \begin{macrocode}
  \cs_set_protected:Npn \@@_parse_auxi:w #1 ;~ #2 ;~ #3 ; #4 \q_stop
    {
      \if_int_compare:w \__str_if_eq_x:nn {#2} { C } = \c_zero
        \if_int_compare:w \tex_lccode:D "#1 = "#3 \scan_stop:
        \else:
          \tl_const:cx
            { c_@@_fold_ \utex_char:D "#1 _tl }
            { \utex_char:D "#3 ~ }
        \fi:
      \else:
        \if_int_compare:w \__str_if_eq_x:nn {#2} { F } = \c_zero
          \@@_parse_auxii:w #1 ~ #3 ~ \q_stop
        \fi:
      \fi:
    }
  \cs_set_protected:Npn \@@_parse_auxii:w #1 ~ #2 ~ #3 ~ #4 \q_stop
    { \@@_store:nnnnn {#1} { fold } {#2} {#3} {#4} }
  \@@_map_inline:n { CaseFolding.txt }
%    \end{macrocode}
% For upper and lower casing special situations, there is a bit more to
% do as we also have title casing to consider.
%    \begin{macrocode}
  \cs_set_protected:Npn \@@_parse_auxi:w #1 ;~ #2 ;~ #3 ;~ #4 ; #5 \q_stop
    {
      \use:n { \@@_parse_auxii:w #1 ~ lower ~ #2 ~ } ~ \q_stop
      \use:n { \@@_parse_auxii:w #1 ~ upper ~ #4 ~ } ~ \q_stop
      \if_int_compare:w \__str_if_eq_x:nn {#3} {#4} = \c_zero
      \else:
        \use:n { \@@_parse_auxii:w #1 ~ title ~ #3 ~ } ~ \q_stop
      \fi:
    }
  \cs_set_protected:Npn \@@_parse_auxii:w #1 ~ #2 ~ #3 ~ #4 ~ #5 \q_stop
    {
      \tl_if_empty:nF {#4}
        { \@@_store:nnnnn {#1} {#2} {#3} {#4} {#5} }
    }
  \@@_map_inline:n { SpecialCasing.txt }
%    \end{macrocode}
% For the $8$-bit engines, the above does nothing but there is some set
% up needed. There is no expandable character generator primitive so some
% alternative is needed. As we've not used up hash space for the above, we
% can go for the fast approach here of one name per letter. Keeping folding
% and lower casing separate makes the use later a bit easier.
%    \begin{macrocode}
  \cs_if_exist:NF \utex_char:D
    {
      \cs_set_protected:Npn \@@_tmp:NN #1#2
        {
          \if_meaning:w \q_recursion_tail #2
            \exp_after:wN \use_none_delimit_by_q_recursion_stop:w
          \fi:
          \tl_const:cn { c_@@_fold_  #1 _tl } {#2}
          \tl_const:cn { c_@@_lower_ #1 _tl } {#2}
          \tl_const:cn { c_@@_upper_ #2 _tl } {#1}
          \@@_tmp:NN
        }
      \@@_tmp:NN
        AaBbCcDdEeFfGgHhIiJjKkLlMmNnOoPpQqRrSsTtUuVvWwXxYyZz
        ? \q_recursion_tail \q_recursion_stop
    }
%    \end{macrocode}
%
% All done: tidy up.
%    \begin{macrocode}
\group_end:
%    \end{macrocode}
%
%    \begin{macrocode}
%</initex|package>
%    \end{macrocode}
%
% \end{implementation}
%
% \PrintIndex