% \iffalse meta-comment
%
%% File: l3str.dtx Copyright (C) 2014-2015 The LaTeX3 Project
%%
%% It may be distributed and/or modified under the conditions of the
%% LaTeX Project Public License (LPPL), either version 1.3c of this
%% license or (at your option) any later version.  The latest version
%% of this license is in the file
%%
%%    http://www.latex-project.org/lppl.txt
%%
%% This file is part of the "l3kernel bundle" (The Work in LPPL)
%% and all files in that bundle must be distributed together.
%%
%% The released version of this bundle is available from CTAN.
%%
%% -----------------------------------------------------------------------
%%
%% The development version of the bundle can be found at
%%
%%    http://www.latex-project.org/svnroot/experimental/trunk/
%%
%% for those people who are interested.
%%
%%%%%%%%%%%
%% NOTE: %%
%%%%%%%%%%%
%%
%%   Snapshots taken from the repository represent work in progress and may
%%   not work or may contain conflicting material!  We therefore ask
%%   people _not_ to put them into distributions, archives, etc. without
%%   prior consultation with the LaTeX3 Project.
%%
%% -----------------------------------------------------------------------
%
%<*driver>
\documentclass[full]{l3doc}
%</driver>
%<*driver|package>
\GetIdInfo$Id$
  {L3 Strings}
%</driver|package>
%<*driver>
\begin{document}
  \DocInput{\jobname.dtx}
\end{document}
%</driver>
% \fi
%
% \title{^^A
%   The \pkg{l3str} package\\Strings^^A
%   \thanks{This file describes v\ExplFileVersion,
%      last revised \ExplFileDate.}^^A
% }
%
% \author{^^A
%  The \LaTeX3 Project\thanks
%    {^^A
%      E-mail:
%        \href{mailto:latex-team@latex-project.org}
%          {latex-team@latex-project.org}^^A
%    }^^A
% }
%
% \date{Released \ExplFileDate}
%
% \maketitle
%
% \begin{documentation}
%
% \TeX{} associates each character with a category code: as such, there is no
% concept of a \enquote{string} as commonly understood in many other
% programming languages. However, there are places where we wish to manipulate
% token lists while in some sense \enquote{ignoring} category codes: this is
% done by treating token lists as strings in a \TeX{} sense.
%
% A \TeX{} string (and thus an \pkg{expl3} string) is a series of characters
% which have category code $12$ (\enquote{other}) with the exception of
% space characters which have category code $10$ (\enquote{space}). Thus
% at a technical level, a \TeX{} string is a token list with the appropriate
% category codes. In this documentation, these will simply be referred to as
% strings: note that they can be stored in token lists as normal.
%
% The functions documented here take literal token lists,
% convert to strings and then carry out manipulations. Thus they may
% informally be described as \enquote{ignoring} category code. Note that
% the functions \cs{cs_to_str:N}, \cs{tl_to_str:n}, \cs{tl_to_str:N} and
% \cs{token_to_str:N} (and variants) will generate strings from the appropriate
% input: these are documented in \pkg{l3basics}, \pkg{l3tl} and \pkg{l3token},
% respectively.
%
% \section{The first character from a string}
%
% \begin{function}[added = 2011-08-10, EXP]{\str_head:n, \str_tail:n}
%   \begin{syntax}
%     \cs{str_head:n} \Arg{token list}
%     \cs{str_tail:n} \Arg{token list}
%   \end{syntax}
%   Converts the \meta{token list} into a string, as described for
%   \cs{tl_to_str:n}. The \cs{str_head:n} function then leaves
%   the first character of this string in the input stream.
%   The \cs{str_tail:n} function leaves all characters except
%   the first in the input stream. The first character may be
%   a space. If the \meta{token list} argument is entirely empty,
%   nothing is left in the input stream.
% \end{function}
%
% \subsection{Tests on strings}
%
% \begin{function}[EXP,pTF]
%   {
%     \str_if_eq:nn, \str_if_eq:Vn, \str_if_eq:on, \str_if_eq:no,
%     \str_if_eq:nV, \str_if_eq:VV
%   }
%   \begin{syntax}
%     \cs{str_if_eq_p:nn} \Arg{tl_1} \Arg{tl_2}
%     \cs{str_if_eq:nnTF} \Arg{tl_1} \Arg{tl_2} \Arg{true code} \Arg{false code}
%   \end{syntax}
%   Compares the two \meta{token lists} on a character by character
%   basis, and is \texttt{true} if the two lists contain the same
%   characters in the same order. Thus for example
%   \begin{verbatim}
%     \str_if_eq_p:no { abc } { \tl_to_str:n { abc } }
%   \end{verbatim}
%   is logically \texttt{true}.
% \end{function}
%
% \begin{function}[EXP,pTF, added = 2012-06-05]{\str_if_eq_x:nn}
%   \begin{syntax}
%     \cs{str_if_eq_x_p:nn} \Arg{tl_1} \Arg{tl_2}
%     \cs{str_if_eq_x:nnTF} \Arg{tl_1} \Arg{tl_2} \Arg{true code} \Arg{false code}
%   \end{syntax}
%   Compares the full expansion of two \meta{token lists} on a character by
%   character basis, and is \texttt{true} if the two lists contain the same
%   characters in the same order. Thus for example
%   \begin{verbatim}
%     \str_if_eq_x_p:nn { abc } { \tl_to_str:n { abc } }
%   \end{verbatim}
%   is logically \texttt{true}.
% \end{function}
%
% \begin{function}[added = 2013-07-24, updated = 2015-02-28, EXP, TF]
%   {\str_case:nn, \str_case:on, \str_case:nV, \str_case:nv}
%   \begin{syntax}
%     \cs{str_case:nnTF} \Arg{test string} \\
%     ~~|{| \\
%     ~~~~\Arg{string case_1} \Arg{code case_1} \\
%     ~~~~\Arg{string case_2} \Arg{code case_2} \\
%     ~~~~\ldots \\
%     ~~~~\Arg{string case_n} \Arg{code case_n} \\
%     ~~|}| \\
%     ~~\Arg{true code}
%     ~~\Arg{false code}
%   \end{syntax}
%   This function compares the \meta{test string} in turn with each
%   of the \meta{string cases}. If the two are equal (as described for
%   \cs{str_if_eq:nnTF} then the
%   associated \meta{code} is left in the input stream. If any of the
%   cases are matched, the \meta{true code} is also inserted into the
%   input stream (after the code for the appropriate case), while if none
%   match then the \meta{false code} is inserted. The function
%   \cs{str_case:nn}, which does nothing if there is no match, is also
%   available.
% \end{function}
%
% \begin{function}[added = 2013-07-24, EXP, TF]{\str_case_x:nn}
%   \begin{syntax}
%     \cs{str_case_x:nnF} \Arg{test string} \\
%     ~~|{| \\
%     ~~~~\Arg{string case_1} \Arg{code case_1} \\
%     ~~~~\Arg{string case_2} \Arg{code case_2} \\
%     ~~~~\ldots \\
%     ~~~~\Arg{string case_n} \Arg{code case_n} \\
%     ~~|}| \\
%     ~~\Arg{true code}
%     ~~\Arg{false code}
%   \end{syntax}
%   This function compares the full expansion of the \meta{test string}
%   in turn with the full expansion of the \meta{string cases}. If the two
%   full expansions are equal (as described for \cs{str_if_eq:nnTF} then the
%   associated \meta{code} is left in the input stream.  If any of the
%   cases are matched, the \meta{true code} is also inserted into the
%   input stream (after the code for the appropriate case), while if none
%   match then the \meta{false code} is inserted. The function
%   \cs{str_case_x:nn}, which does nothing if there is no match, is also
%   available.
%   The \meta{test string} is expanded in each comparison, and must
%   always yield the same result: for example, random numbers must
%   not be used within this string.
% \end{function}
%
% \section{String manipulation}
%
% \begin{function}[rEXP, added = 2015-03-01]
%    {
%      \str_lower_case:n, \str_lower_case:f, 
%      \str_upper_case:n, \str_upper_case:f
%   }
%   \begin{syntax}
%     \cs{str_lower_case:n} \Arg{tokens}
%     \cs{str_upper_case:n} \Arg{tokens}
%   \end{syntax}
%   Converts the input \meta{tokens} to their string representation, as
%   described for \cs{tl_to_str:n}, and then to the lower or upper
%   case representation using a one-to-one mapping as described by the
%   Unicode Consortium file |UnicodeData.txt|.
%   
%   These functions are intended for case changing programmatic data in
%   places where upper/lower case distinctions are meaningful. One example
%   would be automatically generating a function name from user input where
%   some case changing is needed. In this situation the input is programmatic,
%   not textual, case does have meaning and a language-independent one-to-one
%   mapping is appropriate. For example
%   \begin{verbatim}
%     \cs_new_protected:Npn \myfunc:nn #1#2
%       {
%         \cs_set_protected:cpn
%           {
%             user
%             \str_upper_case:f { \tl_head:n {#1} }
%             \str_lower_case:f { \tl_tail:n {#1} }
%           }
%           { #2 }
%       }
%   \end{verbatim}
%   would be used to generate a function with an auto-generated name consisting
%   of the upper case equivalent of the supplied name followed by the lower
%   case equivalent of the rest of the input.
%   
%   These functions should \emph{not} be used for
%   \begin{itemize}
%     \item Caseless comparisons: use \cs{str_fold_case:n} for this
%       situation (case folding is district from lower casing).
%     \item Case changing text for typesetting: see the \cs{tl_lower_case:n(n)},
%       \cs{tl_upper_case:n(n)} and \cs{tl_mixed_case:n(n)} functions which
%       correctly deal with context-dependence and other factors appropriate
%       to text case changing.
%   \end{itemize}
%
%   \begin{texnote}
%     As with all \pkg{expl3} functions, the input supported by
%     \cs{str_fold_case:n} is \emph{engine-native} characters which are or
%     interoperate with \textsc{utf-8}. As such, when used with \pdfTeX{}
%     \emph{only} the Latin alphabet characters A--Z will be case-folded
%     (\emph{i.e.}~the \textsc{ascii} range which coincides with
%     \textsc{utf-8}). Full \textsc{utf-8} support is available with both
%     \XeTeX{} and \LuaTeX{}, subject only to the fact that \XeTeX{} in
%     particular has issues with characters of code above hexadecimal
%     $0\mathrm{xFFF}$ when interacting with \cs{tl_to_str:n}.
%   \end{texnote}
% \end{function}
%
% \begin{function}[rEXP, added = 2014-06-19, updated = 2015-03-01]
%   {\str_fold_case:n}
%   \begin{syntax}
%     \cs{str_fold_case:n} \Arg{tokens}
%   \end{syntax}
%   Converts the input \meta{tokens} to their string representation, as
%   described for \cs{tl_to_str:n}, and then folds the case of the resulting
%   \meta{string} to remove case information. The result of this process is
%   left in the input stream.
%
%   String folding is a process used for material such as identifiers rather
%   than for \enquote{text}. The folding provided by \cs{str_fold_case:n}
%   follows the mappings provided by the \href{http://www.unicode.org}^^A
%   {Unicode Consortium}, who
%   \href{http://www.unicode.org/faq/casemap_charprop.html#2}{state}:
%   \begin{quote}
%     Case folding is primarily used for caseless comparison of text, such
%     as identifiers in a computer program, rather than actual text
%     transformation. Case folding in Unicode is based on the lowercase
%     mapping, but includes additional changes to the source text to help make
%     it language-insensitive and consistent. As a result, case-folded text
%     should be used solely for internal processing and generally should not be
%     stored or displayed to the end user.
%   \end{quote}
%   The folding approach implemented by \cs{str_fold_case:n} follows the
%   \enquote{full} scheme defined by the Unicode Consortium
%   (\emph{e.g.}~\SS folds to \texttt{SS}). As case-folding is
%   a language-insensitive process, there is no special treatment of
%   Turkic input (\emph{i.e.}~\texttt{I} always folds to \texttt{i} and
%   not to \texttt{\i}).
%
%   \begin{texnote}
%     As with all \pkg{expl3} functions, the input supported by
%     \cs{str_fold_case:n} is \emph{engine-native} characters which are or
%     interoperate with \textsc{utf-8}. As such, when used with \pdfTeX{}
%     \emph{only} the Latin alphabet characters A--Z will be case-folded
%     (\emph{i.e.}~the \textsc{ascii} range which coincides with
%     \textsc{utf-8}). Full \textsc{utf-8} support is available with both
%     \XeTeX{} and \LuaTeX{}, subject only to the fact that \XeTeX{} in
%     particular has issues with characters of code above hexadecimal
%     $0\mathrm{xFFF}$ when interacting with \cs{tl_to_str:n}.
%   \end{texnote}
% \end{function}
%
% \subsection{Internal string functions}
%
% \begin{function}[EXP]{\__str_if_eq_x:nn}
%   \begin{syntax}
%     \cs{__str_if_eq_x:nn} \Arg{tl_1} \Arg{tl_2}
%   \end{syntax}
%   Compares the full expansion of two \meta{token lists} on a character by
%   character basis, and is \texttt{true} if the two lists contain the same
%   characters in the same order. Leaves |0| in the input stream if the
%   condition is true, and |+1| or |-1| otherwise.
% \end{function}
%
% \begin{function}{\__str_if_eq_x_return:nn}
%   \begin{syntax}
%     \cs{__str_if_eq_x_return:nn} \Arg{tl_1} \Arg{tl_2}
%   \end{syntax}
%   Compares the full expansion of two \meta{token lists} on a character by
%   character basis, and is \texttt{true} if the two lists contain the same
%   characters in the same order. Either \cs{prg_return_true:} or
%   \cs{prg_return_false:} is then left in the input stream. This is a version
%   of \cs{str_if_eq_x:nn(TF)} coded for speed.
% \end{function}
%
% \end{documentation}
%
% \begin{implementation}
%
% \section{\pkg{l3str} implementation}
%
%    \begin{macrocode}
%<*initex|package>
%    \end{macrocode}
%
%    \begin{macrocode}
%<@@=str>
%    \end{macrocode}
%
% \begin{macro}{\str_head:n, \str_tail:n}
% \begin{macro}[aux]{\__str_head:w}
% \begin{macro}[aux]{\__str_tail:w}
%   After \cs{tl_to_str:n}, we have a list of character tokens,
%   all with category code 12, except the space, which has category
%   code 10. Directly using \cs{tl_head:w} would thus lose leading spaces.
%   Instead, we take an argument delimited by an explicit space, and
%   then only use \cs{tl_head:w}. If the string started with a
%   space, then the argument of \cs{__str_head:w} is empty, and
%   the function correctly returns a space character. Otherwise,
%   it returns the first token of |#1|, which is the first token
%   of the string. If the string is empty, we return an empty result.
%
%   To remove the first character of \cs{tl_to_str:n} |{#1}|,
%   we test it using \cs{if_charcode:w} \cs{scan_stop:},
%   always \texttt{false} for characters. If the argument was non-empty,
%   then \cs{__str_tail:w} returns everything until the first
%   \texttt{X} (with category code letter, no risk of confusing
%   with the user input). If the argument was empty, the first
%   \texttt{X} is taken by \cs{if_charcode:w}, and nothing
%   is returned. We use \texttt{X} as a \meta{marker}, rather than
%   a quark because the test \cs{if_charcode:w} \cs{scan_stop:}
%   \meta{marker} has to be \texttt{false}.
%    \begin{macrocode}
\cs_new:Npn \str_head:n #1
  {
    \exp_after:wN \__str_head:w
    \tl_to_str:n {#1}
    { { } } ~ \q_stop
  }
\cs_new:Npn \__str_head:w #1 ~ %
  { \tl_head:w #1 { ~ } }
\cs_new:Npn \str_tail:n #1
  {
    \exp_after:wN \__str_tail:w
    \reverse_if:N \if_charcode:w
        \scan_stop: \tl_to_str:n {#1} X X \q_stop
  }
\cs_new:Npn \__str_tail:w #1 X #2 \q_stop { \fi: #1 }
%    \end{macrocode}
% \end{macro}
% \end{macro}
% \end{macro}
%
% \subsection{String comparisons}
%
% \begin{macro}[int, EXP]{\__str_if_eq_x:nn}
% \begin{macro}[aux, EXP]{\__str_escape_x:n}
%   String comparisons rely on the primitive \cs{(pdf)strcmp} if available:
%   \LuaTeX{} does not have it, so emulation is required. As the net result
%   is that we do not \emph{always} use the primitive, the correct approach
%   is to wrap up in a function with defined behaviour. That's done by
%   providing a wrapper and then redefining in the \LuaTeX{} case. Note that
%   the necessary Lua code is covered in \pkg{l3boostrap}: long-term this may
%   need to go into a separate Lua file, but at present it's somewhere that
%   spaces are not skipped for ease-of-input. The need to detokenize and force
%   expansion of input arises from the case where a |#| token is used in the
%   input, \emph{e.g.}~|\__str_if_eq_x:nn {#} { \tl_to_str:n {#} }|, which
%   otherwise will fail as \cs{luatex_luaescapestring:D} does not double
%   such tokens.
%    \begin{macrocode}
\cs_new:Npn \__str_if_eq_x:nn #1#2 { \pdftex_strcmp:D {#1} {#2} }
\luatex_if_engine:T
   {
     \cs_set:Npn \__str_if_eq_x:nn #1#2
       {
          \luatex_directlua:D
            {
              l3kernel.strcmp
                (
                  " \__str_escape_x:n {#1} " ,
                  " \__str_escape_x:n {#2} "
                )
            }
       }
     \cs_new:Npn \__str_escape_x:n #1
       {
         \luatex_luaescapestring:D
           {
             \etex_detokenize:D \exp_after:wN { \luatex_expanded:D {#1} }
           }
       }
   }
%    \end{macrocode}
% \end{macro}
% \end{macro}
%
% \begin{macro}[int, EXP]{\__str_if_eq_x_return:nn}
%   It turns out that we often need to compare a token list
%   with the result of applying some function to it, and
%   return with \cs{prg_return_true/false:}. This test is
%   similar to \cs{str_if_eq:nnTF} (see \pkg{l3str}),
%   but is hard-coded for speed.
%    \begin{macrocode}
\cs_new:Npn \__str_if_eq_x_return:nn #1 #2
  {
    \if_int_compare:w \__str_if_eq_x:nn {#1} {#2} = \c_zero
      \prg_return_true:
    \else:
      \prg_return_false:
    \fi:
  }
%    \end{macrocode}
% \end{macro}
%
% \begin{macro}[pTF, EXP]
%   {
%     \str_if_eq:nn, \str_if_eq:Vn, \str_if_eq:on, \str_if_eq:nV,
%     \str_if_eq:no, \str_if_eq:VV,
%     \str_if_eq_x:nn
%   }
%   Modern engines provide a direct way of comparing two token lists,
%   but returning a number. This set of conditionals therefore make life
%   a bit clearer. The \texttt{nn} and \texttt{xx} versions are created
%   directly as this is most efficient.
%    \begin{macrocode}
\prg_new_conditional:Npnn \str_if_eq:nn #1#2 { p , T , F , TF }
  {
    \if_int_compare:w
      \__str_if_eq_x:nn { \exp_not:n {#1} } { \exp_not:n {#2} }
      = \c_zero
      \prg_return_true: \else: \prg_return_false: \fi:
  }
\cs_generate_variant:Nn \str_if_eq_p:nn {  V ,  o }
\cs_generate_variant:Nn \str_if_eq_p:nn { nV , no , VV }
\cs_generate_variant:Nn \str_if_eq:nnT  {  V ,  o }
\cs_generate_variant:Nn \str_if_eq:nnT  { nV , no , VV }
\cs_generate_variant:Nn \str_if_eq:nnF  {  V ,  o }
\cs_generate_variant:Nn \str_if_eq:nnF  { nV , no , VV }
\cs_generate_variant:Nn \str_if_eq:nnTF {  V ,  o }
\cs_generate_variant:Nn \str_if_eq:nnTF { nV , no , VV }
\prg_new_conditional:Npnn \str_if_eq_x:nn #1#2 { p , T , F , TF }
  {
    \if_int_compare:w \__str_if_eq_x:nn {#1} {#2} = \c_zero
      \prg_return_true: \else: \prg_return_false: \fi:
  }
%    \end{macrocode}
% \end{macro}
%
% \begin{macro}[int, EXP]{\@@_if_eq_x_return:nn}
% \end{macro}
%
% \begin{macro}[EXP]
%   {\str_case:nn, \str_case:on, \str_case:nV, \str_case:nv, \str_case_x:nn}
% \begin{macro}[EXP, TF]
%   {\str_case:nn, \str_case:on, \str_case:nV, \str_case:nv, \str_case_x:nn}
% \begin{macro}[EXP, aux]{\@@_case:nnTF, \@@_case_x:nnTF}
% \begin{macro}[aux, EXP]
%   {\@@_case:nw, \@@_case_x:nw, \@@_case_end:nw}
%   Much the same as \cs{tl_case:nn(TF)} here: just a change in the internal
%   comparison.
%    \begin{macrocode}
\cs_new:Npn \str_case:nn #1#2
  {
    \tex_romannumeral:D
    \@@_case:nnTF {#1} {#2} { } { }
  }
\cs_new:Npn \str_case:nnT #1#2#3
  {
    \tex_romannumeral:D
    \@@_case:nnTF {#1} {#2} {#3} { }
  }
\cs_new:Npn \str_case:nnF #1#2
  {
    \tex_romannumeral:D
    \@@_case:nnTF {#1} {#2} { }
  }
\cs_new:Npn \str_case:nnTF #1#2
  {
    \tex_romannumeral:D
    \@@_case:nnTF {#1} {#2}
  }
\cs_new:Npn \@@_case:nnTF #1#2#3#4
  { \@@_case:nw {#1} #2 {#1} { } \q_mark {#3} \q_mark {#4} \q_stop }
\cs_generate_variant:Nn \str_case:nn   { o , nV , nv }
\cs_generate_variant:Nn \str_case:nnT  { o , nV , nv }
\cs_generate_variant:Nn \str_case:nnF  { o , nV , nv }
\cs_generate_variant:Nn \str_case:nnTF { o , nV , nv }
\cs_new:Npn \@@_case:nw #1#2#3
  {
    \str_if_eq:nnTF {#1} {#2}
      { \@@_case_end:nw {#3} }
      { \@@_case:nw {#1} }
  }
\cs_new:Npn \str_case_x:nn #1#2
  {
    \tex_romannumeral:D
    \@@_case_x:nnTF {#1} {#2} { } { }
  }
\cs_new:Npn \str_case_x:nnT #1#2#3
  {
    \tex_romannumeral:D
    \@@_case_x:nnTF {#1} {#2} {#3} { }
  }
\cs_new:Npn \str_case_x:nnF #1#2
  {
    \tex_romannumeral:D
    \@@_case_x:nnTF {#1} {#2} { }
  }
\cs_new:Npn \str_case_x:nnTF #1#2
  {
    \tex_romannumeral:D
    \@@_case_x:nnTF {#1} {#2}
  }
\cs_new:Npn \@@_case_x:nnTF #1#2#3#4
  { \@@_case_x:nw {#1} #2 {#1} { } \q_mark {#3} \q_mark {#4} \q_stop }
\cs_new:Npn \@@_case_x:nw #1#2#3
  {
    \str_if_eq_x:nnTF {#1} {#2}
      { \@@_case_end:nw {#3} }
      { \@@_case_x:nw {#1} }
  }
\cs_new_eq:NN \@@_case_end:nw \__prg_case_end:nw
%    \end{macrocode}
% \end{macro}
% \end{macro}
% \end{macro}
% \end{macro}
%
% \subsection{String manipulation}
%
% \begin{macro}[EXP]
%   {
%     \str_fold_case:n,
%     \str_lower_case:n, \str_lower_case:f,
%     \str_upper_case:n, \str_upper_case:f
%   }
% \begin{macro}[aux, EXP]{\__str_change_case:nn}
% \begin{macro}[aux, EXP]{\__str_change_case_aux:nn}
% \begin{macro}[aux, EXP]{\__str_change_case_loop:nw}
% \begin{macro}[aux, EXP]{\__str_change_case_space:n}
% \begin{macro}[aux, EXP]{\__str_change_case_char:nN}
% \begin{macro}[aux, EXP]{\__str_change_case_char:NNNNNNNNn}
%   Case changing for programmatic reasons is done by first detokenizing
%   input then doing a simple loop that only has to worry about spaces
%   and everything else. The output is detokenized to allow data sharing
%   with text-based case changing. The key idea here of splitting up the
%   data files based on the character code of the current character is
%   shared with the text case changer too.
%    \begin{macrocode}
\cs_new:Npn \str_fold_case:n #1 { \__str_change_case:nn {#1} { fold } }
\cs_new:Npn \str_lower_case:n #1 { \__str_change_case:nn {#1} { lower } }
\cs_new:Npn \str_upper_case:n #1 { \__str_change_case:nn {#1} { upper } }
\cs_generate_variant:Nn \str_lower_case:n { f }
\cs_generate_variant:Nn \str_upper_case:n { f }
\cs_new:Npn \__str_change_case:nn #1
  {
    \exp_after:wN \__str_change_case_aux:nn \exp_after:wN 
      { \tl_to_str:n {#1} }
  }
\cs_new:Npn \__str_change_case_aux:nn #1#2
  {
    \__str_change_case_loop:nw {#2} #1 \q_recursion_tail \q_recursion_stop
  }
\cs_new:Npn \__str_change_case_loop:nw #1#2 \q_recursion_stop
  {
    \tl_if_head_is_space:nTF {#2}
      { \__str_change_case_space:n }
      { \__str_change_case_char:nN }
    {#1} #2 \q_recursion_stop
  }
\use:x
  { \cs_new:Npn \exp_not:N \__str_change_case_space:n ##1 \c_space_tl }
  {
    \c_space_tl
    \__str_change_case_loop:nw {#1}
  }
\cs_new:Npn \__str_change_case_char:nN #1#2
  {
    \quark_if_recursion_tail_stop:N #2
    \exp_args:Nf \tl_to_str:n
      {
        \exp_after:wN \__str_change_case_char:NNNNNNNNn
          \int_use:N \__int_eval:w 1000000 + `#2 \__int_eval_end: #2 {#1}
      }
    \__str_change_case_loop:nw {#1}
  }
\cs_new:Npn \__str_change_case_char:NNNNNNNNn #1#2#3#4#5#6#7#8#9
  {
    \str_case:nvF #8
      { c__unicode_ #9 _ #6 _X_ #7 _tl }
      { #8 }
  }
%    \end{macrocode}
% \end{macro}
% \end{macro}
% \end{macro}
% \end{macro}
% \end{macro}
% \end{macro}
% \end{macro}
%
%    \begin{macrocode}
%</initex|package>
%    \end{macrocode}
%
% \end{implementation}
%
% \PrintIndex