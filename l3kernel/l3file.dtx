% \iffalse meta-comment
%
%% File: l3file.dtx Copyright (C) 1990-2012 The LaTeX3 Project
%%
%% It may be distributed and/or modified under the conditions of the
%% LaTeX Project Public License (LPPL), either version 1.3c of this
%% license or (at your option) any later version.  The latest version
%% of this license is in the file
%%
%%    http://www.latex-project.org/lppl.txt
%%
%% This file is part of the "l3kernel bundle" (The Work in LPPL)
%% and all files in that bundle must be distributed together.
%%
%% The released version of this bundle is available from CTAN.
%%
%% -----------------------------------------------------------------------
%%
%% The development version of the bundle can be found at
%%
%%    http://www.latex-project.org/svnroot/experimental/trunk/
%%
%% for those people who are interested.
%%
%%%%%%%%%%%
%% NOTE: %%
%%%%%%%%%%%
%%
%%   Snapshots taken from the repository represent work in progress and may
%%   not work or may contain conflicting material!  We therefore ask
%%   people _not_ to put them into distributions, archives, etc. without
%%   prior consultation with the LaTeX3 Project.
%%
%% -----------------------------------------------------------------------
%
%<*driver|package>
\RequirePackage{l3bootstrap}
\GetIdInfo$Id$
  {L3 File and I/O operations}
%</driver|package>
%<*driver>
\documentclass[full]{l3doc}
\begin{document}
  \DocInput{\jobname.dtx}
\end{document}
%</driver>
% \fi
%
% \title{^^A
%   The \pkg{l3file} package\\ File and I/O operations^^A
%   \thanks{This file describes v\ExplFileVersion,
%      last revised \ExplFileDate.}^^A
% }
%
% \author{^^A
%  The \LaTeX3 Project\thanks
%    {^^A
%      E-mail:
%        \href{mailto:latex-team@latex-project.org}
%          {latex-team@latex-project.org}^^A
%    }^^A
% }
%
% \date{Released \ExplFileDate}
%
% \maketitle
%
% \begin{documentation}
%
% This module provides functions for working with external files. Some of these
% functions apply to an entire file, and have prefix \cs{file_\ldots}, while
% others are used to work with files on a line by line basis and have prefix
% \cs{ior_\ldots} (reading) or \cs{iow_\ldots} (writing).
%
% It is important to remember that when reading external files \TeX{} will
% attempt to locate them both the operating system path and entries in the
% \TeX{} file database (most \TeX{} systems use such a database). Thus the
% \enquote{current path} for \TeX{} is somewhat broader than that for other
% programs.
%
% For functions which expect a \meta{file name} argument, this argument
% may contain both literal items and expandable content, which should on
% full expansion be the desired file name.  Any active characters (as
% declared in \cs{l_char_active_seq}) will \emph{not} be expanded,
% allowing the direct use of these in file names.
% Spaces are not allowed in file names.
%
% \section{File operation functions}
%
% \begin{variable}{\g_file_current_name_tl}
%   Contains the name of the current \LaTeX{} file. This variable
%   should not be modified: it is intended for information only. It
%   will be equal to \cs{c_job_name_tl} at the start of a \LaTeX{}
%   run and will be modified each time a file is read using
%   \cs{file_input:n}.
% \end{variable}
%
% \begin{function}[TF, updated = 2012-02-10]{\file_if_exist:n}
%   \begin{syntax}
%     \cs{file_if_exist:nTF} \Arg{file name} \Arg{true code} \Arg{false code}
%   \end{syntax}
%   Searches for \meta{file name} using the current \TeX{} search
%   path and the additional paths controlled by
%   \cs{file_path_include:n}).
% \end{function}
%
% \begin{function}[updated = 2012-02-10]{\file_add_path:nN}
%   \begin{syntax}
%     \cs{file_add_path:nN} \Arg{file name} \meta{tl var}
%   \end{syntax}
%   Searches for \meta{file name} in the path as detailed for
%   \cs{file_if_exist:nTF}, and if found sets the \meta{tl var} the
%   fully-qualified name of the file, \emph{i.e.}~the path and file name.
%   If the file is not found then the \meta{tl var} will contain the
%   marker \cs{q_no_value}.
% \end{function}
%
% \begin{function}[updated = 2012-02-17]{\file_input:n}
%   \begin{syntax}
%     \cs{file_input:n} \Arg{file name}
%   \end{syntax}
%   Searches for \meta{file name} in the path as detailed for
%   \cs{file_if_exist:nTF}, and if found reads in the file as
%   additional \LaTeX{} source. All files read are recorded
%   for information and the file name stack is updated by this
%   function. An error will be raised if the file is not found.
% \end{function}
%
% \begin{function}[updated = 2012-07-04]{\file_path_include:n}
%   \begin{syntax}
%     \cs{file_path_include:n} \Arg{path}
%   \end{syntax}
%   Adds \meta{path} to the list of those used to search when reading
%   files. The assignment is local.
%   The \meta{path} is processed in the same way as a \meta{file name},
%   \emph{i.e.}, with \texttt{x}-type expansion except active
%   characters.  Spaces are not allowed in the \meta{path}.
% \end{function}
%
% \begin{function}[updated = 2012-07-04]{\file_path_remove:n}
%   \begin{syntax}
%     \cs{file_path_remove:n} \Arg{path}
%   \end{syntax}
%   Removes \meta{path} from the list of those used to search when reading
%   files. The assignment is local.
%   The \meta{path} is processed in the same way as a \meta{file name},
%   \emph{i.e.}, with \texttt{x}-type expansion except active
%   characters.  Spaces are not allowed in the \meta{path}.
% \end{function}
%
% \begin{function}{\file_list:}
%   \begin{syntax}
%     \cs{file_list:}
%   \end{syntax}
%   This function will list all files loaded using \cs{file_input:n}
%   in the log file.
% \end{function}
%
% \subsection{Input--output stream management}
%
% As \TeX{} is limited to $16$ input streams and $16$ output streams, direct
% use of the streams by the programmer is not supported in \LaTeX3. Instead, an
% internal pool of streams is maintained, and these are allocated and
% deallocated as needed by other modules. As a result, the programmer should
% close streams when they are no longer needed, to release them for other
% processes.
%
% Note that I/O operations are global: streams should all be declared
% with global names and treated accordingly.
%
% \begin{function}[added = 2011-09-26, updated = 2011-12-27]
%   {\ior_new:N, \ior_new:c, \iow_new:N, \iow_new:c}
%   \begin{syntax}
%     \cs{ior_new:N} \meta{stream}
%     \cs{iow_new:N} \meta{stream}
%   \end{syntax}
%   Globally reserves the name of the \meta{stream}, either for reading
%   or for writing as appropriate. The \meta{stream} is not opened until
%   the appropriate \cs{\ldots_open:Nn} function is used. Attempting to
%   use a \meta{stream} which has not been opened is an error, and the
%   \meta{stream} will behave as the corresponding \cs{c_term_\ldots}.
% \end{function}
%
% \begin{function}[updated = 2012-02-10]{\ior_open:Nn, \ior_open:cn}
%   \begin{syntax}
%     \cs{ior_open:Nn} \meta{stream} \Arg{file name}
%   \end{syntax}
%   Opens \meta{file name} for reading using \meta{stream} as the
%   control sequence for file access. If the \meta{stream} was already
%   open it is closed before the new operation begins. The
%   \meta{stream} is available for access immediately and will remain
%   allocated to \meta{file name} until a \cs{ior_close:N} instruction
%   is given or the \TeX{} run ends.
% \end{function}
%
% \begin{function}[updated = 2012-02-09]{\iow_open:Nn, \iow_open:cn}
%   \begin{syntax}
%     \cs{iow_open:Nn} \meta{stream} \Arg{file name}
%   \end{syntax}
%   Opens \meta{file name} for writing using \meta{stream} as the
%   control sequence for file access. If the \meta{stream} was already
%   open it is closed before the new operation begins. The
%   \meta{stream} is available for access immediately and will remain
%   allocated to \meta{file name} until a \cs{iow_close:N} instruction
%   is given or the \TeX{} run ends. Opening a file for writing will clear
%   any existing content in the file (\emph{i.e.}~writing is \emph{not}
%   additive).
% \end{function}
%
% \begin{function}[updated = 2012-07-31]
%   {\ior_close:N, \ior_close:c, \iow_close:N, \iow_close:c}
%   \begin{syntax}
%     \cs{ior_close:N} \meta{stream}
%     \cs{iow_close:N} \meta{stream}
%   \end{syntax}
%   Closes the \meta{stream}. Streams should always be closed when
%   they are finished with as this ensures that they remain available
%   to other programmers.
% \end{function}
%
% \begin{function}{\ior_list_streams:, \iow_list_streams:}
%   \begin{syntax}
%     \cs{ior_list_streams:}
%     \cs{iow_list_streams:}
%   \end{syntax}
%   Displays a list of the file names associated with each open
%   stream: intended for tracking down problems.
% \end{function}
%
% \subsection{Reading from files}
%
% \begin{function}[added = 2012-06-24]{\ior_get:NN}
%   \begin{syntax}
%     \cs{ior_get:NN} \meta{stream} \meta{token list variable}
%   \end{syntax}
%   Function that reads one or more lines (until an equal number of left
%   and right braces are found) from the input \meta{stream} and stores
%   the result locally in the \meta{token list} variable. If the
%   \meta{stream} is not open, input is requested from the terminal.
%   The material read from the \meta{stream} will be tokenized by
%   \TeX{} according to the category codes in force when the function
%   is used.
%   \begin{texnote}
%     This protected macro expands to the \TeX{} primitive \tn{read}
%     along with the |to| keyword.
%   \end{texnote}
% \end{function}
%
% \begin{function}[added = 2012-06-24, updated = 2012-07-31]{\ior_get_str:NN}
%   \begin{syntax}
%     \cs{ior_get_str:NN} \meta{stream} \meta{token list variable}
%   \end{syntax}
%   Function that reads one line from the input \meta{stream} and stores
%   the result locally in the \meta{token list} variable. If the
%   \meta{stream} is not open, input is requested from the terminal.
%   The material is read from the \meta{stream} as a series of tokens with
%   category code $12$ (other), with the exception of space
%   characters which are given category code $10$ (space).
%   \begin{texnote}
%     This protected macro is a wrapper around the \eTeX{} primitive
%     \tn{readline}.  However, the end-line character normally added by
%     this primitive is not included in the result of
%     \cs{ior_get_str:NN}.
%   \end{texnote}
% \end{function}
%
%\begin{function}[updated = 2012-02-10, EXP, pTF]{\ior_if_eof:N}
%  \begin{syntax}
%    \cs{ior_if_eof_p:N} \meta{stream} \\
%    \cs{ior_if_eof:NTF} \meta{stream} \Arg{true code} \Arg{false code}
%  \end{syntax}
%  Tests if the end of a \meta{stream} has been reached during a reading
%  operation. The test will also return a \texttt{true} value if
%  the \meta{stream} is not open.
%\end{function}
%
% \section{Writing to files}
%
% \begin{function}[updated = 2012-06-05]{\iow_now:Nn, \iow_now:Nx}
%   \begin{syntax}
%     \cs{iow_now:Nn} \meta{stream} \Arg{tokens}
%   \end{syntax}
%   This functions writes \meta{tokens} to the specified
%   \meta{stream} immediately (\emph{i.e.}~the write operation is called
%   on expansion of \cs{iow_now:Nn}).
% \end{function}
%
% \begin{function}{\iow_log:n, \iow_log:x}
%   \begin{syntax}
%     \cs{iow_log:n} \Arg{tokens}
%   \end{syntax}
%   This function writes the given \meta{tokens} to the log (transcript)
%   file immediately: it is a dedicated version of \cs{iow_now:Nn}.
% \end{function}
%
% \begin{function}{\iow_term:n, \iow_term:x}
%   \begin{syntax}
%     \cs{iow_term:n} \Arg{tokens}
%   \end{syntax}
%   This function writes the given \meta{tokens} to the terminal
%   file immediately: it is a dedicated version of \cs{iow_now:Nn}.
% \end{function}
%
% \begin{function}{\iow_shipout:Nn, \iow_shipout:Nx}
%   \begin{syntax}
%     \cs{iow_shipout:Nn} \meta{stream} \Arg{tokens}
%   \end{syntax}
%   This functions writes \meta{tokens} to the specified
%   \meta{stream} when the current page is finalised (\emph{i.e.}~at
%   shipout). The \texttt{x}-type variants expand the \meta{tokens}
%   at the point where the function is used but \emph{not} when the
%   resulting tokens are written to the \meta{stream}
%   (\emph{cf.}~\cs{iow_shipout_x:Nn}).
% \end{function}
%
% \begin{function}{\iow_shipout_x:Nn, \iow_shipout_x:Nx}
%   \begin{syntax}
%     \cs{iow_shipout_x:Nn} \meta{stream} \Arg{tokens}
%   \end{syntax}
%   This functions writes \meta{tokens} to the specified
%   \meta{stream} when the current page is finalised (\emph{i.e.}~at
%   shipout). The \meta{tokens} are expanded at the time of writing
%   in addition to any expansion when the function is used. This makes
%   these functions suitable for including material finalised during
%   the page building process (such as the page number integer).
%   \begin{texnote}
%     This is a wrapper around the \TeX{} primitive \tn{write}.
%   \end{texnote}
% \end{function}
%
% \begin{function}[EXP]{\iow_char:N}
%   \begin{syntax}
%     \cs{iow_char:N} |\|\meta{char}
%   \end{syntax}
%   Inserts \meta{char} into the output stream. Useful when trying to
%   write difficult characters such as |%|, |{|, |}|,
%   \emph{etc.}~in messages, for example:
%   \begin{verbatim}
%     \iow_now:Nx \g_my_iow { \iow_char:N \{ text \iow_char:N \} }
%   \end{verbatim}
%   The function has no effect if writing is taking place without
%   expansion (\emph{e.g.}~in the second argument of \cs{iow_now:Nn}).
% \end{function}
%
% \begin{function}[EXP]{\iow_newline:}
%   \begin{syntax}
%     \cs{iow_newline:}
%   \end{syntax}
%   Function to add a new line within the \meta{tokens} written to a
%   file. The function has no effect if writing is taking place without
%   expansion (\emph{e.g.}~in the second argument of \cs{iow_now:Nn}).
% \end{function}
%
% \subsection{Wrapping lines in output}
%
% \begin{function}[added = 2012-06-28]{\iow_wrap:nnnN}
%   \begin{syntax}
%     \cs{iow_wrap:nnnN} \Arg{text} \Arg{run-on text} \Arg{set up} \meta{function}
%   \end{syntax}
%   This function will wrap the \meta{text} to a fixed number of
%   characters per line. At the start of each line which is wrapped,
%   the \meta{run-on text} will be inserted.  The line character count
%   targeted will be the value of \cs{l_iow_line_count_int} minus the
%   number of characters in the \meta{run-on text}.  The \meta{text} and
%   \meta{run-on text} are exhaustively expanded by the function, with the
%   following substitutions:
%   \begin{itemize}
%     \item |\\| may be used to force a new line,
%     \item \verb*|\ | may be used to represent a forced space
%       (for example after a control sequence),
%     \item |\#|, |\%|, |\{|, |\}|, |\~| may be used to represent
%       the corresponding character,
%     \item \cs{iow_indent:n} may be used to indent a part of the
%       message.
%   \end{itemize}
%   Additional functions may be added to the wrapping by using the
%   \meta{set up}, which is executed before the wrapping takes place: this
%   may include overriding the substitutions listed.
%   
%   Any expandable material in the \meta{text} which is not to be expanded
%   on wrapping should be converted to a string using \cs{token_to_str:N},
%   \cs{tl_to_str:n}, \cs{tl_to_str:N}, \emph{etc.}
%   
%   The result of the wrapping operation is passed as a braced argument to the
%   \meta{function}, which will typically be a wrapper around a write
%   operation. The output of \cs{iow_wrap:nnnN} (\emph{i.e.}~the argument
%   passed to the \meta{function}) will consist of characters of category
%   \enquote{other} (category code~12), with the exception of spaces which
%   will have category \enquote{space} (category code~10). This means that the
%   output will \emph{not} expand further when written to a file.
%   
%   \begin{texnote}
%     Internally, \cs{iow_wrap:nnnN} carries out an \texttt{x}-type expansion
%     on the \meta{text} to expand it. This is done in such a way that
%     \cs{exp_not:N} or \cs{exp_not:n} \emph{could} be used to prevent
%     expansion of material. However, this is less conceptually clear than
%     conversion to a string, which is therefore the supported method for
%     handling expandable material in the \meta{text}.
%   \end{texnote}
% \end{function}
%
% \begin{function}[added = 2011-09-21]{\iow_indent:n}
%   \begin{syntax}
%     \cs{iow_indent:n} \Arg{text}
%   \end{syntax}
%   In the context of \cs{iow_wrap:nnnN} (for instance in messages),
%   indents \meta{text} by four spaces. This function will not cause
%   a line break, and only affects lines which start within the scope
%   of the \meta{text}. In case the indented \meta{text} should appear
%   on separate lines from the surrounding text, use |\\| to force
%   line breaks.
% \end{function}
%
% \begin{variable}[added = 2012-06-24]{\l_iow_line_count_int}
%   The maximum number of characters in a line to be written
%   by the \cs{iow_wrap:nnnN}
%   function. This value depends on the \TeX{} system in use: the standard
%   value is $78$, which is typically correct for unmodified \TeX{}live
%   and MiK\TeX{} systems.
% \end{variable}
%
% \begin{variable}[added = 2011-09-05]{\c_catcode_other_space_tl}
%   Token list containing one character with category code $12$,
%   (\enquote{other}), and character code $32$ (space).
% \end{variable}
%
% \subsection{Constant input--output streams}
%
% \begin{variable}{\c_term_ior}
%   Constant input stream for reading from the terminal. Reading from this
%   stream using \cs{ior_get:NN} or similar will result in a prompt from
%   \TeX{} of the form
%   \begin{verbatim}
%     <tl>=
%   \end{verbatim}
% \end{variable}
%
% \begin{variable}{\c_log_iow, \c_term_iow}
%   Constant output streams for writing to the log and to the terminal
%   (plus the log), respectively.
% \end{variable}
%
% \subsection{Primitive conditionals}
%
% \begin{function}[EXP]{\if_eof:w}
%   \begin{syntax}
%     \cs{if_eof:w} \meta{stream}
%     ~~\meta{true code}
%     \cs{else:}
%     ~~\meta{false code}
%     \cs{fi:}
%   \end{syntax}
%   Tests if the \meta{stream} returns \enquote{end of file}, which is true
%   for non-existent files. The \cs{else:} branch is optional.
%   \begin{texnote}
%     This is the \TeX{} primitive \tn{ifeof}.
%   \end{texnote}
% \end{function}
%
% \subsection{Internal file functions and variables}
%
% \begin{variable}{\l__file_internal_name_ior}
%   Used to test for the existence of files when opening.
% \end{variable}
%
% \begin{variable}{\l__file_internal_name_tl}
%   Used to return the full name of a file for internal use.
% \end{variable}
%
% \begin{function}[added = 2012-02-09]{\__file_name_sanitize:nn}
%   \begin{syntax}
%     \cs{__file_name_sanitize:nn} \Arg{name} \Arg{tokens}
%   \end{syntax}
%   Exhaustively-expands the \meta{name} with the exception of any
%   category \meta{active} (catcode~$13$) tokens, which are not expanded.
%   The list of \meta{active} tokens is taken from \cs{l_char_active_seq}.
%   The \meta{sanitized name} is then inserted (in braces) after the
%   \meta{tokens}, which should further process the file name. If any
%   spaces are found in the name after expansion, an error is raised.
% \end{function}
%
% \subsection{Internal input--output functions}
%
% \begin{function}[added = 2012-01-23]{\__ior_open:Nn, \__ior_open:No}
%   \begin{syntax}
%     \cs{__ior_open:Nn} \meta{stream} \Arg{file name}
%   \end{syntax}
%   This function has identical syntax to the public version. However,
%   is does not take precautions against active characters in the
%   \meta{file name}, and it does not attempt to add a \meta{path} to
%   the \meta{file name}: it is therefore intended to be used by
%   higher-level
%   functions which have already fully expanded the \meta{file name} and which
%   need to perform multiple open or close operations. See for example the
%   implementation of \cs{file_add_path:nN},
% \end{function}
%
% \end{documentation}
%
% \begin{implementation}
%
% \section{\pkg{l3file} implementation}
%
% \TestFiles{m3file001}
%
%    \begin{macrocode}
%<*initex|package>
%    \end{macrocode}
%
%    \begin{macrocode}
%<@@=file>
%    \end{macrocode}
%
%    \begin{macrocode}
%<*package>
\ProvidesExplPackage
  {\ExplFileName}{\ExplFileDate}{\ExplFileVersion}{\ExplFileDescription}
\__expl_package_check:
%</package>
%    \end{macrocode}
%
% \subsection{File operations}
%
% \begin{variable}{\g_file_current_name_tl}
%   The name of the current file should be available at all times.
%   For the format the file name needs to be picked up at the start of the
%   file. In package mode the current file name is collected from \LaTeXe{}.
%    \begin{macrocode}
\tl_new:N \g_file_current_name_tl
%<*initex>
\tex_everyjob:D \exp_after:wN
  {
    \tex_the:D \tex_everyjob:D
    \tl_gset:Nx \g_file_current_name_tl { \tex_jobname:D }
  }
%</initex>
%<*package>
\tl_gset_eq:NN \g_file_current_name_tl \@currname
%</package>
%    \end{macrocode}
% \end{variable}
%
% \begin{variable}{\g_@@_stack_seq}
%   The input list of files is stored as a sequence stack.
%    \begin{macrocode}
\seq_new:N \g_@@_stack_seq
%    \end{macrocode}
% \end{variable}
%
% \begin{variable}{\g_@@_record_seq}
%   The total list of files used is recorded separately from the current
%   file stack, as nothing is ever popped from this list.  The current
%   file name should be included in the file list!  In format mode, this
%   is done at the very start of the \TeX{} run.  In package mode we
%   will eventually copy the contents of \cs{@filelist}.
%    \begin{macrocode}
\seq_new:N \g_@@_record_seq
%<*initex>
\tex_everyjob:D \exp_after:wN
  {
    \tex_the:D \tex_everyjob:D
    \seq_gput_right:NV \g_@@_record_seq \g_file_current_name_tl
  }
%</initex>
%    \end{macrocode}
% \end{variable}
%
% \begin{variable}{\l_@@_internal_tl}
%   Used as a short-term scratch variable.  It may be possible to reuse
%   \cs{l_@@_internal_name_tl} there.
%    \begin{macrocode}
\tl_new:N \l_@@_internal_tl
%    \end{macrocode}
% \end{variable}
%
% \begin{variable}{\l_@@_internal_name_tl}
%   Used to return the fully-qualified name of a file.
%    \begin{macrocode}
\tl_new:N \l_@@_internal_name_tl
%    \end{macrocode}
% \end{variable}
%
% \begin{variable}{\l_@@_search_path_seq}
%   The current search path.
%    \begin{macrocode}
\seq_new:N \l_@@_search_path_seq
%    \end{macrocode}
% \end{variable}
%
% \begin{variable}{\l_@@_saved_search_path_seq}
%   The current search path has to be saved for package use.
%    \begin{macrocode}
%<*package>
\seq_new:N \l_@@_saved_search_path_seq
%</package>
%    \end{macrocode}
% \end{variable}
%
% \begin{variable}{\l_@@_internal_seq}
%   Scratch space for comma list conversion in package mode.
%    \begin{macrocode}
%<*package>
\seq_new:N \l_@@_internal_seq
%</package>
%    \end{macrocode}
% \end{variable}
%
% \begin{macro}[aux]{\@@_name_sanitize:nn}
%   For converting a token list to a string where active characters are treated
%   as strings from the start.
%    \begin{macrocode}
\cs_new_protected:Npn \@@_name_sanitize:nn #1#2
  {
    \group_begin:
      \seq_map_inline:Nn \l_char_active_seq
        { \cs_set_nopar:Npx ##1 { \token_to_str:N ##1 } }
      \tl_set:Nx \l_@@_internal_name_tl {#1}
      \tl_set:Nx \l_@@_internal_name_tl
        { \tl_to_str:N \l_@@_internal_name_tl }
      \tl_if_in:NnT \l_@@_internal_name_tl { ~ }
        {
          \__msg_kernel_error:nnx { kernel } { space-in-file-name }
            { \l_@@_internal_name_tl }
          \tl_remove_all:Nn \l_@@_internal_name_tl { ~ }
        }
      \use:x
        {
          \group_end:
          \exp_not:n {#2} { \l_@@_internal_name_tl }
        }
  }
%    \end{macrocode}
% \end{macro}
%
% \begin{macro}{\file_add_path:nN}
% \begin{macro}[aux]{\@@_add_path:nN, \@@_add_path_search:nN}
%   The way to test if a file exists is to try to open it: if it does not
%   exist then \TeX{} will report end-of-file. For files which are in the
%   current directory, this is straight-forward.  For other locations, a
%   search has to be made looking at each potential path in turn. The first
%   location is of course treated as the correct one. If nothing is found,
%   |#2| is returned empty.
%    \begin{macrocode}
\cs_new_protected:Npn \file_add_path:nN #1
  { \@@_name_sanitize:nn {#1} { \@@_add_path:nN } }
\cs_new_protected:Npn \@@_add_path:nN #1#2
  {
    \__ior_open:Nn \g__file_internal_ior {#1}
    \ior_if_eof:NTF \g__file_internal_ior
      { \@@_add_path_search:nN {#1} #2 }
      { \tl_set:Nn #2 {#1} }
    \ior_close:N \g__file_internal_ior
  }
\cs_new_protected:Npn \@@_add_path_search:nN #1#2
  {
    \tl_set:Nn #2 { \q_no_value  }
%<*package>
    \cs_if_exist:NT \input@path
      {
        \seq_set_eq:NN \l_@@_saved_search_path_seq \l_@@_search_path_seq
        \seq_set_split:NnV \l_@@_internal_seq { , } \input@path
        \seq_concat:NNN \l_@@_search_path_seq
          \l_@@_search_path_seq \l_@@_internal_seq
      }
%</package>
    \seq_map_inline:Nn \l_@@_search_path_seq
      {
        \__ior_open:Nn \g__file_internal_ior { ##1 #1 }
        \ior_if_eof:NF \g__file_internal_ior
          {
            \tl_set:Nx #2 { ##1 #1 }
            \seq_map_break:
          }
      }
%<*package>
    \cs_if_exist:NT \input@path
      { \seq_set_eq:NN \l_@@_search_path_seq \l_@@_saved_search_path_seq }
%</package>
  }
%    \end{macrocode}
% \end{macro}
% \end{macro}
%
% \begin{macro}[TF]{\file_if_exist:n}
%   The test for the existence of a file is a wrapper around the function to
%   add a path to a file. If the file was found, the path will contain
%   something, whereas if the file was not located then the return value
%   will be \cs{q_no_value}.
%    \begin{macrocode}
\prg_new_protected_conditional:Npnn \file_if_exist:n #1 { T , F , TF }
  {
    \file_add_path:nN {#1} \l_@@_internal_name_tl
    \quark_if_no_value:NTF \l_@@_internal_name_tl
      { \prg_return_false: }
      { \prg_return_true: }
  }
%    \end{macrocode}
% \end{macro}
%
% \begin{macro}{\file_input:n}
% \begin{macro}[aux]{\@@_input:n \@@_input:V}
% \begin{macro}[aux, EXP]{\@@_input_aux:w}
% \begin{macro}[aux]{\@@_input_aux:n, \@@_input_aux:o}
%   Loading a file is done in a safe way, checking first that the file
%   exists and loading only if it does.  Push the file name on the
%   \cs{g_@@_stack_seq}, and add it to the file list, either
%   \cs{g_@@_record_seq}, or \cs{@filelist} in package mode.
%    \begin{macrocode}
\cs_new_protected:Npn \file_input:n #1
  {
    \file_add_path:nN {#1} \l_@@_internal_name_tl
    \quark_if_no_value:NTF \l_@@_internal_name_tl
      {
        \@@_name_sanitize:nn {#1}
          { \__msg_kernel_error:nnx { kernel } { file-not-found } }
      }
      { \@@_input:V \l_@@_internal_name_tl }
  }
\cs_new_protected:Npn \@@_input:n #1
  {
    \quark_if_nil:oTF { \@@_input_aux:w #1 . \q_nil . \q_stop }
      { \@@_input_aux:o { \tl_to_str:n { #1 .tex } } }
      { \@@_input_aux:n {#1} }
  }
\cs_generate_variant:Nn \@@_input:n { V }
\cs_new:Npn \@@_input_aux:w #1 . #2 . #3 \q_stop { #2 }
\cs_new_protected:Npn \@@_input_aux:n #1
  {
%<*initex>
    \seq_gput_right:Nn \g_@@_record_seq {#1}
%</initex>
%<*package>
    \clist_if_exist:NTF \@filelist
      { \@addtofilelist {#1} }
      { \seq_gput_right:Nn \g_@@_record_seq {#1} }
%</package>
    \seq_gpush:No \g_@@_stack_seq \g_file_current_name_tl
    \tl_gset:Nn \g_file_current_name_tl {#1}
    \tex_input:D #1 \c_space_tl
    \seq_gpop:NN \g_@@_stack_seq \l_@@_internal_tl
    \tl_gset_eq:NN \g_file_current_name_tl \l_@@_internal_tl
  }
\cs_generate_variant:Nn \@@_input_aux:n { o }
%    \end{macrocode}
% \end{macro}
% \end{macro}
% \end{macro}
% \end{macro}
%
% \begin{macro}{\file_path_include:n}
% \begin{macro}{\file_path_remove:n}
% \begin{macro}[aux]{\@@_path_include:n}
%   Wrapper functions to manage the search path.
%    \begin{macrocode}
\cs_new_protected:Npn \file_path_include:n #1
  { \@@_name_sanitize:nn {#1} { \@@_path_include:n } }
\cs_new_protected:Npn \@@_path_include:n #1
  {
    \seq_if_in:NnF \l_@@_search_path_seq {#1}
      { \seq_put_right:Nn \l_@@_search_path_seq {#1} }
  }
\cs_new_protected:Npn \file_path_remove:n #1
  {
    \@@_name_sanitize:nn {#1}
      { \seq_remove_all:Nn \l_@@_search_path_seq }
  }
%    \end{macrocode}
% \end{macro}
% \end{macro}
% \end{macro}
%
% \begin{macro}{\file_list:}
%   A function to list all files used to the log, without duplicates.
%   In package mode, if \cs{@filelist} is still defined, we need to take
%   it into account (we capture it \cs{AtBeginDocument} into
%   \cs{g_@@_record_seq}), turning each file name into a string.
%    \begin{macrocode}
\cs_new_protected_nopar:Npn \file_list:
  {
    \seq_set_eq:NN \l_@@_internal_seq \g_@@_record_seq
%<*package>
    \clist_if_exist:NT \@filelist
      {
        \clist_map_inline:Nn \@filelist
          {
            \seq_put_right:No \l_@@_internal_seq
              { \tl_to_str:n {##1} }
          }
      }
%</package>
    \seq_remove_duplicates:N \l_@@_internal_seq
    \iow_log:n { *~File~List~* }
    \seq_map_inline:Nn \l_@@_internal_seq { \iow_log:n {##1} }
    \iow_log:n { ************* }
  }
%    \end{macrocode}
% \end{macro}
%
% When used as a package, there is a need to hold onto the standard file
% list as well as the new one here.  File names recorded in
% \cs{@filelist} must be turned to strings before being added to
% \cs{g_@@_record_seq}.
%    \begin{macrocode}
%<*package>
\AtBeginDocument
  {
    \clist_map_inline:Nn \@filelist
      { \seq_gput_right:No \g_@@_record_seq { \tl_to_str:n {#1} } }
  }
%</package>
%    \end{macrocode}
%
% \subsection{Input operations}
%
%    \begin{macrocode}
%<@@=ior>
%    \end{macrocode}
%
% \subsubsection{Variables and constants}
%
% \begin{variable}{\c_term_ior}
%   Reading from the terminal (with a prompt) is done using a positive
%   but non-existent stream number. Unlike writing, there is no concept
%   of reading from the log.
%    \begin{macrocode}
\cs_new_eq:NN \c_term_ior \c_sixteen
%    \end{macrocode}
% \end{variable}
%
% \begin{variable}{\c_@@_streams_tl}
%   The list of streams available, by number.
%    \begin{macrocode}
\tl_const:Nn \c_@@_streams_tl
  {
    \c_zero
    \c_one
    \c_two
    \c_three
    \c_four
    \c_five
    \c_six
    \c_seven
    \c_eight
    \c_nine
    \c_ten
    \c_eleven
    \c_twelve
    \c_thirteen
    \c_fourteen
    \c_fifteen
  }
%    \end{macrocode}
% \end{variable}
%
% \begin{variable}{\g_@@_streams_prop}
%   The allocations for streams are stored in property lists, which
%   are set up to have a \enquote{full} set of allocations from the start.
%   In package mode, a few slots are always taken, so these are
%   blocked off from use.
%    \begin{macrocode}
\prop_new:N \g_@@_streams_prop
%<*package>
\prop_put:Nnn \g_@@_streams_prop { 0 } { LaTeX2e~reserved }
%</package>
%    \end{macrocode}
% \end{variable}
%
% \begin{variable}{\l_@@_stream_int}
%   Used to track the number allocated to the stream being created:
%   this is taken from the property list but does alter.
%    \begin{macrocode}
\int_new:N \l_@@_stream_int
%    \end{macrocode}
% \end{variable}
%
% \subsubsection{Stream management}
%
% \begin{macro}[int]{\@@_new:N, \@@_new:c}
%   The lowest level for stream management is actually creating raw \TeX{}
%   streams. As these are very limited (even with \eTeX{}), this should not
%   be addressed directly.
%    \begin{macrocode}
%<*initex>
\__alloc_setup_type:nnn { ior } \c_zero \c_sixteen
\cs_new_protected:Npn \@@_new:N #1
  { \__alloc_reg:nNN { ior } \tex_chardef:D #1 }
%</initex>
%<*package>
\cs_set_eq:NN \@@_new:N \newread
%</package>
\cs_generate_variant:Nn \@@_new:N { c }
%    \end{macrocode}
% \end{macro}
%
% \begin{macro}{\ior_new:N, \ior_new:c}
%   Reserving a new stream is done by defining the name as equal to using the
%   terminal.
%    \begin{macrocode}
\cs_new_protected:Npn \ior_new:N #1 { \cs_new_eq:NN #1 \c_term_ior }
\cs_generate_variant:Nn \ior_new:N { c }
%    \end{macrocode}
% \end{macro}
%
% \begin{variable}{\g__file_internal_ior}
%   Delayed from the \texttt{file} section so that the mechanisms are in place.
%    \begin{macrocode}
\ior_new:N \g__file_internal_ior
%    \end{macrocode}
% \end{variable}
%
% \begin{macro}{\ior_open:Nn, \ior_open:cn}
% \begin{macro}[TF]{\ior_open:Nn}
% \begin{macro}[aux]{\@@_open_aux:Nn}
% \begin{macro}[aux]{\@@_open_aux:NnTF}
% \begin{macro}[aux]{\@@_open:Nn, \@@_open:No}
%   Opening a stream starts with a call to the closing function: this is
%   safest. There is then a loop through the  allocation number list to find
%   the first free stream number. When one is found the allocation can take
%   place, the information can be stored and finally the file can actually be
%   opened. Before any actual file operations there is a precaution against
%   special characters in file names. There is an intermediate auxiliary to
%   allow path addition, keeping the internal function fast and avoiding an
%   infinite loop.
%    \begin{macrocode}
\cs_new_protected:Npn \ior_open:Nn #1#2
  { \__file_name_sanitize:nn {#2} { \@@_open_aux:Nn #1 } }
\cs_generate_variant:Nn \ior_open:Nn { c }
\cs_new_protected:Npn \@@_open_aux:Nn #1#2
  {
    \file_add_path:nN {#2} \l__file_internal_name_tl
    \quark_if_no_value:NTF \l__file_internal_name_tl
      { \__msg_kernel_error:nnx { kernel } { file-not-found } {#2} }
      { \@@_open:No #1 \l__file_internal_name_tl }
  }
\prg_new_protected_conditional:Npnn \ior_open:Nn #1#2 { T , F , TF }
  { \__file_name_sanitize:nn {#2} { \@@_open_aux:NnTF #1 } }
\cs_new_protected:Npn \@@_open_aux:NnTF #1#2
  {
    \file_add_path:nN {#2} \l__file_internal_name_tl
    \quark_if_no_value:NTF \l__file_internal_name_tl
      { \prg_return_false: }
      {
        \@@_open:No #1 \l__file_internal_name_tl
        \prg_return_true:
      }
  }
\cs_generate_variant:Nn \ior_open:NnT  { c }
\cs_generate_variant:Nn \ior_open:NnF  { c }
\cs_generate_variant:Nn \ior_open:NnTF { c }
\cs_new_protected:Npn \@@_open:Nn #1#2
  {
    \ior_close:N #1
    \int_set_eq:NN \l_@@_stream_int \c_sixteen
    \tl_map_function:NN \c_@@_streams_tl \@@_alloc:n
    \int_compare:nNnTF \l_@@_stream_int = \c_sixteen
      { \__msg_kernel_fatal:nn { kernel } { input-streams-exhausted } }
      {
        \@@_alloc:N #1
        \prop_gput:NVn \g_@@_streams_prop \l_@@_stream_int {#2}
        \tex_openin:D #1#2 \scan_stop:
      }
  }
\cs_generate_variant:Nn \@@_open:Nn { No }
%    \end{macrocode}
% \end{macro}
% \end{macro}
% \end{macro}
% \end{macro}
% \end{macro}
%
% \begin{macro}[aux]{\@@_alloc:n}
%   See if a particular stream is available: the property list contains file
%   names for streams in use, so any unused ones are for the taking.
%    \begin{macrocode}
\cs_new_protected:Npn \@@_alloc:n #1
  {
    \prop_if_in:NnF \g_@@_streams_prop {#1}
      {
        \int_set:Nn \l_@@_stream_int {#1}
        \tl_map_break:
      }
  }
%    \end{macrocode}
% \end{macro}
%
% \begin{macro}[aux]{\@@_alloc:N}
% \begin{macro}[aux]{\@@_alloc:}
% \begin{variable}{\g_@@_internal_ior}
%   Allocating a raw stream is much easier in \IniTeX{} mode than for the
%   package. For the format, all streams will be allocated by \pkg{l3file} and
%   so there is a simple check to see if a raw stream is actually available. On
%   the other hand, for the package there will be non-managed streams. So if
%   the managed one is not open, a check is made to see if some other managed
%   stream is available before deciding to open a new one. If a new one is
%   needed, we get the number allocated by \LaTeXe{} to get \enquote{back on
%   track} with allocation.
%    \begin{macrocode}
\ior_new:N \g_@@_internal_ior
\cs_new_protected:Npn \@@_alloc:N #1
  {
    \cs_if_exist:cF { g_ior_ \int_use:N \l_@@_stream_int _ior }
      {
%<*package>
        \@@_alloc:
        \int_compare:nNnT \l_@@_stream_int = \c_sixteen
          {
            \@@_new:N \g_@@_internal_ior
            \int_set:Nn \l_@@_stream_int { \g_@@_internal_ior }
            \cs_gset_eq:cN
              { g_ior_ \int_use:N \l_@@_stream_int _ior } \g_@@_internal_ior
          }
%</package>
%<*initex>
        \@@_new:c { g_ior_ \int_use:N \l_@@_stream_int _ior }
%</initex>
      }
    \cs_gset_eq:Nc #1 { g_ior_ \int_use:N \l_@@_stream_int _ior }
  }
%<*package>
\cs_new_protected_nopar:Npn \@@_alloc:
  {
    \int_incr:N \l_@@_stream_int
    \int_compare:nNnT \l_@@_stream_int < \c_sixteen
      {
        \cs_if_exist:cTF { g_ior_ \int_use:N \l_@@_stream_int _ior }
          {
            \prop_if_in:NVT \g_@@_streams_prop \l_@@_stream_int
              { \@@_alloc: }
          }
          { \@@_alloc: }
      }
  }
%</package>
%    \end{macrocode}
% \end{variable}
% \end{macro}
% \end{macro}
%
% \begin{macro}{\ior_close:N, \ior_close:c}
%   Closing a stream is not quite the reverse of opening one. First,
%   the close operation is easier than the open one, and second as the
%   stream is actually a number we can use it directly to show that the
%   slot has been freed up.
%    \begin{macrocode}
\cs_new_protected:Npn \ior_close:N #1
  {
    \__chk_if_exist_cs:N #1
    \int_compare:nNnF #1 < \c_zero
      {
        \int_compare:nNnF #1 > \c_fifteen
          {
            \tex_closein:D #1
            \prop_gremove:NV \g_@@_streams_prop #1
            \cs_gset_eq:NN #1 \c_term_ior
          }
      }
  }
\cs_generate_variant:Nn \ior_close:N { c }
%    \end{macrocode}
% \end{macro}
%
% \begin{macro}{\ior_list_streams:}
% \begin{macro}[aux]{\@@_list_streams:Nn}
%   Show the property lists, but with some \enquote{pretty printing}.
%   See the \pkg{l3msg} module. If there are no open read streams,
%   issue the message \texttt{show-no-stream}, and show an empty
%   token list. If there are open read streams, format them with
%   \cs{__msg_show_item_unbraced:nn}, and with the message
%   \texttt{show-open-streams}.
%    \begin{macrocode}
\cs_new_protected_nopar:Npn \ior_list_streams:
  { \@@_list_streams:Nn \g_@@_streams_prop { input } }
\cs_new_protected:Npn \@@_list_streams:Nn #1#2
  {
    \__msg_term:nnn { LaTeX / kernel }
      { \prop_if_empty:NTF #1 { show-no-stream } { show-open-streams } }
      {#2}
    \__msg_show_variable:x
      { \prop_map_function:NN #1 \__msg_show_item_unbraced:nn }
  }
%    \end{macrocode}
% \end{macro}
% \end{macro}
%
% \subsubsection{Reading input}
%
% \begin{macro}[int]{\if_eof:w}
%   The primitive conditional
%    \begin{macrocode}
\cs_new_eq:NN \if_eof:w \tex_ifeof:D
%    \end{macrocode}
% \end{macro}
%
% \begin{macro}[pTF]{\ior_if_eof:N}
%   To test if some particular input stream is exhausted the following
%   conditional is provided.
%    \begin{macrocode}
\prg_new_conditional:Nnn \ior_if_eof:N { p , T , F , TF }
  {
    \cs_if_exist:NTF #1
      {
        \if_int_compare:w #1 = \c_sixteen
          \prg_return_true:
        \else:
          \if_eof:w #1
            \prg_return_true:
          \else:
            \prg_return_false:
          \fi:
        \fi:
      }
      { \prg_return_true: }
  }
%    \end{macrocode}
% \end{macro}
%
% \begin{macro}{\ior_get:NN}
%   And here we read from files.
%    \begin{macrocode}
\cs_new_protected:Npn \ior_get:NN #1#2
  { \tex_read:D #1 to #2 }
%    \end{macrocode}
% \end{macro}
%
% \begin{macro}{\ior_get_str:NN}
%   Reading as strings is a more complicated wrapper, as we wish to
%   remove the endline character.
%    \begin{macrocode}
\cs_new_protected:Npn \ior_get_str:NN #1#2
  {
    \use:x
      {
        \int_set_eq:NN \tex_endlinechar:D \c_minus_one
        \exp_not:n { \etex_readline:D #1 to #2 }
        \int_set:Nn \tex_endlinechar:D { \int_use:N \tex_endlinechar:D }
      }
  }
%    \end{macrocode}
% \end{macro}
%
% \subsection{Output operations}
%
%    \begin{macrocode}
%<@@=iow>
%    \end{macrocode}
%
% There is a lot of similarity here to the input operations, at least for
% many of the basics. Thus quite a bit is copied from the earlier material.
%
% \subsubsection{Variables and constants}
%
% \begin{variable}{\c_log_iow, \c_term_iow}
%   Here we allocate two output streams for writing to the transcript
%   file only (\cs{c_log_iow}) and to both the terminal and
%   transcript file (\cs{c_term_iow}).
%    \begin{macrocode}
\cs_new_eq:NN \c_log_iow  \c_minus_one
\cs_new_eq:NN \c_term_iow \c_sixteen
%    \end{macrocode}
% \end{variable}
%
% \begin{variable}{\c_@@_streams_tl}
%   The list of streams available, by number: copied from the input
%   operations.
%    \begin{macrocode}
\cs_new_eq:NN \c_@@_streams_tl \c__ior_streams_tl
%    \end{macrocode}
% \end{variable}
%
% \begin{variable}{\g_@@_streams_prop}
%   As for reads, but with more reserved as \LaTeXe{} takes up a few here.
%    \begin{macrocode}
\prop_new:N \g_@@_streams_prop
%<*package>
\prop_put:Nnn \g_@@_streams_prop { 0 } { LaTeX2e~reserved }
\prop_put:Nnn \g_@@_streams_prop { 1 } { LaTeX2e~reserved }
\prop_put:Nnn \g_@@_streams_prop { 2 } { LaTeX2e~reserved }
%</package>
%    \end{macrocode}
% \end{variable}
%
% \begin{variable}{\l_@@_stream_int}
%   Used to track the number allocated to the stream being created:
%   this is taken from the property list but does alter. The two allocation
%   routines only need one register, so we save one here.
%    \begin{macrocode}
\cs_new_eq:NN \l_@@_stream_int \l__ior_stream_int
%    \end{macrocode}
% \end{variable}
%
% \subsection{Stream management}
%
% \begin{macro}[aux]{\@@_new:N, \@@_new:c}
%   The same idea as for input, but with a separate set of functions.
%    \begin{macrocode}
%<*initex>
\__alloc_setup_type:nnn { iow } \c_zero \c_sixteen
\cs_new_protected:Npn \@@_new:N #1
  { \__alloc_reg:nNN { iow } \tex_chardef:D #1 }
%</initex>
%<*package>
\cs_set_eq:NN \@@_new:N \newwrite
%</package>
\cs_generate_variant:Nn \@@_new:N { c }
%    \end{macrocode}
% \end{macro}
%
% \begin{macro}{\iow_new:N, \iow_new:c}
%   Reserving a new stream is done by defining the name as equal to writing
%   to the terminal: odd but at least consistent.
%    \begin{macrocode}
\cs_new_protected:Npn \iow_new:N #1 { \cs_new_eq:NN #1 \c_term_iow }
\cs_generate_variant:Nn \iow_new:N { c }
%    \end{macrocode}
% \end{macro}
%
% \begin{macro}{\iow_open:Nn, \iow_open:cn}
% \begin{macro}[aux]{\@@_open:Nn}
%   The same idea as for reading, but without the path.
%    \begin{macrocode}
\cs_new_protected:Npn \iow_open:Nn #1#2
  { \__file_name_sanitize:nn {#2} { \@@_open:Nn #1 } }
\cs_generate_variant:Nn \iow_open:Nn { c }
\cs_new_protected:Npn \@@_open:Nn #1#2
  {
    \iow_close:N #1
    \int_set_eq:NN \l_@@_stream_int \c_sixteen
    \tl_map_function:NN \c_@@_streams_tl \@@_alloc:n
    \int_compare:nNnTF \l_@@_stream_int = \c_sixteen
      { \__msg_kernel_fatal:nn { kernel } { output-streams-exhausted } }
      {
        \@@_alloc:N #1
        \prop_gput:NVn \g_@@_streams_prop \l_@@_stream_int {#2}
        \tex_immediate:D \tex_openout:D #1#2 \scan_stop:
      }
  }
%    \end{macrocode}
% \end{macro}
% \end{macro}
%
% \begin{macro}[aux]{\@@_alloc:n}
%   Identical to the input routine, but with the appropriate names used
%   (when \pkg{l3docstrip} has done it's job!).
%    \begin{macrocode}
\cs_new_protected:Npn \@@_alloc:n #1
  {
    \prop_if_in:NnF \g_@@_streams_prop {#1}
      {
        \int_set:Nn \l_@@_stream_int {#1}
        \tl_map_break:
      }
  }
%    \end{macrocode}
% \end{macro}
%
% \begin{macro}[aux]{\@@_alloc:N}
% \begin{macro}[aux]{\@@_alloc:}
% \begin{variable}{\g_@@_internal_iow}
%   Exactly the same as for reading!
%    \begin{macrocode}
\iow_new:N \g_@@_internal_iow
\cs_new_protected:Npn \@@_alloc:N #1
  {
    \cs_if_exist:cF { g_iow_ \int_use:N \l_@@_stream_int _iow }
      {
%<*package>
        \@@_alloc:
        \int_compare:nNnT \l_@@_stream_int = \c_sixteen
          {
            \@@_new:N \g_@@_internal_iow
            \int_set:Nn \l_@@_stream_int { \g_@@_internal_iow }
            \cs_gset_eq:cN
              { g_iow_ \int_use:N \l_@@_stream_int _iow } \g_@@_internal_iow
          }
%</package>
%<*initex>
        \@@_new:c { g_iow_ \int_use:N \l_@@_stream_int _iow }
%</initex>
      }
    \cs_gset_eq:Nc #1 { g_iow_ \int_use:N \l_@@_stream_int _iow }
  }
%<*package>
\cs_new_protected_nopar:Npn \@@_alloc:
  {
    \int_incr:N \l_@@_stream_int
    \int_compare:nNnT \l_@@_stream_int < \c_sixteen
      {
        \cs_if_exist:cTF { g_iow_ \int_use:N \l_@@_stream_int _iow }
          {
            \prop_if_in:NVT \g_@@_streams_prop \l_@@_stream_int
              { \@@_alloc: }
          }
          { \@@_alloc: }
      }
  }
%</package>
%    \end{macrocode}
% \end{variable}
% \end{macro}
% \end{macro}
%
% \begin{macro}{\iow_close:N, \iow_close:c}
%   Closing a stream is not quite the reverse of opening one. First,
%   the close operation is easier than the open one, and second as the
%   stream is actually a number we can use it directly to show that the
%   slot has been freed up.
%    \begin{macrocode}
\cs_new_protected:Npn \iow_close:N #1
  {
    \__chk_if_exist_cs:N #1
    \int_compare:nNnF #1 < \c_zero
      {
        \int_compare:nNnF #1 > \c_fifteen
          {
            \tex_immediate:D \tex_closeout:D #1
            \prop_gremove:NV \g_@@_streams_prop #1
            \cs_gset_eq:NN #1 \c_term_iow
          }
      }
  }
\cs_generate_variant:Nn \iow_close:N { c }
%    \end{macrocode}
% \end{macro}
%
% \begin{macro}{\iow_list_streams:}
% \begin{macro}{\@@_list_streams:Nn}
%   Done as for input, but with a copy of the auxiliary so the name is correct.
%    \begin{macrocode}
\cs_new_protected_nopar:Npn \iow_list_streams:
  { \@@_list_streams:Nn \g_@@_streams_prop { output } }
\cs_new_eq:NN \@@_list_streams:Nn \__ior_list_streams:Nn
%    \end{macrocode}
% \end{macro}
% \end{macro}
%
% \subsubsection{Deferred writing}
%
% \begin{macro}{\iow_shipout_x:Nn, \iow_shipout_x:Nx}
%   First the easy part, this is the primitive, which expects its
%   argument to be braced.
%    \begin{macrocode}
\cs_new_protected:Npn \iow_shipout_x:Nn #1#2
  { \tex_write:D #1 {#2} }
\cs_generate_variant:Nn \iow_shipout_x:Nn { Nx }
%    \end{macrocode}
% \end{macro}
%
% \begin{macro}{\iow_shipout:Nn, \iow_shipout:Nx}
%   With \eTeX{} available deferred writing without expansion is easy.
%    \begin{macrocode}
\cs_new_protected:Npn \iow_shipout:Nn #1#2
  { \iow_shipout_x:Nn #1 { \exp_not:n {#2} } }
\cs_generate_variant:Nn \iow_shipout:Nn { Nx }
%    \end{macrocode}
% \end{macro}
%
% \subsubsection{Immediate writing}
%
% \begin{macro}{\iow_now:Nn, \iow_now:Nx}
%   This routine writes the second argument onto the output stream without
%   expansion. If this stream isn't open, the output goes to the terminal
%   instead. If the first argument is no output stream at all, we get an
%   internal error.  We don't use the expansion done by \tn{write} to
%   get the |Nx| variant, because it differs in subtle ways from
%   \texttt{x}-expansion, namely, macro parameter characters would not
%   need to be doubled.
%    \begin{macrocode}
\cs_new_protected:Npn \iow_now:Nn #1#2
  { \tex_immediate:D \iow_shipout_x:Nn #1 { \exp_not:n {#2} } }
\cs_generate_variant:Nn \iow_now:Nn { Nx }
%    \end{macrocode}
% \end{macro}
%
% \begin{macro}{\iow_log:n, \iow_log:x}
% \begin{macro}{\iow_term:n, \iow_term:x}
%   Writing to the log and the terminal directly are relatively easy.
%    \begin{macrocode}
\cs_set_protected_nopar:Npn \iow_log:x  { \iow_now:Nx \c_log_iow  }
\cs_new_protected_nopar:Npn \iow_log:n  { \iow_now:Nn \c_log_iow  }
\cs_set_protected_nopar:Npn \iow_term:x { \iow_now:Nx \c_term_iow }
\cs_new_protected_nopar:Npn \iow_term:n { \iow_now:Nn \c_term_iow }
%    \end{macrocode}
%\end{macro}
%\end{macro}
%
% \subsubsection{Special characters for writing}
%
% \begin{macro}{\iow_newline:}
%   Global variable holding the character that forces a new line when
%   something is written to an output stream
%    \begin{macrocode}
\cs_new_nopar:Npn \iow_newline: { ^^J }
%    \end{macrocode}
% \end{macro}
%
% \begin{macro}{\iow_char:N}
%   Function to write any escaped char to an output stream.
%    \begin{macrocode}
\cs_new_eq:NN \iow_char:N \cs_to_str:N
%    \end{macrocode}
% \end{macro}
%
% \subsubsection{Hard-wrapping lines to a character count}
%
% The code here implements a generic hard-wrapping function. This is
% used by the messaging system, but is designed such that it is
% available for other uses.
%
% \begin{macro}[var, added = 2012-06-24]{\l_iow_line_count_int}
%   This is the \enquote{raw} number of characters in a line which
%   can be written to the terminal.
%   The standard value is the line length typically used by
%   \TeX{}Live and Mik\TeX{}.
%    \begin{macrocode}
\int_new:N  \l_iow_line_count_int
\int_set:Nn \l_iow_line_count_int { 78 }
%    \end{macrocode}
% \end{macro}
%
% \begin{macro}[aux]{\l_@@_target_count_int}
%   This stores the target line count: the full number of characters
%   in a line, minus any part for a leader at the start of each line.
%    \begin{macrocode}
\int_new:N \l_@@_target_count_int
%    \end{macrocode}
% \end{macro}
%
% \begin{macro}[aux]
%   {
%     \l_@@_current_line_int,
%     \l_@@_current_word_int,
%     \l_@@_current_indentation_int
%   }
%   These store the number of characters in the line and word currently
%   being constructed, and the current indentation, respectively.
%    \begin{macrocode}
\int_new:N \l_@@_current_line_int
\int_new:N \l_@@_current_word_int
\int_new:N \l_@@_current_indentation_int
%    \end{macrocode}
% \end{macro}
%
% \begin{macro}[aux]
%   {
%     \l_@@_current_line_tl,
%     \l_@@_current_word_tl,
%     \l_@@_current_indentation_tl
%   }
%   These hold the current line of text and current word,
%   and a number of spaces for indentation, respectively.
%    \begin{macrocode}
\tl_new:N \l_@@_current_line_tl
\tl_new:N \l_@@_current_word_tl
\tl_new:N \l_@@_current_indentation_tl
%    \end{macrocode}
%\end{macro}
%
% \begin{macro}[aux]{\l_@@_wrap_tl}
%   Used for the expansion step before detokenizing, and for the output
%   from wrapping text: fully expanded and with lines which are not
%   overly long.
%    \begin{macrocode}
\tl_new:N \l_@@_wrap_tl
%    \end{macrocode}
% \end{macro}
%
% \begin{macro}[aux]{\l_@@_newline_tl}
%   The token list inserted to produce the new line, with the
%   \meta{run-on text}.
%    \begin{macrocode}
\tl_new:N \l_@@_newline_tl
%    \end{macrocode}
% \end{macro}
%
% \begin{macro}[aux]{\l_@@_line_start_bool}
%   Boolean to avoid adding a space at the beginning of forced newlines,
%   and to know when to add the indentation.
%    \begin{macrocode}
\bool_new:N \l_@@_line_start_bool
%    \end{macrocode}
% \end{macro}
%
% \begin{macro}{\c_catcode_other_space_tl}
%   Lowercase a character with category code $12$ to produce an
%   \enquote{other} space. We can do everything within the group,
%   because \cs{tl_const:Nn} defines its argument globally.
%    \begin{macrocode}
\group_begin:
  \char_set_catcode_other:N \*
  \char_set_lccode:nn {`\*} {`\ }
  \tl_to_lowercase:n { \tl_const:Nn \c_catcode_other_space_tl { * } }
\group_end:
%    \end{macrocode}
% \end{macro}
%
% \begin{macro}[aux]{\c_@@_wrap_marker_tl}
% \begin{macro}[aux]
%   {
%     \c_@@_wrap_end_marker_tl,
%     \c_@@_wrap_newline_marker_tl,
%     \c_@@_wrap_indent_marker_tl,
%     \c_@@_wrap_unindent_marker_tl
%   }
%   Every special action of the wrapping code is preceeded by
%   the same recognizable string, \cs{c_@@_wrap_marker_tl}.
%   Upon seeing that \enquote{word}, the wrapping code reads
%   one space-delimited argument to know what operation to
%   perform. The setting of \tn{escapechar} here is not
%   very important, but makes \cs{c_@@_wrap_marker_tl} look
%   nicer.
%    \begin{macrocode}
\group_begin:
  \int_set_eq:NN \tex_escapechar:D \c_minus_one
  \tl_const:Nx \c_@@_wrap_marker_tl
    { \tl_to_str:n { \^^I \^^O \^^W \^^_ \^^W \^^R \^^A \^^P } }
\group_end:
\tl_map_inline:nn
  { { end } { newline } { indent } { unindent } }
  {
    \tl_const:cx { c_@@_wrap_ #1 _marker_tl }
      {
        \c_catcode_other_space_tl
        \c_@@_wrap_marker_tl
        \c_catcode_other_space_tl
        #1
        \c_catcode_other_space_tl
      }
  }
%    \end{macrocode}
% \end{macro}
% \end{macro}
%
% \begin{macro}{\iow_indent:n}
% \begin{macro}[aux]{\@@_indent:n}
%   We give a dummy (protected) definition to \cs{iow_indent:n}
%   when outside messages. Within wrapped message, it places
%   the instruction for increasing the indentation before its
%   argument, and the instruction for unindenting afterwards.
%   Note that there will be no forced line-break, so the indentation
%   only changes when the next line is started.
%    \begin{macrocode}
\cs_new_protected:Npn \iow_indent:n #1 { }
\cs_new:Npx \@@_indent:n #1
  {
    \c_@@_wrap_indent_marker_tl
    #1
    \c_@@_wrap_unindent_marker_tl
  }
%    \end{macrocode}
% \end{macro}
% \end{macro}
%
% \begin{macro}{\iow_wrap:nnnN}
%   The main wrapping function works as follows.  First give |\\|,
%   \verb*|\ | and other formatting commands the correct definition for
%   messages, before fully-expanding the input.  In package mode, the
%   expansion uses \LaTeXe{}'s \cs{protect} mechanism.  Afterwards, set
%   the newline marker (two assignments to fully expand, then convert
%   to a string) and its length, and initialize some registers.
%   There is then a loop over each word in the input, which will do the
%   actual wrapping.  After the loop, the resulting text is passed on to
%   the function which has been given as a post-processor.
%   The argument |#4| is available for additional set up steps for
%   the output. The definition of |\\| and \verb*|\ | use an
%   \enquote{other} space rather than a normal space, because the latter
%   might be absorbed by \TeX{} to end a number or other \texttt{f}-type
%   expansions. The \cs{tl_to_str:N} step converts the \enquote{other}
%   space back to a normal space.
%    \begin{macrocode}
\cs_new_protected:Npn \iow_wrap:nnnN #1#2#3#4
  {
    \group_begin:
      \int_set_eq:NN \tex_escapechar:D \c_minus_one
      \cs_set_nopar:Npx \{ { \token_to_str:N \{ }
      \cs_set_nopar:Npx \# { \token_to_str:N \# }
      \cs_set_nopar:Npx \} { \token_to_str:N \} }
      \cs_set_nopar:Npx \% { \token_to_str:N \% }
      \cs_set_nopar:Npx \~ { \token_to_str:N \~ }
      \int_set:Nn \tex_escapechar:D { 92 }
      \cs_set_eq:NN \\ \c_@@_wrap_newline_marker_tl
      \cs_set_eq:NN \  \c_catcode_other_space_tl
      \cs_set_eq:NN \iow_indent:n \@@_indent:n
      #3
%<*initex>
      \tl_set:Nx \l_@@_wrap_tl {#1}
%</initex>
%<*package>
      \protected@edef \l_@@_wrap_tl {#1}
%</package>
%    \end{macrocode}
%   This is a bit of a hack to measure the string length of the run on text
%   without the \pkg{l3str} module (which is still experimental). This should
%   be replaced once the string module is finalised with something a little
%   cleaner.
%    \begin{macrocode}
      \tl_set:Nx \l_@@_newline_tl { \iow_newline: #2 }
      \tl_set:Nx \l_@@_newline_tl { \tl_to_str:N \l_@@_newline_tl }
      \tl_replace_all:Nnn \l_@@_newline_tl { ~ } { \c_space_tl }
      \int_set:Nn \l_@@_target_count_int
        { \l_iow_line_count_int - \tl_count:N \l_@@_newline_tl + \c_one }
      \int_zero:N \l_@@_current_indentation_int
      \tl_clear:N \l_@@_current_indentation_tl
      \int_zero:N \l_@@_current_line_int
      \tl_clear:N \l_@@_current_line_tl
      \bool_set_true:N \l_@@_line_start_bool
      \use:x
        {
          \exp_not:n { \tl_clear:N \l_@@_wrap_tl }
          \@@_wrap_loop:w
          \tl_to_str:N \l_@@_wrap_tl
          \tl_to_str:N \c_@@_wrap_end_marker_tl
          \c_space_tl \c_space_tl
          \exp_not:N \q_stop
        }
    \exp_args:NNo \group_end:
    #4 \l_@@_wrap_tl
  }
%    \end{macrocode}
% \end{macro}
%
% \begin{macro}[aux]{\@@_wrap_loop:w}
%   The loop grabs one word in the input, and checks whether it is
%   the special marker, or a normal word.
%    \begin{macrocode}
\cs_new_protected:Npn \@@_wrap_loop:w #1 ~ %
  {
    \tl_set:Nn \l_@@_current_word_tl {#1}
    \tl_if_eq:NNTF \l_@@_current_word_tl \c_@@_wrap_marker_tl
      { \@@_wrap_special:w }
      { \@@_wrap_word: }
  }
%    \end{macrocode}
% \end{macro}
%
% \begin{macro}[aux]{\@@_wrap_word:}
% \begin{macro}[aux]{\@@_wrap_word_fits:}
% \begin{macro}[aux]{\@@_wrap_word_newline:}
%   For a normal word, update the line count, then test if the current
%   word would fit in the current line, and call the appropriate function.
%   If the word fits in the current line, add it to the line, preceded by
%   a space unless it is the first word of the line.
%   Otherwise, the current line is added to the result, with the run-on text.
%   The current word (and its character count) are then put in the new line.
%    \begin{macrocode}
\cs_new_protected_nopar:Npn \@@_wrap_word:
  {
    \int_set:Nn \l_@@_current_word_int
      { \__str_count_ignore_spaces:N \l_@@_current_word_tl }
    \int_add:Nn \l_@@_current_line_int { \l_@@_current_word_int }
    \int_compare:nNnTF \l_@@_current_line_int < \l_@@_target_count_int
      { \@@_wrap_word_fits: }
      { \@@_wrap_word_newline: }
    \@@_wrap_loop:w
  }
\cs_new_protected_nopar:Npn \@@_wrap_word_fits:
  {
    \bool_if:NTF \l_@@_line_start_bool
      {
        \bool_set_false:N \l_@@_line_start_bool
        \tl_put_right:Nx \l_@@_current_line_tl
          { \l_@@_current_indentation_tl \l_@@_current_word_tl }
        \int_add:Nn \l_@@_current_line_int
          { \l_@@_current_indentation_int }
      }
      {
        \tl_put_right:Nx \l_@@_current_line_tl
          { ~ \l_@@_current_word_tl }
        \int_incr:N \l_@@_current_line_int
      }
  }
\cs_new_protected_nopar:Npn \@@_wrap_word_newline:
  {
    \tl_put_right:Nx \l_@@_wrap_tl
      { \l_@@_current_line_tl \l_@@_newline_tl }
    \int_set:Nn \l_@@_current_line_int
      {
        \l_@@_current_word_int
        + \l_@@_current_indentation_int
      }
    \tl_set:Nx \l_@@_current_line_tl
      { \l_@@_current_indentation_tl \l_@@_current_word_tl }
  }
%    \end{macrocode}
% \end{macro}
% \end{macro}
% \end{macro}
%
% \begin{macro}[aux]{\@@_wrap_special:w}
% \begin{macro}[aux]{\@@_wrap_newline:w}
% \begin{macro}[aux]{\@@_wrap_indent:w}
% \begin{macro}[aux]{\@@_wrap_unindent:w}
% \begin{macro}[aux]{\@@_wrap_end:w}
%   When the \enquote{special} marker is encountered,
%   read what operation to perform, as a space-delimited
%   argument, perform it, and remember to loop.
%   In fact, to avoid spurious spaces when two special
%   actions follow each other, we look ahead for another
%   copy of the marker.
%   Forced newlines are almost identical to those caused by overflow,
%   except that here the word is empty.
%   To indent more, add four spaces to the start of the indentation
%   token list. To reduce indentation, rebuild the indentation token
%   list using \cs{prg_replicate:nn}.
%   At the end, we simply save the last line (without the run-on text),
%   and prevent the loop.
%    \begin{macrocode}
\cs_new_protected:Npn \@@_wrap_special:w #1 ~ #2 ~ #3 ~ %
  {
    \use:c { @@_wrap_#1: }
    \str_if_eq_x:nnTF { #2~#3 } { ~ \c_@@_wrap_marker_tl }
      { \@@_wrap_special:w }
      { \@@_wrap_loop:w #2 ~ #3 ~ }
  }
\cs_new_protected_nopar:Npn \@@_wrap_newline:
  {
    \tl_put_right:Nx \l_@@_wrap_tl
      { \l_@@_current_line_tl \l_@@_newline_tl }
    \int_zero:N \l_@@_current_line_int
    \tl_clear:N \l_@@_current_line_tl
    \bool_set_true:N \l_@@_line_start_bool
  }
\cs_new_protected_nopar:Npx \@@_wrap_indent:
  {
    \int_add:Nn \l_@@_current_indentation_int \c_four
    \tl_put_right:Nx \exp_not:N \l_@@_current_indentation_tl
      { \c_space_tl \c_space_tl \c_space_tl \c_space_tl }
  }
\cs_new_protected_nopar:Npn \@@_wrap_unindent:
  {
    \int_sub:Nn \l_@@_current_indentation_int \c_four
    \tl_set:Nx \l_@@_current_indentation_tl
      { \prg_replicate:nn \l_@@_current_indentation_int { ~ } }
  }
\cs_new_protected_nopar:Npn \@@_wrap_end:
  {
    \tl_put_right:Nx \l_@@_wrap_tl
      { \l_@@_current_line_tl }
    \use_none_delimit_by_q_stop:w
  }
%    \end{macrocode}
% \end{macro}
% \end{macro}
% \end{macro}
% \end{macro}
% \end{macro}
%
% \begin{macro}[aux]{\__str_count_ignore_spaces:N}
% \begin{macro}[aux]{\__str_count_ignore_spaces:n}
% \begin{macro}[aux]{\__str_count_loop:NNNNNNNNN}
%   The wrapping code requires to measure the number of character
%   in each word. This could be done with \cs{tl_count:n}, but
%   it is ten times faster (literally) to use the code below.
%    \begin{macrocode}
\cs_new_nopar:Npn \__str_count_ignore_spaces:N
  { \exp_args:No \__str_count_ignore_spaces:n }
\cs_new:Npn \__str_count_ignore_spaces:n #1
  {
    \__int_value:w \__int_eval:w
      \exp_after:wN \__str_count_loop:NNNNNNNNN \tl_to_str:n {#1}
        { X8 } { X7 } { X6 } { X5 } { X4 } { X3 } { X2 } { X1 } { X0 } \q_stop
    \__int_eval_end:
  }
\cs_new:Npn \__str_count_loop:NNNNNNNNN #1#2#3#4#5#6#7#8#9
  {
    \if_catcode:w X #9
      \exp_after:wN \use_none_delimit_by_q_stop:w
    \else:
      9 +
      \exp_after:wN \__str_count_loop:NNNNNNNNN
    \fi:
  }
%    \end{macrocode}
% \end{macro}
% \end{macro}
% \end{macro}
%
% \subsection{Messages}
%
%    \begin{macrocode}
\__msg_kernel_new:nnnn { kernel } { file-not-found }
  { File~'#1'~not~found. }
  {
    The~requested~file~could~not~be~found~in~the~current~directory,~
    in~the~TeX~search~path~or~in~the~LaTeX~search~path.
  }
\__msg_kernel_new:nnnn { kernel } { input-streams-exhausted }
  { Input~streams~exhausted }
  {
    TeX~can~only~open~up~to~16~input~streams~at~one~time.\\
    All~16~are~currently~in~use,~and~something~wanted~to~open~
    another~one.
  }
\__msg_kernel_new:nnnn { kernel } { output-streams-exhausted }
  { Output~streams~exhausted }
  {
    TeX~can~only~open~up~to~16~output~streams~at~one~time.\\
    All~16~are~currently~in~use,~and~something~wanted~to~open~
    another~one.
  }
\__msg_kernel_new:nnnn { kernel } { space-in-file-name }
  { Space~in~file~name~'#1'. }
  {
    Spaces~are~not~permitted~in~files~loaded~by~LaTeX: \\
    Further~errors~may~follow!
  }
%    \end{macrocode}
%
% \subsection{Deprecated functions}
%
% Deprecated on 2012-06-28, for removal by 2012-12-31.
%
% \begin{macro}{\iow_wrap:xnnnN}
%   This was renamed and one unneed argument was removed.
%    \begin{macrocode}
\cs_new_protected:Npn \iow_wrap:xnnnN #1#2#3#4#5
  { \iow_wrap:nnnN {#1} {#2} {#4} #5 }
%    \end{macrocode}
% \end{macro}
%
% Deprecated on 2012-06-24, for removal by 2012-12-31.
%
% \begin{variable}{\l_iow_line_length_int}
%   Simple rename.  Here we copy the \TeX{} register.
%    \begin{macrocode}
\cs_new_eq:NN \l_iow_line_length_int \l_iow_line_count_int
%    \end{macrocode}
% \end{variable}
%
% \begin{macro}{\ior_to:NN, \ior_gto:NN, \ior_str_to:NN, \ior_str_gto:NN}
%   The local variants are renames, while the global variants are
%   deprecated and add the \TeX{} primitive \tn{global}.
%    \begin{macrocode}
\cs_new_eq:NN \ior_to:NN \ior_get:NN
\cs_new_protected_nopar:Npn \ior_gto:NN { \tex_global:D \ior_to:NN }
\cs_new_eq:NN \ior_str_to:NN \ior_get_str:NN
\cs_new_protected_nopar:Npn \ior_str_gto:NN { \tex_global:D \ior_str_to:NN }
%    \end{macrocode}
% \end{macro}
%
% Deprecated on 2012-02-10, for removal by 2012-05-31.
%
% \begin{macro}{\iow_now_when_avail:Nn, \iow_now_when_avail:Nx}
% For writing only if the stream requested is open at all.
%    \begin{macrocode}
%<*deprecated>
\cs_new_protected:Npn \iow_now_when_avail:Nn #1
  { \cs_if_free:NTF #1 { \use_none:n } { \iow_now:Nn #1 } }
\cs_new_protected:Npn \iow_now_when_avail:Nx #1
  { \cs_if_free:NTF #1 { \use_none:n } { \iow_now:Nx #1 } }
%</deprecated>
%    \end{macrocode}
% \end{macro}
%
% Deprecated on 2011-05-27, for removal by 2011-08-31.
%
% \begin{macro}{\iow_now_buffer_safe:Nn, \iow_now_buffer_safe:Nx}
%   This is much more easily done using the wrapping system: there is
%   an expansion there, so a bit of a hack is needed.
%    \begin{macrocode}
%<*deprecated>
\cs_new_protected:Npn \iow_now_buffer_safe:Nn #1#2
  { \iow_wrap:xnnnN { \exp_not:n {#2} } { } \c_zero { } \iow_now:Nn #1 }
\cs_new_protected:Npn \iow_now_buffer_safe:Nx #1#2
  { \iow_wrap:xnnnN {#2} { } \c_zero { } \iow_now:Nn #1 }
%</deprecated>
%    \end{macrocode}
% \end{macro}
%
% \begin{macro}{\ior_open_streams:}
% \begin{macro}{\iow_open_streams:}
%   Slightly misleading names.
%    \begin{macrocode}
%<*deprecated>
\cs_new_eq:NN \ior_open_streams: \ior_list_streams:
\cs_new_eq:NN \iow_open_streams: \iow_list_streams:
%</deprecated>
%    \end{macrocode}
% \end{macro}
% \end{macro}
%
%    \begin{macrocode}
%</initex|package>
%    \end{macrocode}
%
% \end{implementation}
%
% \PrintIndex
