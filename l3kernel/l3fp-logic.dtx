% \iffalse meta-comment
%
%% File: l3fp-logic.dtx Copyright (C) 2011-2018 The LaTeX3 Project
%
% It may be distributed and/or modified under the conditions of the
% LaTeX Project Public License (LPPL), either version 1.3c of this
% license or (at your option) any later version.  The latest version
% of this license is in the file
%
%    https://www.latex-project.org/lppl.txt
%
% This file is part of the "l3kernel bundle" (The Work in LPPL)
% and all files in that bundle must be distributed together.
%
% -----------------------------------------------------------------------
%
% The development version of the bundle can be found at
%
%    https://github.com/latex3/latex3
%
% for those people who are interested.
%
%<*driver>
\documentclass[full,kernel]{l3doc}
\begin{document}
  \DocInput{\jobname.dtx}
\end{document}
%</driver>
% \fi
%
% \title{The \textsf{l3fp-logic} package\\
%   Floating point conditionals}
% \author{^^A
%  The \LaTeX3 Project\thanks
%    {^^A
%      E-mail:
%        \href{mailto:latex-team@latex-project.org}
%          {latex-team@latex-project.org}^^A
%    }^^A
% }
% \date{Released 2018/04/30}
%
% \maketitle
%
% \begin{documentation}
%
% \end{documentation}
%
% \begin{implementation}
%
% \section{\pkg{l3fp-logic} Implementation}
%
%    \begin{macrocode}
%<*initex|package>
%    \end{macrocode}
%
%    \begin{macrocode}
%<@@=fp>
%    \end{macrocode}
%
% \begin{macro}[EXP]{\@@_parse_word_max:N  , \@@_parse_word_min:N}
%   Those functions may receive a variable number of arguments.
%    \begin{macrocode}
\cs_new:Npn \@@_parse_word_max:N
  { \@@_parse_function:NNN \@@_minmax_o:Nw 2 }
\cs_new:Npn \@@_parse_word_min:N
  { \@@_parse_function:NNN \@@_minmax_o:Nw 0 }
%    \end{macrocode}
% \end{macro}
%
% \subsection{Syntax of internal functions}
%
% \begin{itemize}
%   \item \cs{@@_compare_npos:nwnw} \Arg{expo_1} \meta{body_1} |;|
%     \Arg{expo_2} \meta{body_2} |;|
%   \item \cs{@@_minmax_o:Nw} \meta{sign} \meta{floating point array}
%   \item \cs{@@_not_o:w} |?| \meta{floating point array} (with one floating point number only)
%   \item \cs{@@_&_o:ww} \meta{floating point} \meta{floating point}
%   \item \cs{@@_|_o:ww} \meta{floating point} \meta{floating point}
%   \item \cs{@@_ternary:NwwN}, \cs{@@_ternary_auxi:NwwN},
%     \cs{@@_ternary_auxii:NwwN} have to be understood.
% \end{itemize}
%
% \subsection{Existence test}
%
% \begin{macro}[pTF]{\fp_if_exist:N, \fp_if_exist:c}
%   Copies of the \texttt{cs} functions defined in \pkg{l3basics}.
%    \begin{macrocode}
\prg_new_eq_conditional:NNn \fp_if_exist:N \cs_if_exist:N { TF , T , F , p }
\prg_new_eq_conditional:NNn \fp_if_exist:c \cs_if_exist:c { TF , T , F , p }
%    \end{macrocode}
% \end{macro}
%
% \subsection{Comparison}
%
% \begin{macro}[pTF, EXP]{\fp_compare:n}
% \begin{macro}[EXP]{\@@_compare_return:w}
%   Within floating point expressions, comparison operators are treated
%   as operations, so we evaluate |#1|, then compare with $\pm 0$.
%   Tuples are \texttt{true}.
%    \begin{macrocode}
\prg_new_conditional:Npnn \fp_compare:n #1 { p , T , F , TF }
  {
    \exp_after:wN \@@_compare_return:w
    \exp:w \exp_end_continue_f:w \@@_parse:n {#1}
  }
\cs_new:Npn \@@_compare_return:w #1#2#3;
  {
    \if_charcode:w 0
          \@@_if_type_fp:NTwFw
            #1 { \use_i_delimit_by_q_stop:nw #3 \q_stop }
            \s_@@ 1 \q_stop
      \prg_return_false:
    \else:
      \prg_return_true:
    \fi:
  }
%    \end{macrocode}
% \end{macro}
% \end{macro}
%
% \begin{macro}[pTF, EXP]{\fp_compare:nNn}
% \begin{macro}[EXP]{\@@_compare_aux:wn}
%   Evaluate |#1| and |#3|, using an auxiliary to expand both, and feed
%   the two floating point numbers swapped to \cs{@@_compare_back_any:ww},
%   defined below.  Compare the result with |`#2-`=|, which is $-1$ for
%   |<|, $0$ for |=|, $1$ for |>| and $2$ for |?|.
%    \begin{macrocode}
\prg_new_conditional:Npnn \fp_compare:nNn #1#2#3 { p , T , F , TF }
  {
    \if_int_compare:w
        \exp_after:wN \@@_compare_aux:wn
          \exp:w \exp_end_continue_f:w \@@_parse:n {#1} {#3}
        = \@@_int_eval:w `#2 - `= \@@_int_eval_end:
      \prg_return_true:
    \else:
      \prg_return_false:
    \fi:
  }
\cs_new:Npn \@@_compare_aux:wn #1; #2
  {
    \exp_after:wN \@@_compare_back_any:ww
      \exp:w \exp_end_continue_f:w \@@_parse:n {#2} #1;
  }
%    \end{macrocode}
% \end{macro}
% \end{macro}
%
% \begin{macro}[EXP]{\@@_compare_back_any:ww, \@@_compare_back:ww, \@@_compare_nan:w}
%   \begin{quote}
%     \cs{@@_compare_back_any:ww} \meta{y} |;| \meta{x} |;|
%   \end{quote}
%   Expands (in the same way as \cs{int_eval:n}) to $-1$ if $x<y$, $0$
%   if $x=y$, $1$ if $x>y$, and $2$ otherwise (denoted as $x?y$).  If
%   either operand is \texttt{nan}, stop the comparison with
%   \cs{@@_compare_nan:w} returning $2$.  If $x$ is negative, swap the
%   outputs $1$ and $-1$ (\emph{i.e.}, $>$ and $<$); we can henceforth
%   assume that $x\geq 0$.  If $y\geq 0$, and they have the same type,
%   either they are normal and we compare them with
%   \cs{@@_compare_npos:nwnw}, or they are equal.  If $y\geq 0$, but of
%   a different type, the highest type is a larger number.  Finally, if
%   $y\leq 0$, then $x>y$, unless both are zero.
%    \begin{macrocode}
\cs_new:Npn \@@_compare_back_any:ww #1#2; #3
  {
    \@@_if_type_fp:NTwFw
      #1 { \@@_if_type_fp:NTwFw #3 \use_i:nn \s_@@ \use_ii:nn \q_stop }
      \s_@@ \use_ii:nn \q_stop
    \@@_compare_back:ww
    {
      \cs:w
        @@
        \@@_type_from_scan:N #1
        _compare_back
        \@@_type_from_scan:N #3
        :ww
      \cs_end:
    }
    #1#2 ; #3
  }
\cs_new:Npn \@@_compare_back:ww
    \s_@@ \@@_chk:w #1 #2 #3;
    \s_@@ \@@_chk:w #4 #5 #6;
  {
    \int_value:w
      \if_meaning:w 3 #1 \exp_after:wN \@@_compare_nan:w \fi:
      \if_meaning:w 3 #4 \exp_after:wN \@@_compare_nan:w \fi:
      \if_meaning:w 2 #5 - \fi:
      \if_meaning:w #2 #5
        \if_meaning:w #1 #4
          \if_meaning:w 1 #1
            \@@_compare_npos:nwnw #6; #3;
          \else:
            0
          \fi:
        \else:
          \if_int_compare:w #4 < #1 - \fi: 1
        \fi:
      \else:
        \if_int_compare:w #1#4 = 0 \exp_stop_f:
          0
        \else:
          1
        \fi:
      \fi:
    \exp_stop_f:
  }
\cs_new:Npn \@@_compare_nan:w #1 \fi: \exp_stop_f: { 2 \exp_stop_f: }
%    \end{macrocode}
% \end{macro}
%
% \begin{macro}[EXP]{\@@_compare_back_tuple:ww, \@@_tuple_compare_back:ww, \@@_tuple_compare_back_tuple:ww}
% \begin{macro}[EXP]{\@@_tuple_compare_back_loop:w}
%   Tuple and floating point numbers are not comparable so return $2$ in
%   mixed cases or when tuples have a different number of items.
%   Otherwise compare pairs of items with \cs{@@_compare_back_any:ww}
%   and if any don't match return~$2$ (as \cs{int_value:w} |02|
%   \cs{exp_stop_f:}).
%    \begin{macrocode}
\cs_new:Npn \@@_compare_back_tuple:ww #1; #2; { 2 }
\cs_new:Npn \@@_tuple_compare_back:ww #1; #2; { 2 }
\cs_new:Npn \@@_tuple_compare_back_tuple:ww
  \s_@@_tuple \@@_tuple_chk:w #1;
  \s_@@_tuple \@@_tuple_chk:w #2;
  {
    \int_compare:nNnTF { \@@_array_count:n {#1} } =
      { \@@_array_count:n {#2} }
      {
        \int_value:w 0
          \@@_tuple_compare_back_loop:w
              #1 { \s_@@ \prg_break: } ; @
              #2 { \s_@@ \prg_break: } ;
            \prg_break_point:
        \exp_stop_f:
      }
      { 2 }
  }
\cs_new:Npn \@@_tuple_compare_back_loop:w #1#2 ; #3 @ #4#5 ;
  {
    \use_none:n #1
    \use_none:n #4
    \if_int_compare:w
        \@@_compare_back_any:ww #1 #2 ; #4 #5 ; = 0 \exp_stop_f:
    \else:
      2 \exp_after:wN \prg_break:
    \fi:
    \@@_tuple_compare_back_loop:w #3 @
  }
%    \end{macrocode}
% \end{macro}
% \end{macro}
%
% \begin{macro}[EXP]{\@@_compare_npos:nwnw}
% \begin{macro}[EXP]{\@@_compare_significand:nnnnnnnn}
%   \begin{quote}
%     \cs{@@_compare_npos:nwnw}
%       \Arg{expo_1} \meta{body_1} |;|
%       \Arg{expo_2} \meta{body_2} |;|
%   \end{quote}
%   Within an \cs{int_value:w} \ldots{} \cs{exp_stop_f:} construction,
%   this expands to $0$ if the two numbers are equal, $-1$ if the first
%   is smaller, and $1$ if the first is bigger.  First compare the
%   exponents: the larger one denotes the larger number.  If they are
%   equal, we must compare significands.  If both the first $8$ digits and
%   the next $8$ digits coincide, the numbers are equal.  If only the
%   first $8$ digits coincide, the next $8$ decide.  Otherwise, the
%   first $8$ digits are compared.
%    \begin{macrocode}
\cs_new:Npn \@@_compare_npos:nwnw #1#2; #3#4;
  {
    \if_int_compare:w #1 = #3 \exp_stop_f:
      \@@_compare_significand:nnnnnnnn #2 #4
    \else:
      \if_int_compare:w #1 < #3 - \fi: 1
    \fi:
  }
\cs_new:Npn \@@_compare_significand:nnnnnnnn #1#2#3#4#5#6#7#8
  {
    \if_int_compare:w #1#2 = #5#6 \exp_stop_f:
      \if_int_compare:w #3#4 = #7#8 \exp_stop_f:
        0
      \else:
        \if_int_compare:w #3#4 < #7#8 - \fi: 1
      \fi:
    \else:
      \if_int_compare:w #1#2 < #5#6 - \fi: 1
    \fi:
  }
%    \end{macrocode}
% \end{macro}
% \end{macro}
%
% \subsection{Floating point expression loops}
%
% \begin{macro}[rEXP]
%   {
%     \fp_do_until:nn, \fp_do_while:nn,
%     \fp_until_do:nn, \fp_while_do:nn
%   }
%   These are quite easy given the above functions. The |do_until| and
%   |do_while| versions execute the body, then test.  The |until_do| and
%   |while_do| do it the other way round.
%    \begin{macrocode}
\cs_new:Npn \fp_do_until:nn #1#2
  {
    #2
    \fp_compare:nF {#1}
      { \fp_do_until:nn {#1} {#2} }
  }
\cs_new:Npn \fp_do_while:nn #1#2
  {
    #2
    \fp_compare:nT {#1}
      { \fp_do_while:nn {#1} {#2} }
  }
\cs_new:Npn \fp_until_do:nn #1#2
  {
    \fp_compare:nF {#1}
      {
        #2
        \fp_until_do:nn {#1} {#2}
      }
  }
\cs_new:Npn \fp_while_do:nn #1#2
  {
    \fp_compare:nT {#1}
      {
        #2
        \fp_while_do:nn {#1} {#2}
      }
  }
%    \end{macrocode}
% \end{macro}
%
% \begin{macro}[rEXP]
%   {
%     \fp_do_until:nNnn, \fp_do_while:nNnn,
%     \fp_until_do:nNnn, \fp_while_do:nNnn
%   }
%   As above but not using the |nNn| syntax.
%    \begin{macrocode}
\cs_new:Npn \fp_do_until:nNnn #1#2#3#4
  {
    #4
    \fp_compare:nNnF {#1} #2 {#3}
      { \fp_do_until:nNnn {#1} #2 {#3} {#4} }
  }
\cs_new:Npn \fp_do_while:nNnn #1#2#3#4
  {
    #4
    \fp_compare:nNnT {#1} #2 {#3}
      { \fp_do_while:nNnn {#1} #2 {#3} {#4} }
  }
\cs_new:Npn \fp_until_do:nNnn #1#2#3#4
  {
    \fp_compare:nNnF {#1} #2 {#3}
      {
        #4
        \fp_until_do:nNnn {#1} #2 {#3} {#4}
      }
  }
\cs_new:Npn \fp_while_do:nNnn #1#2#3#4
  {
    \fp_compare:nNnT {#1} #2 {#3}
      {
        #4
        \fp_while_do:nNnn {#1} #2 {#3} {#4}
      }
  }
%    \end{macrocode}
% \end{macro}
%
% \begin{macro}[EXP]{\fp_step_function:nnnN, \fp_step_function:nnnc}
% \begin{macro}[EXP]{\@@_step:wwwN, \@@_step_fp:wwwN}
% \begin{macro}[EXP]{\@@_step:NnnnnN, \@@_step:NfnnnN}
%   The approach here is somewhat similar to
%   \cs{int_step_function:nnnN}.  There are two subtleties: we use the
%   internal parser \cs{@@_parse:n} to avoid converting back and forth
%   from the internal representation; and (due to rounding) even a
%   non-zero step does not guarantee that the loop counter increases.
%    \begin{macrocode}
\cs_new:Npn \fp_step_function:nnnN #1#2#3
  {
    \exp_after:wN \@@_step:wwwN
      \exp:w \exp_end_continue_f:w \@@_parse_o:n {#1}
      \exp:w \exp_end_continue_f:w \@@_parse_o:n {#2}
      \exp:w \exp_end_continue_f:w \@@_parse:n {#3}
  }
\cs_generate_variant:Nn \fp_step_function:nnnN { nnnc }
%      \end{macrocode}
%   Only floating point numbers (not tuples) are allowed arguments.
%   Only \enquote{normal} floating points (not $\pm 0$,
%   $\pm\texttt{inf}$, \texttt{nan}) can be used as step; if positive,
%   call \cs{@@_step:NnnnnN} with argument |>| otherwise~|<|.  This
%   function has one more argument than its integer counterpart, namely
%   the previous value, to catch the case where the loop has made no
%   progress.  Conversion to decimal is done just before calling the
%   user's function.
%    \begin{macrocode}
\cs_new:Npn \@@_step:wwwN #1#2; #3#4; #5#6; #7
  {
    \@@_if_type_fp:NTwFw #1 { } \s_@@ \prg_break: \q_stop
    \@@_if_type_fp:NTwFw #3 { } \s_@@ \prg_break: \q_stop
    \@@_if_type_fp:NTwFw #5 { } \s_@@ \prg_break: \q_stop
    \use_i:nnnn { \@@_step_fp:wwwN #1#2; #3#4; #5#6; #7 }
    \prg_break_point:
    \use:n
      {
        \@@_error:nfff { fp-step-tuple } { \fp_to_tl:n { #1#2 ; } }
          { \fp_to_tl:n { #3#4 ; } } { \fp_to_tl:n { #5#6 ; } }
      }
  }
\cs_new:Npn \@@_step_fp:wwwN #1 ; \s_@@ \@@_chk:w #2#3#4 ; #5; #6
  {
    \token_if_eq_meaning:NNTF #2 1
      {
        \token_if_eq_meaning:NNTF #3 0
          { \@@_step:NnnnnN > }
          { \@@_step:NnnnnN < }
      }
      {
        \token_if_eq_meaning:NNTF #2 0
          {
            \__kernel_msg_expandable_error:nnn { kernel }
              { zero-step } {#6}
          }
          {
            \@@_error:nnfn { fp-bad-step } { }
              { \fp_to_tl:n { \s_@@ \@@_chk:w #2#3#4 ; } } {#6}
          }
        \use_none:nnnnn
      }
      { #1 ; } { \c_nan_fp } { \s_@@ \@@_chk:w #2#3#4 ; } { #5 ; } #6
  }
\cs_new:Npn \@@_step:NnnnnN #1#2#3#4#5#6
  {
    \fp_compare:nNnTF {#2} = {#3}
      {
        \@@_error:nffn { fp-tiny-step }
          { \fp_to_tl:n {#3} } { \fp_to_tl:n {#4} } {#6}
      }
      {
        \fp_compare:nNnF {#2} #1 {#5}
          {
            \exp_args:Nf #6 { \@@_to_decimal_dispatch:w #2 }
            \@@_step:NfnnnN
              #1 { \@@_parse:n { #2 + #4 } } {#2} {#4} {#5} #6
          }
      }
  }
\cs_generate_variant:Nn \@@_step:NnnnnN { Nf }
%    \end{macrocode}
% \end{macro}
% \end{macro}
% \end{macro}
%
% \begin{macro}{\fp_step_inline:nnnn, \fp_step_variable:nnnNn}
% \begin{macro}{\@@_step:NNnnnn}
%   As for \cs{int_step_inline:nnnn}, create a global function and apply it,
%   following up with a break point.
%    \begin{macrocode}
\cs_new_protected:Npn \fp_step_inline:nnnn
  {
    \int_gincr:N \g__kernel_prg_map_int
    \exp_args:NNc \@@_step:NNnnnn
      \cs_gset_protected:Npn
      { @@_map_ \int_use:N \g__kernel_prg_map_int :w }
  }
\cs_new_protected:Npn \fp_step_variable:nnnNn #1#2#3#4#5
  {
    \int_gincr:N \g__kernel_prg_map_int
    \exp_args:NNc \@@_step:NNnnnn
      \cs_gset_protected:Npx
      { @@_map_ \int_use:N \g__kernel_prg_map_int :w }
      {#1} {#2} {#3}
      {
        \tl_set:Nn \exp_not:N #4 {##1}
        \exp_not:n {#5}
      }
  }
\cs_new_protected:Npn \@@_step:NNnnnn #1#2#3#4#5#6
  {
    #1 #2 ##1 {#6}
    \fp_step_function:nnnN {#3} {#4} {#5} #2
    \prg_break_point:Nn \scan_stop: { \int_gdecr:N \g__kernel_prg_map_int }
  }
%    \end{macrocode}
% \end{macro}
% \end{macro}
%
%    \begin{macrocode}
\__kernel_msg_new:nnn { kernel } { fp-step-tuple }
  { Tuple~argument~in~fp_step_...~{#1}{#2}{#3}. }
\__kernel_msg_new:nnn { kernel } { fp-bad-step }
  { Invalid~step~size~#2~in~step~function~#3. }
\__kernel_msg_new:nnn { kernel } { fp-tiny-step }
  { Tiny~step~size~(#1+#2=#1)~in~step~function~#3. }
%    \end{macrocode}
%
% \subsection{Extrema}
%
% \begin{macro}[EXP]{\@@_minmax_o:Nw, \@@_minmax_aux_o:Nw}
%   First check all operands are floating point numbers.
%   The argument~|#1| is $2$~to find the maximum of an array~|#2| of
%   floating point numbers, and $0$~to find the minimum.  We read
%   numbers sequentially, keeping track of the largest (smallest) number
%   found so far.  If numbers are equal (for instance~$\pm0$), the first
%   is kept.  We append $-\infty$ ($\infty$), for the case of an empty
%   array.  Since no number is smaller (larger) than that, this
%   additional item only affects the maximum (minimum) in the case of
%   |max()| and |min()| with no argument.  The weird
%   fp-like trailing marker breaks the loop correctly: see the precise
%   definition of \cs{@@_minmax_loop:Nww}.
%    \begin{macrocode}
\cs_new:Npn \@@_minmax_o:Nw #1
  {
    \@@_parse_function_all_fp_o:fnw
      { \token_if_eq_meaning:NNTF 0 #1 { min } { max } }
      { \@@_minmax_aux_o:Nw #1 }
  }
\cs_new:Npn \@@_minmax_aux_o:Nw #1#2 @
  {
    \if_meaning:w 0 #1
      \exp_after:wN \@@_minmax_loop:Nww \exp_after:wN +
    \else:
      \exp_after:wN \@@_minmax_loop:Nww \exp_after:wN -
    \fi:
    #2
    \s_@@ \@@_chk:w 2 #1 \s_@@_exact ;
    \s_@@ \@@_chk:w { 3 \@@_minmax_break_o:w } ;
  }
%    \end{macrocode}
% \end{macro}
%
% \begin{macro}[EXP]{\@@_minmax_loop:Nww}
%   The first argument is $-$ or $+$ to denote the case where the
%   currently largest (smallest) number found (first floating point
%   argument) should be replaced by the new number (second floating
%   point argument).  If the new number is \texttt{nan}, keep that as
%   the extremum, unless that extremum is already a \texttt{nan}.
%   Otherwise, compare the two numbers.  If the new number is larger (in
%   the case of |max|) or smaller (in the case of |min|), the test
%   yields \texttt{true}, and we keep the second number as a new
%   maximum; otherwise we keep the first number.  Then loop.
%    \begin{macrocode}
\cs_new:Npn \@@_minmax_loop:Nww
    #1 \s_@@ \@@_chk:w #2#3; \s_@@ \@@_chk:w #4#5;
  {
    \if_meaning:w 3 #4
      \if_meaning:w 3 #2
        \@@_minmax_auxi:ww
      \else:
        \@@_minmax_auxii:ww
      \fi:
    \else:
      \if_int_compare:w
          \@@_compare_back:ww
            \s_@@ \@@_chk:w #4#5;
            \s_@@ \@@_chk:w #2#3;
          = #1 1 \exp_stop_f:
        \@@_minmax_auxii:ww
      \else:
        \@@_minmax_auxi:ww
      \fi:
    \fi:
    \@@_minmax_loop:Nww #1
      \s_@@ \@@_chk:w #2#3;
      \s_@@ \@@_chk:w #4#5;
  }
%    \end{macrocode}
% \end{macro}
%
% \begin{macro}[EXP]{\@@_minmax_auxi:ww, \@@_minmax_auxii:ww}
%   Keep the first/second number, and remove the other.
%    \begin{macrocode}
\cs_new:Npn \@@_minmax_auxi:ww  #1 \fi: \fi: #2 \s_@@ #3 ; \s_@@ #4;
  { \fi: \fi: #2 \s_@@ #3 ; }
\cs_new:Npn \@@_minmax_auxii:ww #1 \fi: \fi: #2 \s_@@ #3 ;
  { \fi: \fi: #2 }
%    \end{macrocode}
% \end{macro}
%
% \begin{macro}[EXP]{\@@_minmax_break_o:w}
%   This function is called from within an \cs{if_meaning:w} test.  Skip
%   to the end of the tests, close the current test with \cs{fi:}, clean
%   up, and return the appropriate number with one post-expansion.
%    \begin{macrocode}
\cs_new:Npn \@@_minmax_break_o:w #1 \fi: \fi: #2 \s_@@ #3; #4;
  { \fi: \@@_exp_after_o:w \s_@@ #3; }
%    \end{macrocode}
% \end{macro}
%
% \subsection{Boolean operations}
%
% \begin{macro}[EXP]{\@@_not_o:w, \@@_tuple_not_o:w}
%   Return \texttt{true} or \texttt{false}, with two expansions, one to
%   exit the conditional, and one to please \pkg{l3fp-parse}.  The first
%   argument is provided by \pkg{l3fp-parse} and is ignored.
%    \begin{macrocode}
\cs_new:Npn \@@_not_o:w #1 \s_@@ \@@_chk:w #2#3; @
  {
    \if_meaning:w 0 #2
      \exp_after:wN \exp_after:wN \exp_after:wN \c_one_fp
    \else:
      \exp_after:wN \exp_after:wN \exp_after:wN \c_zero_fp
    \fi:
  }
\cs_new:Npn \@@_tuple_not_o:w #1 @ { \exp_after:wN \c_zero_fp }
%    \end{macrocode}
% \end{macro}
%
% \begin{macro}[EXP]{\@@_&_o:ww, \@@_tuple_&_o:ww, \@@_&_tuple_o:ww, \@@_tuple_&_tuple_o:ww}
% \begin{macro}[EXP]{\@@_|_o:ww, \@@_tuple_|_o:ww, \@@_|_tuple_o:ww, \@@_tuple_|_tuple_o:ww}
% \begin{macro}[EXP]{\@@_and_return:wNw}
%   For \texttt{and}, if the first number is zero, return it (with the
%   same sign).  Otherwise, return the second one.  For \texttt{or}, the
%   logic is reversed: if the first number is non-zero, return it,
%   otherwise return the second number: we achieve that by hi-jacking
%   \cs{@@_&_o:ww}, inserting an extra argument, \cs{else:}, before
%   \cs{s_@@}.  In all cases, expand after the floating point number.
%    \begin{macrocode}
\group_begin:
  \char_set_catcode_letter:N &
  \char_set_catcode_letter:N |
  \cs_new:Npn \@@_&_o:ww #1 \s_@@ \@@_chk:w #2#3;
    {
      \if_meaning:w 0 #2 #1
        \@@_and_return:wNw \s_@@ \@@_chk:w #2#3;
      \fi:
      \@@_exp_after_o:w
    }
  \cs_new:Npn \@@_&_tuple_o:ww #1 \s_@@ \@@_chk:w #2#3;
    {
      \if_meaning:w 0 #2 #1
        \@@_and_return:wNw \s_@@ \@@_chk:w #2#3;
      \fi:
      \@@_exp_after_tuple_o:w
    }
  \cs_new:Npn \@@_tuple_&_o:ww #1; { \@@_exp_after_o:w }
  \cs_new:Npn \@@_tuple_&_tuple_o:ww #1; { \@@_exp_after_tuple_o:w }
  \cs_new:Npn \@@_|_o:ww { \@@_&_o:ww \else: }
  \cs_new:Npn \@@_|_tuple_o:ww { \@@_&_tuple_o:ww \else: }
  \cs_new:Npn \@@_tuple_|_o:ww #1; #2; { \@@_exp_after_tuple_o:w #1; }
  \cs_new:Npn \@@_tuple_|_tuple_o:ww #1; #2;
    { \@@_exp_after_tuple_o:w #1; }
\group_end:
\cs_new:Npn \@@_and_return:wNw #1; \fi: #2;
  { \fi: \@@_exp_after_o:w #1; }
%    \end{macrocode}
% \end{macro}
% \end{macro}
% \end{macro}
%
% \subsection{Ternary operator}
%
% \begin{macro}[EXP]
%   {\@@_ternary:NwwN, \@@_ternary_auxi:NwwN, \@@_ternary_auxii:NwwN}
%   The first function receives the test and the true branch of the |?:|
%   ternary operator.  It calls \cs{@@_ternary_auxii:NwwN} if the test
%   branch is a floating point number $\pm 0$, and otherwise calls
%   \cs{@@_ternary_auxi:NwwN}.  These functions select one of their two
%   arguments.
%    \begin{macrocode}
\cs_new:Npn \@@_ternary:NwwN #1 #2#3@ #4@ #5
  {
    \if_meaning:w \@@_parse_infix_::N #5
      \if_charcode:w 0
            \@@_if_type_fp:NTwFw
              #2 { \use_i:nn \use_i_delimit_by_q_stop:nw #3 \q_stop }
              \s_@@ 1 \q_stop
        \exp_after:wN \exp_after:wN \exp_after:wN \@@_ternary_auxii:NwwN
      \else:
        \exp_after:wN \exp_after:wN \exp_after:wN \@@_ternary_auxi:NwwN
      \fi:
      \exp_after:wN #1
      \exp:w \exp_end_continue_f:w
      \@@_exp_after_array_f:w #4 \s_@@_stop
      \exp_after:wN @
      \exp:w
        \@@_parse_operand:Nw \c_@@_prec_colon_int
        \@@_parse_expand:w
    \else:
      \__kernel_msg_expandable_error:nnnn
        { kernel } { fp-missing } { : } { ~for~?: }
      \exp_after:wN \@@_parse_continue:NwN
      \exp_after:wN #1
      \exp:w \exp_end_continue_f:w
      \@@_exp_after_array_f:w #4 \s_@@_stop
      \exp_after:wN #5
      \exp_after:wN #1
    \fi:
  }
\cs_new:Npn \@@_ternary_auxi:NwwN #1#2@#3@#4
  {
    \exp_after:wN \@@_parse_continue:NwN
    \exp_after:wN #1
    \exp:w \exp_end_continue_f:w
    \@@_exp_after_array_f:w #2 \s_@@_stop
    #4 #1
  }
\cs_new:Npn \@@_ternary_auxii:NwwN #1#2@#3@#4
  {
    \exp_after:wN \@@_parse_continue:NwN
    \exp_after:wN #1
    \exp:w \exp_end_continue_f:w
    \@@_exp_after_array_f:w #3 \s_@@_stop
    #4 #1
  }
%    \end{macrocode}
% \end{macro}
%
%    \begin{macrocode}
%</initex|package>
%    \end{macrocode}
%
% \end{implementation}
%
% \PrintChanges
%
% \PrintIndex
