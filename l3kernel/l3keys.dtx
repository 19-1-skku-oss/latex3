% \iffalse meta-comment
%
%% File: l3keys.dtx Copyright (C) 2006-2015 The LaTeX3 Project
%%
%% It may be distributed and/or modified under the conditions of the
%% LaTeX Project Public License (LPPL), either version 1.3c of this
%% license or (at your option) any later version.  The latest version
%% of this license is in the file
%%
%%    http://www.latex-project.org/lppl.txt
%%
%% This file is part of the "l3kernel bundle" (The Work in LPPL)
%% and all files in that bundle must be distributed together.
%%
%% The released version of this bundle is available from CTAN.
%%
%% -----------------------------------------------------------------------
%%
%% The development version of the bundle can be found at
%%
%%    http://www.latex-project.org/svnroot/experimental/trunk/
%%
%% for those people who are interested.
%%
%%%%%%%%%%%
%% NOTE: %%
%%%%%%%%%%%
%%
%%   Snapshots taken from the repository represent work in progress and may
%%   not work or may contain conflicting material!  We therefore ask
%%   people _not_ to put them into distributions, archives, etc. without
%%   prior consultation with the LaTeX3 Project.
%%
%% -----------------------------------------------------------------------
%
%<*driver>
\documentclass[full]{l3doc}
%</driver>
%<*driver|package>
\GetIdInfo$Id$
  {L3 Key-value interfaces}
%</driver|package>
%<*driver>
\begin{document}
  \DocInput{\jobname.dtx}
\end{document}
%</driver>
% \fi
%
% \title{^^A
%   The \pkg{l3keys} package\\ Key--value interfaces^^A
%   \thanks{This file describes v\ExplFileVersion,
%      last revised \ExplFileDate.}^^A
% }
%
% \author{^^A
%  The \LaTeX3 Project\thanks
%    {^^A
%      E-mail:
%        \href{mailto:latex-team@latex-project.org}
%          {latex-team@latex-project.org}^^A
%    }^^A
% }
%
% \date{Released \ExplFileDate}
%
% \maketitle
%
% \begin{documentation}
%
% The key--value method is a popular system for creating large numbers
% of settings for controlling function or package behaviour.  The
% system normally results in input of the form
% \begin{verbatim}
%   \MyModuleSetup{
%     key-one = value one,
%     key-two = value two
%   }
% \end{verbatim}
% or
% \begin{verbatim}
%   \MyModuleMacro[
%     key-one = value one,
%     key-two = value two
%   ]{argument}
% \end{verbatim}
% for the user.
%
% The high level functions here are intended as a method to create
% key--value controls. Keys are themselves created using a key--value
% interface, minimising the number of functions and arguments
% required. Each key is created by setting one or more \emph{properties}
% of the key:
% \begin{verbatim}
%   \keys_define:nn { mymodule }
%     {
%       key-one .code:n   = code including parameter #1,
%       key-two .tl_set:N = \l_mymodule_store_tl
%     }
% \end{verbatim}
% These values can then be set as with other key--value approaches:
% \begin{verbatim}
%   \keys_set:nn { mymodule }
%     {
%       key-one = value one,
%       key-two = value two
%     }
% \end{verbatim}
%
% At a document level, \cs{keys_set:nn} will be used within a
% document function, for example
% \begin{verbatim}
%   \DeclareDocumentCommand \MyModuleSetup { m }
%     { \keys_set:nn { mymodule } { #1 }  }
%   \DeclareDocumentCommand \MyModuleMacro { o m }
%     {
%       \group_begin:
%         \keys_set:nn { mymodule } { #1 }
%         % Main code for \MyModuleMacro
%       \group_end:
%     }
% \end{verbatim}
%
% Key names may contain any tokens, as they are handled internally
% using \cs{tl_to_str:n}. As will be discussed in
% section~\ref{sec:l3keys:subdivision}, it is suggested that the character
% |/| is reserved for sub-division of keys into logical
% groups. Functions and variables are \emph{not} expanded when creating
% key names, and so
% \begin{verbatim}
%   \tl_set:Nn \l_mymodule_tmp_tl { key }
%   \keys_define:nn { mymodule }
%     {
%       \l_mymodule_tmp_tl .code:n = code
%     }
% \end{verbatim}
% will create a key called \cs{l_mymodule_tmp_tl}, and not one called
% \texttt{key}.
%
% \section{Creating keys}
%
% \begin{function}{\keys_define:nn}
%   \begin{syntax}
%     \cs{keys_define:nn} \Arg{module} \Arg{keyval list}
%   \end{syntax}
%   Parses the \meta{keyval list} and defines the keys listed there for
%   \meta{module}. The \meta{module} name should be a text value, but
%   there are no restrictions on the nature of the text. In practice the
%   \meta{module} should be chosen to be unique to the module in question
%   (unless deliberately adding keys to an existing module).
%
%   The \meta{keyval list} should consist of one or more key names along
%   with an associated key \emph{property}. The properties of a key
%   determine how it acts. The individual properties are described
%   in the following text; a typical use of \cs{keys_define:nn} might
%   read
%   \begin{verbatim}
%     \keys_define:nn { mymodule }
%       {
%         keyname .code:n = Some~code~using~#1,
%         keyname .value_required:
%       }
%   \end{verbatim}
%   where the properties of the key begin from the |.| after the key
%   name.
% \end{function}
%
% The various properties available take either no arguments at
% all, or require one or more arguments. This is indicated in the
% name of the property using an argument specification. In the following
% discussion, each property is illustrated attached to an
% arbitrary \meta{key}, which when used may be supplied with a
% \meta{value}. All key \emph{definitions} are local.
%
% \begin{function}[updated = 2013-07-08]
%   {.bool_set:N, .bool_set:c, .bool_gset:N, .bool_gset:c}
%   \begin{syntax}
%     \meta{key} .bool_set:N = \meta{boolean}
%   \end{syntax}
%   Defines \meta{key} to set \meta{boolean} to \meta{value} (which
%   must be either \texttt{true} or \texttt{false}).  If the variable
%   does not exist, it will be created globally at the point that
%   the key is set up.
% \end{function}
%
% \begin{function}[added = 2011-08-28, updated = 2013-07-08]
%   {
%     .bool_set_inverse:N, .bool_set_inverse:c,
%     .bool_gset_inverse:N, .bool_gset_inverse:c
%   }
%   \begin{syntax}
%     \meta{key} .bool_set_inverse:N = \meta{boolean}
%   \end{syntax}
%   Defines \meta{key} to set \meta{boolean} to the logical
%   inverse of \meta{value} (which  must be either \texttt{true} or
%   \texttt{false}).
%   If the \meta{boolean} does not exist, it will be created globally
%   at the point that the key is set up.
% \end{function}
%
% \begin{function}{.choice:}
%   \begin{syntax}
%     \meta{key} .choice:
%   \end{syntax}
%   Sets \meta{key} to act as a choice key. Each valid choice
%   for \meta{key} must then be created, as discussed in
%   section~\ref{sec:l3keys:choice}.
% \end{function}
%
% \begin{function}[added = 2011-08-21, updated = 2013-07-10]
%   {.choices:nn, .choices:Vn, .choices:on, .choices:xn}
%   \begin{syntax}
%     \meta{key} .choices:nn = \Arg{choices} \Arg{code}
%   \end{syntax}
%   Sets \meta{key} to act as a choice key, and defines a series \meta{choices}
%   which are implemented using the \meta{code}. Inside \meta{code},
%   \cs{l_keys_choice_tl} will be the name of the choice made, and
%   \cs{l_keys_choice_int} will be the position of the choice in the list
%   of \meta{choices} (indexed from~$1$).
%   Choices are discussed in detail in section~\ref{sec:l3keys:choice}.
% \end{function}
%
% \begin{function}[added=2011-09-11]
%   {.clist_set:N, .clist_set:c, .clist_gset:N, .clist_gset:c}
%   \begin{syntax}
%     \meta{key} .clist_set:N = \meta{comma list variable}
%   \end{syntax}
%   Defines \meta{key} to set \meta{comma list variable} to \meta{value}.
%   Spaces around commas and empty items will be stripped.
%   If the variable does not exist, it
%   will be created globally at the point that the key is set up.
% \end{function}
%
% \begin{function}[updated = 2013-07-10]{.code:n}
%   \begin{syntax}
%     \meta{key} .code:n = \Arg{code}
%   \end{syntax}
%   Stores the \meta{code} for execution when \meta{key} is used.
%   The \meta{code} can include one parameter (|#1|), which will be the
%   \meta{value} given for the \meta{key}. The \texttt{x}-type variant
%   will expand \meta{code} at the point  where the \meta{key} is
%   created.
% \end{function}
%
% \begin{function}[updated = 2013-07-09]
%   {.default:n, .default:V, .default:o, .default:x}
%   \begin{syntax}
%     \meta{key} .default:n = \Arg{default}
%   \end{syntax}
%   Creates a \meta{default} value for \meta{key}, which is used if no
%   value is given. This will be used if only the key name is given,
%   but not if a blank \meta{value} is given:
%   \begin{verbatim}
%     \keys_define:nn { mymodule }
%       {
%         key .code:n    = Hello~#1,
%         key .default:n = World
%       }
%     \keys_set:nn { mymodule }
%       {
%         key = Fred, % Prints 'Hello Fred'
%         key,        % Prints 'Hello World'
%         key = ,     % Prints 'Hello '
%       }
%   \end{verbatim}
% \end{function}
%
% \begin{function}{.dim_set:N, .dim_set:c, .dim_gset:N, .dim_gset:c}
%   \begin{syntax}
%     \meta{key} .dim_set:N = \meta{dimension}
%   \end{syntax}
%   Defines \meta{key} to set \meta{dimension} to \meta{value} (which
%   must a dimension expression).  If the variable does not exist, it
%   will be created globally at the point that the key is set up.
% \end{function}
%
% \begin{function}{.fp_set:N, .fp_set:c, .fp_gset:N, .fp_gset:c}
%   \begin{syntax}
%     \meta{key} .fp_set:N = \meta{floating point}
%   \end{syntax}
%   Defines \meta{key} to set \meta{floating point} to \meta{value}
%   (which must a floating point expression).  If the variable does not exist,
%   it will be created globally at the point that the key is set up.
% \end{function}
%
% \begin{function}[added = 2013-07-14]
%   {.groups:n}
%   \begin{syntax}
%     \meta{key} .groups:n = \Arg{groups}
%   \end{syntax}
%   Defines \meta{key} as belonging to the \meta{groups} declared. Groups
%   provide a \enquote{secondary axis} for selectively setting keys, and are
%   described in Section~\ref{sec:l3keys:selective}.
% \end{function}
%
% \begin{function}[updated = 2013-07-09]
%   {.initial:n, .initial:V, .initial:o, .initial:x}
%   \begin{syntax}
%     \meta{key} .initial:n = \Arg{value}
%   \end{syntax}
%   Initialises the \meta{key} with the \meta{value}, equivalent to
%   \begin{quote}
%     \cs{keys_set:nn} \Arg{module} \{ \meta{key} = \meta{value} \}
%   \end{quote}
% \end{function}
%
% \begin{function}{.int_set:N, .int_set:c, .int_gset:N, .int_gset:c}
%   \begin{syntax}
%     \meta{key} .int_set:N = \meta{integer}
%   \end{syntax}
%   Defines \meta{key} to set \meta{integer} to \meta{value} (which
%   must be an integer expression).  If the variable does not exist, it
%   will be created globally at the point that the key is set up.
% \end{function}
%
% \begin{function}[updated = 2013-07-10]{.meta:n}
%   \begin{syntax}
%     \meta{key} .meta:n = \Arg{keyval list}
%   \end{syntax}
%   Makes \meta{key} a meta-key, which will set \meta{keyval list} in
%   one go.  If \meta{key} is given with a value at the time the key
%   is used, then the value will be passed through to the subsidiary
%   \meta{keys} for processing (as |#1|).
% \end{function}
%
% \begin{function}[added = 2013-07-10]{.meta:nn}
%   \begin{syntax}
%     \meta{key} .meta:nn = \Arg{path} \Arg{keyval list}
%   \end{syntax}
%   Makes \meta{key} a meta-key, which will set \meta{keyval list} in
%   one go using the \meta{path} in place of the current one.
%   If \meta{key} is given with a value at the time the key
%   is used, then the value will be passed through to the subsidiary
%   \meta{keys} for processing (as |#1|).
% \end{function}
%
% \begin{function}[added = 2011-08-21]{.multichoice:}
%   \begin{syntax}
%     \meta{key} .multichoice:
%   \end{syntax}
%   Sets \meta{key} to act as a multiple choice key. Each valid choice
%   for \meta{key} must then be created, as discussed in
%   section~\ref{sec:l3keys:choice}.
% \end{function}
%
% \begin{function}[added = 2011-08-21, updated = 2013-07-10]
%   {.multichoices:nn, .multichoices:Vn, .multichoices:on, .multichoices:xn}
%   \begin{syntax}
%     \meta{key} .multichoices:nn \Arg{choices} \Arg{code}
%   \end{syntax}
%   Sets \meta{key} to act as a multiple choice key, and defines a series
%   \meta{choices}
%   which are implemented using the \meta{code}. Inside \meta{code},
%   \cs{l_keys_choice_tl} will be the name of the choice made, and
%   \cs{l_keys_choice_int} will be the position of the choice in the list
%   of \meta{choices} (indexed from~$1$).
%   Choices are discussed in detail in section~\ref{sec:l3keys:choice}.
% \end{function}
%
% \begin{function}{.skip_set:N, .skip_set:c, .skip_gset:N, .skip_gset:c}
%   \begin{syntax}
%     \meta{key} .skip_set:N = \meta{skip}
%   \end{syntax}
%   Defines \meta{key} to set \meta{skip} to \meta{value} (which
%   must be a skip expression). If the variable does not exist, it
%   will be created globally at the point that the key is set up.
% \end{function}
%
% \begin{function}{.tl_set:N, .tl_set:c, .tl_gset:N, .tl_gset:c}
%   \begin{syntax}
%     \meta{key} .tl_set:N = \meta{token list variable}
%   \end{syntax}
%   Defines \meta{key} to set \meta{token list variable} to \meta{value}.
%   If the variable does not exist, it will be created globally
%   at the point that the key is set up.
% \end{function}
%
% \begin{function}{.tl_set_x:N, .tl_set_x:c, .tl_gset_x:N, .tl_gset_x:c}
%   \begin{syntax}
%     \meta{key} .tl_set_x:N = \meta{token list variable}
%   \end{syntax}
%   Defines \meta{key} to set \meta{token list variable} to \meta{value},
%   which will be subjected to an \texttt{x}-type expansion
%   (\emph{i.e.}~using \cs{tl_set:Nx}). If the variable does not exist,
%   it will be created globally at the point that the key is set up.
% \end{function}
%
% \begin{function}{.value_forbidden:}
%   \begin{syntax}
%     \meta{key} .value_forbidden:
%   \end{syntax}
%   Specifies that \meta{key} cannot receive a \meta{value} when used.
%   If a \meta{value} is given then an error will be issued.
% \end{function}
%
% \begin{function}{.value_required:}
%   \begin{syntax}
%      \meta{key} .value_required:
%   \end{syntax}
%   Specifies that \meta{key} must receive a \meta{value} when used.
%   If a \meta{value} is not given then an error will be issued.
% \end{function}
%
% \section{Sub-dividing keys}
% \label{sec:l3keys:subdivision}
%
% When creating large numbers of keys, it may be desirable to divide
% them into several sub-groups for a given module. This can be achieved
% either by adding a sub-division to the module name:
% \begin{verbatim}
%   \keys_define:nn { module / subgroup }
%     { key .code:n = code }
% \end{verbatim}
% or to the key name:
% \begin{verbatim}
%   \keys_define:nn { mymodule }
%     { subgroup / key .code:n = code }
% \end{verbatim}
% As illustrated, the best choice of token for sub-dividing keys in
% this way is |/|. This is because of the method that is
% used to represent keys internally. Both of the above code fragments
% set the same key, which has full name \texttt{module/subgroup/key}.
%
% As will be illustrated in the next section, this subdivision is
% particularly relevant to making multiple choices.
%
% \section{Choice and multiple choice keys}
% \label{sec:l3keys:choice}
%
% The \pkg{l3keys} system supports two types of choice key, in which a series
% of pre-defined input values are linked to varying implementations. Choice
% keys are usually created so that the various values are mutually-exclusive:
% only one can apply at any one time. \enquote{Multiple} choice keys are also
% supported: these allow a selection of values to be chosen at the same time.
%
% Mutually-exclusive choices are created by setting the \texttt{.choice:}
% property:
% \begin{verbatim}
%   \keys_define:nn { mymodule }
%     { key .choice: }
% \end{verbatim}
% For keys which are set up as choices, the valid choices are generated
% by creating sub-keys of the choice key. This can be carried out in
% two ways.
%
% In many cases, choices execute similar code which is dependant only
% on the name of the choice or the position of the choice in the
% list of all possibilities. Here, the keys can share the same code, and can
% be rapidly created using the  \texttt{.choices:nn} property.
% \begin{verbatim}
%   \keys_define:nn { mymodule }
%     {
%       key .choices:nn =
%         { choice-a, choice-b, choice-c }
%         {
%           You~gave~choice~'\tl_use:N \l_keys_choice_tl',~
%           which~is~in~position~\int_use:N \l_keys_choice_int \c_space_tl
%           in~the~list.
%         }
%     }
% \end{verbatim}
% The index \cs{l_keys_choice_int} in the list of choices starts at~$1$.
%
% \begin{variable}{\l_keys_choice_int, \l_keys_choice_tl}
%   Inside the code block for a choice generated using \texttt{.choices:nn},
%   the variables \cs{l_keys_choice_tl} and \cs{l_keys_choice_int} are
%   available to indicate the name of the current choice, and its position in
%   the comma list.  The position is indexed from~$1$. Note that, as with
%   standard key code generated using \texttt{.code:n}, the value passed to
%   the key (i.e.~the choice name) is also available as |#1|.
% \end{variable}
%
% On the other hand, it is sometimes useful to create choices which
% use entirely different code from one another. This can be achieved
% by setting the \texttt{.choice:} property of a key, then manually
% defining sub-keys.
% \begin{verbatim}
%   \keys_define:nn { mymodule }
%     {
%       key .choice:,
%       key / choice-a .code:n = code-a,
%       key / choice-b .code:n = code-b,
%       key / choice-c .code:n = code-c,
%     }
% \end{verbatim}
%
% It is possible to mix the two methods, but manually-created choices
% should \emph{not} use \cs{l_keys_choice_tl} or \cs{l_keys_choice_int}.
% These variables do not have defined behaviour when used outside of
% code created using \texttt{.choices:nn}
% (\emph{i.e.}~anything might happen).
%
% It is possible to allow choice keys to take values which have not previously
% been defined by adding code for the special \texttt{unknown} choice. The
% general behavior of the \texttt{unknown} key is described in
% Section~\ref{sec:l3keys:unknown}. A typical example in the case of a choice
% would be to issue a custom error message:
% \begin{verbatim}
%   \keys_define:nn { mymodule }
%     {
%       key .choice:,
%       key / choice-a .code:n = code-a,
%       key / choice-b .code:n = code-b,
%       key / choice-c .code:n = code-c,
%       key / unknown  .code:n =
%         \msg_error:nnxxx { mymodule } { unknown-choice }
%           { key }                              % Name of choice key
%           { choice-a , choice-b ,  choice-c }  % Valid choices
%           { \exp_not:n {#1} }                  % Invalid choice given
%       %
%       %
%     }
% \end{verbatim}
%
% Multiple choices are created in a very similar manner to mutually-exclusive
% choices, using the properties \texttt{.multichoice:} and
% \texttt{.multichoices:nn}. As with mutually exclusive choices, multiple
% choices are define as sub-keys. Thus both
% \begin{verbatim}
%   \keys_define:nn { mymodule }
%     {
%       key .multichoices:nn =
%         { choice-a, choice-b, choice-c }
%         {
%           You~gave~choice~'\tl_use:N \l_keys_choice_tl',~
%           which~is~in~position~
%           \int_use:N \l_keys_choice_int \c_space_tl
%           in~the~list.
%         }
%     }
% \end{verbatim}
% and
% \begin{verbatim}
%   \keys_define:nn { mymodule }
%     {
%       key .multichoice:,
%       key / choice-a .code:n = code-a,
%       key / choice-b .code:n = code-b,
%       key / choice-c .code:n = code-c,
%     }
% \end{verbatim}
% are valid.
%
% When a multiple choice key is set
% \begin{verbatim}
%   \keys_set:nn { mymodule }
%     {
%       key = { a , b , c } % 'key' defined as a multiple choice
%     }
% \end{verbatim}
% each choice is applied in turn, equivalent to a \texttt{clist} mapping or
% to applying each value individually:
% \begin{verbatim}
%   \keys_set:nn { mymodule }
%     {
%       key = a ,
%       key = b ,
%       key = c ,
%     }
% \end{verbatim}
% Thus each separate choice will have passed to it the
% \cs{l_keys_choice_tl} and \cs{l_keys_choice_int} in exactly
% the same way as described for \texttt{.choices:nn}.
%
% \section{Setting keys}
%
% \begin{function}
%   {\keys_set:nn, \keys_set:nV, \keys_set:nv, \keys_set:no}
%   \begin{syntax}
%     \cs{keys_set:nn} \Arg{module} \Arg{keyval list}
%   \end{syntax}
%   Parses the \meta{keyval list}, and sets those keys which are defined
%   for \meta{module}. The behaviour on finding an unknown key can be set
%   by defining a special \texttt{unknown} key: this will be illustrated
%   later.
% \end{function}
%
% \begin{variable}{\l_keys_key_tl, \l_keys_path_tl, \l_keys_value_tl}
%   For each key processed, information of the full \emph{path} of the
%   key, the \emph{name} of the key and the \emph{value} of the key is
%   available within three token list variables. These may be used within
%   the code of the key.
%
%   The \emph{value} is everything after the \texttt{=}, which may be
%   empty if no value was given. This is stored in \cs{l_keys_value_tl}, and
%   is not processed in any way by \cs{keys_set:nn}.
%
%   The \emph{path} of the key is a \enquote{full} description of the key,
%   and is unique for each key. It consists of the module and full key name,
%   thus for example
%   \begin{verbatim}
%     \keys_set:nn { mymodule } { key-a = some-value }
%  \end{verbatim}
%  has path \texttt{mymodule/key-a} while
%   \begin{verbatim}
%     \keys_set:nn { mymodule } { subset  / key-a = some-value }
%  \end{verbatim}
%  has path \texttt{mymodule/subset/key-a}. This information is stored in
%  \cs{l_keys_path_tl}, and will have been processed by \cs{tl_to_str:n}.
%
%  The \emph{name} of the key is the part of the path after the last
%  \texttt{/}, and thus is not unique. In the preceding examples, both keys
%  have name \texttt{key-a} despite having different paths.  This information
%  is stored in \cs{l_keys_key_tl}, and will have been processed by
%  \cs{tl_to_str:n}.
% \end{variable}
%
% \section{Handling of unknown keys}
% \label{sec:l3keys:unknown}
%
% If a key has not previously been defined (is unknown), \cs{keys_set:nn} will
% look for a special \texttt{unknown} key for the same module, and if this is
% not defined raises an error indicating that the key name was unknown. This
% mechanism can be used for example to issue custom error texts.
% \begin{verbatim}
%   \keys_define:nn { mymodule }
%     {
%       unknown .code:n =
%         You~tried~to~set~key~'\l_keys_key_tl'~to~'#1'.
%     }
% \end{verbatim}
%
% \begin{function}[added = 2011-08-23, updated = 2014-04-27]
%   {
%     \keys_set_known:nnN, \keys_set_known:nVN,
%     \keys_set_known:nvN, \keys_set_known:noN,
%     \keys_set_known:nn, \keys_set_known:nV,
%     \keys_set_known:nv, \keys_set_known:no
%   }
%   \begin{syntax}
%     \cs{keys_set_known:nnN} \Arg{module} \Arg{keyval list} \meta{tl}
%   \end{syntax}
%   In some cases, the desired behavior is to simply ignore unknown keys,
%   collecting up information on these for later processing. The
%   \cs{keys_set_known:nnN} function parses the \meta{keyval list}, and sets
%   those keys which are defined for \meta{module}. Any keys which are unknown
%   are not processed further by the parser.
%   The key--value pairs for each \emph{unknown} key name will be
%   stored in the \meta{tl} in a comma-separated form (\emph{i.e.}~an edited
%   version of the \meta{keyval list}). The \cs{keys_set_known:nn} version
%   skips this stage.
%
%   Use of \cs{keys_set_known:nnN} can be nested, with the correct residual
%   \meta{keyval list} returned at each stage.
% \end{function}
%
% \section{Selective key setting}
% \label{sec:l3keys:selective}
%
% In some cases it may be useful to be able to select only some keys for
% setting, even though these keys have the same path. For example, with
% a set of keys defined using
% \begin{verbatim}
%   \keys define:nn { mymodule }
%     {
%       key-one   .code:n   = { \my_func:n {#1} } ,
%       key-two   .tl_set:N = \l_my_a_tl          ,
%       key-three .tl_set:N = \l_my_b_tl          ,
%       key-four  .fp_set:N = \l_my_a_fp          ,
%     }
% \end{verbatim}
% the use of \cs{keys_set:nn} will attempt to set all four keys. However, in
% some contexts it may only be sensible to set some keys, or to control the
% order of setting. To do this, keys may be assigned to \emph{groups}:
% arbitrary sets which are independent of the key tree. Thus modifying the
% example to read
% \begin{verbatim}
%   \keys define:nn { mymodule }
%     {
%       key-one   .code:n   = { \my_func:n {#1} } ,
%       key-one   .groups:n = { first }           ,
%       key-two   .tl_set:N = \l_my_a_tl          ,
%       key-two   .groups:n = { first }           ,
%       key-three .tl_set:N = \l_my_b_tl          ,
%       key-three .groups:n = { second }          ,
%       key-four  .fp_set:N = \l_my_a_fp          ,
%     }
% \end{verbatim}
% will assign \texttt{key-one} and \texttt{key-two} to group \texttt{first},
% \texttt{key-three} to group \texttt{second}, while \texttt{key-four} is
% not assigned to a group.
%
% Selective key setting may be achieved either by selecting one or more
% groups to be made \enquote{active}, or by marking one or more groups to
% be ignored in key setting.
%
% \begin{function}[added = 2013-07-14, updated = 2014-04-27]
%   {
%     \keys_set_filter:nnnN, \keys_set_filter:nnVN,
%     \keys_set_filter:nnvN, \keys_set_filter:nnoN,
%     \keys_set_filter:nnn, \keys_set_filter:nnV,
%     \keys_set_filter:nnv, \keys_set_filter:nno
%   }
%   \begin{syntax}
%     \cs{keys_set_filter:nnnN} \Arg{module} \Arg{groups} \Arg{keyval list} \meta{tl}
%   \end{syntax}
%   Actives key filtering in an \enquote{opt-out} sense: keys assigned to any
%   of the \meta{groups} specified will be ignored. The \meta{groups} are
%   given as a comma-separated list. Unknown keys are not assigned to any
%   group and will thus always be set. The key--value pairs for each
%   key which is filtered out will be stored in the \meta{tl} in a
%   comma-separated form (\emph{i.e.}~an edited version of the \meta{keyval
%   list}). The \cs{keys_set_filter:nnn} version skips this stage.
%
%   Use of \cs{keys_set_filter:nnnN} can be nested, with the correct residual
%   \meta{keyval list} returned at each stage.
% \end{function}
%
% \begin{function}[added = 2013-07-14]
%   {
%     \keys_set_groups:nnn, \keys_set_groups:nnV,
%     \keys_set_groups:nnv, \keys_set_groups:nno
%   }
%   \begin{syntax}
%     \cs{keys_set_groups:nnn} \Arg{module} \Arg{groups} \Arg{keyval list}
%   \end{syntax}
%   Actives key filtering in an \enquote{opt-in} sense: only keys assigned to
%   one or more of the \meta{groups} specified will be set. The \meta{groups} are
%   given as a comma-separated list. Unknown keys are not assigned to any
%   group and will thus never be set.
% \end{function}
%
% \section{Utility functions for keys}
%
% \begin{function}[EXP,pTF]{\keys_if_exist:nn}
%   \begin{syntax}
%     \cs{keys_if_exist_p:nn} \Arg{module} \Arg{key} \\
%     \cs{keys_if_exist:nnTF} \Arg{module} \Arg{key} \Arg{true code} \Arg{false code}
%   \end{syntax}
%   Tests if the \meta{key} exists for \meta{module}, \emph{i.e.}~if any code
%   has been defined for \meta{key}.
% \end{function}
%
% \begin{function}[added = 2011-08-21,EXP,pTF]{\keys_if_choice_exist:nnn}
%   \begin{syntax}
%     \cs{keys_if_choice_exist_p:nnn} \Arg{module} \Arg{key} \Arg{choice} \\
%     \cs{keys_if_choice_exist:nnnTF} \Arg{module} \Arg{key} \Arg{choice} \Arg{true code} \Arg{false code}
%   \end{syntax}
%   Tests if the \meta{choice} is defined for the \meta{key} within the
%   \meta{module}, \emph{i.e.}~if any code has been defined for
%   \meta{key}/\meta{choice}. The test is \texttt{false} if the \meta{key}
%   itself is not defined.
% \end{function}
%
% \begin{function}{\keys_show:nn}
%   \begin{syntax}
%     \cs{keys_show:nn} \Arg{module} \Arg{key}
%   \end{syntax}
%   Shows the function which is used to actually implement a
%   \meta{key} for a \meta{module}.
% \end{function}
%
% \section{Low-level interface for parsing key--val lists}
%
% To re-cap from earlier, a key--value list is input of the form
% \begin{verbatim}
%   KeyOne = ValueOne ,
%   KeyTwo = ValueTwo ,
%   KeyThree
% \end{verbatim}
% where each key--value pair is separated by a comma from the rest of
% the list, and each key--value pair does not necessarily contain an
% equals sign or a value! Processing this type of input correctly
% requires a number of careful steps, to correctly account for
% braces, spaces and the category codes of separators.
%
% While the functions described earlier are used as a high-level interface
% for processing such input, in special circumstances you may wish to use
% a lower-level approach.
% The low-level parsing system converts a \meta{key--value list}
% into \meta{keys} and associated \meta{values}. After the parsing phase
% is completed, the resulting keys and values (or keys alone) are
% available for further processing. This processing is not carried out by the
% low-level parser itself, and so the parser requires the names of
% two functions along with the key--value list. One function is
% needed to process key--value pairs (it receives two arguments),
% and a second function is required for keys given without any value
% (it is called with a single argument).
%
% The parser does not double |#| tokens or expand any input. Active
% tokens |=| and |,| appearing at the outer level of braces are converted
% to category \enquote{other} (12) so that the parser does not \enquote{miss}
% any due to category code changes. Spaces are removed from the ends
% of the keys and values. Keys and values which are given in braces
% will have exactly one set removed (after space trimming), thus
% \begin{verbatim}
%    key = {value here},
% \end{verbatim}
% and
% \begin{verbatim}
%   key = value here,
% \end{verbatim}
% are treated identically.
%
% \begin{function}[updated = 2011-09-08]{\keyval_parse:NNn}
%   \begin{syntax}
%     \cs{keyval_parse:NNn} \meta{function_1} \meta{function_2} \Arg{key--value list}
%   \end{syntax}
%   Parses the \meta{key--value list} into a series of \meta{keys} and
%   associated \meta{values}, or keys alone (if no \meta{value} was
%   given).  \meta{function_1} should take one argument, while
%   \meta{function_2} should absorb two arguments. After
%   \cs{keyval_parse:NNn} has parsed the \meta{key--value list},
%   \meta{function_1} will be used to process keys given with no value
%   and \meta{function_2} will be used to process keys given with a
%   value. The order of the \meta{keys} in the \meta{key--value list}
%   will be preserved. Thus
%   \begin{verbatim}
%     \keyval_parse:NNn \function:n \function:nn
%       { key1 = value1 , key2 = value2, key3 = , key4 }
%   \end{verbatim}
%   will be converted into an input stream
%   \begin{verbatim}
%     \function:nn { key1 } { value1 }
%     \function:nn { key2 } { value2 }
%     \function:nn { key3 } { }
%     \function:n  { key4 }
%   \end{verbatim}
%   Note that there is a difference between an empty value (an equals
%   sign followed by nothing) and a missing value (no equals sign at
%   all). Spaces are trimmed from the ends of the \meta{key} and \meta{value},
%   then one \emph{outer} set of braces is removed from the \meta{key}
%   and \meta{value} as part of the processing.
% \end{function}
%
% \end{documentation}
%
% \begin{implementation}
%
% \section{\pkg{l3keys} Implementation}
%
%    \begin{macrocode}
%<*initex|package>
%    \end{macrocode}
%
% \subsection{Low-level interface}
%
%    \begin{macrocode}
%<@@=keyval>
%    \end{macrocode}
%
% For historical reasons this code uses the `keyval' module prefix.
%
% \begin{variable}{\g_@@_level_int}
%   To allow nesting of \cs{keyval_parse:NNn}, an integer is needed for
%   the current level.
%    \begin{macrocode}
\int_new:N \g_@@_level_int
%    \end{macrocode}
% \end{variable}
%
% \begin{variable}{\l_@@_key_tl, \l_@@_value_tl}
%   The current key name and value.
%    \begin{macrocode}
\tl_new:N \l_@@_key_tl
\tl_new:N \l_@@_value_tl
%    \end{macrocode}
% \end{variable}
%
% \begin{variable}{\l_@@_sanitise_tl}
% \begin{variable}{\l_@@_parse_tl}
%   Token list variables for dealing with awkward category codes in the
%   input.
%    \begin{macrocode}
\tl_new:N \l_@@_sanitise_tl
\tl_new:N \l_@@_parse_tl
%    \end{macrocode}
% \end{variable}
% \end{variable}
%
% \begin{macro}{\@@_parse:n}
%   The parsing function first deals with the category codes for
%   |=| and |,|, so that there are no odd events. The input is then
%   handed off to the element by element system.
%    \begin{macrocode}
\group_begin:
  \char_set_catcode_active:n { `\= }
  \char_set_catcode_active:n { `\, }
  \char_set_lccode:nn { `\8 } { `\= }
  \char_set_lccode:nn { `\9 } { `\, }
\tl_to_lowercase:n
  {
    \group_end:
    \cs_new_protected:Npn \@@_parse:n #1
      {
        \group_begin:
          \tl_set:Nn \l_@@_sanitise_tl {#1}
          \tl_replace_all:Nnn \l_@@_sanitise_tl { = } { 8 }
          \tl_replace_all:Nnn \l_@@_sanitise_tl { , } { 9 }
          \tl_clear:N \l_@@_parse_tl
          \exp_after:wN \@@_parse_elt:w \exp_after:wN
            \q_nil \l_@@_sanitise_tl 9 \q_recursion_tail 9 \q_recursion_stop
        \exp_after:wN \group_end:
        \l_@@_parse_tl
      }
  }
%    \end{macrocode}
% \end{macro}
%
% \begin{macro}{\@@_parse_elt:w}
%   Each item to be parsed will have \cs{q_nil} added to the front.
%   Hence the blank test here can always be used to find a totally
%   empty argument. To allow rapid matching for an |=| while not stripping
%   any braces, another \cs{q_nil} needed before the next phase of the
%   parser. Finally, loop around for the next item, adding in the
%   \cs{q_nil}: this happens whatever the nature of the current argument
%   as the end-of-recursion will clear up in all cases.  
%    \begin{macrocode}
\cs_new_protected:Npn \@@_parse_elt:w #1 ,
  {
    \tl_if_blank:oF { \use_none:n #1 }
      {
        \quark_if_recursion_tail_stop:o { \use_none:n #1 }
        \@@_split_key_value:w #1 \q_nil = = \q_stop
      }
    \@@_parse_elt:w \q_nil
  }
%    \end{macrocode}
% \end{macro}
%
% \begin{macro}{\@@_split_key_value:w}
% \begin{macro}[aux]{\@@_split_key:w}
%   Split the key and value using a delimited argument. The \cs{q_nil}
%   values added earlier ensure that no braces will be stripped as part
%   of this process. A blank test can then be used on |#3|: it is only
%   empty if there was no |=| in the original input. In that case, strip
%   a \cs{q_nil} from the end of the key name then hand on to remove other
%   things and store as \cs{l_@@_key_tl} before adding to the output token
%   list. In the case where there is an |=|, first tidy up the key, this time
%   without a trailing \cs{q_nil}, then do a check to ensure that |#3| is
%   exactly one token (|=|). With that done, the final stage is to hand off
%   to tidy up the value.
%    \begin{macrocode}
\cs_new_protected:Npn \@@_split_key_value:w #1 = #2 = #3 \q_stop
  {
    \tl_if_blank:nTF {#3}
      {
        \@@_split_key:w #1 \q_stop
        \tl_put_right:Nx \l_@@_parse_tl
          {
            \exp_not:c
              {
                @@_key_no_value_elt_
                \int_use:N \g_@@_level_int
                :n
              }
              { \exp_not:o \l_@@_key_tl }
          }
      }
      {
        \@@_split:Nn \l_@@_key_tl {#1}
        \tl_if_blank:oTF { \use_none:n #3 }
          { \@@_split_value:w \q_nil #2 \q_stop }
          { \__msg_kernel_error:nn { kernel } { misplaced-equals-sign } }
      }
  }
\cs_new_protected:Npn \@@_split_key:w #1 \q_nil \q_stop
  { \@@_split:Nn \l_@@_key_tl {#1} }
%    \end{macrocode}
% \end{macro}
% \end{macro}
%
% \begin{macro}{\@@_split:Nn}
% \begin{macro}[aux]{\@@_split:Nw}
%   There are two possible cases here. The first case is that |#1| is
%   surrounded by braces, in which case the |\use_none:nnn #1 \q_nil \q_nil|
%   will yield \cs{q_nil}. There, we can remove the leading \cs{q_nil}, the
%   braces and any spaces around the outside with \cs{use_ii:nnn}. On the
%   other hand, if there are no braces then the second branch removes the
%   leading \cs{q_nil} and any surrounding spaces.
%    \begin{macrocode}
\cs_new_protected:Npn \@@_split:Nn #1#2
  {
    \quark_if_nil:oTF { \use_none:nnn #2 \q_nil \q_nil }
      { \tl_set:Nx #1 { \exp_not:o { \use_ii:nnn #2 \q_nil } } }
      { \@@_split:Nw #1 #2 \q_stop }
  }
\cs_new_protected:Npn \@@_split:Nw #1 \q_nil #2 \q_stop
  { \tl_set:Nx #1 { \tl_trim_spaces:n {#2} } }
%    \end{macrocode}
% \end{macro}
% \end{macro}
%
% \begin{macro}{\@@_split_value:w}
%   As this stage there is just the value to deal with. The leading and
%   trailing \cs{q_nil} tokens are removed in two steps before storing the
%   value with spaces stripped (see \cs{@@_split:Nn}). Doing the storage
%   of key and value in one shot will put exactly the right number of
%   brace groups into the output.
%    \begin{macrocode}
\cs_new_protected:Npn \@@_split_value:w #1 \q_nil \q_stop
  {
    \@@_split:Nn \l_@@_value_tl {#1}
    \tl_put_right:Nx \l_@@_parse_tl
      {
        \exp_not:c
          { @@_key_value_elt_ \int_use:N \g_@@_level_int :nn }
          { \exp_not:o \l_@@_key_tl }
          { \exp_not:o \l_@@_value_tl }
      }
  }
%    \end{macrocode}
% \end{macro}
%
% \begin{macro}{\keyval_parse:NNn}
%   The outer parsing routine just sets up the processing functions and
%   hands off.
%    \begin{macrocode}
\cs_new_protected:Npn \keyval_parse:NNn #1#2#3
  {
    \int_gincr:N \g_@@_level_int
    \cs_gset_eq:cN
      { @@_key_no_value_elt_ \int_use:N \g_@@_level_int :n } #1
    \cs_gset_eq:cN
      { @@_key_value_elt_ \int_use:N \g_@@_level_int :nn }   #2
    \@@_parse:n {#3}
    \int_gdecr:N \g_@@_level_int
  }
%    \end{macrocode}
% \end{macro}
%
% One message for the low level parsing system.
%    \begin{macrocode}
\__msg_kernel_new:nnnn { kernel } { misplaced-equals-sign }
  { Misplaced~equals~sign~in~key-value~input~\msg_line_number: }
  {
    LaTeX~is~attempting~to~parse~some~key-value~input~but~found~
    two~equals~signs~not~separated~by~a~comma.
  }
%    \end{macrocode}
%
% \subsection{Constants and variables}
%
%    \begin{macrocode}
%<@@=keys>
%    \end{macrocode}
%
% \begin{variable}{\c_@@_code_root_tl, \c_@@_info_root_tl}
%   The prefixes for the code and variables of the keys themselves.
%    \begin{macrocode}
\tl_const:Nn \c_@@_code_root_tl { key~code~>~ }
\tl_const:Nn \c_@@_info_root_tl { key~info~>~ }
%    \end{macrocode}
% \end{variable}
%
% \begin{variable}{\c_@@_props_root_tl}
%   The prefix for storing properties.
%    \begin{macrocode}
\tl_const:Nn \c_@@_props_root_tl { key~prop~>~ }
%    \end{macrocode}
% \end{variable}
%
% \begin{variable}{\l_keys_choice_int, \l_keys_choice_tl}
%   Publicly accessible data on which choice is being used when several
%   are generated as a set.
%    \begin{macrocode}
\int_new:N \l_keys_choice_int
\tl_new:N \l_keys_choice_tl
%    \end{macrocode}
% \end{variable}
%
% \begin{variable}{\l_@@_groups_clist}
%   Used for storing and recovering the list of groups which apply to a key:
%   set as a comma list but at one point we have to use this for a token
%   list recovery.
%    \begin{macrocode}
\clist_new:N \l_@@_groups_clist
%    \end{macrocode}
% \end{variable}
%
% \begin{variable}{\l_keys_key_tl}
%   The name of a key itself: needed when setting keys.
%    \begin{macrocode}
\tl_new:N \l_keys_key_tl
%    \end{macrocode}
% \end{variable}
%
% \begin{variable}{\l_@@_module_tl}
%   The module for an entire set of keys.
%    \begin{macrocode}
\tl_new:N \l_@@_module_tl
%    \end{macrocode}
% \end{variable}
%
% \begin{variable}{\l_@@_no_value_bool}
%   A marker is needed internally to show if only a key or a key plus a
%   value was seen: this is recorded here.
%    \begin{macrocode}
\bool_new:N \l_@@_no_value_bool
%    \end{macrocode}
% \end{variable}
%
% \begin{variable}{\l_@@_only_known_bool}
%   Used to track if only \enquote{known} keys are being set.
%    \begin{macrocode}
\bool_new:N \l_@@_only_known_bool
%    \end{macrocode}
% \end{variable}
%
% \begin{variable}{\l_keys_path_tl}
%   The \enquote{path} of the current key is stored here: this is
%   available to the programmer and so is public.
%    \begin{macrocode}
\tl_new:N \l_keys_path_tl
%    \end{macrocode}
% \end{variable}
%
% \begin{variable}{\l_@@_property_tl}
%   The \enquote{property} begin set for a key at definition time is
%   stored here.
%    \begin{macrocode}
\tl_new:N \l_@@_property_tl
%    \end{macrocode}
% \end{variable}
%
% \begin{variable}{\l_@@_selective_bool, \l_@@_filtered_bool}
%   Two flags for using key groups: one to indicate that \enquote{selective}
%   setting is active, a second to specify which type (\enquote{opt-in}
%   or \enquote{opt-out}).
%    \begin{macrocode}
\bool_new:N \l_@@_selective_bool
\bool_new:N \l_@@_filtered_bool
%    \end{macrocode}
% \end{variable}
%
% \begin{variable}{\l_@@_selective_seq}
%   The list of key groups being filtered in or out during selective setting.
%    \begin{macrocode}
\seq_new:N \l_@@_selective_seq
%    \end{macrocode}
% \end{variable}
%
% \begin{variable}{\l_@@_unused_clist}
%   Used when setting only some keys to store those left over.
%    \begin{macrocode}
\tl_new:N \l_@@_unused_clist
%    \end{macrocode}
% \end{variable}
%
% \begin{variable}{\l_keys_value_tl}
%   The value given for a key: may be empty if no value was given.
%    \begin{macrocode}
\tl_new:N \l_keys_value_tl
%    \end{macrocode}
% \end{variable}
%
% \begin{variable}{\l_@@_tmp_bool}
%   Scratch space.
%    \begin{macrocode}
\bool_new:N \l_@@_tmp_bool
%    \end{macrocode}
% \end{variable}
%
% \subsection{The key defining mechanism}
%
% \begin{macro}{\keys_define:nn}
% \begin{macro}[aux]{\@@_define:nnn, \@@_define:onn}
%   The public function for definitions is just a wrapper for the lower
%   level mechanism, more or less. The outer function is designed to
%   keep a track of the current module, to allow safe nesting. The module is set
%   removing any leading |/| (which is not needed here).
%    \begin{macrocode}
\cs_new_protected:Npn \keys_define:nn
  { \@@_define:onn \l_@@_module_tl }
\cs_new_protected:Npn \@@_define:nnn #1#2#3
  {
    \tl_set:Nx \l_@@_module_tl { \tl_to_str:n {#2} }
    \keyval_parse:NNn \@@_define_elt:n \@@_define_elt:nn {#3}
    \tl_set:Nn \l_@@_module_tl {#1}
  }
\cs_generate_variant:Nn \@@_define:nnn { o }
%    \end{macrocode}
% \end{macro}
% \end{macro}
%
% \begin{macro}[int]{\@@_define_elt:n}
% \begin{macro}[int]{\@@_define_elt:nn}
% \begin{macro}[aux]{\@@_define_elt_aux:nn}
%   The outer functions here record whether a value was given and then
%   converge on a common internal mechanism. There is first a search for
%   a property in the current key name, then a check to make sure it is
%   known before the code hands off to the next step.
%    \begin{macrocode}
\cs_new_protected:Npn \@@_define_elt:n #1
  {
    \bool_set_true:N \l_@@_no_value_bool
    \@@_define_elt_aux:nn {#1} { }
  }
\cs_new_protected:Npn \@@_define_elt:nn #1#2
  {
    \bool_set_false:N \l_@@_no_value_bool
    \@@_define_elt_aux:nn {#1} {#2}
  }
\cs_new_protected:Npn \@@_define_elt_aux:nn #1#2
  {
    \@@_property_find:n {#1}
    \cs_if_exist:cTF { \c_@@_props_root_tl \l_@@_property_tl }
      { \@@_define_key:n {#2} }
      {
        \str_if_eq_x:nnF { \l_@@_property_tl } { .abort: }
          {
            \__msg_kernel_error:nnxx { kernel } { property-unknown }
              { \l_@@_property_tl } { \l_keys_path_tl }
           }
      }
  }
%    \end{macrocode}
% \end{macro}
% \end{macro}
% \end{macro}
%
% \begin{macro}[int]{\@@_property_find:n}
% \begin{macro}[aux]{\@@_property_find:w}
%   Searching for a property means finding the last |.| in the input,
%   and storing the text before and after it. Everything is turned into
%   strings, so there is no problem using an \texttt{x}-type expansion.
%    \begin{macrocode}
\cs_new_protected:Npn \@@_property_find:n #1
  {
    \tl_set:Nx \l_keys_path_tl { \l_@@_module_tl / }
    \tl_if_in:nnTF {#1} { . }
      { \@@_property_find:w #1 \q_stop }
      {
        \__msg_kernel_error:nnx { kernel } { key-no-property } {#1}
        \tl_set:Nn \l_@@_property_tl { .abort: }
      }
  }
\cs_new_protected:Npn \@@_property_find:w #1 . #2 \q_stop
  {
    \tl_set:Nx \l_keys_path_tl { \l_keys_path_tl \tl_to_str:n {#1} }
    \tl_if_in:nnTF {#2} { . }
      {
        \tl_set:Nx \l_keys_path_tl { \l_keys_path_tl . }
        \@@_property_find:w #2 \q_stop
      }
      { \tl_set:Nn \l_@@_property_tl { . #2 } }
  }
%    \end{macrocode}
% \end{macro}
% \end{macro}
%
% \begin{macro}[int]{\@@_define_key:n}
% \begin{macro}[aux]{\@@_define_key:w}
%   Two possible cases. If there is a value for the key, then just use
%   the function. If not, then a check to make sure there is no need for
%   a value with the property. If there should be one then complain,
%   otherwise execute it. There is no need to check for a |:| as if it
%   is missing the earlier tests will have failed.
%    \begin{macrocode}
\cs_new_protected:Npn \@@_define_key:n #1
  {
    \bool_if:NTF \l_@@_no_value_bool
      {
        \exp_after:wN \@@_define_key:w
          \l_@@_property_tl \q_stop
          { \use:c { \c_@@_props_root_tl \l_@@_property_tl } }
          {
            \__msg_kernel_error:nnxx { kernel }
              { property-requires-value } { \l_@@_property_tl }
              { \l_keys_path_tl }
            }
      }
      { \use:c { \c_@@_props_root_tl \l_@@_property_tl } {#1} }
  }
\cs_new_protected:Npn \@@_define_key:w #1 : #2 \q_stop
  { \tl_if_empty:nTF {#2} }
%    \end{macrocode}
% \end{macro}
% \end{macro}
%
% \subsection{Turning properties into actions}
%
% \begin{macro}[int]{\@@_bool_set:Nn, \@@_bool_set:cn}
%   Boolean keys are really just choices, but all done by hand. The
%   second argument here is the scope: either empty or \texttt{ g } for
%   global.
%    \begin{macrocode}
\cs_new_protected:Npn \@@_bool_set:Nn #1#2
  {
    \bool_if_exist:NF #1 { \bool_new:N #1 }
    \@@_choice_make:
    \@@_cmd_set:nx { \l_keys_path_tl / true }
      { \exp_not:c { bool_ #2 set_true:N } \exp_not:N #1 }
    \@@_cmd_set:nx { \l_keys_path_tl / false }
      { \exp_not:c { bool_ #2 set_false:N } \exp_not:N #1 }
    \@@_cmd_set:nn { \l_keys_path_tl / unknown }
      {
        \__msg_kernel_error:nnx { kernel } { boolean-values-only }
          { \l_keys_key_tl }
      }
    \@@_default_set:n { true }
  }
\cs_generate_variant:Nn \@@_bool_set:Nn { c }
%    \end{macrocode}
% \end{macro}
%
% \begin{macro}[int]{\@@_bool_set_inverse:Nn, \@@_bool_set_inverse:cn}
%   Inverse boolean setting is much the same.
%    \begin{macrocode}
\cs_new_protected:Npn \@@_bool_set_inverse:Nn #1#2
  {
    \bool_if_exist:NF #1 { \bool_new:N #1 }
    \@@_choice_make:
    \@@_cmd_set:nx { \l_keys_path_tl / true }
      { \exp_not:c { bool_ #2 set_false:N } \exp_not:N #1 }
    \@@_cmd_set:nx { \l_keys_path_tl / false }
      { \exp_not:c { bool_ #2 set_true:N } \exp_not:N #1 }
    \@@_cmd_set:nn { \l_keys_path_tl / unknown }
      {
        \__msg_kernel_error:nnx { kernel } { boolean-values-only }
          { \l_keys_key_tl }
      }
    \@@_default_set:n { true }
  }
\cs_generate_variant:Nn \@@_bool_set_inverse:Nn { c }
%    \end{macrocode}
% \end{macro}
%
% \begin{macro}[int]{\@@_choice_make:, \@@_multichoice_make:}
% \begin{macro}[int]{\@@_choice_make:N}
% \begin{macro}[aux]{\@@_choice_make_aux:N}
% \begin{macro}[aux]{\@@_parent:n, \@@_parent:o}
% \begin{macro}[aux]{\@@_parent:wn}
%   To make a choice from a key, two steps: set the code, and set the
%   unknown key. There is one point to watch here: choice keys cannot be
%   nested! As multichoices and choices are essentially the same bar one
%   function, the code is given together.
%    \begin{macrocode}
\cs_new_protected_nopar:Npn \@@_choice_make:
  { \@@_choice_make:N \@@_choice_find:n }
\cs_new_protected_nopar:Npn \@@_multichoice_make:
  { \@@_choice_make:N \@@_multichoice_find:n }
\cs_new_protected_nopar:Npn \@@_choice_make:N #1
  {
    \prop_if_exist:cTF
      { \c_@@_info_root_tl \@@_parent:o \l_keys_path_tl }
      {
        \prop_get:cnNTF
          { \c_@@_info_root_tl \@@_parent:o \l_keys_path_tl }
          { choice } \l_keys_value_tl
          {
            \__msg_kernel_error:nnxx { kernel } { nested-choice-key }
              { \l_keys_path_tl } { \@@_parent:o \l_keys_path_tl }
          }
          { \@@_choice_make_aux:N #1 }
      }
      { \@@_choice_make_aux:N #1 }
  }
\cs_new_protected_nopar:Npn \@@_choice_make_aux:N #1
  {
    \@@_cmd_set:nn { \l_keys_path_tl } { #1 {##1} }
    \prop_put:cnn { \c__keys_info_root_tl \l_keys_path_tl } { choice }
      { true }
    \@@_cmd_set:nn { \l_keys_path_tl / unknown }
      {
        \__msg_kernel_error:nnxx { kernel } { key-choice-unknown }
          { \l_keys_path_tl } {##1}
      }
  }
\cs_new:Npn \@@_parent:n #1
  { \@@_parent:wn #1 / / \q_stop { } }
\cs_generate_variant:Nn \@@_parent:n { o }
\cs_new:Npn \@@_parent:wn #1 / #2 / #3 \q_stop #4
  {
    \tl_if_blank:nTF {#2}
      { \use_none:n #4 }
      {
        \@@_parent:wn #2 / #3 \q_stop { #4 / #1 }
      }
  }
%    \end{macrocode}
% \end{macro}
% \end{macro}
% \end{macro}
% \end{macro}
% \end{macro}
%
% \begin{macro}[int]{\@@_choices_make:nn, \@@_multichoices_make:nn}
% \begin{macro}[aux]{\@@_choices_make:Nnn}
%   Auto-generating choices means setting up the root key as a choice, then
%   defining each choice in turn.
%    \begin{macrocode}
\cs_new_protected_nopar:Npn \@@_choices_make:nn 
  { \@@_choices_make:Nnn \@@_choice_make: }
\cs_new_protected_nopar:Npn \@@_multichoices_make:nn 
  { \@@_choices_make:Nnn \@@_multichoice_make: }  
\cs_new_protected:Npn \@@_choices_make:Nnn #1#2#3
  {
    #1
    \int_zero:N \l_keys_choice_int
    \clist_map_inline:nn {#2}
      {
        \int_incr:N \l_keys_choice_int
        \@@_cmd_set:nx { \l_keys_path_tl / ##1 }
          {
            \tl_set:Nn \exp_not:N \l_keys_choice_tl {##1}
            \int_set:Nn \exp_not:N \l_keys_choice_int
              { \int_use:N \l_keys_choice_int }
            \exp_not:n {#3}
          }
      }
  }
%    \end{macrocode}
% \end{macro}
% \end{macro}
%
% \begin{macro}[int]
%   {\@@_cmd_set:nn, \@@_cmd_set:nx, \@@_cmd_set:Vn, \@@_cmd_set:Vo}
%   Creating a new command means tidying up the properties and then making
%   the internal function which actually does the work.
%    \begin{macrocode}
\cs_new_protected:Npn \@@_cmd_set:nn #1#2
  {
    \prop_clear_new:c { \c_@@_info_root_tl #1 }
    \cs_set:cpn { \c_@@_code_root_tl #1 } ##1 {#2}
  }
\cs_generate_variant:Nn \@@_cmd_set:nn { nx , Vn , Vo }
%    \end{macrocode}
% \end{macro}
%
% \begin{macro}[int]{\@@_default_set:n}
%   Setting a default value is easy.
%    \begin{macrocode}
\cs_new_protected:Npn \@@_default_set:n #1
  {
    \prop_if_exist:cT { \c_@@_info_root_tl \l_keys_path_tl }
      {
        \prop_put:cnn { \c_@@_info_root_tl \l_keys_path_tl }
          { default } {#1}
      }
  }
%    \end{macrocode}
% \end{macro}
%
% \begin{macro}[int]{\@@_groups_set:n}
%   Assigning a key to one or more groups uses comma lists. So that the comma
%   list is \enquote{well-behaved} later, the storage is done via a stored
%   list here, which does the normalisation.
%    \begin{macrocode}
\cs_new_protected:Npn \@@_groups_set:n #1
  {
    \prop_if_exist:cT { \c_@@_info_root_tl \l_keys_path_tl }
      {
        \clist_set:Nn \l_@@_groups_clist {#1}
        \prop_put:cnV { \c_@@_info_root_tl \l_keys_path_tl }
          { groups } \l_@@_groups_clist
      }
  }
%    \end{macrocode}
% \end{macro}
%
% \begin{macro}[int]{\@@_initialise:n}
% \begin{macro}[aux]{\@@_initialise:wn}
%   A set up for initialisation from which the key system requires that
%   the path is split up into a module and a key name. At this stage,
%   \cs{l_keys_path_tl} will contain \texttt{/} so a split is easy to do.
%    \begin{macrocode}
\cs_new_protected:Npn \@@_initialise:n #1
  { \exp_after:wN \@@_initialise:wn \l_keys_path_tl \q_stop {#1} }
\cs_new_protected:Npn \@@_initialise:wn #1 / #2 \q_stop #3
  { \keys_set:nn {#1} { #2 = {#3} } }
%    \end{macrocode}
% \end{macro}
% \end{macro}
%
% \begin{macro}[int]{\@@_meta_make:n}
% \begin{macro}[int]{\@@_meta_make:nn}
%   To create a meta-key, simply set up to pass data through.
%    \begin{macrocode}
\cs_new_protected:Npn \@@_meta_make:n #1
  {
    \@@_cmd_set:Vo \l_keys_path_tl
      {
        \exp_after:wN \keys_set:nn
        \exp_after:wN { \l_@@_module_tl } {#1}
      }
  }
\cs_new_protected:Npn \@@_meta_make:nn #1#2
  { \@@_cmd_set:Vn \l_keys_path_tl { \keys_set:nn {#1} {#2} } }
%    \end{macrocode}
% \end{macro}
% \end{macro}
%
% \begin{macro}[int]{\@@_value_requirement:n}
%   Values can be required or forbidden by having the appropriate marker
%   set. First, both the required and forbidden ones are clear, just in case!
%    \begin{macrocode}
\cs_new_protected:Npn \@@_value_requirement:n #1
  {
    \prop_if_exist:cT { \c_@@_info_root_tl \l_keys_path_tl }
      {
        \prop_remove:cn { \c_@@_info_root_tl \l_keys_path_tl }
          { required }
        \prop_remove:cn { \c_@@_info_root_tl \l_keys_path_tl }
          { forbidden }
        \prop_put:cnn { \c_@@_info_root_tl \l_keys_path_tl }
          {#1} { true }
    }
  }
%    \end{macrocode}
% \end{macro}
%
% \begin{macro}[int]{\@@_variable_set:NnnN, \@@_variable_set:cnnN}
%   Setting a variable takes the type and scope separately so that
%   it is easy to make a new variable if needed.
%    \begin{macrocode}
\cs_new_protected:Npn \@@_variable_set:NnnN #1#2#3#4
  {
    \use:c { #2_if_exist:NF } #1 { \use:c { #2 _new:N } #1 }
    \@@_cmd_set:nx { \l_keys_path_tl }
      {
        \exp_not:c { #2 _ #3 set:N #4 }
        \exp_not:N #1
        \exp_not:n  { {##1} }
      }
  }
\cs_generate_variant:Nn \@@_variable_set:NnnN { c }
%    \end{macrocode}
% \end{macro}
%
% \subsection{Creating key properties}
%
% The key property functions are all wrappers for internal functions,
% meaning that things stay readable and can also be altered later on.
%
% Importantly, while key properties have \enquote{normal} argument specs, the
% underlying code always supplies one braced argument to these. As such, argument
% expansion is handled by hand rather than using the standard tools. This shows
% up particularly for the two-argument properties, where things would otherwise
% go badly wrong.
%
% \begin{macro}{.bool_set:N, .bool_set:c}
% \begin{macro}{.bool_gset:N, .bool_gset:c}
%   One function for this.
%    \begin{macrocode}
\cs_new_protected:cpn { \c_@@_props_root_tl .bool_set:N } #1
  { \@@_bool_set:Nn #1 { } }
\cs_new_protected:cpn { \c_@@_props_root_tl .bool_set:c } #1
  { \@@_bool_set:cn {#1} { } }
\cs_new_protected:cpn { \c_@@_props_root_tl .bool_gset:N } #1
  { \@@_bool_set:Nn #1 { g } }
\cs_new_protected:cpn { \c_@@_props_root_tl .bool_gset:c } #1
  { \@@_bool_set:cn {#1} { g } }
%    \end{macrocode}
% \end{macro}
% \end{macro}
%
% \begin{macro}{.bool_set_inverse:N, .bool_set_inverse:c}
% \begin{macro}{.bool_gset_inverse:N, .bool_gset_inverse:c}
%   One function for this.
%    \begin{macrocode}
\cs_new_protected:cpn { \c_@@_props_root_tl .bool_set_inverse:N } #1
  { \@@_bool_set_inverse:Nn #1 { } }
\cs_new_protected:cpn { \c_@@_props_root_tl .bool_set_inverse:c } #1
  { \@@_bool_set_inverse:cn {#1} { } }
\cs_new_protected:cpn { \c_@@_props_root_tl .bool_gset_inverse:N } #1
  { \@@_bool_set_inverse:Nn #1 { g } }
\cs_new_protected:cpn { \c_@@_props_root_tl .bool_gset_inverse:c } #1
  { \@@_bool_set_inverse:cn {#1} { g } }
%    \end{macrocode}
% \end{macro}
% \end{macro}
%
% \begin{macro}{.choice:}
%   Making a choice is handled internally, as it is also needed by
%   \texttt{.generate_choices:n}.
%    \begin{macrocode}
\cs_new_protected_nopar:cpn { \c_@@_props_root_tl .choice: }
  { \@@_choice_make: }
%    \end{macrocode}
% \end{macro}
%
% \begin{macro}
%   {.choices:nn, .choices:Vn, .choices:on, .choices:xn}
%   For auto-generation of a series of mutually-exclusive choices.
%   Here, |#1| will consist of two separate
%   arguments, hence the slightly odd-looking implementation.
%    \begin{macrocode}
\cs_new_protected:cpn { \c_@@_props_root_tl .choices:nn } #1
  { \@@_choices_make:nn #1 }
\cs_new_protected:cpn { \c_@@_props_root_tl .choices:Vn } #1
  { \exp_args:NV \@@_choices_make:nn #1 }
\cs_new_protected:cpn { \c_@@_props_root_tl .choices:on } #1
  { \exp_args:No \@@_choices_make:nn #1 }
\cs_new_protected:cpn { \c_@@_props_root_tl .choices:xn } #1
  { \exp_args:Nx \@@_choices_make:nn #1 }
%    \end{macrocode}
% \end{macro}
%
% \begin{macro}{.code:n}
%   Creating code is simply a case of passing through to the underlying
%   \texttt{set} function.
%    \begin{macrocode}
\cs_new_protected:cpn { \c_@@_props_root_tl .code:n } #1
  { \@@_cmd_set:nn { \l_keys_path_tl } {#1} }
%    \end{macrocode}
% \end{macro}
%
% \begin{macro}{.clist_set:N, .clist_set:c}
% \begin{macro}{.clist_gset:N, .clist_gset:c}
%    \begin{macrocode}
\cs_new_protected:cpn { \c_@@_props_root_tl .clist_set:N } #1
  { \@@_variable_set:NnnN #1 { clist } { } n }
\cs_new_protected:cpn { \c_@@_props_root_tl .clist_set:c } #1
  { \@@_variable_set:cnnN {#1} { clist } { } n }
\cs_new_protected:cpn { \c_@@_props_root_tl .clist_gset:N } #1
  { \@@_variable_set:NnnN #1 { clist } { g } n }
\cs_new_protected:cpn { \c_@@_props_root_tl .clist_gset:c } #1
  { \@@_variable_set:cnnN {#1} { clist } { g } n }
%    \end{macrocode}
% \end{macro}
% \end{macro}
%
% \begin{macro}{.default:n, .default:V, .default:o, .default:x}
%   Expansion is left to the internal functions.
%    \begin{macrocode}
\cs_new_protected:cpn { \c_@@_props_root_tl .default:n } #1
  { \@@_default_set:n {#1} }
\cs_new_protected:cpn { \c_@@_props_root_tl .default:V } #1
  { \exp_args:NV \@@_default_set:n #1 }
\cs_new_protected:cpn { \c_@@_props_root_tl .default:o } #1
  { \exp_args:No \@@_default_set:n {#1} }
\cs_new_protected:cpn { \c_@@_props_root_tl .default:x } #1
  { \exp_args:Nx \@@_default_set:n {#1} }
%    \end{macrocode}
% \end{macro}
%
% \begin{macro}{.dim_set:N, .dim_set:c}
% \begin{macro}{.dim_gset:N, .dim_gset:c}
% Setting a variable is very easy: just pass the data along.
%    \begin{macrocode}
\cs_new_protected:cpn { \c_@@_props_root_tl .dim_set:N } #1
  { \@@_variable_set:NnnN #1 { dim } { } n }
\cs_new_protected:cpn { \c_@@_props_root_tl .dim_set:c } #1
  { \@@_variable_set:cnnN {#1} { dim } { } n }
\cs_new_protected:cpn { \c_@@_props_root_tl .dim_gset:N } #1
  { \@@_variable_set:NnnN #1 { dim } { g } n }
\cs_new_protected:cpn { \c_@@_props_root_tl .dim_gset:c } #1
  { \@@_variable_set:cnnN {#1} { dim } { g } n }
%    \end{macrocode}
% \end{macro}
% \end{macro}
%
% \begin{macro}{.fp_set:N, .fp_set:c}
% \begin{macro}{.fp_gset:N, .fp_gset:c}
%   Setting a variable is very easy: just pass the data along.
%    \begin{macrocode}
\cs_new_protected:cpn { \c_@@_props_root_tl .fp_set:N } #1
  { \@@_variable_set:NnnN #1 { fp } { } n }
\cs_new_protected:cpn { \c_@@_props_root_tl .fp_set:c } #1
  { \@@_variable_set:cnnN {#1} { fp } { } n }
\cs_new_protected:cpn { \c_@@_props_root_tl .fp_gset:N } #1
  { \@@_variable_set:NnnN #1 { fp } { g } n }
\cs_new_protected:cpn { \c_@@_props_root_tl .fp_gset:c } #1
  { \@@_variable_set:cnnN {#1} { fp } { g } n }
%    \end{macrocode}
% \end{macro}
% \end{macro}
%
% \begin{macro}{.groups:n}
%   A single property to create groups of keys.
%    \begin{macrocode}
\cs_new_protected:cpn { \c_@@_props_root_tl .groups:n } #1
  { \@@_groups_set:n {#1} }
%    \end{macrocode}
% \end{macro}
%
% \begin{macro}{.initial:n, .initial:V, .initial:o, .initial:x}
%   The standard hand-off approach.
%    \begin{macrocode}
\cs_new_protected:cpn { \c_@@_props_root_tl .initial:n } #1
  { \@@_initialise:n {#1} }
\cs_new_protected:cpn { \c_@@_props_root_tl .initial:V } #1
  { \exp_args:NV \@@_initialise:n #1 }
\cs_new_protected:cpn { \c_@@_props_root_tl .initial:o } #1
  { \exp_args:No \@@_initialise:n {#1} }
\cs_new_protected:cpn { \c_@@_props_root_tl .initial:x } #1
  { \exp_args:Nx \@@_initialise:n {#1} }
%    \end{macrocode}
% \end{macro}
%
% \begin{macro}{.int_set:N, .int_set:c}
% \begin{macro}{.int_gset:N, .int_gset:c}
%   Setting a variable is very easy: just pass the data along.
%    \begin{macrocode}
\cs_new_protected:cpn { \c_@@_props_root_tl .int_set:N } #1
  { \@@_variable_set:NnnN #1 { int } { } n }
\cs_new_protected:cpn { \c_@@_props_root_tl .int_set:c } #1
  { \@@_variable_set:cnnN {#1} { int } { } n }
\cs_new_protected:cpn { \c_@@_props_root_tl .int_gset:N } #1
  { \@@_variable_set:NnnN #1 { int } { g } n }
\cs_new_protected:cpn { \c_@@_props_root_tl .int_gset:c } #1
  { \@@_variable_set:cnnN {#1} { int } { g } n }
%    \end{macrocode}
% \end{macro}
% \end{macro}
%
% \begin{macro}{.meta:n}
%   Making a meta is handled internally.
%    \begin{macrocode}
\cs_new_protected:cpn { \c_@@_props_root_tl .meta:n } #1
  { \@@_meta_make:n {#1} }
%    \end{macrocode}
% \end{macro}
%
% \begin{macro}{.meta:nn}
%   Meta with path: potentially lots of variants, but for the moment
%   no so many defined.
%    \begin{macrocode}
\cs_new_protected:cpn { \c_@@_props_root_tl .meta:nn } #1
  { \@@_meta_make:nn #1 }
%    \end{macrocode}
% \end{macro}
%
% \begin{macro}{.multichoice:}
% \begin{macro}
%   {
%     .multichoices:nn, .multichoices:Vn, .multichoices:on,
%       .multichoices:xn,
%   }
%   The same idea as \texttt{.choice:} and \texttt{.choices:nn}, but
%   where more than one choice is allowed.
%    \begin{macrocode}
\cs_new_protected_nopar:cpn { \c_@@_props_root_tl .multichoice: }
  { \@@_multichoice_make: }
\cs_new_protected:cpn { \c_@@_props_root_tl .multichoices:nn } #1
  { \@@_multichoices_make:nn #1 }
\cs_new_protected:cpn { \c_@@_props_root_tl .multichoices:Vn } #1
  { \exp_args:NV \@@_multichoices_make:nn #1 }
\cs_new_protected:cpn { \c_@@_props_root_tl .multichoices:on } #1
  { \exp_args:No \@@_multichoices_make:nn #1 }
\cs_new_protected:cpn { \c_@@_props_root_tl .multichoices:xn } #1
  { \exp_args:Nx \@@_multichoices_make:nn #1 }
%    \end{macrocode}
% \end{macro}
% \end{macro}
%
% \begin{macro}{.skip_set:N, .skip_set:c}
% \begin{macro}{.skip_gset:N, .skip_gset:c}
%   Setting a variable is very easy: just pass the data along.
%    \begin{macrocode}
\cs_new_protected:cpn { \c_@@_props_root_tl .skip_set:N } #1
  { \@@_variable_set:NnnN #1 { skip } { } n }
\cs_new_protected:cpn { \c_@@_props_root_tl .skip_set:c } #1
  { \@@_variable_set:cnnN {#1} { skip } { } n }
\cs_new_protected:cpn { \c_@@_props_root_tl .skip_gset:N } #1
  { \@@_variable_set:NnnN #1 { skip } { g } n }
\cs_new_protected:cpn { \c_@@_props_root_tl .skip_gset:c } #1
  { \@@_variable_set:cnnN {#1} { skip } { g } n }
%    \end{macrocode}
% \end{macro}
% \end{macro}
%
%
% \begin{macro}{.tl_set:N, .tl_set:c}
% \begin{macro}{.tl_gset:N, .tl_gset:c}
% \begin{macro}{.tl_set_x:N, .tl_set_x:c}
% \begin{macro}{.tl_gset_x:N, .tl_gset_x:c}
%   Setting a variable is very easy: just pass the data along.
%    \begin{macrocode}
\cs_new_protected:cpn { \c_@@_props_root_tl .tl_set:N } #1
  { \@@_variable_set:NnnN #1 { tl } { } n }
\cs_new_protected:cpn { \c_@@_props_root_tl .tl_set:c } #1
  { \@@_variable_set:cnnN {#1} { tl } { } n }
\cs_new_protected:cpn { \c_@@_props_root_tl .tl_set_x:N } #1
  { \@@_variable_set:NnnN #1 { tl } { } x }
\cs_new_protected:cpn { \c_@@_props_root_tl .tl_set_x:c } #1
  { \@@_variable_set:cnnN {#1} { tl } { } x }
\cs_new_protected:cpn { \c_@@_props_root_tl .tl_gset:N } #1
  { \@@_variable_set:NnnN #1 { tl } { g } n }
\cs_new_protected:cpn { \c_@@_props_root_tl .tl_gset:c } #1
  { \@@_variable_set:cnnN {#1} { tl } { g } n }
\cs_new_protected:cpn { \c_@@_props_root_tl .tl_gset_x:N } #1
  { \@@_variable_set:NnnN #1 { tl } { g } x }
\cs_new_protected:cpn { \c_@@_props_root_tl .tl_gset_x:c } #1
  { \@@_variable_set:cnnN {#1} { tl } { g } x }
%    \end{macrocode}
% \end{macro}
% \end{macro}
% \end{macro}
% \end{macro}
%
% \begin{macro}{.value_forbidden:}
% \begin{macro}{.value_required:}
%   These are very similar, so both call the same function.
%    \begin{macrocode}
\cs_new_protected_nopar:cpn { \c_@@_props_root_tl .value_forbidden: }
  { \@@_value_requirement:n { forbidden } }
\cs_new_protected_nopar:cpn { \c_@@_props_root_tl .value_required: }
  { \@@_value_requirement:n { required } }
%    \end{macrocode}
% \end{macro}
% \end{macro}
%
% \subsection{Setting keys}
%
% \begin{macro}{\keys_set:nn, \keys_set:nV, \keys_set:nv, \keys_set:no}
% \begin{macro}[aux]{\@@_set:nnn, \@@_set:onn}
%   A simple wrapper again.
%    \begin{macrocode}
\cs_new_protected_nopar:Npn \keys_set:nn
  { \@@_set:onn { \l_@@_module_tl } }
\cs_new_protected:Npn \@@_set:nnn #1#2#3
  {
    \tl_set:Nx \l_@@_module_tl { \tl_to_str:n {#2} }
    \keyval_parse:NNn \@@_set_elt:n \@@_set_elt:nn {#3}
    \tl_set:Nn \l_@@_module_tl {#1}
  }
\cs_generate_variant:Nn \keys_set:nn { nV , nv , no }
\cs_generate_variant:Nn \@@_set:nnn { o }
%    \end{macrocode}
% \end{macro}
% \end{macro}
%
% \begin{macro}
%   {
%     \keys_set_known:nnN, \keys_set_known:nVN,
%     \keys_set_known:nvN, \keys_set_known:noN
%   }
% \begin{macro}[aux]{\@@_set_known:nnnN, \@@_set_known:onnN}
% \begin{macro}
%   {
%     \keys_set_known:nn, \keys_set_known:nV,
%     \keys_set_known:nv, \keys_set_known:no
%   }
%   Setting known keys simply means setting the appropriate flag, then
%   running the standard code. To allow for nested setting, any existing
%   value of \cs{l_@@_unused_clist} is saved on the stack and reset
%   afterwards.Note that for speed/simplicity reasons we use a \texttt{tl}
%   operation to set the \texttt{clist} here!
%    \begin{macrocode}
\cs_new_protected_nopar:Npn \keys_set_known:nnN
  { \@@_set_known:onnN \l_@@_unused_clist }
\cs_generate_variant:Nn \keys_set_known:nnN { nV , nv , no }
\cs_new_protected:Npn \@@_set_known:nnnN #1#2#3#4
  {
    \clist_clear:N \l_@@_unused_clist
    \keys_set_known:nn {#2} {#3}
    \tl_set:Nx #4 { \exp_not:o { \l_@@_unused_clist } }
    \tl_set:Nn \l_@@_unused_clist {#1}
  }
\cs_generate_variant:Nn \@@_set_known:nnnN { o }
\cs_new_protected:Npn \keys_set_known:nn #1#2
  {
    \bool_set_true:N \l_@@_only_known_bool
    \keys_set:nn {#1} {#2}
    \bool_set_false:N \l_@@_only_known_bool
  }
\cs_generate_variant:Nn \keys_set_known:nn { nV , nv , no }
%    \end{macrocode}
% \end{macro}
% \end{macro}
% \end{macro}
%
% \begin{macro}
%   {
%     \keys_set_filter:nnnN, \keys_set_filter:nnVN, \keys_set_filter:nnvN
%       \keys_set_filter:nnoN
%   }
% \begin{macro}[aux]{\@@_set_filter:nnnnN, \@@_set_filter:onnnN}
% \begin{macro}
%   {
%     \keys_set_filter:nnn, \keys_set_filter:nnV, \keys_set_filter:nnv
%       \keys_set_filter:nno
%   }
% \begin{macro}
%   {
%     \keys_set_groups:nnn, \keys_set_groups:nnV, \keys_set_groups:nnv
%       \keys_set_groups:nno
%   }
%   The idea of setting keys in a selective manner again uses flags
%   wrapped around the basic code. The comments on \cs{keys_set_known:nnN}
%   also apply here.
%    \begin{macrocode}
\cs_new_protected_nopar:Npn \keys_set_filter:nnnN
  {  \@@_set_filter:onnnN \l_@@_unused_clist }
\cs_generate_variant:Nn \keys_set_filter:nnnN { nnV , nnv , nno }
\cs_new_protected:Npn \@@_set_filter:nnnnN #1#2#3#4#5
  {
    \clist_clear:N \l_@@_unused_clist
    \keys_set_filter:nnn {#2} {#3} {#4}
    \tl_set:Nx #5 { \exp_not:o { \l_@@_unused_clist } }
    \tl_set:Nn \l_@@_unused_clist {#1}
  }
\cs_generate_variant:Nn \@@_set_filter:nnnnN { o }
\cs_new_protected:Npn \keys_set_filter:nnn #1#2#3
  {
    \bool_set_true:N \l_@@_selective_bool
    \bool_set_true:N \l_@@_filtered_bool
    \seq_set_from_clist:Nn \l_@@_selective_seq {#2}
    \keys_set:nn {#1} {#3}
    \bool_set_false:N \l_@@_selective_bool
  }
\cs_generate_variant:Nn \keys_set_filter:nnn { nnV , nnv , nno }
\cs_new_protected:Npn \keys_set_groups:nnn #1#2#3
  {
    \bool_set_true:N \l_@@_selective_bool
    \bool_set_false:N \l_@@_filtered_bool
    \seq_set_from_clist:Nn \l_@@_selective_seq {#2}
    \keys_set:nn {#1} {#3}
    \bool_set_false:N \l_@@_selective_bool
  }
\cs_generate_variant:Nn \keys_set_groups:nnn { nnV , nnv , nno }
%    \end{macrocode}
% \end{macro}
% \end{macro}
% \end{macro}
% \end{macro}
%
% \begin{macro}[int]{\@@_set_elt:n, \@@_set_elt:nn}
% \begin{macro}[aux]{\@@_set_elt_aux:nn}
% \begin{macro}[aux]{\@@_set_elt_aux:, \@@_set_elt_selective:}
%   A shared system once again. First, set the current path and add a
%   default if needed. There are then checks to see if the a value is
%   required or forbidden. If everything passes, move on to execute the
%   code.
%    \begin{macrocode}
\cs_new_protected:Npn \@@_set_elt:n #1
  {
    \bool_set_true:N \l_@@_no_value_bool
    \@@_set_elt_aux:nn {#1} { }
  }
\cs_new_protected:Npn \@@_set_elt:nn #1#2
  {
    \bool_set_false:N \l_@@_no_value_bool
    \@@_set_elt_aux:nn {#1} {#2}
  }
\cs_new_protected:Npn \@@_set_elt_aux:nn #1#2
  {
    \tl_set:Nx \l_keys_key_tl { \tl_to_str:n {#1} }
    \tl_set:Nx \l_keys_path_tl { \l_@@_module_tl / \l_keys_key_tl }
    \@@_value_or_default:n {#2}
    \bool_if:NTF \l_@@_selective_bool
      { \@@_set_elt_selective: }
      { \@@_set_elt_aux: }
  }
\cs_new_protected_nopar:Npn \@@_set_elt_aux:
  {
    \bool_if:nTF
      {
        \@@_if_value_p:n { required } &&
        \l_@@_no_value_bool
      }
      {
        \__msg_kernel_error:nnx { kernel } { value-required }
          { \l_keys_path_tl }
      }
      {
        \bool_if:nTF
          {
              \@@_if_value_p:n { forbidden } &&
            ! \l_@@_no_value_bool
          }
          {
            \__msg_kernel_error:nnxx { kernel } { value-forbidden }
              { \l_keys_path_tl } { \l_keys_value_tl }
          }
          { \@@_execute: }
      }
  }
%    \end{macrocode}
%  If selective setting is active, there are a number of possible sub-cases
%  to consider. The key name may not be known at all or if it is, it may not
%  have any groups assigned. There is then the question of whether the
%  selection is opt-in or opt-out.
%    \begin{macrocode}
\cs_new_protected_nopar:Npn \@@_set_elt_selective:
  {
    \prop_if_exist:cTF { \c_@@_info_root_tl \l_keys_path_tl }
      {
        \prop_get:cnNTF { \c_@@_info_root_tl \l_keys_path_tl }
          { groups } \l_@@_groups_clist
          { \@@_check_groups: }
          {
            \bool_if:NTF \l_@@_filtered_bool
              { \@@_set_elt_aux: }
              { \@@_store_unused: }
          }
      }
      {
        \bool_if:NTF \l_@@_filtered_bool
          { \@@_set_elt_aux: }
          { \@@_store_unused: }
      }
  }
%    \end{macrocode}
%    In the case where selective setting requires a comparison of the list
%    of groups which apply to a key with the list of those which have been
%    set active. That requires two mappings, and again a different outcome
%    depending on whether opt-in or opt-out is set.
%    \begin{macrocode}
\cs_new_protected_nopar:Npn \@@_check_groups:
  {
    \bool_set_false:N \l_@@_tmp_bool
    \seq_map_inline:Nn \l_@@_selective_seq
      {
        \clist_map_inline:Nn \l_@@_groups_clist
          {
            \str_if_eq:nnT {##1} {####1}
              {
                \bool_set_true:N \l_@@_tmp_bool
                \clist_map_break:n { \seq_map_break: }
              }
          }
      }
    \bool_if:NTF \l_@@_tmp_bool
      {
        \bool_if:NTF \l_@@_filtered_bool
          { \@@_store_unused: }
          { \@@_set_elt_aux: }
      }
      {
        \bool_if:NTF \l_@@_filtered_bool
          { \@@_set_elt_aux: }
          { \@@_store_unused: }
      }
  }
%    \end{macrocode}
% \end{macro}
% \end{macro}
% \end{macro}
%
% \begin{macro}[int]{\@@_value_or_default:n}
%   If a value is given, return it as |#1|, otherwise send a default if
%   available.
%    \begin{macrocode}
\cs_new_protected:Npn \@@_value_or_default:n #1
  {
    \bool_if:NTF \l_@@_no_value_bool
      {
        \prop_get:cnNF { \c_@@_info_root_tl \l_keys_path_tl }
          { default } \l_keys_value_tl
          { \tl_clear:N \l_keys_value_tl }
      }
      { \tl_set:Nn \l_keys_value_tl {#1} }
  }
%    \end{macrocode}
% \end{macro}
%
% \begin{macro}[int]{\@@_if_value_p:n}
%   To test if a value is required or forbidden. A simple check for
%   the existence of the appropriate marker.
%    \begin{macrocode}
\prg_new_conditional:Npnn \@@_if_value:n #1 { p }
  {
    \prop_if_exist:cTF { \c_@@_info_root_tl \l_keys_path_tl }
      {
        \prop_if_in:cnTF { \c_@@_info_root_tl \l_keys_path_tl } {#1}
          { \prg_return_true: }
          { \prg_return_false: }
      }
      { \prg_return_false: }
  }
%    \end{macrocode}
% \end{macro}
%
% \begin{macro}[int]{\@@_execute:}
% \begin{macro}[aux]{\@@_execute_unknown:}
% \begin{macro}[aux, EXP]{\@@_execute:nn}
% \begin{macro}[aux]{\@@_store_unused:}
%   Actually executing a key is done in two parts. First, look for the
%   key itself, then look for the \texttt{unknown} key with the same
%   path. If both of these fail, complain. What exactly happens if a key
%   is unknown depends on whether unknown keys are being skipped or if
%   an error should be raised.
%    \begin{macrocode}
\cs_new_protected_nopar:Npn \@@_execute:
  { \@@_execute:nn { \l_keys_path_tl } { \@@_execute_unknown: } }
\cs_new_protected_nopar:Npn \@@_execute_unknown:
  {
    \bool_if:NTF \l_@@_only_known_bool
      { \@@_store_unused: }
      {
        \@@_execute:nn { \l_@@_module_tl / unknown }
          {
            \__msg_kernel_error:nnxx { kernel } { key-unknown }
              { \l_keys_path_tl } { \l_@@_module_tl }
          }
      }
  }
\cs_new:Npn \@@_execute:nn #1#2
  {
    \cs_if_exist:cTF { \c_@@_code_root_tl #1 }
      {
        \exp_args:Nc \exp_args:No { \c_@@_code_root_tl #1 }
          \l_keys_value_tl
      }
      {#2}
  }
\cs_new_protected_nopar:Npn \@@_store_unused:
  {
    \clist_put_right:Nx \l_@@_unused_clist
      {
        \exp_not:o \l_keys_key_tl
        \bool_if:NF \l_@@_no_value_bool
          { = { \exp_not:o \l_keys_value_tl } }
      }
  }
%    \end{macrocode}
% \end{macro}
% \end{macro}
% \end{macro}
% \end{macro}
%
% \begin{macro}[int, EXP]{\@@_choice_find:n}
% \begin{macro}[int, EXP]{\@@_multichoice_find:n}
%   Executing a choice has two parts. First, try the choice given, then
%   if that fails call the unknown key. That will exist, as it is created
%   when a choice is first made. So there is no need for any escape code.
%   For multiple choices, the same code ends up used in a mapping.
%    \begin{macrocode}
\cs_new:Npn \@@_choice_find:n #1
  {
    \@@_execute:nn { \l_keys_path_tl / \tl_to_str:n {#1} }
      { \@@_execute:nn { \l_keys_path_tl / unknown } { } }
  }
\cs_new:Npn \@@_multichoice_find:n #1
  { \clist_map_function:nN {#1} \@@_choice_find:n }
%    \end{macrocode}
% \end{macro}
% \end{macro}
%
% \subsection{Utilities}
%
% \begin{macro}[EXP,pTF]{\keys_if_exist:nn}
% A utility for others to see if a key exists.
%    \begin{macrocode}
\prg_new_conditional:Npnn \keys_if_exist:nn #1#2 { p , T , F , TF }
  {
    \cs_if_exist:cTF { \c_@@_code_root_tl \tl_to_str:n { #1 / #2 } }
      { \prg_return_true: }
      { \prg_return_false: }
  }
%    \end{macrocode}
% \end{macro}
%
% \begin{macro}[EXP,pTF]{\keys_if_choice_exist:nnn}
%   Just an alternative view on \cs{keys_if_exist:nn(TF)}.
%    \begin{macrocode}
\prg_new_conditional:Npnn \keys_if_choice_exist:nnn #1#2#3
  { p , T , F , TF }
  {
    \cs_if_exist:cTF { \c_@@_code_root_tl \tl_to_str:n { #1 / #2 / #3 } }
      { \prg_return_true: }
      { \prg_return_false: }
  }
%    \end{macrocode}
% \end{macro}
%
% \begin{macro}{\keys_show:nn}
%   Showing a key is just a question of using the correct name.
%    \begin{macrocode}
\cs_new_protected:Npn \keys_show:nn #1#2
  { \cs_show:c { \c_@@_code_root_tl #1 / \tl_to_str:n {#2} } }
%    \end{macrocode}
% \end{macro}
%
% \subsection{Messages}
%
% For when there is a need to complain.
%    \begin{macrocode}
\__msg_kernel_new:nnnn { kernel } { boolean-values-only }
  { Key~'#1'~accepts~boolean~values~only. }
  { The~key~'#1'~only~accepts~the~values~'true'~and~'false'. }
\__msg_kernel_new:nnnn { kernel } { choice-unknown }
  { Choice~'#2'~unknown~for~key~'#1'. }
  {
    The~key~'#1'~takes~a~limited~number~of~values.\\
    The~input~given,~'#2',~is~not~on~the~list~accepted.
  }
\__msg_kernel_new:nnnn { kernel } { key-choice-unknown }
  { Key~'#1'~accepts~only~a~fixed~set~of~choices. }
  {
    The~key~'#1'~only~accepts~predefined~values,~
    and~'#2'~is~not~one~of~these.
  }
\__msg_kernel_new:nnnn { kernel } { key-no-property }
  { No~property~given~in~definition~of~key~'#1'. }
  {
    \c__msg_coding_error_text_tl
    Inside~\keys_define:nn  each~key~name~
    needs~a~property:  \\ \\
    \iow_indent:n { #1 .<property> } \\ \\
    LaTeX~did~not~find~a~'.'~to~indicate~the~start~of~a~property.
  }
\__msg_kernel_new:nnnn { kernel } { key-unknown }
  { The~key~'#1'~is~unknown~and~is~being~ignored. }
  {
    The~module~'#2'~does~not~have~a~key~called~#1'.\\
    Check~that~you~have~spelled~the~key~name~correctly.
  }
\__msg_kernel_new:nnnn { kernel } { nested-choice-key }
  { Attempt~to~define~'#1'~as~a~nested~choice~key. }
  {
    The~key~'#1'~cannot~be~defined~as~a~choice~as~the~parent~key~'#2'~is~
    itself~a~choice.
  }
\__msg_kernel_new:nnnn { kernel } { property-requires-value }
  { The~property~'#1'~requires~a~value. }
  {
    \c__msg_coding_error_text_tl
    LaTeX~was~asked~to~set~property~'#1'~for~key~'#2'.\\
    No~value~was~given~for~the~property,~and~one~is~required.
  }
\__msg_kernel_new:nnnn { kernel } { property-unknown }
  { The~key~property~'#1'~is~unknown. }
  {
    \c__msg_coding_error_text_tl
    LaTeX~has~been~asked~to~set~the~property~'#1'~for~key~'#2':~
    this~property~is~not~defined.
  }
\__msg_kernel_new:nnnn { kernel } { value-forbidden }
  { The~key~'#1'~does~not~taken~a~value. }
  {
    The~key~'#1'~should~be~given~without~a~value.\\
    LaTeX~will~ignore~the~given~value~'#2'.
  }
\__msg_kernel_new:nnnn { kernel } { value-required }
  { The~key~'#1'~requires~a~value. }
  {
    The~key~'#1'~must~have~a~value.\\
    No~value~was~present:~the~key~will~be~ignored.
  }
%    \end{macrocode}
%
% \subsection{Deprecated functions}
%
% \begin{macro}[int]{\@@_choice_code_store:n, \@@_choice_code_store:x}
% \begin{macro}{.choice_code:n, .choice_code:x}
% \begin{macro}[int]{\@@_choices_generate:n}
% \begin{macro}[aux]{\@@_choices_generate_aux:n}
% \begin{macro}{.generate_choices:n}
%   Deprecated on 2013-07-09.
%    \begin{macrocode}
\cs_new_protected:Npn \@@_choice_code_store:n #1
  {
    \cs_if_exist:cF
      { \c_@@_info_root_tl \l_keys_path_tl .choice~code }
      {
        \tl_new:c
          { \c_@@_info_root_tl \l_keys_path_tl .choice~code }
      }
    \tl_set:cn { \c_@@_info_root_tl \l_keys_path_tl .choice~code }
      {#1}
  }
\cs_generate_variant:Nn \@@_choice_code_store:n { x }
\cs_new_protected:cpn { \c_@@_props_root_tl .choice_code:n } #1
  { \@@_choice_code_store:n {#1} }
\cs_new_protected:cpn { \c_@@_props_root_tl .choice_code:x } #1
  { \@@_choice_code_store:x {#1} }
\cs_new_protected:Npn \@@_choices_generate:n #1
  {
    \cs_if_exist:cTF
      { \c_@@_info_root_tl \l_keys_path_tl .choice~code }
      {
        \@@_choice_make:
        \int_zero:N \l_keys_choice_int
        \clist_map_function:nN {#1} \@@_choices_generate_aux:n
      }
      {
        \__msg_kernel_error:nnx { kernel }
          { generate-choices-before-code } { \l_keys_path_tl }
      }
  }
\cs_new_protected:Npn \@@_choices_generate_aux:n #1
  {
    \int_incr:N \l_keys_choice_int
    \@@_cmd_set:nx { \l_keys_path_tl / #1 }
      {
        \tl_set:Nn \exp_not:N \l_keys_choice_tl {#1}
        \int_set:Nn \exp_not:N \l_keys_choice_int
          { \int_use:N \l_keys_choice_int }
        \exp_not:v
          { \c_@@_info_root_tl \l_keys_path_tl .choice~code }
      }
  }
\__msg_kernel_new:nnnn { kernel } { generate-choices-before-code }
  { No~code~available~to~generate~choices~for~key~'#1'. }
  {
    \c__msg_coding_error_text_tl
    Before~using~.generate_choices:n~the~code~should~be~defined~
    with~'.choice_code:n'~or~'.choice_code:x'.
  }
\cs_new_protected:cpn { \c_@@_props_root_tl .generate_choices:n } #1
  { \@@_choices_generate:n {#1} }
%    \end{macrocode}
% \end{macro}
% \end{macro}
% \end{macro}
% \end{macro}
% \end{macro}
%
%    \begin{macrocode}
%</initex|package>
%    \end{macrocode}
%
%\end{implementation}
%
%\PrintIndex
