% \iffalse meta-comment
%
%% File: l3tl.dtx Copyright (C) 1990-2018 The LaTeX3 Project
%
% It may be distributed and/or modified under the conditions of the
% LaTeX Project Public License (LPPL), either version 1.3c of this
% license or (at your option) any later version.  The latest version
% of this license is in the file
%
%    https://www.latex-project.org/lppl.txt
%
% This file is part of the "l3kernel bundle" (The Work in LPPL)
% and all files in that bundle must be distributed together.
%
% -----------------------------------------------------------------------
%
% The development version of the bundle can be found at
%
%    https://github.com/latex3/latex3
%
% for those people who are interested.
%
%<*driver>
\documentclass[full,kernel]{l3doc}
\begin{document}
  \DocInput{\jobname.dtx}
\end{document}
%</driver>
% \fi
%
% \title{^^A
%   The \pkg{l3tl} package\\ Token lists^^A
% }
%
% \author{^^A
%  The \LaTeX3 Project\thanks
%    {^^A
%      E-mail:
%        \href{mailto:latex-team@latex-project.org}
%          {latex-team@latex-project.org}^^A
%    }^^A
% }
%
% \date{Released 2018-05-12}
%
% \maketitle
%
% \begin{documentation}
%
% \TeX{} works with tokens, and \LaTeX3 therefore provides a number of
% functions to deal with lists of tokens.  Token lists may be present
% directly in the argument to a function:
% \begin{verbatim}
%   \foo:n { a collection of \tokens }
% \end{verbatim}
% or may be stored in a so-called \enquote{token list variable}, which
% have the suffix \texttt{tl}: a token list variable can also be used as
% the argument to a function, for example
% \begin{verbatim}
%   \foo:N \l_some_tl
% \end{verbatim}
% In both cases, functions are available to test and manipulate the lists
% of tokens, and these have the module prefix \texttt{tl}.
% In many cases, functions which can be applied to token list variables
% are paired with similar functions for application to explicit lists
% of tokens: the two \enquote{views} of a token list are therefore collected
% together here.
%
% A token list (explicit, or stored in a variable) can be seen either
% as a list of \enquote{items},
% or a list of \enquote{tokens}. An item is whatever \cs{use:n} would
% grab as its argument: a single non-space token or a brace group,
% with optional leading explicit space characters (each item is thus
% itself a token list). A token is either a normal \texttt{N} argument,
% or \verb*| |, |{|, or |}| (assuming normal \TeX{} category codes).
% Thus for example
% \begin{verbatim}
%   { Hello } ~ world
% \end{verbatim}
% contains six items (\texttt{Hello}, \texttt{w}, \texttt{o}, \texttt{r},
% \texttt{l} and \texttt{d}), but thirteen tokens (|{|, \texttt{H}, \texttt{e},
% \texttt{l}, \texttt{l}, \texttt{o}, |}|, \verb*| |, \texttt{w}, \texttt{o},
% \texttt{r}, \texttt{l} and \texttt{d}).
% Functions which act on items are often faster than their analogue acting
% directly on tokens.
%
% \section{Creating and initialising token list variables}
%
% \begin{function}{\tl_new:N, \tl_new:c}
%   \begin{syntax}
%     \cs{tl_new:N} \meta{tl~var}
%   \end{syntax}
%   Creates a new \meta{tl~var} or raises an error if the
%   name is already taken. The declaration is global. The
%   \meta{tl~var} is initially empty.
% \end{function}
%
% \begin{function}{\tl_const:Nn, \tl_const:Nx, \tl_const:cn, \tl_const:cx}
%   \begin{syntax}
%     \cs{tl_const:Nn} \meta{tl~var} \Arg{token list}
%   \end{syntax}
%   Creates a new constant \meta{tl~var} or raises an error
%   if the name is already taken. The value of the
%   \meta{tl~var} is set globally to the \meta{token list}.
% \end{function}
%
% \begin{function}{\tl_clear:N, \tl_clear:c, \tl_gclear:N, \tl_gclear:c}
%   \begin{syntax}
%     \cs{tl_clear:N} \meta{tl~var}
%   \end{syntax}
%   Clears all entries from the \meta{tl~var}.
% \end{function}
%
% \begin{function}
%   {\tl_clear_new:N, \tl_clear_new:c, \tl_gclear_new:N, \tl_gclear_new:c}
%   \begin{syntax}
%     \cs{tl_clear_new:N} \meta{tl~var}
%   \end{syntax}
%   Ensures that the \meta{tl~var} exists globally by applying
%   \cs{tl_new:N} if necessary, then applies \cs[index=tl_clear:N]{tl_(g)clear:N} to leave
%   the \meta{tl~var} empty.
% \end{function}
%
% \begin{function}
%   {
%     \tl_set_eq:NN,  \tl_set_eq:cN,  \tl_set_eq:Nc,  \tl_set_eq:cc,
%     \tl_gset_eq:NN, \tl_gset_eq:cN, \tl_gset_eq:Nc, \tl_gset_eq:cc
%   }
%   \begin{syntax}
%     \cs{tl_set_eq:NN} \meta{tl~var_1} \meta{tl~var_2}
%   \end{syntax}
%   Sets the content of \meta{tl~var_1} equal to that of
%   \meta{tl~var_2}.
% \end{function}
%
% \begin{function}[added = 2012-05-18]
%   {
%     \tl_concat:NNN,  \tl_concat:ccc,
%     \tl_gconcat:NNN, \tl_gconcat:ccc
%   }
%   \begin{syntax}
%     \cs{tl_concat:NNN} \meta{tl~var_1} \meta{tl~var_2} \meta{tl~var_3}
%   \end{syntax}
%   Concatenates the content of \meta{tl~var_2} and \meta{tl~var_3}
%   together and saves the result in \meta{tl~var_1}. The \meta{tl~var_2}
%   is placed at the left side of the new token list.
% \end{function}
%
% \begin{function}[EXP, pTF, added=2012-03-03]{\tl_if_exist:N, \tl_if_exist:c}
%   \begin{syntax}
%     \cs{tl_if_exist_p:N} \meta{tl~var}
%     \cs{tl_if_exist:NTF} \meta{tl~var} \Arg{true code} \Arg{false code}
%   \end{syntax}
%   Tests whether the \meta{tl~var} is currently defined.  This does not
%   check that the \meta{tl~var} really is a token list variable.
% \end{function}
%
% \section{Adding data to token list variables}
%
% \begin{function}
%   {
%     \tl_set:Nn, \tl_set:NV, \tl_set:Nv, \tl_set:No, \tl_set:Nf, \tl_set:Nx,
%     \tl_set:cn, \tl_set:cV, \tl_set:cv, \tl_set:co, \tl_set:cf, \tl_set:cx,
%     \tl_gset:Nn, \tl_gset:NV, \tl_gset:Nv,
%     \tl_gset:No, \tl_gset:Nf, \tl_gset:Nx,
%     \tl_gset:cn, \tl_gset:cV, \tl_gset:cv,
%     \tl_gset:co, \tl_gset:cf, \tl_gset:cx
%   }
%   \begin{syntax}
%     \cs{tl_set:Nn} \meta{tl~var} \Arg{tokens}
%   \end{syntax}
%   Sets \meta{tl~var} to contain \meta{tokens},
%   removing any previous content from the variable.
% \end{function}
%
% \begin{function}
%   {
%     \tl_put_left:Nn,  \tl_put_left:NV,  \tl_put_left:No,  \tl_put_left:Nx,
%     \tl_put_left:cn,  \tl_put_left:cV,  \tl_put_left:co,  \tl_put_left:cx,
%     \tl_gput_left:Nn, \tl_gput_left:NV, \tl_gput_left:No, \tl_gput_left:Nx,
%     \tl_gput_left:cn, \tl_gput_left:cV, \tl_gput_left:co, \tl_gput_left:cx
%   }
%   \begin{syntax}
%     \cs{tl_put_left:Nn} \meta{tl~var} \Arg{tokens}
%   \end{syntax}
%   Appends \meta{tokens} to the left side of the current content of
%   \meta{tl~var}.
% \end{function}
%
% \begin{function}
%   {
%     \tl_put_right:Nn, \tl_put_right:NV, \tl_put_right:No, \tl_put_right:Nx,
%     \tl_put_right:cn, \tl_put_right:cV, \tl_put_right:co, \tl_put_right:cx,
%     \tl_gput_right:Nn, \tl_gput_right:NV, \tl_gput_right:No,
%     \tl_gput_right:Nx,
%     \tl_gput_right:cn, \tl_gput_right:cV, \tl_gput_right:co,
%     \tl_gput_right:cx
%   }
%   \begin{syntax}
%     \cs{tl_put_right:Nn} \meta{tl~var} \Arg{tokens}
%   \end{syntax}
%   Appends \meta{tokens} to the right side of the current content of
%   \meta{tl~var}.
% \end{function}
%
% \section{Modifying token list variables}
%
% \begin{function}[updated = 2011-08-11]
%   {
%     \tl_replace_once:Nnn,  \tl_replace_once:cnn,
%     \tl_greplace_once:Nnn, \tl_greplace_once:cnn
%   }
%   \begin{syntax}
%     \cs{tl_replace_once:Nnn} \meta{tl~var} \Arg{old tokens} \Arg{new tokens}
%   \end{syntax}
%   Replaces the first (leftmost) occurrence of \meta{old tokens} in the
%   \meta{tl~var} with \meta{new tokens}. \meta{Old tokens}
%   cannot contain |{|, |}| or |#|
%   (more precisely, explicit character tokens with category code $1$
%   (begin-group) or $2$ (end-group), and tokens with category code $6$).
% \end{function}
%
% \begin{function}[updated = 2011-08-11]
%   {
%     \tl_replace_all:Nnn, \tl_replace_all:cnn,
%     \tl_greplace_all:Nnn, \tl_greplace_all:cnn
%   }
%   \begin{syntax}
%     \cs{tl_replace_all:Nnn} \meta{tl~var} \Arg{old tokens} \Arg{new tokens}
%   \end{syntax}
%   Replaces all occurrences of \meta{old tokens} in the
%   \meta{tl~var} with \meta{new tokens}. \meta{Old tokens}
%   cannot contain |{|, |}| or |#|
%   (more precisely, explicit character tokens with category code $1$
%   (begin-group) or $2$ (end-group), and tokens with category code $6$).
%   As this function
%   operates from left to right, the pattern \meta{old tokens}
%   may remain after the replacement (see \cs{tl_remove_all:Nn}
%   for an example).
% \end{function}
%
% \begin{function}[updated = 2011-08-11]
%   {
%     \tl_remove_once:Nn,  \tl_remove_once:cn,
%     \tl_gremove_once:Nn, \tl_gremove_once:cn
%   }
%   \begin{syntax}
%     \cs{tl_remove_once:Nn} \meta{tl~var} \Arg{tokens}
%   \end{syntax}
%   Removes the first (leftmost) occurrence of \meta{tokens} from the
%   \meta{tl~var}. \meta{Tokens} cannot contain |{|, |}| or |#|
%   (more precisely, explicit character tokens with category code $1$
%   (begin-group) or $2$ (end-group), and tokens with category code $6$).
% \end{function}
%
% \begin{function}[updated = 2011-08-11]
%   {
%     \tl_remove_all:Nn,  \tl_remove_all:cn,
%     \tl_gremove_all:Nn, \tl_gremove_all:cn
%   }
%   \begin{syntax}
%     \cs{tl_remove_all:Nn} \meta{tl~var} \Arg{tokens}
%   \end{syntax}
%   Removes all occurrences of \meta{tokens} from the
%   \meta{tl~var}. \meta{Tokens} cannot contain |{|, |}| or |#|
%   (more precisely, explicit character tokens with category code $1$
%   (begin-group) or $2$ (end-group), and tokens with category code $6$).
%   As this function
%   operates from left to right, the pattern \meta{tokens}
%   may remain after the removal, for instance,
%   \begin{quote}
%     \cs{tl_set:Nn} \cs{l_tmpa_tl} |{abbccd}|
%     \cs{tl_remove_all:Nn} \cs{l_tmpa_tl} |{bc}|
%   \end{quote}
%   results in \cs{l_tmpa_tl} containing \texttt{abcd}.
% \end{function}
%
% \section{Reassigning token list category codes}
%
% These functions allow the rescanning of tokens: re-apply \TeX{}'s
% tokenization process to apply category codes different from those
% in force when the tokens were absorbed. Whilst this functionality is
% supported, it is often preferable to find alternative approaches
% to achieving outcomes rather than rescanning tokens (for example
% construction of token lists token-by-token with intervening category
% code changes).
%
% \begin{function}[updated = 2015-08-11]
%   {
%     \tl_set_rescan:Nnn,  \tl_set_rescan:Nno,  \tl_set_rescan:Nnx,
%     \tl_set_rescan:cnn,  \tl_set_rescan:cno,  \tl_set_rescan:cnx,
%     \tl_gset_rescan:Nnn, \tl_gset_rescan:Nno, \tl_gset_rescan:Nnx,
%     \tl_gset_rescan:cnn, \tl_gset_rescan:cno, \tl_gset_rescan:cnx
%   }
%   \begin{syntax}
%     \cs{tl_set_rescan:Nnn} \meta{tl~var} \Arg{setup} \Arg{tokens}
%   \end{syntax}
%   Sets \meta{tl~var} to contain \meta{tokens}, applying the category
%   code r\'{e}gime specified in the \meta{setup} before carrying out
%   the assignment. (Category codes applied to tokens not explicitly covered
%   by the \meta{setup} are those in force at the point of use of
%   \cs{tl_set_rescan:Nnn}.)
%   This allows the \meta{tl~var} to contain material
%   with category codes other than those that apply when \meta{tokens}
%   are absorbed.  The \meta{setup} is run within a group and may
%   contain any valid input, although only changes in category codes
%   are relevant. See also \cs{tl_rescan:nn}.
%   \begin{texnote}
%     The \meta{tokens} are first turned into a string (using
%     \cs{tl_to_str:n}).  If the string contains one or more characters
%     with character code \tn{newlinechar} (set equal to
%     \tn{endlinechar} unless that is equal to $32$, before the user
%     \meta{setup}), then it is split into lines at these characters,
%     then read as if reading multiple lines from a file, ignoring
%     spaces (catcode $10$) at the beginning and spaces and tabs
%     (character code $32$ or $9$) at the end of every line.
%     Otherwise, spaces (and tabs) are retained at both ends of the
%     single-line string, as if it appeared in the middle of a line
%     read from a file.
%   \end{texnote}
% \end{function}
%
% \begin{function}[updated = 2015-08-11]{\tl_rescan:nn}
%   \begin{syntax}
%     \cs{tl_rescan:nn} \Arg{setup} \Arg{tokens}
%   \end{syntax}
%   Rescans \meta{tokens} applying the category code r\'{e}gime
%   specified in the \meta{setup}, and leaves the resulting tokens in
%   the input stream. (Category codes applied to tokens not explicitly covered
%   by the \meta{setup} are those in force at the point of use of
%   \cs{tl_rescan:nn}.)
%   The \meta{setup} is run within a group and may
%   contain any valid input, although only changes in category codes
%   are relevant.  See also \cs{tl_set_rescan:Nnn}, which is more
%   robust than using \cs{tl_set:Nn} in the \meta{tokens} argument of
%   \cs{tl_rescan:nn}.
%   \begin{texnote}
%     The \meta{tokens} are first turned into a string (using
%     \cs{tl_to_str:n}).  If the string contains one or more characters
%     with character code \tn{newlinechar} (set equal to
%     \tn{endlinechar} unless that is equal to $32$, before the user
%     \meta{setup}), then it is split into lines at these characters,
%     then read as if reading multiple lines from a file, ignoring
%     spaces (catcode $10$) at the beginning and spaces and tabs
%     (character code $32$ or $9$) at the end of every line.
%     Otherwise, spaces (and tabs) are retained at both ends of the
%     single-line string, as if it appeared in the middle of a line
%     read from a file.
%   \end{texnote}
% \end{function}
%
% \section{Token list conditionals}
%
% \begin{function}[EXP,pTF]{\tl_if_blank:n, \tl_if_blank:V, \tl_if_blank:o}
%   \begin{syntax}
%     \cs{tl_if_blank_p:n} \Arg{token list}
%     \cs{tl_if_blank:nTF} \Arg{token list} \Arg{true code} \Arg{false code}
%   \end{syntax}
%   Tests if the \meta{token list} consists only of blank spaces
%   (\emph{i.e.}~contains no item). The test is \texttt{true} if
%   \meta{token list} is zero or more explicit space characters
%   (explicit tokens with character code~$32$ and category code~$10$),
%   and is \texttt{false} otherwise.
% \end{function}
%
% \begin{function}[EXP,pTF]{\tl_if_empty:N, \tl_if_empty:c}
%   \begin{syntax}
%     \cs{tl_if_empty_p:N} \meta{tl~var}
%     \cs{tl_if_empty:NTF} \meta{tl~var} \Arg{true code} \Arg{false code}
%   \end{syntax}
%   Tests if the \meta{token list variable} is entirely empty
%   (\emph{i.e.}~contains no tokens at all).
% \end{function}
%
% \begin{function}[added = 2012-05-24, updated = 2012-06-05, EXP,pTF]
%   {\tl_if_empty:n, \tl_if_empty:V, \tl_if_empty:o}
%   \begin{syntax}
%     \cs{tl_if_empty_p:n} \Arg{token list}
%     \cs{tl_if_empty:nTF} \Arg{token list} \Arg{true code} \Arg{false code}
%   \end{syntax}
%   Tests if the \meta{token list} is entirely empty
%   (\emph{i.e.}~contains no tokens at all).
% \end{function}
%
% \begin{function}[EXP,pTF]
%   {\tl_if_eq:NN, \tl_if_eq:Nc, \tl_if_eq:cN, \tl_if_eq:cc}
%   \begin{syntax}
%     \cs{tl_if_eq_p:NN} \meta{tl~var_1} \meta{tl~var_2}
%     \cs{tl_if_eq:NNTF} \meta{tl~var_1} \meta{tl~var_2} \Arg{true code} \Arg{false code}
%   \end{syntax}
%   Compares the content of two \meta{token list variables} and
%   is logically \texttt{true} if the two contain the same list of
%   tokens (\emph{i.e.}~identical in both the list of characters they
%   contain and the category codes of those characters). Thus for example
%   \begin{verbatim}
%     \tl_set:Nn \l_tmpa_tl { abc }
%     \tl_set:Nx \l_tmpb_tl { \tl_to_str:n { abc } }
%     \tl_if_eq:NNTF \l_tmpa_tl \l_tmpb_tl { true } { false }
%   \end{verbatim}
%   yields \texttt{false}.
% \end{function}
%
% \begin{function}[TF]{\tl_if_eq:nn}
%   \begin{syntax}
%     \cs{tl_if_eq:nnTF} \Arg{token list_1} \Arg{token list_2} \Arg{true code} \Arg{false code}
%   \end{syntax}
%   Tests if \meta{token list_1} and \meta{token list_2} contain the
%   same list of tokens, both in respect of character codes and category
%   codes.
% \end{function}
%
% \begin{function}[TF]{\tl_if_in:Nn, \tl_if_in:cn}
%   \begin{syntax}
%     \cs{tl_if_in:NnTF} \meta{tl~var} \Arg{token list} \Arg{true code} \Arg{false code}
%   \end{syntax}
%   Tests if the \meta{token list} is found in the content of the
%   \meta{tl~var}. The \meta{token list} cannot contain
%   the tokens |{|, |}| or |#|
%   (more precisely, explicit character tokens with category code $1$
%   (begin-group) or $2$ (end-group), and tokens with category code $6$).
% \end{function}
%
% \begin{function}[TF]
%   {\tl_if_in:nn, \tl_if_in:Vn, \tl_if_in:on, \tl_if_in:no}
%   \begin{syntax}
%     \cs{tl_if_in:nnTF} \Arg{token list_1} \Arg{token list_2} \Arg{true code} \Arg{false code}
%   \end{syntax}
%   Tests if \meta{token list_2} is found inside \meta{token list_1}.
%   The \meta{token list_2} cannot contain the tokens |{|, |}| or |#|
%   (more precisely, explicit character tokens with category code $1$
%   (begin-group) or $2$ (end-group), and tokens with category code $6$).
% \end{function}
%
% \begin{function}[added = 2017-11-14, EXP,pTF]{\tl_if_novalue:n}
%   \begin{syntax}
%     \cs{tl_if_novalue_p:n} \Arg{token list}
%     \cs{tl_if_novalue:nTF} \Arg{token list} \Arg{true code} \Arg{false code}
%   \end{syntax}
%   Tests if the \meta{token list} is exactly equal to the special
%   \cs{c_novalue_tl} marker. This function is intended to allow construction
%   of flexible document interface structures in which missing optional
%   arguments are detected.
% \end{function}
%
% \begin{function}[updated = 2011-08-13, EXP,pTF]
%   {\tl_if_single:N, \tl_if_single:c}
%   \begin{syntax}
%     \cs{tl_if_single_p:N} \meta{tl~var}
%     \cs{tl_if_single:NTF} \meta{tl~var} \Arg{true code} \Arg{false code}
%   \end{syntax}
%   Tests if the content of the \meta{tl~var} consists of a single item,
%   \emph{i.e.}~is a single normal token (neither an explicit space
%   character nor a begin-group character) or a single brace group,
%   surrounded by optional spaces on both sides. In other words, such a
%   token list has token count $1$ according to \cs{tl_count:N}.
% \end{function}
%
% \begin{function}[updated = 2011-08-13, EXP,pTF]{\tl_if_single:n}
%   \begin{syntax}
%     \cs{tl_if_single_p:n} \Arg{token list}
%     \cs{tl_if_single:nTF} \Arg{token list} \Arg{true code} \Arg{false code}
%   \end{syntax}
%   Tests if the \meta{token list} has exactly one item, \emph{i.e.}~is
%   a single normal token (neither an explicit space character nor a
%   begin-group character) or a single brace group, surrounded by
%   optional spaces on both sides. In other words, such a token list has
%   token count $1$ according to \cs{tl_count:n}.
% \end{function}
%
% \begin{function}[added = 2013-07-24, EXP, noTF]{\tl_case:Nn, \tl_case:cn}
%   \begin{syntax}
%     \cs{tl_case:NnTF} \meta{test token list variable} \\
%     ~~"{" \\
%     ~~~~\meta{token list variable case_1} \Arg{code case_1} \\
%     ~~~~\meta{token list variable case_2} \Arg{code case_2} \\
%     ~~~~\ldots \\
%     ~~~~\meta{token list variable case_n} \Arg{code case_n} \\
%     ~~"}" \\
%     ~~\Arg{true code}
%     ~~\Arg{false code}
%   \end{syntax}
%   This function compares the \meta{test token list variable} in turn
%   with each of the \meta{token list variable cases}. If the two
%   are equal (as described for \cs{tl_if_eq:NNTF})
%   then the associated \meta{code} is left in the input
%   stream and other cases are discarded. If any of the
%   cases are matched, the \meta{true code} is also inserted into the
%   input stream (after the code for the appropriate case), while if none
%   match then the \meta{false code} is inserted. The function
%   \cs{tl_case:Nn}, which does nothing if there is no match, is also
%   available.
% \end{function}
%
% \section{Mapping to token lists}
%
% \begin{function}[updated = 2012-06-29, rEXP]
%   {\tl_map_function:NN, \tl_map_function:cN}
%   \begin{syntax}
%     \cs{tl_map_function:NN} \meta{tl~var} \meta{function}
%   \end{syntax}
%   Applies \meta{function} to every \meta{item} in the \meta{tl~var}.
%   The \meta{function} receives one argument for each iteration.
%   This may be a number of tokens if the \meta{item} was stored within
%   braces. Hence the \meta{function} should anticipate receiving
%   \texttt{n}-type arguments. See also \cs{tl_map_function:nN}.
% \end{function}
%
% \begin{function}[updated = 2012-06-29, rEXP]{\tl_map_function:nN}
%   \begin{syntax}
%     \cs{tl_map_function:nN} \Arg{token list} \meta{function}
%   \end{syntax}
%   Applies \meta{function} to every \meta{item} in the \meta{token list},
%   The \meta{function} receives one argument for each iteration.
%   This may be a number of tokens if the \meta{item} was stored within
%   braces. Hence the \meta{function} should anticipate receiving
%   \texttt{n}-type arguments. See also \cs{tl_map_function:NN}.
% \end{function}
%
% \begin{function}[updated = 2012-06-29]
%   {\tl_map_inline:Nn, \tl_map_inline:cn}
%   \begin{syntax}
%     \cs{tl_map_inline:Nn} \meta{tl~var} \Arg{inline function}
%   \end{syntax}
%   Applies the \meta{inline function} to every \meta{item} stored within the
%   \meta{tl~var}. The \meta{inline function} should consist of code which
%   receives the \meta{item} as |#1|. See also \cs{tl_map_function:NN}.
% \end{function}
%
% \begin{function}[updated = 2012-06-29]{\tl_map_inline:nn}
%   \begin{syntax}
%     \cs{tl_map_inline:nn} \Arg{token list} \Arg{inline function}
%   \end{syntax}
%   Applies the \meta{inline function} to every \meta{item} stored within the
%   \meta{token list}. The \meta{inline function}  should consist of code which
%   receives the \meta{item} as |#1|. See also \cs{tl_map_function:nN}.
% \end{function}
%
% \begin{function}[updated = 2012-06-29]
%   {\tl_map_variable:NNn, \tl_map_variable:cNn}
%   \begin{syntax}
%     \cs{tl_map_variable:NNn} \meta{tl~var} \meta{variable} \Arg{code}
%   \end{syntax}
%   Stores each \meta{item} of the \meta{tl~var} in turn in the (token
%   list) \meta{variable} and applies the \meta{code}.  The \meta{code}
%   will usually make use of the \meta{variable}, but this is not
%   enforced.  The assignments to the \meta{variable} are local.  See
%   also \cs{tl_map_inline:Nn}.
% \end{function}
%
% \begin{function}[updated = 2012-06-29]{\tl_map_variable:nNn}
%   \begin{syntax}
%     \cs{tl_map_variable:nNn} \Arg{token list} \meta{variable} \Arg{code}
%   \end{syntax}
%   Stores each \meta{item} of the \meta{token list} in turn in the
%   (token list) \meta{variable} and applies the \meta{code}.  The
%   \meta{code} will usually make use of the \meta{variable}, but this
%   is not enforced.  The assignments to the \meta{variable} are local.
%   See also \cs{tl_map_inline:nn}.
% \end{function}
%
% \begin{function}[updated = 2012-06-29, rEXP]{\tl_map_break:}
%   \begin{syntax}
%     \cs{tl_map_break:}
%   \end{syntax}
%   Used to terminate a \cs[no-index]{tl_map_\ldots} function before all
%   entries in the \meta{token list variable} have been processed. This
%   normally takes place within a conditional statement, for example
%   \begin{verbatim}
%     \tl_map_inline:Nn \l_my_tl
%       {
%         \str_if_eq:nnT { #1 } { bingo } { \tl_map_break: }
%         % Do something useful
%       }
%   \end{verbatim}
%   See also \cs{tl_map_break:n}.
%   Use outside of a \cs[no-index]{tl_map_\ldots} scenario leads to low
%   level \TeX{} errors.
%   \begin{texnote}
%     When the mapping is broken, additional tokens may be inserted
%     before the \meta{tokens} are
%     inserted into the input stream.
%     This depends on the design of the mapping function.
%   \end{texnote}
% \end{function}
%
% \begin{function}[updated = 2012-06-29, rEXP]{\tl_map_break:n}
%   \begin{syntax}
%     \cs{tl_map_break:n} \Arg{code}
%   \end{syntax}
%   Used to terminate a \cs[no-index]{tl_map_\ldots} function before all
%   entries in the \meta{token list variable} have been processed, inserting
%   the \meta{code} after the mapping has ended. This
%   normally takes place within a conditional statement, for example
%   \begin{verbatim}
%     \tl_map_inline:Nn \l_my_tl
%       {
%         \str_if_eq:nnT { #1 } { bingo }
%           { \tl_map_break:n { <code> } }
%         % Do something useful
%       }
%   \end{verbatim}
%   Use outside of a \cs[no-index]{tl_map_\ldots} scenario leads to low
%   level \TeX{} errors.
%   \begin{texnote}
%     When the mapping is broken, additional tokens may be inserted
%     before the \meta{code} is
%     inserted into the input stream.
%     This depends on the design of the mapping function.
%   \end{texnote}
% \end{function}
%
% \section{Using token lists}
%
% \begin{function}[EXP]{\tl_to_str:n, \tl_to_str:V}
%   \begin{syntax}
%     \cs{tl_to_str:n} \Arg{token list}
%   \end{syntax}
%   Converts the \meta{token list} to a \meta{string}, leaving the
%   resulting character tokens in the input stream. A \meta{string}
%   is a series of tokens with category code $12$ (other) with the exception
%   of spaces, which retain category code $10$ (space).
%   This function requires only a single expansion.
%   Its argument \emph{must} be braced.
%   \begin{texnote}
%     This is the \eTeX{} primitive \tn{detokenize}.
%     Converting a \meta{token list} to a \meta{string} yields a
%     concatenation of the string representations of every token in the
%     \meta{token list}.
%     The string representation of a control sequence is
%     \begin{itemize}
%       \item an escape character, whose character code is given by the
%         internal parameter \tn{escapechar}, absent if the
%         \tn{escapechar} is negative or greater than the largest
%         character code;
%       \item the control sequence name, as defined by \cs{cs_to_str:N};
%       \item a space, unless the control sequence name is a single
%         character whose category at the time of expansion of
%         \cs{tl_to_str:n} is not \enquote{letter}.
%     \end{itemize}
%     The string representation of an explicit character token is that
%     character, doubled in the case of (explicit) macro parameter
%     characters (normally |#|).
%     In particular, the string representation of a token list may
%     depend on the category codes in effect when it is evaluated, and
%     the value of the \tn{escapechar}: for instance |\tl_to_str:n {\a}|
%     normally produces the three character \enquote{backslash},
%     \enquote{lower-case a}, \enquote{space}, but it may also produce a
%     single \enquote{lower-case a} if the escape character is negative
%     and \texttt{a} is currently not a letter.
%   \end{texnote}
% \end{function}
%
% \begin{function}[EXP]{\tl_to_str:N, \tl_to_str:c}
%   \begin{syntax}
%     \cs{tl_to_str:N} \meta{tl~var}
%   \end{syntax}
%   Converts the content of the  \meta{tl~var} into a series of characters
%   with category code $12$ (other) with the exception of spaces, which
%   retain category code $10$ (space). This \meta{string} is then left
%   in the input stream. For low-level details, see the notes given for
%   \cs{tl_to_str:n}.
% \end{function}
%
% \begin{function}[EXP]{\tl_use:N, \tl_use:c}
%   \begin{syntax}
%     \cs{tl_use:N} \meta{tl~var}
%   \end{syntax}
%   Recovers the content of a \meta{tl~var} and places it
%   directly in the input stream. An error is raised if the variable
%   does not exist or if it is invalid. Note that it is possible to use
%   a \meta{tl~var} directly without an accessor function.
% \end{function}
%
% \section{Working with the content of token lists}
%
% \begin{function}[added = 2012-05-13, EXP]
%   {\tl_count:n, \tl_count:V, \tl_count:o}
%   \begin{syntax}
%     \cs{tl_count:n} \Arg{tokens}
%   \end{syntax}
%   Counts the number of \meta{items} in \meta{tokens} and leaves this
%   information in the input stream. Unbraced tokens count as one
%   element as do each token group (|{|\ldots|}|). This process
%   ignores any unprotected spaces within \meta{tokens}. See also
%   \cs{tl_count:N}. This function requires three expansions,
%   giving an \meta{integer denotation}.
% \end{function}
%
% \begin{function}[added = 2012-05-13, EXP]{\tl_count:N, \tl_count:c}
%   \begin{syntax}
%     \cs{tl_count:N} \meta{tl~var}
%   \end{syntax}
%   Counts the number of token groups in the \meta{tl~var}
%   and leaves this information in the input stream. Unbraced tokens
%   count as one element as do each token group (|{|\ldots|}|). This
%   process ignores any unprotected spaces within the \meta{tl~var}.
%   See also \cs{tl_count:n}. This function requires three expansions,
%   giving an \meta{integer denotation}.
% \end{function}
%
% \begin{function}[updated = 2012-01-08, EXP]
%   {\tl_reverse:n, \tl_reverse:V, \tl_reverse:o}
%   \begin{syntax}
%     \cs{tl_reverse:n} \Arg{token list}
%   \end{syntax}
%   Reverses the order of the \meta{items} in the \meta{token list},
%   so that \meta{item_1}\meta{item_2}\meta{item_3} \ldots \meta{item_n}
%   becomes \meta{item_n}\ldots \meta{item_3}\meta{item_2}\meta{item_1}.
%   This process preserves unprotected space within the
%   \meta{token list}. Tokens are not reversed within braced token
%   groups, which keep their outer set of braces.
%   In situations where performance is important,
%   consider \cs{tl_reverse_items:n}.
%   See also \cs{tl_reverse:N}.
%   \begin{texnote}
%     The result is returned within \tn{unexpanded}, which means that the token
%     list does not expand further when appearing in an \texttt{x}-type
%     argument expansion.
%   \end{texnote}
% \end{function}
%
% \begin{function}[updated = 2012-01-08]
%   {\tl_reverse:N, \tl_reverse:c, \tl_greverse:N, \tl_greverse:c}
%   \begin{syntax}
%     \cs{tl_reverse:N} \meta{tl~var}
%   \end{syntax}
%   Reverses the order of the \meta{items} stored in \meta{tl~var}, so
%   that \meta{item_1}\meta{item_2}\meta{item_3} \ldots \meta{item_n}
%   becomes \meta{item_n}\ldots \meta{item_3}\meta{item_2}\meta{item_1}.
%   This process preserves unprotected spaces within the
%   \meta{token list variable}. Braced token groups are copied without
%   reversing the order of tokens, but keep the outer set of braces.
%   See also \cs{tl_reverse:n}, and, for improved performance,
%   \cs{tl_reverse_items:n}.
% \end{function}
%
% \begin{function}[added = 2012-01-08, EXP]{\tl_reverse_items:n}
%   \begin{syntax}
%     \cs{tl_reverse_items:n} \Arg{token list}
%   \end{syntax}
%   Reverses the order of the \meta{items} stored in \meta{tl~var},
%   so that \Arg{item_1}\Arg{item_2}\Arg{item_3} \ldots \Arg{item_n}
%   becomes \Arg{item_n} \ldots{} \Arg{item_3}\Arg{item_2}\Arg{item_1}.
%   This process removes any unprotected space within the
%   \meta{token list}. Braced token groups are copied without
%   reversing the order of tokens, and keep the outer set of braces.
%   Items which are initially not braced are copied with braces in
%   the result. In cases where preserving spaces is important,
%   consider the slower function \cs{tl_reverse:n}.
%   \begin{texnote}
%     The result is returned within \tn{unexpanded}, which means that the token
%     list does not expand further when appearing in an \texttt{x}-type
%     argument expansion.
%   \end{texnote}
% \end{function}
%
% \begin{function}[added = 2011-07-09, updated = 2012-06-25, EXP]
%   {\tl_trim_spaces:n, \tl_trim_spaces:o}
%   \begin{syntax}
%     \cs{tl_trim_spaces:n} \Arg{token list}
%   \end{syntax}
%   Removes any leading and trailing explicit space characters
%   (explicit tokens with character code~$32$ and category code~$10$)
%   from the \meta{token list} and leaves the result in the input
%   stream.
%   \begin{texnote}
%     The result is returned within \tn{unexpanded}, which means that the token
%     list does not expand further when appearing in an \texttt{x}-type
%     argument expansion.
%   \end{texnote}
% \end{function}
%
% \begin{function}[added = 2018-04-12, EXP]
%   {\tl_trim_spaces_apply:nN, \tl_trim_spaces_apply:oN}
%   \begin{syntax}
%     \cs{tl_trim_spaces_apply:nN} \Arg{token list} \meta{function}
%   \end{syntax}
%   Removes any leading and trailing explicit space characters (explicit
%   tokens with character code~$32$ and category code~$10$) from the
%   \meta{token list} and passes the result to the \meta{function} as an
%   \texttt{n}-type argument.
% \end{function}
%
% \begin{function}[added = 2011-07-09]
%   {
%     \tl_trim_spaces:N,  \tl_trim_spaces:c,
%     \tl_gtrim_spaces:N, \tl_gtrim_spaces:c
%   }
%   \begin{syntax}
%     \cs{tl_trim_spaces:N} \meta{tl~var}
%   \end{syntax}
%   Removes any leading and trailing explicit space characters
%   (explicit tokens with character code~$32$ and category code~$10$)
%   from the content of the \meta{tl~var}. Note that this therefore
%   \emph{resets} the content of the variable.
% \end{function}
%
% \begin{function}[added = 2017-02-06]
%   {\tl_sort:Nn, \tl_sort:cn, \tl_gsort:Nn, \tl_gsort:cn}
%   \begin{syntax}
%     \cs{tl_sort:Nn} \meta{tl var} \Arg{comparison code}
%   \end{syntax}
%   Sorts the items in the \meta{tl var} according to the
%   \meta{comparison code}, and assigns the result to
%   \meta{tl var}. The details of sorting comparison are
%   described in Section~\ref{sec:l3sort:mech}.
% \end{function}
%
% \begin{function}[added = 2017-02-06, EXP]{\tl_sort:nN}
%   \begin{syntax}
%     \cs{tl_sort:nN} \Arg{token list} \meta{conditional}
%   \end{syntax}
%   Sorts the items in the \meta{token list}, using the
%   \meta{conditional} to compare items, and leaves the result in the
%   input stream.  The \meta{conditional} should have signature |:nnTF|,
%   and return \texttt{true} if the two items being compared should be
%   left in the same order, and \texttt{false} if the items should be
%   swapped. The details of sorting comparison are
%   described in Section~\ref{sec:l3sort:mech}.
%   \begin{texnote}
%     The result is returned within \cs{exp_not:n}, which means that the
%     token list does not expand further when appearing in an
%     \texttt{x}-type argument expansion.
%   \end{texnote}
% \end{function}
%
% \section{The first token from a token list}
%
% Functions which deal with either only the very first item (balanced
% text or single normal token) in a token list, or the remaining tokens.
%
% \begin{function}[updated = 2012-09-09, EXP]
%   {\tl_head:N, \tl_head:n, \tl_head:V, \tl_head:v, \tl_head:f}
%   \begin{syntax}
%     \cs{tl_head:n} \Arg{token list}
%   \end{syntax}
%   Leaves in the input stream the first \meta{item} in the
%   \meta{token list}, discarding the rest of the \meta{token list}.
%   All leading explicit space characters
%   (explicit tokens with character code~$32$ and category code~$10$)
%   are discarded; for example
%   \begin{verbatim}
%     \tl_head:n { abc }
%   \end{verbatim}
%   and
%   \begin{verbatim}
%     \tl_head:n { ~ abc }
%   \end{verbatim}
%   both leave |a| in the input stream. If the \enquote{head} is a
%   brace group, rather than a single token, the braces are removed, and
%   so
%   \begin{verbatim}
%     \tl_head:n { ~ { ~ ab } c }
%   \end{verbatim}
%   yields \verb*| ab|.
%   A blank \meta{token list} (see \cs{tl_if_blank:nTF}) results in
%   \cs{tl_head:n} leaving nothing in the input stream.
%   \begin{texnote}
%     The result is returned within \cs{exp_not:n}, which means that the token
%     list does not expand further when appearing in an \texttt{x}-type
%     argument expansion.
%   \end{texnote}
% \end{function}
%
% \begin{function}[EXP]{\tl_head:w}
%   \begin{syntax}
%     \cs{tl_head:w} \meta{token list} | { } | \cs{q_stop}
%   \end{syntax}
%   Leaves in the input stream the first \meta{item} in the
%   \meta{token list}, discarding the rest of the \meta{token list}.
%   All leading explicit space characters
%   (explicit tokens with character code~$32$ and category code~$10$)
%   are discarded.
%   A blank \meta{token list} (which consists only of space characters)
%   results in a low-level \TeX{} error, which may be avoided by the
%   inclusion of an empty group in the input (as shown), without the need
%   for an explicit test. Alternatively, \cs{tl_if_blank:nF} may be used to
%   avoid using the function with a \enquote{blank} argument.
%   This function requires only a single expansion, and thus is suitable for
%   use within an \texttt{o}-type expansion. In general, \cs{tl_head:n} should
%   be preferred if the number of expansions is not critical.
% \end{function}
%
% \begin{function}[updated = 2012-09-01, EXP]
%   {\tl_tail:N, \tl_tail:n, \tl_tail:V, \tl_tail:v, \tl_tail:f}
%   \begin{syntax}
%     \cs{tl_tail:n} \Arg{token list}
%   \end{syntax}
%   Discards all leading explicit space characters
%   (explicit tokens with character code~$32$ and category code~$10$)
%   and the first \meta{item} in the \meta{token list}, and leaves the
%   remaining tokens in the input stream.  Thus for example
%   \begin{verbatim}
%     \tl_tail:n { a ~ {bc} d }
%   \end{verbatim}
%   and
%   \begin{verbatim}
%     \tl_tail:n { ~ a ~ {bc} d }
%   \end{verbatim}
%   both leave \verb*| {bc}d| in the input stream.  A blank
%   \meta{token list} (see \cs{tl_if_blank:nTF}) results
%   in \cs{tl_tail:n} leaving nothing in the input stream.
%   \begin{texnote}
%     The result is returned within \cs{exp_not:n}, which means that the
%     token list does not expand further when appearing in an \texttt{x}-type
%     argument expansion.
%   \end{texnote}
% \end{function}
%
% \begin{function}[updated = 2012-07-09, EXP, pTF]{\tl_if_head_eq_catcode:nN}
%   \begin{syntax}
%     \cs{tl_if_head_eq_catcode_p:nN} \Arg{token list} \meta{test token}
%     \cs{tl_if_head_eq_catcode:nNTF} \Arg{token list} \meta{test token}
%     ~~\Arg{true code} \Arg{false code}
%   \end{syntax}
%   Tests if the first \meta{token} in the \meta{token list} has the
%   same category code as the \meta{test token}.  In the case where the
%   \meta{token list} is empty, the test is always \texttt{false}.
% \end{function}
%
% \begin{function}[updated = 2012-07-09, EXP, pTF]
%   {\tl_if_head_eq_charcode:nN, \tl_if_head_eq_charcode:fN}
%   \begin{syntax}
%     \cs{tl_if_head_eq_charcode_p:nN} \Arg{token list} \meta{test token}
%     \cs{tl_if_head_eq_charcode:nNTF} \Arg{token list} \meta{test token}
%     ~~\Arg{true code} \Arg{false code}
%   \end{syntax}
%   Tests if the first \meta{token} in the \meta{token list} has the
%   same character code as the \meta{test token}.  In the case where the
%   \meta{token list} is empty, the test is always \texttt{false}.
% \end{function}
%
% \begin{function}[updated = 2012-07-09, EXP, pTF]{\tl_if_head_eq_meaning:nN}
%   \begin{syntax}
%     \cs{tl_if_head_eq_meaning_p:nN} \Arg{token list} \meta{test token}
%     \cs{tl_if_head_eq_meaning:nNTF} \Arg{token list} \meta{test token}
%     ~~\Arg{true code} \Arg{false code}
%   \end{syntax}
%   Tests if the first \meta{token} in the \meta{token list} has the
%   same meaning as the \meta{test token}.  In the case where
%   \meta{token list} is empty, the test is always \texttt{false}.
% \end{function}
%
% \begin{function}[added = 2012-07-08, EXP, pTF]{\tl_if_head_is_group:n}
%   \begin{syntax}
%     \cs{tl_if_head_is_group_p:n} \Arg{token list}
%     \cs{tl_if_head_is_group:nTF} \Arg{token list} \Arg{true code} \Arg{false code}
%   \end{syntax}
%   Tests if the first \meta{token} in the \meta{token list}
%   is an explicit begin-group character (with category code~$1$
%   and any character code), in other words, if the \meta{token list}
%   starts with a brace group. In particular, the test is \texttt{false}
%   if the \meta{token list} starts with an implicit token such as
%   \cs{c_group_begin_token}, or if it is empty.
%   This function is useful to implement actions on token lists on
%   a token by token basis.
% \end{function}
%
% \begin{function}[added = 2012-07-08, EXP, pTF]{\tl_if_head_is_N_type:n}
%   \begin{syntax}
%     \cs{tl_if_head_is_N_type_p:n} \Arg{token list}
%     \cs{tl_if_head_is_N_type:nTF} \Arg{token list} \Arg{true code} \Arg{false code}
%   \end{syntax}
%   Tests if the first \meta{token} in the \meta{token list}
%   is a normal \texttt{N}-type argument. In other words,
%   it is neither an explicit space character
%   (explicit token with character code~$32$ and category code~$10$)
%   nor an explicit begin-group character
%   (with category code~1 and any character code). An empty
%   argument yields \texttt{false}, as it does not have a \enquote{normal}
%   first token.
%   This function is useful to implement actions on token lists on
%   a token by token basis.
% \end{function}
%
% \begin{function}[updated = 2012-07-08, EXP, pTF]{\tl_if_head_is_space:n}
%   \begin{syntax}
%     \cs{tl_if_head_is_space_p:n} \Arg{token list}
%     \cs{tl_if_head_is_space:nTF} \Arg{token list} \Arg{true code} \Arg{false code}
%   \end{syntax}
%   Tests if the first \meta{token} in the \meta{token list}
%   is an explicit space character
%   (explicit token with character code~$12$ and category code~$10$).
%   In particular, the test is \texttt{false} if the \meta{token list}
%   starts with an implicit token such as \cs{c_space_token}, or if it
%   is empty.
%   This function is useful to implement actions on token lists on
%   a token by token basis.
% \end{function}
%
% \section{Using a single item}
%
% \begin{function}[added = 2014-07-17, EXP]
%   {\tl_item:nn, \tl_item:Nn, \tl_item:cn}
%   \begin{syntax}
%     \cs{tl_item:nn} \Arg{token list} \Arg{integer expression}
%   \end{syntax}
%   Indexing items in the \meta{token list} from~$1$ on the left, this
%   function evaluates the \meta{integer expression} and leaves the
%   appropriate item from the \meta{token list} in the input stream.
%   If the \meta{integer expression} is negative, indexing occurs from
%   the right of the token list, starting at $-1$ for the right-most item.
%   If the index is out of bounds, then the function expands to nothing.
%   \begin{texnote}
%     The result is returned within the \tn{unexpanded}
%     primitive (\cs{exp_not:n}), which means that the \meta{item}
%     does not expand further when appearing in an \texttt{x}-type
%     argument expansion.
%   \end{texnote}
% \end{function}
%
% \section{Viewing token lists}
%
% \begin{function}[updated = 2015-08-01]{\tl_show:N, \tl_show:c}
%   \begin{syntax}
%     \cs{tl_show:N} \meta{tl~var}
%   \end{syntax}
%   Displays the content of the \meta{tl~var} on the terminal.
%   \begin{texnote}
%     This is similar to the \TeX{} primitive \tn{show}, wrapped to a
%     fixed number of characters per line.
%   \end{texnote}
% \end{function}
%
% \begin{function}[updated = 2015-08-07]{\tl_show:n}
%   \begin{syntax}
%     \cs{tl_show:n} \Arg{token list}
%   \end{syntax}
%   Displays the \meta{token list} on the terminal.
%   \begin{texnote}
%     This is similar to the \eTeX{} primitive \tn{showtokens}, wrapped
%     to a fixed number of characters per line.
%   \end{texnote}
% \end{function}
%
% \begin{function}[added = 2014-08-22, updated = 2015-08-01]{\tl_log:N, \tl_log:c}
%   \begin{syntax}
%     \cs{tl_log:N} \meta{tl~var}
%   \end{syntax}
%   Writes the content of the \meta{tl~var} in the log file.  See also
%   \cs{tl_show:N} which displays the result in the terminal.
% \end{function}
%
% \begin{function}[added = 2014-08-22, updated = 2015-08-07]{\tl_log:n}
%   \begin{syntax}
%     \cs{tl_log:n} \Arg{token list}
%   \end{syntax}
%   Writes the \meta{token list} in the log file.  See also
%   \cs{tl_show:n} which displays the result in the terminal.
% \end{function}
%
% \section{Constant token lists}
%
% \begin{variable}{\c_empty_tl}
%   Constant that is always empty.
% \end{variable}
%
% \begin{variable}[added = 2017-11-14]{\c_novalue_tl}
%   A marker for the absence of an argument. This constant |tl| can safely
%   be typeset (\emph{cf.}~|\q_nil|), with the result being |-NoValue-|.
%   It is important to note that \cs{c_novalue_tl} is constructed such that it
%   will \emph{not} match the simple text input |-NoValue-|, \emph{i.e.}
%   that
%   \begin{verbatim}
%     \tl_if_eq:VnTF \c_novalue_tl { -NoValue- }
%   \end{verbatim}
%   is logically \texttt{false}. The \cs{c_novalue_tl} marker is intended for
%   use in creating document-level interfaces, where it serves as an indicator
%   that an (optional) argument was omitted. In particular, it is distinct
%   from a simple empty |tl|.
% \end{variable}
%
% \begin{variable}{\c_space_tl}
%   An explicit space character contained in a token list (compare this with
%   \cs{c_space_token}). For use where an explicit space is required.
% \end{variable}
%
% \section{Scratch token lists}
%
% \begin{variable}{\l_tmpa_tl, \l_tmpb_tl}
%   Scratch token lists for local assignment. These are never used by
%   the kernel code, and so are safe for use with any \LaTeX3-defined
%   function. However, they may be overwritten by other non-kernel
%   code and so should only be used for short-term storage.
% \end{variable}
%
% \begin{variable}{\g_tmpa_tl, \g_tmpb_tl}
%   Scratch token lists for global assignment. These are never used by
%   the kernel code, and so are safe for use with any \LaTeX3-defined
%   function. However, they may be overwritten by other non-kernel
%   code and so should only be used for short-term storage.
% \end{variable}
%
% \end{documentation}
%
% \begin{implementation}
%
% \section{\pkg{l3tl} implementation}
%
%    \begin{macrocode}
%<*initex|package>
%    \end{macrocode}
%
%    \begin{macrocode}
%<@@=tl>
%    \end{macrocode}
%
% A token list variable is a \TeX{} macro that holds tokens. By using the
% \eTeX{} primitive \tn{unexpanded} inside a \TeX{} \tn{edef} it is
% possible to store any tokens, including |#|, in this way.
%
% \subsection{Functions}
%
% \begin{macro}{\tl_new:N, \tl_new:c}
%   Creating new token list variables is a case of checking for an
%   existing definition and doing the definition.
%    \begin{macrocode}
\cs_new_protected:Npn \tl_new:N #1
  {
    \__kernel_chk_if_free_cs:N #1
    \cs_gset_eq:NN #1 \c_empty_tl
  }
\cs_generate_variant:Nn \tl_new:N { c }
%    \end{macrocode}
% \end{macro}
%
% \begin{macro}{\tl_const:Nn, \tl_const:Nx, \tl_const:cn, \tl_const:cx}
%   Constants are also easy to generate.
%    \begin{macrocode}
\__kernel_patch:nnNNpn { \__kernel_chk_var_scope:NN c #1 } { }
\cs_new_protected:Npn \tl_const:Nn #1#2
  {
    \__kernel_chk_if_free_cs:N #1
    \cs_gset_nopar:Npx #1 { \exp_not:n {#2} }
  }
\__kernel_patch:nnNNpn { \__kernel_chk_var_scope:NN c #1 } { }
\cs_new_protected:Npn \tl_const:Nx #1#2
  {
    \__kernel_chk_if_free_cs:N #1
    \cs_gset_nopar:Npx #1 {#2}
  }
\cs_generate_variant:Nn \tl_const:Nn { c }
\cs_generate_variant:Nn \tl_const:Nx { c }
%    \end{macrocode}
% \end{macro}
%
% \begin{macro}{\tl_clear:N, \tl_clear:c}
% \begin{macro}{\tl_gclear:N, \tl_gclear:c}
%   Clearing a token list variable means setting it to an empty value.
%   Error checking is sorted out by the parent function.
%    \begin{macrocode}
\cs_new_protected:Npn \tl_clear:N  #1
  { \tl_set_eq:NN #1 \c_empty_tl }
\cs_new_protected:Npn \tl_gclear:N #1
  { \tl_gset_eq:NN #1 \c_empty_tl }
\cs_generate_variant:Nn \tl_clear:N  { c }
\cs_generate_variant:Nn \tl_gclear:N { c }
%    \end{macrocode}
% \end{macro}
% \end{macro}
%
% \begin{macro}{\tl_clear_new:N, \tl_clear_new:c}
% \begin{macro}{\tl_gclear_new:N, \tl_gclear_new:c}
%   Clearing a token list variable means setting it to an empty value.
%   Error checking is sorted out by the parent function.
%    \begin{macrocode}
\cs_new_protected:Npn \tl_clear_new:N  #1
  { \tl_if_exist:NTF #1 { \tl_clear:N #1 } { \tl_new:N #1 } }
\cs_new_protected:Npn \tl_gclear_new:N #1
  { \tl_if_exist:NTF #1 { \tl_gclear:N #1 } { \tl_new:N #1 } }
\cs_generate_variant:Nn \tl_clear_new:N  { c }
\cs_generate_variant:Nn \tl_gclear_new:N { c }
%    \end{macrocode}
% \end{macro}
% \end{macro}
%
% \begin{macro}{\tl_set_eq:NN, \tl_set_eq:Nc, \tl_set_eq:cN, \tl_set_eq:cc}
% \begin{macro}{\tl_gset_eq:NN, \tl_gset_eq:Nc, \tl_gset_eq:cN, \tl_gset_eq:cc}
%   For setting token list variables equal to each other.  When checking
%   is turned on, make sure both variables exist.
%    \begin{macrocode}
\__kernel_if_debug:TF
  {
    \cs_new_protected:Npn \tl_set_eq:NN #1#2
      {
        \__kernel_chk_var_local:N #1
        \__kernel_chk_var_exist:N #2
        \cs_set_eq:NN #1 #2
      }
    \cs_new_protected:Npn \tl_gset_eq:NN #1#2
      {
        \__kernel_chk_var_global:N #1
        \__kernel_chk_var_exist:N #2
        \cs_gset_eq:NN #1 #2
      }
  }
  {
    \cs_new_eq:NN \tl_set_eq:NN  \cs_set_eq:NN
    \cs_new_eq:NN \tl_gset_eq:NN \cs_gset_eq:NN
  }
\cs_generate_variant:Nn \tl_set_eq:NN { cN, Nc, cc }
\cs_generate_variant:Nn \tl_gset_eq:NN { cN, Nc, cc }
%    \end{macrocode}
% \end{macro}
% \end{macro}
%
% \begin{macro}{\tl_concat:NNN, \tl_concat:ccc}
% \begin{macro}{\tl_gconcat:NNN, \tl_gconcat:ccc}
%   Concatenating token lists is easy.  When checking is turned on, all
%   three arguments must be checked: a token list |#2| or |#3| equal to
%   \cs{scan_stop:} would lead to problems later on.
%    \begin{macrocode}
\__kernel_patch:nnNNpn
  {
    \__kernel_chk_var_exist:N #2
    \__kernel_chk_var_exist:N #3
  }
  { }
\cs_new_protected:Npn \tl_concat:NNN #1#2#3
  { \tl_set:Nx #1 { \exp_not:o {#2} \exp_not:o {#3} } }
\__kernel_patch:nnNNpn
  {
    \__kernel_chk_var_exist:N #2
    \__kernel_chk_var_exist:N #3
  }
  { }
\cs_new_protected:Npn \tl_gconcat:NNN #1#2#3
  { \tl_gset:Nx #1 { \exp_not:o {#2} \exp_not:o {#3} } }
\cs_generate_variant:Nn \tl_concat:NNN  { ccc }
\cs_generate_variant:Nn \tl_gconcat:NNN { ccc }
%    \end{macrocode}
% \end{macro}
% \end{macro}
%
% \begin{macro}[pTF]{\tl_if_exist:N, \tl_if_exist:c}
%   Copies of the \texttt{cs} functions defined in \pkg{l3basics}.
%    \begin{macrocode}
\prg_new_eq_conditional:NNn \tl_if_exist:N \cs_if_exist:N { TF , T , F , p }
\prg_new_eq_conditional:NNn \tl_if_exist:c \cs_if_exist:c { TF , T , F , p }
%    \end{macrocode}
% \end{macro}
%
% \subsection{Constant token lists}
%
% \begin{variable}{\c_empty_tl}
%   Never full. We need to define that constant before using \cs{tl_new:N}.
%    \begin{macrocode}
\tl_const:Nn \c_empty_tl { }
%    \end{macrocode}
% \end{variable}
%
% \begin{variable}{\c_novalue_tl}
%   A special marker: as we don't have |\char_generate:nn| yet, has to be
%   created the old-fashioned way.
%    \begin{macrocode}
\group_begin:
\tex_lccode:D `A = `-
\tex_lccode:D `N = `N
\tex_lccode:D `V = `V
\tex_lowercase:D
  {
    \group_end:
    \tl_const:Nn \c_novalue_tl { ANoValue- }
  }
%    \end{macrocode}
% \end{variable}
%
% \begin{variable}{\c_space_tl}
%   A space as a token list (as opposed to as a character).
%    \begin{macrocode}
\tl_const:Nn \c_space_tl { ~ }
%    \end{macrocode}
% \end{variable}
%
% \subsection{Adding to token list variables}
%
% \begin{macro}
%   {
%     \tl_set:Nn, \tl_set:NV, \tl_set:Nv, \tl_set:No, \tl_set:Nf, \tl_set:Nx,
%     \tl_set:cn, \tl_set:cV, \tl_set:cv, \tl_set:co, \tl_set:cf, \tl_set:cx
%   }
% \begin{macro}
%   {
%     \tl_gset:Nn, \tl_gset:NV, \tl_gset:Nv,
%     \tl_gset:No, \tl_gset:Nf, \tl_gset:Nx,
%     \tl_gset:cn, \tl_gset:cV, \tl_gset:cv,
%     \tl_gset:co, \tl_gset:cf, \tl_gset:cx
%   }
%   By using \cs{exp_not:n} token list variables can contain |#| tokens,
%   which makes the token list registers provided by \TeX{}
%   more or less redundant. The \cs{tl_set:No} version is done
%   \enquote{by hand} as it is used quite a lot.  Each definition is
%   prefixed by a call to \cs{__kernel_patch:nnNNpn} which adds an
%   existence check to the definition.
%    \begin{macrocode}
\__kernel_patch:nnNNpn { \__kernel_chk_var_local:N #1 } { }
\cs_new_protected:Npn \tl_set:Nn #1#2
  { \cs_set_nopar:Npx #1 { \exp_not:n {#2} } }
\__kernel_patch:nnNNpn { \__kernel_chk_var_local:N #1 } { }
\cs_new_protected:Npn \tl_set:No #1#2
  { \cs_set_nopar:Npx #1 { \exp_not:o {#2} } }
\__kernel_patch:nnNNpn { \__kernel_chk_var_local:N #1 } { }
\cs_new_protected:Npn \tl_set:Nx #1#2
  { \cs_set_nopar:Npx #1 {#2} }
\__kernel_patch:nnNNpn { \__kernel_chk_var_global:N #1 } { }
\cs_new_protected:Npn \tl_gset:Nn #1#2
  { \cs_gset_nopar:Npx #1 { \exp_not:n {#2} } }
\__kernel_patch:nnNNpn { \__kernel_chk_var_global:N #1 } { }
\cs_new_protected:Npn \tl_gset:No #1#2
  { \cs_gset_nopar:Npx #1 { \exp_not:o {#2} } }
\__kernel_patch:nnNNpn { \__kernel_chk_var_global:N #1 } { }
\cs_new_protected:Npn \tl_gset:Nx #1#2
  { \cs_gset_nopar:Npx #1 {#2} }
\cs_generate_variant:Nn \tl_set:Nn  {         NV , Nv , Nf }
\cs_generate_variant:Nn \tl_set:Nx  { c }
\cs_generate_variant:Nn \tl_set:Nn  { c, co , cV , cv , cf }
\cs_generate_variant:Nn \tl_gset:Nn {         NV , Nv , Nf }
\cs_generate_variant:Nn \tl_gset:Nx { c }
\cs_generate_variant:Nn \tl_gset:Nn { c, co , cV , cv , cf }
%    \end{macrocode}
% \end{macro}
% \end{macro}
%
% \begin{macro}
%   {
%     \tl_put_left:Nn, \tl_put_left:NV, \tl_put_left:No, \tl_put_left:Nx,
%     \tl_put_left:cn, \tl_put_left:cV, \tl_put_left:co, \tl_put_left:cx
%   }
% \begin{macro}
%   {
%     \tl_gput_left:Nn, \tl_gput_left:NV, \tl_gput_left:No, \tl_gput_left:Nx,
%     \tl_gput_left:cn, \tl_gput_left:cV, \tl_gput_left:co, \tl_gput_left:cx
%   }
% Adding to the left is done directly to gain a little performance.
%    \begin{macrocode}
\__kernel_patch:nnNNpn { \__kernel_chk_var_local:N #1 } { }
\cs_new_protected:Npn \tl_put_left:Nn #1#2
  { \cs_set_nopar:Npx #1 { \exp_not:n {#2} \exp_not:o #1 } }
\__kernel_patch:nnNNpn { \__kernel_chk_var_local:N #1 } { }
\cs_new_protected:Npn \tl_put_left:NV #1#2
  { \cs_set_nopar:Npx #1 { \exp_not:V #2 \exp_not:o #1 } }
\__kernel_patch:nnNNpn { \__kernel_chk_var_local:N #1 } { }
\cs_new_protected:Npn \tl_put_left:No #1#2
  { \cs_set_nopar:Npx #1 { \exp_not:o {#2} \exp_not:o #1 } }
\__kernel_patch:nnNNpn { \__kernel_chk_var_local:N #1 } { }
\cs_new_protected:Npn \tl_put_left:Nx #1#2
  { \cs_set_nopar:Npx #1 { #2 \exp_not:o #1 } }
\__kernel_patch:nnNNpn { \__kernel_chk_var_global:N #1 } { }
\cs_new_protected:Npn \tl_gput_left:Nn #1#2
  { \cs_gset_nopar:Npx #1 { \exp_not:n {#2} \exp_not:o #1 } }
\__kernel_patch:nnNNpn { \__kernel_chk_var_global:N #1 } { }
\cs_new_protected:Npn \tl_gput_left:NV #1#2
  { \cs_gset_nopar:Npx #1 { \exp_not:V #2 \exp_not:o #1 } }
\__kernel_patch:nnNNpn { \__kernel_chk_var_global:N #1 } { }
\cs_new_protected:Npn \tl_gput_left:No #1#2
  { \cs_gset_nopar:Npx #1 { \exp_not:o {#2} \exp_not:o #1 } }
\__kernel_patch:nnNNpn { \__kernel_chk_var_global:N #1 } { }
\cs_new_protected:Npn \tl_gput_left:Nx #1#2
  { \cs_gset_nopar:Npx #1 { #2 \exp_not:o {#1} } }
\cs_generate_variant:Nn \tl_put_left:Nn  { c }
\cs_generate_variant:Nn \tl_put_left:NV  { c }
\cs_generate_variant:Nn \tl_put_left:No  { c }
\cs_generate_variant:Nn \tl_put_left:Nx  { c }
\cs_generate_variant:Nn \tl_gput_left:Nn { c }
\cs_generate_variant:Nn \tl_gput_left:NV { c }
\cs_generate_variant:Nn \tl_gput_left:No { c }
\cs_generate_variant:Nn \tl_gput_left:Nx { c }
%    \end{macrocode}
% \end{macro}
% \end{macro}
%
% \begin{macro}
%   {
%     \tl_put_right:Nn, \tl_put_right:NV, \tl_put_right:No, \tl_put_right:Nx,
%     \tl_put_right:cn, \tl_put_right:cV, \tl_put_right:co, \tl_put_right:cx
%   }
% \begin{macro}
%   {
%     \tl_gput_right:Nn, \tl_gput_right:NV, \tl_gput_right:No,
%     \tl_gput_right:Nx,
%     \tl_gput_right:cn, \tl_gput_right:cV, \tl_gput_right:co,
%     \tl_gput_right:cx
%   }
% The same on the right.
%    \begin{macrocode}
\__kernel_patch:nnNNpn { \__kernel_chk_var_local:N #1 } { }
\cs_new_protected:Npn \tl_put_right:Nn #1#2
  { \cs_set_nopar:Npx #1 { \exp_not:o #1 \exp_not:n {#2} } }
\__kernel_patch:nnNNpn { \__kernel_chk_var_local:N #1 } { }
\cs_new_protected:Npn \tl_put_right:NV #1#2
  { \cs_set_nopar:Npx #1 { \exp_not:o #1 \exp_not:V #2 } }
\__kernel_patch:nnNNpn { \__kernel_chk_var_local:N #1 } { }
\cs_new_protected:Npn \tl_put_right:No #1#2
  { \cs_set_nopar:Npx #1 { \exp_not:o #1 \exp_not:o {#2} } }
\__kernel_patch:nnNNpn { \__kernel_chk_var_local:N #1 } { }
\cs_new_protected:Npn \tl_put_right:Nx #1#2
  { \cs_set_nopar:Npx #1 { \exp_not:o #1 #2 } }
\__kernel_patch:nnNNpn { \__kernel_chk_var_global:N #1 } { }
\cs_new_protected:Npn \tl_gput_right:Nn #1#2
  { \cs_gset_nopar:Npx #1 { \exp_not:o #1 \exp_not:n {#2} } }
\__kernel_patch:nnNNpn { \__kernel_chk_var_global:N #1 } { }
\cs_new_protected:Npn \tl_gput_right:NV #1#2
  { \cs_gset_nopar:Npx #1 { \exp_not:o #1 \exp_not:V #2 } }
\__kernel_patch:nnNNpn { \__kernel_chk_var_global:N #1 } { }
\cs_new_protected:Npn \tl_gput_right:No #1#2
  { \cs_gset_nopar:Npx #1 { \exp_not:o #1 \exp_not:o {#2} } }
\__kernel_patch:nnNNpn { \__kernel_chk_var_global:N #1 } { }
\cs_new_protected:Npn \tl_gput_right:Nx #1#2
  { \cs_gset_nopar:Npx #1 { \exp_not:o {#1} #2 } }
\cs_generate_variant:Nn \tl_put_right:Nn  { c }
\cs_generate_variant:Nn \tl_put_right:NV  { c }
\cs_generate_variant:Nn \tl_put_right:No  { c }
\cs_generate_variant:Nn \tl_put_right:Nx  { c }
\cs_generate_variant:Nn \tl_gput_right:Nn { c }
\cs_generate_variant:Nn \tl_gput_right:NV { c }
\cs_generate_variant:Nn \tl_gput_right:No { c }
\cs_generate_variant:Nn \tl_gput_right:Nx { c }
%    \end{macrocode}
% \end{macro}
% \end{macro}
%
% \subsection{Reassigning token list category codes}
%
% \begin{variable}{\c_@@_rescan_marker_tl}
%   The rescanning code needs a special token list containing the same
%   character (chosen here to be a colon) with two different category
%   codes: it cannot appear in the tokens being rescanned since all
%   colons have the same category code.
%    \begin{macrocode}
\tl_const:Nx \c_@@_rescan_marker_tl { : \token_to_str:N : }
%    \end{macrocode}
% \end{variable}
%
% \begin{macro}
%   {
%     \tl_set_rescan:Nnn, \tl_set_rescan:Nno, \tl_set_rescan:Nnx,
%     \tl_set_rescan:cnn, \tl_set_rescan:cno, \tl_set_rescan:cnx
%   }
% \begin{macro}
%   {
%     \tl_gset_rescan:Nnn, \tl_gset_rescan:Nno, \tl_gset_rescan:Nnx,
%     \tl_gset_rescan:cnn, \tl_gset_rescan:cno, \tl_gset_rescan:cnx
%   }
% \begin{macro}{\tl_rescan:nn}
% \begin{macro}{\@@_set_rescan:NNnn, \@@_set_rescan_multi:n}
% \begin{macro}[EXP]{\@@_rescan:w}
%   These functions use a common auxiliary.  After some initial setup
%   explained below, and the user setup |#3| (followed by
%   \cs{scan_stop:} to be safe), the tokens are rescanned by
%   \cs{@@_set_rescan:n} and stored into \cs{l_@@_internal_a_tl}, then
%   passed to |#1#2| outside the group after expansion.  The auxiliary
%   \cs{@@_set_rescan:n} is defined later: in the simplest case, this
%   auxiliary calls \cs{@@_set_rescan_multi:n}, whose code is included
%   here to help understand the approach.
%
%   One difficulty when rescanning is that \tn{scantokens} treats the
%   argument as a file, and without the correct settings a \TeX{} error
%   occurs:
%   \begin{verbatim}
%    ! File ended while scanning definition of ...
%   \end{verbatim}
%   The standard solution is to use an \texttt{x}-expanding assignment
%   and set \tn{everyeof} to \cs{exp_not:N} to suppress the error at
%   the end of the file.  Since the rescanned tokens should not be
%   expanded, they are taken as a delimited argument of an
%   auxiliary which wraps them in \cs{exp_not:n} (in fact
%   \cs{exp_not:o}, as there is a \cs{prg_do_nothing:} to avoid losing
%   braces).  The delimiter cannot appear within the rescanned token
%   list because it contains twice the same character, with different
%   catcodes.
%
%   The difference between single-line and multiple-line files
%   complicates the story, as explained below.
%    \begin{macrocode}
\cs_new_protected:Npn \tl_set_rescan:Nnn
  { \@@_set_rescan:NNnn \tl_set:Nn }
\cs_new_protected:Npn \tl_gset_rescan:Nnn
  { \@@_set_rescan:NNnn \tl_gset:Nn }
\cs_new_protected:Npn \tl_rescan:nn
  { \@@_set_rescan:NNnn \prg_do_nothing: \use:n }
\cs_new_protected:Npn \@@_set_rescan:NNnn #1#2#3#4
  {
    \tl_if_empty:nTF {#4}
      {
        \group_begin:
          #3
        \group_end:
        #1 #2 { }
      }
      {
        \group_begin:
          \exp_args:No \tex_everyeof:D
            { \c_@@_rescan_marker_tl \exp_not:N }
          \int_compare:nNnT \tex_endlinechar:D = { 32 }
            { \int_set:Nn \tex_endlinechar:D { -1 } }
          \tex_newlinechar:D \tex_endlinechar:D
          #3 \scan_stop:
          \exp_args:No \@@_set_rescan:n { \tl_to_str:n {#4} }
          \exp_args:NNNo
        \group_end:
        #1 #2 \l_@@_internal_a_tl
    }
  }
\cs_new_protected:Npn \@@_set_rescan_multi:n #1
  {
    \tl_set:Nx \l_@@_internal_a_tl
      {
        \exp_after:wN \@@_rescan:w
        \exp_after:wN \prg_do_nothing:
        \tex_scantokens:D {#1}
      }
  }
\exp_args:Nno \use:nn
  { \cs_new:Npn \@@_rescan:w #1 } \c_@@_rescan_marker_tl
  { \exp_not:o {#1} }
\cs_generate_variant:Nn \tl_set_rescan:Nnn  {     Nno , Nnx }
\cs_generate_variant:Nn \tl_set_rescan:Nnn  { c , cno , cnx }
\cs_generate_variant:Nn \tl_gset_rescan:Nnn {     Nno , Nnx }
\cs_generate_variant:Nn \tl_gset_rescan:Nnn { c , cno }
%    \end{macrocode}
% \end{macro}
% \end{macro}
% \end{macro}
% \end{macro}
% \end{macro}
%
% \begin{macro}{\@@_set_rescan:n, \@@_set_rescan:NnTF}
% \begin{macro}{\@@_set_rescan_single:nn, \@@_set_rescan_single_aux:nn}
%   This function calls \cs{@@_set_rescan_multiple:n} or
%   \cs{@@_set_rescan_single:nn} |{ ' }| depending on whether its
%   argument is a single-line fragment of code/data or is made of
%   multiple lines by testing for the presence of a \tn{newlinechar}
%   character.  If \tn{newlinechar} is out of range, the argument is
%   assumed to be a single line.
%
%   The case of multiple lines is a straightforward application of
%   \tn{scantokens} as described above.  The only subtlety is that
%   \tn{newlinechar} should be equal to \tn{endlinechar} because
%   \tn{newlinechar} characters become new lines and then become
%   \tn{endlinechar} characters when writing to an abstract file and
%   reading back.  This equality is ensured by setting \tn{newlinechar}
%   equal to \tn{endlinechar}.  Prior to this, \tn{endlinechar} is set
%   to $-1$ if it was $32$ (in particular true after \cs{ExplSyntaxOn})
%   to avoid unreasonable line-breaks at every space for instance in
%   error messages triggered by the user setup.  Another side effect of
%   reading back from the file is that spaces (catcode $10$) are
%   ignored at the beginning of lines, and spaces and tabs (character
%   code $32$ and $9$) are ignored at the end of lines.
%
%   For a single line, no \tn{endlinechar} should be added, so it is
%   set to $-1$, and spaces should not be removed.
%
%   Trailing spaces and tabs are a difficult matter, as \TeX{} removes
%   these at a very low level.  The only way to preserve them is to
%   rescan not the argument but the argument followed by a character
%   with a reasonable category code.  Here, $11$ (letter), $12$ (other)
%   and $13$ (active) are accepted, as these are suitable for
%   delimiting an argument, and it is very unlikely that none of the
%   ASCII characters are in one of these categories.  To avoid
%   selecting one particular character to put at the end, whose
%   category code may have been modified, there is a loop through
%   characters from |'| (ASCII $39$) to |~| (ASCII $127$).  The choice
%   of starting point was made because this is the start of a very long
%   range of characters whose standard category is letter or other,
%   thus minimizing the number of steps needed by the loop (most often
%   just a single one).  Once a valid character is found, run some code
%   very similar to \cs{@@_set_rescan_multi:n}, except that
%   \cs{@@_rescan:w} must be redefined to also remove the additional
%   character (with the appropriate catcode).  Getting the delimiter
%   with the right catcode requires using \tn{scantokens} inside an
%   \texttt{x}-expansion, hence using the previous definition of
%   \cs{@@_rescan:w} as well.  The odd \cs{exp_not:N} \cs{use:n}
%   ensures that the trailing \cs{exp_not:N} in \tn{everyeof} does
%   not prevent the expansion of \cs{c_@@_rescan_marker_tl}, but
%   rather of a closing brace (this does nothing).
%   If no valid character is found, similar code is ran,
%   and the only difference is that trailing spaces are not preserved
%   (bear in mind that this only happens if no character between $39$
%   and $127$ has catcode letter, other or active).
%
%   There is also some work to preserve leading spaces: test whether
%   the first character (given by \cs{str_head:n}, with an extra space
%   to circumvent a limitation of \texttt{f}-expansion) has catcode
%   $10$ and add what \TeX{} would add in the middle of a line for any
%   sequence of such characters: a single space with catcode $10$ and
%   character code $32$.
%    \begin{macrocode}
\group_begin:
  \tex_catcode:D `\^^@ = 12 \scan_stop:
  \cs_new_protected:Npn \@@_set_rescan:n #1
    {
      \int_compare:nNnTF \tex_newlinechar:D < 0
        { \use_ii:nn }
        {
          \char_set_lccode:nn { 0 } { \tex_newlinechar:D }
          \tex_lowercase:D { \@@_set_rescan:NnTF ^^@ } {#1}
        }
          { \@@_set_rescan_multi:n }
          { \@@_set_rescan_single:nn { ' } }
      {#1}
    }
  \cs_new_protected:Npn \@@_set_rescan:NnTF #1#2
    { \tl_if_in:nnTF {#2} {#1} }
  \cs_new_protected:Npn \@@_set_rescan_single:nn #1
    {
      \int_compare:nNnTF
        { \char_value_catcode:n { `#1 } / 3 } = 4
        { \@@_set_rescan_single_aux:nn {#1} }
        {
          \int_compare:nNnTF { `#1 } < { `\~ }
            {
              \char_set_lccode:nn { 0 } { `#1 + 1 }
              \tex_lowercase:D { \@@_set_rescan_single:nn { ^^@ } }
            }
            { \@@_set_rescan_single_aux:nn { } }
        }
    }
  \cs_new_protected:Npn \@@_set_rescan_single_aux:nn #1#2
    {
      \int_set:Nn \tex_endlinechar:D { -1 }
      \use:x
        {
          \exp_not:N \use:n
            {
              \exp_not:n { \cs_set:Npn \@@_rescan:w ##1 }
              \exp_after:wN \@@_rescan:w
              \exp_after:wN \prg_do_nothing:
              \tex_scantokens:D {#1}
            }
          \c_@@_rescan_marker_tl
        }
        { \exp_not:o {##1} }
      \tl_set:Nx \l_@@_internal_a_tl
        {
          \int_compare:nNnT
            {
              \char_value_catcode:n
                { \exp_last_unbraced:Nf ` { \str_head:n {#2} } ~ }
            }
            = { 10 } { ~ }
          \exp_after:wN \@@_rescan:w
          \exp_after:wN \prg_do_nothing:
          \tex_scantokens:D { #2 #1 }
        }
    }
\group_end:
%    \end{macrocode}
% \end{macro}
% \end{macro}
%
% \subsection{Modifying token list variables}
%
% \begin{macro}{\tl_replace_all:Nnn, \tl_replace_all:cnn}
% \begin{macro}{\tl_greplace_all:Nnn, \tl_greplace_all:cnn}
% \begin{macro}{\tl_replace_once:Nnn, \tl_replace_once:cnn}
% \begin{macro}{\tl_greplace_once:Nnn, \tl_greplace_once:cnn}
%   All of the \texttt{replace} functions call \cs{@@_replace:NnNNNnn}
%   with appropriate arguments.  The first two arguments are explained
%   later.  The next controls whether the replacement function calls
%   itself (\cs{@@_replace_next:w}) or stops (\cs{@@_replace_wrap:w})
%   after the first replacement.  Next comes an \texttt{x}-type
%   assignment function \cs{tl_set:Nx} or \cs{tl_gset:Nx} for local or
%   global replacements.  Finally, the three arguments \meta{tl~var}
%   \Arg{pattern} \Arg{replacement} provided by the user.  When
%   describing the auxiliary functions below, we denote the contents of
%   the \meta{tl~var} by \meta{token list}.
%    \begin{macrocode}
\cs_new_protected:Npn \tl_replace_once:Nnn
  { \@@_replace:NnNNNnn \q_mark ? \@@_replace_wrap:w \tl_set:Nx  }
\cs_new_protected:Npn \tl_greplace_once:Nnn
  { \@@_replace:NnNNNnn \q_mark ? \@@_replace_wrap:w \tl_gset:Nx }
\cs_new_protected:Npn \tl_replace_all:Nnn
  { \@@_replace:NnNNNnn \q_mark ? \@@_replace_next:w \tl_set:Nx  }
\cs_new_protected:Npn \tl_greplace_all:Nnn
  { \@@_replace:NnNNNnn \q_mark ? \@@_replace_next:w \tl_gset:Nx }
\cs_generate_variant:Nn \tl_replace_once:Nnn  { c }
\cs_generate_variant:Nn \tl_greplace_once:Nnn { c }
\cs_generate_variant:Nn \tl_replace_all:Nnn   { c }
\cs_generate_variant:Nn \tl_greplace_all:Nnn  { c }
%    \end{macrocode}
% \end{macro}
% \end{macro}
% \end{macro}
% \end{macro}
%
% \begin{macro}
%   {
%     \@@_replace:NnNNNnn,
%     \@@_replace_auxi:NnnNNNnn,
%     \@@_replace_auxii:nNNNnn,
%     \@@_replace_next:w,
%     \@@_replace_wrap:w,
%   }
%   To implement the actual replacement auxiliary
%   \cs{@@_replace_auxii:nNNNnn} we need a \meta{delimiter} with
%   the following properties:
%   \begin{itemize}
%     \item all occurrences of the \meta{pattern}~|#6| in
%       \enquote{\meta{token list} \meta{delimiter}} belong to the
%       \meta{token list} and have no overlap with the \meta{delimiter},
%     \item the first occurrence of the \meta{delimiter} in
%       \enquote{\meta{token list} \meta{delimiter}} is the trailing
%       \meta{delimiter}.
%   \end{itemize}
%   We first find the building blocks for the \meta{delimiter}, namely
%   two tokens \meta{A} and~\meta{B} such that \meta{A} does not appear
%   in~|#6| and |#6| is not~\meta{B} (this condition is trivial if |#6|
%   has more than one token).  Then we consider the delimiters
%   \enquote{\meta{A}} and \enquote{\meta{A} \meta{A}$^n$ \meta{B}
%     \meta{A}$^n$ \meta{B}}, for $n\geq 1$, where $\meta{A}^n$ denotes
%   $n$~copies of \meta{A}, and we choose as our \meta{delimiter} the
%   first one which is not in the \meta{token list}.
%
%   Every delimiter in the set obeys the first condition: |#6|~does not
%   contain~\meta{A} hence cannot be overlapping with the \meta{token
%     list} and the \meta{delimiter}, and it cannot be within the
%   \meta{delimiter} since it would have to be in one of the two
%   \meta{B} hence be equal to this single token (or empty, but this is
%   an error case filtered separately).  Given the particular form of
%   these delimiters, for which no prefix is also a suffix, the second
%   condition is actually a consequence of the weaker condition that the
%   \meta{delimiter} we choose does not appear in the \meta{token list}.
%   Additionally, the set of delimiters is such that a \meta{token list}
%   of $n$~tokens can contain at most $O(n^{1/2})$ of them, hence we
%   find a \meta{delimiter} with at most $O(n^{1/2})$ tokens in a time
%   at most $O(n^{3/2})$.  Bear in mind that these upper bounds are
%   reached only in very contrived scenarios: we include the case
%   \enquote{\meta{A}} in the list of delimiters to try, so that the
%   \meta{delimiter} is simply \cs{q_mark} in the most common
%   situation where neither the \meta{token list} nor the \meta{pattern}
%   contains \cs{q_mark}.
%
%   Let us now ahead, optimizing for this most common case.  First, two
%   special cases: an empty \meta{pattern}~|#6| is an error, and if
%   |#1|~is absent from both the \meta{token list}~|#5| and the
%   \meta{pattern}~|#6| then we can use it as the \meta{delimiter}
%   through \cs{@@_replace_auxii:nNNNnn} |{#1}|.  Otherwise, we end up
%   calling \cs{@@_replace:NnNNNnn} repeatedly with the first two
%   arguments |\q_mark| |{?}|, |\?| |{??}|, |\??| |{???}|, and so on,
%   until |#6|~does not contain the control sequence~|#1|, which we take
%   as our~\meta{A}.  The argument~|#2| only serves to collect~|?|
%   characters for~|#1|.  Note that the order of the tests means that
%   the first two are done every time, which is wasteful (for instance,
%   we repeatedly test for the emptyness of~|#6|).  However, this is
%   rare enough not to matter.  Finally, choose~\meta{B} to be
%   \cs{q_nil} or~\cs{q_stop} such that it is not equal to~|#6|.
%
%   The \cs{@@_replace_auxi:NnnNNNnn} auxiliary receives \Arg{A} and
%   |{|\meta{A}$^n$\meta{B}|}| as its arguments, initially with $n=1$.
%   If \enquote{\meta{A} \meta{A}$^n$\meta{B} \meta{A}$^n$\meta{B}} is
%   in the \meta{token list} then increase~$n$ and try again.  Once it
%   is not anymore in the \meta{token list} we take it as our
%   \meta{delimiter} and pass this to the \texttt{auxii} auxiliary.
%    \begin{macrocode}
\cs_new_protected:Npn \@@_replace:NnNNNnn #1#2#3#4#5#6#7
  {
    \tl_if_empty:nTF {#6}
      {
        \__kernel_msg_error:nnx { kernel } { empty-search-pattern }
          { \tl_to_str:n {#7} }
      }
      {
        \tl_if_in:onTF { #5 #6 } {#1}
          {
            \tl_if_in:nnTF {#6} {#1}
              { \exp_args:Nc \@@_replace:NnNNNnn {#2} {#2?} }
              {
                \quark_if_nil:nTF {#6}
                  { \@@_replace_auxi:NnnNNNnn #5 {#1} { #1 \q_stop } }
                  { \@@_replace_auxi:NnnNNNnn #5 {#1} { #1 \q_nil  } }
              }
          }
          { \@@_replace_auxii:nNNNnn {#1} }
          #3#4#5 {#6} {#7}
      }
  }
\cs_new_protected:Npn \@@_replace_auxi:NnnNNNnn #1#2#3
  {
    \tl_if_in:NnTF #1 { #2 #3 #3 }
      { \@@_replace_auxi:NnnNNNnn #1 { #2 #3 } {#2} }
      { \@@_replace_auxii:nNNNnn { #2 #3 #3 } }
  }
%    \end{macrocode}
%   The auxiliary \cs{@@_replace_auxii:nNNNnn} receives the following
%   arguments:
%   \begin{quote}
%     \Arg{delimiter} \meta{function} \meta{assignment} \\
%       \meta{tl~var} \Arg{pattern} \Arg{replacement}
%   \end{quote}
%   All of its work is done between
%   \cs{group_align_safe_begin:} and \cs{group_align_safe_end:} to avoid
%   issues in alignments.  It does the actual replacement within
%   |#3|~|#4|~|{...}|, an \texttt{x}-expanding \meta{assignment}~|#3| to
%   the \meta{tl~var}~|#4|.  The auxiliary \cs{@@_replace_next:w} is
%   called, followed by the \meta{token list}, some tokens including the
%   \meta{delimiter}~|#1|, followed by the \meta{pattern}~|#5|.
%   This auxiliary finds an argument delimited by~|#5| (the presence of
%   a trailing~|#5| avoids runaway arguments) and calls
%   \cs{@@_replace_wrap:w} to test whether this |#5| is found within the
%   \meta{token list} or is the trailing one.
%
%   If on the one hand it is found within the \meta{token list}, then
%   |##1| cannot contain the \meta{delimiter}~|#1| that we worked so
%   hard to obtain, thus \cs{@@_replace_wrap:w} gets~|##1| as its own
%   argument~|##1|, and protects it against
%   the \texttt{x}-expanding assignment.  It also finds \cs{exp_not:n}
%   as~|##2| and does nothing to it, thus letting through \cs{exp_not:n}
%   \Arg{replacement} into the assignment.  Note that
%   \cs{@@_replace_next:w} and \cs{@@_replace_wrap:w} are always called
%   followed by two empty brace groups.  These are safe because no
%   delimiter can match them.  They prevent losing braces when grabbing
%   delimited arguments, but require the use of \cs{exp_not:o} and
%   \cs{use_none:nn}, rather than simply \cs{exp_not:n}.
%   Afterwards, \cs{@@_replace_next:w} is called
%   to repeat the replacement, or \cs{@@_replace_wrap:w} if we only want
%   a single replacement.  In this second case, |##1| is the
%   \meta{remaining tokens} in the \meta{token list} and |##2| is some
%   \meta{ending code} which ends the assignment and removes the
%   trailing tokens |#5| using some \cs{if_false:} |{| \cs{fi:} |}|
%   trickery because~|#5| may contain any delimiter.
%
%   If on the other hand the argument~|##1| of \cs{@@_replace_next:w} is
%   delimited by the trailing \meta{pattern}~|#5|, then |##1| is
%   \enquote{\{ \} \{ \} \meta{token list} \meta{delimiter}
%     \Arg{ending code}}, hence \cs{@@_replace_wrap:w} finds
%   \enquote{\{ \} \{ \} \meta{token list}} as |##1| and the
%   \meta{ending code} as~|##2|.  It leaves the \meta{token list} into
%   the assignment and unbraces the \meta{ending code} which removes
%   what remains (essentially the \meta{delimiter} and
%   \meta{replacement}).
%    \begin{macrocode}
\cs_new_protected:Npn \@@_replace_auxii:nNNNnn #1#2#3#4#5#6
  {
    \group_align_safe_begin:
    \cs_set:Npn \@@_replace_wrap:w ##1 #1 ##2
      { \exp_not:o { \use_none:nn ##1 } ##2 }
    \cs_set:Npx \@@_replace_next:w ##1 #5
      {
        \exp_not:N \@@_replace_wrap:w ##1
        \exp_not:n { #1 }
        \exp_not:n { \exp_not:n {#6} }
        \exp_not:n { #2 { } { } }
      }
    #3 #4
      {
        \exp_after:wN \@@_replace_next:w
        \exp_after:wN { \exp_after:wN }
        \exp_after:wN { \exp_after:wN }
        #4
        #1
        {
          \if_false: { \fi: }
          \exp_after:wN \use_none:n \exp_after:wN { \if_false: } \fi:
        }
        #5
      }
    \group_align_safe_end:
  }
\cs_new_eq:NN \@@_replace_wrap:w ?
\cs_new_eq:NN \@@_replace_next:w ?
%    \end{macrocode}
% \end{macro}
%
% \begin{macro}{\tl_remove_once:Nn, \tl_remove_once:cn}
% \begin{macro}{\tl_gremove_once:Nn, \tl_gremove_once:cn}
%   Removal is just a special case of replacement.
%    \begin{macrocode}
\cs_new_protected:Npn \tl_remove_once:Nn #1#2
  { \tl_replace_once:Nnn #1 {#2} { } }
\cs_new_protected:Npn \tl_gremove_once:Nn #1#2
  { \tl_greplace_once:Nnn #1 {#2} { } }
\cs_generate_variant:Nn \tl_remove_once:Nn  { c }
\cs_generate_variant:Nn \tl_gremove_once:Nn { c }
%    \end{macrocode}
% \end{macro}
% \end{macro}
%
% \begin{macro}{\tl_remove_all:Nn, \tl_remove_all:cn}
% \begin{macro}{\tl_gremove_all:Nn, \tl_gremove_all:cn}
%   Removal is just a special case of replacement.
%    \begin{macrocode}
\cs_new_protected:Npn \tl_remove_all:Nn #1#2
  { \tl_replace_all:Nnn #1 {#2} { } }
\cs_new_protected:Npn \tl_gremove_all:Nn #1#2
  { \tl_greplace_all:Nnn #1 {#2} { } }
\cs_generate_variant:Nn \tl_remove_all:Nn  { c }
\cs_generate_variant:Nn \tl_gremove_all:Nn { c }
%    \end{macrocode}
% \end{macro}
% \end{macro}
%
% \subsection{Token list conditionals}
%
% \begin{macro}[pTF]{\tl_if_blank:n, \tl_if_blank:V, \tl_if_blank:o}
% \begin{macro}{\@@_if_blank_p:NNw}
%   \TeX{} skips spaces when reading a non-delimited arguments. Thus,
%   a \meta{token list} is blank if and only if \cs{use_none:n}
%   \meta{token list} |?| is empty after one expansion.  The auxiliary
%   \cs{@@_if_empty_if:o} is a fast emptyness test, converting its
%   argument to a string (after one expansion) and using the test
%   \cs{if_meaning:w} \cs{q_nil} |...| \cs{q_nil}.
%    \begin{macrocode}
\prg_new_conditional:Npnn \tl_if_blank:n #1 { p , T , F , TF }
  {
    \@@_if_empty_if:o { \use_none:n #1 ? }
      \prg_return_true:
    \else:
      \prg_return_false:
    \fi:
  }
\prg_generate_conditional_variant:Nnn \tl_if_blank:n
  { V , o } { p , T , F , TF }
%    \end{macrocode}
% \end{macro}
% \end{macro}
%
% \begin{macro}[pTF]{\tl_if_empty:N, \tl_if_empty:c}
%    These functions check whether the token list in the argument is
%    empty and execute the proper code from their argument(s).
%    \begin{macrocode}
\prg_new_conditional:Npnn \tl_if_empty:N #1 { p , T , F , TF }
  {
    \if_meaning:w #1 \c_empty_tl
      \prg_return_true:
    \else:
      \prg_return_false:
    \fi:
  }
\prg_generate_conditional_variant:Nnn \tl_if_empty:N
  { c } { p , T , F , TF }
%    \end{macrocode}
% \end{macro}
%
% \begin{macro}[pTF]{\tl_if_empty:n, \tl_if_empty:V}
%   Convert the argument to a string: this is empty if and only if
%   the argument is.  Then |\if_meaning:w \q_nil ... \q_nil| is
%   \texttt{true} if and only if the string |...| is empty.
%   It could be tempting to use |\if_meaning:w \q_nil #1 \q_nil| directly.
%   This fails on a token
%   list starting with \cs{q_nil} of course but more troubling is the
%   case where argument is a complete conditional such as \cs{if_true:}
%   a \cs{else:} b \cs{fi:} because then \cs{if_true:} is used by
%   \cs{if_meaning:w}, the test turns out \texttt{false}, the \cs{else:}
%   executes the \texttt{false} branch, the \cs{fi:} ends it and the
%   \cs{q_nil} at the end
%   starts executing\dots{}
%    \begin{macrocode}
\prg_new_conditional:Npnn \tl_if_empty:n #1 { p , TF , T , F }
  {
    \exp_after:wN \if_meaning:w \exp_after:wN \q_nil
        \tl_to_str:n {#1} \q_nil
      \prg_return_true:
    \else:
      \prg_return_false:
    \fi:
  }
\prg_generate_conditional_variant:Nnn \tl_if_empty:n
  { V } { p , TF , T , F }
%    \end{macrocode}
% \end{macro}
%
% \begin{macro}[pTF,documented-as=\tl_if_empty:nTF]{\tl_if_empty:o}
% \begin{macro}[EXP]{\@@_if_empty_if:o}
%   The auxiliary function \cs{@@_if_empty_if:o} is for use
%   in various token list conditionals which reduce to testing
%   if a given token list is empty after applying a simple function
%   to it.
%   The test for emptiness is based on \cs{tl_if_empty:nTF}, but
%   the expansion is hard-coded for efficiency, as this auxiliary
%   function is used in several places.
%   We don't put \cs{prg_return_true:} and so on in the definition of
%   the auxiliary, because that would prevent an optimization applied to
%   conditionals that end with this code.
%    \begin{macrocode}
\cs_new:Npn \@@_if_empty_if:o #1
  {
    \exp_after:wN \if_meaning:w \exp_after:wN \q_nil
      \__kernel_tl_to_str:w \exp_after:wN {#1} \q_nil
  }
\prg_new_conditional:Npnn \tl_if_empty:o #1 { p , TF , T , F }
  {
    \@@_if_empty_if:o {#1}
      \prg_return_true:
    \else:
      \prg_return_false:
    \fi:
 }
%    \end{macrocode}
% \end{macro}
% \end{macro}
%
% \begin{macro}[pTF]{\tl_if_eq:NN, \tl_if_eq:Nc, \tl_if_eq:cN, \tl_if_eq:cc}
%   Returns \cs{c_true_bool} if and only if the two token list variables are
%   equal.
%    \begin{macrocode}
\prg_new_conditional:Npnn \tl_if_eq:NN #1#2 { p , T , F , TF }
  {
    \if_meaning:w #1 #2
      \prg_return_true:
    \else:
      \prg_return_false:
    \fi:
  }
\prg_generate_conditional_variant:Nnn \tl_if_eq:NN
  { Nc , c , cc } { p , TF , T , F }
%    \end{macrocode}
% \end{macro}
%
% \begin{macro}[TF]{\tl_if_eq:nn}
% \begin{variable}{\l_@@_internal_a_tl, \l_@@_internal_b_tl}
%   A simple store and compare routine.
%    \begin{macrocode}
\prg_new_protected_conditional:Npnn \tl_if_eq:nn #1#2 { T , F ,  TF }
  {
    \group_begin:
      \tl_set:Nn \l_@@_internal_a_tl {#1}
      \tl_set:Nn \l_@@_internal_b_tl {#2}
      \exp_after:wN
    \group_end:
    \if_meaning:w \l_@@_internal_a_tl \l_@@_internal_b_tl
      \prg_return_true:
    \else:
      \prg_return_false:
    \fi:
  }
\tl_new:N \l_@@_internal_a_tl
\tl_new:N \l_@@_internal_b_tl
%    \end{macrocode}
% \end{variable}
% \end{macro}
%
% \begin{macro}[TF]{\tl_if_in:Nn, \tl_if_in:cn}
%   See \cs{tl_if_in:nnTF} for further comments. Here we simply
%   expand the token list variable and pass it to \cs{tl_if_in:nnTF}.
%    \begin{macrocode}
\cs_new_protected:Npn \tl_if_in:NnT  { \exp_args:No \tl_if_in:nnT  }
\cs_new_protected:Npn \tl_if_in:NnF  { \exp_args:No \tl_if_in:nnF  }
\cs_new_protected:Npn \tl_if_in:NnTF { \exp_args:No \tl_if_in:nnTF }
\prg_generate_conditional_variant:Nnn \tl_if_in:Nn
  { c } { T , F , TF }
%    \end{macrocode}
% \end{macro}
%
% \begin{macro}[TF]{\tl_if_in:nn, \tl_if_in:Vn, \tl_if_in:on, \tl_if_in:no}
%   Once more, the test relies on the emptiness test for robustness.
%   The function \cs{@@_tmp:w} removes tokens until the first occurrence
%   of |#2|. If this does not appear in |#1|, then the final |#2| is removed,
%   leaving an empty token list. Otherwise some tokens remain, and the
%   test is \texttt{false}. See \cs{tl_if_empty:nTF} for details on
%   the emptiness test.
%
%   Treating correctly cases like
%   |\tl_if_in:nnTF {a state}{states}|, where |#1#2| contains |#2| before
%   the end, requires special care.
%   To cater for this case, we insert |{}{}| between the two token
%   lists. This marker may not appear in |#2| because of \TeX{} limitations
%   on what can delimit a parameter, hence we are safe. Using two brace
%   groups makes the test work also for empty arguments.
%   The \cs{if_false:} constructions are a faster way to do
%   \cs{group_align_safe_begin:} and \cs{group_align_safe_end:}.
%    \begin{macrocode}
\prg_new_protected_conditional:Npnn \tl_if_in:nn #1#2 { T  , F , TF }
  {
    \if_false: { \fi:
    \cs_set:Npn \@@_tmp:w ##1 #2 { }
    \tl_if_empty:oTF { \@@_tmp:w #1 {} {} #2 }
      { \prg_return_false: } { \prg_return_true: }
    \if_false: } \fi:
  }
\prg_generate_conditional_variant:Nnn \tl_if_in:nn
  { V , o , no } { T , F , TF }
%    \end{macrocode}
% \end{macro}
%
% \begin{macro}[pTF, EXP]{\tl_if_novalue:n}
% \begin{macro}[EXP]{\@@_if_novalue:w}
%   Tests for |-NoValue-|: this is similar to \cs{tl_if_in:nn} but set
%   up to be expandable and to check the value exactly.  The question
%   mark prevents the auxiliary from losing braces.
%    \begin{macrocode}
\cs_set_protected:Npn \@@_tmp:w #1
  {
    \prg_new_conditional:Npnn \tl_if_novalue:n ##1
      { p , T ,  F , TF }
      {
        \str_if_eq:onTF
          { \@@_if_novalue:w ? ##1 { } #1 }
          { ? { } #1 }
          { \prg_return_true: }
          { \prg_return_false: }
      }
    \cs_new:Npn \@@_if_novalue:w ##1 #1 {##1}
  }
\exp_args:No \@@_tmp:w { \c_novalue_tl }
%    \end{macrocode}
% \end{macro}
% \end{macro}
%
% \begin{macro}[EXP,pTF]{\tl_if_single:N}
%   Expand the token list and feed it to \cs{tl_if_single:n}.
%    \begin{macrocode}
\cs_new:Npn \tl_if_single_p:N { \exp_args:No \tl_if_single_p:n }
\cs_new:Npn \tl_if_single:NT  { \exp_args:No \tl_if_single:nT  }
\cs_new:Npn \tl_if_single:NF  { \exp_args:No \tl_if_single:nF  }
\cs_new:Npn \tl_if_single:NTF { \exp_args:No \tl_if_single:nTF }
%    \end{macrocode}
% \end{macro}
%
% \begin{macro}[EXP,pTF]{\tl_if_single:n}
% \begin{macro}[EXP,pTF]{\@@_if_single:n}
%   This test is similar to \cs{tl_if_empty:nTF}.  Expanding
%   \cs{use_none:nn} |#1| |??| once yields an empty result if |#1| is
%   blank, a single~|?| if |#1| has a single item, and otherwise yields
%   some tokens ending with |??|.  Then, \cs{tl_to_str:n} makes sure
%   there are no odd category codes.  An earlier version would compare
%   the result to a single~|?| using string comparison, but the Lua call
%   is slow in \LuaTeX{}.  Instead, \cs{@@_if_single:nnw} picks the
%   second token in front of it.  If |#1| is empty, this token is
%   the trailing~|?| and the catcode test yields \texttt{false}.  If
%   |#1| has a single item, the token is~|^| and the catcode test
%   yields \texttt{true}.  Otherwise, it is one of the characters
%   resulting from \cs{tl_to_str:n}, and the catcode test yields
%   \texttt{false}.  Note that \cs{if_catcode:w} and
%   \cs{__kernel_tl_to_str:w} are primitives that take care of
%   expansion.
%    \begin{macrocode}
\prg_new_conditional:Npnn \tl_if_single:n #1 { p , T , F , TF }
  {
    \if_catcode:w ^ \exp_after:wN \@@_if_single:nnw
        \__kernel_tl_to_str:w
          \exp_after:wN { \use_none:nn #1 ?? } ^ ? \q_stop
      \prg_return_true:
    \else:
      \prg_return_false:
    \fi:
  }
\cs_new:Npn \@@_if_single:nnw #1#2#3 \q_stop {#2}
%    \end{macrocode}
% \end{macro}
% \end{macro}
%
% \begin{macro}[EXP, noTF]{\tl_case:Nn, \tl_case:cn}
% \begin{macro}[EXP]{\@@_case:nnTF}
% \begin{macro}[EXP]{\@@_case:Nw}
% \begin{macro}[EXP]{\@@_case_end:nw}
%   The aim here is to allow the case statement to be evaluated
%   using a known number of expansion steps (two), and without
%   needing to use an explicit \enquote{end of recursion} marker.
%   That is achieved by using the test input as the final case,
%   as this is always true. The trick is then to tidy up
%   the output such that the appropriate case code plus either
%   the \texttt{true} or \texttt{false} branch code is inserted.
%    \begin{macrocode}
\cs_new:Npn \tl_case:Nn #1#2
  {
    \exp:w
    \@@_case:NnTF #1 {#2} { } { }
  }
\cs_new:Npn \tl_case:NnT #1#2#3
  {
    \exp:w
    \@@_case:NnTF #1 {#2} {#3} { }
  }
\cs_new:Npn \tl_case:NnF #1#2#3
  {
    \exp:w
    \@@_case:NnTF #1 {#2} { } {#3}
  }
\cs_new:Npn \tl_case:NnTF #1#2
  {
    \exp:w
    \@@_case:NnTF #1 {#2}
  }
\cs_new:Npn \@@_case:NnTF #1#2#3#4
  { \@@_case:Nw #1 #2 #1 { } \q_mark {#3} \q_mark {#4} \q_stop }
\cs_new:Npn \@@_case:Nw #1#2#3
  {
    \tl_if_eq:NNTF #1 #2
      { \@@_case_end:nw {#3} }
      { \@@_case:Nw #1 }
  }
\cs_generate_variant:Nn \tl_case:Nn   { c }
\prg_generate_conditional_variant:Nnn \tl_case:Nn
  { c } { T , F , TF }
%    \end{macrocode}
%   To tidy up the recursion, there are two outcomes. If there was a hit to
%   one of the cases searched for, then |#1| is the code to insert,
%   |#2| is the \emph{next} case to check on and |#3| is all of
%   the rest of the cases code. That means that |#4| is the \texttt{true}
%   branch code, and |#5| tidies up the spare \cs{q_mark} and the
%   \texttt{false} branch. On the other hand, if none of the cases matched
%   then we arrive here using the \enquote{termination} case of comparing
%   the search with itself. That means that |#1| is empty, |#2| is
%   the first \cs{q_mark} and so |#4| is the \texttt{false} code (the
%   \texttt{true} code is mopped up by |#3|).
%    \begin{macrocode}
\cs_new:Npn \@@_case_end:nw #1#2#3 \q_mark #4#5 \q_stop
  { \exp_end: #1 #4 }
%    \end{macrocode}
% \end{macro}
% \end{macro}
% \end{macro}
% \end{macro}
%
% \subsection{Mapping to token lists}
%
% \begin{macro}{\tl_map_function:nN}
% \begin{macro}{\tl_map_function:NN, \tl_map_function:cN}
% \begin{macro}{\@@_map_function:Nn}
%   Expandable loop macro for token lists. These have the advantage of not
%   needing to test if the argument is empty, because if it is, the stop
%   marker is read immediately and the loop terminated.
%    \begin{macrocode}
\cs_new:Npn \tl_map_function:nN #1#2
  {
    \@@_map_function:Nn #2 #1
      \q_recursion_tail
    \prg_break_point:Nn \tl_map_break: { }
  }
\cs_new:Npn \tl_map_function:NN
  { \exp_args:No \tl_map_function:nN }
\cs_new:Npn \@@_map_function:Nn #1#2
  {
    \quark_if_recursion_tail_break:nN {#2} \tl_map_break:
    #1 {#2} \@@_map_function:Nn #1
  }
\cs_generate_variant:Nn \tl_map_function:NN { c }
%    \end{macrocode}
% \end{macro}
% \end{macro}
% \end{macro}
%
% \begin{macro}{\tl_map_inline:nn}
% \begin{macro}{\tl_map_inline:Nn, \tl_map_inline:cn}
%   The inline functions are straight forward by now. We use a little
%   trick with the counter \cs{g__kernel_prg_map_int} to make
%   them nestable. We can also make use of \cs{@@_map_function:Nn}
%   from before.
%    \begin{macrocode}
\cs_new_protected:Npn \tl_map_inline:nn #1#2
  {
    \int_gincr:N \g__kernel_prg_map_int
    \cs_gset_protected:cpn
      { @@_map_ \int_use:N \g__kernel_prg_map_int :w } ##1 {#2}
    \exp_args:Nc \@@_map_function:Nn
      { @@_map_ \int_use:N \g__kernel_prg_map_int :w }
      #1 \q_recursion_tail
    \prg_break_point:Nn \tl_map_break:
      { \int_gdecr:N \g__kernel_prg_map_int }
  }
\cs_new_protected:Npn \tl_map_inline:Nn
  { \exp_args:No \tl_map_inline:nn }
\cs_generate_variant:Nn \tl_map_inline:Nn { c }
%    \end{macrocode}
% \end{macro}
% \end{macro}
%
% \begin{macro}{\tl_map_variable:nNn}
% \begin{macro}{\tl_map_variable:NNn, \tl_map_variable:cNn}
% \begin{macro}{\@@_map_variable:Nnn}
%   \cs{tl_map_variable:nNn} \meta{token list} \meta{temp} \meta{action}
%   assigns
%   \meta{temp} to each element and executes \meta{action}.
%    \begin{macrocode}
\cs_new_protected:Npn \tl_map_variable:nNn #1#2#3
  {
    \@@_map_variable:Nnn #2 {#3} #1
      \q_recursion_tail
    \prg_break_point:Nn \tl_map_break: { }
  }
\cs_new_protected:Npn \tl_map_variable:NNn
  { \exp_args:No \tl_map_variable:nNn }
\cs_new_protected:Npn \@@_map_variable:Nnn #1#2#3
  {
    \tl_set:Nn #1 {#3}
    \quark_if_recursion_tail_break:NN #1 \tl_map_break:
    \use:n {#2}
    \@@_map_variable:Nnn #1 {#2}
  }
\cs_generate_variant:Nn \tl_map_variable:NNn { c }
%    \end{macrocode}
% \end{macro}
% \end{macro}
% \end{macro}
%
% \begin{macro}{\tl_map_break:}
% \begin{macro}{\tl_map_break:n}
%   The break statements use the general \cs{prg_map_break:Nn}.
%    \begin{macrocode}
\cs_new:Npn \tl_map_break:
  { \prg_map_break:Nn \tl_map_break: { } }
\cs_new:Npn \tl_map_break:n
  { \prg_map_break:Nn \tl_map_break: }
%    \end{macrocode}
% \end{macro}
% \end{macro}
%
% \subsection{Using token lists}
%
% \begin{macro}{\tl_to_str:n, \tl_to_str:V}
%   Another name for a primitive: defined in \pkg{l3basics}.
%    \begin{macrocode}
\cs_generate_variant:Nn \tl_to_str:n { V }
%    \end{macrocode}
% \end{macro}
%
% \begin{macro}{\tl_to_str:N, \tl_to_str:c}
%    These functions return the replacement text of a token list as a
%    string.
%    \begin{macrocode}
\cs_new:Npn \tl_to_str:N #1 { \__kernel_tl_to_str:w \exp_after:wN {#1} }
\cs_generate_variant:Nn \tl_to_str:N { c }
%    \end{macrocode}
% \end{macro}
%
% \begin{macro}{\tl_use:N, \tl_use:c}
% Token lists which are simply not defined give a clear \TeX{}
% error here. No such luck for ones equal to \cs{scan_stop:} so
% instead a test is made and if there is an issue an error is forced.
%    \begin{macrocode}
\cs_new:Npn \tl_use:N #1
  {
    \tl_if_exist:NTF #1 {#1}
      {
        \__kernel_msg_expandable_error:nnn
          { kernel } { bad-variable } {#1}
      }
  }
\cs_generate_variant:Nn \tl_use:N { c }
%    \end{macrocode}
% \end{macro}
%
% \subsection{Working with the contents of token lists}
%
% \begin{macro}{\tl_count:n, \tl_count:V, \tl_count:o}
% \begin{macro}{\tl_count:N, \tl_count:c}
% \begin{macro}{\@@_count:n}
%   Count number of elements within a token list or token list
%   variable. Brace groups within the list are read as a single
%   element. Spaces are ignored.
%   \cs{@@_count:n} grabs the element and replaces it by |+1|.
%   The |0| ensures that it works on an empty list.
%    \begin{macrocode}
\cs_new:Npn \tl_count:n #1
  {
    \int_eval:n
      { 0 \tl_map_function:nN {#1} \@@_count:n }
  }
\cs_new:Npn \tl_count:N #1
  {
    \int_eval:n
      { 0 \tl_map_function:NN #1 \@@_count:n }
  }
\cs_new:Npn \@@_count:n #1 { + 1 }
\cs_generate_variant:Nn \tl_count:n { V , o }
\cs_generate_variant:Nn \tl_count:N { c }
%    \end{macrocode}
% \end{macro}
% \end{macro}
% \end{macro}
%
% \begin{macro}{\tl_reverse_items:n}
% \begin{macro}{\@@_reverse_items:nwNwn}
% \begin{macro}{\@@_reverse_items:wn}
%    Reversal of a token list is done by taking one item at a time
%    and putting it after \cs{q_stop}.
%    \begin{macrocode}
\cs_new:Npn \tl_reverse_items:n #1
  {
    \@@_reverse_items:nwNwn #1 ?
      \q_mark \@@_reverse_items:nwNwn
      \q_mark \@@_reverse_items:wn
      \q_stop { }
  }
\cs_new:Npn \@@_reverse_items:nwNwn #1 #2 \q_mark #3 #4 \q_stop #5
  {
    #3 #2
      \q_mark \@@_reverse_items:nwNwn
      \q_mark \@@_reverse_items:wn
      \q_stop { {#1} #5 }
  }
\cs_new:Npn \@@_reverse_items:wn #1 \q_stop #2
  { \exp_not:o { \use_none:nn #2 } }
%    \end{macrocode}
% \end{macro}
% \end{macro}
% \end{macro}
%
% \begin{macro}{\tl_trim_spaces:n, \tl_trim_spaces:o}
% \begin{macro}{\tl_trim_spaces_apply:nN, \tl_trim_spaces_apply:oN}
% \begin{macro}
%   {
%     \tl_trim_spaces:N, \tl_trim_spaces:c,
%     \tl_gtrim_spaces:N, \tl_gtrim_spaces:c
%   }
%   Trimming spaces from around the input is deferred to an internal
%   function whose first argument is the token list to trim, augmented
%   by an initial \cs{q_mark}, and whose second argument is a
%   \meta{continuation}, which receives as a braced argument
%   \cs{use_none:n} \cs{q_mark} \meta{trimmed token list}.  In the case
%   at hand, we take \cs{exp_not:o} as our continuation, so that space
%   trimming behaves correctly within an \texttt{x}-type expansion.
%    \begin{macrocode}
\cs_new:Npn \tl_trim_spaces:n #1
  { \@@_trim_spaces:nn { \q_mark #1 } \exp_not:o }
\cs_generate_variant:Nn \tl_trim_spaces:n { o }
\cs_new:Npn \tl_trim_spaces_apply:nN #1#2
  { \@@_trim_spaces:nn { \q_mark #1 } { \exp_args:No #2 } }
\cs_generate_variant:Nn \tl_trim_spaces_apply:nN { o }
\cs_new_protected:Npn \tl_trim_spaces:N #1
  { \tl_set:Nx #1 { \exp_args:No \tl_trim_spaces:n {#1} } }
\cs_new_protected:Npn \tl_gtrim_spaces:N #1
  { \tl_gset:Nx #1 { \exp_args:No \tl_trim_spaces:n {#1} } }
\cs_generate_variant:Nn \tl_trim_spaces:N  { c }
\cs_generate_variant:Nn \tl_gtrim_spaces:N { c }
%    \end{macrocode}
% \end{macro}
% \end{macro}
%
% \begin{macro}{\@@_trim_spaces:nn}
% \begin{macro}
%   {
%     \@@_trim_spaces_auxi:w, \@@_trim_spaces_auxii:w,
%     \@@_trim_spaces_auxiii:w, \@@_trim_spaces_auxiv:w
%   }
%   Trimming spaces from around the input is done using delimited
%   arguments and quarks, and to get spaces at odd places in the
%   definitions, we nest those in \cs{@@_tmp:w}, which then receives
%   a single space as its argument: |#1| is \verb*+ +.
%   Removing leading spaces is done with \cs{@@_trim_spaces_auxi:w},
%   which loops until \cs{q_mark}\verb*+ + matches the end of the token
%   list: then |##1| is the token list and |##3| is
%   \cs{@@_trim_spaces_auxii:w}. This hands the relevant tokens to the
%   loop \cs{@@_trim_spaces_auxiii:w}, responsible for trimming
%   trailing spaces. The end is reached when \verb*+ + \cs{q_nil}
%   matches the one present in the definition of \cs{tl_trim_spacs:n}.
%   Then \cs{@@_trim_spaces_auxiv:w} puts the token list into a group,
%   with \cs{use_none:n} placed there to gobble a lingering \cs{q_mark},
%   and feeds this to the \meta{continuation}.
%    \begin{macrocode}
\cs_set:Npn \@@_tmp:w #1
  {
    \cs_new:Npn \@@_trim_spaces:nn ##1
      {
        \@@_trim_spaces_auxi:w
          ##1
          \q_nil
          \q_mark #1 { }
          \q_mark \@@_trim_spaces_auxii:w
          \@@_trim_spaces_auxiii:w
          #1 \q_nil
          \@@_trim_spaces_auxiv:w
        \q_stop
      }
    \cs_new:Npn \@@_trim_spaces_auxi:w ##1 \q_mark #1 ##2 \q_mark ##3
      {
        ##3
        \@@_trim_spaces_auxi:w
        \q_mark
        ##2
        \q_mark #1 {##1}
      }
    \cs_new:Npn \@@_trim_spaces_auxii:w
        \@@_trim_spaces_auxi:w \q_mark \q_mark ##1
      {
        \@@_trim_spaces_auxiii:w
        ##1
      }
    \cs_new:Npn \@@_trim_spaces_auxiii:w ##1 #1 \q_nil ##2
      {
        ##2
        ##1 \q_nil
        \@@_trim_spaces_auxiii:w
      }
    \cs_new:Npn \@@_trim_spaces_auxiv:w ##1 \q_nil ##2 \q_stop ##3
      { ##3 { \use_none:n ##1 } }
  }
\@@_tmp:w { ~ }
%    \end{macrocode}
% \end{macro}
% \end{macro}
% \end{macro}
%
% \begin{macro}
%   {\tl_sort:Nn, \tl_sort:cn, \tl_gsort:Nn, \tl_gsort:cn, \tl_sort:nN}
%   Implemented in \pkg{l3sort}.
% \end{macro}
%
% \subsection{Token by token changes}
%
% \begin{variable}{\q_@@_act_mark, \q_@@_act_stop}
%   The \cs[no-index]{@@_act_\ldots{}} functions may be applied to any token list.
%   Hence, we use two private quarks, to allow any token, even quarks,
%   in the token list.
%   Only \cs{q_@@_act_mark} and \cs{q_@@_act_stop} may not appear
%   in the token lists manipulated by \cs{@@_act:NNNnn} functions.
%   No quark module yet, so do things by hand.
%    \begin{macrocode}
\cs_new_nopar:Npn \q_@@_act_mark { \q_@@_act_mark }
\cs_new_nopar:Npn \q_@@_act_stop { \q_@@_act_stop }
%    \end{macrocode}
% \end{variable}
%
% \begin{macro}[EXP]{\@@_act:NNNnn}
% \begin{macro}[EXP]{\@@_act_output:n, \@@_act_reverse_output:n}
% \begin{macro}[EXP]{\@@_act_loop:w}
% \begin{macro}[EXP]{\@@_act_normal:NwnNNN}
% \begin{macro}[EXP]{\@@_act_group:nwnNNN}
% \begin{macro}[EXP]{\@@_act_space:wwnNNN}
% \begin{macro}[EXP]{\@@_act_end:w}
%   To help control the expansion, \cs{@@_act:NNNnn} should always
%   be proceeded by \cs{exp:w} and ends by producing \cs{exp_end:}
%   once the result has been obtained. Then loop over tokens,
%   groups, and spaces in |#5|. The marker \cs{q_@@_act_mark}
%   is used both to avoid losing outer braces and to detect the
%   end of the token list more easily. The result is stored
%   as an argument for the dummy function \cs{@@_act_result:n}.
%    \begin{macrocode}
\cs_new:Npn \@@_act:NNNnn #1#2#3#4#5
  {
    \group_align_safe_begin:
    \@@_act_loop:w #5 \q_@@_act_mark \q_@@_act_stop
    {#4} #1 #2 #3
    \@@_act_result:n { }
  }
%    \end{macrocode}
%   In the loop, we check how the token list begins and act
%   accordingly. In the \enquote{normal} case, we may have
%   reached \cs{q_@@_act_mark}, the end of the list. Then
%   leave \cs{exp_end:} and the result in the input stream,
%   to terminate the expansion of \cs{exp:w}.
%   Otherwise, apply the relevant function to the
%   \enquote{arguments}, |#3|
%   and to the head of the token list. Then repeat the loop.
%   The scheme is the same if the token list starts with a
%   group or with a space. Some extra work is needed to
%   make \cs{@@_act_space:wwnNNN} gobble the space.
%    \begin{macrocode}
\cs_new:Npn \@@_act_loop:w #1 \q_@@_act_stop
  {
    \tl_if_head_is_N_type:nTF {#1}
      { \@@_act_normal:NwnNNN }
      {
        \tl_if_head_is_group:nTF {#1}
          { \@@_act_group:nwnNNN }
          { \@@_act_space:wwnNNN }
      }
    #1 \q_@@_act_stop
  }
\cs_new:Npn \@@_act_normal:NwnNNN #1 #2 \q_@@_act_stop #3#4
  {
    \if_meaning:w \q_@@_act_mark #1
      \exp_after:wN \@@_act_end:wn
    \fi:
    #4 {#3} #1
    \@@_act_loop:w #2 \q_@@_act_stop
    {#3} #4
  }
\cs_new:Npn \@@_act_end:wn #1 \@@_act_result:n #2
  { \group_align_safe_end: \exp_end: #2 }
\cs_new:Npn \@@_act_group:nwnNNN #1 #2 \q_@@_act_stop #3#4#5
  {
    #5 {#3} {#1}
    \@@_act_loop:w #2 \q_@@_act_stop
    {#3} #4 #5
  }
\exp_last_unbraced:NNo
  \cs_new:Npn \@@_act_space:wwnNNN \c_space_tl #1 \q_@@_act_stop #2#3#4#5
  {
    #5 {#2}
    \@@_act_loop:w #1 \q_@@_act_stop
    {#2} #3 #4 #5
  }
%    \end{macrocode}
%   Typically, the output is done to the right of what was already output,
%   using \cs{@@_act_output:n}, but for the \cs{@@_act_reverse} functions,
%   it should be done to the left.
%    \begin{macrocode}
\cs_new:Npn \@@_act_output:n #1 #2 \@@_act_result:n #3
  { #2 \@@_act_result:n { #3 #1 } }
\cs_new:Npn \@@_act_reverse_output:n #1 #2 \@@_act_result:n #3
  { #2 \@@_act_result:n { #1 #3 } }
%    \end{macrocode}
% \end{macro}
% \end{macro}
% \end{macro}
% \end{macro}
% \end{macro}
% \end{macro}
% \end{macro}
%
% \begin{macro}[EXP]{\tl_reverse:n, \tl_reverse:o, \tl_reverse:V}
% \begin{macro}[EXP]{\@@_reverse_normal:nN}
% \begin{macro}[EXP]{\@@_reverse_group_preserve:nn}
% \begin{macro}[EXP]{\@@_reverse_space:n}
%   The goal here is to reverse without losing spaces nor braces.
%   This is done using the general internal function \cs{@@_act:NNNnn}.
%   Spaces and \enquote{normal} tokens are output on the left of the current
%   output. Grouped tokens are output to the left but without any reversal
%   within the group. All of the internal functions here drop one argument:
%   this is needed by \cs{@@_act:NNNnn} when changing case (to record
%   which direction the change is in), but not when reversing the tokens.
%    \begin{macrocode}
\cs_new:Npn \tl_reverse:n #1
  {
    \__kernel_exp_not:w \exp_after:wN
      {
        \exp:w
        \@@_act:NNNnn
          \@@_reverse_normal:nN
          \@@_reverse_group_preserve:nn
          \@@_reverse_space:n
          { }
          {#1}
      }
  }
\cs_generate_variant:Nn \tl_reverse:n { o , V }
\cs_new:Npn \@@_reverse_normal:nN #1#2
  { \@@_act_reverse_output:n {#2} }
\cs_new:Npn \@@_reverse_group_preserve:nn #1#2
  { \@@_act_reverse_output:n { {#2} } }
\cs_new:Npn \@@_reverse_space:n #1
  { \@@_act_reverse_output:n { ~ } }
%    \end{macrocode}
% \end{macro}
% \end{macro}
% \end{macro}
% \end{macro}
%
% \begin{macro}{\tl_reverse:N, \tl_reverse:c, \tl_greverse:N, \tl_greverse:c}
%   This reverses the list, leaving \cs{exp_stop_f:} in front,
%   which stops the \texttt{f}-expansion.
%    \begin{macrocode}
\cs_new_protected:Npn \tl_reverse:N #1
  { \tl_set:Nx #1 { \exp_args:No \tl_reverse:n { #1 } } }
\cs_new_protected:Npn \tl_greverse:N #1
  { \tl_gset:Nx #1 { \exp_args:No \tl_reverse:n { #1 } } }
\cs_generate_variant:Nn \tl_reverse:N  { c }
\cs_generate_variant:Nn \tl_greverse:N { c }
%    \end{macrocode}
% \end{macro}
%
% \subsection{The first token from a token list}
%
% \begin{macro}{\tl_head:N, \tl_head:n, \tl_head:V, \tl_head:v, \tl_head:f}
% \begin{macro}{\@@_head_auxi:nw, \@@_head_auxii:n}
% \begin{macro}{\tl_head:w}
% \begin{macro}{\tl_tail:N, \tl_tail:n, \tl_tail:V, \tl_tail:v, \tl_tail:f}
%   Finding the head of a token list expandably always strips braces, which
%   is fine as this is consistent with for example mapping to a list. The
%   empty brace groups in \cs{tl_head:n} ensure that a blank argument gives an
%   empty result. The result is returned within the \tn{unexpanded} primitive.
%   The approach here is to use \cs{if_false:} to allow us to use |}| as
%   the closing delimiter: this is the only safe choice, as any other token
%   would not be able to parse it's own code. Using a marker, we can see if
%   what we are grabbing is exactly the marker, or there is anything else to
%   deal with. Is there is, there is a loop. If not, tidy up and leave the
%   item in the output stream. More detail in
%   \url{http://tex.stackexchange.com/a/70168}.
%    \begin{macrocode}
\cs_new:Npn \tl_head:n #1
  {
    \__kernel_exp_not:w
      \if_false: { \fi: \@@_head_auxi:nw #1 { } \q_stop }
  }
\cs_new:Npn \@@_head_auxi:nw #1#2 \q_stop
  {
    \exp_after:wN \@@_head_auxii:n \exp_after:wN {
      \if_false: } \fi: {#1}
  }
\cs_new:Npn \@@_head_auxii:n #1
  {
    \exp_after:wN \if_meaning:w \exp_after:wN \q_nil
      \__kernel_tl_to_str:w \exp_after:wN { \use_none:n #1 } \q_nil
      \exp_after:wN \use_i:nn
    \else:
      \exp_after:wN \use_ii:nn
    \fi:
      {#1}
      { \if_false: { \fi: \@@_head_auxi:nw #1 } }
  }
\cs_generate_variant:Nn \tl_head:n { V , v , f }
\cs_new:Npn \tl_head:w #1#2 \q_stop {#1}
\cs_new:Npn \tl_head:N { \exp_args:No \tl_head:n }
%    \end{macrocode}
%   To correctly leave the tail of a token list, it's important \emph{not} to
%   absorb any of the tail part as an argument. For example, the simple
%   definition
%   \begin{verbatim}
%     \cs_new:Npn \tl_tail:n #1 { \tl_tail:w #1 \q_stop }
%     \cs_new:Npn \tl_tail:w #1#2 \q_stop
%   \end{verbatim}
%   would give the wrong result for |\tl_tail:n { a { bc } }| (the braces would
%   be stripped). Thus the only safe way to proceed is to first check that
%   there is an item to grab (\emph{i.e.}~that the argument is not blank) and
%   assuming there is to dispose of the first item.  As with \cs{tl_head:n},
%   the result is protected from further expansion by \tn{unexpanded}.
%   While we could optimise the test here, this would leave some tokens
%   \enquote{banned} in the input, which we do not have with this definition.
%    \begin{macrocode}
\cs_new:Npn \tl_tail:n #1
  {
    \__kernel_exp_not:w
      \tl_if_blank:nTF {#1}
        { { } }
        { \exp_after:wN { \use_none:n #1 } }
  }
\cs_generate_variant:Nn \tl_tail:n { V , v , f }
\cs_new:Npn \tl_tail:N { \exp_args:No \tl_tail:n }
%    \end{macrocode}
% \end{macro}
% \end{macro}
% \end{macro}
% \end{macro}
%
% \begin{macro}[pTF]{\tl_if_head_eq_meaning:nN}
% \begin{macro}[pTF]{\tl_if_head_eq_charcode:nN}
% \begin{macro}[pTF]{\tl_if_head_eq_charcode:fN}
% \begin{macro}[pTF]{\tl_if_head_eq_catcode:nN}
%   Accessing the first token of a token list is tricky in three cases:
%   when it has category code $1$ (begin-group token), when it is an
%   explicit space, with category code $10$ and character code $32$, or
%   when the token list is empty (obviously).
%
%   Forgetting temporarily about this issue we would use the following
%   test in \cs{tl_if_head_eq_charcode:nN}.  Here, \cs{tl_head:w} yields
%   the first token of the token list, then passed to \cs{exp_not:N}.
% \begin{verbatim}
% \if_charcode:w
%     \exp_after:wN \exp_not:N \tl_head:w #1 \q_nil \q_stop
%     \exp_not:N #2
% \end{verbatim}
%   The two first special cases are detected by testing if the token
%   list starts with an \texttt{N}-type token (the extra |?| sends empty
%   token lists to the \texttt{true} branch of this test).  In those
%   cases, the first token is a character, and since we only care about
%   its character code, we can use \cs{str_head:n} to access it (this
%   works even if it is a space character).  An empty argument
%   results in \cs{tl_head:w} leaving two tokens: |?| which is taken in
%   the \cs{if_charcode:w} test, and \cs{use_none:nn}, which ensures
%   that \cs{prg_return_false:} is returned regardless of whether the
%   charcode test was \texttt{true} or \texttt{false}.
%    \begin{macrocode}
\prg_new_conditional:Npnn \tl_if_head_eq_charcode:nN #1#2 { p , T , F , TF }
  {
    \if_charcode:w
        \exp_not:N #2
        \tl_if_head_is_N_type:nTF { #1 ? }
          {
            \exp_after:wN \exp_not:N
            \tl_head:w #1 { ? \use_none:nn } \q_stop
          }
          { \str_head:n {#1} }
      \prg_return_true:
    \else:
      \prg_return_false:
    \fi:
  }
\prg_generate_conditional_variant:Nnn \tl_if_head_eq_charcode:nN
  { f } { p , TF , T , F }
%    \end{macrocode}
%   For \cs{tl_if_head_eq_catcode:nN}, again we detect special cases
%   with a \cs{tl_if_head_is_N_type:n}.  Then we need to test if the
%   first token is a begin-group token or an explicit space token, and
%   produce the relevant token, either \cs{c_group_begin_token} or
%   \cs{c_space_token}.  Again, for an empty argument, a hack is used,
%   removing \cs{prg_return_true:} and \cs{else:} with \cs{use_none:nn}
%   in case the catcode test with the (arbitrarily chosen) |?| is
%   \texttt{true}.
%    \begin{macrocode}
\prg_new_conditional:Npnn \tl_if_head_eq_catcode:nN #1 #2 { p , T , F , TF }
  {
    \if_catcode:w
        \exp_not:N #2
        \tl_if_head_is_N_type:nTF { #1 ? }
          {
            \exp_after:wN \exp_not:N
            \tl_head:w #1 { ? \use_none:nn } \q_stop
          }
          {
            \tl_if_head_is_group:nTF {#1}
              { \c_group_begin_token }
              { \c_space_token }
          }
      \prg_return_true:
    \else:
      \prg_return_false:
    \fi:
  }
%    \end{macrocode}
%   For \cs{tl_if_head_eq_meaning:nN}, again, detect special cases.  In
%   the normal case, use \cs{tl_head:w}, with no \cs{exp_not:N} this
%   time, since \cs{if_meaning:w} causes no expansion.  With an empty
%   argument, the test is \texttt{true}, and \cs{use_none:nnn} removes
%   |#2| and the usual \cs{prg_return_true:} and \cs{else:}.
%   In the special cases, we know that the first token is a character,
%   hence \cs{if_charcode:w} and \cs{if_catcode:w} together are enough.
%   We combine them in some order, hopefully faster than the reverse.
%   Tests are not nested because the arguments may contain unmatched
%   primitive conditionals.
%    \begin{macrocode}
\prg_new_conditional:Npnn \tl_if_head_eq_meaning:nN #1#2 { p , T , F , TF }
  {
    \tl_if_head_is_N_type:nTF { #1 ? }
      { \@@_if_head_eq_meaning_normal:nN }
      { \@@_if_head_eq_meaning_special:nN }
    {#1} #2
  }
\cs_new:Npn \@@_if_head_eq_meaning_normal:nN #1 #2
  {
    \exp_after:wN \if_meaning:w
        \tl_head:w #1 { ?? \use_none:nnn } \q_stop #2
      \prg_return_true:
    \else:
      \prg_return_false:
    \fi:
  }
\cs_new:Npn \@@_if_head_eq_meaning_special:nN #1 #2
  {
    \if_charcode:w \str_head:n {#1} \exp_not:N #2
      \exp_after:wN \use:n
    \else:
      \prg_return_false:
      \exp_after:wN \use_none:n
    \fi:
    {
      \if_catcode:w \exp_not:N #2
                    \tl_if_head_is_group:nTF {#1}
                      { \c_group_begin_token }
                      { \c_space_token }
        \prg_return_true:
      \else:
        \prg_return_false:
      \fi:
    }
  }
%    \end{macrocode}
% \end{macro}
% \end{macro}
% \end{macro}
% \end{macro}
%
% \begin{macro}[pTF]{\tl_if_head_is_N_type:n}
% \begin{macro}[EXP]{\@@_if_head_is_N_type:w}
%   A token list can be empty, can start with an explicit space
%   character (catcode 10 and charcode 32), can start with a begin-group
%   token (catcode 1), or start with an \texttt{N}-type argument.  In
%   the first two cases, the line involving \cs{@@_if_head_is_N_type:w}
%   produces~|^| (and otherwise nothing).  In the third case
%   (begin-group token), the lines involving \cs{exp_after:wN} produce a
%   single closing brace.  The category code test is thus true exactly
%   in the fourth case, which is what we want.  One cannot optimize by
%   moving one of the |*| to the beginning: if |#1| contains primitive
%   conditionals, all of its occurrences must be dealt with before the
%   \cs{if_catcode:w} tries to skip the \texttt{true} branch of the
%   conditional.
%    \begin{macrocode}
\prg_new_conditional:Npnn \tl_if_head_is_N_type:n #1 { p , T , F , TF }
  {
    \if_catcode:w
        \if_false: { \fi: \@@_if_head_is_N_type:w ? #1 ~ }
        \exp_after:wN \use_none:n
          \exp_after:wN { \exp_after:wN { \token_to_str:N #1 ? } }
        * *
      \prg_return_true:
    \else:
      \prg_return_false:
    \fi:
  }
\cs_new:Npn \@@_if_head_is_N_type:w #1 ~
  {
    \tl_if_empty:oTF { \use_none:n #1 } { ^ } { }
    \exp_after:wN \use_none:n \exp_after:wN { \if_false: } \fi:
  }
%    \end{macrocode}
% \end{macro}
% \end{macro}
%
% \begin{macro}[EXP,pTF]{\tl_if_head_is_group:n}
%   Pass the first token of |#1| through \cs{token_to_str:N},
%   then check for the brace balance. The extra \texttt{?}
%   caters for an empty argument.\footnote{Bruno: this could
%     be made faster, but we don't: if we hope to ever have
%     an e-type argument, we need all brace \enquote{tricks}
%     to happen in one step of expansion, keeping the token
%     list brace balanced at all times.}
%    \begin{macrocode}
\prg_new_conditional:Npnn \tl_if_head_is_group:n #1 { p , T , F , TF }
  {
    \if_catcode:w
        \exp_after:wN \use_none:n
          \exp_after:wN { \exp_after:wN { \token_to_str:N #1 ? } }
        * *
      \prg_return_false:
    \else:
      \prg_return_true:
    \fi:
  }
%    \end{macrocode}
% \end{macro}
%
% \begin{macro}[EXP,pTF]{\tl_if_head_is_space:n}
% \begin{macro}[EXP]{\@@_if_head_is_space:w}
%   The auxiliary's argument is all that is before the first explicit
%   space in |?#1?~|.  If that is a single~|?| the test yields
%   \texttt{true}.  Otherwise, that is more than one token, and the test
%   yields \texttt{false}.  The work is done within braces (with an
%   |\if_false: { \fi: ... }| construction) both to hide potential
%   alignment tab characters from \TeX{} in a table, and to allow for
%   removing what remains of the token list after its first space.  The
%   \cs{exp:w} and \cs{exp_end:} ensure that the result of a
%   single step of expansion directly yields a balanced token list (no
%   trailing closing brace).
%    \begin{macrocode}
\prg_new_conditional:Npnn \tl_if_head_is_space:n #1 { p , T , F , TF }
  {
    \exp:w \if_false: { \fi:
      \@@_if_head_is_space:w ? #1 ? ~ }
  }
\cs_new:Npn \@@_if_head_is_space:w #1 ~
  {
    \tl_if_empty:oTF { \use_none:n #1 }
      { \exp_after:wN \exp_end: \exp_after:wN \prg_return_true: }
      { \exp_after:wN \exp_end: \exp_after:wN \prg_return_false: }
    \exp_after:wN \use_none:n \exp_after:wN { \if_false: } \fi:
  }
%    \end{macrocode}
% \end{macro}
% \end{macro}
%
% \subsection{Using a single item}
%
% \begin{macro}{\tl_item:nn, \tl_item:Nn, \tl_item:cn}
% \begin{macro}{\@@_item_aux:nn, \@@_item:nn}
%   The idea here is to find the offset of the item from the left, then use
%   a loop to grab the correct item. If the resulting offset is too large,
%   then \cs{quark_if_recursion_tail_stop:n} terminates the loop, and returns
%   nothing at all.
%    \begin{macrocode}
\cs_new:Npn \tl_item:nn #1#2
  {
    \exp_args:Nf \@@_item:nn
      { \exp_args:Nf \@@_item_aux:nn { \int_eval:n {#2} } {#1} }
    #1
    \q_recursion_tail
    \prg_break_point:
  }
\cs_new:Npn \@@_item_aux:nn #1#2
  {
    \int_compare:nNnTF {#1} < 0
      { \int_eval:n { \tl_count:n {#2} + 1 + #1 } }
      {#1}
  }
\cs_new:Npn \@@_item:nn #1#2
  {
    \quark_if_recursion_tail_break:nN {#2} \prg_break:
    \int_compare:nNnTF {#1} = 1
      { \prg_break:n { \exp_not:n {#2} } }
      { \exp_args:Nf \@@_item:nn { \int_eval:n { #1 - 1 } } }
  }
\cs_new:Npn \tl_item:Nn { \exp_args:No \tl_item:nn }
\cs_generate_variant:Nn \tl_item:Nn { c }
%    \end{macrocode}
% \end{macro}
% \end{macro}
%
% \subsection{Viewing token lists}
%
% \begin{macro}{\tl_show:N, \tl_show:c, \tl_log:N, \tl_log:c, \@@_show:NN}
%   Showing token list variables is done after checking that the
%   variable is defined (see \cs{__kernel_register_show:N}).
%    \begin{macrocode}
\cs_new_protected:Npn \tl_show:N { \@@_show:NN \tl_show:n }
\cs_generate_variant:Nn \tl_show:N { c }
\cs_new_protected:Npn \tl_log:N { \@@_show:NN \tl_log:n }
\cs_generate_variant:Nn \tl_log:N { c }
\cs_new_protected:Npn \@@_show:NN #1#2
  {
    \__kernel_chk_defined:NT #2
      { \exp_args:Nx #1 { \token_to_str:N #2 = \exp_not:o {#2} } }
  }
%    \end{macrocode}
% \end{macro}
%
% \begin{macro}{\tl_show:n, \@@_show:n}
% \begin{macro}[EXP]{\@@_show:w}
%   Many |show| functions are based on \cs{tl_show:n}.
%   The argument of \cs{tl_show:n} is line-wrapped using
%   \cs{iow_wrap:nnnN} but with a leading |>~| and trailing period, both
%   removed before passing the wrapped text to the \tn{showtokens}
%   primitive.  This primitive shows the result with a leading |>~| and
%   trailing period.
%
%   The token list \cs{l_@@_internal_a_tl} containing the result
%   of all these manipulations is displayed to the terminal using
%   \cs{tex_showtokens:D} and an odd \cs{exp_after:wN} which expand the
%   closing brace to improve the output slightly.  The calls to
%   \cs{__kernel_iow_with:Nnn} ensure that the \tn{newlinechar} is set to~$10$
%   so that the \cs{iow_newline:} inserted by the line-wrapping code
%   are correctly recognized by \TeX{}, and that \tn{errorcontextlines}
%   is $-1$ to avoid printing irrelevant context.
%    \begin{macrocode}
\cs_new_protected:Npn \tl_show:n #1
  { \iow_wrap:nnnN { >~ \tl_to_str:n {#1} . } { } { } \@@_show:n }
\cs_new_protected:Npn \@@_show:n #1
  {
    \tl_set:Nf \l_@@_internal_a_tl { \@@_show:w #1 \q_stop }
    \__kernel_iow_with:Nnn \tex_newlinechar:D { 10 }
      {
        \__kernel_iow_with:Nnn \tex_errorcontextlines:D { -1 }
          {
            \tex_showtokens:D \exp_after:wN \exp_after:wN \exp_after:wN
              { \exp_after:wN \l_@@_internal_a_tl }
          }
      }
  }
\cs_new:Npn \@@_show:w #1 > #2 . \q_stop {#2}
%    \end{macrocode}
% \end{macro}
% \end{macro}
%
% \begin{macro}{\tl_log:n}
%   Logging is much easier, simply line-wrap.  The |>~| and trailing
%   period is there to match the output of \cs{tl_show:n}.
%    \begin{macrocode}
\cs_new_protected:Npn \tl_log:n #1
  { \iow_wrap:nnnN { > ~ \tl_to_str:n {#1} . } { } { } \iow_log:n }
%    \end{macrocode}
% \end{macro}
%
% \subsection{Scratch token lists}
%
% \begin{variable}{\g_tmpa_tl, \g_tmpb_tl}
%    Global temporary token list variables.
%    They are supposed to be set and used immediately,
%    with no delay between the definition and the use because you
%    can't count on other macros not to redefine them from under you.
%    \begin{macrocode}
\tl_new:N \g_tmpa_tl
\tl_new:N \g_tmpb_tl
%    \end{macrocode}
% \end{variable}
%
% \begin{variable}{\l_tmpa_tl, \l_tmpb_tl}
%    These are local temporary token list variables. Be sure not to assume
%    that the value you put into them will survive for
%    long---see discussion above.
%    \begin{macrocode}
\tl_new:N \l_tmpa_tl
\tl_new:N \l_tmpb_tl
%    \end{macrocode}
% \end{variable}
%
%    \begin{macrocode}
%</initex|package>
%    \end{macrocode}
%
% \end{implementation}
%
% \PrintIndex
