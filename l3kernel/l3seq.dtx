% \iffalse meta-comment
%
%% File: l3seq.dtx Copyright (C) 1990-2011 The LaTeX3 Project
%%
%% It may be distributed and/or modified under the conditions of the
%% LaTeX Project Public License (LPPL), either version 1.3c of this
%% license or (at your option) any later version.  The latest version
%% of this license is in the file
%%
%%    http://www.latex-project.org/lppl.txt
%%
%% This file is part of the "expl3 bundle" (The Work in LPPL)
%% and all files in that bundle must be distributed together.
%%
%% The released version of this bundle is available from CTAN.
%%
%% -----------------------------------------------------------------------
%%
%% The development version of the bundle can be found at
%%
%%    http://www.latex-project.org/svnroot/experimental/trunk/
%%
%% for those people who are interested.
%%
%%%%%%%%%%%
%% NOTE: %%
%%%%%%%%%%%
%%
%%   Snapshots taken from the repository represent work in progress and may
%%   not work or may contain conflicting material!  We therefore ask
%%   people _not_ to put them into distributions, archives, etc. without
%%   prior consultation with the LaTeX3 Project.
%%
%% -----------------------------------------------------------------------
%
%<*driver|package>
\RequirePackage{l3names}
\GetIdInfo$Id$
  {L3 Experimental sequences and stacks}
%</driver|package>
%<*driver>
\documentclass[full]{l3doc}
\begin{document}
  \DocInput{\jobname.dtx}
\end{document}
%</driver>
% \fi
%
% \title{^^A
%   The \pkg{l3seq} package\\ Sequences and stacks^^A
%   \thanks{This file describes v\fileversion, last revised \filedate.}^^A
% }
%
% \author{^^A
%  The \LaTeX3 Project\thanks
%    {^^A
%      E-mail:
%        \href{mailto:latex-team@latex-project.org}
%          {latex-team@latex-project.org}^^A
%    }^^A
% }
%
% \date{Released \filedate}
%
% \maketitle
%
% \begin{documentation}
%
% \LaTeX3 implements a \enquote{sequence} data type, which contain
% an ordered list of entries which may contain any \meta{balanced text}.
% It is possible to map functions to sequences such that the function
% is applied to every item in the sequence.
%
% Sequences are also used to implement stack functions in \LaTeX3. This
% is achieved using a number of dedicated stack functions.
%
% \section{Creating and initialising sequences}
%
% \begin{function}{\seq_new:N, \seq_new:c}
%   \begin{syntax}
%     \cs{seq_new:N} \meta{sequence}
%   \end{syntax}
%   Creates a new \meta{sequence} or raises an error if the name is
%   already taken. The declaration is global. The \meta{sequence} will
%   initially contain no items.
% \end{function}
%
% \begin{function}{\seq_clear:N, \seq_clear:c}
%   \begin{syntax}
%     \cs{seq_clear:N} \meta{sequence}
%   \end{syntax}
%   Clears all items from the \meta{sequence} within the scope of
%   the current \TeX{} group.
% \end{function}
%
% \begin{function}{\seq_gclear:N, \seq_gclear:c}
%   \begin{syntax}
%     \cs{seq_gclear:N} \meta{sequence}
%   \end{syntax}
%   Clears all entries from the \meta{sequence} globally.
% \end{function}
%
% \begin{function}{\seq_clear_new:N, \seq_clear_new:c}
%   \begin{syntax}
%     \cs{seq_clear_new:N} \meta{sequence}
%   \end{syntax}
%   If the \meta{sequence} already exists, clears it within the scope
%   of the current \TeX{} group. If the \meta{sequence} is not defined,
%   it will be created (using \cs{seq_new:N}). Thus the sequence is
%   guaranteed to be available and clear within the current \TeX{}
%   group. The \meta{sequence} will exist globally, but the content
%   outside of the current \TeX{} group is not specified.
% \end{function}
%
% \begin{function}{\seq_gclear_new:N, \seq_gclear_new:c}
%   \begin{syntax}
%     \cs{seq_gclear_new:N} \meta{sequence}
%   \end{syntax}
%   If the \meta{sequence} already exists, clears it globally. If the
%   \meta{sequence} is not defined, it will be created (using
%   \cs{seq_new:N}). Thus the sequence is guaranteed to be available
%   and globally clear.
% \end{function}
%
% \begin{function}
%   {\seq_set_eq:NN, \seq_set_eq:cN, \seq_set_eq:Nc, \seq_set_eq:cc}
%   \begin{syntax}
%     \cs{seq_set_eq:NN} \meta{sequence1} \meta{sequence2}
%   \end{syntax}
%   Sets the content of \meta{sequence1} equal to that of
%   \meta{sequence2}. This assignment is restricted to the current
%   \TeX{} group level.
% \end{function}
%
% \begin{function}
%   {\seq_gset_eq:NN, \seq_gset_eq:cN, \seq_gset_eq:Nc, \seq_gset_eq:cc}
%   \begin{syntax}
%     \cs{seq_gset_eq:NN} \meta{sequence1} \meta{sequence2}
%   \end{syntax}
%   Sets the content of \meta{sequence1} equal to that of
%   \meta{sequence2}. This assignment is global and so is not
%   limited by the current \TeX{} group level.
% \end{function}
%
% \begin{function}{\seq_concat:NNN, \seq_concat:ccc}
%   \begin{syntax}
%     \cs{seq_concat:NNN} \meta{sequence1} \meta{sequence2} \meta{sequence3}
%   \end{syntax}
%   Concatenates the content of \meta{sequence2} and \meta{sequence3}
%   together and saves the result in \meta{sequence1}. The items in
%   \meta{sequence2} will be placed at the left side of the new sequence.
%   This  operation is local to the current \TeX{} group and will
%   remove any existing content in \meta{sequence1}.
% \end{function}
%
% \begin{function}{\seq_gconcat:NNN, \seq_gconcat:ccc}
%   \begin{syntax}
%     \cs{seq_gconcat:NNN} \meta{sequence1} \meta{sequence2} \meta{sequence3}
%   \end{syntax}
%   Concatenates the content of \meta{sequence2} and \meta{sequence3}
%   together and saves the result in \meta{sequence1}. The items in
%   \meta{sequence2}  will be placed at the left side of the new sequence.
%   This operation is global and will remove any existing content in
%   \meta{sequence1}.
% \end{function}
%
% \section{Appending data to sequences}
%
% \begin{function}{
%   \seq_put_left:Nn, \seq_put_left:NV, \seq_put_left:Nv,
%   \seq_put_left:No, \seq_put_left:Nx,
%   \seq_put_left:cn, \seq_put_left:cV, \seq_put_left:cv,
%   \seq_put_left:co, \seq_put_left:cx
% }
%   \begin{syntax}
%     \cs{seq_put_left:Nn} \meta{sequence} \Arg{item}
%   \end{syntax}
%   Appends the \meta{item} to the left of the \meta{sequence}.
%   The assignment is restricted to the current \TeX{} group.
% \end{function}
%
% \begin{function}{
%   \seq_gput_left:Nn, \seq_gput_left:NV, \seq_gput_left:Nv,
%   \seq_gput_left:No, \seq_gput_left:Nx,
%   \seq_gput_left:cn, \seq_gput_left:cV, \seq_gput_left:cv,
%   \seq_gput_left:co, \seq_gput_left:cx
% }
%   \begin{syntax}
%     \cs{seq_gput_left:Nn} \meta{sequence} \Arg{item}
%   \end{syntax}
%   Appends the \meta{item} to the left of the \meta{sequence}.
%   The assignment is global.
% \end{function}
%
% \begin{function}{
%   \seq_put_right:Nn, \seq_put_right:NV, \seq_put_right:Nv,
%   \seq_put_right:No, \seq_put_right:Nx,
%   \seq_put_right:cn, \seq_put_right:cV, \seq_put_right:cv,
%   \seq_put_right:co, \seq_put_right:cx
% }
%   \begin{syntax}
%     \cs{seq_put_right:Nn} \meta{sequence} \Arg{item}
%   \end{syntax}
%   Appends the \meta{item} to the right of the \meta{sequence}.
%   The assignment is restricted to the current \TeX{} group.
% \end{function}
%
% \begin{function}{
%   \seq_gput_right:Nn, \seq_gput_right:NV, \seq_gput_right:Nv,
%   \seq_gput_right:No, \seq_gput_right:Nx,
%   \seq_gput_right:cn, \seq_gput_right:cV, \seq_gput_right:cv,
%   \seq_gput_right:co, \seq_gput_right:cx
% }
%   \begin{syntax}
%     \cs{seq_gput_right:Nn} \meta{sequence} \Arg{item}
%   \end{syntax}
%   Appends the \meta{item} to the right of the \meta{sequence}.
%   The assignment is global.
% \end{function}
%
% \section{Recovering items from sequences}
%
% Items can be recovered from either the left or the right of sequences.
% For implementation reasons, the actions at the left of the sequence are
% faster than those acting on the right. These functions all assign the
% recovered material locally, \emph{i.e.}~setting the
% \meta{token list variable} used with \cs{tl_set:Nn} and \emph{never}
% \cs{tl_gset:Nn}.
%
% \begin{function}{\seq_get_left:NN, \seq_get_left:cN}
%   \begin{syntax}
%     \cs{seq_get_left:NN} \meta{sequence} \meta{token list variable}
%   \end{syntax}
%   Stores the left-most item from a \meta{sequence} in the
%   \meta{token list variable} without removing it from the
%   \meta{sequence}. The \meta{token list variable} is assigned locally.
%   If \meta{sequence} is empty an error will be raised.
% \end{function}
%
% \begin{function}{\seq_get_right:NN, \seq_get_right:cN}
%   \begin{syntax}
%     \cs{seq_get_right:NN} \meta{sequence} \meta{token list variable}
%   \end{syntax}
%   Stores the right-most item from a \meta{sequence} in the
%   \meta{token list variable} without removing it from the
%   \meta{sequence}. The \meta{token list variable} is assigned locally.
%   If \meta{sequence} is empty an error will be raised.
% \end{function}
%
% \begin{function}{\seq_pop_left:NN, \seq_pop_left:cN}
%   \begin{syntax}
%     \cs{seq_pop_left:NN} \meta{sequence} \meta{token list variable}
%   \end{syntax}
%   Pops the left-most item from a \meta{sequence} into the
%   \meta{token list variable}, \emph{i.e.}~removes the item from the
%   sequence and stores it in the \meta{token list variable}.
%   Both of the variables are assigned locally. If \meta{sequence} is
%   empty an error will be raised.
% \end{function}
%
% \begin{function}{\seq_gpop_left:NN, \seq_gpop_left:cN}
%   \begin{syntax}
%     \cs{seq_gpop_left:NN} \meta{sequence} \meta{token list variable}
%   \end{syntax}
%   Pops the left-most item from a \meta{sequence} into the
%   \meta{token list variable}, \emph{i.e.}~removes the item from the
%   sequence and stores it in the \meta{token list variable}.
%   The \meta{sequence} is modified globally, while the assignment of
%   the \meta{token list variable} is local.
%   If \meta{sequence} is  empty an error will be raised.
% \end{function}
%
% \begin{function}{\seq_pop_right:NN, \seq_pop_right:cN}
%   \begin{syntax}
%     \cs{seq_pop_right:NN} \meta{sequence} \meta{token list variable}
%   \end{syntax}
%   Pops the right-most item from a \meta{sequence} into the
%   \meta{token list variable}, \emph{i.e.}~removes the item from the
%   sequence and stores it in in the \meta{token list variable}.
%   Both of the variables are assigned locally. If \meta{sequence} is
%   empty an error will be raised.
% \end{function}
%
% \begin{function}{\seq_gpop_right:NN, \seq_gpop_right:cN}
%   \begin{syntax}
%     \cs{seq_gpop_right:NN} \meta{sequence} \meta{token list variable}
%   \end{syntax}
%   Pops the right-most item from a \meta{sequence} into the
%   \meta{token list variable}, \emph{i.e.}~removes the item from the
%   sequence and stores it in the \meta{token list variable}.
%   The \meta{sequence} is modified globally, while the assignment of
%   the \meta{token list variable} is local.
%   If \meta{sequence} is  empty an error will be raised.
% \end{function}
%
% \section{Modifying sequences}
%
%  While sequences are normally used as ordered lists, it may be
%  necessary to modify the content. The functions here may be used
%  to update sequences, while retaining the order of the unaffected
%  entries.
%
% \begin{function}{\seq_remove_duplicates:N, \seq_remove_duplicates:c}
%   \begin{syntax}
%     \cs{seq_remove_duplicates:N} \meta{sequence}
%   \end{syntax}
%   Removes duplicate items from the \meta{sequence}, leaving the
%   left most copy of each item in the \meta{sequence}.  The \meta{item}
%   comparison takes place on a token basis, as for \cs{tl_if_eq:nn(TF)}.
%   The removal is local to the current \TeX{} group.
%   \begin{texnote}
%     This function iterates through every item in the \meta{sequence} and
%     does a comparison with the \meta{items} already checked. It is therefore
%     relatively slow with large sequences.
%   \end{texnote}
% \end{function}
%
% \begin{function}{\seq_gremove_duplicates:N, \seq_gremove_duplicates:c}
%   \begin{syntax}
%     \cs{seq_gremove_duplicates:N} \meta{sequence}
%   \end{syntax}
%   Removes duplicate items from the \meta{sequence}, leaving the
%   left most copy of each item in the \meta{sequence}.  The \meta{item}
%   comparison takes place on a token basis, as for \cs{tl_if_eq:nn(TF)}.
%   The removal is applied globally.
%   \begin{texnote}
%     This function iterates through every item in the \meta{sequence} and
%     does a comparison with the \meta{items} already checked. It is therefore
%     relatively slow with large sequences.
%   \end{texnote}
% \end{function}
%
% \begin{function}{\seq_remove_all:Nn, \seq_remove_all:cn}
%   \begin{syntax}
%     \cs{seq_remove_all:Nn} \meta{sequence} \Arg{item}
%   \end{syntax}
%   Removes every occurrence of \meta{item} from the \meta{sequence}.
%   The \meta{item} comparison takes place on a token basis, as for
%   \cs{tl_if_eq:nn(TF)}. The  removal is local to the current \TeX{} group.
% \end{function}
%
% \begin{function}{\seq_gremove_all:Nn, \seq_gremove_all:cn}
%   \begin{syntax}
%     \cs{seq_gremove_all:Nn} \meta{sequence} \Arg{item}
%   \end{syntax}
%   Removes each occurrence of \meta{item} from the \meta{sequence}.
%   The \meta{item} comparison takes place on a token basis, as for
%   \cs{tl_if_eq:nn(TF)}. The removal is applied globally.
% \end{function}
%
% \section{Sequence conditionals}
%
% \begin{function}[EXP,pTF]{\seq_if_empty:N, \seq_if_empty:c}
%   \begin{syntax}
%     \cs{seq_if_empty_p:N} \meta{sequence}
%     \cs{seq_if_empty:NTF} \meta{sequence} \Arg{true code} \Arg{false code}
%   \end{syntax}
%   Tests if the \meta{sequence} is empty (containing no items). The
%   branching versions then leave either \meta{true code} or
%   \meta{false code} in the input stream, as appropriate to the truth of
%   the test and the variant of the function chosen. The logical truth of
%   the test is left in the input stream by the predicate version.
% \end{function}
%
% \begin{function}[TF]{
%    \seq_if_in:Nn, \seq_if_in:NV, \seq_if_in:Nv, \seq_if_in:No, \seq_if_in:Nx,
%    \seq_if_in:cn, \seq_if_in:cV, \seq_if_in:cv, \seq_if_in:co, \seq_if_in:cx
% }
%   \begin{syntax}
%     \cs{seq_if_in:NnTF} \meta{sequence} \Arg{item}
%     ~~\Arg{true code} \Arg{false code}
%   \end{syntax}
%   Tests if the \meta{item} is present in the \meta{sequence}.
%   Either the \meta{true code} or \meta{false code} is left in the input
%   stream, as appropriate to the truth of the test and the variant of the
%   function
%   chosen.
% \end{function}
%
% \section{Mapping to sequences}
%
% \begin{function}[EXP]{\seq_map_function:NN, \seq_map_function:cN}
%   \begin{syntax}
%     \cs{seq_map_function:NN} \meta{sequence} \meta{function}
%   \end{syntax}
%   Applies \meta{function} to every \meta{item} stored in the
%   \meta{sequence}. The \meta{function} will receive one argument for
%   each iteration. The \meta{items} are returned from left to right.
%   The function \cs{seq_map_inline:Nn} is in general more efficient
%   than \cs{seq_map_function:NN}.
%   One mapping may be nested inside another.
% \end{function}
%
% \begin{function}{\seq_map_inline:Nn, \seq_map_inline:cn}
%   \begin{syntax}
%     \cs{seq_map_inline:Nn} \meta{sequence} \Arg{inline function}
%   \end{syntax}
%   Applies \meta{inline function} to every \meta{item} stored
%   within the \meta{sequence}. The \meta{inline function} should
%   consist of code which will receive the \meta{item} as |#1|.
%   One in line mapping can be nested inside another. The \meta{items}
%   are returned from left to right.
% \end{function}
%
% \begin{function}{
%   \seq_map_variable:NNn, \seq_map_variable:Ncn,
%   \seq_map_variable:cNn, \seq_map_variable:ccn
% }
%   \begin{syntax}
%     \cs{seq_map_variable:NNn} \meta{sequence}
%     ~~\meta{tl~var.} \Arg{function using tl~var.}
%   \end{syntax}
%   Stores each entry in the \meta{sequence} in turn in the
%   \meta{tl~var.}\ and applies the \meta{function using tl~var.}
%   The \meta{function} will usually consist of code making use of
%   the \meta{tl~var.}, but this is not enforced.  One variable
%   mapping can be nested inside another. The \meta{items}
%   are returned from left to right.
% \end{function}
%
% \begin{function}[EXP]{\seq_map_break:}
%   \begin{syntax}
%     \cs{seq_map_break:}
%   \end{syntax}
%   Used to terminate a \cs{seq_map_\ldots} function before all
%   entries in the \meta{sequence} have been processed. This will
%   normally take place within a conditional statement, for example
%   \begin{verbatim}
%     \seq_map_inline:Nn \l_my_seq
%       {
%         \str_if_eq:nnTF { #1 } { bingo }
%           { \seq_map_break: }
%           {
%             % Do something useful
%           }
%       }
%   \end{verbatim}
%   Use outside of a \cs{seq_map_\ldots} scenario will lead to low
%   level \TeX{} errors.
%   \begin{texnote}
%     When the mapping is broken, additional tokens may be inserted by the
%     internal macro \cs{seq_break_point:n} before further items are taken
%     from the input stream. This will depend on the design of the mapping
%     function.
%   \end{texnote}
% \end{function}
%
% \begin{function}[EXP]{\seq_map_break:n}
%   \begin{syntax}
%     \cs{seq_map_break:n} \Arg{tokens}
%   \end{syntax}
%   Used to terminate a \cs{seq_map_\ldots} function before all
%   entries in the \meta{sequence} have been processed, inserting
%   the \meta{tokens} after the mapping has ended. This will
%   normally take place within a conditional statement, for example
%   \begin{verbatim}
%     \seq_map_inline:Nn \l_my_seq
%       {
%         \str_if_eq:nnTF { #1 } { bingo }
%           { \seq_map_break:n { <tokens> } }
%           {
%             % Do something useful
%           }
%       }
%   \end{verbatim}
%   Use outside of a \cs{seq_map_\ldots} scenario will lead to low
%   level \TeX{} errors.
%   \begin{texnote}
%     When the mapping is broken, additional tokens may be inserted by the
%     internal macro \cs{seq_break_point:n} before the \meta{tokens} are
%     inserted into the input stream.
%     This will depend on the design of the mapping function.
%   \end{texnote}
% \end{function}
%
% \section{Sequences as stacks}
%
% Sequences can be used as stacks, where data is pushed to and popped
% from the top of the sequence. (The left of a sequence is the top, for
% performance reasons.) The stack functions for sequences are not
% intended to be mixed with the general ordered data functions detailed
% in the previous section: a sequence should either be used as an
% ordered data type or as a stack, but not in both ways.
%
% \begin{function}{\seq_get:NN, \seq_get:cN}
%   \begin{syntax}
%     \cs{seq_get:NN} \meta{sequence} \meta{token list variable}
%   \end{syntax}
%   Reads the top item from a \meta{sequence} into the
%   \meta{token list variable} without removing it from the
%   \meta{sequence}. The \meta{token list variable} is assigned locally.
%   If \meta{sequence} is empty an error will be raised.
% \end{function}
%
% \begin{function}{\seq_pop:NN, \seq_pop:cN}
%   \begin{syntax}
%     \cs{seq_pop:NN} \meta{sequence} \meta{token list variable}
%   \end{syntax}
%   Pops the top item from a \meta{sequence} into the
%   \meta{token list variable}. Both of the variables are assigned
%   locally. If \meta{sequence} is empty an error will be raised.
% \end{function}
%
% \begin{function}{\seq_gpop:NN, \seq_gpop:cN}
%   \begin{syntax}
%     \cs{seq_gpop:NN} \meta{sequence} \meta{token list variable}
%   \end{syntax}
%   Pops the top item from a \meta{sequence} into the
%   \meta{token list variable}. The \meta{sequence} is modified globally,
%   while the \meta{token list variable} is assigned locally. If
%   \meta{sequence} is empty an error will be raised.
% \end{function}
%
% \begin{function}
%   {
%     \seq_push:Nn, \seq_push:NV, \seq_push:Nv, \seq_push:No, \seq_push:Nx,
%      seq_push:cn, \seq_push:cV, \seq_push:cv, \seq_push:co, \seq_push:cx
%   }
%   \begin{syntax}
%     \cs{seq_push:Nn} \meta{sequence} \Arg{item}
%   \end{syntax}
%   Adds the \Arg{item} to the top of the \meta{sequence}.
%   The assignment is restricted to the current \TeX{}  group.
% \end{function}
%
% \begin{function}
%   {
%     \seq_gpush:Nn, \seq_gpush:NV, \seq_gpush:Nv,
%     \seq_gpush:No, \seq_gpush:Nx,
%     \seq_gpush:cn, \seq_gpush:cV, \seq_gpush:cv,
%     \seq_gpush:co, \seq_gpush:cx
%   }
%   \begin{syntax}
%     \cs{seq_gpush:Nn} \meta{sequence} \Arg{item}
%   \end{syntax}
%   Pushes the \meta{item} onto the end of the top of the
%   \meta{sequence}. The assignment is global.
% \end{function}
%
% \section{Viewing sequences}
%
% \begin{function}{\seq_show:N, \seq_show:c}
%   \begin{syntax}
%     \cs{seq_show:N} \meta{sequence}
%   \end{syntax}
%   Displays the entries in the \meta{sequence} in the terminal.
% \end{function}
%
% \section{Experimental sequence functions}
%
% This section contains functions which may or may not be retained, depending
% on how useful they are found to be.
%
% \begin{function}[TF]{\seq_get_left:NN, \seq_get_left:cN}
%   \begin{syntax}
%     \cs{seq_get_left:NNTF} \meta{sequence} \meta{token list variable}
%     ~~\Arg{true code} \Arg{false code}
%   \end{syntax}
%   If the \meta{sequence} is empty, leaves the \meta{false code} in the
%   input stream and leaves the \meta{token list variable} unchanged. If the
%   \meta{sequence} is non-empty, stores the left-most item from a \meta{sequence}
%   in the \meta{token list variable} without removing it from a
%   \meta{sequence}. The \meta{true code} is then left in the input stream.
%   The \meta{token list variable} is assigned locally.
% \end{function}
%
% \begin{function}[TF]{\seq_get_right:NN, \seq_get_right:cN}
%   \begin{syntax}
%     \cs{seq_get_right:NNTF} \meta{sequence} \meta{token list variable}
%     ~~\Arg{true code} \Arg{false code}
%   \end{syntax}
%   If the \meta{sequence} is empty, leaves the \meta{false code} in the
%   input stream and leaves the \meta{token list variable} unchanged. If the
%   \meta{sequence} is non-empty, stores the right-most item from a \meta{sequence}
%   in the \meta{token list variable} without removing it from a
%   \meta{sequence}. The \meta{true code} is then left in the input stream.
%   The \meta{token list variable} is assigned locally.
% \end{function}
%
% \begin{function}[TF]{\seq_pop_left:NN, \seq_pop_left:cN}
%   \begin{syntax}
%     \cs{seq_pop_left:NNTF} \meta{sequence} \meta{token list variable}
%     ~~\Arg{true code} \Arg{false code}
%   \end{syntax}
%   If the \meta{sequence} is empty, leaves the \meta{false code} in the
%   input stream and leaves the \meta{token list variable} unchanged. If the
%   \meta{sequence} is non-empty, pops the left-most item from a \meta{sequence}
%   in the \meta{token list variable}, \emph{i.e.}~removes the item from a
%   \meta{sequence}. The \meta{true code} is then left in the input stream.
%   Both the \meta{sequence} and the \meta{token list variable} are assigned
%   locally.
% \end{function}
%
% \begin{function}[TF]{\seq_gpop_left:NN, \seq_gpop_left:cN}
%   \begin{syntax}
%     \cs{seq_gpop_left:NNTF} \meta{sequence} \meta{token list variable}
%     ~~\Arg{true code} \Arg{false code}
%   \end{syntax}
%   If the \meta{sequence} is empty, leaves the \meta{false code} in the
%   input stream and leaves the \meta{token list variable} unchanged. If the
%   \meta{sequence} is non-empty, pops the left-most item from a \meta{sequence}
%   in the \meta{token list variable}, \emph{i.e.}~removes the item from a
%   \meta{sequence}. The \meta{true code} is then left in the input stream.
%   The \meta{sequence} is modified globally, while the \meta{token list variable}
%   is assigned locally.
% \end{function}
%
% \begin{function}[TF]{\seq_pop_right:NN, \seq_pop_right:cN}
%   \begin{syntax}
%     \cs{seq_pop_right:NNTF} \meta{sequence} \meta{token list variable}
%     ~~\Arg{true code} \Arg{false code}
%   \end{syntax}
%   If the \meta{sequence} is empty, leaves the \meta{false code} in the
%   input stream and leaves the \meta{token list variable} unchanged. If the
%   \meta{sequence} is non-empty, pops the right-most item from a \meta{sequence}
%   in the \meta{token list variable}, \emph{i.e.}~removes the item from a
%   \meta{sequence}. The \meta{true code} is then left in the input stream.
%   Both the \meta{sequence} and the \meta{token list variable} are assigned
%   locally.
% \end{function}
%
% \begin{function}[TF]{\seq_gpop_right:NN, \seq_gpop_right:cN}
%   \begin{syntax}
%     \cs{seq_gpop_right:NNTF} \meta{sequence} \meta{token list variable}
%     ~~\Arg{true code} \Arg{false code}
%   \end{syntax}
%   If the \meta{sequence} is empty, leaves the \meta{false code} in the
%   input stream and leaves the \meta{token list variable} unchanged. If the
%   \meta{sequence} is non-empty, pops the right-most item from a \meta{sequence}
%   in the \meta{token list variable}, \emph{i.e.}~removes the item from a
%   \meta{sequence}. The \meta{true code} is then left in the input stream.
%   The \meta{sequence} is modified globally, while the \meta{token list variable}
%   is assigned locally.
% \end{function}
%
% \begin{function}[EXP]{\seq_length:N, \seq_length:c}
%   \begin{syntax}
%     \cs{seq_length:N} \meta{sequence}
%   \end{syntax}
%   Leaves the number of items in the \meta{sequence} in the input
%   stream as an \meta{integer denotation}. The total number of items
%   in a \meta{sequence} will include those which are empty and duplicates,
%   \emph{i.e.}~every item in a \meta{sequence} is unique.
% \end{function}
%
% \begin{function}[EXP]{\seq_item:Nn, \seq_item:cn}
%   \begin{syntax}
%     \cs{seq_item:Nn} \meta{sequence} \Arg{integer expression}
%   \end{syntax}
%   Indexing items in the \meta{sequence} from $0$ at the top (left), this
%   function will evaluate the \meta{integer expression} and leave the
%   appropriate item from the sequence in the input stream. If the
%   \meta{integer expression} is negative, indexing occurs from the
%   bottom (right) of the sequence. When the \meta{integer expression}
%   is larger than the number of items in the \meta{sequence} (as
%   calculated by \cs{seq_length:N}) then the function will expand to
%   nothing.
% \end{function}
%
% \begin{function}[EXP]{\seq_use:N, \seq_use:c}
%   \begin{syntax}
%     \cs{seq_use:N} \meta{sequence}
%   \end{syntax}
%   Places each \meta{item} in the \meta{sequence} in turn in the input stream.
%   This occurs in an expandable fashion, and is implemented as a mapping.
%   This means that the process may be prematurely terminated using
%   \cs{seq_map_break:} or \cs{seq_map_break:n}. The \meta{items} in the
%   \meta{sequence} will be used from left (top) to right (bottom).
% \end{function}
%
% \begin{function}[EXP]
%   {
%     \seq_mapthread_function:NNN, \seq_mapthread_function:NcN,
%     \seq_mapthread_function:cNN, \seq_mapthread_function:ccN
%   }
%   \begin{syntax}
%     \cs{seq_mapthread_function:NNN} \meta{seq1} \meta{seq2} \meta{function}
%   \end{syntax}
%   Applies \meta{function} to every pair of items
%   \meta{seq1-item}--\meta{seq2-item} from the two sequences, returning
%   items from both sequences from left to right.   The \meta{function} will
%   receive two \texttt{n}-type arguments for each iteration. The  mapping
%   will terminate when
%   the end of either sequence is reached (\emph{i.e.}~whichever sequence has
%   fewer items determines how many iterations
%   occur).
% \end{function}
%
% \begin{function}
%   {
%     \seq_set_from_clist:NN, \seq_set_from_clist:cN,
%     \seq_set_from_clist:Nc, \seq_set_from_clist:cc,
%     \seq_set_from_clist:Nn, \seq_set_from_clist:cn
%   }
%   \begin{syntax}
%     \cs{seq_set_from_clist:NN} \meta{sequence} \meta{comma-list}
%   \end{syntax}
%   Sets the \meta{sequence} within the current \TeX{} group to be equal
%   to the content of the \meta{comma-list}.
% \end{function}
%
% \begin{function}
%   {
%     \seq_gset_from_clist:NN, \seq_gset_from_clist:cN,
%     \seq_gset_from_clist:Nc, \seq_gset_from_clist:cc,
%     \seq_gset_from_clist:Nn, \seq_gset_from_clist:cn
%   }
%   \begin{syntax}
%     \cs{seq_gset_from_clist:NN} \meta{sequence} \meta{comma-list}
%   \end{syntax}
%   Sets the \meta{sequence} globally to equal to the content of the
%   \meta{comma-list}.
% \end{function}
%
% \section{Internal sequence functions}
%
% \begin{function}{\seq_if_empty_err_break:N}
%   \begin{syntax}
%     \cs{seq_if_empty_err_break:N} \meta{sequence}
%   \end{syntax}
%   Tests if the \meta{sequence} is empty, and if so issues an error
%   message before skipping over any tokens up to \cs{seq_break_point:n}.
%   This function is used to avoid more serious errors which would
%   otherwise occur if some internal functions were applied to an
%   empty \meta{sequence}.
% \end{function}
%
% \begin{function}[EXP]{\seq_item:n}
%   \begin{syntax}
%     \cs{seq_item:n} \meta{item}
%   \end{syntax}
%   The internal token used to begin each sequence entry. If expanded
%   outside of a mapping or manipulation function, an error will be
%   raised. The definition should always be set globally.
% \end{function}
%
% \begin{function}{\seq_push_item_def:n, \seq_push_item_def:x}
%   \begin{syntax}
%     \cs{seq_push_item_def:n} \Arg{code}
%   \end{syntax}
%   Saves the definition of \cs{seq_item:n} and redefines it to
%   accept one parameter and expand to \meta{code}. This function
%   should always be balanced by use of \cs{seq_pop_item_def:}.
% \end{function}
%
% \begin{function}{\seq_pop_item_def:}
%   \begin{syntax}
%     \cs{seq_pop_item_def:}
%   \end{syntax}
%   Restores the definition of \cs{seq_item:n} most recently saved by
%   \cs{seq_push_item_def:n}. This function should always be used in
%   a balanced pair with \cs{seq_push_item_def:n}.
% \end{function}
%
% \begin{function}[EXP]{\seq_break:}
%   \begin{syntax}
%     \cs{seq_break:}
%   \end{syntax}
%   Used to terminate sequence functions by gobbling all tokens
%   up to \cs{seq_break_point:n}. This function is a copy of
%   \cs{seq_map_break:}, but is used in situations which are
%   not mappings.
% \end{function}
%
% \begin{function}[EXP]{\seq_break:n}
%   \begin{syntax}
%     \cs{seq_break:n} \Arg{tokens}
%   \end{syntax}
%   Used to terminate sequence functions by gobbling all tokens
%   up to \cs{seq_break_point:n}, then inserting the \meta{tokens}
%   before continuing reading the input stream. This function is a copy
%   of \cs{seq_map_break:n}, but is used in situations which are
%   not mappings.
% \end{function}
%
% \begin{function}[EXP]{\seq_break_point:n}
%   \begin{syntax}
%     \cs{seq_break_point:n} \meta{tokens}
%   \end{syntax}
%   Used to mark the end of a recursion or mapping: the functions
%   \cs{seq_map_break:} and \cs{seq_map_break:n} use this to break out
%   of the loop. After the loop ends, the \meta{tokens} are inserted into
%   the input stream. This occurs even if the the break functions are
%   \emph{not} applied: \cs{seq_break_point:n} is functionally-equivalent
%   in these cases to \cs{use:n}.
% \end{function}
%
% \end{documentation}
%
% \begin{implementation}
%
% \section{\pkg{l3seq} implementation}
%
% \TestFiles{m3seq002,m3seq003}
%
%    \begin{macrocode}
%<*initex|package>
%    \end{macrocode}
%
%    \begin{macrocode}
%<*package>
\ProvidesExplPackage
  {\filename}{\filedate}{\fileversion}{\filedescription}
\package_check_loaded_expl:
%</package>
%    \end{macrocode}
%
% A sequence is a control sequence whose top-level expansion is of
% the form \enquote{\cs{seq_item:n} \marg{item$_0$}
% \ldots \cs{seq_item:n} \marg{item$_{n-1}$}}. An earlier implementation
% used the structure \enquote{\cs{seq_elt:w} \meta{item$_1$}
% \cs{seq_elt_end:} \ldots \cs{seq_elt:w} \meta{item$_n$}
% \cs{seq_elt_end:}}. This allows rapid searching using a delimited
% function, but is not suitable for items containing |{|, |}| and |#|
% tokens, and also leads to the loss of surrounding braces
% around items.
%
% \begin{macro}[int]{\seq_item:n}
%   The delimiter is always defined, but when used incorrectly simply
%   removes its argument and hits an undefined control sequence to
%   raise an error.
%    \begin{macrocode}
\cs_new:Npn \seq_item:n
  {
    \seq_use_error:
    \use_none:n
  }
%    \end{macrocode}
% \end{macro}
%
% \begin{variable}{\l_seq_tmpa_tl, \l_seq_tmpb_tl}
%   Scratch space for various internal uses.
%    \begin{macrocode}
\tl_new:N \l_seq_tmpa_tl
\tl_new:N \l_seq_tmpb_tl
%    \end{macrocode}
% \end{variable}
%
% \subsection{Allocation and initialisation}
%
% \begin{macro}{\seq_new:N,\seq_new:c}
% \UnitTested
%   Internally, sequences are just token lists.
%    \begin{macrocode}
\cs_new_eq:NN \seq_new:N \tl_new:N
\cs_new_eq:NN \seq_new:c \tl_new:c
%    \end{macrocode}
% \end{macro}
%
% \begin{macro}{\seq_clear:N, \seq_clear:c}
% \UnitTested
% \begin{macro}{\seq_gclear:N, \seq_gclear:c}
% \UnitTested
%   Clearing sequences is just the same as clearing token lists.
%    \begin{macrocode}
\cs_new_eq:NN \seq_clear:N  \tl_clear:N
\cs_new_eq:NN \seq_clear:c  \tl_clear:c
\cs_new_eq:NN \seq_gclear:N \tl_gclear:N
\cs_new_eq:NN \seq_gclear:c \tl_gclear:c
%    \end{macrocode}
% \end{macro}
% \end{macro}
%
% \begin{macro}{\seq_clear_new:N, \seq_clear_new:c}
% \UnitTested
% \begin{macro}{\seq_gclear_new:N, \seq_gclear_new:c}
% \UnitTested
%   Once again a copy from the token list functions.
%    \begin{macrocode}
\cs_new_eq:NN \seq_clear_new:N  \tl_clear_new:N
\cs_new_eq:NN \seq_clear_new:c  \tl_clear_new:c
\cs_new_eq:NN \seq_gclear_new:N \tl_gclear_new:N
\cs_new_eq:NN \seq_gclear_new:c \tl_gclear_new:c
%    \end{macrocode}
% \end{macro}
% \end{macro}
%
% \begin{macro}{\seq_set_eq:NN, \seq_set_eq:cN, \seq_set_eq:Nc, \seq_set_eq:cc}
% \UnitTested
% \begin{macro}
%   {\seq_gset_eq:NN, \seq_gset_eq:cN, \seq_gset_eq:Nc, \seq_gset_eq:cc}
% \UnitTested
%   Once again, these are simple copies from the token list functions.
%    \begin{macrocode}
\cs_new_eq:NN \seq_set_eq:NN  \tl_set_eq:NN
\cs_new_eq:NN \seq_set_eq:Nc  \tl_set_eq:Nc
\cs_new_eq:NN \seq_set_eq:cN  \tl_set_eq:cN
\cs_new_eq:NN \seq_set_eq:cc  \tl_set_eq:cc
\cs_new_eq:NN \seq_gset_eq:NN \tl_gset_eq:NN
\cs_new_eq:NN \seq_gset_eq:Nc \tl_gset_eq:Nc
\cs_new_eq:NN \seq_gset_eq:cN \tl_gset_eq:cN
\cs_new_eq:NN \seq_gset_eq:cc \tl_gset_eq:cc
%    \end{macrocode}
% \end{macro}
% \end{macro}
%
% \begin{macro}{\seq_concat:NNN, \seq_concat:ccc}
% \UnitTested
% \begin{macro}{\seq_gconcat:NNN, \seq_gconcat:ccc}
% \UnitTested
%   Concatenating sequences is easy.
%    \begin{macrocode}
\cs_new_protected_nopar:Npn \seq_concat:NNN #1#2#3
  { \tl_set:Nx #1 { \exp_not:o {#2} \exp_not:o {#3} } }
\cs_new_protected_nopar:Npn \seq_gconcat:NNN #1#2#3
  { \tl_gset:Nx #1 { \exp_not:o {#2} \exp_not:o {#3} } }
\cs_generate_variant:Nn \seq_concat:NNN  { ccc }
\cs_generate_variant:Nn \seq_gconcat:NNN { ccc }
%    \end{macrocode}
% \end{macro}
% \end{macro}
%
% \subsection{Appending data to either end}
%
% \begin{macro}{
%   \seq_put_left:Nn, \seq_put_left:NV, \seq_put_left:Nv,
%   \seq_put_left:No, \seq_put_left:Nx,
%   \seq_put_left:cn, \seq_put_left:cV, \seq_put_left:cv,
%   \seq_put_left:co, \seq_put_left:cx
% }
% \UnitTested
% \begin{macro}{
%   \seq_put_right:Nn, \seq_put_right:NV, \seq_put_right:Nv,
%   \seq_put_right:No, \seq_put_right:Nx,
%   \seq_put_right:cn, \seq_put_right:cV, \seq_put_right:cv,
%   \seq_put_right:co, \seq_put_right:cx
% }
% \UnitTested
%   The code here is just a wrapper for adding to token lists.
%    \begin{macrocode}
\cs_new_protected:Npn \seq_put_left:Nn #1#2
  { \tl_put_left:Nn #1 { \seq_item:n {#2} } }
\cs_new_protected:Npn \seq_put_right:Nn #1#2
  { \tl_put_right:Nn #1 { \seq_item:n {#2} } }
\cs_generate_variant:Nn \seq_put_left:Nn  {     NV , Nv , No , Nx }
\cs_generate_variant:Nn \seq_put_left:Nn  { c , cV , cv , co , cx }
\cs_generate_variant:Nn \seq_put_right:Nn {     NV , Nv , No , Nx }
\cs_generate_variant:Nn \seq_put_right:Nn { c , cV , cv , co , cx }
%    \end{macrocode}
% \end{macro}
% \end{macro}
%
% \begin{macro}{
%   \seq_gput_left:Nn, \seq_gput_left:NV, \seq_gput_left:Nv,
%   \seq_gput_left:No, \seq_gput_left:Nx,
%   \seq_gput_left:cn, \seq_gput_left:cV, \seq_gput_left:cv,
%   \seq_gput_left:co, \seq_gput_left:cx
% }
% \begin{macro}{
%   \seq_gput_right:Nn, \seq_gput_right:NV, \seq_gput_right:Nv,
%   \seq_gput_right:No, \seq_gput_right:Nx,
%   \seq_gput_right:cn, \seq_gput_right:cV,\seq_gput_right:cv,
%   \seq_gput_right:co, \seq_gput_right:cx
% }
%   The same for global addition.
%    \begin{macrocode}
\cs_new_protected:Npn \seq_gput_left:Nn #1#2
  { \tl_gput_left:Nn #1 { \seq_item:n {#2} } }
\cs_new_protected:Npn \seq_gput_right:Nn #1#2
  { \tl_gput_right:Nn #1 { \seq_item:n {#2} } }
\cs_generate_variant:Nn \seq_gput_left:Nn  {     NV , Nv , No , Nx }
\cs_generate_variant:Nn \seq_gput_left:Nn  { c , cV , cv , co , cx }
\cs_generate_variant:Nn \seq_gput_right:Nn {     NV , Nv , No , Nx }
\cs_generate_variant:Nn \seq_gput_right:Nn { c , cV , cv , co , cx }
%    \end{macrocode}
% \end{macro}
% \end{macro}
%
% \subsection{Modifying sequences}
%
% \begin{variable}{\l_seq_remove_seq}
%   An internal sequence for the removal routines.
%    \begin{macrocode}
\seq_new:N \l_seq_remove_seq
%    \end{macrocode}
% \end{variable}
%
% \begin{macro}{\seq_remove_duplicates:N, \seq_remove_duplicates:c}
% \UnitTested
% \begin{macro}{\seq_gremove_duplicates:N, \seq_gremove_duplicates:c}
% \UnitTested
% \begin{macro}[aux]{\seq_remove_duplicates_aux:NN}
%   Removing duplicates means making a new list then copying it.
%    \begin{macrocode}
\cs_new_protected:Npn \seq_remove_duplicates:N
  { \seq_remove_duplicates_aux:NN \seq_set_eq:NN }
\cs_new_protected:Npn \seq_gremove_duplicates:N
  { \seq_remove_duplicates_aux:NN \seq_gset_eq:NN }
\cs_new_protected:Npn \seq_remove_duplicates_aux:NN #1#2
  {
    \seq_clear:N \l_seq_remove_seq
    \seq_map_inline:Nn #2
      {
        \seq_if_in:NnF \l_seq_remove_seq {##1}
          { \seq_put_right:Nn \l_seq_remove_seq {##1} }
      }
    #1 #2 \l_seq_remove_seq
  }
\cs_generate_variant:Nn \seq_remove_duplicates:N  { c }
\cs_generate_variant:Nn \seq_gremove_duplicates:N { c }
%    \end{macrocode}
% \end{macro}
% \end{macro}
% \end{macro}
%
% \begin{macro}{\seq_remove_all:Nn, \seq_remove_all:cn}
% \UnitTested
% \begin{macro}{\seq_gremove_all:Nn, \seq_gremove_all:cn}
% \UnitTested
% \begin{macro}[aux]{\seq_remove_all_aux:NNn}
%   The idea of the code here is to avoid a relatively expensive addition of
%   items one at a time to an intermediate sequence.
%   The approach taken is therefore similar to
%   that in \cs{seq_pop_right_aux_ii:NNN}, using a \enquote{flexible}
%   \texttt{x}-type expansion to do most of the work. As \cs{tl_if_eq:nnT}
%   is not expandable, a two-part strategy is needed. First, the
%   \texttt{x}-type expansion uses \cs{str_if_eq:nnT} to find potential
%   matches. If one is found, the expansion is halted and the necessary
%   set up takes place to use the \cs{tl_if_eq:NNT} test. The \texttt{x}-type
%   is started again, including all of the items copied already. This will
%   happen repeatedly until the entire sequence has been scanned. The code
%   is set up to avoid needing and intermediate scratch list: the lead-off
%   \texttt{x}-type expansion (|#1 #2 {#2}|) will ensure that nothing is
%   lost.
%    \begin{macrocode}
\cs_new_protected:Npn \seq_remove_all:Nn
  { \seq_remove_all_aux:NNn \tl_set:Nx }
\cs_new_protected:Npn \seq_gremove_all:Nn
  { \seq_remove_all_aux:NNn \tl_gset:Nx }
\cs_new_protected:Npn \seq_remove_all_aux:NNn #1#2#3
  {
    \seq_push_item_def:n
      {
        \str_if_eq:nnT {##1} {#3}
          {
            \if_false: { \fi: }
            \tl_set:Nn \l_seq_tmpb_tl {##1}
            #1 #2
               { \if_false: } \fi:
                 \exp_not:o {#2}
                 \tl_if_eq:NNT \l_seq_tmpa_tl \l_seq_tmpb_tl
                   { \use_none:nn }
          }
        \exp_not:n { \seq_item:n {##1} }
      }
    \tl_set:Nn \l_seq_tmpa_tl {#3}
    #1 #2 {#2}
    \seq_pop_item_def:
  }
\cs_generate_variant:Nn \seq_remove_all:Nn  { c }
\cs_generate_variant:Nn \seq_gremove_all:Nn { c }
%    \end{macrocode}
% \end{macro}
% \end{macro}
% \end{macro}
%
% \subsection{Sequence conditionals}
%
% \begin{macro}[pTF]{\seq_if_empty:N, \seq_if_empty:c}
% \UnitTested
%   Simple copies from the token list variable material.
%    \begin{macrocode}
\prg_new_eq_conditional:NNn \seq_if_empty:N \tl_if_empty:N
  { p , T , F , TF }
\prg_new_eq_conditional:NNn \seq_if_empty:c \tl_if_empty:c
  { p , T , F , TF }
%    \end{macrocode}
% \end{macro}
%
% \begin{macro}[TF]{
%   \seq_if_in:Nn, \seq_if_in:NV, \seq_if_in:Nv, \seq_if_in:No, \seq_if_in:Nx,
%   \seq_if_in:cn, \seq_if_in:cV, \seq_if_in:cv, \seq_if_in:co, \seq_if_in:cx
% }
% \UnitTested
% \begin{macro}[aux]{\seq_if_in_aux:}
%   The approach here is to define \cs{seq_item:n} to compare its
%   argument with the test sequence. If the two items are equal, the
%   mapping is terminated and \cs{prg_return_true:} is inserted. On the
%   other hand, if there is no match then the loop will break returning
%   \cs{prg_return_false:}. In either case, \cs{seq_break_point:n}
%   ensures that the group ends before the logical value is returned.
%   Everything is inside a group so that \cs{seq_item:n} is preserved
%   in nested situations.
%    \begin{macrocode}
\prg_new_protected_conditional:Npnn \seq_if_in:Nn #1#2
  { T , F , TF }
  {
    \group_begin:
      \tl_set:Nn \l_seq_tmpa_tl {#2}
      \cs_set_protected:Npn \seq_item:n ##1
        {
          \tl_set:Nn \l_seq_tmpb_tl {##1}
          \if_meaning:w \l_seq_tmpa_tl \l_seq_tmpb_tl
            \exp_after:wN \seq_if_in_aux:
          \fi:
        }
      #1
      \seq_break:n { \prg_return_false: }
    \seq_break_point:n { \group_end: }
  }
\cs_new_nopar:Npn \seq_if_in_aux: { \seq_break:n { \prg_return_true: } }
\cs_generate_variant:Nn \seq_if_in:NnT  {     NV , Nv , No , Nx }
\cs_generate_variant:Nn \seq_if_in:NnT  { c , cV , cv , co , cx }
\cs_generate_variant:Nn \seq_if_in:NnF  {     NV , Nv , No , Nx }
\cs_generate_variant:Nn \seq_if_in:NnF  { c , cV , cv , co , cx }
\cs_generate_variant:Nn \seq_if_in:NnTF {     NV , Nv , No , Nx }
\cs_generate_variant:Nn \seq_if_in:NnTF { c , cV , cv , co , cx }
%    \end{macrocode}
% \end{macro}
% \end{macro}
%
% \subsection{Recovering data from sequences}
%
% \begin{macro}{\seq_get_left:NN, \seq_get_left:cN}
% \UnitTested
% \begin{macro}[aux]{\seq_get_left_aux:NnwN}
%   Getting an item from the left of a sequence is pretty easy: just
%   trim off the first item after removing the \cs{seq_item:n} at
%   the start.
%    \begin{macrocode}
\cs_new_protected_nopar:Npn \seq_get_left:NN #1#2
  {
    \seq_if_empty_err_break:N #1
    \exp_after:wN \seq_get_left_aux:NnwN #1 \q_stop #2
    \seq_break_point:n { }
  }
\cs_new_protected:Npn \seq_get_left_aux:NnwN \seq_item:n #1#2 \q_stop #3
  { \tl_set:Nn #3 {#1} }
\cs_generate_variant:Nn \seq_get_left:NN { c }
%    \end{macrocode}
% \end{macro}
% \end{macro}
%
% \begin{macro}{\seq_pop_left:NN, \seq_pop_left:cN}
% \UnitTested
% \begin{macro}{\seq_gpop_left:NN, \seq_gpop_left:cN}
% \UnitTested
% \begin{macro}[aux]{\seq_pop_left_aux:NNN}
% \begin{macro}[aux]{\seq_pop_left_aux:NnwNNN}
%   The approach to popping an item is pretty similar to that to get
%   an item, with the only difference being that the sequence itself has
%   to be redefined. This makes it more sensible to use an auxiliary
%   function for the local and global cases.
%    \begin{macrocode}
\cs_new_protected_nopar:Npn \seq_pop_left:NN
  { \seq_pop_left_aux:NNN \tl_set:Nn }
\cs_new_protected_nopar:Npn \seq_gpop_left:NN
  { \seq_pop_left_aux:NNN \tl_gset:Nn }
\cs_new_protected_nopar:Npn \seq_pop_left_aux:NNN #1#2#3
  {
    \seq_if_empty_err_break:N #2
    \exp_after:wN \seq_pop_left_aux:NnwNNN #2 \q_stop #1#2#3
    \seq_break_point:n { }
  }
\cs_new_protected:Npn \seq_pop_left_aux:NnwNNN \seq_item:n #1#2 \q_stop #3#4#5
  {
    #3 #4 {#2}
    \tl_set:Nn #5 {#1}
  }
\cs_generate_variant:Nn \seq_pop_left:NN  { c }
\cs_generate_variant:Nn \seq_gpop_left:NN { c }
%    \end{macrocode}
% \end{macro}
% \end{macro}
% \end{macro}
% \end{macro}
%
% \begin{macro}{\seq_get_right:NN, \seq_get_right:cN}
% \UnitTested
% \begin{macro}[aux]{\seq_get_right_aux:NN}
% \begin{macro}[aux]{\seq_get_right_loop:nn}
%   The idea here is to remove the very first \cs{seq_item:n} from the
%   sequence, leaving a token list starting with the first braced entry.
%   Two arguments at a time are then grabbed: apart from the right-hand end of
%   the sequence, this will be a brace group followed by \cs{seq_item:n}. The
%   set up code means that these all disappear. At the end of the sequence,
%   the assignment is placed in front of the very last entry in the sequence,
%   before a tidying-up step takes place to remove the loop and reset the
%   meaning of \cs{seq_item:n}.
%    \begin{macrocode}
\cs_new_protected_nopar:Npn \seq_get_right:NN #1#2
  {
    \seq_if_empty_err_break:N #1
    \seq_get_right_aux:NN #1#2
    \seq_break_point:n { }
  }
\cs_new_protected_nopar:Npn \seq_get_right_aux:NN #1#2
  {
    \seq_push_item_def:n { }
    \exp_after:wN \exp_after:wN \exp_after:wN \seq_get_right_loop:nn
      \exp_after:wN \use_none:n #1
      { \tl_set:Nn #2 }
      { }
      {
        \seq_pop_item_def:
        \seq_break:
      }
}
\cs_new:Npn \seq_get_right_loop:nn #1#2
  {
    #2 {#1}
    \seq_get_right_loop:nn
  }
\cs_generate_variant:Nn \seq_get_right:NN { c }
%    \end{macrocode}
% \end{macro}
% \end{macro}
% \end{macro}
%
% \begin{macro}{\seq_pop_right:NN, \seq_pop_right:cN}
% \UnitTested
% \begin{macro}{\seq_gpop_right:NN, \seq_gpop_right:cN}
% \UnitTested
% \begin{macro}[aux]{\seq_pop_right_aux:NNN, \seq_pop_right_aux_ii:NNN}
%   The approach to popping from the right is a bit more involved, but does
%   use some of the same ideas as getting from the right. What is needed is a
%   \enquote{flexible length} way to set a token list variable. This is
%   supplied by the |{ \if_false:} \fi:| \ldots
%   |\if_false: { \fi: }| construct. Using an \texttt{x}-type
%   expansion and a \enquote{non-expanding} definition for \cs{seq_item:n},
%   the left-most $n - 1$ entries in a sequence of $n$ items will be stored
%   back in the sequence. That needs a loop of unknown length, hence using the
%   strange \cs{if_false:} way of including brackets. When the last item
%   of the sequence is reached, the closing bracket for the assignment is
%   inserted, and |\tl_set:Nn #3| is inserted in front of the final entry.
%   This therefore does the pop assignment, then a final loop clears up the
%   code.
%    \begin{macrocode}
\cs_new_protected_nopar:Npn \seq_pop_right:NN
  { \seq_pop_right_aux:NNN \tl_set:Nx }
\cs_new_protected_nopar:Npn \seq_gpop_right:NN
  { \seq_pop_right_aux:NNN \tl_gset:Nx }
\cs_new_protected_nopar:Npn \seq_pop_right_aux:NNN #1#2#3
  {
    \seq_if_empty_err_break:N #2
    \seq_pop_right_aux_ii:NNN #1 #2 #3
    \seq_break_point:n { }
  }
\cs_new_protected_nopar:Npn \seq_pop_right_aux_ii:NNN #1#2#3
  {
    \seq_push_item_def:n { \exp_not:n { \seq_item:n {##1} } }
    #1 #2 { \if_false: } \fi:
      \exp_after:wN \exp_after:wN \exp_after:wN \seq_get_right_loop:nn
         \exp_after:wN \use_none:n #2
        {
          \if_false: { \fi: }
          \tl_set:Nn #3
        }
        { }
        {
          \seq_pop_item_def:
          \seq_break:
        }
  }
\cs_generate_variant:Nn \seq_pop_right:NN  { c }
\cs_generate_variant:Nn \seq_gpop_right:NN { c }
%    \end{macrocode}
% \end{macro}
% \end{macro}
% \end{macro}
%
% \subsection{Mapping to sequences}
%
% \begin{macro}[int]{\seq_break:}
% \begin{macro}[int]{\seq_break:n}
%   To break a function, the special token \cs{seq_break_point:n} is
%   used to find the end of the code. Any ending code is then inserted
%   before the return value of \cs{seq_map_break:n} is inserted.
%    \begin{macrocode}
\cs_new:Npn \seq_break:  #1   \seq_break_point:n #2 {#2}
\cs_new:Npn \seq_break:n #1#2 \seq_break_point:n #3 { #3 #1 }
%    \end{macrocode}
% \end{macro}
% \end{macro}
%
% \begin{macro}{\seq_map_break:}
% \UnitTested
% \begin{macro}{\seq_map_break:n}
% \UnitTested
%   Semantically-logical copies of the break functions for use inside
%   mappings.
%    \begin{macrocode}
\cs_new_eq:NN \seq_map_break:  \seq_break:
\cs_new_eq:NN \seq_map_break:n \seq_break:n
%    \end{macrocode}
% \end{macro}
% \end{macro}
%
% \begin{macro}[int]{\seq_break_point:n}
%   Normally, the marker token will not be executed, but if it is then
%   the end code is simply inserted.
%    \begin{macrocode}
\cs_new_eq:NN \seq_break_point:n \use:n
%    \end{macrocode}
% \end{macro}
%
% \begin{macro}[int]{\seq_if_empty_err_break:N}
%   A function to check that sequences really have some content. This
%   is optimised for speed, hence the direct primitive use.
%    \begin{macrocode}
\cs_new_protected_nopar:Npn \seq_if_empty_err_break:N #1
  {
    \if_meaning:w #1 \c_empty_tl
      \msg_kernel_error:nnx { seq } { empty-sequence } { \token_to_str:N #1 }
      \exp_after:wN \seq_break:
    \fi:
  }
%    \end{macrocode}
% \end{macro}
%
% \begin{macro}{\seq_map_function:NN, \seq_map_function:cN}
% \UnitTested
% \begin{macro}[aux]{\seq_map_function_aux:NNn}
%   The idea here is to apply the code of |#2| to each item in the
%   sequence without altering the definition of \cs{seq_item:n}. This
%   is done as by noting that every odd token in the sequence must be
%   \cs{seq_item:n}, which can be gobbled by \cs{use_none:n}. At the end of
%   the loop, |#2| is instead |? \seq_map_break:|, which therefore breaks the
%   loop without needing to do a (relatively-expensive) quark test.
%    \begin{macrocode}
\cs_new:Npn \seq_map_function:NN #1#2
  {
    \exp_after:wN \seq_map_function_aux:NNn \exp_after:wN #2 #1
      { ? \seq_map_break: } { }
    \seq_break_point:n { }
  }
\cs_new:Npn \seq_map_function_aux:NNn #1#2#3
  {
    \use_none:n #2
    #1 {#3}
    \seq_map_function_aux:NNn #1
  }
\cs_generate_variant:Nn \seq_map_function:NN { c }
%    \end{macrocode}
% \end{macro}
% \end{macro}
%
% \begin{variable}{\g_seq_nesting_depth_int}
% A counter to keep track of nested functions: defined in \pkg{l3int}.
% \end{variable}
%
% \begin{macro}[int]{\seq_push_item_def:n, \seq_push_item_def:x}
% \begin{macro}[aux]{\seq_push_item_def_aux:}
% \begin{macro}[int]{\seq_pop_item_def:}
%   The definition of \cs{seq_item:n} needs to be saved and restored at
%   various points within the mapping and manipulation code. That is handled
%   here: as always, this approach uses global assignments.
%    \begin{macrocode}
\cs_new_protected:Npn \seq_push_item_def:n
  {
    \seq_push_item_def_aux:
    \cs_gset:Npn \seq_item:n ##1
  }
\cs_new_protected:Npn \seq_push_item_def:x
  {
    \seq_push_item_def_aux:
    \cs_gset:Npx \seq_item:n ##1
  }
\cs_new_protected:Npn \seq_push_item_def_aux:
  {
    \cs_gset_eq:cN { seq_item_ \int_use:N \g_seq_nesting_depth_int :n }
      \seq_item:n
    \int_gincr:N \g_seq_nesting_depth_int
  }
\cs_new_protected_nopar:Npn \seq_pop_item_def:
  {
    \int_gdecr:N \g_seq_nesting_depth_int
    \cs_gset_eq:Nc \seq_item:n
      { seq_item_ \int_use:N \g_seq_nesting_depth_int :n }
  }
%    \end{macrocode}
% \end{macro}
% \end{macro}
% \end{macro}
%
% \begin{macro}{\seq_map_inline:Nn, \seq_map_inline:cn}
% \UnitTested
%   The idea here is that \cs{seq_item:n} is already \enquote{applied} to
%   each item in a sequence, and so an in-line mapping is just a case of
%   redefining \cs{seq_item:n}.
%    \begin{macrocode}
\cs_new_protected:Npn \seq_map_inline:Nn #1#2
  {
    \seq_push_item_def:n {#2}
    #1
    \seq_break_point:n { \seq_pop_item_def: }
  }
\cs_generate_variant:Nn \seq_map_inline:Nn { c }
%    \end{macrocode}
% \end{macro}
%
% \begin{macro}
%   {
%     \seq_map_variable:NNn,\seq_map_variable:Ncn,
%     \seq_map_variable:cNn,\seq_map_variable:ccn
%   }
% \UnitTested
%   This is just a specialised version of the in-line mapping function,
%   using an \texttt{x}-type expansion for the code set up so that the
%   number of |#| tokens required is as expected.
%    \begin{macrocode}
\cs_new_protected:Npn \seq_map_variable:NNn #1#2#3
  {
    \seq_push_item_def:x
      {
        \tl_set:Nn \exp_not:N #2 {##1}
        \exp_not:n {#3}
      }
    #1
    \seq_break_point:n { \seq_pop_item_def: }
  }
\cs_generate_variant:Nn \seq_map_variable:NNn {     Nc }
\cs_generate_variant:Nn \seq_map_variable:NNn { c , cc }
%    \end{macrocode}
% \end{macro}
%
% \subsection{Sequence stacks}
%
% The same functions as for sequences, but with the correct naming.
%
% \begin{macro}{
%   \seq_push:Nn, \seq_push:NV, \seq_push:Nv, \seq_push:No, \seq_push:Nx,
%   \seq_push:cn, \seq_push:cV, \seq_push:cV, \seq_push:co, \seq_push:cx
% }
% \UnitTested
% \begin{macro}{
%   \seq_gpush:Nn, \seq_gpush:NV, \seq_gpush:Nv, \seq_gpush:No, \seq_gpush:Nx,
%   \seq_gpush:cn, \seq_gpush:cV, \seq_gpush:cv, \seq_gpush:co, \seq_gpush:cx
% }
% \UnitTested
%   Pushing to a sequence is the same as adding on the left.
%    \begin{macrocode}
\cs_new_eq:NN \seq_push:Nn  \seq_put_left:Nn
\cs_new_eq:NN \seq_push:NV  \seq_put_left:NV
\cs_new_eq:NN \seq_push:Nv  \seq_put_left:Nv
\cs_new_eq:NN \seq_push:No  \seq_put_left:No
\cs_new_eq:NN \seq_push:Nx  \seq_put_left:Nx
\cs_new_eq:NN \seq_push:cn  \seq_put_left:cn
\cs_new_eq:NN \seq_push:cV  \seq_put_left:cV
\cs_new_eq:NN \seq_push:cv  \seq_put_left:cv
\cs_new_eq:NN \seq_push:co  \seq_put_left:co
\cs_new_eq:NN \seq_push:cx  \seq_put_left:cx
\cs_new_eq:NN \seq_gpush:Nn \seq_gput_left:Nn
\cs_new_eq:NN \seq_gpush:NV \seq_gput_left:NV
\cs_new_eq:NN \seq_gpush:Nv \seq_gput_left:Nv
\cs_new_eq:NN \seq_gpush:No \seq_gput_left:No
\cs_new_eq:NN \seq_gpush:Nx \seq_gput_left:Nx
\cs_new_eq:NN \seq_gpush:cn \seq_gput_left:cn
\cs_new_eq:NN \seq_gpush:cV \seq_gput_left:cV
\cs_new_eq:NN \seq_gpush:cv \seq_gput_left:cv
\cs_new_eq:NN \seq_gpush:co \seq_gput_left:co
\cs_new_eq:NN \seq_gpush:cx \seq_gput_left:cx
%    \end{macrocode}
% \end{macro}
% \end{macro}
%
% \begin{macro}{\seq_get:NN, \seq_get:cN}
% \UnitTested
% \begin{macro}{\seq_pop:NN, \seq_pop:cN}
% \UnitTested
% \begin{macro}{\seq_gpop:NN, \seq_gpop:cN}
% \UnitTested
%   In most cases, getting items from the stack does not need to specify
%   that this is from the left. So alias are provided.
%    \begin{macrocode}
\cs_new_eq:NN \seq_get:NN \seq_get_left:NN
\cs_new_eq:NN \seq_get:cN \seq_get_left:cN
\cs_new_eq:NN \seq_pop:NN \seq_pop_left:NN
\cs_new_eq:NN \seq_pop:cN \seq_pop_left:cN
\cs_new_eq:NN \seq_gpop:NN \seq_gpop_left:NN
\cs_new_eq:NN \seq_gpop:cN \seq_gpop_left:cN
%    \end{macrocode}
% \end{macro}
% \end{macro}
% \end{macro}
%
% \subsection{Viewing sequences}
%
% \begin{variable}{\l_seq_show_tl}
%   Used to store the material for display.
%    \begin{macrocode}
\tl_new:N \l_seq_show_tl
%    \end{macrocode}
% \end{variable}
%
% \begin{macro}{\seq_show:N, \seq_show:c}
% \UnitTested
% \begin{macro}[aux]{\seq_show_aux:n}
% \begin{macro}[aux]{\seq_show_aux:w}
%   The aim of the mapping here is to create a token list containing the
%   formatted sequence. The very first item needs the new line and \verb*|> |
%   removing, which is achieved using a \texttt{w}-type auxiliary. To avoid
%   a low-level \TeX{} error if there is an empty sequence, a simple test is
%   used to keep the output \enquote{clean}.
%    \begin{macrocode}
\cs_new_protected_nopar:Npn \seq_show:N #1
  {
    \seq_if_empty:NTF #1
      {
        \iow_term:x { Sequence~\token_to_str:N #1 \c_space_tl is~empty }
        \tl_show:n { }
      }
      {
        \iow_term:x
          {
            Sequence~\token_to_str:N #1 \c_space_tl
            contains~the~items~(without~outer~braces):
          }
        \tl_set:Nx \l_seq_show_tl
          { \seq_map_function:NN #1 \seq_show_aux:n }
          \etex_showtokens:D \exp_after:wN \exp_after:wN \exp_after:wN
            { \exp_after:wN \seq_show_aux:w \l_seq_show_tl }
      }
  }
\cs_new:Npn \seq_show_aux:n #1
  {
    \iow_newline: > \c_space_tl \c_space_tl
    \iow_char:N \{ \exp_not:n {#1} \iow_char:N \}
  }
\cs_new:Npn \seq_show_aux:w #1 > ~ { }
\cs_generate_variant:Nn \seq_show:N { c }
%    \end{macrocode}
% \end{macro}
% \end{macro}
% \end{macro}
%
% \subsection{Experimental functions}
%
% \begin{macro}[aux]{\seq_if_empty_break_return_false:N}
% The name says it all: of the sequence is empty, returns logical
% \texttt{false}.
%    \begin{macrocode}
\cs_new_nopar:Npn \seq_if_empty_break_return_false:N #1
  {
    \if_meaning:w #1 \c_empty_tl
      \prg_return_false:
      \exp_after:wN \seq_break:
    \fi:
  }
%    \end{macrocode}
% \end{macro}
%
% \begin{macro}[TF]{\seq_get_left:NN, \seq_get_left:cN}
% \begin{macro}[TF]{\seq_get_right:NN, \seq_get_right:cN}
% Getting from the left or right with a check on the results.
%    \begin{macrocode}
\prg_new_protected_conditional:Npnn \seq_get_left:NN #1 #2 { T , F , TF }
  {
    \seq_if_empty_break_return_false:N #1
    \exp_after:wN \seq_get_left_aux:Nw #1 \q_stop #2
    \prg_return_true:
    \seq_break:
    \seq_break_point:n { }
  }
\prg_new_protected_conditional:Npnn \seq_get_right:NN #1#2 { T , F , TF }
  {
    \seq_if_empty_break_return_false:N #1
    \seq_get_right_aux:NN #1#2
    \prg_return_true: \seq_break:
    \seq_break_point:n { }
  }
\cs_generate_variant:Nn \seq_get_left:NNT   { c }
\cs_generate_variant:Nn \seq_get_left:NNF   { c }
\cs_generate_variant:Nn \seq_get_left:NNTF  { c }
\cs_generate_variant:Nn \seq_get_right:NNT  { c }
\cs_generate_variant:Nn \seq_get_right:NNF  { c }
\cs_generate_variant:Nn \seq_get_right:NNTF { c }
%    \end{macrocode}
% \end{macro}
% \end{macro}
%
% \begin{macro}[TF]{\seq_pop_left:NN, \seq_pop_left:cN}
% \begin{macro}[TF]{\seq_gpop_left:NN, \seq_gpop_left:cN}
% \begin{macro}[TF]{\seq_pop_right:NN, \seq_pop_right:cN}
% \begin{macro}[TF]{\seq_gpop_right:NN, \seq_gpop_right:cN}
% More or less the same for popping.
%    \begin{macrocode}
\prg_new_protected_conditional:Npnn \seq_pop_left:NN #1#2 { T , F , TF }
  {
    \seq_if_empty_break_return_false:N #1
    \exp_after:wN \seq_pop_left_aux:NnwNNN #1 \q_stop \tl_set:Nn #1#2
    \prg_return_true: \seq_break:
    \seq_break_point:n { }
  }
\prg_new_protected_conditional:Npnn \seq_gpop_left:NN #1#2 { T , F , TF }
  {
    \seq_if_empty_break_return_false:N #1
    \exp_after:wN \seq_pop_left_aux:NnwNNN #1 \q_stop \tl_gset:Nn #1#2
    \prg_return_true: \seq_break:
    \seq_break_point:n { }
  }
\prg_new_protected_conditional:Npnn \seq_pop_right:NN #1#2 { T , F , TF }
  {
    \seq_if_empty_break_return_false:N #1
    \seq_pop_right_aux_ii:NNN \tl_set:Nx #1 #2
    \prg_return_true: \seq_break:
    \seq_break_point:n { }
  }
\prg_new_protected_conditional:Npnn \seq_gpop_right:NN #1#2 { T , F , TF }
  {
    \seq_if_empty_break_return_false:N #1
    \seq_pop_right_aux_ii:NNN \tl_gset:Nx #1 #2
    \prg_return_true: \seq_break:
    \seq_break_point:n { }
  }
\cs_generate_variant:Nn \seq_pop_left:NNT    { c }
\cs_generate_variant:Nn \seq_pop_left:NNF    { c }
\cs_generate_variant:Nn \seq_pop_left:NNTF   { c }
\cs_generate_variant:Nn \seq_gpop_left:NNT   { c }
\cs_generate_variant:Nn \seq_gpop_left:NNF   { c }
\cs_generate_variant:Nn \seq_gpop_left:NNTF  { c }
\cs_generate_variant:Nn \seq_pop_right:NNT   { c }
\cs_generate_variant:Nn \seq_pop_right:NNF   { c }
\cs_generate_variant:Nn \seq_pop_right:NNTF  { c }
\cs_generate_variant:Nn \seq_gpop_right:NNT  { c }
\cs_generate_variant:Nn \seq_gpop_right:NNF  { c }
\cs_generate_variant:Nn \seq_gpop_right:NNTF { c }
%    \end{macrocode}
% \end{macro}
% \end{macro}
% \end{macro}
% \end{macro}
%
% \begin{macro}{\seq_length:N, \seq_length:c}
% \begin{macro}[aux]{\seq_length_aux:n}
%   Counting the items in a sequence is done using the same approach as for
%   other length functions: turn each entry into a \texttt{+1} then use
%   integer evaluation to actually do the mathematics.
%    \begin{macrocode}
\cs_new:Npn \seq_length:N #1
  {
    \int_eval:n
      {
        0
        \seq_map_function:NN #1 \seq_length_aux:n
      }
  }
\cs_new:Npn \seq_length_aux:n #1 { +1 }
\cs_generate_variant:Nn \seq_length:N { c }
%    \end{macrocode}
% \end{macro}
% \end{macro}
%
% \begin{macro}{\seq_item:Nn, \seq_item:cn}
% \begin{macro}[aux]{\seq_item_aux:nnn}
%   The idea here is to find the offset of the item from the left, then use
%   a loop to grab the correct item. If the resulting offset is too large,
%   then the stop code |{ ? \seq_break } { }| will be used by the auxiliary,
%   terminating the loop and returning nothing at all.
%    \begin{macrocode}
\cs_new_nopar:Npn \seq_item:Nn #1#2
  {
    \exp_last_unbraced:Nfo \seq_item_aux:nnn
      {
        \int_eval:n
          {
            \int_compare:nNnT {#2} < \c_zero
              { \seq_length:N #1 + }
            #2
          }
      }
    #1
    { ? \seq_break: }
    { }
    \seq_break_point:n { }
  }
\cs_new_nopar:Npn \seq_item_aux:nnn #1#2#3
  {
    \use_none:n #2
    \int_compare:nNnTF {#1} = \c_zero
      { \seq_break:n {#3} }
      { \exp_args:Nf \seq_item_aux:nnn { #1 - 1 } }
  }
\cs_generate_variant:Nn \seq_item:Nn { c }
%    \end{macrocode}
% \end{macro}
% \end{macro}
%
% \begin{macro}{\seq_use:N, \seq_use:c}
%   A simple short cut for a mapping.
%    \begin{macrocode}
\cs_new_nopar:Npn \seq_use:N #1 { \seq_map_function:NN #1 \use:n }
\cs_generate_variant:Nn \seq_use:N { c }
%    \end{macrocode}
% \end{macro}
%
% \begin{macro}
%   {
%     \seq_mapthread_function:NNN, \seq_mapthread_function:NcN,
%     \seq_mapthread_function:cNN, \seq_mapthread_function:ccN
%   }
% \begin{macro}[aux]{\seq_mapthread_function_aux:NN}
% \begin{macro}[aux]{\seq_mapthread_function_aux:Nnnwnn}
%   The idea here is to first expand both of the sequences, adding the usual
%   |{ ? \seq_break: } { }| to the end of each on. This is most conveniently
%   done in two steps using an auxiliary function. The mapping then throws
%   away the first token of |#2| and |#5|, which for items in the sequences
%   will both be \cs{seq_item:n}. The function to be mapped will then be
%   applied to the two entries. When the code hits the end of one of the
%   sequences, the break material will stop the entire loop and tidy up. This
%   avoids needing to find the length of the two sequences, or worrying about
%   which is longer.
%    \begin{macrocode}
\cs_new_nopar:Npn \seq_mapthread_function:NNN #1#2#3
  {
    \exp_after:wN \seq_mapthread_function_aux:NN
      \exp_after:wN #3
      \exp_after:wN #1
      #2
      { ? \seq_break: } { }
    \seq_break_point:n { }
  }
\cs_new_nopar:Npn \seq_mapthread_function_aux:NN #1#2
  {
    \exp_after:wN \seq_mapthread_function_aux:Nnnwnn
      \exp_after:wN #1
      #2
      { ? \seq_break: } { }
      \q_stop
  }
\cs_new:Npn \seq_mapthread_function_aux:Nnnwnn #1#2#3#4 \q_stop #5#6
  {
    \use_none:n #2
    \use_none:n #5
    #1 {#3} {#6}
    \seq_mapthread_function_aux:Nnnwnn #1 #4 \q_stop
  }
\cs_generate_variant:Nn \seq_mapthread_function:NNN {     Nc }
\cs_generate_variant:Nn \seq_mapthread_function:NNN { c , cc }
%    \end{macrocode}
% \end{macro}
% \end{macro}
% \end{macro}
%
% \begin{macro}
%   {
%     \seq_set_from_clist:NN, \seq_set_from_clist:cN,
%     \seq_set_from_clist:Nc, \seq_set_from_clist:cc,
%     \seq_set_from_clist:Nn, \seq_set_from_clist:cn
%   }
% \begin{macro}
%   {
%     \seq_gset_from_clist:NN, \seq_gset_from_clist:cN,
%     \seq_gset_from_clist:Nc, \seq_gset_from_clist:cc,
%     \seq_gset_from_clist:Nn, \seq_gset_from_clist:cn
%   }
% \begin{macro}[aux]{\seq_wrap_item:n}
%   Setting a sequence from a comma-separated list is done using a simple
%   mapping.
%    \begin{macrocode}
\cs_new_protected:Npn \seq_set_from_clist:NN #1#2
  {
    \tl_set:Nx #1
      { \clist_map_function:NN #2 \seq_wrap_item:n }
  }
\cs_new_protected:Npn \seq_set_from_clist:Nn #1#2
  {
    \tl_set:Nx #1
      { \clist_map_function:nN {#2} \seq_wrap_item:n }
  }
\cs_new_protected:Npn \seq_gset_from_clist:NN #1#2
  {
    \tl_gset:Nx #1
      { \clist_map_function:NN #2 \seq_wrap_item:n }
  }
\cs_new_protected:Npn \seq_gset_from_clist:Nn #1#2
  {
    \tl_gset:Nx #1
      { \clist_map_function:nN {#2} \seq_wrap_item:n }
  }
\cs_new:Npn \seq_wrap_item:n #1 { \exp_not:n { \seq_item:n {#1} } }
\cs_generate_variant:Nn \seq_set_from_clist:NN  {     Nc }
\cs_generate_variant:Nn \seq_set_from_clist:NN  { c , cc }
\cs_generate_variant:Nn \seq_set_from_clist:Nn  { c      }
\cs_generate_variant:Nn \seq_gset_from_clist:NN {     Nc }
\cs_generate_variant:Nn \seq_gset_from_clist:NN { c , cc }
\cs_generate_variant:Nn \seq_gset_from_clist:Nn { c      }
%    \end{macrocode}
% \end{macro}
% \end{macro}
% \end{macro}
%
% \subsection{Depreciated interfaces}
%
%   A few functions which are no longer documented: these were moved
%   here on or before 2011-04-20, and will be removed entirely by
%   2011-07-20.
%
% \begin{macro}{\seq_top:NN, \seq_top:cN}
%   These are old stack functions.
%    \begin{macrocode}
\cs_new_eq:NN \seq_top:NN \seq_get_left:NN
\cs_new_eq:NN \seq_top:cN \seq_get_left:cN
%    \end{macrocode}
% \end{macro}
%
% \begin{macro}{\seq_display:N, \seq_display:c}
%   An older name for \cs{seq_show:N}.
%    \begin{macrocode}
\cs_new_eq:NN \seq_display:N \seq_show:N
\cs_new_eq:NN \seq_display:c \seq_show:c
%    \end{macrocode}
% \end{macro}
%
%    \begin{macrocode}
%</initex|package>
%    \end{macrocode}
%
% \end{implementation}
%
% \PrintIndex