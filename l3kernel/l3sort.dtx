% \iffalse
%
%% File l3sort.dtx (C) Copyright 2012,2014-2017 The LaTeX3 Project
%
% It may be distributed and/or modified under the conditions of the
% LaTeX Project Public License (LPPL), either version 1.3c of this
% license or (at your option) any later version.  The latest version
% of this license is in the file
%
%    http://www.latex-project.org/lppl.txt
%
% This file is part of the "l3kernel bundle" (The Work in LPPL)
% and all files in that bundle must be distributed together.
%
% -----------------------------------------------------------------------
%
% The development version of the bundle can be found at
%
%    https://github.com/latex3/latex3
%
% for those people who are interested.
%
%<*driver>
\documentclass[full]{l3doc}
%</driver>
%<*driver|package>
\GetIdInfo$Id$
  {L3 Sorting functions}
%</driver|package>
%<*driver>
\begin{document}
  \DocInput{\jobname.dtx}
\end{document}
%</driver>
% \fi
%
% \title{^^A
%   The \pkg{l3sort} package\\ Sorting functions^^A
%   \thanks{This file describes v\ExplFileVersion,
%      last revised \ExplFileDate.}^^A
% }
%
% \author{^^A
%  The \LaTeX3 Project\thanks
%    {^^A
%      E-mail:
%        \href{mailto:latex-team@latex-project.org}
%          {latex-team@latex-project.org}^^A
%    }^^A
% }
%
% \date{Released \ExplFileDate}
%
% \maketitle
%
% \begin{documentation}
%
% \section{Controlling sorting}
%
% \label{sec:l3sort:mech}
%
% \LaTeX3 comes with a facility to sort list variables (sequences,
% token lists, or comma-lists) according to some user-defined
% comparison. For instance,
% \begin{verbatim}
%   \clist_set:Nn \l_foo_clist { 3 , 01 , -2 , 5 , +1 }
%   \clist_sort:Nn \l_foo_clist
%     {
%       \int_compare:nNnTF { #1 } > { #2 }
%         { \sort_return_swapped: }
%         { \sort_return_same: }
%     }
% \end{verbatim}
% will result in \cs{l_foo_clist} holding the values
% |{ -2 , 01 , +1 , 3 , 5 }| sorted in non-decreasing order.
%
% The code defining the comparison should call
% \cs{sort_return_swapped:} if the two items given as |#1|
% and |#2| are not in the correct order, and otherwise it
% should call \cs{sort_return_same:} to indicate that
% the order of this pair of items should not be changed.
%
% For instance, a \meta{comparison code} consisting only
% of \cs{sort_return_same:} with no test will yield a trivial
% sort: the final order is identical to the original order.
% Conversely, using a \meta{comparison code} consisting only
% of \cs{sort_return_swapped:} will reverse the list (in a fairly
% inefficient way).
%
% \begin{texnote}
%   The current implementation is limited to sorting approximately
%   $20000$ items ($40000$ in Lua\TeX{}), depending on what other
%   packages are loaded.
%
%   Internally, the code from \pkg{l3sort} stores items in \tn{toks}
%   registers allocated locally.  Thus, the \meta{comparison code}
%   should not call \tn{newtoks} or other commands that allocate new
%   \tn{toks} registers.  On the other hand, altering the value of a
%   previously allocated \tn{toks} register is not a problem.
% \end{texnote}
%
% \end{documentation}
%
% \begin{implementation}
%
% \section{\pkg{l3sort} implementation}
%
%    \begin{macrocode}
%<*initex|package>
%    \end{macrocode}
%
%    \begin{macrocode}
%<@@=sort>
%    \end{macrocode}
%
% \subsection{Variables}
%
% \begin{variable}
%   {
%     \l_@@_length_int, \l_@@_min_int, \l_@@_top_int, \l_@@_max_int,
%     \l_@@_true_max_int
%   }
%   The sequence has \cs{l_@@_length_int} items and is stored from
%   \cs{l_@@_min_int} to $\cs{l_@@_top_int}-1$.  While reading the
%   sequence in memory, we check that \cs{l_@@_top_int} remains at most
%   \cs{l_@@_max_int}, precomputed by \cs{@@_compute_range:}.  That
%   bound is such that the merge sort will only use \tn{toks} registers
%   less than \cs{l_@@_true_max_int}, namely those that have not been
%   allocated for use in other code: the user's comparison code could
%   alter these.
%    \begin{macrocode}
\int_new:N \l_@@_length_int
\int_new:N \l_@@_min_int
\int_new:N \l_@@_top_int
\int_new:N \l_@@_max_int
\int_new:N \l_@@_true_max_int
%    \end{macrocode}
% \end{variable}
%
% \begin{variable}{\l_@@_block_int}
%   Merge sort is done in several passes. In each pass, blocks of size
%   \cs{l_@@_block_int} are merged in pairs. The block size starts
%   at $1$, and, for a length in the range $[2^k+1,2^{k+1}]$, reaches
%   $2^{k}$ in the last pass.
%    \begin{macrocode}
\int_new:N \l_@@_block_int
%    \end{macrocode}
% \end{variable}
%
% \begin{variable}{\l_@@_begin_int}
% \begin{variable}{\l_@@_end_int}
%   When merging two blocks, \cs{l_@@_begin_int} marks the lowest
%   index in the two blocks, and \cs{l_@@_end_int} marks the
%   highest index, plus $1$.
%    \begin{macrocode}
\int_new:N \l_@@_begin_int
\int_new:N \l_@@_end_int
%    \end{macrocode}
% \end{variable}
% \end{variable}
%
% \begin{variable}{\l_@@_A_int}
% \begin{variable}{\l_@@_B_int}
% \begin{variable}{\l_@@_C_int}
%   When merging two blocks (whose end-points are \texttt{beg}
%   and \texttt{end}), $A$ starts from the high end of the low
%   block, and decreases until reaching \texttt{beg}. The index
%   $B$ starts from the top of the range and marks the register
%   in which a sorted item should be put. Finally, $C$ points
%   to the copy of the high block in the interval of registers
%   starting at \cs{l_@@_length_int}, upwards. $C$ starts
%   from the upper limit of that range.
%    \begin{macrocode}
\int_new:N \l_@@_A_int
\int_new:N \l_@@_B_int
\int_new:N \l_@@_C_int
%    \end{macrocode}
% \end{variable}
% \end{variable}
% \end{variable}
%
% \subsection{Finding available \tn{toks} registers}
%
% \begin{macro}[int]{\@@_shrink_range:}
% \begin{macro}[aux]{\@@_shrink_range_loop:}
%   After \cs{@@_compute_range:} (defined below) determines that
%   \tn{toks} registers between \cs{l_@@_min_int} (included) and
%   \cs{l_@@_true_max_int} (excluded) have not yet been assigned,
%   \cs{@@_shrink_range:} computes \cs{l_@@_max_int} to reflect the need
%   for a buffer when merging blocks in the merge sort.  Given
%   $2^{n}\leq A\leq 2^{n}+2^{n-1}$ registers we can sort $\lfloor
%   A/2\rfloor+2^{n-2}$ items while if we have $2^{n}+2^{n-1}\leq A\leq
%   2^{n+1}$ registers we can sort $A-2^{n-1}$ items.  We first find out
%   a power $2^{n}$ such that $2^{n}\leq A\leq 2^{n+1}$ by repeatedly
%   halving \cs{l_@@_block_int}, starting at $2^{15}$ or $2^{14}$ namely
%   half the total number of registers, then we use the formulas and set
%   \cs{l_@@_max_int}.
%    \begin{macrocode}
\cs_new_protected:Npn \@@_shrink_range:
  {
    \int_set:Nn \l_@@_A_int
      { \l_@@_true_max_int - \l_@@_min_int + 1 }
    \int_set:Nn \l_@@_block_int { \c_max_register_int / 2 }
    \@@_shrink_range_loop:
    \int_set:Nn \l_@@_max_int
      {
        \int_compare:nNnTF
          { \l_@@_block_int * 3 / 2 } > \l_@@_A_int
          {
            \l_@@_min_int
            + ( \l_@@_A_int - 1 ) / 2
            + \l_@@_block_int / 4
            - 1
          }
          { \l_@@_true_max_int - \l_@@_block_int / 2 }
      }
  }
\cs_new_protected:Npn \@@_shrink_range_loop:
  {
    \if_int_compare:w \l_@@_A_int < \l_@@_block_int
      \tex_divide:D \l_@@_block_int 2 \exp_stop_f:
      \exp_after:wN \@@_shrink_range_loop:
    \fi:
  }
%    \end{macrocode}
% \end{macro}
% \end{macro}
%
% \begin{macro}[int]{\@@_compute_range:, \@@_redefine_compute_range:}
% \begin{variable}{\c_@@_max_length_int}
%   First find out what \tn{toks} have not yet been assigned.  There are
%   many cases.  In \LaTeXe{} with no package, available \tn{toks} range
%   from $\tn{count}15+1$ to \cs{c_max_register_int} included (this was
%   not altered despite the 2015 changes).  When \tn{loctoks} is
%   defined, namely in plain (e)\TeX{}, or when the package \pkg{etex}
%   is loaded in \LaTeXe{}, redefine \cs{@@_compute_range:} to use the
%   range $\tn{count}265$ to $\tn{count}275-1$.  The \pkg{elocalloc}
%   package also defines \tn{loctoks} but uses yet another number for
%   the upper bound, namely \cs{e@alloc@top} (minus one).  We must check
%   for \tn{loctoks} every time a sorting function is called, as
%   \pkg{etex} or \pkg{elocalloc} could be loaded.
%
%   In Con\TeX{}t MkIV the range is from
%   $|\c_syst_last_allocated_toks|+1$ to \cs{c_max_register_int}, and in
%   MkII it is from $|\lastallocatedtoks|+1$ to \cs{c_max_register_int}.
%   In all these cases, call \cs{@@_shrink_range:}.  The \LaTeX3 format
%   mode is easiest: no \tn{toks} are ever allocated so available
%   \tn{toks} range from $0$ to \cs{c_max_register_int} and we
%   precompute the result of \cs{@@_shrink_range:}.
%    \begin{macrocode}
%<*package>
\cs_new_protected:Npn \@@_compute_range:
  {
    \int_set:Nn \l_@@_min_int { \tex_count:D 15 + 1 }
    \int_set:Nn \l_@@_true_max_int { \c_max_register_int + 1 }
    \@@_shrink_range:
    \if_meaning:w \loctoks \tex_undefined:D \else:
      \if_meaning:w \loctoks \scan_stop: \else:
        \@@_redefine_compute_range:
        \@@_compute_range:
      \fi:
    \fi:
  }
\cs_new_protected:Npn \@@_redefine_compute_range:
  {
    \cs_if_exist:cTF { ver@elocalloc.sty }
      {
        \cs_gset_protected:Npn \@@_compute_range:
          {
            \int_set:Nn \l_@@_min_int { \tex_count:D 265 }
            \int_set_eq:NN \l_@@_true_max_int \e@alloc@top
            \@@_shrink_range:
          }
      }
      {
        \cs_gset_protected:Npn \@@_compute_range:
          {
            \int_set:Nn \l_@@_min_int { \tex_count:D 265 }
            \int_set:Nn \l_@@_true_max_int { \tex_count:D 275 }
            \@@_shrink_range:
          }
      }
  }
\cs_if_exist:NT \loctoks { \@@_redefine_compute_range: }
\tl_map_inline:nn { \lastallocatedtoks \c_syst_last_allocated_toks }
  {
    \cs_if_exist:NT #1
      {
        \cs_gset_protected:Npn \@@_compute_range:
          {
            \int_set:Nn \l_@@_min_int { #1 + 1 }
            \int_set:Nn \l_@@_true_max_int { \c_max_register_int + 1 }
            \@@_shrink_range:
          }
      }
  }
%</package>
%<*initex>
\int_const:Nn \c_@@_max_length_int
  { ( \c_max_register_int + 1 ) * 3 / 4 }
\cs_new_protected:Npn \@@_compute_range:
  {
    \int_set:Nn \l_@@_min_int { 0 }
    \int_set:Nn \l_@@_true_max_int { \c_max_register_int + 1 }
    \int_set:Nn \l_@@_max_int { \c_@@_max_length_int }
  }
%</initex>
%    \end{macrocode}
% \end{variable}
% \end{macro}
%
% \subsection{Protected user commands}
%
% \begin{macro}[int]{\@@_main:NNNnNn}
%   Sorting happens in three steps. First store items in \tn{toks}
%   registers ranging from \cs{l_@@_min_int} to $\cs{l_@@_top_int}-1$,
%   while checking that the list is not too
%   long. If we reach the maximum length, all further
%   items are entirely ignored after raising an error.
%   Secondly, sort the array of \tn{toks} registers,
%   using the user-defined sorting function, |#6|.
%   Finally, unpack the \tn{toks} registers (now sorted)
%   into a variable of the right type, by \texttt{x}-expanding
%   the code in |#4|, specific to each type of list.
%    \begin{macrocode}
\cs_new_protected:Npn \@@_main:NNNnNn #1#2#3#4#5#6
  {
    \group_begin:
%<package>      \@@_disable_toksdef:
      \@@_compute_range:
      \int_set_eq:NN \l_@@_top_int \l_@@_min_int
      #2 #5
        {
          \if_int_compare:w \l_@@_top_int = \l_@@_max_int
            \@@_too_long_error:NNw #3 #5
          \fi:
          \tex_toks:D \l_@@_top_int {##1}
          \tex_advance:D \l_@@_top_int \c_one
        }
      \int_set:Nn \l_@@_length_int
        { \l_@@_top_int - \l_@@_min_int }
      \cs_set:Npn \@@_compare:nn ##1 ##2 { #6 }
      \int_set_eq:NN \l_@@_block_int \c_one
      \@@_level:
      \use:x
        {
          \group_end:
          #1 \exp_not:N #5 {#4}
        }
  }
%    \end{macrocode}
% \end{macro}
%
% \begin{macro}{\seq_sort:Nn, \seq_gsort:Nn}
%   The first argument to \cs{@@_main:NNNnNn} is the final
%   assignment function used, either \cs{tl_set:Nn} or
%   \cs{tl_gset:Nn} to control local versus global results.
%   The second argument is what mapping function is used when storing
%   items to \tn{toks} registers, and the third breaks away from the
%   loop. The fourth is used to build back the correct kind of list
%   from the contents of the \tn{toks} registers, including the leading
%   \cs{s__seq}.  Fifth and sixth
%   arguments are the variable to sort, and the sorting method
%   as inline code.
%    \begin{macrocode}
\cs_new_protected:Npn \seq_sort:Nn
  {
    \@@_main:NNNnNn \tl_set:Nn
      \seq_map_inline:Nn \seq_map_break:n
      { \s__seq \@@_toks:NN \exp_not:N \__seq_item:n }
  }
\cs_generate_variant:Nn \seq_sort:Nn { c }
\cs_new_protected:Npn \seq_gsort:Nn
  {
    \@@_main:NNNnNn \tl_gset:Nn
      \seq_map_inline:Nn \seq_map_break:n
      { \s__seq \@@_toks:NN \exp_not:N \__seq_item:n }
  }
\cs_generate_variant:Nn \seq_gsort:Nn { c }
%    \end{macrocode}
% \end{macro}
%
% \begin{macro}{\tl_sort:Nn, \tl_sort:cn, \tl_gsort:Nn, \tl_gsort:cn}
%   Again, use \cs{tl_set:Nn} or \cs{tl_gset:Nn} to control
%   the scope of the assignment. Mapping through the token
%   list is done with \cs{tl_map_inline:Nn}, and producing
%   the token list is very similar to sequences, removing
%   \cs{__seq_item:n}.
%    \begin{macrocode}
\cs_new_protected:Npn \tl_sort:Nn
  {
    \@@_main:NNNnNn \tl_set:Nn
      \tl_map_inline:Nn \tl_map_break:n
      { \@@_toks:NN \prg_do_nothing: \prg_do_nothing: }
  }
\cs_generate_variant:Nn \tl_sort:Nn { c }
\cs_new_protected:Npn \tl_gsort:Nn
  {
    \@@_main:NNNnNn \tl_gset:Nn
      \tl_map_inline:Nn \tl_map_break:n
      { \@@_toks:NN \prg_do_nothing: \prg_do_nothing: }
  }
\cs_generate_variant:Nn \tl_gsort:Nn { c }
%    \end{macrocode}
% \end{macro}
%
% \begin{macro}
%   {\clist_sort:Nn, \clist_sort:cn, \clist_gsort:Nn, \clist_gsort:cn}
% \begin{macro}[aux]{\@@_clist:NNn}
%   The case of empty comma-lists is a little bit special as usual,
%   and filtered out: there is nothing to sort in that case.
%   Otherwise, the input is done with \cs{clist_map_inline:Nn},
%   and the output requires some more elaborate processing than
%   for sequences and token lists. The first comma must be removed.
%   An item must be wrapped in an extra set of braces if it contains
%   either the space or the comma characters. This is taken care of
%   by \cs{clist_wrap_item:n}, but \cs{@@_toks:NN} would simply
%   feed \cs{tex_the:D} \cs{tex_toks:D} \meta{number} as an
%   argument to that function; hence we need to expand this argument
%   once to unpack the register.
%    \begin{macrocode}
\cs_new_protected:Npn \clist_sort:Nn
  { \@@_clist:NNn \tl_set:Nn }
\cs_new_protected:Npn \clist_gsort:Nn
  { \@@_clist:NNn \tl_gset:Nn }
\cs_generate_variant:Nn \clist_sort:Nn  { c }
\cs_generate_variant:Nn \clist_gsort:Nn { c }
\cs_new_protected:Npn \@@_clist:NNn #1#2#3
  {
    \clist_if_empty:NF #2
      {
        \@@_main:NNNnNn #1
          \clist_map_inline:Nn \clist_map_break:n
          {
            \exp_last_unbraced:Nf \use_none:n
              { \@@_toks:NN \exp_args:No \__clist_wrap_item:n }
          }
          #2 {#3}
      }
  }
%    \end{macrocode}
% \end{macro}
% \end{macro}
%
% \begin{macro}{\@@_toks:NN, \@@_toks:NNw}
%   Unpack the various \tn{toks} registers, from \cs{l_@@_min_int} to
%   $\cs{l_@@_top_int}-1$. The functions |#1| and |#2| allow us to treat
%   the three data structures in a unified way:
%   \begin{itemize}
%   \item for sequences, they are \cs{exp_not:N} \cs{__seq_item:n},
%     expanding to the \cs{__seq_item:n} separator, as expected;
%   \item for token lists, they expand to nothing;
%   \item for comma lists, they expand to \cs{exp_args:No}
%     \cs{clist_wrap_item:n}, taking care of unpacking the register
%     before letting the undocumented internal \pkg{clist} function
%     \cs{clist_wrap_item:n} do the work of putting a comma and possibly
%     braces.
%   \end{itemize}
%    \begin{macrocode}
\cs_new:Npn \@@_toks:NN #1#2
  { \@@_toks:NNw #1 #2 \l_@@_min_int ; }
\cs_new:Npn \@@_toks:NNw #1#2#3 ;
  {
    \if_int_compare:w #3 < \l_@@_top_int
      #1 #2 { \tex_the:D \tex_toks:D #3 }
      \exp_after:wN \@@_toks:NNw \exp_after:wN #1 \exp_after:wN #2
      \__int_value:w \__int_eval:w #3 + \c_one \exp_after:wN ;
    \fi:
  }
%    \end{macrocode}
% \end{macro}
%
% \subsection{Merge sort}
%
% \begin{macro}[int]{\@@_level:}
%   This function is called once blocks of size \cs{l_@@_block_int}
%   (initially $1$) are each sorted. If the whole list fits in one
%   block, then we are done (this also takes care of the case of an
%   empty list or a list with one item). Otherwise, go through pairs
%   of blocks starting from $0$, then double the block size, and repeat.
%    \begin{macrocode}
\cs_new_protected:Npn \@@_level:
  {
    \if_int_compare:w \l_@@_block_int < \l_@@_length_int
      \l_@@_end_int \l_@@_min_int
      \@@_merge_blocks:
      \tex_advance:D \l_@@_block_int \l_@@_block_int
      \exp_after:wN \@@_level:
    \fi:
  }
%    \end{macrocode}
% \end{macro}
%
% \begin{macro}[int]{\@@_merge_blocks:}
%   This function is called to merge a pair of blocks, starting at
%   the last value of \cs{l_@@_end_int} (end-point of the previous
%   pair of blocks). If shifting by one block to the right we reach
%   the end of the list, then this pass has ended: the end of the
%   list is sorted already. Otherwise, store the result of that shift in $A$,
%   which will index the first block starting from the top end.
%   Then locate the end-point (maximum) of the second block: shift
%   \texttt{end} upwards by one more block, but keeping it
%   $\leq\texttt{top}$. Copy this upper block of \tn{toks}
%   registers in registers above \texttt{length}, indexed by $C$:
%   this is covered by \cs{@@_copy_block:}. Once this is done we
%   are ready to do the actual merger using \cs{@@_merge_blocks_aux:},
%   after shifting $A$, $B$ and $C$ so that they point to the largest
%   index in their respective ranges rather than pointing just beyond
%   those ranges. Of course, once that pair of blocks is merged,
%   move on to the next pair.
%    \begin{macrocode}
\cs_new_protected:Npn \@@_merge_blocks:
  {
    \l_@@_begin_int \l_@@_end_int
    \tex_advance:D \l_@@_end_int \l_@@_block_int
    \if_int_compare:w \l_@@_end_int < \l_@@_top_int
      \l_@@_A_int \l_@@_end_int
      \tex_advance:D \l_@@_end_int \l_@@_block_int
      \if_int_compare:w \l_@@_end_int > \l_@@_top_int
        \l_@@_end_int \l_@@_top_int
      \fi:
      \l_@@_B_int \l_@@_A_int
      \l_@@_C_int \l_@@_top_int
      \@@_copy_block:
      \tex_advance:D \l_@@_A_int - \c_one
      \tex_advance:D \l_@@_B_int - \c_one
      \tex_advance:D \l_@@_C_int - \c_one
      \exp_after:wN \@@_merge_blocks_aux:
      \exp_after:wN \@@_merge_blocks:
    \fi:
  }
%    \end{macrocode}
% \end{macro}
%
% \begin{macro}[int]{\@@_copy_block:}
%   We wish to store a copy of the \enquote{upper} block of
%   \tn{toks} registers, ranging between the initial value of
%   \cs{l_@@_B_int} (included) and \cs{l_@@_end_int}
%   (excluded) into a new range starting at the initial value
%   of \cs{l_@@_C_int}, namely \cs{l_@@_top_int}.
%    \begin{macrocode}
\cs_new_protected:Npn \@@_copy_block:
  {
    \tex_toks:D \l_@@_C_int \tex_toks:D \l_@@_B_int
    \tex_advance:D \l_@@_C_int \c_one
    \tex_advance:D \l_@@_B_int \c_one
    \if_int_compare:w \l_@@_B_int = \l_@@_end_int
      \use_i:nn
    \fi:
    \@@_copy_block:
  }
%    \end{macrocode}
% \end{macro}
%
% \begin{macro}[aux]{\@@_merge_blocks_aux:}
%   At this stage, the first block starts at \cs{l_@@_begin_int},
%   and ends at \cs{l_@@_A_int}, and the second block starts at
%   \cs{l_@@_top_int} and ends at \cs{l_@@_C_int}. The result
%   of the merger is stored at positions indexed by \cs{l_@@_B_int},
%   which starts at $\cs{l_@@_end_int}-1$ and decreases down to
%   \cs{l_@@_begin_int}, covering the full range of the two blocks.
%   In other words, we are building the merger starting with the
%   largest values.
%   The comparison function is defined to return either
%   \texttt{swapped} or \texttt{same}. Of course, this
%   means the arguments need to be given in the order they
%   appear originally in the list.
%    \begin{macrocode}
\cs_new_protected:Npn \@@_merge_blocks_aux:
  {
    \exp_after:wN \@@_compare:nn \exp_after:wN
      { \tex_the:D \tex_toks:D \exp_after:wN \l_@@_A_int \exp_after:wN }
      \exp_after:wN { \tex_the:D \tex_toks:D \l_@@_C_int }
    \prg_do_nothing:
    \@@_return_mark:N
    \@@_return_mark:N
    \@@_return_none_error:
  }
%    \end{macrocode}
% \end{macro}
%
% \begin{macro}{\sort_return_same:, \sort_return_swapped:}
% \begin{macro}[aux]{\@@_return_mark:N}
% \begin{macro}[aux]{\@@_return_none_error:, \@@_return_two_error:w}
%   The marker removes one token.  Each comparison should call
%   \cs{sort_return_same:} or \cs{sort_return_swapped:} exactly
%   once.  If neither is called, \cs{@@_return_none_error:} is called.
%    \begin{macrocode}
\cs_new_protected:Npn \sort_return_same: #1 \@@_return_mark:N
  { #1 \@@_return_mark:N \@@_return_two_error:w \@@_return_same: }
\cs_new_protected:Npn \sort_return_swapped: #1 \@@_return_mark:N
  { #1 \@@_return_mark:N \@@_return_two_error:w \@@_return_swapped: }
\cs_new_protected:Npn \@@_return_mark:N #1 { }
\cs_new_protected:Npn \@@_return_none_error:
  {
    \__msg_kernel_error:nnxx { sort } { return-none }
      { \tex_the:D \tex_toks:D \l_@@_A_int }
      { \tex_the:D \tex_toks:D \l_@@_C_int }
    \@@_return_same:
  }
\cs_new_protected:Npn \@@_return_two_error:w
    #1 \@@_return_none_error:
  { \__msg_kernel_error:nn { sort } { return-two } }
%    \end{macrocode}
% \end{macro}
% \end{macro}
% \end{macro}
%
% \begin{macro}[aux]{\@@_return_same:}
%   If the comparison function returns \texttt{same},
%   then the second argument fed to \cs{@@_compare:nn}
%   should remain to the right of the other one. Since
%   we build the merger starting from the right, we copy
%   that \tn{toks} register into the allotted range, then
%   shift the pointers $B$ and $C$, and go on to do one
%   more step in the merger, unless the second block has
%   been exhausted: then the remainder of the first block
%   is already in the correct registers and we are done
%   with merging those two blocks.
%    \begin{macrocode}
\cs_new_protected:Npn \@@_return_same:
  {
    \tex_toks:D \l_@@_B_int \tex_toks:D \l_@@_C_int
    \tex_advance:D \l_@@_B_int - \c_one
    \tex_advance:D \l_@@_C_int - \c_one
    \if_int_compare:w \l_@@_C_int < \l_@@_top_int
      \use_i:nn
    \fi:
    \@@_merge_blocks_aux:
  }
%    \end{macrocode}
% \end{macro}
%
% \begin{macro}[aux]{\@@_return_swapped:}
%   If the comparison function returns \texttt{swapped},
%   then the next item to add to the merger is the first
%   argument, contents of the \tn{toks} register $A$.
%   Then shift the pointers $A$ and $B$ to the left, and
%   go for one more step for the merger, unless the left
%   block was exhausted ($A$ goes below the threshold).
%   In that case, all remaining \tn{toks} registers in
%   the second block, indexed by $C$, are copied
%   to the merger by \cs{@@_merge_blocks_end:}.
%    \begin{macrocode}
\cs_new_protected:Npn \@@_return_swapped:
  {
    \tex_toks:D \l_@@_B_int \tex_toks:D \l_@@_A_int
    \tex_advance:D \l_@@_B_int - \c_one
    \tex_advance:D \l_@@_A_int - \c_one
    \if_int_compare:w \l_@@_A_int < \l_@@_begin_int
      \@@_merge_blocks_end: \use_i:nn
    \fi:
    \@@_merge_blocks_aux:
  }
%    \end{macrocode}
% \end{macro}
%
% \begin{macro}[aux]{\@@_merge_blocks_end:}
%   This function's task is to copy the \tn{toks} registers
%   in the block indexed by $C$ to the merger indexed by $B$.
%   The end can equally be detected by checking when $B$ reaches
%   the threshold \texttt{begin}, or when $C$ reaches
%   \texttt{top}.
%    \begin{macrocode}
\cs_new_protected:Npn \@@_merge_blocks_end:
  {
    \tex_toks:D \l_@@_B_int \tex_toks:D \l_@@_C_int
    \tex_advance:D \l_@@_B_int - \c_one
    \tex_advance:D \l_@@_C_int - \c_one
    \if_int_compare:w \l_@@_B_int < \l_@@_begin_int
      \use_i:nn
    \fi:
    \@@_merge_blocks_end:
  }
%    \end{macrocode}
% \end{macro}
%
% \subsection{Expandable sorting}
%
% Sorting expandably is very different from sorting and assigning to a
% variable.  Since tokens cannot be stored, they must remain in the
% input stream, and be read through at every step.  It is thus
% necessarily much slower (at best $O(n^2\ln n)$) than non-expandable
% sorting functions ($O(n\ln n)$).
%
% A prototypical version of expandable quicksort is as follows.  If the
% argument has no item, return nothing, otherwise partition, using the
% first item as a pivot (argument |#4| of \cs{@@:nnNnn}).  The
% arguments of \cs{@@:nnNnn} are 1.~items less than |#4|, 2.~items
% greater or equal to |#4|, 3.~comparison, 4.~pivot, 5.~next item to
% test.  If |#5| is the tail of the list, call \cs{tl_sort:nN} on |#1|
% and on |#2|, placing |#4| in between; |\use:ff| expands the parts to
% make \cs{tl_sort:nN} \texttt{f}-expandable.  Otherwise, compare |#4|
% and |#5| using |#3|.  If they are ordered, place |#5| amongst the
% \enquote{greater} items, otherwise amongst the \enquote{lesser} items,
% and continue partitioning.
% \begin{verbatim}
% \cs_new:Npn \tl_sort:nN #1#2
%   {
%     \tl_if_blank:nF {#1}
%       {
%         \__sort:nnNnn { } { } #2
%           #1 \q_recursion_tail \q_recursion_stop
%       }
%   }
% \cs_new:Npn \__sort:nnNnn #1#2#3#4#5
%   {
%     \quark_if_recursion_tail_stop_do:nn {#5}
%       { \use:ff { \tl_sort:nN {#1} #3 {#4} } { \tl_sort:nN {#2} #3 } }
%     #3 {#4} {#5}
%       { \__sort:nnNnn {#1} { #2 {#5} } #3 {#4} }
%       { \__sort:nnNnn { #1 {#5} } {#2} #3 {#4} }
%   }
% \cs_generate_variant:Nn \use:nn { ff }
% \end{verbatim}
% There are quite a few optimizations available here: the code below is
% less legible, but more than twice as fast.
%
% In the simple version of the code, \cs{@@:nnNnn} is called
% \(O(n\ln n)\) times on average (the number of comparisons required by
% the quicksort algorithm).  Hence most of our focus will be on
% optimizing that function.
%
% The first speed up is to avoid testing for the end of the list at
% every call to \cs{@@:nnNnn}.  For this, the list is prepared by
% changing each \meta{item} of the original token list into
% \meta{command} \Arg{item}, just like sequences are stored.  We arrange
% things such that the \meta{command} is the \meta{conditional} provided
% by the user: the loop over the \meta{prepared tokens} then looks like
% \begin{quote}
%   \ttfamily
%   \cs{cs_new:Npn}~\cs{@@_loop:wNn}~\ldots{}~|#6#7|\\
%   ~~|{|\\
%   ~~~~|#6|~\Arg{pivot}~|{#7}|~\meta{loop big}~\meta{loop small}\\
%   ~~~~~~\meta{extra arguments}\\
%   ~~|}|\\
%   \cs{@@_loop:wNn}~\ldots{}~\meta{prepared tokens}\\
%   ~~\meta{end-loop}~|{}|~|\q_stop|
% \end{quote}
% In this example, which matches the structure of
% \cs{@@_quick_split_i:NnnnnNn} and a few other functions below, the
% \cs{@@_loop:wNn} auxiliary normally receives the user's
% \meta{conditional} as~|#6| and an \meta{item} as~|#7|.  This is
% compared to the \meta{pivot} (the argument~|#5|, not shown here), and
% the \meta{conditional} leaves the \meta{loop big} or \meta{loop small}
% auxiliary, which both have the same form as \cs{@@_loop:wNn},
% receiving the next pair \meta{conditional} \Arg{item} as |#6|
% and~|#7|.  At the end, |#6| is the \meta{end-loop} function, which
% terminates the loop.
%
% The second speed up is to minimize the duplicated tokens between the
% \texttt{true} and \texttt{false} branches of the conditional.  For
% this, we introduce two versions of \cs{@@:nnNnn}, which receive
% the new item as~|#1| and place it either into the list~|#2| of items
% less than the pivot~|#4| or into the list~|#3| of items greater or
% equal to the pivot.
% \begin{verbatim}
% \cs_new:Npn \__sort_i:nnnnNn #1#2#3#4#5#6
%   {
%     #5 {#4} {#6} \__sort_ii:nnnnNn \__sort_i:nnnnNn
%       {#6} { #2 {#1} } {#3} {#4}
%   }
% \cs_new:Npn \__sort_ii:nnnnNn #1#2#3#4#5#6
%   {
%     #5 {#4} {#6} \__sort_ii:nnnnNn \__sort_i:nnnnNn
%       {#6} {#2} { #3 {#1} } {#4}
%   }
% \end{verbatim}
% Note that the two functions have the form of \cs{@@_loop:wNn} above,
% receiving as~|#5| the conditional or a function to end the loop.  In
% fact, the lists~|#2| and~|#3| must be made of pairs \meta{conditional}
% \Arg{item}, so we have to replace~|{#6}| above by |{|~|#5|~|{#6}|~|}|,
% and |{#1}|~by~|#1|.  The actual functions have one more argument, so
% all argument numbers are shifted compared to this code.
%
% The third speed up is to avoid |\use:ff| using a continuation-passing
% style: \cs{@@_quick_split:NnNn} expects a list followed by
% \cs{q_mark} \Arg{code}, and expands to \meta{code} \meta{sorted list}.
% Sorting the two parts of the list around the pivot is done with
% \begin{quote}
%   \ttfamily
%   \cs{@@_quick_split:NnNn} |#2| \ldots{} \cs{q_mark}\\
%   ~~|{|\\
%   ~~~~\cs{@@_quick_split:NnNn} |#1| \ldots{} \cs{q_mark} \Arg{code}\\
%   ~~~~\Arg{pivot}\\
%   ~~|}|
% \end{quote}
% Items which are larger than the \meta{pivot} are sorted, then placed
% after code that sorts the smaller items, and after the (braced)
% \meta{pivot}.
%
% The fourth speed up is avoid the recursive call to \cs{tl_sort:nN}
% with an empty first argument.  For this, we introduce functions
% similar to the \cs{@@_i:nnnnNn} of the last example, but aware of
% whether the list of \meta{conditional} \Arg{item} read so far that are
% less than the pivot, and the list of those greater or equal, are empty
% or not: see \cs{@@_quick_split:NnNn} and functions defined below.
% Knowing whether the lists are empty or not is useless if we do not use
% distinct ending codes as appropriate.  The splitting auxiliaries
% communicate to the \meta{end-loop} function (that is initially placed
% after the ``prepared'' list) by placing a specific ending function,
% ignored when looping, but useful at the end.  In fact, the
% \meta{end-loop} function does nothing but place the appropriate ending
% function in front of all its arguments.  The ending functions take
% care of sorting non-empty sublists, placing the pivot in between, and
% the continuation before.
%
% The final change in fact slows down the code a little, but is required
% to avoid memory issues: schematically, when \TeX{} encounters
% \begin{verbatim}
%   \use:n { \use:n { \use:n { ... } ... } ... }
% \end{verbatim}
% the argument of the first \cs{use:n} is not completely read by the
% second \cs{use:n}, hence must remain in memory; then the argument of
% the second \cs{use:n} is not completely read when grabbing the
% argument of the third \cs{use:n}, hence must remain in memory, and so
% on.  The memory consumption grows quadratically with the number of
% nested \cs{use:n}.  In practice, this means that we must read
% everything until a trailing \cs{q_stop} once in a while, otherwise
% sorting lists of more than a few thousand items would exhaust a
% typical \TeX{}'s memory.
%
% \begin{macro}[EXP]{\tl_sort:nN}
% \begin{macro}[aux, EXP]
%   {
%     \@@_quick_prepare:Nnnn,
%     \@@_quick_prepare_end:NNNnw,
%     \@@_quick_cleanup:w
%   }
%   The code within the \cs{exp_not:f} sorts the list, leaving in most
%   cases a leading \cs{exp_not:f}, which stops the expansion, letting
%   the result be return within \cs{exp_not:n}.  We filter out the case
%   of a list with no item, which would otherwise cause problems.  Then
%   prepare the token list~|#1| by inserting the conditional~|#2| before
%   each item.  The \texttt{prepare} auxiliary receives the conditional
%   as~|#1|, the prepared token list so far as~|#2|, the next prepared
%   item as~|#3|, and the item after that as~|#4|.  The loop ends
%   when~|#4| contains \cs{__prg_break_point:}, then the
%   \texttt{prepare_end} auxiliary finds the prepared token list
%   as~|#4|.  The scene is then set up for \cs{@@_quick_split:NnNn},
%   which will sort the prepared list and perform the post action placed
%   after \cs{q_mark}, namely removing the trailing \cs{s__stop} and
%   \cs{q_stop} and leaving \cs{exp_stop_f:} to stop
%   \texttt{f}-expansion.
%    \begin{macrocode}
\cs_new:Npn \tl_sort:nN #1#2
  {
    \exp_not:f
      {
        \tl_if_blank:nF {#1}
          {
            \@@_quick_prepare:Nnnn #2 { } { }
              #1
              { \__prg_break_point: \@@_quick_prepare_end:NNNnw }
            \q_stop
          }
      }
  }
\cs_new:Npn \@@_quick_prepare:Nnnn #1#2#3#4
  {
    \__prg_break: #4 \__prg_break_point:
    \@@_quick_prepare:Nnnn #1 { #2 #3 } { #1 {#4} }
  }
\cs_new:Npn \@@_quick_prepare_end:NNNnw #1#2#3#4#5 \q_stop
  {
    \@@_quick_split:NnNn #4 \@@_quick_end:nnTFNn { }
    \q_mark { \@@_quick_cleanup:w \exp_stop_f: }
    \s__stop \q_stop
  }
\cs_new:Npn \@@_quick_cleanup:w #1 \s__stop \q_stop {#1}
%    \end{macrocode}
% \end{macro}
% \end{macro}
%
% \begin{macro}[EXP, aux]
%   {
%     \@@_quick_split:NnNn,
%     \@@_quick_only_i:NnnnnNn,
%     \@@_quick_only_ii:NnnnnNn,
%     \@@_quick_split_i:NnnnnNn,
%     \@@_quick_split_ii:NnnnnNn
%   }
%   The \texttt{only_i}, \texttt{only_ii}, \texttt{split_i} and
%   \texttt{split_ii} auxiliaries receive a useless first argument, the
%   new item~|#2| (that they append to either one of the next two
%   arguments), the list~|#3| of items less than the pivot, bigger
%   items~|#4|, the pivot~|#5|, a \meta{function}~|#6|, and an
%   item~|#7|.  The \meta{function} is the user's \meta{conditional}
%   except at the end of the list where it is
%   \cs{@@_quick_end:nnTFNn}.  The comparison is applied to the
%   \meta{pivot} and the \meta{item}, and calls the \texttt{only_i} or
%   \texttt{split_i} auxiliaries if the \meta{item} is smaller, and the
%   \texttt{only_ii} or \texttt{split_ii} auxiliaries otherwise.  In
%   both cases, the next auxiliary goes to work right away, with no
%   intermediate expansion that would slow down operations.  Note that
%   the argument~|#2| left for the next call has the form
%   \meta{conditional} \Arg{item}, so that the lists~|#3| and~|#4| keep
%   the right form to be fed to the next sorting function.
%   The \texttt{split} auxiliary differs from these in that it is
%   missing three of the arguments, which would be empty, and its first
%   argument is always the user's \meta{conditional} rather than an
%   ending function.
%    \begin{macrocode}
\cs_new:Npn \@@_quick_split:NnNn #1#2#3#4
  {
    #3 {#2} {#4} \@@_quick_only_ii:NnnnnNn \@@_quick_only_i:NnnnnNn
      \@@_quick_single_end:nnnwnw
      { #3 {#4} } { } { } {#2}
  }
\cs_new:Npn \@@_quick_only_i:NnnnnNn #1#2#3#4#5#6#7
  {
    #6 {#5} {#7} \@@_quick_split_ii:NnnnnNn \@@_quick_only_i:NnnnnNn
      \@@_quick_only_i_end:nnnwnw
      { #6 {#7} } { #3 #2 } { } {#5}
  }
\cs_new:Npn \@@_quick_only_ii:NnnnnNn #1#2#3#4#5#6#7
  {
    #6 {#5} {#7} \@@_quick_only_ii:NnnnnNn \@@_quick_split_i:NnnnnNn
      \@@_quick_only_ii_end:nnnwnw
      { #6 {#7} } { } { #4 #2 } {#5}
  }
\cs_new:Npn \@@_quick_split_i:NnnnnNn #1#2#3#4#5#6#7
  {
    #6 {#5} {#7} \@@_quick_split_ii:NnnnnNn \@@_quick_split_i:NnnnnNn
      \@@_quick_split_end:nnnwnw
      { #6 {#7} } { #3 #2 } {#4} {#5}
  }
\cs_new:Npn \@@_quick_split_ii:NnnnnNn #1#2#3#4#5#6#7
  {
    #6 {#5} {#7} \@@_quick_split_ii:NnnnnNn \@@_quick_split_i:NnnnnNn
      \@@_quick_split_end:nnnwnw
      { #6 {#7} } {#3} { #4 #2 } {#5}
  }
%    \end{macrocode}
% \end{macro}
%
% \begin{macro}[EXP, aux]
%   {
%     \@@_quick_end:nnTFNn,
%     \@@_quick_single_end:nnnwnw,
%     \@@_quick_only_i_end:nnnwnw,
%     \@@_quick_only_ii_end:nnnwnw,
%     \@@_quick_split_end:nnnwnw,
%   }
%   The \cs{@@_quick_end:nnTFNn} appears instead of the user's
%   conditional, and receives as its arguments the pivot~|#1|, a fake
%   item~|#2|, a \texttt{true} and a \texttt{false} branches |#3|
%   and~|#4|, followed by an ending function~|#5| (one of the four
%   auxiliaries here) and another copy~|#6| of the fake item.  All those
%   are discarded except the function~|#5|.  This function receives
%   lists~|#1| and~|#2| of items less than or greater than the
%   pivot~|#3|, then a continuation code~|#5| just after \cs{q_mark}.
%   To avoid a memory problem described earlier, all of the ending
%   functions read~|#6| until \cs{q_stop} and place~|#6| back into the
%   input stream.  When the lists |#1| and~|#2| are empty, the
%   \texttt{single} auxiliary simply places the continuation~|#5| before
%   the pivot~|{#3}|.  When |#2|~is empty, |#1|~is sorted and placed
%   before the pivot~|{#3}|, taking care to feed the continuation~|#5|
%   as a continuation for the function sorting~|#1|.  When |#1|~is
%   empty, |#2|~is sorted, and the continuation argument is used to
%   place the continuation~|#5| and the pivot~|{#3}| before the sorted
%   result.  Finally, when both lists are non-empty, items larger than
%   the pivot are sorted, then items less than the pivot, and the
%   continuations are done in such a way to place the pivot in between.
%    \begin{macrocode}
\cs_new:Npn \@@_quick_end:nnTFNn #1#2#3#4#5#6 {#5}
\cs_new:Npn \@@_quick_single_end:nnnwnw #1#2#3#4 \q_mark #5#6 \q_stop
  { #5 {#3} #6 \q_stop }
\cs_new:Npn \@@_quick_only_i_end:nnnwnw #1#2#3#4 \q_mark #5#6 \q_stop
  {
    \@@_quick_split:NnNn #1
      \@@_quick_end:nnTFNn { } \q_mark {#5}
    {#3}
    #6 \q_stop
  }
\cs_new:Npn \@@_quick_only_ii_end:nnnwnw #1#2#3#4 \q_mark #5#6 \q_stop
  {
    \@@_quick_split:NnNn #2
      \@@_quick_end:nnTFNn { } \q_mark { #5 {#3} }
    #6 \q_stop
  }
\cs_new:Npn \@@_quick_split_end:nnnwnw #1#2#3#4 \q_mark #5#6 \q_stop
  {
    \@@_quick_split:NnNn #2 \@@_quick_end:nnTFNn { } \q_mark
      {
        \@@_quick_split:NnNn #1
          \@@_quick_end:nnTFNn { } \q_mark {#5}
        {#3}
      }
    #6 \q_stop
  }
%    \end{macrocode}
% \end{macro}
%
% \subsection{Messages}
%
% \begin{macro}[aux]{\@@_error:}
%   Bailing out of the sorting code is a bit tricky.  It may not be safe
%   to use a delimited argument, so instead we redefine many
%   \pkg{l3sort} commands to be trivial, with \cs{@@_level:} getting rid
%   of the final assignment.  This error recovery won't work in a group.
%    \begin{macrocode}
\cs_new_protected:Npn \@@_error:
  {
    \cs_set_eq:NN \@@_merge_blocks_aux: \prg_do_nothing:
    \cs_set_eq:NN \@@_merge_blocks: \prg_do_nothing:
    \cs_set_protected:Npn \@@_level: \use:x ##1 { \group_end: }
  }
%    \end{macrocode}
% \end{macro}
%
% \begin{macro}[aux]{\@@_disable_toksdef:, \@@_disabled_toksdef:n}
%   While sorting, \tn{toksdef} is locally disabled to prevent users
%   from using \tn{newtoks} or similar commands in their comparison
%   code: the \tn{toks} registers that would be assigned are in use by
%   \pkg{l3sort}.  In format mode, none of this is needed since there is
%   no \tn{toks} allocator.
%    \begin{macrocode}
%<*package>
\cs_new_protected:Npn \@@_disable_toksdef:
  { \cs_set_eq:NN \toksdef \@@_disabled_toksdef:n }
\cs_new_protected:Npn \@@_disabled_toksdef:n #1
  {
    \__msg_kernel_error:nnx { sort } { toksdef }
      { \token_to_str:N #1 }
    \@@_error:
    \tex_toksdef:D #1
  }
\__msg_kernel_new:nnnn { sort } { toksdef }
  { Allocation~of~\iow_char:N\\toks~registers~impossible~while~sorting. }
  {
    The~comparison~code~used~for~sorting~a~list~has~attempted~to~
    define~#1~as~a~new~\iow_char:N\\toks~register~using~\iow_char:N\\newtoks~
    or~a~similar~command.~The~list~will~not~be~sorted.
  }
%</package>
%    \end{macrocode}
% \end{macro}
%
% \begin{macro}[aux]{\@@_too_long_error:NNw}
%   When there are too many items in a sequence, this is an error, and
%   we clean up properly the mapping over items in the list: break using
%   the type-specific breaking function |#1|.
%    \begin{macrocode}
\cs_new_protected:Npn \@@_too_long_error:NNw #1#2 \fi:
  {
    \fi:
    \__msg_kernel_error:nnxxx { sort } { too-large }
      { \token_to_str:N #2 }
      { \int_eval:n { \l_@@_true_max_int - \l_@@_min_int } }
      { \int_eval:n { \l_@@_top_int - \l_@@_min_int } }
    #1 \@@_error:
  }
\__msg_kernel_new:nnnn { sort } { too-large }
  { The~list~#1~is~too~long~to~be~sorted~by~TeX. }
  {
    TeX~has~#2~toks~registers~still~available:~
    this~only~allows~to~sort~with~up~to~#3~
    items.~All~extra~items~will~be~deleted.
  }
%    \end{macrocode}
% \end{macro}
%
%    \begin{macrocode}
\__msg_kernel_new:nnnn { sort } { return-none }
  { The~comparison~code~did~not~return. }
  {
    When~sorting~a~list,~the~code~to~compare~items~#1~and~#2~
    did~not~call~
    \iow_char:N\\sort_return_same: ~nor~
    \iow_char:N\\sort_return_swapped: .~
    Exactly~one~of~these~should~be~called.
  }
\__msg_kernel_new:nnnn { sort } { return-two }
  { The~comparison~code~returned~multiple~times. }
  {
    When~sorting~a~list,~the~code~to~compare~items~called~
    \iow_char:N\\sort_return_same: ~or~
    \iow_char:N\\sort_return_swapped: ~multiple~times.~
    Exactly~one~of~these~should~be~called.
  }
%    \end{macrocode}
%
% \subsection{Deprecated functions}
%
% \begin{macro}[deprecated = 2018-12-31]{\sort_ordered:, \sort_reversed:}
%   These functions were renamed for consistency.
%    \begin{macrocode}
\cs_new_protected:Npn \sort_ordered:
  {
    \__msg_kernel_warning:nnxxx { kernel } { deprecated-command }
      { 2018-12-31 }
      { \token_to_str:N \sort_ordered: }
      { \token_to_str:N \sort_return_same: }
    \cs_gset_eq:NN \sort_ordered: \sort_return_same:
    \sort_return_same:
  }
\cs_new_protected:Npn \sort_reversed:
  {
    \__msg_kernel_warning:nnxxx { kernel } { deprecated-command }
      { 2018-12-31 }
      { \token_to_str:N \sort_reversed: }
      { \token_to_str:N \sort_return_swapped: }
    \cs_gset_eq:NN \sort_reversed: \sort_return_swapped:
    \sort_return_swapped:
  }
%    \end{macrocode}
% \end{macro}
%
%    \begin{macrocode}
%</initex|package>
%    \end{macrocode}
%
% \end{implementation}
%
% \PrintIndex
