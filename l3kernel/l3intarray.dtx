% \iffalse meta-comment
%
%% File: l3intarray.dtx Copyright (C) 2017-2018 The LaTeX3 Project
%
% It may be distributed and/or modified under the conditions of the
% LaTeX Project Public License (LPPL), either version 1.3c of this
% license or (at your option) any later version.  The latest version
% of this license is in the file
%
%    https://www.latex-project.org/lppl.txt
%
% This file is part of the "l3kernel bundle" (The Work in LPPL)
% and all files in that bundle must be distributed together.
%
% -----------------------------------------------------------------------
%
% The development version of the bundle can be found at
%
%    https://github.com/latex3/latex3
%
% for those people who are interested.
%
%<*driver>
\documentclass[full,kernel]{l3doc}
\begin{document}
  \DocInput{\jobname.dtx}
\end{document}
%</driver>
% \fi
%
%
% \title{^^A
%   The \textsf{l3intarray} package: fast global integer arrays^^A
% }
%
% \author{^^A
%  The \LaTeX3 Project\thanks
%    {^^A
%      E-mail:
%        \href{mailto:latex-team@latex-project.org}
%          {latex-team@latex-project.org}^^A
%    }^^A
% }
%
% \date{Released 2018-04-30}
%
% \maketitle
%
% \begin{documentation}
%
% \section{\pkg{l3intarray} documentation}
%
% For applications requiring heavy use of integers, this module provides
% arrays which can be accessed in constant time (contrast \pkg{l3seq},
% where access time is linear). These arrays have several important
% features
% \begin{itemize}
%   \item The size of the array is fixed and must be given at
%     point of initialisation
%   \item The absolute value of each entry has maximum $2^{30}-1$
%     (\emph{i.e.}~one power lower than the usual \cs{c_max_int}
%     ceiling of $2^{31}-1$)
% \end{itemize}
% The use of \texttt{intarray} data is therefore recommended for cases where
% the need for fast access is of paramount importance.
%
% \begin{function}[added = 2018-03-29]{\intarray_new:Nn}
%   \begin{syntax}
%     \cs{intarray_new:Nn} \meta{intarray~var} \Arg{size}
%   \end{syntax}
%   Evaluates the integer expression \meta{size} and allocates an
%   \meta{integer array variable} with that number of (zero) entries.
%   The variable name should start with |\g_| because assignments are
%   always global.
% \end{function}
%
% \begin{function}[EXP, added = 2018-03-29]{\intarray_count:N}
%   \begin{syntax}
%     \cs{intarray_count:N} \meta{intarray~var}
%   \end{syntax}
%   Expands to the number of entries in the \meta{integer array variable}.
%   Contrarily to \cs{seq_count:N} this is performed in constant time.
% \end{function}
%
% \begin{function}[added = 2018-03-29]{\intarray_gset:Nnn}
%   \begin{syntax}
%     \cs{intarray_gset:Nnn} \meta{intarray~var} \Arg{position} \Arg{value}
%   \end{syntax}
%   Stores the result of evaluating the integer expression \meta{value}
%   into the \meta{integer array variable} at the (integer expression)
%   \meta{position}.  If the \meta{position} is not between $1$ and the
%   \cs{intarray_count:N}, or the \meta{value}'s absolute value is
%   bigger than $2^{30}-1$, an error occurs.  Assignments are always
%   global.
% \end{function}
%
% \begin{function}[added = 2018-05-04]{\intarray_gzero:N}
%   \begin{syntax}
%     \cs{intarray_gzero:N} \meta{intarray~var}
%   \end{syntax}
%   Sets all entries of the \meta{integer array variable} to zero.
%   Assignments are always global.
% \end{function}
%
% \begin{function}[EXP, added = 2018-03-29]{\intarray_item:Nn}
%   \begin{syntax}
%     \cs{intarray_item:Nn} \meta{intarray~var} \Arg{position}
%   \end{syntax}
%   Expands to the integer entry stored at the (integer expression)
%   \meta{position} in the \meta{integer array variable}.  If the
%   \meta{position} is not between $1$ and the \cs{intarray_count:N}, an
%   error occurs.
% \end{function}
%
% \subsection{Implementation notes}
%
% It is a wrapper around the \tn{fontdimen} primitive, used to store
% arrays of integers (with a restricted range: absolute value at most
% $2^{30}-1$).  In contrast to \pkg{l3seq} sequences the access to
% individual entries is done in constant time rather than linear time,
% but only integers can be stored.  More precisely, the primitive
% \tn{fontdimen} stores dimensions but the \pkg{l3intarray} package
% transparently converts these from/to integers.  Assignments are always
% global.
%
% While \LuaTeX{}'s memory is extensible, other engines can
% \enquote{only} deal with a bit less than $4\times 10^6$ entries in all
% \tn{fontdimen} arrays combined (with default \TeX{}Live settings).
%
% \end{documentation}
%
% \begin{implementation}
%
% \section{\pkg{l3intarray} implementation}
%
%    \begin{macrocode}
%<*initex|package>
%    \end{macrocode}
%
%    \begin{macrocode}
%<@@=intarray>
%    \end{macrocode}
%
% \subsection{Allocating arrays}
%
% \begin{macro}{\@@_entry:w, \@@_count:w}
%   We use these primitives quite a lot in this module.
%    \begin{macrocode}
\cs_new_eq:NN \@@_entry:w \tex_fontdimen:D
\cs_new_eq:NN \@@_count:w \tex_hyphenchar:D
%    \end{macrocode}
% \end{macro}
%
% \begin{variable}{\l_@@_loop_int}
%   A loop index.
%    \begin{macrocode}
\int_new:N \l_@@_loop_int
%    \end{macrocode}
% \end{variable}
%
% \begin{variable}{\c_@@_sp_dim}
%   Used to convert integers to dimensions fast.
%    \begin{macrocode}
\dim_const:Nn \c_@@_sp_dim { 1 sp }
%    \end{macrocode}
% \end{variable}
%
% \begin{variable}{\g_@@_font_int}
%   Used to assign one font per array.
%    \begin{macrocode}
\int_new:N \g_@@_font_int
%    \end{macrocode}
% \end{variable}
%
%    \begin{macrocode}
\__kernel_msg_new:nnn { kernel } { negative-array-size }
  { Size~of~array~may~not~be~negative:~#1 }
%    \end{macrocode}
%
% \begin{macro}{\intarray_new:Nn, \@@_new:N}
%   Declare |#1| to be a font (arbitrarily |cmr10| at a never-used
%   size).  Store the array's size as the \tn{hyphenchar} of that font
%   and make sure enough \tn{fontdimen} are allocated, by setting the
%   last one.  Then clear any \tn{fontdimen} that |cmr10| starts with.
%   It seems \LuaTeX{}'s |cmr10| has an extra \tn{fontdimen} parameter
%   number $8$ compared to other engines (for a math font we would
%   replace $8$ by $22$ or some such).
%   Every \texttt{intarray} must be global; it's enough to run this
%   check in \cs{intarray_new:Nn}.
%    \begin{macrocode}
\cs_new_protected:Npn \@@_new:N #1
  {
    \__kernel_chk_if_free_cs:N #1
    \int_gincr:N \g_@@_font_int
    \tex_global:D \tex_font:D #1
      = cmr10~at~ \g_@@_font_int \c_@@_sp_dim \scan_stop:
    \int_step_inline:nn { 8 }
      { \__kernel_intarray_gset:Nnn #1 {##1} \c_zero }
  }
\__kernel_patch:nnNNpn { \__kernel_chk_var_scope:NN g #1 } { }
\cs_new_protected:Npn \intarray_new:Nn #1#2
  {
    \@@_new:N #1
    \@@_count:w #1 = \int_eval:n {#2} \scan_stop:
    \int_compare:nNnT { \intarray_count:N #1 } < 0
      {
        \__kernel_msg_error:nnx { kernel } { negative-array-size }
          { \intarray_count:N #1 }
      }
    \int_compare:nNnT { \intarray_count:N #1 } > 0
      { \__kernel_intarray_gset:Nnn #1 { \intarray_count:N #1 } { 0 } }
  }
%    \end{macrocode}
% \end{macro}
%
% \begin{macro}[EXP]{\intarray_count:N}
%   Size of an array.
%    \begin{macrocode}
\cs_new:Npn \intarray_count:N #1 { \int_value:w \@@_count:w #1 }
%    \end{macrocode}
% \end{macro}
%
% \subsection{Array items}
%
% \begin{macro}[EXP]{\@@_signed_max_dim:n}
%   Used when an item to be stored is larger than \cs{c_max_dim} in
%   absolute value; it is replaced by $\pm\cs{c_max_dim}$.
%    \begin{macrocode}
\cs_new:Npn \@@_signed_max_dim:n #1
  { \int_value:w \int_compare:nNnT {#1} < 0 { - } \c_max_dim }
%    \end{macrocode}
% \end{macro}
%
% \begin{macro}[EXP]{\@@_bounds:NNnTF, \@@_bounds_error:NNn}
%   The functions \cs{intarray_gset:Nnn} and \cs{intarray_item:Nn} share
%   bounds checking.  The |T| branch is used if |#3| is within bounds of
%   the array |#2|.
%    \begin{macrocode}
\cs_new:Npn \@@_bounds:NNnTF #1#2#3#4#5
  {
    \if_int_compare:w 1 > #3 \exp_stop_f:
      \@@_bounds_error:NNn #1 #2 {#3}
      #5
    \else:
      \if_int_compare:w #3 > \intarray_count:N #2 \exp_stop_f:
        \@@_bounds_error:NNn #1 #2 {#3}
        #5
      \else:
        #4
      \fi:
    \fi:
  }
\cs_new:Npn \@@_bounds_error:NNn #1#2#3
  {
    #1 { kernel } { out-of-bounds }
      { \token_to_str:N #2 } {#3} { \intarray_count:N #2 }
  }
%    \end{macrocode}
% \end{macro}
%
% \begin{macro}{\intarray_gset:Nnn, \__kernel_intarray_gset:Nnn}
% \begin{macro}{\@@_gset:Nnn, \@@_gset_overflow:Nnn}
%   Set the appropriate \tn{fontdimen}.  The
%   \cs{__kernel_intarray_gset:Nnn} function does not use
%   \cs{int_eval:n}, namely its arguments must be suitable for
%   \cs{int_value:w}.  The user version checks the position and value
%   are within bounds.
%    \begin{macrocode}
\cs_new_protected:Npn \__kernel_intarray_gset:Nnn #1#2#3
  { \@@_entry:w #2 #1 #3 \c_@@_sp_dim }
\cs_new_protected:Npn \intarray_gset:Nnn #1#2#3
  {
    \exp_after:wN \@@_gset:Nww
    \exp_after:wN #1
    \int_value:w \int_eval:n {#2} \exp_after:wN ;
    \int_value:w \int_eval:n {#3} ;
  }
\cs_new_protected:Npn \@@_gset:Nww #1#2 ; #3 ;
  {
    \@@_bounds:NNnTF \__kernel_msg_error:nnxxx #1 {#2}
      {
        \@@_gset_overflow_test:nw {#3}
        \__kernel_intarray_gset:Nnn #1 {#2} {#3}
      }
      { }
  }
\cs_if_exist:NTF \tex_ifabsnum:D
  {
    \cs_new_protected:Npn \@@_gset_overflow_test:nw #1
      {
        \tex_ifabsnum:D #1 > \c_max_dim
          \exp_after:wN \@@_gset_overflow:NNnn
        \fi:
      }
  }
  {
    \cs_new_protected:Npn \@@_gset_overflow_test:nw #1
      {
        \if_int_compare:w \int_abs:n {#1} > \c_max_dim
          \exp_after:wN \@@_gset_overflow:NNnn
        \fi:
      }
  }
\cs_new_protected:Npn \@@_gset_overflow:NNnn #1#2#3#4
  {
    \__kernel_msg_error:nnxxxx { kernel } { overflow }
      { \token_to_str:N #2 } {#3} {#4} {  \@@_signed_max_dim:n {#4} }
    #1 #2 {#3} { \@@_signed_max_dim:n {#4} }
  }
%    \end{macrocode}
% \end{macro}
% \end{macro}
%
% \begin{macro}{\intarray_gzero:N}
%   Set the appropriate \tn{fontdimen} to zero.  No bound checking
%   needed.  The \cs{prg_replicate:nn} possibly uses quite a lot of
%   memory, but this is somewhat comparable to the size of the array,
%   and it is much faster than an \cs{int_step_inline:nn} loop.
%    \begin{macrocode}
\cs_new_protected:Npn \intarray_gzero:N #1
  {
    \int_zero:N \l_@@_loop_int
    \prg_replicate:nn { \intarray_count:N #1 }
      {
        \int_incr:N \l_@@_loop_int
        \@@_entry:w \l_@@_loop_int #1 \c_zero_dim
      }
  }
%    \end{macrocode}
% \end{macro}
%
% \begin{macro}[EXP]{\intarray_item:Nn, \__kernel_intarray_item:Nn}
% \begin{macro}{\@@_item:Nn}
%   Get the appropriate \tn{fontdimen} and perform bound checks.  The
%   \cs{__kernel_intarray_item:Nn} function omits bound checks and omits
%   \cs{int_eval:n}, namely its argument must be a \TeX{} integer
%   suitable for \cs{int_value:w}.
%    \begin{macrocode}
\cs_new:Npn \__kernel_intarray_item:Nn #1#2
  { \int_value:w \@@_entry:w #2 #1 }
\cs_new:Npn \intarray_item:Nn #1#2
  {
    \exp_after:wN \@@_item:Nw
    \exp_after:wN #1
    \int_value:w \int_eval:n {#2} ;
  }
\cs_new:Npn \@@_item:Nw #1#2 ;
  {
    \@@_bounds:NNnTF \__kernel_msg_expandable_error:nnfff #1 {#2}
      { \__kernel_intarray_item:Nn #1 {#2} }
      { 0 }
  }
%    \end{macrocode}
% \end{macro}
% \end{macro}
%
% \begin{macro}{\intarray_rand_item:N}
%   Importantly, \cs{intarray_item:Nn} only evaluates its argument once.
%    \begin{macrocode}
\cs_new:Npn \intarray_rand_item:N #1
  { \intarray_item:Nn #1 { \int_rand:n { \intarray_count:N #1 } } }
%    \end{macrocode}
% \end{macro}
%
% \subsection{Working with contents of integer arrays}
%
% At the time of writing these are candidates, but we need at least
% \cs{intarray_const_from_clist:Nn} in \pkg{l3fp} so before
% \pkg{l3candidates}.
%
% \begin{macro}{\intarray_const_from_clist:Nn, \@@_const_from_clist:nN}
%   Similar to \cs{intarray_new:Nn} (which we don't use because when
%   debugging is enabled that function checks the variable name starts
%   with |g_|).  We make use of the fact that \TeX{} allows allocation
%   of successive \tn{fontdimen} as long as no other font has been
%   declared: no need to count the comma list items first.  We need the
%   code in \cs{intarray_gset:Nnn} that checks the item value is not too
%   big, namely \cs{@@_gset_overflow_test:nw}, but not the code that
%   checks bounds.  At the end, set the size of the intarray.
%    \begin{macrocode}
\__kernel_patch:nnNNpn { \__kernel_chk_var_scope:NN c #1 } { }
\cs_new_protected:Npn \intarray_const_from_clist:Nn #1#2
  {
    \@@_new:N #1
    \int_zero:N \l_@@_loop_int
    \clist_map_inline:nn {#2}
      { \exp_args:Nf \@@_const_from_clist:nN { \int_eval:n {##1} } #1 }
    \@@_count:w #1 \l_@@_loop_int
  }
\cs_new_protected:Npn \@@_const_from_clist:nN #1#2
  {
    \int_incr:N \l_@@_loop_int
    \@@_gset_overflow_test:nw {#1}
    \__kernel_intarray_gset:Nnn #2 \l_@@_loop_int {#1}
  }
%    \end{macrocode}
% \end{macro}
%
% \begin{macro}[rEXP]{\intarray_to_clist:N, \@@_to_clist:Nn, \@@_to_clist:w}
%   Loop through the array, putting a comma before each item.  Remove
%   the leading comma with |f|-expansion.  We also use the auxiliary in
%   \cs{intarray_show:N} with argument comma, space.
%    \begin{macrocode}
\cs_new:Npn \intarray_to_clist:N #1 { \@@_to_clist:Nn #1 { , } }
\cs_new:Npn \@@_to_clist:Nn #1#2
  {
    \int_compare:nNnF { \intarray_count:N #1 } = \c_zero
      {
        \exp_last_unbraced:Nf \use_none:n
          { \@@_to_clist:w 1 ; #1 {#2} \prg_break_point: }
      }
  }
\cs_new:Npn \@@_to_clist:w #1 ; #2#3
  {
    \if_int_compare:w #1 > \@@_count:w #2
      \prg_break:n
    \fi:
    #3 \__kernel_intarray_item:Nn #2 {#1}
    \exp_after:wN \@@_to_clist:w
    \int_value:w \int_eval:w #1 + \c_one ; #2 {#3}
  }
%    \end{macrocode}
% \end{macro}
%
% \begin{macro}{\intarray_show:N, \intarray_log:N}
%   Convert the list to a comma list (with spaces after each comma)
%    \begin{macrocode}
\cs_new_protected:Npn \intarray_show:N { \@@_show:NN \msg_show:nnxxxx }
\cs_generate_variant:Nn \intarray_show:N { c }
\cs_new_protected:Npn \intarray_log:N { \@@_show:NN \msg_log:nnxxxx }
\cs_generate_variant:Nn \intarray_log:N { c }
\cs_new_protected:Npn \@@_show:NN #1#2
  {
    \__kernel_chk_defined:NT #2
      {
        #1 { LaTeX/kernel } { show-intarray }
          { \token_to_str:N #2 }
          { \intarray_count:N #2 }
          { >~ \@@_to_clist:Nn #2 { , ~ } }
          { }
      }
  }
%    \end{macrocode}
% \end{macro}
%
% \subsection{Random arrays}
%
% \begin{macro}
%   {
%     \intarray_gset_rand:Nn,
%     \intarray_gset_rand:Nnn,
%     \@@_gset_rand:Nnn,
%     \@@_gset_rand:Nff,
%     \@@_gset_rand_auxi:Nnnn,
%     \@@_gset_rand_auxii:Nnnn,
%     \@@_gset_rand_auxiii:Nnnn,
%     \@@_gset_all_same:Nn,
%   }
%   We only perform the bounds checks once.  This is done by two
%   \cs{@@_gset_overflow_test:nw}, with an appropriate empty argument to
%   avoid a spurious \enquote{at position \texttt{\#1}} part in the
%   error message.  Then calculate the number of choices: this is at
%   most $(2^{30}-1)-(-(2^{30}-1))+1=2^{31}-1$, which just barely does
%   not overflow.  For small ranges use \cs{__kernel_randint:nn},
%   otherwise use the (very much slower) floating point |randint|.
%   Finally, if there are no random numbers do not define any of the
%   auxiliaries.
%    \begin{macrocode}
\cs_new_protected:Npn \intarray_gset_rand:Nn #1
  { \intarray_gset_rand:Nnn #1 { 1 } }
\sys_if_rand_exist:TF
  {
    \cs_new_protected:Npn \intarray_gset_rand:Nnn #1#2#3
      {
        \@@_gset_rand:Nff #1
          { \int_eval:n {#2} } { \int_eval:n {#3} }
      }
    \cs_new_protected:Npn \@@_gset_rand:Nnn #1#2#3
      {
        \int_compare:nNnTF {#2} > {#3}
          {
            \__kernel_msg_expandable_error:nnnn
              { kernel } { randint-backward-range } {#2} {#3}
            \@@_gset_rand:Nnn #1 {#3} {#2}
          }
          {
            \@@_gset_overflow_test:nw {#2}
            \@@_gset_rand_auxi:Nnnn #1 { } {#2} {#3}
          }
      }
    \cs_generate_variant:Nn \@@_gset_rand:Nnn { Nff }
    \cs_new_protected:Npn \@@_gset_rand_auxi:Nnnn #1#2#3#4
      {
        \@@_gset_overflow_test:nw {#4}
        \@@_gset_rand_auxii:Nnnn #1 { } {#4} {#3}
      }
    \cs_new_protected:Npn \@@_gset_rand_auxii:Nnnn #1#2#3#4
      {
        \exp_args:NNf \@@_gset_rand_auxiii:Nnnn #1
          { \int_eval:n { #3 - #4 + 1 } } {#4} {#3}
      }
    \cs_new_protected:Npn \@@_gset_rand_auxiii:Nnnn #1#2#3#4
      {
        \exp_args:NNf \@@_gset_all_same:Nn #1
          {
            \int_compare:nNnTF {#2} > \c__kernel_randint_max_int
              { \exp_stop_f: \fp_to_int:n { randint(#3,#4) } }
              {
                \exp_stop_f:
                \int_eval:n { \__kernel_randint:nn {#3} {#2} }
              }
          }
      }
    \cs_new_protected:Npn \@@_gset_all_same:Nn #1#2
      {
        \int_zero:N \l_@@_loop_int
        \prg_replicate:nn { \intarray_count:N #1 }
          {
            \int_incr:N \l_@@_loop_int
            \__kernel_intarray_gset:Nnn #1 \l_@@_loop_int {#2}
          }
      }
  }
  {
    \cs_new_protected:Npn \intarray_gset_rand:Nnn #1#2#3
      {
        \__kernel_msg_error:nnn { kernel } { fp-no-random }
          { \intarray_gset_rand:Nnn #1 {#2} {#3} }
      }
  }
%    \end{macrocode}
% \end{macro}
%
%    \begin{macrocode}
%</initex|package>
%    \end{macrocode}
%
% \end{implementation}
%
% \PrintIndex
