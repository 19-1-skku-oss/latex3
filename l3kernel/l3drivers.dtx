% \iffalse meta-comment
%
%% File: l3drivers.dtx
%
% Copyright (C) 2011-2019 The LaTeX3 Project
%
% It may be distributed and/or modified under the conditions of the
% LaTeX Project Public License (LPPL), either version 1.3c of this
% license or (at your option) any later version.  The latest version
% of this license is in the file
%
%    https://www.latex-project.org/lppl.txt
%
% This file is part of the "l3kernel bundle" (The Work in LPPL)
% and all files in that bundle must be distributed together.
%
% -----------------------------------------------------------------------
%
% The development version of the bundle can be found at
%
%    https://github.com/latex3/latex3
%
% for those people who are interested.
%
%<*driver>
\documentclass[full,kernel]{l3doc}
\begin{document}
  \DocInput{\jobname.dtx}
\end{document}
%</driver>
% \fi
%
% \title{^^A
%   The \textsf{l3drivers} package\\ Drivers^^A
% }
%
% \author{^^A
%  The \LaTeX3 Project\thanks
%    {^^A
%      E-mail:
%        \href{mailto:latex-team@latex-project.org}
%          {latex-team@latex-project.org}^^A
%    }^^A
% }
%
% \date{Released 2019-03-05}
%
% \maketitle
%
% \begin{documentation}
%
% \TeX{} relies on drivers in order to carry out a number of tasks, such
% as using color, including graphics and setting up hyper-links. The nature
% of the code required depends on the exact driver in use. Currently,
% \LaTeX3 is aware of the following drivers:
% \begin{itemize}
%   \item \texttt{pdfmode}:  The \enquote{driver} for direct PDF output by
%     \emph{both} \pdfTeX{} and \LuaTeX{} (no separate driver is used in this
%     case: the engine deals with PDF creation itself).
%   \item \texttt{dvips}: The \texttt{dvips} program, which works in
%     conjugation with \pdfTeX{} or \LuaTeX{} in DVI mode.
%   \item \texttt{dvipdfmx}: The \texttt{dvipdfmx} program, which works in
%     conjugation with \pdfTeX{} or \LuaTeX{} in DVI mode.
%   \item \texttt{dvisvgm}:  The \texttt{dvisvgm} program, which works in
%     conjugation with \pdfTeX{} or \LuaTeX{} when run in DVI mode as well
%     as with (u)\pTeX{} and \XeTeX{}.
%   \item \texttt{xdvipdfmx}: The driver used by \XeTeX{}.
% \end{itemize}
%
% This module provides code closely tied to the exact driver in use: broadly,
% the functions here are implemented entirely independently for each case.
% As such, they often rely on higher-level code to provide necessary but
% shared operations. For example, in box rotation and scaling the functions
% here do no correct the final size of the box: this will always be required
% and thus is handled in the \pkg{box} module.
%
% Several of the operations here are low-level, and so may be used only in
% restricted contexts. Some also require understanding of PostScript/PDF
% concepts to be used corrected as they take \enquote{raw} arguments, similar
% in format to those used by the underlying driver.
%
% The functions in this module should be regarded as experimental with
% the following exceptions:
% \begin{itemize}
%   \item \dots
% \end{itemize}
%
% \section{Box clipping}
%
% \begin{function}[added = 2017-12-13]{\driver_box_use_clip:N}
%   \begin{syntax}
%     \cs{driver_box_use_clip:N} \meta{box}
%   \end{syntax}
%   Inserts the content of the \meta{box} at the current insertion point
%   such that any material outside of the bounding box is not displayed
%   by the driver. The material in the \meta{box} is still placed in the
%   output stream: the clipping takes place at a driver level.
% \end{function}
%
% \section{Box rotation and scaling}
%
% \begin{function}[added = 2017-12-13, updated = 2018-04-26]
%   {\driver_box_use_rotate:Nn}
%   \begin{syntax}
%     \cs{driver_box_use_rotate:Nn} \meta{box} \Arg{angle}
%   \end{syntax}
%   Inserts the content of the \meta{box} at the current insertion point
%   rotated by the \meta{angle} (an \meta{fp expression} expressed in degrees).
%   The material is
%   rotated such the the \TeX{} reference point of the box is the center of
%   rotation and remains the reference point after rotation. It is the
%   responsibility of the code using this function to adjust the apparent
%   size of the inserted material.
% \end{function}
%
% \begin{function}[added = 2017-12-13, updated = 2018-04-26]
%   {\driver_box_use_scale:Nnn}
%   \begin{syntax}
%     \cs{driver_box_use_scale:Nnn} \meta{box} \Arg{x-scale} \Arg{y-scale}
%   \end{syntax}
%   Inserts the content of the \meta{box} at the current insertion point
%   scale by the \meta{x-scale} and \meta{y-scale} (both \meta{fp expressions}).
%   The reference point
%   of the material will be unchanged. It is the responsibility of the
%   code using this function to adjust the apparent size of the inserted
%   material.
% \end{function}
%
% \section{Color support}
%
% \begin{function}[added = 2018-02-20]{\driver_color_cmyk:nnnn}
%   \begin{syntax}
%     \cs{driver_color_cmyk:nnnn} \Arg{cyan} \Arg{magenta} \Arg{yellow}
%       \Arg{black}
%   \end{syntax}
%   Sets the color to the CMYK values specified, all of which are
%   fp denotations in the range $0$ and $1$. For drawing colors, see
%   \cs{driver_draw_stroke_cmyk:nnnn}, \emph{etc.}
% \end{function}
%
% \begin{function}[added = 2018-02-20]{\driver_color_gray:n}
%   \begin{syntax}
%     \cs{driver_color_gray:n} \Arg{gray}
%   \end{syntax}
%   Sets the color to the grayscale value specified, which is
%   fp denotations in the range $0$ and $1$. For drawing colors, see
%   \cs{driver_draw_stroke_gray:n}, \emph{etc.}
% \end{function}
%
% \begin{function}[added = 2018-02-20]{\driver_color_rgb:nnn}
%   \begin{syntax}
%     \cs{driver_color_rgb:nnn} \Arg{red} \Arg{green} \Arg{blue}
%   \end{syntax}
%   Sets the color to the RGB values specified, all of which are
%   fp denotations in the range $0$ and $1$. For drawing colors, see
%   \cs{driver_draw_stroke_rgb:nnn}, \emph{etc.}
% \end{function}
%
% \begin{function}[added = 2018-02-20]{\driver_color_pickup:N}
%   \begin{syntax}
%     \cs{driver_color_pickup:N} \meta{tl}
%   \end{syntax}
%   In \LaTeXe{} package mode, collects data on the current color from
%   \tn{current@color} and stores it in the low-level format used by \pkg{expl3}
%   in the \meta{tl}.
% \end{function}
%
% \section{Drawing}
%
% The drawing functions provided here are \emph{highly} experimental. They
% are inspired heavily by the system layer of \pkg{pgf} (most have the
% same interface as the same functions in the latter's \cs{pgfsys@\ldots}
% namespace). They are intended to form the basis for higher level drawing
% interfaces, which themselves are likely to be further abstracted for user
% access. Again, this model is heavily inspired by \pkg{pgf} and
% Ti\textit{k}z.
%
% These low level drawing interfaces abstract from the driver raw requirements
% but still require an appreciation of the concepts of PostScript/PDF/SVG
% graphic creation.
%
% \begin{function}
%   {\driver_draw_begin:, \driver_draw_end:}
%   \begin{syntax}
%     \cs{driver_draw_begin:}
%     \meta{content}
%     \cs{driver_draw_end:}
%   \end{syntax}
%   Defines a drawing environment. This is a scope for the purposes of
%   the graphics state. Depending on the driver, other set up may or may not
%   take place here. The natural size of the \meta{content} should be zero
%   from the \TeX{} perspective: allowance for the size of the content must
%   be made at a higher level (or indeed this can be skipped if the content is
%   to overlap other material).
% \end{function}
%
% \begin{function}
%   {\driver_draw_scope_begin:, \driver_draw_scope_end:}
%   \begin{syntax}
%     \cs{driver_draw_scope_begin:}
%     \meta{content}
%     \cs{driver_draw_scope_end:}
%   \end{syntax}
%   Defines a scope for drawing settings and so on. Changes to the graphic
%   state and concepts such as color or linewidth are localised to a scope.
%   This function pair must never be used if an partial path is under
%   construction: such paths must be entirely contained at one unbroken
%   scope level. Note that scopes do not form \TeX{} groups and may not
%   be aligned with them.
% \end{function}
%
% \subsection{Path construction}
%
% \begin{function}{\driver_draw_moveto:nn}
%   \begin{syntax}
%     \cs{driver_draw_move:nn} \Arg{x} \Arg{y}
%   \end{syntax}
%   Moves the current drawing reference point to (\meta{x}, \meta{y});
%   any active transformation matrix applies.
% \end{function}
%
% \begin{function}{\driver_draw_lineto:nn}
%   \begin{syntax}
%     \cs{driver_draw_lineto:nn} \Arg{x} \Arg{y}
%   \end{syntax}
%   Adds a path from the current drawing reference point to
%   (\meta{x}, \meta{y}); any active transformation matrix applies. Note
%   that nothing is drawn until a fill or stroke operation is applied, and that
%   the path may be discarded or used as a clip without appearing itself.
% \end{function}
%
% \begin{function}{\driver_draw_curveto:nnnnnn}
%   \begin{syntax}
%     \cs{driver_draw_curveto:nnnnnn} \Arg{x_1} \Arg{y_1}
%       \Arg{x_2} \Arg{y_2} \Arg{x_3} \Arg{y_3}
%   \end{syntax}
%   Adds a Bezier curve path from the current drawing reference point to
%   (\meta{x_3}, \meta{y_3}), using (\meta{x_1}, \meta{y_1}) and
%   (\meta{x_2}, \meta{y_2}) as control points; any active transformation
%   matrix applies.  Note that nothing is drawn until a fill or stroke
%   operation is applied, and that the path may be discarded or used as a clip
%   without appearing itself.
% \end{function}
%
% \begin{function}{\driver_draw_rectangle:nnnn}
%   \begin{syntax}
%     \cs{driver_draw_rectangle:nnnn} \Arg{x} \Arg{y} \Arg{width} \Arg{height}
%   \end{syntax}
%   Adds rectangular path from (\meta{x_1}, \meta{y_1}) of \meta{height}
%   and \meta{width}; any active transformation matrix applies.  Note that
%   nothing is drawn until a fill or stroke operation is applied, and that the
%   path may be discarded or used as a clip without appearing itself.
% \end{function}
%
% \begin{function}{\driver_draw_closepath:}
%   \begin{syntax}
%     \cs{driver_draw_closepath:}
%   \end{syntax}
%   Closes an existing path, adding a line from the current point to the
%   start of path. Note that nothing is drawn until a fill or stroke
%   operation is applied, and that the path may be discarded or used as a clip
%   without appearing itself.
% \end{function}
%
% \subsection{Stroking and filling}
%
% \begin{function}{\driver_draw_stroke:, \driver_draw_closestroke:}
%   \begin{syntax}
%     \meta{path construction}
%     \cs{driver_draw_stroke:}
%   \end{syntax}
%   Draws a line along the current path, which is also closed in the case of
%   \cs{driver_draw_closestroke:}. The nature of the line drawn
%   is influenced by settings for
%   \begin{itemize}
%     \item Line thickness
%     \item Stroke color (or the current color if no specific stroke color
%       is set)
%     \item Line capping (how non-closed line ends should look)
%     \item Join style (how a bend in the path should be rendered)
%     \item Dash pattern
%   \end{itemize}
%   The path may also be used for clipping.
% \end{function}
%
% \begin{function}{\driver_draw_fill:, \driver_draw_fillstroke:}
%   \begin{syntax}
%     \meta{path construction}
%     \cs{driver_draw_fill:}
%   \end{syntax}
%   Fills the area surrounded by the current path: this will be closed prior
%   to filling if it is not already. The \texttt{fillstroke} version also
%   strokes the path as described for \cs{driver_draw_stroke:}. The fill is
%   influenced by the setting for fill color (or the current color if no
%   specific stroke color is set). The path may also be used for clipping.
%   For paths which are self-intersecting or comprising multiple parts, the
%   determination of which areas are inside the path is made using the non-zero
%   winding number rule unless the even-odd rule is active.
% \end{function}
%
% \begin{function}{\driver_draw_nonzero_rule:, \driver_draw_evenodd_rule:}
%   \begin{syntax}
%     \cs{driver_draw_nonzero_rule:}
%   \end{syntax}
%   Active either the non-zero winding number or the even-odd rule,
%   respectively, for determining what is inside a fill or clip area.
%   For technical reasons, these command are not influenced by scoping
%   and apply on an ongoing basis.
% \end{function}
%
% \begin{function}{\driver_draw_clip:}
%   \begin{syntax}
%     \meta{path construction}
%     \cs{driver_draw_clip:}
%   \end{syntax}
%   Indicates that the current path should be used for clipping, such that
%   any subsequent material outside of the path (but within the current
%   scope) will not be shown. This command should be given once a path is
%   complete but before it is stroked or filled (if appropriate). This
%   command is \emph{not} affected by scoping: it applies to exactly one
%   path as shown.
% \end{function}
%
% \begin{function}{\driver_draw_discardpath:}
%   \begin{syntax}
%     \meta{path construction}
%     \cs{driver_draw_discardpath:}
%   \end{syntax}
%   Discards the current path without stroking or filling. This is primarily
%   useful for paths constructed purely for clipping, as this alone does not
%   end the paths existence.
% \end{function}
%
% \subsection{Stroke options}
%
% \begin{function}{\driver_draw_linewidth:n}
%   \begin{syntax}
%     \cs{driver_draw_linewidth:n} \Arg{dimexpr}
%   \end{syntax}
%   Sets the width to be used for stroking to \meta{dimexpr}.
% \end{function}
%
% \begin{function}{\driver_draw_dash_pattern:nn}
%   \begin{syntax}
%     \cs{driver_draw_dash:nn} \Arg{dash pattern} \Arg{phase}
%   \end{syntax}
%   Sets the pattern of dashing to be used when stroking a line. The
%   \meta{dash pattern} should be a comma-separated list of dimension
%   expressions. This is then interpreted as a series of pairs of line-on
%   and line-off lengths. For example \texttt{3pt, 4pt} means that $3$\,pt on,
%   $4$\,pt off, $3$\,pt on, and so on. A more complex pattern will also
%   repeat: \texttt{3pt, 4pt, 1pt, 2pt} results in $3$\,pt on, $4$\,pt off,
%   $1$\,pt on, $2$\,pt off, $3$\,pt on, and so on. An odd number of entries
%   means that the last is repeated, for example \texttt{3pt} is equal to
%   \texttt{3pt, 3pt}. An empty pattern yields a solid line.
%
%   The \meta{phase} specifies an offset at the start of the cycle. For
%   example, with a pattern \texttt{3pt} a phase of \texttt{1pt} means
%   that the output is $2$\,pt on, $3$\,pt off, $3$\,pt on, $3$\,pt on,
%   \emph{etc.}
% \end{function}
%
% \begin{function}
%   {
%     \driver_draw_cap_butt:      ,
%     \driver_draw_cap_rectangle: ,
%     \driver_draw_cap_round:
%   }
%   \begin{syntax}
%     \cs{driver_draw_cap_butt:}
%   \end{syntax}
%   Sets the style of terminal stroke position to one of butt, rectangle or
%   round.
% \end{function}
%
% \begin{function}
%   {
%     \driver_draw_join_bevel: ,
%     \driver_draw_join_miter: ,
%     \driver_draw_join_round:
%   }
%   \begin{syntax}
%     \cs{driver_draw_cap_butt:}
%   \end{syntax}
%   Sets the style of stroke joins to one of bevel, miter or round.
% \end{function}
%
% \begin{function}{\driver_draw_miterlimit:n}
%   \begin{syntax}
%     \cs{driver_draw_miterlimit:n} \Arg{factor}
%   \end{syntax}
%   Sets the miter limit of lines joined as a miter, as described in the
%   PDF and PostScript manuals. The \meta{factor} here is an
%   \meta{fp expression}.
% \end{function}
%
% \subsection{Color}
%
% \begin{function}
%   {
%     \driver_draw_color_fill_cmyk:nnnn  ,
%     \driver_draw_color_stroke_cmyk:nnnn
%   }
%   \begin{syntax}
%     \cs{driver_draw_color_fill_cmyk:nnnn} \Arg{cyan} \Arg{magenta} \Arg{yellow}
%       \Arg{black}
%   \end{syntax}
%   Sets the color for drawing to the CMYK values specified, all of which are
%   fp denotations in the range $0$ and $1$.
% \end{function}
%
% \begin{function}
%   {
%     \driver_draw_color_fill_gray:n  ,
%     \driver_draw_color_stroke_gray:n
%   }
%   \begin{syntax}
%     \cs{driver_draw_color_fill_gray:n} \Arg{gray}
%   \end{syntax}
%   Sets the color for drawing to the grayscale value specified, which is
%   fp denotations in the range $0$ and $1$.
% \end{function}
%
% \begin{function}
%   {
%     \driver_draw_color_fill_rgb:nnn  ,
%     \driver_draw_color_stroke_rgb:nnn
%   }
%   \begin{syntax}
%     \cs{driver_draw_color_fill_rgb:nnn} \Arg{red} \Arg{green} \Arg{blue}
%   \end{syntax}
%   Sets the color for drawing to the RGB values specified, all of which are
%   fp denotations in the range $0$ and $1$.
% \end{function}
%
% \subsection{Inserting \TeX{} material}
%
% \begin{function}{\driver_draw_box_use:Nnnnn}
%   \begin{syntax}
%     \cs{driver_draw_box:Nnnnnnn} \meta{box}
%       \Arg{a} \Arg{b} \Arg{c} \Arg{d} \Arg{x} \Arg{y}
%   \end{syntax}
%   Inserts the \meta{box} as an hbox with the box reference point placed
%   at ($x$, $y$). The transformation matrix $[a b c d]$ is applied
%   to the box, allowing it to be in synchronisation with any scaling, rotation
%   or skewing applying more generally. Note that \TeX{} material should not
%   be inserted directly into a drawing as it would not be in the correct
%   location. Also note that as for other drawing elements the box here
%   has no size from a \TeX{} perspective.
% \end{function}
%
% \subsection{Coordinate system transformations}
%
% \begin{function}{\driver_draw_cm:nnnn}
%   \begin{syntax}
%     \cs{driver_draw_cm:nnnn} \Arg{a} \Arg{b} \Arg{c} \Arg{d}
%   \end{syntax}
%   Applies the transformation matrix $[a b c d]$ to the current graphic state.
%   This affects any subsequent items in the same scope but not those already
%   given.
% \end{function}
%
% \section{PDF Features}
%
% A range of PDF features are exposed by \pdfTeX{} and \LuaTeX{} in direct PDF
% output mode, and the vast majority of these are also controllable using
% the \texttt{(x)dvipdfmx} driver (as DVI instructions are converted directly
% to PDF). Some of these functions are also available for cases where PDFs
% are generated by \texttt{dvips}: this depends on being able to pass
% information through correctly.
%
% \subsection{PDF Objects}
%
% Objects are used to provide a range of data structures in a PDF. At the
% driver level, different PDF object types are declared separately. Objects
% are only \emph{written} to the PDF when referenced.
%
% \begin{function}{\driver_pdf_object_new:nn}
%   \begin{syntax}
%     \cs{driver_pdf_object_new:n} \Arg{name} \Arg{type}
%   \end{syntax}
%   Declares \meta{name} as a PDF object. The \texttt{type} should be one of
%   |array| or |dict|, |fstream| or |stream|.
% \end{function}
%
% \begin{function}[EXP]{\driver_pdf_object_ref:n}
%   \begin{syntax}
%     \cs{driver_pdf_object_ref:n} \Arg{object}
%   \end{syntax}
%   Inserts the appropriate information to reference the \meta{object}
%   in for example page resource allocation.
% \end{function}
%
% \begin{function}{\driver_pdf_object_write:nn}
%   \begin{syntax}
%     \cs{driver_pdf_object_write:nn} \Arg{name} \Arg{data}
%   \end{syntax}
%   Writes the \meta{data} as content of the \meta{object}. Depending on the
%   \meta{type} declared for the object, the format required for the
%   \meta{data} will vary
%   \begin{itemize}
%     \item[\texttt{array}] A space-separated list of values
%     \item[\texttt{dict}] Key--value pairs in the form
%       \texttt{/\meta{key} \meta{value}}
%     \item[\texttt{fstream}] Two brace groups: \meta{content} and
%       \meta{file name}
%     \item[\texttt{stream}] Two brace groups: \meta{content} and
%       \meta{additional attributes}
%   \end{itemize}
% \end{function}
%
% \subsection{PDF structure}
%
% \begin{function}{\driver_pdf_compresslevel:n}
%   \begin{syntax}
%     \cs{driver_pdf_compresslevel:n} \Arg{level}
%   \end{syntax}
%   Sets the degree of compression used for PDF files: the \meta{level} should
%   be in the range $0$ to $9$ (higher is more compression). Typically, either
%   compression is disables ($0$) or maximised ($9$). When used with
%   \texttt{(x)dvipdfmx}, this setting may only be applied globally: it should
%   be set only once.
% \end{function}
%
% \begin{function}{\driver_pdf_objects_enable:, \driver_pdf_objects_disable:}
%   \begin{syntax}
%     \cs{driver_pdf_objects_disable:}
%   \end{syntax}
%   Enables or disables the creation of PDF objects. These objects are used to
%   reduce the size of PDFs, and typically are enabled as standard. When used
%   with \texttt{(x)dvipdfmx}, object creation can be disabled but not
%   re-enabled, and this setting may only be applied globally: it should
%   be set only once.
% \end{function}
%
% \end{documentation}
%
% \begin{implementation}
%
% \section{\pkg{l3drivers} Implementation}
%
%    \begin{macrocode}
%<*initex|package>
%<@@=driver>
%    \end{macrocode}
%
% Whilst there is a reasonable amount of code overlap between drivers,
% it is much clearer to have the blocks more-or-less separated than run
% in together and DocStripped out in parts. As such, most of the following
% is set up on a per-driver basis, though there is some common code (again
% given in blocks not interspersed with other material).
%
% All the file identifiers are up-front so that they come out in the right
% place in the files.
%    \begin{macrocode}
%<*package>
\ProvidesExplFile
%<*dvipdfmx>
  {l3dvipdfmx.def}{2019-03-05}{}
  {L3 Experimental driver: dvipdfmx}
%</dvipdfmx>
%<*dvips>
  {l3dvips.def}{2019-03-05}{}
  {L3 Experimental driver: dvips}
%</dvips>
%<*dvisvgm>
  {l3dvisvgm.def}{2019-03-05}{}
  {L3 Experimental driver: dvisvgm}
%</dvisvgm>
%<*pdfmode>
  {l3pdfmode.def}{2019-03-05}{}
  {L3 Experimental driver: PDF mode}
%</pdfmode>
%<*xdvipdfmx>
  {l3xdvipdfmx.def}{2019-03-05}{}
  {L3 Experimental driver: xdvipdfmx}
%</xdvipdfmx>
%</package>
%    \end{macrocode}
%
% The order of the driver code here is such that we get somewhat logical
% outcomes in terms of code sharing whilst keeping things readable. (Trying to
% mix all of the code by concept is almost unmanageable.) The key parts which
% are shared are
% \begin{itemize}
%   \item Color support is either \texttt{dvips}-like or \texttt{pdfmode}-like.
%   \item \texttt{pdfmode} and \texttt{(x)dvipdfmx} share drawing routines.
%   \item \texttt{xdvipdfmx} is largely the same as \texttt{dvipdfmx} so
%     takes most of the same code.
% \end{itemize}
%
% \begin{macro}{\@@_literal:e, \@@_literal:n, \@@_literal:x}
%  The one shared function for all drivers is access to the basic
%  \tn{special} primitive: it has slightly odd expansion behaviour
%  so a wrapper is provided.
%    \begin{macrocode}
\cs_new_eq:NN \@@_literal:e \tex_special:D
\cs_new_protected:Npn \@@_literal:n #1
  { \@@_literal:e { \exp_not:n {#1} } }
\cs_generate_variant:Nn \@@_literal:n { x }
%    \end{macrocode}
% \end{macro}
%
% \subsection{Color support}
%
% Color support is split into two parts: a \enquote{general} concept and
% one directly linked to drawings (or rather the split between filling
% and stroking). General color is relatively easy to handle: we have a color
% stack available with all modern drivers, and can use that.%
% Whilst \texttt{(x)dvipdfmx} does have its own approach to color specials,
% it is easier to use \texttt{dvips}-like ones for all cases except direct
% PDF output.
%
% \subsubsection{\texttt{dvips}-style}
%
%    \begin{macrocode}
%<*dvisvgm|dvipdfmx|dvips|xdvipdfmx>
%    \end{macrocode}
%
% \begin{macro}{\driver_color_pickup:N}
% \begin{macro}{\@@_color_pickup:w}
%   Allow for \LaTeXe{} color. Here, the possible input values are limited:
%   \texttt{dvips}-style colors can mainly be taken as-is with the exception
%   spot ones (here we need a model and a tint).
%    \begin{macrocode}
%<*package>
\cs_new_protected:Npn \driver_color_pickup:N #1 { }
\AtBeginDocument
  {
    \@ifpackageloaded { color }
      {
        \cs_set_protected:Npn \driver_color_pickup:N #1
          {
            \exp_args:NV \tl_if_head_is_space:nTF \current@color
              {
                \tl_set:Nx #1
                   {
                     spot ~
                     \exp_after:wN \use:n \current@color \c_space_tl 1
                   }
              }
              {
                \exp_last_unbraced:Nx \@@_color_pickup:w
                  { \current@color } \q_stop #1
              }
          }
        \cs_new_protected:Npn \@@_color_pickup:w #1 ~ #2 \q_stop #3
          { \tl_set:Nn #3 { #1 ~ #2 } }
      }
      { }
  }
%</package>
%    \end{macrocode}
% \end{macro}
% \end{macro}
%
% \begin{macro}{\driver_color_cmyk:nnnn}
% \begin{macro}{\driver_color_gray:n}
% \begin{macro}{\driver_color_rgb:nnn}
% \begin{macro}{\driver_color_spot:nn}
% \begin{macro}{\@@_color_select:n, \@@_color_select:x}
% \begin{macro}{\@@_color_reset:}
%    Push the data to the stack. In the case of \texttt{dvips} also reset the
%    drawing fill color in raw PostScript.
%    \begin{macrocode}
\cs_new_protected:Npn \driver_color_cmyk:nnnn #1#2#3#4
  {
    \@@_color_select:x
      {
        cmyk~
        \fp_eval:n {#1} ~ \fp_eval:n {#2} ~
        \fp_eval:n {#3} ~ \fp_eval:n {#4}
      }
  }
\cs_new_protected:Npn \driver_color_gray:n #1
  { \@@_color_select:x { gray~ \fp_eval:n {#1} } }
\cs_new_protected:Npn \driver_color_rgb:nnn #1#2#3
  {
    \@@_color_select:x
      { rgb~ \fp_eval:n {#1} ~ \fp_eval:n {#2} ~ \fp_eval:n {#3} }
  }
\cs_new_protected:Npn \driver_color_spot:nn #1#2
  { \@@_color_select:n { #1 } }
\cs_new_protected:Npn \@@_color_select:n #1
  {
    \@@_literal:n { color~push~ #1 }
%<*dvips>
    \@@_literal_postscript:n { /l3fc~{ }~def }
%</dvips>
    \group_insert_after:N \@@_color_reset:
  }
\cs_generate_variant:Nn \@@_color_select:n { x }
\cs_new_protected:Npn \@@_color_reset:
  { \@@_literal:n { color~pop } }
%    \end{macrocode}
% \end{macro}
% \end{macro}
% \end{macro}
% \end{macro}
% \end{macro}
% \end{macro}
%
%    \begin{macrocode}
%</dvisvgm|dvipdfmx|dvips|xdvipdfmx>
%    \end{macrocode}
%
% \subsubsection{\texttt{pdfmode}}
%
%    \begin{macrocode}
%<*pdfmode>
%    \end{macrocode}
%
% \begin{macro}{\driver_color_pickup:N}
% \begin{macro}{\@@_color_pickup:w}
%   The current color in driver-dependent format: pick up the package-mode
%   data if available. We end up converting back and forward in this route as
%   we store our color data in \texttt{dvips} format.
%   The \tn{current@color} needs to be \texttt{x}-expanded before
%   \cs{@@_color_pickup:w} breaks it apart, because for instance
%   \pkg{xcolor} sets it to be instructions to generate a colour
%    \begin{macrocode}
%<*package>
\cs_new_protected:Npn \driver_color_pickup:N #1 { }
\AtBeginDocument
  {
    \@ifpackageloaded { color }
      {
        \cs_set_protected:Npn \driver_color_pickup:N #1
          {
            \exp_last_unbraced:Nx \@@_color_pickup:w
              { \current@color } ~ 0 ~ 0 ~ 0 \q_stop #1
          }
        \cs_new_protected:Npn \@@_color_pickup:w
          #1 ~ #2 ~ #3 ~ #4 ~ #5 ~ #6 \q_stop #7
          {
            \str_if_eq:nnTF {#2} { g }
              { \tl_set:Nn #7 { gray ~ #1 } }
              {
                \str_if_eq:nnTF {#4} { rg }
                  { \tl_set:Nn #7 { rgb ~ #1 ~ #2 ~ #3 } }
                  {
                     \str_if_eq:nnTF {#5} { k }
                       { \tl_set:Nn #7 { cmyk ~ #1 ~ #2 ~ #3 ~ #4 } }
                       {
                         \str_if_eq:nnTF {#2} { cs }
                           {
                             \tl_set:Nx #7 { spot ~ \use_none:n #1 ~ #5 }
                           }
                           {
                             \tl_set:Nn #7 { gray ~ 0 }
                           }
                       }
                  }
              }
          }
      }
      { }
  }
%</package>
%    \end{macrocode}
% \end{macro}
% \end{macro}
%
% \begin{variable}{\l_@@_color_stack_int}
%   \pdfTeX{} and \LuaTeX{} have multiple stacks available, and to track
%   which one is in use a variable is required.
%    \begin{macrocode}
\int_new:N \l_@@_color_stack_int
%    \end{macrocode}
% \end{variable}
%
% \begin{macro}{\driver_color_cmyk:nnnn, \@@_color_cmyk:nnnn}
% \begin{macro}{\driver_color_gray:n, \@@_color_gray:n}
% \begin{macro}{\driver_color_rgb:nnn, \@@_color_rgb:nnn}
% \begin{macro}{\driver_color_spot:nn}
% \begin{macro}{\@@_color_select:n, \@@_color_select:x}
% \begin{macro}{\@@_color_reset:}
%   Simply dump the data, but allowing for \LuaTeX{}.
%    \begin{macrocode}
\cs_new_protected:Npn \driver_color_cmyk:nnnn #1#2#3#4
  {
     \use:x
       {
         \@@_color_cmyk:nnnn
           { \fp_eval:n {#1} }
           { \fp_eval:n {#2} }
           { \fp_eval:n {#3} }
           { \fp_eval:n {#4} }
       } 
  }
\cs_new_protected:Npn \@@_color_cmyk:nnnn #1#2#3#4
  {
    \@@_color_select:n
      { #1 ~ #2 ~ #3 ~ #4 ~ k ~ #1 ~ #2 ~ #3 ~ #4 ~ K }
  }
\cs_new_protected:Npn \driver_color_gray:n #1
  { \exp_args:Nx \@@_color_gray:n { \fp_eval:n {#1} } }
\cs_new_protected:Npn \@@_color_gray:n #1
  { \@@_color_select:n { #1 ~ g ~ #1 ~ G } }
\cs_new_protected:Npn \driver_color_rgb:nnn #1#2#3
  {
     \use:x
       {
         \@@_color_rgb:nnn
           { \fp_eval:n {#1} }
           { \fp_eval:n {#2} }
           { \fp_eval:n {#3} }
       } 
  }
\cs_new_protected:Npn \@@_color_rgb:nnn #1#2#3
  { \@@_color_select:n { #1 ~ #2 ~ #3 ~ rg ~ #1 ~ #2 ~ #3 ~ RG } }
\cs_new_protected:Npn \driver_color_spot:nn #1#2
  { \@@_color_select:n { /#1 ~ cs ~ /#1 ~ CS ~ #2 ~ sc ~ #2 ~ SC } }
\cs_new_protected:Npx \@@_color_select:n #1
  {
    \cs_if_exist:NTF \tex_pdfextension:D
      { \tex_pdfextension:D colorstack }
      { \tex_pdfcolorstack:D }
        \exp_not:N \l_@@_color_stack_int push {#1}
      \group_insert_after:N \exp_not:N \@@_color_reset:
  }
\cs_generate_variant:Nn \@@_color_select:n { x }
\cs_new_protected:Npx \@@_color_reset:
  {
    \cs_if_exist:NTF \tex_pdfextension:D
      { \tex_pdfextension:D colorstack }
      { \tex_pdfcolorstack:D }
        \exp_not:N \l_@@_color_stack_int pop \scan_stop:
  }
%    \end{macrocode}
% \end{macro}
% \end{macro}
% \end{macro}
% \end{macro}
% \end{macro}
% \end{macro}
%
%    \begin{macrocode}
%</pdfmode>
%    \end{macrocode}
%
% \subsection{\texttt{dvips} driver}
%
%    \begin{macrocode}
%<*dvips>
%    \end{macrocode}
%
% \subsubsection{Basics}
%
% \begin{macro}{\@@_literal_postscript:n, \@@_literal_postscript:x}
%   Literal PostScript can be included using a few low-level formats. Here,
%   we use the form with no positioning: this is overall more convenient as
%   a wrapper. Note that this does require that where position is important,
%   an appropriate wrapper is included.
%    \begin{macrocode}
\cs_new_protected:Npn \@@_literal_postscript:n #1
  { \@@_literal:n { ps:: #1 } }
\cs_generate_variant:Nn \@@_literal_postscript:n { x }
%    \end{macrocode}
% \end{macro}
%
% \begin{macro}{\@@_align_currentpoint_begin:, \@@_align_currentpoint_end:}
%   In \texttt{dvips} there is no build-in saving of the current
%   position, and so some additional PostScript is required to set up the
%   transformation matrix and also to restore it afterwards. Notice the use
%   of the stack to save the current position \enquote{up front} and to
%   move back to it at the end of the process. Notice that the |[begin]|/^^A
%   |[end]| pair here mean that we can use a run of PostScript statements
%   in separate lines: not \emph{required} but does make the code and
%   output more clear.
%    \begin{macrocode}
\cs_new_protected:Npn \@@_align_currentpoint_begin:
  {
    \@@_literal:n { ps::[begin] }
    \@@_literal_postscript:n { currentpoint }
    \@@_literal_postscript:n { currentpoint~translate }
  }
\cs_new_protected:Npn \@@_align_currentpoint_end:
  {
    \@@_literal_postscript:n { neg~exch~neg~exch~translate }
    \@@_literal:n { ps::[end] }
  }
%    \end{macrocode}
% \end{macro}
%
% \begin{macro}{\@@_scope_begin:, \@@_scope_end:}
%   Saving/restoring scope for general operations needs to be done with
%   \texttt{dvips} positioning (try without to see this!). Thus we need the
%   |ps:| version of the special here. As only the graphics state is ever
%   altered within this pairing, we use the lower-cost |g|-versions.
%    \begin{macrocode}
\cs_new_protected:Npn \@@_scope_begin:
  { \@@_literal:n { ps:gsave } }
\cs_new_protected:Npn \@@_scope_end:
  { \@@_literal:n { ps:grestore } }
%    \end{macrocode}
% \end{macro}
%
% \subsubsection{Box operations}
%
% \begin{macro}{\driver_box_use_clip:N}
%   The \texttt{dvips} driver scales all absolute dimensions based on the
%   output resolution selected and any \TeX{} magnification. Thus for any
%   operation involving absolute lengths there is a correction to make. See
%   \texttt{normalscale} from \texttt{special.pro} for the variables, noting
%   that here everything is saved on the stack rather than as a separate
%   variable. Once all of that is done, the actual clipping is trivial.
%    \begin{macrocode}
\cs_new_protected:Npn \driver_box_use_clip:N #1
  {
    \@@_scope_begin:
    \@@_align_currentpoint_begin:
    \@@_literal_postscript:n { matrix~currentmatrix }
    \@@_literal_postscript:n
      { Resolution~72~div~VResolution~72~div~scale }
    \@@_literal_postscript:n { DVImag~dup~scale }
    \@@_literal_postscript:x
      {
        0 ~
        \dim_to_decimal_in_bp:n { \box_dp:N #1 } ~
        \dim_to_decimal_in_bp:n { \box_wd:N #1 } ~
        \dim_to_decimal_in_bp:n { -\box_ht:N #1 - \box_dp:N #1 } ~
        rectclip
      }
    \@@_literal_postscript:n { setmatrix }
    \@@_align_currentpoint_end:
    \hbox_overlap_right:n { \box_use:N #1 }
    \@@_scope_end:
    \skip_horizontal:n { \box_wd:N #1 }
  }
%    \end{macrocode}
% \end{macro}
%
% \begin{macro}{\driver_box_use_rotate:Nn}
% \begin{macro}{\@@_box_use_rotate:Nn}
%   Rotating using \texttt{dvips} does not require that the box dimensions
%   are altered and has a very convenient built-in operation. Zero rotation
%   must be written as |0| not |-0| so there is a quick test.
%    \begin{macrocode}
\cs_new_protected:Npn \driver_box_use_rotate:Nn #1#2
  { \exp_args:NNf \@@_box_use_rotate:Nn #1 { \fp_eval:n {#2} } }
\cs_new_protected:Npn \@@_box_use_rotate:Nn #1#2
  {
    \@@_scope_begin:
    \@@_align_currentpoint_begin:
    \@@_literal_postscript:x
      {
        \fp_compare:nNnTF {#2} = \c_zero_fp
          { 0 }
          { \fp_eval:n { round ( -(#2) , 5 ) } } ~
        rotate
      }
   \@@_align_currentpoint_end:
   \box_use:N #1
   \@@_scope_end:
  }
%    \end{macrocode}
% \end{macro}
% \end{macro}
%
% \begin{macro}{\driver_box_use_scale:Nnn}
%   The \texttt{dvips} driver once again has a dedicated operation we can
%   use here.
%    \begin{macrocode}
\cs_new_protected:Npn \driver_box_use_scale:Nnn #1#2#3
  {
    \@@_scope_begin:
    \@@_align_currentpoint_begin:
    \@@_literal_postscript:x
      {
        \fp_eval:n { round ( #2 , 5 ) } ~
        \fp_eval:n { round ( #3 , 5 ) } ~
        scale
      }
    \@@_align_currentpoint_end:
    \hbox_overlap_right:n { \box_use:N #1 }
    \@@_scope_end:
  }
%    \end{macrocode}
% \end{macro}
%
% \subsubsection{Images}
%
% \begin{macro}{\@@_image_getbb_eps:n}
%   Simply use the generic function.
%    \begin{macrocode}
\cs_new_eq:NN \@@_image_getbb_eps:n \image_read_bb:n
%    \end{macrocode}
% \end{macro}
%
% \begin{macro}{\@@_image_include_eps:n}
%  The special syntax is relatively clear here: remember we need PostScript
%  sizes here.
%    \begin{macrocode}
\cs_new_protected:Npn \@@_image_include_eps:n #1
  { \@@_literal:n { PSfile = #1 } }
%    \end{macrocode}
% \end{macro}
%
% \subsubsection{Drawing}
%
% \begin{macro}{\@@_draw_literal:n, \@@_draw_literal:x}
%   The same as literal PostScript: same arguments about positioning apply
%   her.
%    \begin{macrocode}
\cs_new_eq:NN \@@_draw_literal:n \@@_literal_postscript:n
\cs_generate_variant:Nn \@@_draw_literal:n { x }
%    \end{macrocode}
% \end{macro}
%
% \begin{macro}{\driver_draw_begin:, \driver_draw_end:}
%   The |ps::[begin]| special here deals with positioning but allows us to
%   continue on to a matching |ps::[end]|: contrast with |ps:|, which positions
%   but where we can't split material between separate calls. The
%   |@beginspecial|/|@endspecial| pair are from |special.pro| and correct the
%   scale and $y$-axis direction. The definition of |/l3fc| deals with fill
%   color in paths. In contrast to \pkg{pgf}, we don't save the current point:
%   discussion with Tom Rokici suggested a better way to handle the necessary
%   translations (see \cs{driver_draw_box_use:Nnnnn}). (Note that
%   |@beginspecial|/|@endspecial| forms a driver scope.) The  |[begin]|/^^A
%   |[end]| lines are handled differently from the rest as they are
%   conceptually different: not really drawing literals but instructions to
%   \texttt{dvips} itself.
%    \begin{macrocode}
\cs_new_protected:Npn \driver_draw_begin:
  {
    \@@_literal:n { ps::[begin] }
    \@@_draw_literal:n { @beginspecial }
    \@@_draw_literal:n { /l3fc~{ }~def }
  }
\cs_new_protected:Npn \driver_draw_end:
  {
    \@@_draw_literal:n { @endspecial }
    \@@_literal:n { ps::[end] }
  }
%    \end{macrocode}
% \end{macro}
%
% \begin{macro}{\driver_draw_scope_begin:, \driver_draw_scope_end:}
%   Scope here may need to contain saved definitions, so the entire memory
%   rather than just the graphic state has to be sent to the stack.
%    \begin{macrocode}
\cs_new_protected:Npn \driver_draw_scope_begin:
  { \@@_draw_literal:n { save } }
\cs_new_protected:Npn \driver_draw_scope_end:
  { \@@_draw_literal:n { restore } }
%    \end{macrocode}
% \end{macro}
%
% \begin{macro}{\driver_draw_moveto:nn, \driver_draw_lineto:nn}
% \begin{macro}{\driver_draw_rectangle:nnnn}
% \begin{macro}{\driver_draw_curveto:nnnnnn}
%   Path creation operations mainly resolve directly to PostScript primitive
%   steps, with only the need to convert to \texttt{bp}. Notice that
%   \texttt{x}-type expansion is included here to ensure that any variable
%   values are forced to literals before any possible caching. There is
%   no native rectangular path command (without also clipping, filling or
%   stroking), so that task is done using a small amount of PostScript.
%    \begin{macrocode}
\cs_new_protected:Npn \driver_draw_moveto:nn #1#2
  {
    \@@_draw_literal:x
      {
        \dim_to_decimal_in_bp:n {#1} ~ 
        \dim_to_decimal_in_bp:n {#2} ~ moveto
      }
  }
\cs_new_protected:Npn \driver_draw_lineto:nn #1#2
  {
    \@@_draw_literal:x
      {
        \dim_to_decimal_in_bp:n {#1} ~
        \dim_to_decimal_in_bp:n {#2} ~ lineto
      }
  }
\cs_new_protected:Npn \driver_draw_rectangle:nnnn #1#2#3#4
  {
     \@@_draw_literal:x
       {
         \dim_to_decimal_in_bp:n {#4} ~ \dim_to_decimal_in_bp:n {#3} ~
         \dim_to_decimal_in_bp:n {#1} ~ \dim_to_decimal_in_bp:n {#2} ~
         moveto~dup~0~rlineto~exch~0~exch~rlineto~neg~0~rlineto~closepath
      }
  }
\cs_new_protected:Npn \driver_draw_curveto:nnnnnn #1#2#3#4#5#6
  {
    \@@_draw_literal:x
      {
        \dim_to_decimal_in_bp:n {#1} ~ \dim_to_decimal_in_bp:n {#2} ~
        \dim_to_decimal_in_bp:n {#3} ~ \dim_to_decimal_in_bp:n {#4} ~
        \dim_to_decimal_in_bp:n {#5} ~ \dim_to_decimal_in_bp:n {#6} ~
        curveto
      }
 }
%    \end{macrocode}
% \end{macro}
% \end{macro}
% \end{macro}
%
% \begin{macro}{\driver_draw_evenodd_rule:, \driver_draw_nonzero_rule:}
% \begin{variable}{\g_@@_draw_eor_bool}
%    The even-odd rule here can be implemented as a simply switch.
%    \begin{macrocode}
\cs_new_protected:Npn \driver_draw_evenodd_rule:
  { \bool_gset_true:N \g_@@_draw_eor_bool }
\cs_new_protected:Npn \driver_draw_nonzero_rule:
  { \bool_gset_false:N \g_@@_draw_eor_bool }
\bool_new:N \g_@@_draw_eor_bool
%    \end{macrocode}
% \end{variable}
% \end{macro}
%
% \begin{macro}
%   {
%     \driver_draw_closepath:   ,
%     \driver_draw_stroke:      ,
%     \driver_draw_closestroke: ,
%     \driver_draw_fill:        ,
%     \driver_draw_fillstroke:  ,
%     \driver_draw_clip:        ,
%     \driver_draw_discardpath:
%   }
% \begin{variable}{\g_@@_draw_clip_bool}
%   Unlike PDF, PostScript doesn't track separate colors for strokes and other
%   elements. It is also desirable to have the |clip| keyword after a stroke or
%   fill. To achieve those outcomes, there is some work to do. For color, the
%   stoke color is simple but the fill one has to be inserted by hand. For
%   clipping, the required ordering is achieved using a \TeX{} switch. All of
%   the operations end with a new path instruction  as they do not terminate
%   (again in contrast to PDF).
%    \begin{macrocode}
\cs_new_protected:Npn \driver_draw_closepath:
  { \@@_draw_literal:n { closepath } }
\cs_new_protected:Npn \driver_draw_stroke:
  {
    \@@_draw_literal:n { stroke }
    \bool_if:NT \g_@@_draw_clip_bool
      {
        \@@_draw_literal:x
          {
            \bool_if:NT \g_@@_draw_eor_bool { eo }
            clip
          }
      }
    \@@_draw_literal:n { newpath }
    \bool_gset_false:N \g_@@_draw_clip_bool
  }
\cs_new_protected:Npn \driver_draw_closestroke:
  {
    \driver_draw_closepath:
    \driver_draw_stroke:
  }
\cs_new_protected:Npn \driver_draw_fill:
  {
    \@@_draw_literal:n { gsave }
    \@@_draw_literal:n { l3fc }
    \@@_draw_literal:x
      {
        \bool_if:NT \g_@@_draw_eor_bool { eo }
        fill
      }
    \@@_draw_literal:n { grestore }
    \bool_if:NT \g_@@_draw_clip_bool
      {
        \@@_draw_literal:x
          {
            \bool_if:NT \g_@@_draw_eor_bool { eo }
            clip
          }
      }
    \@@_draw_literal:n { newpath }
    \bool_gset_false:N \g_@@_draw_clip_bool
  }
\cs_new_protected:Npn \driver_draw_fillstroke:
  {
    \@@_draw_literal:n { gsave }
    \@@_draw_literal:n { l3fc }
    \@@_draw_literal:x
      {
        \bool_if:NT \g_@@_draw_eor_bool { eo }
        fill
      }
    \@@_draw_literal:n { grestore }
    \@@_draw_literal:n { stroke }
    \bool_if:NT \g_@@_draw_clip_bool
      {
        \@@_draw_literal:x
          {
            \bool_if:NT \g_@@_draw_eor_bool { eo }
            clip
          }
      }
    \@@_draw_literal:n { newpath }
    \bool_gset_false:N \g_@@_draw_clip_bool
  }
\cs_new_protected:Npn \driver_draw_clip:
  { \bool_gset_true:N \g_@@_draw_clip_bool }
\bool_new:N \g_@@_draw_clip_bool
\cs_new_protected:Npn \driver_draw_discardpath:
  {
    \bool_if:NT \g_@@_draw_clip_bool
      {
        \@@_draw_literal:x
          {
            \bool_if:NT \g_@@_draw_eor_bool { eo }
            clip
          }
      }
    \@@_draw_literal:n { newpath }
    \bool_gset_false:N \g_@@_draw_clip_bool
  }
%    \end{macrocode}
% \end{variable}
% \end{macro}
%
% \begin{macro}{\driver_draw_dash_pattern:nn}
% \begin{macro}{\@@_draw_dash:n}
% \begin{macro}{\driver_draw_linewidth:n}
% \begin{macro}{\driver_draw_miterlimit:n}
% \begin{macro}
%   {
%     \driver_draw_cap_butt:, \driver_draw_cap_round:, \driver_draw_cap_rectangle:,
%     \driver_draw_join_miter:, \driver_draw_join_round:, \driver_draw_join_bevel:
%   }
%   Converting paths to output is again a case of mapping directly to
%   PostScript operations.
%    \begin{macrocode}
\cs_new_protected:Npn \driver_draw_dash_pattern:nn #1#2
  {
    \@@_draw_literal:x
      {
        [ 
          \exp_args:Nf \use:n
            { \clist_map_function:nN {#1} \@@_draw_dash:n }
        ] ~
        \dim_to_decimal_in_bp:n {#2} ~ setdash
      }
  }
\cs_new:Npn \@@_draw_dash:n #1
  { ~ \dim_to_decimal_in_bp:n {#1} }
\cs_new_protected:Npn \driver_draw_linewidth:n #1
  {
    \@@_draw_literal:x
      { \dim_to_decimal_in_bp:n {#1} ~ setlinewidth }
  }
\cs_new_protected:Npn \driver_draw_miterlimit:n #1
  { \@@_draw_literal:x { \fp_eval:n {#1} ~ setmiterlimit } }
\cs_new_protected:Npn \driver_draw_cap_butt:
  { \@@_draw_literal:n { 0 ~ setlinecap } }
\cs_new_protected:Npn \driver_draw_cap_round:
  { \@@_draw_literal:n { 1 ~ setlinecap } }
\cs_new_protected:Npn \driver_draw_cap_rectangle:
  { \@@_draw_literal:n { 2 ~ setlinecap } }
\cs_new_protected:Npn \driver_draw_join_miter:
  { \@@_draw_literal:n { 0 ~ setlinejoin } }
\cs_new_protected:Npn \driver_draw_join_round:
  { \@@_draw_literal:n { 1 ~ setlinejoin } }
\cs_new_protected:Npn \driver_draw_join_bevel:
  { \@@_draw_literal:n { 2 ~ setlinejoin } }
%    \end{macrocode}
% \end{macro}
% \end{macro}
% \end{macro}
% \end{macro}
% \end{macro}
%
% \begin{macro}
%   {
%     \driver_draw_color_fill_cmyk:nnnn   ,
%     \driver_draw_color_stroke_cmyk:nnnn
%   }
% \begin{macro}
%   {
%     \driver_draw_color_fill_gray:n   ,
%     \driver_draw_color_stroke_gray:n
%   }
% \begin{macro}
%   {
%     \driver_draw_color_fill_rgb:nnn   ,
%     \driver_draw_color_stroke_rgb:nnn
%   }
% \begin{macro}
%   {
%     \@@_draw_color_fill:n, \@@_draw_color_fill:x,
%     \@@_draw_color_stroke:n, \@@_draw_color_stroke:x
%   }
%   For \texttt{dvips}, we can use the standard color stack to deal with
%   stroke color, but for fills have to switch to raw PostScript. This is
%   thus not handled by the stack, but the context is very restricted. See
%   also how fills are implemented.
%    \begin{macrocode}
\cs_new_protected:Npn \driver_draw_color_fill_cmyk:nnnn #1#2#3#4
  {
    \@@_draw_color_fill:x
      {
        \fp_eval:n {#1} ~ \fp_eval:n {#2} ~
        \fp_eval:n {#3} ~ \fp_eval:n {#4} ~
        setcmykcolor
      }
  }
\cs_new_protected:Npn \driver_draw_color_stroke_cmyk:nnnn #1#2#3#4
  {
    \@@_draw_color_stroke:x
      {
        cmyk ~
        \fp_eval:n {#1} ~ \fp_eval:n {#2} ~
        \fp_eval:n {#3} ~ \fp_eval:n {#4}
      }
  }
\cs_new_protected:Npn \driver_draw_color_fill_gray:n #1
  { \@@_draw_color_fill:x { \fp_eval:n {#1} ~ setgray } }
\cs_new_protected:Npn \driver_draw_color_stroke_gray:n #1
  { \@@_draw_color_stroke:x { gray ~ \fp_eval:n {#1} } }
\cs_new_protected:Npn \driver_draw_color_fill_rgb:nnn #1#2#3
  {
    \@@_draw_color_fill:x
      { \fp_eval:n {#1} ~ \fp_eval:n {#2} ~ \fp_eval:n {#3} ~ setrgbcolor }
  }
\cs_new_protected:Npn \driver_draw_color_stroke_rgb:nnn #1#2#3
  {
    \@@_draw_color_stroke:x
      { rgb ~ \fp_eval:n {#1} ~ \fp_eval:n {#2} ~ \fp_eval:n {#3} }
  }
\cs_new_protected:Npn \@@_draw_color_fill:n #1
  { \@@_draw_literal:n { /l3fc ~ { #1 } ~ def } }
\cs_generate_variant:Nn \@@_draw_color_fill:n { x }
\cs_new_protected:Npn \@@_draw_color_stroke:n #1
  {
    \@@_literal:n { color~push~#1 } 
    \group_insert_after:N \@@_color_reset:
  }
\cs_generate_variant:Nn \@@_draw_color_stroke:n { x }
%    \end{macrocode}
% \end{macro}
% \end{macro}
% \end{macro}
% \end{macro}
%
% \begin{macro}{\driver_draw_cm:nnnn}
%   In \texttt{dvips}, keeping the transformations in line with the engine
%   is unfortunately not possible for scaling and rotations: even if we
%   decompose the matrix into those operations, there is still no driver
%   tracking (\emph{cf.}~\texttt{(x)dvipdfmx}). Thus we take the shortest
%   path available and simply dump the matrix as given.
%    \begin{macrocode}
\cs_new_protected:Npn \driver_draw_cm:nnnn #1#2#3#4
  {
    \@@_draw_literal:n
      {
        [
          \fp_eval:n {#1} ~ \fp_eval:n {#2} ~
          \fp_eval:n {#3} ~ \fp_eval:n {#4} ~
          0 ~ 0
        ] ~
        concat
      }
  }
%    \end{macrocode}
% \end{macro}
%
% \begin{macro}{\driver_draw_box_use:Nnnnn}
%   Inside a picture |@beginspecial|/|@endspecial| are active, which is
%   normally a good thing but means that the position and scaling would be off
%   if the box was inserted directly. To deal with that, there are a number of
%   possible approaches. The implementation here was suggested by Tom Rokici
%   (author of \texttt{dvips}). We end the current special placement, then
%   set the current point with a literal |[begin]|. As for general
%   literals, we then use the stack to store the current point and move to
%   it. To insert the required transformation, we have to flip the $y$-axis,
%   once before and once after it. Then we get back to the \TeX{} reference
%   point to insert our content. The clean up has to happen in the right
%   places, hence the |[begin]|/|[end]| pair around |restore|. Finally,
%   we can return to \enquote{normal} drawing mode. Notice that the set up
%   here is very similar to that in \cs{@@_align_currentpoint_\ldots}, but
%   the ordering of saving and restoring is different (intermixed).
%    \begin{macrocode}
\cs_new_protected:Npn \driver_draw_box_use:Nnnnn #1#2#3#4#5
  {
    \@@_draw_literal:n { @endspecial }
    \@@_draw_literal:n { [end] }
    \@@_draw_literal:n { [begin] }
    \@@_draw_literal:n { save }
    \@@_draw_literal:n { currentpoint }
    \@@_draw_literal:n { currentpoint~translate }
    \driver_draw_cm:nnnn { 1 } { 0 } { 0 } { -1 }
    \driver_draw_cm:nnnn {#2} {#3} {#4} {#5}
    \driver_draw_cm:nnnn { 1 } { 0 } { 0 } { -1 }
    \@@_draw_literal:n { neg~exch~neg~exch~translate }
    \@@_draw_literal:n { [end] }
    \hbox_overlap_right:n { \box_use:N #1 }
    \@@_draw_literal:n { [begin] }
    \@@_draw_literal:n { restore }
    \@@_draw_literal:n { [end] }
    \@@_draw_literal:n { [begin] }
    \@@_draw_literal:n { @beginspecial }
  }
%    \end{macrocode}
% \end{macro}
%
% \subsubsection{PDF Features}
%
% \begin{variable}{\g_@@_pdf_object_int, \g_@@_pdf_object_prop}
%   For tracking objects to allow finalisation.
%    \begin{macrocode}
\int_new:N \g_@@_pdf_object_int
\prop_new:N \g_@@_pdf_object_prop
%    \end{macrocode}
% \end{variable}
%
% \begin{macro}{\driver_pdf_object_new:nn}
% \begin{macro}[EXP]{\driver_pdf_object_ref:n}
%   Tracking objects is similar to \texttt{dvipdfmx}.
%    \begin{macrocode}
\cs_new_protected:Npn \driver_pdf_object_new:nn #1#2
  {
    \int_gincr:N \g_@@_pdf_object_int
    \int_const:cn
      { g_@@_pdf_object_ \tl_to_str:n {#1} _int }
      { \g_@@_pdf_object_int }
    \prop_gput:Nnn \g_@@_pdf_object_prop {#1} {#2}
  }
\cs_new:Npn \driver_pdf_object_ref:n #1
  { { l3obj \int_use:c { g_@@_pdf_object_ \tl_to_str:n {#1} _int } } }
%    \end{macrocode}
% \end{macro}
% \end{macro}
%
% \begin{macro}{\driver_pdf_object_write:nn}
% \begin{macro}
%   {
%     \@@_pdf_object_write_array:nn ,
%     \@@_pdf_object_write_dict:nn  ,
%     \@@_pdf_object_write_stream:nn
%   }
% \begin{macro}{\@@_pdf_object_write_stream:nnn}
%   This is where we choose the actual type: some work to get things
%   right.
%    \begin{macrocode}
\cs_new_protected:Npn \driver_pdf_object_write:nn #1#2
  {
    \@@_literal_postscript:x
      {
        mark ~ /_objdef ~ \driver_pdf_object_ref:n {#1} ~
        /type
        \str_case_e:nn
          { \prop_item:Nn \g_@@_pdf_object_prop {#1} }
          {
            { array }   { /array }
            { dict }    { /dict }
            { fstream } { /stream }
            { stream }  { /stream }
          }
        /OBJ ~ pdfmark
      }
    \use:c
      { @@_pdf_object_write_ \prop_item:Nn \g_@@_pdf_object_prop {#1} :nn }
      {#1} {#2}
  }
\cs_new_protected:Npn \@@_pdf_object_write_array:nn #1#2
  {
    \@@_literal_postscript:x
      {
        mark ~ \driver_pdf_object_ref:n {#1} ~
          [ ~ \exp_not:n {#2} ~ ] ~ /PUTINTERVAL ~ pdfmark
      }
  }
\cs_new_protected:Npn \@@_pdf_object_write_dict:nn #1#2
  {
    \@@_literal_postscript:x
      {
        mark ~ \driver_pdf_object_ref:n {#1} ~
          << ~ \exp_not:n {#2} ~ >> ~ /PUT ~ pdfmark
      }
  }
\cs_new_protected:Npn \@@_pdf_object_write_stream:nn #1#2
  {
    \exp_args:Nx
      \@@_pdf_object_write_stream:nnn
        { \driver_pdf_object_ref:n {#1} }
        #2
  }
\cs_new_protected:Npn \@@_pdf_object_write_stream:nnn #1#2#3
  {
    \@@_literal_postscript:n
      {
        [nobreak] ~
        mark ~ #1 ~ ( #3 ) ~ /PUT ~ pdfmark ~
        mark ~ #1 ~ << #2 >> ~ /PUT ~ pdfmark
      }
  }
%    \end{macrocode}
% \end{macro}
% \end{macro}
% \end{macro}
%
% \begin{macro}{\driver_pdf_compresslevel:n}
% \begin{macro}{\driver_pdf_objects_enable:, \driver_pdf_objects_disable:}
%   These are all no-ops.
%    \begin{macrocode}
\cs_new_protected:Npn \driver_pdf_compresslevel:n #1 { }
\cs_new_protected:Npn \driver_pdf_objects_enable: { }
\cs_new_protected:Npn \driver_pdf_objects_disable: { }
%    \end{macrocode}
% \end{macro}
% \end{macro}
%
%    \begin{macrocode}
%</dvips>
%    \end{macrocode}
%
% \subsection{\texttt{pdfmode} driver}
%
%    \begin{macrocode}
%<*pdfmode>
%    \end{macrocode}
%
% The direct PDF driver covers both \pdfTeX{} and \LuaTeX{}. The latter
% renames and restructures the driver primitives but this can be handled
% at one level of abstraction. As such, we avoid using two separate drivers
% for this material at the cost of some \texttt{x}-type definitions to get
% everything expanded up-front.
%
% \subsubsection{Basics}
%
% \begin{macro}{\@@_literal_pdf:n, \@@_literal_pdf:x}
%   This is equivalent to \verb|\special{pdf:}| but the engine can
%   track it. Without the \texttt{direct} keyword everything is kept in
%   sync: the transformation matrix is set to the current point automatically.
%   Note that this is still inside the text (\texttt{BT} \dots \texttt{ET}
%   block).
%    \begin{macrocode}
\cs_new_protected:Npx \@@_literal_pdf:n #1
  {
    \cs_if_exist:NTF \tex_pdfextension:D
      { \tex_pdfextension:D literal }
      { \tex_pdfliteral:D }
        { \exp_not:N \exp_not:n {#1} }
  }
\cs_generate_variant:Nn \@@_literal_pdf:n { x }
%    \end{macrocode}
% \end{macro}
%
% \begin{macro}{\@@_scope_begin:, \@@_scope_end:}
%  Higher-level interfaces for saving and restoring the graphic state.
%    \begin{macrocode}
\cs_new_protected:Npx \@@_scope_begin:
  {
    \cs_if_exist:NTF \tex_pdfextension:D
      { \tex_pdfextension:D save \scan_stop: }
      { \tex_pdfsave:D }
  }
\cs_new_protected:Npx \@@_scope_end:
  {
    \cs_if_exist:NTF \tex_pdfextension:D
      { \tex_pdfextension:D restore \scan_stop: }
      { \tex_pdfrestore:D }
  }
%    \end{macrocode}
% \end{macro}
%
% \begin{macro}{\@@_matrix:n, \@@_matrix:x}
%   Here the appropriate function is set up to insert an affine matrix
%   into the PDF. With \pdfTeX{} and \LuaTeX{} in direct PDF output mode there
%   is a primitive for this, which only needs the rotation/scaling/skew part.
%    \begin{macrocode}
\cs_new_protected:Npx \@@_matrix:n #1
  {
    \cs_if_exist:NTF \tex_pdfextension:D
      { \tex_pdfextension:D setmatrix }
      { \tex_pdfsetmatrix:D }
        { \exp_not:N \exp_not:n {#1} }
  }
\cs_generate_variant:Nn \@@_matrix:n { x }
%    \end{macrocode}
% \end{macro}
%
% \subsubsection{Box operations}
%
% \begin{macro}{\driver_box_use_clip:N}
%   The general method is to save the current location, define a clipping path
%   equivalent to the bounding box, then insert the content at the current
%   position and in a zero width box. The \enquote{real} width is then made up
%   using a horizontal skip before tidying up. There are other approaches that
%   can be taken (for example using XForm objects), but the logic here shares
%   as much code as possible and uses the same conversions (and so same
%   rounding errors) in all cases.
%    \begin{macrocode}
\cs_new_protected:Npn \driver_box_use_clip:N #1
  {
    \@@_scope_begin:
    \@@_literal_pdf:x
      {
        0~
        \dim_to_decimal_in_bp:n { -\box_dp:N #1 } ~
        \dim_to_decimal_in_bp:n { \box_wd:N #1 } ~
        \dim_to_decimal_in_bp:n { \box_ht:N #1 + \box_dp:N #1 } ~
        re~W~n
      }
    \hbox_overlap_right:n { \box_use:N #1 }
    \@@_scope_end:
    \skip_horizontal:n { \box_wd:N #1 }
  }
%    \end{macrocode}
% \end{macro}
%
% \begin{macro}{\driver_box_use_rotate:Nn}
% \begin{macro}{\@@_box_use_rotate:Nn}
% \begin{variable}{\l_@@_cos_fp, \l_@@_sin_fp}
%   Rotations are set using an affine transformation matrix which therefore
%   requires sine/cosine values not the angle itself. We store the rounded
%   values to avoid rounding twice. There are also a couple of comparisons to
%   ensure that |-0| is not written to the output, as this avoids any issues
%   with problematic display programs.  Note that numbers are compared to~$0$
%   after rounding.
%    \begin{macrocode}
\cs_new_protected:Npn \driver_box_use_rotate:Nn #1#2
  { \exp_args:NNf \@@_box_use_rotate:Nn #1 { \fp_eval:n {#2} } }
\cs_new_protected:Npn \@@_box_use_rotate:Nn #1#2
  {
    \@@_scope_begin:
    \box_set_wd:Nn #1 { 0pt }
    \fp_set:Nn \l_@@_cos_fp { round ( cosd ( #2 ) , 5 ) }
    \fp_compare:nNnT \l_@@_cos_fp = \c_zero_fp
      { \fp_zero:N \l_@@_cos_fp }
    \fp_set:Nn \l_@@_sin_fp { round ( sind ( #2 ) , 5 ) }
    \@@_matrix:x
      {
        \fp_use:N \l_@@_cos_fp \c_space_tl
        \fp_compare:nNnTF \l_@@_sin_fp = \c_zero_fp
          { 0~0 }
          {
            \fp_use:N \l_@@_sin_fp
            \c_space_tl
            \fp_eval:n { -\l_@@_sin_fp }
          }
        \c_space_tl
        \fp_use:N \l_@@_cos_fp
      }
   \box_use:N #1
   \@@_scope_end:
  }
\fp_new:N \l_@@_cos_fp
\fp_new:N \l_@@_sin_fp
%    \end{macrocode}
% \end{variable}
% \end{macro}
% \end{macro}
%
% \begin{macro}{\driver_box_use_scale:Nnn}
%   The same idea as for rotation but without the complexity of signs and
%   cosines.
%    \begin{macrocode}
\cs_new_protected:Npn \driver_box_use_scale:Nnn #1#2#3
  {
    \@@_scope_begin:
    \@@_matrix:x
      {
        \fp_eval:n { round ( #2 , 5 ) } ~
        0~0~
        \fp_eval:n { round ( #3 , 5 ) }
      }
    \hbox_overlap_right:n { \box_use:N #1 }
    \@@_scope_end:
  }
%    \end{macrocode}
% \end{macro}
%
% \subsubsection{Images}
%
% \begin{variable}{\l_@@_image_attr_tl}
%   In PDF mode, additional attributes of an image (such as page number) are
%   needed both to obtain the bounding box and when inserting the image: this
%   occurs as the image dictionary approach means they are read as part of
%   the bounding box operation. As such, it is easier to track additional
%   attributes using a dedicated |tl| rather than build up the same data
%   twice.
%    \begin{macrocode}
\tl_new:N \l_@@_image_attr_tl
%    \end{macrocode}
% \end{variable}
%
% \begin{macro}
%   {\@@_image_getbb_jpg:n, \@@_image_getbb_pdf:n, \@@_image_getbb_png:n}
% \begin{macro}
%   {\@@_image_getbb_auxi:n, \@@_image_getbb_auxii:n}
%   Getting the bounding box here requires us to box up the image and
%   measure it. To deal with the difference in feature support in bitmap
%   and vector images but keeping the common parts, there is a little work
%   to do in terms of auxiliaries. The key here is to notice that we need
%   two forms of the attributes: a \enquote{short} set to allow us to
%   track for caching, and the full form to pass to the primitive.
%    \begin{macrocode}
\cs_new_protected:Npn \@@_image_getbb_jpg:n #1
  {
    \int_zero:N \l_image_page_int
    \tl_clear:N \l_image_pagebox_tl
    \tl_set:Nx \l_@@_image_attr_tl
      {
        \tl_if_empty:NF \l_image_decode_tl
          { :D \l_image_decodearray_tl }
        \bool_if:NT \l_image_interpolate_bool
          { :I }
      }
    \tl_clear:N \l_@@_image_attr_tl
    \@@_image_getbb_auxi:n {#1}
  }
\cs_new_eq:NN \@@_image_getbb_png:n \@@_image_getbb_jpg:n
\cs_new_protected:Npn \@@_image_getbb_pdf:n #1
  {
    \tl_clear:N \l_image_decode_tl
    \bool_set_false:N \l_image_interpolate_bool
    \tl_set:Nx \l_@@_image_attr_tl
      {
        : \l_image_pagebox_tl
        \int_compare:nNnT \l_image_page_int > 1
          { :P \int_use:N \l_image_page_int }
      }
    \@@_image_getbb_auxi:n {#1}
  }
\cs_new_protected:Npn \@@_image_getbb_auxi:n #1
  {
    \image_bb_restore:xF { #1 \l_@@_image_attr_tl }
      { \@@_image_getbb_auxii:n {#1} }
  }
%    \begin{macrocode}
%   Measuring the image is done by boxing up: for PDF images we could
%   use |\tex_pdfximagebbox:D|, but if doesn't work for other types.
%   As the box always starts at $(0,0)$ there is no need to worry about
%   the lower-left position.
%    \begin{macrocode}
\cs_new_protected:Npn \@@_image_getbb_auxii:n #1
  {
    \tex_immediate:D \tex_pdfximage:D
      \bool_lazy_or:nnT
        { \l_image_interpolate_bool }
        { ! \tl_if_empty_p:N \l_image_decodearray_tl }
        {
          attr ~
            {
              \tl_if_empty:NF \l_image_decode_tl
                { /Decode~[ \l_image_decodearray_tl ] }
              \bool_if:NT \l_image_interpolate_bool
                { /Interpolate~true }
            }
        }
      \int_compare:nNnT \l_image_page_int > 0
        { page ~ \int_use:N \l_image_page_int }
      \tl_if_empty:NF \l_image_pagebox_tl
        { \l_image_pagebox_tl }
      {#1}
    \hbox_set:Nn \l_@@_tmp_box
      { \tex_pdfrefximage:D \tex_pdflastximage:D }
    \dim_set:Nn \l_image_urx_dim { \box_wd:N \l_@@_tmp_box }
    \dim_set:Nn \l_image_ury_dim { \box_ht:N \l_@@_tmp_box }
    \int_const:cn { c_@@_image_ #1 \l_@@_image_attr_tl _int }
      { \tex_the:D \tex_pdflastximage:D }
    \image_bb_save:x { #1 \l_@@_image_attr_tl }
  }
%    \end{macrocode}
% \end{macro}
% \end{macro}
%
% \begin{macro}
%   {\@@_image_include_jpg:n, \@@_image_include_pdf:n, \@@_image_include_png:n}
%   Images are already loaded for the measurement part of the code, so
%   inclusion is straight-forward, with only any attributes to worry about. The
%   latter carry through from determination of the bounding box.
%    \begin{macrocode}
\cs_new_protected:Npn \@@_image_include_jpg:n #1
  {
    \tex_pdfrefximage:D
      \int_use:c { c_@@_image_ #1 \l_@@_image_attr_tl _int }
  }
\cs_new_eq:NN \@@_image_include_pdf:n \@@_image_include_jpg:n
\cs_new_eq:NN \@@_image_include_png:n \@@_image_include_jpg:n
%    \end{macrocode}
% \end{macro}
%
% \subsubsection{PDF Objects}
%
% \begin{variable}{\g_@@_pdf_object_prop}
%   For tracking objects to allow finalisation.
%    \begin{macrocode}
\prop_new:N \g_@@_pdf_object_prop
%    \end{macrocode}
% \end{variable}
%
% \begin{macro}{\driver_pdf_object_new:nn}
% \begin{macro}[EXP]{\driver_pdf_object_ref:n}
%   Declaring objects means reserving at the PDF level plus starting
%   tracking.
%    \begin{macrocode}
\group_begin:
  \cs_set_protected:Npn \@@_tmp:w #1#2
    {
      \cs_new_protected:Npx \driver_pdf_object_new:nn ##1##2
        {
          #1 reserveobjnum ~
          \int_const:cn
            { g_@@_pdf_object_ \exp_not:N \tl_to_str:n {##1} _int }
            {#2}
          \prop_gput:Nnn \exp_not:N \g_@@_pdf_object_prop {##1} {##2}
        }
    }
  \cs_if_exist:NTF \tex_pdfextension:D
    {
      \@@_tmp:w
        { \tex_pdfextension:D obj ~ }
        { \tex_pdffeedback:D lastobj \scan_stop: }
    }
    { \@@_tmp:w { \tex_pdfobj:D } { \tex_pdflastobj:D } }
\group_end:
\cs_new:Npn \driver_pdf_object_ref:n #1
  { \int_use:c { g_@@_pdf_object_ \tl_to_str:n {#1} _int } ~ 0 ~ R }
%    \end{macrocode}
% \end{macro}
% \end{macro}
%
% \begin{macro}{\driver_pdf_object_write:nn}
% \begin{macro}[EXP]{\@@_exp_not_i:nn, \@@_exp_not_ii:nn}
%   Writing the data needs a little information about the structure of the
%   object.
%    \begin{macrocode}
\group_begin:
  \cs_set_protected:Npn \@@_tmp:w #1
    {
      \cs_new_protected:Npn \driver_pdf_object_write:nn ##1##2
        {
          \tex_immediate:D #1 useobjnum ~
          \int_use:c
            { g_@@_pdf_object_ \tl_to_str:n {##1} _int }
            \str_case_e:nn
              { \prop_item:Nn \g_@@_pdf_object_prop {##1} }
              {
                { array } { { [ ~ \exp_not:n {##2} ~ ] } }
                { dict }  { { << ~ \exp_not:n {##2} ~ >> } }
                { fstream }
                  {
                    stream ~ attr ~ { \@@_exp_not_i:nn ##2 } ~
                      file ~ { \@@_exp_not_ii:nn ##2 }
                  }
                { stream }
                  {
                    stream ~ attr ~ { \@@_exp_not_i:nn ##2 } ~ 
                      { \@@_exp_not_ii:nn ##2 }
                  }
              }
        }
    }
  \cs_if_exist:NTF \tex_pdfextension:D
    { \@@_tmp:w { \tex_pdfextension:D obj ~ } }
    { \@@_tmp:w { \tex_pdfobj:D } }
\group_end:
\cs_new:Npn \@@_exp_not_i:nn #1#2 { \exp_not:n {#1} }
\cs_new:Npn \@@_exp_not_ii:nn #1#2 { \exp_not:n {#2} }
%    \end{macrocode}
% \end{macro}
% \end{macro}
%
% \subsubsection{PDF Structure}
%
% \begin{macro}{\driver_pdf_compresslevel:n}
% \begin{macro}{\driver_pdf_objects_enable:, \driver_pdf_objects_disable:}
% \begin{macro}{\@@_pdf_objectlevel:n}
%   Simply pass data to the engine.
%    \begin{macrocode}
\cs_new_protected:Npx \driver_pdf_compresslevel:n #1
  {
    \cs_if_exist:NTF \tex_pdfcompresslevel:D
      { \tex_pdfcompresslevel:D }
      { \tex_pdfvariable:D compresslevel }
      \exp_not:N \int_value:w \exp_not:N \int_eval:n {#1} \scan_stop:
  }
\cs_new_protected:Npn \driver_pdf_objects_enable:
  { \@@_pdf_objectlevel:n { 2 } }
\cs_new_protected:Npn \driver_pdf_objects_disable:
  { \@@_pdf_objectlevel:n { 0 } }
\cs_new_protected:Npx \@@_pdf_objectlevel:n #1
  {
    \cs_if_exist:NTF \tex_pdfobjcompresslevel:D
      { \tex_pdfobjcompresslevel:D }
      { \tex_pdfvariable:D objcompresslevel }
      #1 \scan_stop:
  }
%    \end{macrocode}
% \end{macro}
% \end{macro}
% \end{macro}
%
%    \begin{macrocode}
%</pdfmode>
%    \end{macrocode}
%
% \subsection{\texttt{dvipdfmx} driver}
%
%    \begin{macrocode}
%<*dvipdfmx|xdvipdfmx>
%    \end{macrocode}
%
% The \texttt{dvipdfmx} shares code with the PDF mode one (using the common
% section to this file) but also with \texttt{xdvipdfmx}. The latter is close
% to identical to \texttt{dvipdfmx} and so all of the code here is extracted
% for both drivers, with some \texttt{clean up} for \texttt{xdvipdfmx} as
% required.
%
% \subsubsection{Basics}
%
% \begin{macro}{\@@_literal_pdf:n, \@@_literal_pdf:x}
%   Equivalent to \texttt{pdf:content} but favored as the link to
%   the \pdfTeX{} primitive approach is clearer.
%    \begin{macrocode}
\cs_new_protected:Npn \@@_literal_pdf:n #1
  { \@@_literal:n { pdf:literal~ #1 } }
\cs_generate_variant:Nn \@@_literal_pdf:n { x }
%    \end{macrocode}
% \end{macro}
%
% \begin{macro}{\@@_scope_begin:, \@@_scope_end:}
%   Scoping is done using the driver-specific specials.
%    \begin{macrocode}
\cs_new_protected:Npn \@@_scope_begin:
  { \@@_literal:n { x:gsave } }
\cs_new_protected:Npn \@@_scope_end:
  { \@@_literal:n { x:grestore } }
%    \end{macrocode}
% \end{macro}
%
% \subsubsection{Box operations}
%
% \begin{macro}{\driver_box_use_clip:N}
%   The code here is identical to that for \texttt{pdfmode}: unlike rotation and
%   scaling, there is no higher-level support in the driver for clipping.
%    \begin{macrocode}
\cs_new_protected:Npn \driver_box_use_clip:N #1
  {
    \@@_scope_begin:
    \@@_literal_pdf:x
      {
        0~
        \dim_to_decimal_in_bp:n { -\box_dp:N #1 } ~
        \dim_to_decimal_in_bp:n { \box_wd:N #1 } ~
        \dim_to_decimal_in_bp:n { \box_ht:N #1 + \box_dp:N #1 } ~
        re~W~n
      }
    \hbox_overlap_right:n { \box_use:N #1 }
    \@@_scope_end:
    \skip_horizontal:n { \box_wd:N #1 }
  }
%    \end{macrocode}
% \end{macro}
%
% \begin{macro}{\driver_box_use_rotate:Nn}
% \begin{macro}{\@@_box_use_rotate:Nn}
%   Rotating in \texttt{(x)}dvipdmfx can be implemented using either PDF or
%   driver-specific code. The former approach however is not \enquote{aware}
%   of the content of boxes: this means that any embedded links would not be
%   adjusted by the rotation. As such, the driver-native approach is preferred:
%   the code therefore is similar (though not identical) to the \texttt{dvips}
%   version (notice the rotation angle here is positive). As for
%   \texttt{dvips}, zero rotation is written as |0| not |-0|.
%    \begin{macrocode}
\cs_new_protected:Npn \driver_box_use_rotate:Nn #1#2
  { \exp_args:NNf \@@_box_use_rotate:Nn #1 { \fp_eval:n {#2} } }
\cs_new_protected:Npn \@@_box_use_rotate:Nn #1#2
  {
    \@@_scope_begin:
    \@@_literal:x
      {
        x:rotate~
        \fp_compare:nNnTF {#2} = \c_zero_fp
          { 0 }
          { \fp_eval:n { round ( #2 , 5 ) } }
      }
    \box_use:N #1
    \@@_scope_end:
  }
%    \end{macrocode}
% \end{macro}
% \end{macro}
%
% \begin{macro}{\driver_box_use_scale:Nnn}
%   Much the same idea for scaling: use the higher-level driver operation to allow
%   for box content.
%    \begin{macrocode}
\cs_new_protected:Npn \driver_box_use_scale:Nnn #1#2#3
  {
    \@@_scope_begin:
    \@@_literal:x
      {
        x:scale~
        \fp_eval:n { round ( #2 , 5 ) } ~
        \fp_eval:n { round ( #3 , 5 ) }
      }
    \hbox_overlap_right:n { \box_use:N #1 }
    \@@_scope_end:
  }
%    \end{macrocode}
% \end{macro}
%
% \subsubsection{Images}
%
% \begin{macro}
%   {
%     \@@_image_getbb_eps:n, \@@_image_getbb_jpg:n,
%     \@@_image_getbb_pdf:n, \@@_image_getbb_png:n
%   }
%   Simply use the generic functions: only for \texttt{dvipdfmx} in the
%   extraction cases.
%    \begin{macrocode}
\cs_new_eq:NN \@@_image_getbb_eps:n \image_read_bb:n
%<*dvipdfmx>
\cs_new_protected:Npn \@@_image_getbb_jpg:n #1
  {
    \int_zero:N \l_image_page_int
    \tl_clear:N \l_image_pagebox_tl
    \image_extract_bb:n {#1}
  }
\cs_new_eq:NN \@@_image_getbb_png:n \@@_image_getbb_jpg:n
\cs_new_protected:Npn \@@_image_getbb_pdf:n #1
  {
    \tl_clear:N \l_image_decode_tl
    \bool_set_false:N \l_image_interpolate_bool
    \image_extract_bb:n {#1}
  }
%</dvipdfmx>
%    \end{macrocode}
% \end{macro}
%
% \begin{variable}{\g_@@_image_int}
%   Used to track the object number associated with each image.
%    \begin{macrocode}
\int_new:N \g_@@_image_int
%    \end{macrocode}
% \end{variable}
%
% \begin{macro}
%   {
%     \@@_image_include_eps:n, \@@_image_include_jpg:n,
%     \@@_image_include_pdf:n, \@@_image_include_png:n
%   }
%  \begin{macro}{\@@_image_include_auxi:nn}
%  \begin{macro}{\@@_image_include_auxii:nnn, \@@_image_include_auxii:xnn}
%  \begin{macro}{\@@_image_include_auxiii:nn}
%   The special syntax depends on the file type. There is a difference in
%   how PDF images are best handled between |dvipdfmx| and |xdvipdfmx|: for
%   the latter it is better to use the primitive route. The relevant code for
%   that is included later in this file.
%    \begin{macrocode}
\cs_new_protected:Npn \@@_image_include_eps:n #1
  {
    \@@_literal:n { PSfile = #1 }
  }
\cs_new_protected:Npn \@@_image_include_jpg:n #1
  { \@@_image_include_auxi:nn {#1} { image } }
\cs_new_eq:NN \@@_image_include_png:n \@@_image_include_jpg:n
%<*dvipdfmx>
\cs_new_protected:Npn \@@_image_include_pdf:n #1
  { \@@_image_include_auxi:nn {#1} { epdf } }
%</dvipdfmx>
%    \end{macrocode}
%   Image inclusion is set up to use the fact that each image is stored in
%   the PDF as an XObject. This means that we can include repeated images
%   only once and refer to them. To allow that, track the nature of each
%   image: much the same as for the direct PDF mode case.
%    \begin{macrocode}
\cs_new_protected:Npn \@@_image_include_auxi:nn #1#2
  {
    \@@_image_include_auxii:xnn
      {
        \tl_if_empty:NF \l_image_pagebox_tl
          { : \l_image_pagebox_tl }
        \int_compare:nNnT \l_image_page_int > 1
          { :P \int_use:N \l_image_page_int }
        \tl_if_empty:NF \l_image_decode_tl
          { :D \l_image_decodearray_tl }
        \bool_if:NT \l_image_interpolate_bool
           { :I }
      }
      {#1} {#2}
  }
\cs_new_protected:Npn \@@_image_include_auxii:nnn #1#2#3
  {
    \int_if_exist:cTF { c_@@_image_ #2#1 _int }
      {
        \@@_literal:x
          { pdf:usexobj~@image \int_use:c { c_@@_image_ #2#1 _int } }
      }
      { \@@_image_include_auxiii:nn {#2} {#1} {#3} }
  }
\cs_generate_variant:Nn \@@_image_include_auxii:nnn { x }
%    \end{macrocode}
%  Inclusion using the specials is relatively straight-forward, but there
%  is one wrinkle. To get the |pagebox| correct for PDF images in all cases,
%  it is necessary to provide both that information and the |bbox| argument:
%  odd things happen otherwise!
%    \begin{macrocode}
\cs_new_protected:Npn \@@_image_include_auxiii:nnn #1#2#3
  {
    \int_gincr:N \g_@@_image_int
    \int_const:cn { c_@@_image_ #1#2 _int } { \g_@@_image_int }
    \@@_literal:x
      {
        pdf:#3~
        @image \int_use:c { c_@@_image_ #1#2 _int }
        \int_compare:nNnT \l_image_page_int > 1
          { page ~ \int_use:N \l_image_page_int \c_space_tl }
        \tl_if_empty:NF \l_image_pagebox_tl
          {
            pagebox ~ \l_image_pagebox_tl \c_space_tl
            bbox ~
              \dim_to_decimal_in_bp:n \l_image_llx_dim \c_space_tl
              \dim_to_decimal_in_bp:n \l_image_lly_dim \c_space_tl
              \dim_to_decimal_in_bp:n \l_image_urx_dim \c_space_tl
              \dim_to_decimal_in_bp:n \l_image_ury_dim \c_space_tl
          }
        (#1)
        \bool_lazy_or:nnT
          { \l_image_interpolate_bool }
          { ! \tl_if_empty_p:N \l_image_decodearray_tl }
          {
            <<
              \tl_if_empty:NF \l_image_decode_tl
                { /Decode~[ \l_image_decodearray_tl ] }
              \bool_if:NT \l_image_interpolate_bool
                { /Interpolate~true> }
            >>
          }
      }
  }
%    \end{macrocode}
% \end{macro}
% \end{macro}
% \end{macro}
% \end{macro}
%
% \subsubsection{PDF Objects}
%
% \begin{variable}{\g_@@_pdf_object_int, \g_@@_pdf_object_prop}
%   For tracking objects to allow finalisation.
%    \begin{macrocode}
\int_new:N \g_@@_pdf_object_int
\prop_new:N \g_@@_pdf_object_prop
%    \end{macrocode}
% \end{variable}
%
% \begin{macro}{\driver_pdf_object_new:nn}
% \begin{macro}[EXP]{\driver_pdf_object_ref:n}
%   Objects are tracked at the macro level, but we don't have to do anything
%   at this stage.
%    \begin{macrocode}
\cs_new_protected:Npn \driver_pdf_object_new:nn #1#2
  {
    \int_gincr:N \g_@@_pdf_object_int
    \int_const:cn
      { g_@@_pdf_object_ \tl_to_str:n {#1} _int }
      { \g_@@_pdf_object_int }
    \prop_gput:Nnn \g_@@_pdf_object_prop {#1} {#2}
  }
\cs_new:Npn \driver_pdf_object_ref:n #1
  { @l3obj \int_use:c { g_@@_pdf_object_ \tl_to_str:n {#1} _int } }
%    \end{macrocode}
% \end{macro}
% \end{macro}
%
% \begin{macro}{\driver_pdf_object_write:nn}
% \begin{macro}{\@@_pdf_object_write:nnn}
% \begin{macro}
%   {
%     \@@_pdf_object_write_array:nn   ,
%     \@@_pdf_object_write_dict:nn    ,
%     \@@_pdf_object_write_fstream:nn ,
%     \@@_pdf_object_write_stream:nn
%   }
% \begin{macro}{\@@_pdf_object_write_stream:nnnn}
%   This is where we choose the actual type.
%    \begin{macrocode}
\cs_new_protected:Npn \driver_pdf_object_write:nn #1#2
  {
    \exp_args:Nx \@@_pdf_object_write:nnn
      { \prop_item:Nn \g_@@_pdf_object_prop {#1} } {#1} {#2}
  }
\cs_new_protected:Npn \@@_pdf_object_write:nnn #1#2#3
  { \use:c { @@_pdf_object_write_ #1 :nn } {#2} {#3} }
\cs_new_protected:Npn \@@_pdf_object_write_array:nn #1#2
  {
    \@@_literal:x
      {
        pdf:obj ~ \driver_pdf_object_ref:n {#1} ~
          [ ~ \exp_not:n {#2} ~ ]
      }
  }
\cs_new_protected:Npn \@@_pdf_object_write_dict:nn #1#2
  {
    \@@_literal:x
      {
        pdf:obj ~ \driver_pdf_object_ref:n {#1} ~
          << ~ \exp_not:n {#2} ~ >>
      }
  }
\cs_new_protected:Npn \@@_pdf_object_write_fstream:nn #1#2
  { \@@_pdf_object_write_stream:nnnn { f } {#1} #2 }
\cs_new_protected:Npn \@@_pdf_object_write_stream:nn #1#2
  { \@@_pdf_object_write_stream:nnnn { } {#1} #2 }
\cs_new_protected:Npn \@@_pdf_object_write_stream:nnnn #1#2#3#4
  {
    \@@_literal:x
      {
        pdf: #1 stream ~ \driver_pdf_object_ref:n {#2} ~
          ( \exp_not:n {#4} ) ~ << \exp_not:n {#3} >>
        
      }
  }
%    \end{macrocode}
% \end{macro}
% \end{macro}
% \end{macro}
% \end{macro}
%
% \subsubsection{PDF Structure}
%
% \begin{macro}{\driver_pdf_compresslevel:n}
% \begin{macro}{\driver_pdf_objects_enable:, \driver_pdf_objects_disable:}
%   Pass data to the driver: these are a one-shot.
%    \begin{macrocode}
\cs_new_protected:Npn \driver_pdf_compresslevel:n #1
  { \@@_literal:x { dvipdfmx:config~z~ \int_eval:n {#1} } }
\cs_new_protected:Npn \driver_pdf_objects_enable: { }
\cs_new_protected:Npn \driver_pdf_objects_disable:
  { \@@_literal:n { dvipdfmx:config~C~0x40 } }
%    \end{macrocode}
% \end{macro}
% \end{macro}
%
%    \begin{macrocode}
%</dvipdfmx|xdvipdfmx>
%    \end{macrocode}
%
% \subsection{\texttt{xdvipdfmx} driver}
%
%    \begin{macrocode}
%<*xdvipdfmx>
%    \end{macrocode}
%
% \subsubsection{Images}
%
% \begin{macro}
%   {\@@_image_getbb_jpg:n, \@@_image_getbb_pdf:n, \@@_image_getbb_png:n}
% \begin{macro}{\@@_image_getbb_auxi:nN}
% \begin{macro}{\@@_image_getbb_auxii:nnN, \@@_image_getbb_auxii:VnN}
% \begin{macro}{\@@_image_getbb_auxiii:nNnn}
% \begin{macro}{\@@_image_getbb_auxiv:nnNnn, \@@_image_getbb_auxiv:VnNnn}
% \begin{macro}{\@@_image_getbb_auxv:nNnn, \@@_image_getbb_auxv:nNnn}
% \begin{macro}[EXP]{\@@_image_getbb_pagebox:w}
%   For \texttt{xdvipdfmx}, there are two primitives that allow us to obtain
%   the bounding box without needing \texttt{extractbb}. The only complexity
%   is passing the various minor variations to a common core process. The
%   \XeTeX{} primitive omits the text |box| from the page box specification,
%   so there is also some \enquote{trimming} to do here.
%    \begin{macrocode}
\cs_new_protected:Npn \@@_image_getbb_jpg:n #1
  {
    \int_zero:N \l_image_page_int
    \tl_clear:N \l_image_pagebox_tl
    \@@_image_getbb_auxi:nN {#1} \tex_XeTeXpicfile:D
  }
\cs_new_eq:NN \@@_image_getbb_png:n \@@_image_getbb_jpg:n
\cs_new_protected:Npn \@@_image_getbb_pdf:n #1
  {
    \tl_clear:N \l_image_decode_tl
    \bool_set_false:N \l_image_interpolate_bool
    \@@_image_getbb_auxi:nN {#1} \tex_XeTeXpdffile:D
  }
\cs_new_protected:Npn \@@_image_getbb_auxi:nN #1#2
  {
    \int_compare:nNnTF \l_image_page_int > 1
      { \@@_image_getbb_auxii:VnN \l_image_page_int {#1} #2  }
      { \@@_image_getbb_auxiii:nNnn {#1} #2 }
  }
\cs_new_protected:Npn \@@_image_getbb_auxii:nnN #1#2#3
  { \@@_image_getbb_aux:nNnn {#2} #3 { :P #1 } { page #1 } }
\cs_generate_variant:Nn \@@_image_getbb_auxii:nnN { V }
\cs_new_protected:Npn \@@_image_getbb_auxiii:nNnn #1#2#3#4
  {
    \tl_if_empty:NTF \l_image_pagebox_tl
      { \@@_image_getbb_auxiv:VnNnn \l_image_pagebox_tl }
      { \@@_image_getbb_auxv:nNnn }
      {#1} #2 {#3} {#4}
  }
\cs_new_protected:Npn \@@_image_getbb_auxiv:nnNnn #1#2#3#4#5
  {
    \use:x
      {
        \@@_image_getbb_auxv:nNnn {#2} #3 { : #1 #4 }
          { #5 ~ \@@_image_getbb_pagebox:w #1 }
      }
  }
\cs_generate_variant:Nn \@@_image_getbb_auxiv:nnNnn { V }
\cs_new_protected:Npn \@@_image_getbb_auxv:nNnn #1#2#3#4
  {
    \image_bb_restore:nF {#1#3}
      { \@@_image_getbb_auxvi:nNnn {#1} #2 {#3} {#4} }
  }
\cs_new_protected:Npn \@@_image_getbb_auxvi:nNnn #1#2#3#4
  {
    \hbox_set:Nn \l_@@_tmp_box { #2 #1 ~ #4 }
    \dim_set:Nn \l_image_utx_dim { \box_wd:N \l_@@_tmp_box }
    \dim_set:Nn \l_image_ury_dim { \box_ht:N \l_@@_tmp_box }
    \image_bb_save:n {#1#3}
  }
\cs_new:Npn \@@_image_getbb_pagebox:w #1 box {#1}
%    \end{macrocode}
% \end{macro}
% \end{macro}
% \end{macro}
% \end{macro}
% \end{macro}
% \end{macro}
% \end{macro}
%
% \begin{macro}{\@@_image_include_pdf:n}
%   For PDF images, properly supporting the |pagebox| concept in \XeTeX{}
%   is best done using the |\tex_XeTeXpdffile:D| primitive. The syntax here
%   is the same as for the image measurement part, although we know at this
%   stage that there must be some valid setting for \cs{l_image_pagebox_tl}.
%    \begin{macrocode}
\cs_new_protected:Npn \@@_image_include_pdf:n #1
  {
    \tex_XeTeXpdffile:D "#1" ~
      \int_compare:nNnT \l_image_page_int > 0
        { page~ \int_use:N \l_image_page_int }
      \@@_image_getbb_auxiv:VnNnn \l_image_pagebox_tl
  }
%    \end{macrocode}
% \end{macro}
%
%    \begin{macrocode}
%</xdvipdfmx>
%    \end{macrocode}
%
% \subsection{Drawing commands: \texttt{pdfmode} and \texttt{(x)dvipdfmx}}
%
% Both \texttt{pdfmode} and \texttt{(x)dvipdfmx} directly produce PDF output
% and understand a shared set of specials for drawing commands.
%
%    \begin{macrocode}
%<*dvipdfmx|pdfmode|xdvipdfmx>
%    \end{macrocode}
%
% \subsubsection{Drawing}
%
% \begin{macro}{\@@_draw_literal:n, \@@_draw_literal:x}
%   Pass data through using a dedicated interface.
%    \begin{macrocode}
\cs_new_eq:NN \@@_draw_literal:n \@@_literal_pdf:n
\cs_generate_variant:Nn \@@_draw_literal:n { x }
%    \end{macrocode}
% \end{macro}
%
% \begin{macro}{\driver_draw_begin:, \driver_draw_end:}
%   No special requirements here, so simply set up a drawing scope.
%    \begin{macrocode}
\cs_new_protected:Npn \driver_draw_begin:
  { \driver_draw_scope_begin: }
\cs_new_protected:Npn \driver_draw_end:
  { \driver_draw_scope_end: }
%    \end{macrocode}
% \end{macro}
%
% \begin{macro}{\driver_draw_scope_begin:, \driver_draw_scope_end:}
%   Use the driver-level scope mechanisms.
%    \begin{macrocode}
\cs_new_eq:NN \driver_draw_scope_begin: \@@_scope_begin:
\cs_new_eq:NN \driver_draw_scope_end: \@@_scope_end:
%    \end{macrocode}
% \end{macro}
%
% \begin{macro}{\driver_draw_moveto:nn, \driver_draw_lineto:nn}
% \begin{macro}{\driver_draw_curveto:nnnnnn}
% \begin{macro}{\driver_draw_rectangle:nnnn}
%   Path creation operations all resolve directly to PDF primitive steps, with
%   only the need to convert to \texttt{bp}.
%    \begin{macrocode}
\cs_new_protected:Npn \driver_draw_moveto:nn #1#2
  {
    \@@_draw_literal:x
      { \dim_to_decimal_in_bp:n {#1} ~ \dim_to_decimal_in_bp:n {#2} ~ m }
  }
\cs_new_protected:Npn \driver_draw_lineto:nn #1#2
  {
    \@@_draw_literal:x
      { \dim_to_decimal_in_bp:n {#1} ~ \dim_to_decimal_in_bp:n {#2} ~ l }
  }
\cs_new_protected:Npn \driver_draw_curveto:nnnnnn #1#2#3#4#5#6
  {
    \@@_draw_literal:x
      {
        \dim_to_decimal_in_bp:n {#1} ~ \dim_to_decimal_in_bp:n {#2} ~
        \dim_to_decimal_in_bp:n {#3} ~ \dim_to_decimal_in_bp:n {#4} ~
        \dim_to_decimal_in_bp:n {#5} ~ \dim_to_decimal_in_bp:n {#6} ~
        c
      }
 }
\cs_new_protected:Npn \driver_draw_rectangle:nnnn #1#2#3#4
  {
     \@@_draw_literal:x
      {
        \dim_to_decimal_in_bp:n {#1} ~ \dim_to_decimal_in_bp:n {#2} ~
        \dim_to_decimal_in_bp:n {#3} ~ \dim_to_decimal_in_bp:n {#4} ~
        re
      }
  }
%    \end{macrocode}
% \end{macro}
% \end{macro}
% \end{macro}
%
% \begin{macro}{\driver_draw_evenodd_rule:, \driver_draw_nonzero_rule:}
% \begin{variable}{\g_@@_draw_eor_bool}
%    The even-odd rule here can be implemented as a simply switch.
%    \begin{macrocode}
\cs_new_protected:Npn \driver_draw_evenodd_rule:
  { \bool_gset_true:N \g_@@_draw_eor_bool }
\cs_new_protected:Npn \driver_draw_nonzero_rule:
  { \bool_gset_false:N \g_@@_draw_eor_bool }
\bool_new:N \g_@@_draw_eor_bool
%    \end{macrocode}
% \end{variable}
% \end{macro}
%
% \begin{macro}
%   {
%     \driver_draw_closepath:   ,
%     \driver_draw_stroke:      ,
%     \driver_draw_closestroke: ,
%     \driver_draw_fill:        ,
%     \driver_draw_fillstroke:  ,
%     \driver_draw_clip:        ,
%     \driver_draw_discardpath:
%   }
%   Converting paths to output is again a case of mapping directly to
%   PDF operations.
%    \begin{macrocode}
\cs_new_protected:Npn \driver_draw_closepath:
  { \@@_draw_literal:n { h } }
\cs_new_protected:Npn \driver_draw_stroke:
  { \@@_draw_literal:n { S } }
\cs_new_protected:Npn \driver_draw_closestroke:
  { \@@_draw_literal:n { s } }
\cs_new_protected:Npn \driver_draw_fill:
  {
    \@@_draw_literal:x
      { f \bool_if:NT \g_@@_draw_eor_bool * }
  }
\cs_new_protected:Npn \driver_draw_fillstroke:
  {
    \@@_draw_literal:x
      { B \bool_if:NT \g_@@_draw_eor_bool * }
  }
\cs_new_protected:Npn \driver_draw_clip:
  {
    \@@_draw_literal:x
      { W \bool_if:NT \g_@@_draw_eor_bool * }
  }
\cs_new_protected:Npn \driver_draw_discardpath:
  { \@@_draw_literal:n { n } }
%    \end{macrocode}
% \end{macro}
%
% \begin{macro}{\driver_draw_dash_pattern:nn}
% \begin{macro}{\@@_draw_dash:n}
% \begin{macro}{\driver_draw_linewidth:n}
% \begin{macro}{\driver_draw_miterlimit:n}
% \begin{macro}
%   {
%     \driver_draw_cap_butt:, \driver_draw_cap_round:, \driver_draw_cap_rectangle:,
%     \driver_draw_join_miter:, \driver_draw_join_round:, \driver_draw_join_bevel:
%   }
%   Converting paths to output is again a case of mapping directly to
%   PDF operations.
%    \begin{macrocode}
\cs_new_protected:Npn \driver_draw_dash_pattern:nn #1#2
  {
    \@@_draw_literal:x
      {
        [
          \exp_args:Nf \use:n
            { \clist_map_function:nN {#1} \@@_draw_dash:n }
        ] ~
        \dim_to_decimal_in_bp:n {#2} ~ d
      }
  }
\cs_new:Npn \@@_draw_dash:n #1
  { ~ \dim_to_decimal_in_bp:n {#1} }
\cs_new_protected:Npn \driver_draw_linewidth:n #1
  {
    \@@_draw_literal:x
      { \dim_to_decimal_in_bp:n {#1} ~ w }
  }
\cs_new_protected:Npn \driver_draw_miterlimit:n #1
  { \@@_draw_literal:x { \fp_eval:n {#1} ~ M } }
\cs_new_protected:Npn \driver_draw_cap_butt:
  { \@@_draw_literal:n { 0 ~ J } }
\cs_new_protected:Npn \driver_draw_cap_round:
  { \@@_draw_literal:n { 1 ~ J } }
\cs_new_protected:Npn \driver_draw_cap_rectangle:
  { \@@_draw_literal:n { 2 ~ J } }
\cs_new_protected:Npn \driver_draw_join_miter:
  { \@@_draw_literal:n { 0 ~ j } }
\cs_new_protected:Npn \driver_draw_join_round:
  { \@@_draw_literal:n { 1 ~ j } }
\cs_new_protected:Npn \driver_draw_join_bevel:
  { \@@_draw_literal:n { 2 ~ j } }
%    \end{macrocode}
% \end{macro}
% \end{macro}
% \end{macro}
% \end{macro}
% \end{macro}
%
% \begin{macro}
%   {
%     \driver_draw_color_fill_cmyk:nnnn   ,
%     \driver_draw_color_stroke_cmyk:nnnn
%   }
% \begin{macro}
%   {
%     \driver_draw_color_fill_gray:n   ,
%     \driver_draw_color_stroke_gray:n
%   }
% \begin{macro}
%   {
%     \driver_draw_color_fill_rgb:nnn   ,
%     \driver_draw_color_stroke_rgb:nnn
%   }
% \begin{macro}{\@@_color_fill_select:n, \@@_color_fill_select:x}
%    For the stroke color, all engines here can use the color stack to handle
%    the setting. However, that is not the case for fill color: the stack in
%    \texttt{(x)dvipdfmx} only covers one type of color. So we have to use
%    different approaches for the two sets of engines.
%    \begin{macrocode}
\cs_new_protected:Npn \driver_draw_color_fill_cmyk:nnnn #1#2#3#4
  {
    \@@_color_fill_select:x
      {
        \fp_eval:n {#1} ~ \fp_eval:n {#2} ~ 
        \fp_eval:n {#3} ~ \fp_eval:n {#4} ~
        k
      }
  }
\cs_new_protected:Npn \driver_draw_color_stroke_cmyk:nnnn #1#2#3#4
  {
    \@@_color_select:x
      {
        \fp_eval:n {#1} ~ \fp_eval:n {#2} ~ 
        \fp_eval:n {#3} ~ \fp_eval:n {#4} ~
        k
      }
  }
\cs_new_protected:Npn \driver_draw_color_fill_gray:n #1
  { \@@_color_fill_select:x { \fp_eval:n {#1} ~ g } }
\cs_new_protected:Npn \driver_draw_color_stroke_gray:n #1
  { \@@_color_select:x { \fp_eval:n {#1} ~ G } }
\cs_new_protected:Npn \driver_draw_color_fill_rgb:nnn #1#2#3
  {
    \@@_color_fill_select:x
      { \fp_eval:n {#1} ~ \fp_eval:n {#2} ~ \fp_eval:n {#3} ~ rg }
  }
\cs_new_protected:Npn \driver_draw_color_stroke_rgb:nnn #1#2#3
  {
    \@@_color_select:x
      { \fp_eval:n {#1} ~ \fp_eval:n {#2} ~ \fp_eval:n {#3} ~ RG }
  }
%<*pdfmode>
\cs_new_eq:NN \@@_color_fill_select:n \@@_color_select:n
%</pdfmode>
%<*dvipdfmx|xdvipdfmx>
\cs_new_eq:NN \@@_color_fill_select:n \@@_draw_literal:n
%</dvipdfmx|xdvipdfmx>
\cs_generate_variant:Nn \@@_color_fill_select:n { x }
%    \end{macrocode}
% \end{macro}
% \end{macro}
% \end{macro}
% \end{macro}
%
% \begin{macro}{\driver_draw_cm:nnnn}
% \begin{macro}{\@@_draw_cm:nnnn}
%   Another split here between \texttt{pdfmode} and \texttt{(x)dvipdfmx}.
%   In the former, we have a direct method to maintain alignment: the driver
%   can use a matrix itself. For \texttt{(x)dvipdfmx}, we can to decompose the
%   matrix into rotations and a scaling, then use those operations as they
%   are handled by the driver. (There is driver support for matrix operations in
%   \texttt{(x)dvipdfmx}, but as a matched pair so not suitable for the
%   \enquote{stand alone} transformation set up here.)
%    \begin{macrocode}
\cs_new_protected:Npn \driver_draw_cm:nnnn #1#2#3#4
  {
%<*pdfmode>
    \@@_matrix:x
      {
        \fp_eval:n {#1} ~ \fp_eval:n {#2} ~
        \fp_eval:n {#3} ~ \fp_eval:n {#4}
      }
%</pdfmode>
%<*dvipdfmx|xdvipdfmx>
    \@@_draw_cm_decompose:nnnnN {#1} {#2} {#3} {#4}
      \@@_draw_cm:nnnn
%</dvipdfmx|xdvipdfmx>
  }
%<*dvipdfmx|xdvipdfmx>
\cs_new_protected:Npn \@@_draw_cm:nnnn #1#2#3#4
  {
    \@@_literal:x
      {
        x:rotate~
        \fp_compare:nNnTF {#1} = \c_zero_fp
          { 0 }
          { \fp_eval:n { round ( -#1 , 5 ) } }
      }
    \@@_literal:x
      {
        x:scale~
        \fp_eval:n { round ( #2 , 5 ) } ~
        \fp_eval:n { round ( #3 , 5 ) }
      }
    \@@_literal:x
      {
        x:rotate~
        \fp_compare:nNnTF {#4} = \c_zero_fp
          { 0 }
          { \fp_eval:n { round ( -#4 , 5 ) } }
      }
  }
%</dvipdfmx|xdvipdfmx>
%    \end{macrocode}
% \end{macro}
% \end{macro}
%
% \begin{macro}{\@@_draw_cm_decompose:nnnnN}
% \begin{macro}
%   {
%     \@@_draw_cm_decompose_auxi:nnnnN,
%     \@@_draw_cm_decompose_auxii:nnnnN,
%     \@@_draw_cm_decompose_auxiii:nnnnN,
%   }
%   Internally, transformations for drawing are tracked as a matrix. Not all
%   engines provide a way of dealing with this: if we use a raw matrix, the
%   engine looses track of positions (for example for hyperlinks), and this is
%   not desirable. They do, however, allow us to track rotations and scalings.
%   Luckily, we can decompose any (two-dimensional) matrix into two rotations
%   and a single scaling:
%   \[
%     \begin{bmatrix}
%         A & B \\ C & D
%     \end{bmatrix}
%     =
%     \begin{bmatrix}
%       \cos\beta & \sin\beta \\ -\sin\beta & \cos\beta
%     \end{bmatrix}
%     \begin{bmatrix}
%       w_{1} & 0 \\ 0 & w_{2}
%     \end{bmatrix}
%     \begin{bmatrix}
%       \cos\gamma & \sin\gamma \\ -\sin\gamma & \cos\gamma
%     \end{bmatrix} 
%   \]
%   The parent matrix can be converted to
%   \[
%     \begin{bmatrix}
%       A & B \\ C & D
%     \end{bmatrix}
%      =
%     \begin{bmatrix}
%       E & H \\-H & E
%     \end{bmatrix}
%     +
%     \begin{bmatrix}
%       F & G \\ G & -F
%     \end{bmatrix}
%   \]
%   From these, we can find that
%   \begin{align*}
%     \frac{w_{1} + w_{2}}{2} &= \sqrt{E^{2} + H^{2}} \\
%     \frac{w_{1} - w_{2}}{2} &= \sqrt{F^{2} + G^{2}} \\
%     \gamma - \beta &= \tan^{-1}(G/F) \\
%     \gamma + \beta &= \tan^{-1}(H/E)
%   \end{align*}
%   at which point we just have to do various pieces of re-arrangement to
%   get all of the values. (See J.~Blinn, \emph{IEEE Comput.\ Graph.\ Appl.},
%   1996, \textbf{16}, 82--88.) There is one wrinkle: the PostScript (and PDF)
%   way of specifying a transformation matrix exchanges where one would
%   normally expect $B$ and $C$ to be.
%    \begin{macrocode}
%<*dvipdfmx|xdvipdfmx>
\cs_new_protected:Npn \@@_draw_cm_decompose:nnnnN #1#2#3#4#5
  {
    \use:x
      {
        \@@_draw_cm_decompose_auxi:nnnnN
          { \fp_eval:n { (#1 + #4) / 2 } }
          { \fp_eval:n { (#1 - #4) / 2 } }
          { \fp_eval:n { (#3 + #2) / 2 } }
          { \fp_eval:n { (#3 - #2) / 2 } }
      }
        #5
  }
\cs_new_protected:Npn \@@_draw_cm_decompose_auxi:nnnnN #1#2#3#4#5
  {
    \use:x
      {
        \@@_draw_cm_decompose_auxii:nnnnN
          { \fp_eval:n { 2 * sqrt ( #1 * #1 + #4 * #4 ) } }
          { \fp_eval:n { 2 * sqrt ( #2 * #2 + #3 * #3 ) } }
          { \fp_eval:n { atand ( #3 , #2 ) } }
          { \fp_eval:n { atand ( #4 , #1 ) } }
      }
         #5
  }
\cs_new_protected:Npn \@@_draw_cm_decompose_auxii:nnnnN #1#2#3#4#5
  {
    \use:x
      {
        \@@_draw_cm_decompose_auxiii:nnnnN
          { \fp_eval:n { ( #4 - #3 ) / 2 } }
          { \fp_eval:n { ( #1 + #2 ) / 2 } }
          { \fp_eval:n { ( #1 - #2 ) / 2 } }
          { \fp_eval:n { ( #4 + #3 ) / 2 } }
      }
        #5
  }
\cs_new_protected:Npn \@@_draw_cm_decompose_auxiii:nnnnN #1#2#3#4#5
  {
    \fp_compare:nNnTF { abs( #2 ) } > { abs ( #3 ) }
      { #5 {#1} {#2} {#3} {#4} }
      { #5 {#1} {#3} {#2} {#4} }
  }
%</dvipdfmx|xdvipdfmx>
%    \end{macrocode}
% \end{macro}
% \end{macro}
%
% \begin{macro}{\driver_draw_box_use:Nnnnn}
%   Inserting a \TeX{} box transformed to the requested position and using
%   the current matrix is done using a mixture of \TeX{} and low-level
%   manipulation. The offset can be handled by \TeX{}, so only any rotation/^^A
%   skew/scaling component needs to be done using the matrix operation. As this
%   operation can never be cached, the scope is set directly not using the
%   \texttt{draw} version.
%    \begin{macrocode}
\cs_new_protected:Npn \driver_draw_box_use:Nnnnn #1#2#3#4#5
  {
    \@@_scope_begin:
%<*pdfmode>
    \driver_draw_cm:nnnn {#2} {#3} {#4} {#5}
%</pdfmode>
%<*dvipdfmx|xdvipdfmx>
    \@@_literal:x
      {
        pdf:btrans~matrix~
        \fp_eval:n {#2} ~ \fp_eval:n {#3} ~
        \fp_eval:n {#4} ~ \fp_eval:n {#5} ~
        0 ~ 0
      }
%</dvipdfmx|xdvipdfmx>
    \hbox_overlap_right:n { \box_use:N #1 }
%<*dvipdfmx|xdvipdfmx>
    \@@_literal:n { pdf:etrans }
%</dvipdfmx|xdvipdfmx>
    \@@_scope_end:
  }
%    \end{macrocode}
% \end{macro}
%
%    \begin{macrocode}
%</dvipdfmx|pdfmode|xdvipdfmx>
%    \end{macrocode}
%
% \subsection{\texttt{dvisvgm} driver}
%
%    \begin{macrocode}
%<*dvisvgm>
%    \end{macrocode}
%
% \subsubsection{Basics}
%
% \begin{macro}{\@@_literal_svg:n, \@@_literal_svg:x}
%   Unlike the other drivers, the requirements for making SVG files mean
%   that we can't conveniently transform all operations to the current point.
%   That makes life a bit more tricky later as that needs to be accounted for.
%   A new line is added after each call to help to keep the output readable
%   for debugging.
%    \begin{macrocode}
\cs_new_protected:Npn \@@_literal_svg:n #1
  { \@@_literal:n { dvisvgm:raw~ #1 { ?nl } } }
\cs_generate_variant:Nn \@@_literal_svg:n { x }
%    \end{macrocode}
% \end{macro}
%
% \begin{macro}{\@@_scope_begin:, \@@_scope_end:}
%   A scope in SVG terms is slightly different to the other drivers as
%   operations have to be \enquote{tied} to these not simply inside them.
%    \begin{macrocode}
\cs_new_protected:Npn \@@_scope_begin:
  { \@@_literal_svg:n { <g> } }
\cs_new_protected:Npn \@@_scope_end:
  { \@@_literal_svg:n { </g> } }
%    \end{macrocode}
% \end{macro}
%
% \subsubsection{Driver-specific auxiliaries}
%
% \begin{macro}{\@@_scope_begin:n, \@@_scope_begin:x}
%   In SVG transformations, clips and so on are attached directly to scopes so
%   we need a way or allowing for that. This is rather more useful than
%   \cs{@@_scope_begin:} as a result. No assumptions are made about the nature
%   of the scoped operation(s).
%    \begin{macrocode}
\cs_new_protected:Npn \@@_scope_begin:n #1
  { \@@_literal_svg:n { <g~ #1 > } }
\cs_generate_variant:Nn \@@_scope_begin:n { x }
%    \end{macrocode}
% \end{macro}
%
% \subsubsection{Box operations}
%
% \begin{macro}{\driver_box_use_clip:N}
% \begin{variable}{\g_@@_clip_path_int}
%   Clipping in SVG is more involved than with other drivers. The first issue
%   is that the clipping path must be defined separately from where it is used,
%   so we need to track how many paths have applied. The naming here uses
%   \texttt{l3cp} as the namespace with a number following. Rather than use
%   a rectangular operation, we define the path manually as this allows it to
%   have a depth: easier than the alternative approach of shifting content
%   up and down using scopes to allow for the depth of the \TeX{} box and
%   keep the reference point the same!
%    \begin{macrocode}
\cs_new_protected:Npn \driver_box_use_clip:N #1
  {
    \int_gincr:N \g_@@_clip_path_int
    \@@_literal_svg:x
      { < clipPath~id = " l3cp \int_use:N \g_@@_clip_path_int " > }
    \@@_literal_svg:x
      {
        <
          path ~ d =
            "
              M ~ 0 ~
                  \dim_to_decimal:n { -\box_dp:N #1 } ~
              L ~ \dim_to_decimal:n { \box_wd:N #1 } ~
                  \dim_to_decimal:n { -\box_dp:N #1 } ~
              L ~ \dim_to_decimal:n { \box_wd:N #1 }  ~
                  \dim_to_decimal:n { \box_ht:N #1 + \box_dp:N #1 } ~
              L ~ 0 ~
                  \dim_to_decimal:n { \box_ht:N #1 + \box_dp:N #1 } ~
              Z
            "
        />
      }
    \@@_literal_svg:n
      { < /clipPath > }
%    \end{macrocode}
%   In general the SVG set up does not try to transform coordinates to the
%   current point. For clipping we need to do that, so have a transformation
%   here to get us to the right place, and a matching one just before the
%   \TeX{} box is inserted to get things back on track. The clip path needs to
%   come between those two such that if lines up with the current point, as
%   does the \TeX{} box.
%    \begin{macrocode}
    \@@_scope_begin:n
      {
        transform =
          "
            translate ( { ?x } , { ?y } ) ~
            scale ( 1 , -1 )
          "
      }
    \@@_scope_begin:x
      {
        clip-path =
          "url ( \c_hash_str l3cp \int_use:N \g_@@_clip_path_int ) "
      }
    \@@_scope_begin:n
      {
        transform =
          "
            scale ( -1 , 1 ) ~
            translate ( { ?x } , { ?y } ) ~
            scale ( -1 , -1 )
          "
      }
    \box_use:N #1
    \@@_scope_end:
    \@@_scope_end:
    \@@_scope_end:
%    \skip_horizontal:n { \box_wd:N #1 }
  }
\int_new:N \g_@@_clip_path_int
%    \end{macrocode}
% \end{variable}
% \end{macro}
%
% \begin{macro}{\driver_box_use_rotate:Nn}
%   Rotation has a dedicated operation which includes a centre-of-rotation
%   optional pair. That can be picked up from the driver syntax, so there is
%   no need to worry about the transformation matrix.
%    \begin{macrocode}
\cs_new_protected:Npn \driver_box_use_rotate:Nn #1#2
  {
    \@@_scope_begin:x
      {
        transform =
          "
            rotate
            ( \fp_eval:n { round ( -(#2) , 5 ) } , ~ { ?x } , ~ { ?y } )
          "
      }
    \box_use:N #1
    \@@_scope_end:
  }
%    \end{macrocode}
% \end{macro}
%
% \begin{macro}{\driver_box_use_scale:Nnn}
%   In contrast to rotation, we have to account for the current position in this
%   case. That is done using a couple of translations in addition to the scaling
%   (which is therefore done backward with a flip).
%    \begin{macrocode}
\cs_new_protected:Npn \driver_box_use_scale:Nnn #1#2#3
  {
    \@@_scope_begin:x
      {
        transform =
          "
            translate ( { ?x } , { ?y } ) ~
            scale
              (
                \fp_eval:n { round ( -#2 , 5 ) } ,
                \fp_eval:n { round ( -#3 , 5 ) }
              ) ~
            translate ( { ?x } , { ?y } ) ~
            scale ( -1 )
          "
      }
    \hbox_overlap_right:n { \box_use:N #1 }
    \@@_scope_end:
  }
%    \end{macrocode}
% \end{macro}
%
% \subsubsection{Images}
%
% \begin{macro}{\@@_image_getbb_png:n, \@@_image_getbb_jpg:n}
%   These can be included by extracting the bounding box data.
%    \begin{macrocode}
\cs_new_eq:NN \@@_image_getbb_png:n \image_extract_bb:n
\cs_new_eq:NN \@@_image_getbb_jpg:n \image_extract_bb:n
%    \end{macrocode}
% \end{macro}
%
% \begin{macro}{\@@_image_include_png:n, \@@_image_include_jpg:n}
% \begin{macro}{\@@_image_include_bitmap_quote:w}
%   The driver here has built-in support for basic image inclusion (see
%   \texttt{dvisvgm.def} for a more complex approach, needed if clipping,
%   \emph{etc.}, is covered at the image driver level). The only issue is
%   that |#1| must be quote-corrected. The \texttt{dvisvgm:img} operation
%   quotes the file name, but if it is already quoted (contains spaces)
%   then we have an issue: we simply strip off any quotes as a result.
%    \begin{macrocode}
\cs_new_protected:Npn \@@_image_include_png:n #1
  {
     \@@_literal:x
       {
         dvisvgm:img~
         \dim_to_decimal:n { \l_image_ury_dim } ~
         \dim_to_decimal:n { \l_image_ury_dim } ~
         \@@_image_include_bitmap_quote:w #1 " " \q_stop
       }
  }
\cs_new_eq:NN \@@_image_include_jpg:n \@@_image_include_png:n
\cs_new:Npn \@@_image_include_bitmap_quote:w #1 " #2 " #3 \q_stop
  { #1#2 }
%    \end{macrocode}
% \end{macro}
% \end{macro}
%
% \subsubsection{PDF Features}
%
% \begin{macro}{\driver_pdf_object_new:n}
% \begin{macro}[EXP]{\driver_pdf_object_ref:n}
% \begin{macro}{\driver_pdf_ojbect_write:nn}
% \begin{macro}{\driver_pdf_compresslevel:n}
% \begin{macro}{\driver_pdf_objects_enable:, \driver_pdf_objects_disable:}
%   These are all no-ops.
%    \begin{macrocode}
\cs_new_protected:Npn \driver_pdf_object_new:n #1 { }
\cs_new:Npn \driver_pdf_object_ref:n #1 { }
\cs_new_protected:Npn \driver_pdf_object_write:nn #1#2 { }
\cs_new_protected:Npn \driver_pdf_compresslevel:n #1 { }
\cs_new_protected:Npn \driver_pdf_objects_enable: { }
\cs_new_protected:Npn \driver_pdf_objects_disable: { }
%    \end{macrocode}
% \end{macro}
% \end{macro}
% \end{macro}
% \end{macro}
% \end{macro}
%
% \subsubsection{Drawing}
%
% \begin{macro}{\@@_draw_literal:n, \@@_draw_literal:x}
%   The same as the more general literal call.
%    \begin{macrocode}
\cs_new_eq:NN \@@_draw_literal:n \@@_literal_svg:n
\cs_generate_variant:Nn \@@_draw_literal:n { x }
%    \end{macrocode}
% \end{macro}
%
% \begin{macro}{\driver_draw_begin:, \driver_draw_end:}
%   A drawing needs to be set up such that the co-ordinate system is
%   translated. That is done inside a scope, which as described below
%    \begin{macrocode}
\cs_new_protected:Npn \driver_draw_begin:
  {
    \driver_draw_scope_begin:
    \@@_draw_scope:n { transform="translate({?x},{?y})~scale(1,-1)" }
  }
\cs_new_protected:Npn \driver_draw_end:
  { \driver_draw_scope_end: }
%    \end{macrocode}
% \end{macro}
%
% \begin{macro}{\driver_draw_scope_begin:, \driver_draw_scope_end:}
% \begin{macro}{\@@_draw_scope:n, \@@_draw_scope:x}
% \begin{variable}{\g_@@_draw_scope_int, \l_@@_draw_scope_int}
%   Several settings that with other drivers are \enquote{stand alone} have
%   to be given as part of a scope in SVG. As a result, there is a need to
%   provide a mechanism to automatically close these extra scopes. That is
%   done using a dedicated function and a pair of tracking variables. Within
%   each graphics scope we use a global variable to do the work, with a group
%   used to save the value between scopes. The result is that no direct action
%   is needed when creating a scope.
%    \begin{macrocode}
\cs_new_protected:Npn \driver_draw_scope_begin:
  {
    \int_set_eq:NN
      \l_@@_draw_scope_int
      \g_@@_draw_scope_int
    \group_begin:
      \int_gzero:N \g_@@_draw_scope_int
  }
\cs_new_protected:Npn \driver_draw_scope_end:
  {
      \prg_replicate:nn
        { \g_@@_draw_scope_int }
        { \@@_draw_literal:n { </g> } }
    \group_end:
    \int_gset_eq:NN
      \g_@@_draw_scope_int
      \l_@@_draw_scope_int
  }
\cs_new_protected:Npn \@@_draw_scope:n #1
  {
    \@@_draw_literal:n { <g~ #1 > }
    \int_gincr:N \g_@@_draw_scope_int
  }
\cs_generate_variant:Nn \@@_draw_scope:n { x }
\int_new:N \g_@@_draw_scope_int
\int_new:N \l_@@_draw_scope_int
%    \end{macrocode}
% \end{variable}
% \end{macro}
% \end{macro}
%
% \begin{macro}{\driver_draw_moveto:nn, \driver_draw_lineto:nn}
% \begin{macro}{\driver_draw_rectangle:nnnn}
% \begin{macro}{\driver_draw_curveto:nnnnnn}
% \begin{macro}{\@@_draw_add_to_path:n}
% \begin{variable}{\g_@@_draw_path_tl}
%   Once again, some work is needed to get path constructs correct. Rather
%   then write the values as they are given, the entire path needs to be
%   collected up before being output in one go. For that we use a dedicated
%   storage routine, which adds spaces as required. Since paths should
%   be fully expanded there is no need to worry about the internal
%   \texttt{x}-type expansion.
%    \begin{macrocode}
\cs_new_protected:Npn \driver_draw_moveto:nn #1#2
  {
    \@@_draw_add_to_path:n
      { M ~ \dim_to_decimal:n {#1} ~ \dim_to_decimal:n {#2} }
  }
\cs_new_protected:Npn \driver_draw_lineto:nn #1#2
  {
    \@@_draw_add_to_path:n
      { L ~ \dim_to_decimal:n {#1} ~ \dim_to_decimal:n {#2} }
  }
\cs_new_protected:Npn \driver_draw_rectangle:nnnn #1#2#3#4
  {
    \@@_draw_add_to_path:n
      {
        M ~ \dim_to_decimal:n {#1} ~ \dim_to_decimal:n {#2}
        h ~ \dim_to_decimal:n {#3} ~
        v ~ \dim_to_decimal:n {#4} ~
        h ~ \dim_to_decimal:n { -#3 } ~
        Z
      }
  }
\cs_new_protected:Npn \driver_draw_curveto:nnnnnn #1#2#3#4#5#6
  {
    \@@_draw_add_to_path:n
      {
        C ~
        \dim_to_decimal:n {#1} ~ \dim_to_decimal:n {#2} ~
        \dim_to_decimal:n {#3} ~ \dim_to_decimal:n {#4} ~
        \dim_to_decimal:n {#5} ~ \dim_to_decimal:n {#6}
      }
  }
\cs_new_protected:Npn \@@_draw_add_to_path:n #1
  {
    \tl_gset:Nx \g_@@_draw_path_tl
      {
        \g_@@_draw_path_tl
        \tl_if_empty:NF \g_@@_draw_path_tl { \c_space_tl }
        #1
      }
  }
\tl_new:N \g_@@_draw_path_tl
%    \end{macrocode}
% \end{variable}
% \end{macro}
% \end{macro}
% \end{macro}
% \end{macro}
%
% \begin{macro}{\driver_draw_evenodd_rule:, \driver_draw_nonzero_rule:}
%   The fill rules here have to be handled as scopes.
%    \begin{macrocode}
\cs_new_protected:Npn \driver_draw_evenodd_rule:
  { \@@_draw_scope:n { fill-rule="evenodd" } }
\cs_new_protected:Npn \driver_draw_nonzero_rule:
  { \@@_draw_scope:n { fill-rule="nonzero" } }
%    \end{macrocode}
% \end{macro}
%
% \begin{macro}{\@@_draw_path:n}
% \begin{macro}
%   {
%     \driver_draw_closepath:   ,
%     \driver_draw_stroke:      ,
%     \driver_draw_closestroke: ,
%     \driver_draw_fill:        ,
%     \driver_draw_fillstroke:  ,
%     \driver_draw_clip:        ,
%     \driver_draw_discardpath:
%   }
% \begin{variable}{\g_@@_draw_clip_bool}
% \begin{variable}{\g_@@_draw_path_int}
%   Setting fill and stroke effects and doing clipping all has to be done using
%   scopes. This means setting up the various requirements in a shared
%   auxiliary which deals with the bits and pieces. Clipping paths are reused
%   for path drawing: not essential but avoids constructing them twice.
%   Discarding a path needs a separate function as it's not quite the same.
%    \begin{macrocode}
\cs_new_protected:Npn \driver_draw_closepath:
  { \@@_draw_add_to_path:n { Z } }
\cs_new_protected:Npn \@@_draw_path:n #1
  {
    \bool_if:NTF \g_@@_draw_clip_bool
      {
        \int_gincr:N \g_@@_clip_path_int
        \@@_draw_literal:x
          {
            < clipPath~id = " l3cp \int_use:N \g_@@_clip_path_int " >
              { ?nl }
            <path~d=" \g_@@_draw_path_tl "/> { ?nl }
            < /clipPath > { ? nl }
            <
              use~xlink:href =
                "\c_hash_str l3path \int_use:N \g_@@_path_int " ~
                #1
            />
          }
        \@@_draw_scope:x
          {
            clip-path =
              "url( \c_hash_str l3cp \int_use:N \g_@@_clip_path_int)"
          }
      }
      {
        \@@_draw_literal:x
          { <path ~ d=" \g_@@_draw_path_tl " ~ #1 /> }
      }
    \tl_gclear:N \g_@@_draw_path_tl
    \bool_gset_false:N \g_@@_draw_clip_bool
  }
\int_new:N \g_@@_path_int
\cs_new_protected:Npn \driver_draw_stroke:
  { \@@_draw_path:n { style="fill:none" } }
\cs_new_protected:Npn \driver_draw_closestroke:
  {
    \driver_draw_closepath:
    \driver_draw_stroke:
  }
\cs_new_protected:Npn \driver_draw_fill:
  { \@@_draw_path:n { style="stroke:none" } }
\cs_new_protected:Npn \driver_draw_fillstroke:
  { \@@_draw_path:n { } }
\cs_new_protected:Npn \driver_draw_clip:
  { \bool_gset_true:N \g_@@_draw_clip_bool }
\bool_new:N \g_@@_draw_clip_bool
\cs_new_protected:Npn \driver_draw_discardpath:
  {
    \bool_if:NT \g_@@_draw_clip_bool
      {
        \int_gincr:N \g_@@_clip_path_int
        \@@_draw_literal:x
          {
            < clipPath~id = " l3cp \int_use:N \g_@@_clip_path_int " >
              { ?nl }
            <path~d=" \g_@@_draw_path_tl "/> { ?nl }
            < /clipPath >
          }
        \@@_draw_scope:x
          {
            clip-path =
              "url( \c_hash_str l3cp \int_use:N \g_@@_clip_path_int)"
          }
      }
    \tl_gclear:N \g_@@_draw_path_tl
    \bool_gset_false:N \g_@@_draw_clip_bool
  }
%    \end{macrocode}
% \end{variable}
% \end{variable}
% \end{macro}
% \end{macro}
%
% \begin{macro}{\driver_draw_dash_pattern:nn}
% \begin{macro}{\@@_draw_dash:n}
% \begin{macro}{\@@_draw_dash_aux:nn}
% \begin{macro}{\driver_draw_linewidth:n}
% \begin{macro}{\driver_draw_miterlimit:n}
% \begin{macro}
%   {
%     \driver_draw_cap_butt:, \driver_draw_cap_round:, \driver_draw_cap_rectangle:,
%     \driver_draw_join_miter:, \driver_draw_join_round:, \driver_draw_join_bevel:
%   }
%   All of these ideas are properties of scopes in SVG. The only slight
%   complexity is converting the dash array properly (doing any required
%   maths).
%    \begin{macrocode}
\cs_new_protected:Npn \driver_draw_dash_pattern:nn #1#2
  {
    \use:x
      {
        \@@_draw_dash_aux:nn
          { \clist_map_function:nn {#1} \@@_draw_dash:n }
          { \dim_to_decimal:n {#2} }
      }
  }
\cs_new:Npn \@@_draw_dash:n #1
  { , \dim_to_decimal_in_bp:n {#1} }
\cs_new_protected:Npn \@@_draw_dash_aux:nn #1#2
  {
    \@@_draw_scope:x
      {
        stroke-dasharray =
          "
            \tl_if_empty:oTF { \use_none:n #1 }
              { none }
              { \use_none:n #1 }
          " ~
          stroke-offset=" #2 "
      }
  }
\cs_new_protected:Npn \driver_draw_linewidth:n #1
  { \@@_draw_scope:x { stroke-width=" \dim_to_decimal:n {#1} " } }
\cs_new_protected:Npn \driver_draw_miterlimit:n #1
  { \@@_draw_scope:x { stroke-miterlimit=" \fp_eval:n {#1} " } }
\cs_new_protected:Npn \driver_draw_cap_butt:
  { \@@_draw_scope:n { stroke-linecap="butt" } }
\cs_new_protected:Npn \driver_draw_cap_round:
  { \@@_draw_scope:n { stroke-linecap="round" } }
\cs_new_protected:Npn \driver_draw_cap_rectangle:
  { \@@_draw_scope:n { stroke-linecap="square" } }
\cs_new_protected:Npn \driver_draw_join_miter:
  { \@@_draw_scope:n { stroke-linejoin="miter" } }
\cs_new_protected:Npn \driver_draw_join_round:
  { \@@_draw_scope:n { stroke-linejoin="round" } }
\cs_new_protected:Npn \driver_draw_join_bevel:
  { \@@_draw_scope:n { stroke-linejoin="bevel" } }
%    \end{macrocode}
% \end{macro}
% \end{macro}
% \end{macro}
% \end{macro}
% \end{macro}
% \end{macro}
%
% \begin{macro}
%   {
%     \driver_draw_color_fill_cmyk:nnnn   ,
%     \driver_draw_color_stroke_cmyk:nnnn
%   }
% \begin{macro}
%   {
%     \driver_draw_color_fill_gray:n   ,
%     \driver_draw_color_stroke_gray:n
%   }
% \begin{macro}
%   {
%     \driver_draw_color_fill_rgb:nnn   ,
%     \driver_draw_color_stroke_rgb:nnn
%   }
% \begin{macro}{\@@_draw_color_fill:nnn}
%  SVG fill color has to be covered outside of the stack, as for
%  \texttt{dvips}. Here, we are only allowed RGB colors so there is some
%  conversion to do.
%    \begin{macrocode}
\cs_new_protected:Npn \driver_draw_color_stroke_cmyk:nnnn #1#2#3#4
  {
    \use:x
      {
        \@@_draw_color_fill:nnn
          { \fp_eval:n { -100 * ( (#1) * ( 1 - (#4) ) - 1 ) } }
          { \fp_eval:n { -100 * ( (#2) * ( 1 - (#4) ) + #4 - 1 ) } }
          { \fp_eval:n { -100 * ( (#3) * ( 1 - (#4) ) + #4 - 1 ) } }
      }
  }
\cs_new_eq:NN \driver_draw_color_stroke_cmyk:nnnn \driver_color_cmyk:nnnn
\cs_new_protected:Npn \driver_draw_color_gray:n #1
  {
    \use:x
      {
        \@@_draw_color_gray_aux:n
          { \fp_eval:n { 100 * (#3) } }
      }
  }
\cs_new_protected:Npn \@@_draw_color_gray_aux:n #1
  { \@@_draw_color_fill:nnn {#1} {#1} {#1} }
\cs_new_eq:NN \driver_draw_color_stroke_gray:n \driver_color_gray:n
\cs_new_protected:Npn \driver_draw_color_rgb:nnn #1#2#3
  {
    \use:x
      {
        \@@_draw_color_fill:nnn
          { \fp_eval:n { 100 * (#1) } }
          { \fp_eval:n { 100 * (#2) } }
          { \fp_eval:n { 100 * (#3) } }
      }
  }
\cs_new_protected:Npn \@@_draw_color_fill:nnn #1#2#3
  {
    \@@_draw_scope:x
      {
        fill =
         "
           rgb
             (
               #1 \c_percent_str ,
               #2 \c_percent_str ,
               #3 \c_percent_str
             )
         "
      }
  }
\cs_new_eq:NN \driver_draw_color_stroke_rgb:nnn \driver_color_rgb:nnn
%    \end{macrocode}
% \end{macro}
% \end{macro}
% \end{macro}
% \end{macro}
%
% \begin{macro}{\driver_draw_cm:nnnn}
%   The four arguments here are floats (the affine matrix), the last
%   two are a displacement vector. 
%    \begin{macrocode}
\cs_new_protected:Npn \driver_draw_cm:nnnn #1#2#3#4
  {
    \@@_draw_scope:n
      {
       transform =
         "
           matrix
             (
               \fp_eval:n {#1} , \fp_eval:n {#2} ,
               \fp_eval:n {#3} , \fp_eval:n {#4} ,
               0pt , 0pt
             )
         "
      }
  }
%    \end{macrocode}
% \end{macro}
%
% \begin{macro}{\driver_draw_box_use:Nnnnn}
%   No special savings can be made here: simply displace the box inside
%   a scope. As there is nothing to re-box, just make the box passed of
%   zero size.
%    \begin{macrocode}
\cs_new_protected:Npn \driver_draw_box_use:Nnnnn #1#2#3#4#5#6#7
  {
    \@@_scope_begin:
    \driver_draw_cm:nnnn {#2} {#3} {#4} {#5}
    \@@_literal_svg:n
      {
        < g~
            stroke="none"~
            transform="scale(-1,1)~translate({?x},{?y})~scale(-1,-1)"
        >
      }
    \box_set_wd:Nn #1 { 0pt }
    \box_set_ht:Nn #1 { 0pt }
    \box_set_dp:Nn #1 { 0pt }
    \box_use:N #1
    \@@_literal_svg:n { </g> }
    \@@_scope_end:
  }
%    \end{macrocode}
% \end{macro}
%
%    \begin{macrocode}
%</dvisvgm>
%    \end{macrocode}
%
%    \begin{macrocode}
%</initex|package>
%    \end{macrocode}
%
% \end{implementation}
%
% \PrintIndex
