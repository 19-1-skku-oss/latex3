% \iffalse meta-comment
%
%% File: l3drivers.dtx Copyright(C) 2011-2016 The LaTeX3 Project
%%
%% It may be distributed and/or modified under the conditions of the
%% LaTeX Project Public License (LPPL), either version 1.3c of this
%% license or (at your option) any later version.  The latest version
%% of this license is in the file
%%
%%    http://www.latex-project.org/lppl.txt
%%
%% This file is part of the "l3kernel bundle" (The Work in LPPL)
%% and all files in that bundle must be distributed together.
%%
%% The released version of this bundle is available from CTAN.
%%
%% -----------------------------------------------------------------------
%%
%% The development version of the bundle can be found at
%%
%%    http://www.latex-project.org/svnroot/experimental/trunk/
%%
%% for those people who are interested.
%%
%%%%%%%%%%%
%% NOTE: %%
%%%%%%%%%%%
%%
%%   Snapshots taken from the repository represent work in progress and may
%%   not work or may contain conflicting material!  We therefore ask
%%   people _not_ to put them into distributions, archives, etc. without
%%   prior consultation with the LaTeX Project Team.
%%
%% -----------------------------------------------------------------------
%%
%
%<*driver>
\documentclass[full]{l3doc}
%</driver>
%<*driver|package>
\GetIdInfo$Id$
  {L3 Experimental drivers}
%</driver|package>
%<*driver>
\begin{document}
  \DocInput{\jobname.dtx}
\end{document}
%</driver>
% \fi
%
% \title{^^A
%   The \textsf{l3drivers} package\\ Drivers^^A
%   \thanks{This file describes v\ExplFileVersion,
%     last revised \ExplFileDate.}^^A
% }
%
% \author{^^A
%  The \LaTeX3 Project\thanks
%    {^^A
%      E-mail:
%        \href{mailto:latex-team@latex-project.org}
%          {latex-team@latex-project.org}^^A
%    }^^A
% }
%
% \date{Released \ExplFileDate}
%
% \maketitle
%
% \begin{documentation}
%
% \TeX{} relies on drivers in order to carry out a number of tasks, such
% as using color, including graphics and setting up hyper-links. The nature
% of the code required depends on the exact driver in use. Currently,
% \LaTeX3 is aware of the following drivers:
% \begin{itemize}
%   \item \texttt{pdfmode}:  The \enquote{driver} for direct PDF output by
%     \emph{both} \pdfTeX{} and \LuaTeX{} (no separate driver is used in this
%     case: the engine deals with PDF creation itself).
%   \item \texttt{dvips}: The \texttt{dvips} program, which works in
%     conjugation with \pdfTeX{} or \LuaTeX{} in DVI mode.
%   \item \texttt{dvipdfmx}: The \texttt{dvipdfmx} program, which works in
%     conjugation with \pdfTeX{} or \LuaTeX{} in DVI mode.
%   \item \texttt{dvisvgm}:  The \texttt{dvisvgm} program, which works in
%     conjugation with \pdfTeX{} or \LuaTeX{} in DVI mode to create SVG
%     output.
%   \item \texttt{xdvipdfmx}: The driver used by \XeTeX{}.
% \end{itemize}
%
% The code here is all very low-level, and should not in general be used
% outside of the kernel. It is also important to note that many of the
% functions here are closely tied to the immediate level \enquote{up},
% and they must be used in the correct contexts.d
%
% \section{Box clipping}
%
% \begin{function}[added = 2011-11-11]{\__driver_box_use_clip:N}
%   \begin{syntax}
%     \cs{__driver_box_use_clip:N} \meta{box}
%   \end{syntax}
%   Inserts the content of the \meta{box} at the current insertion point
%   such that any material outside of the bounding box will not be displayed
%   by the driver. The material in the \meta{box} is still placed in the
%   output stream: the clipping takes place at a driver level.
%
%   This function should only be used within a surrounding horizontal
%   box construct.
% \end{function}
%
% \section{Box rotation and scaling}
%
% \begin{function}[added = 2016-05-12]{\__driver_box_use_rotate:Nn}
%   \begin{syntax}
%     \cs{__driver_box_use_rotate:Nn} \meta{box} \Arg{angle}
%   \end{syntax}
%   Inserts the content of the \meta{box} at the current insertion point
%   rotated by the \meta{angle} (expressed in degrees). The material is
%   inserted with no apparent height or width, and is rotated such the
%   the \TeX{} reference point of the box is the center of rotation and
%   remains the reference point after rotation. It is the responsibly of
%   the code using this function to adjust the apparent size of the box to
%   be correct at the \TeX{} side.
%
%   This function should only be used within a surrounding horizontal
%   box construct.
% \end{function}
%
% \begin{function}[added = 2016-05-12]{\__driver_box_use_scale:Nnn}
%   \begin{syntax}
%     \cs{__driver_box_use_scale:Nnn} \meta{box} \Arg{x-scale} \Arg{y-scale}
%   \end{syntax}
%   Inserts the content of the \meta{box} at the current insertion point
%   scale by the \meta{x-scale} and \meta{y-scale}. The material is
%   inserted with no apparent height or width. It is the responsibly of
%   the code using this function to adjust the apparent size of the box to
%   be correct at the \TeX{} side.
%
%   This function should only be used within a surrounding horizontal
%   box construct.
% \end{function}
%
% \section{Color support}
%
% \begin{function}[added = 2011-09-03, updated = 2012-05-18]
%   {\__driver_color_ensure_current:}
%   \begin{syntax}
%     \cs{__driver_color_ensure_current:}
%   \end{syntax}
%   Ensures that the color used to typeset material is that which was
%   set when the material was placed in a box. This function is therefore
%   required inside any \enquote{color safe} box to ensure that the box may
%   be inserted in a location where the foreground color has been altered,
%   while preserving the color used in the box.
% \end{function}
%
% \end{documentation}
%
% \begin{implementation}
%
% \section{\pkg{l3drivers} Implementation}
%
%    \begin{macrocode}
%<*initex|package>
%<@@=driver>
%    \end{macrocode}
%
% Whilst there is a reasonable amount of code overlap between drivers,
% it is much clearer to have the blocks more-or-less separated than run
% in together and DocStripped out in parts. As such, most of the following
% is set up on a per-driver basis, though there is some common code (again
% given in blocks not interspersed with other material).
%
% All the file identifiers are up-front so that they come out in the right
% place in the files.
%    \begin{macrocode}
%<*package>
\ProvidesExplFile
%<*dvipdfmx>
  {l3dvidpfmx.def}{\ExplFileDate}{\ExplFileVersion}
  {L3 Experimental driver: dvipdfmx}
%</dvipdfmx>
%<*dvips>
  {l3dvips.def}{\ExplFileDate}{\ExplFileVersion}
  {L3 Experimental driver: dvips}
%</dvips>
%<*dvisvgm>
  {l3dvisvgm.def}{\ExplFileDate}{\ExplFileVersion}
  {L3 Experimental driver: dvisvgm}
%</dvisvgm>
%<*pdfmode>
  {l3pdfmode.def}{\ExplFileDate}{\ExplFileVersion}
  {L3 Experimental driver: PDF mode}
%</pdfmode>
%<*xdvipdfmx>
  {l3xdvidpfmx.def}{\ExplFileDate}{\ExplFileVersion}
  {L3 Experimental driver: xdvipdfmx}
%</xdvipdfmx>
%</package>
%    \end{macrocode}
%
% \subsection{\texttt{pdfmode} driver}
%
%    \begin{macrocode}
%<*pdfmode>
%    \end{macrocode}
%
% The direct PDF driver covers both \pdfTeX{} and \LuaTeX{}. The latter
% renames/restructures the driver primitives but this can be handled
% at one level of abstraction. As such, we avoid using two separate drivers
% for this material at the cost of some \texttt{x}-type definitions to get
% everything expanded up-front.
%
% \subsubsection{Basics}
%
% \begin{macro}[int]{\@@_literal:n}
%   This is equivalent to \verb|\special{pdf:}| but the engine can
%   track it. Without the \texttt{direct} keyword everything is kept in
%   sync: the transformation matrix is set to the current point automatically.
%   Note that this is still inside the text (\texttt{BT} \dots \texttt{ET}
%   block).
%    \begin{macrocode}
\cs_new_protected:Npx \@@_literal:n #1
  {
    \cs_if_exist:NTF \luatex_pdfextension:D
      { \luatex_pdfextension:D literal }
      { \pdftex_pdfliteral:D }
        {#1}
  }
%    \end{macrocode}
% \end{macro}
%
% \begin{macro}[int]{\@@_scope_begin:, \@@_scope_end:}
%  Higher-level interfaces for saving and restoring the graphic state.
%    \begin{macrocode}
\cs_new_protected_nopar:Npx \@@_scope_begin:
  {
    \cs_if_exist:NTF \luatex_pdfextension:D
      { \luatex_pdfextension:D save \scan_stop: }
      { \pdftex_pdfsave:D }
  }
\cs_new_protected_nopar:Npx \@@_scope_end:
  {
    \cs_if_exist:NTF \luatex_pdfextension:D
      { \luatex_pdfextension:D restore \scan_stop: }
      { \pdftex_pdfrestore:D }
  }
%    \end{macrocode}
% \end{macro}
%
% \begin{macro}[int]{\@@_matrix:n}
%   Here the appropriate function is set up to insert an affine matrix
%   into the PDF. With \pdfTeX{} and \LuaTeX{} in direct PDF output mode there
%   is a primitive for this, which only needs the rotation/scaling/skew part.
%    \begin{macrocode}
\cs_new_protected:Npx \@@_matrix:n #1
  {
    \cs_if_exist:NTF \luatex_pdfextension:D
      { \luatex_pdfextension:D setmatrix }
      { \pdftex_pdfsetmatrix:D }
        {#1}
  }
%    \end{macrocode}
% \end{macro}
%
% \subsubsection{Color}
%
% \begin{variable}{\l_@@_current_color_tl}
%   The current color in driver-dependent format: pick up the package-mode
%   data if available.
%    \begin{macrocode}
\tl_new:N \l_@@_current_color_tl
\tl_set:Nn \l_@@_current_color_tl { 0~g~0~G }
%<*package>
\AtBeginDocument
  {
    \@ifpackageloaded { color }
      { \tl_set:Nn \l_@@_current_color_tl { \current@color } }
      { }
  }
%</package>
%    \end{macrocode}
% \end{variable}
%
% \begin{variable}{\l_@@_color_stack_int}
%   \pdfTeX{} and \LuaTeX{} have multiple stacks available, and to track
%   which one is in use a variable is required.
%    \begin{macrocode}
\int_new:N \l_@@_color_stack_int
%    \end{macrocode}
% \end{variable}
%
% \begin{macro}[int]{\@@_color_ensure_current:}
% \begin{macro}[aux]{\@@_color_reset:}
%   There is a dedicated primitive/primitive interface for setting colors.
%   As with scoping, this approach is not suitable for cached operations.
%    \begin{macrocode}
\cs_new_protected_nopar:Npx \@@_color_ensure_current:
  {
    \cs_if_exist:NTF \luatex_pdfextension:D
      { \luatex_pdfextension:D colorstack }
      { \pdftex_pdfcolorstack:D }
        \exp_not:N \l_@@_color_stack_int push
          { \exp_not:N \l_@@_current_color_tl }
    \group_insert_after:N \exp_not:N \@@_color_reset:
  }
\cs_new_protected_nopar:Npx \@@_color_reset:
  {
    \cs_if_exist:NTF \luatex_pdfextension:D
      { \luatex_pdfextension:D colorstack }
      { \pdftex_pdfcolorstack:D }
        \exp_not:N \l_@@_color_stack_int pop \scan_stop:
  }
%    \end{macrocode}
% \end{macro}
% \end{macro}
%
%    \begin{macrocode}
%</pdfmode>
%    \end{macrocode}
%
% \subsection{\texttt{dvipdfmx} driver}
%
%    \begin{macrocode}
%<*dvipdfmx|xdvipdfmx>
%    \end{macrocode}
%
% The \texttt{dvipdfmx} shares code with the PDF mode one (using the common
% section to this file) but also with \texttt{xdvipdfmx}. The latter is close
% to identical to \texttt{dvipdfmx} and so all of the code here is extracted
% for both drivers, with some \texttt{clean up} for \texttt{xdvipdfmx} as
% required.
%
% \subsubsection{Basics}
%
% \begin{macro}[int]{\@@_literal:n}
%   Equivalent to \texttt{pdf:content} but favoured as the link to
%   the \pdfTeX{} primitive approach is clearer.
%    \begin{macrocode}
\cs_new_protected:Npn \@@_literal:n #1
  { \tex_special:D { pdf:literal~ #1 } }
%    \end{macrocode}
% \end{macro}
%
% \begin{macro}[int]{\@@_scope_begin:, \@@_scope_end:}
%   Scoping is done using direct PDF operations here.
%    \begin{macrocode}
\cs_new_protected_nopar:Npn \@@_scope_begin:
  { \@@_literal:n { q } }
\cs_new_protected_nopar:Npn \@@_scope_end:
  { \@@_literal:n { Q } }
%    \end{macrocode}
% \end{macro}
%
% \begin{macro}[int]{\@@_matrix:n}
%   With \texttt{(x)dvipdfmx} the matrix has to include a translation
%   part: that is always zero and so is built in here so that the same
%   internal interface works for all PDF-related drivers.
%    \begin{macrocode}
\cs_new_protected:Npn \@@_matrix:n #1
  { \@@_literal:n { #1 \c_space_tl 0~0~cm } }
%    \end{macrocode}
% \end{macro}
%
% \subsubsection{Color}
%
% \begin{variable}{\l_@@_current_color_tl}
%   The current color in driver-dependent format.
%    \begin{macrocode}
\tl_new:N \l_@@_current_color_tl
\tl_set:Nn \l_@@_current_color_tl { [ 0 ] }
%<*package>
\AtBeginDocument
  {
    \@ifpackageloaded { color }
      { \tl_set:Nn \l_@@_current_color_tl { \current@color } }
      { }
  }
%</package>
%    \end{macrocode}
% \end{variable}
%
% \begin{macro}[int]{\@@_color_ensure_current:}
% \begin{macro}[aux]{\@@_color_reset:}
%   Directly set the color using the specials with optimisation support.
%    \begin{macrocode}
\cs_new_protected_nopar:Npx \@@_color_ensure_current:
  {
    \tex_special:D { pdf:bcolor~\l_@@_current_color_tl }
    \group_insert_after:N \exp_not:N \@@_color_reset:
  }
\cs_new_protected_nopar:Npx \@@_color_reset:
  { \tex_special:D { pdf:ecolor } }
%    \end{macrocode}
% \end{macro}
% \end{macro}
%
%    \begin{macrocode}
%</dvipdfmx|xdvipdfmx>
%    \end{macrocode}
%
% \subsection{\texttt{xdvipdfmx} driver}
%
%    \begin{macrocode}
%<*xdvipdfmx>
%    \end{macrocode}
%
% \subsubsection{Color}
%
% \begin{macro}[int]{\@@_color_ensure_current:}
% \begin{macro}[aux]{\@@_color_reset:}
%   The \LaTeXe{} driver uses \texttt{dvips}-like specials so there has to
%   be a change of set up if \pkg{color} is loaded.
%    \begin{macrocode}
%<*package>
\AtBeginDocument
  {
    \@ifpackageloaded { color }
      {
        \cs_set_protected_nopar:Npn \@@_color_ensure_current:
          {
            \tex_special:D { color~push~\l_@@_current_color_tl }
            \group_insert_after:N \exp_not:N \@@_color_reset:
          }

        \cs_set_protected_nopar:Npn \@@_color_reset:
          { \tex_special:D { color~pop } }
      }
      { }
  }
%</package>
%    \end{macrocode}
% \end{macro}
% \end{macro}
%
%    \begin{macrocode}
%</xdvipdfmx>
%    \end{macrocode}
%
% \subsection{Common code for PDF production}
%
% As all of the drivers which understand PDF-targeted specials act in much
% the same way there is a lot of shared code. Rather than try to DocStrip it
% interspersed with the above, we collect all of it here.
%
%    \begin{macrocode}
%<*dvipdfmx|pdfmode|xdvipdfmx>
%    \end{macrocode}
%
% \subsubsection{Box operations}
%
% \begin{macro}{\@@_box_use_clip:N}
%   The general method is to save the current location, define a clipping path
%   equivalent to the bounding box, then insert the content at the current
%   position and in a zero width box. The \enquote{real} width is then made up
%   using a horizontal skip before tidying up. There are other approaches that
%   can be taken (for example using XForm objects), but the logic here shares
%   as much code as possible and uses the same conversions (and so same
%   rounding errors) in all cases.
%    \begin{macrocode}
\cs_new_protected:Npn \@@_box_use_clip:N #1
  {
    \@@_scope_begin:
    \@@_literal:n
      {
        0~
        \dim_to_decimal_in_bp:n { -\box_dp:N #1 } ~
        \dim_to_decimal_in_bp:n { \box_wd:N #1 } ~
        \dim_to_decimal_in_bp:n { \box_ht:N #1 + \box_dp:N #1 } ~
        re~W~n
      }
    \hbox_overlap_right:n { \box_use:N #1 }
    \@@_scope_end:
    \skip_horizontal:n { \box_wd:N #1 }
  }
%    \end{macrocode}
% \end{macro}
%
% \begin{macro}[int]{\@@_box_use_rotate:Nn}
% \begin{variable}{\l_@@_cos_fp, \l_@@_sin_fp}
%   Rotations are set using an affine transformation matrix which therefore
%   requires sine/cosine values not the angle itself. We store the rounded
%   values to avoid rounding twice. There are also a couple of comparisons to
%   ensure that |-0| is not written to the output, as this avoids any issues
%   with problematic display programs.  Note that numbers are compared to~$0$
%   after rounding.
%    \begin{macrocode}
\cs_new_protected:Npn \@@_box_use_rotate:Nn #1#2
  {
    \@@_scope_begin:
    \box_set_wd:Nn #1 \c_zero_dim
    \fp_set:Nn \l_@@_cos_fp { round ( cosd ( #2 ) , 5 ) }
    \fp_compare:nNnT \l_@@_cos_fp = \c_zero_fp
      { \fp_zero:N \l_@@_cos_fp }
    \fp_set:Nn \l_@@_sin_fp { round ( sind ( #2 ) , 5 ) }
    \@@_matrix:n
      {
        \fp_use:N \l_@@_cos_fp \c_space_tl
        \fp_compare:nNnTF \l_@@_sin_fp = \c_zero_fp
          { 0~0 }
          {
            \fp_use:N \l_@@_sin_fp
            \c_space_tl
            \fp_eval:n { -\l_@@_sin_fp }
          }
        \c_space_tl
        \fp_use:N \l_@@_cos_fp
      }
   \box_use:N #1
   \@@_scope_end:
  }
\fp_new:N \l_@@_cos_fp
\fp_new:N \l_@@_sin_fp
%    \end{macrocode}
% \end{variable}
% \end{macro}
%
% \begin{macro}{\@@_box_use_scale:Nnn}
%   The same idea as for rotation but without the complexity of signs and
%   cosines.
%    \begin{macrocode}
\cs_new_protected:Npn \@@_box_use_scale:Nnn #1#2#3
  {
    \@@_scope_begin:
    \@@_matrix:n
      {
        \fp_eval:n { round ( #2 , 5 ) } ~
        0~0~
        \fp_eval:n { round ( #3 , 5 ) }
      }
    \hbox_overlap_right:n { \box_use:N #1 }
    \@@_scope_end:
  }
%    \end{macrocode}
% \end{macro}
%
%    \begin{macrocode}
%</dvipdfmx|pdfmode|xdvipdfmx>
%    \end{macrocode}
%
% \subsection{\texttt{dvips} driver}
%
%    \begin{macrocode}
%<*dvips>
%    \end{macrocode}
%
% \subsubsection{Basics}
%
% \begin{macro}[int]{\@@_literal:n}
%   In the case of \texttt{dvips} there is no build-in saving of the current
%   position, and so some additional PostScript is required to set up the
%   transformation matrix and also to restore it afterwards. Notice the use
%   of the stack to save the current position \enquote{up front} and to
%   move back to it at the end of the process.
%    \begin{macrocode}
\cs_new_protected:Npx \@@_literal:n #1
  {
    \tex_special:D
      {
        ps:
          currentpoint~
          currentpoint~translate~
          #1 ~
          neg~exch~neg~exch~translate
      }
  }
%    \end{macrocode}
% \end{macro}
%
% \begin{macro}{\@@_scope_begin:, \@@_scope_end:}
%   Scope saving/restoring is done directly with no need to worry about the
%   transformation matrix.
%    \begin{macrocode}
\cs_new_protected_nopar:Npn \@@_scope_begin:
  { \tex_special:D { ps:gsave } }
\cs_new_protected_nopar:Npn \@@_scope_end:
  { \tex_special:D { ps:grestore } }
%    \end{macrocode}
% \end{macro}
%
% \subsection{Driver-specific auxiliaries}
%
% \begin{macro}[int, EXP]{\@@_absolute_lengths:n}
%   The \texttt{dvips} driver scales all absolute dimensions based
%   on the output resolution selected and any \TeX{} magnification. Thus
%   for any operation involving absolute lengths there is a correction to
%   make. This is based on \texttt{normalscale} from \texttt{special.pro}
%   but using the stack rather than a definition to save the current matrix.
%    \begin{macrocode}
\cs_new:Npn \@@_absolute_lengths:n #1
  {
     matrix~currentmatrix~
     Resolution~72~div~VResolution~72~div~scale~
     DVImag~dup~scale~
     #1 ~
     setmatrix
  }
%    \end{macrocode}
% \end{macro}
%
% \subsubsection{Box operations}
%
% \begin{macro}{\@@_box_use_clip:N}
%   Much the same idea as for the PDF mode version but with a slightly
%   different syntax for creating the clip path. To avoid any scaling
%   issues we need the absolute length auxiliary here.
%    \begin{macrocode}
\cs_new_protected:Npn \@@_box_use_clip:N #1
  {
    \@@_scope_begin:
    \@@_literal:n
      {
        \@@_absolute_lengths:n
          {
            0 ~
            \dim_to_decimal_in_bp:n { \box_dp:N #1 } ~
            \dim_to_decimal_in_bp:n { \box_wd:N #1 } ~
            \dim_to_decimal_in_bp:n { -\box_ht:N #1 - \box_dp:N #1 } ~
            rectclip
          }
      }
    \hbox_overlap_right:n { \box_use:N #1 }
    \@@_scope_end:
    \skip_horizontal:n { \box_wd:N #1 }
  }
%    \end{macrocode}
% \end{macro}
%
% \begin{macro}{\@@_box_use_rotate:Nn}
%   Rotating using \texttt{dvips} does not require that the box dimensions
%   are altered and has a very convenient built-in operation. Zero rotation
%   must be written as |0| not |-0| so there is a quick test.
%    \begin{macrocode}
\cs_new_protected:Npn \@@_box_use_rotate:Nn #1#2
  {
    \@@_scope_begin:
    \@@_literal:n
      {
        \fp_compare:nNnTF {#2} = \c_zero_fp
          { 0 }
          { \fp_eval:n { round ( -#2 , 5 ) } } ~
        rotate
      }
   \box_use:N #1
   \@@_scope_end:
  }
% \end{macro}
%
% \begin{macro}{\@@_box_use_scale:Nnn}
%   The \texttt{dvips} driver once again has a dedicated operation we can
%   use here.
%    \begin{macrocode}
\cs_new_protected:Npn \@@_box_use_scale:Nnn #1#2#3
  {
    \@@_scope_begin:
    \@@_literal:n
      {
        \fp_eval:n { round ( #2 , 5 ) } ~
        \fp_eval:n { round ( #3 , 5 ) } ~
        scale
      }
    \hbox_overlap_right:n { \box_use:N #1 }
    \@@_scope_end:
  }
%    \end{macrocode}
% \end{macro}
%
% \subsubsection{Color}
%
% \begin{variable}{\l_@@_current_color_tl}
%   The current color in driver-dependent format.
%    \begin{macrocode}
\tl_new:N \l_@@_current_color_tl
\tl_set:Nn \l_@@_current_color_tl { gray~0 }
%<*package>
\AtBeginDocument
  {
    \@ifpackageloaded { color }
      { \tl_set:Nn \l_@@_current_color_tl { \current@color } }
      { }
  }
%</package>
%    \end{macrocode}
% \end{variable}
%
% \begin{macro}[int]{\@@_color_ensure_current:}
% \begin{macro}[aux]{\@@_color_reset:}
%   Directly set the color using the specials: no optimisation here.
%    \begin{macrocode}
\cs_new_protected_nopar:Npn \@@_color_ensure_current:
  {
    \tex_special:D { color~push~\l_@@_current_color_tl }
    \group_insert_after:N \@@_color_reset:
  }

\cs_new_protected_nopar:Npn \@@_color_reset:
  { \tex_special:D { color~pop } }
%    \end{macrocode}
% \end{macro}
% \end{macro}
%
%    \begin{macrocode}
%</dvips>
%    \end{macrocode}
%
% \subsection{\texttt{dvisvgm} driver}
%
%    \begin{macrocode}
%<*dvisvgm>
%    \end{macrocode}
%
% \subsubsection{Basics}
%
% \begin{macro}[int]{\@@_literal:n}
%   Unlike the other drivers, the requirements for making SVG files mean
%   that we can't conveniently transform all operations to the current point.
%   That makes life a bit more tricky later as that needs to be accounted for.
%   A new line is added after each call to help to keep the output readable
%   for debugging.
%    \begin{macrocode}
\cs_new_protected:Npn \@@_literal:n #1
  { \tex_special:D { dvisvgm:raw~ #1 { ?nl } } }
%    \end{macrocode}
% \end{macro}
%
% \begin{macro}[int]{\@@_scope_begin:, \@@_scope_end:}
%   A scope in SVG terms is slightly different to the other drivers as
%   operations have to be \enquote{tied} to these not simply inside them.
%    \begin{macrocode}
\cs_new_protected_nopar:Npn \@@_scope_begin:
  { \@@_literal:n { <g> } }
\cs_new_protected_nopar:Npn \@@_scope_end:
  { \@@_literal:n { </g> } }
%    \end{macrocode}
% \end{macro}
%
% \subsection{Driver-specific auxiliaries}
%
% \begin{macro}[int]{\@@_scope_begin:n}
%   In SVG transformations, clips and so on are attached directly to scopes so
%   we need a way or allowing for that. This is rather more useful that
%   \cs{@@_scope_begin:} as a result. No assumptions are made about the nature
%   of the scoped operation(s).
%    \begin{macrocode}
\cs_new_protected:Npn \@@_scope_begin:n #1
  { \@@_literal:n { <g~ #1 > } }
%    \end{macrocode}
% \end{macro}
%
% \subsubsection{Box operations}
%
% \begin{macro}[int]{\@@_box_use_clip:N}
% \begin{variable}[aux]{\g_@@_clip_path_int}
%   Clipping in SVG is more involved than with other drivers. The first issue
%   is that the clipping path must be defined separately from where it is used,
%   so we need to track how many paths have applied. The naming here uses
%   \texttt{l3cp} as the namespace with a number following. Rather than use
%   a rectangular operation, we define the path manually as this allows it to
%   have a depth: easier than the alternative approach of shifting content
%   up and down using scopes to allow for the depth of the \TeX{} box and
%   keep the reference point the same!
%    \begin{macrocode}
\cs_new_protected:Npn \@@_box_use_clip:N #1
  {
    \int_gincr:N \g_@@_clip_path_int
    \@@_literal:n
      { < clipPath~id = " l3cp \int_use:N \g_@@_clip_path_int " > }
    \@@_literal:n
      {
        <
          path ~ d =
            "
              M ~ 0 ~
                  \dim_to_decimal:n { -\box_dp:N #1 } ~
              L ~ \dim_to_decimal:n { \box_wd:N #1 } ~
                  \dim_to_decimal:n { -\box_dp:N #1 } ~
              L ~ \dim_to_decimal:n { \box_wd:N #1 }  ~
                  \dim_to_decimal:n { \box_ht:N #1 + \box_dp:N #1 } ~
              L ~ 0 ~
                  \dim_to_decimal:n { \box_ht:N #1 + \box_dp:N #1 } ~
              Z
            "
        />
      }
    \@@_literal:n
      { < /clipPath > }
%    \end{macrocode}
%   In general the SVG set up does not try to transform coordinates to the
%   current point. For clipping we need to do that, so have a transformation
%   here to get us to the right place, and a matching one just before the
%   \TeX{} box is inserted to get things back on track. The clip path needs to
%   come between those two such that if lines up with the current point, as
%   does the \TeX{} box.
%    \begin{macrocode}
    \@@_scope_begin:n
      {
        transform =
          "
            translate ( { ?x } , { ?y } ) ~
            scale ( 1 , -1 )
          "
      }
    \@@_scope_begin:n
      {
        clip-path = "url ( \c_hash_str l3cp \int_use:N \g_@@_clip_path_int ) "
      }
    \@@_scope_begin:n
      {
        transform =
          "
            scale ( -1 , 1 ) ~
            translate ( { ?x } , { ?y } ) ~
            scale ( -1 , -1 )
          "
      }
    \box_use:N #1
    \@@_scope_end:
    \@@_scope_end:
    \@@_scope_end:
%    \skip_horizontal:n { \box_wd:N #1 }
  }
\int_new:N \g_@@_clip_path_int
%    \end{macrocode}
% \end{variable}
% \end{macro}
%
% \begin{macro}[int]{\@@_box_use_rotate:Nn}
%   Rotation has a dedicated operation which includes a centre-of-rotation
%   optional pair. That can be picked up from the driver syntax, so there is
%   no need to worry about the transformation matrix.
%    \begin{macrocode}
\cs_new_protected:Npn \@@_box_use_rotate:Nn #1#2
  {
    \@@_scope_begin:n
      {
        transform =
          "
            rotate
            ( \fp_eval:n { round ( -#2 , 5 ) } , ~ { ?x } , ~ { ?y } )
          "
      }
    \box_use:N #1
    \@@_scope_end:
  }
%    \end{macrocode}
% \end{macro}
%
% \begin{macro}{\@@_box_use_scale:Nnn}
%   In contrast to rotation, we have to account for the current position in this
%   case. That is done using a couple of translations in addition to the scaling
%   (which is therefore done backward with a flip).
%    \begin{macrocode}
\cs_new_protected:Npn \@@_box_use_scale:Nnn #1#2#3
  {
    \@@_scope_begin:n
      {
        transform =
          "
            translate ( { ?x } , { ?y } ) ~
            scale
              (
                \fp_eval:n { round ( -#2 , 5 ) } ,
                \fp_eval:n { round ( -#3 , 5 ) }
              ) ~
            translate ( { ?x } , { ?y } ) ~
            scale ( -1 )
          "
      }
    \hbox_overlap_right:n { \box_use:N #1 }
    \@@_scope_end:
  }
%    \end{macrocode}
% \end{macro}
%
% \subsubsection{Color}
%
% \begin{variable}{\l_@@_current_color_tl}
%   The current color in driver-dependent format: the same as for
%   \texttt{dvips}.
%    \begin{macrocode}
\tl_new:N \l_@@_current_color_tl
\tl_set:Nn \l_@@_current_color_tl { gray~0 }
%<*package>
\AtBeginDocument
  {
    \@ifpackageloaded { color }
      { \tl_set:Nn \l_@@_current_color_tl { \current@color } }
      { }
  }
%</package>
%    \end{macrocode}
% \end{variable}
%
% \begin{macro}[int]{\@@_color_ensure_current:}
% \begin{macro}[aux]{\@@_color_reset:}
%   Directly set the color: same as \texttt{dvips}.
%    \begin{macrocode}
\cs_new_protected_nopar:Npn \@@_color_ensure_current:
  {
    \tex_special:D { color~push~\l_@@_current_color_tl }
    \group_insert_after:N \exp_not:N \@@_color_reset:
  }

\cs_new_protected_nopar:Npn \@@_color_reset:
  { \tex_special:D { color~pop } }
%    \end{macrocode}
% \end{macro}
% \end{macro}
%
%    \begin{macrocode}
%</dvisvgm>
%    \end{macrocode}
%
%    \begin{macrocode}
%</initex|package>
%    \end{macrocode}
%
% \end{implementation}
%
% \PrintIndex