% \iffalse meta-comment
%
%% File: l3token.dtx Copyright (C) 2005-2011 by The LaTeX3 Project
%%
%% It may be distributed and/or modified under the conditions of the
%% LaTeX Project Public License (LPPL), either version 1.3c of this
%% license or (at your option) any later version.  The latest version
%% of this license is in the file
%%
%%    http://www.latex-project.org/lppl.txt
%%
%% This file is part of the "expl3 bundle" (The Work in LPPL)
%% and all files in that bundle must be distributed together.
%%
%% The released version of this bundle is available from CTAN.
%%
%% -----------------------------------------------------------------------
%%
%% The development version of the bundle can be found at
%%
%%    http://www.latex-project.org/svnroot/experimental/trunk/
%%
%% for those people who are interested.
%%
%%%%%%%%%%%
%% NOTE: %%
%%%%%%%%%%%
%%
%%   Snapshots taken from the repository represent work in progress and may
%%   not work or may contain conflicting material!  We therefore ask
%%   people _not_ to put them into distributions, archives, etc. without
%%   prior consultation with the LaTeX3 Project.
%%
%% -----------------------------------------------------------------------
%
%<*driver|package>
\RequirePackage{l3names}
\GetIdInfo$Id$
  {L3 Experimental token manipulation}
%</driver|package>
%<*driver>
\documentclass[full]{l3doc}
\begin{document}
  \DocInput{\jobname.dtx}
\end{document}
%</driver>
% \fi
%
% \title{^^A
%   The \pkg{l3token} package: Token manipulation^^A
%   \thanks{This file describes v\fileversion, last revised \filedate.}^^A
% }
%
% \author{^^A
%  The \LaTeX3 Project\thanks
%    {^^A
%      E-mail:
%        \href{mailto:latex-team@latex-project.org}
%          {latex-team@latex-project.org}^^A
%    }^^A
% }
%
% \date{Released \filedate}
%
% \maketitle
%
% \begin{documentation}
%
% This module deals with tokens. Now this is perhaps not the most
% precise description so let's try with a better description: When
% programming in \Te{}X, it is often desirable to know just what a
% certain token is: is it a control sequence or something
% else. Similarly one often needs to know if a control sequence is
% expandable or not, a macro or a primitive, how many arguments it
% takes etc. Another thing of great importance (especially when it
% comes to document commands) is looking ahead in the token stream to
% see if a certain character is present and maybe even remove it or
% disregard other tokens while scanning. This module provides
% functions for both and as such will have two primary function
% categories: |\token| for anything that deals with tokens and
% |\peek| for looking ahead in the token stream.
%
% Most of the time we will be using the term \enquote{token} but most of the
% time the function we're describing can equally well by used on a
% control sequence as such one is one token as well.
%
% We shall refer to list of tokens as |tlist|s and such lists
% represented by a single control sequence is a \enquote{token list variable}
% |tl var|. Functions for these two types are found in the \textsf{l3tl}
% module.
%
% \section{Character tokens}
%
% \begin{function}
%   {
%     \char_make_escape:N           ,
%     \char_make_begin_group:N      ,
%     \char_make_end_group:N        ,
%     \char_make_math_shift:N       ,
%     \char_make_alignment:N        ,
%     \char_make_end_line:N         ,
%     \char_make_parameter:N        ,
%     \char_make_math_superscript:N ,
%     \char_make_math_subscript:N   ,
%     \char_make_ignore:N           ,
%     \char_make_space:N            ,
%     \char_make_letter:N           ,
%     \char_make_other:N            ,
%     \char_make_active:N           ,
%     \char_make_comment:N          ,
%     \char_make_invalid:N
%   }
%   \begin{syntax}
%     \cs{char_make_letter:N} \meta{character}
%   \end{syntax}
%   Sets the category code of the \meta{character} to that indicated in
%   the function name. Depending on the current category code of the
%   \meta{token} the escape token may also be needed:
%   \begin{verbatim}
%     \char_make_other:N \%
%   \end{verbatim}
%   The assignment is local.
% \end{function}
%
% \begin{function}
%   {
%     \char_make_escape:n           ,
%     \char_make_begin_group:n      ,
%     \char_make_end_group:n        ,
%     \char_make_math_shift:n       ,
%     \char_make_alignment:n        ,
%     \char_make_end_line:n         ,
%     \char_make_parameter:n        ,
%     \char_make_math_superscript:n ,
%     \char_make_math_subscript:n   ,
%     \char_make_ignore:n           ,
%     \char_make_space:n            ,
%     \char_make_letter:n           ,
%     \char_make_other:n            ,
%     \char_make_active:n           ,
%     \char_make_comment:n          ,
%     \char_make_invalid:n
%   }
%   \begin{syntax}
%     \cs{char_make_letter:n} \Arg{integer expression}
%   \end{syntax}
%   Sets the category code of the \meta{character} which has character
%   code as given by the \meta{integer expression}. This version can be
%   used to set up characters which cannot otherwise be given
%   (\emph{cf.}~the \texttt{N}-type variants). The assignment is local.
% \end{function}
%
% \begin{function}{\char_set_catcode:nn}
%   \begin{syntax}
%     \cs{char_set_catcode:nn} \Arg{intexpr1} \Arg{intexpr2}
%   \end{syntax}
%   These functions set the category code of the \meta{character} which
%   has character code as given by the \meta{integer expression}.
%   The first \meta{integer expression}
%   is the character code and the second is the category code to apply.
%   The setting applies within the current \TeX{} group. In general, the
%   symbolic functions \cs{char_make_\meta{type}} should be preferred,
%   but there are cases where these lower-level functions may be useful.
% \end{function}
%
% \begin{function}[EXP]{\char_value_catcode:n}
%   \begin{syntax}
%     \cs{char_value_catcode:n} \Arg{integer expression}
%   \end{syntax}
%   Expands to the current category code of the \meta{character} with
%   character code given by the
%   \meta{integer expression}.
% \end{function}
%
% \begin{function}{\char_show_value_catcode:n}
%   \begin{syntax}
%     \cs{char_show_value_catcode:n} \Arg{integer expression}
%   \end{syntax}
%   Displays the current category code of the \meta{character} with
%   character code given by the  \meta{integer expression} on the
%   terminal.
% \end{function}
%
% \begin{function}{\char_set_lccode:nn}
%   \begin{syntax}
%     \cs{char_set_lcode:nn} \Arg{intexpr1} \Arg{intexpr2}
%   \end{syntax}
%   This function set up the behaviour of \meta{character} when
%   found inside \cs{tl_to_lowercase:n}, such that \meta{character1}
%   will be converted into \meta{character2}. The two \meta{characters}
%   may be specified using an \meta{integer expression} for the character code
%   concerned. This may include the \TeX{} |`|\meta{character}
%   method for converting a single character into its character
%   code:
%   \begin{verbatim}
%     \char_set_lccode:nn { `\A } { `\a } % Standard behaviour
%     \char_set_lccode:nn { `\A } { `\A + 32 }
%     \char_set_lccode:nn { 50 } { 60 }
%   \end{verbatim}
%   The setting applies within the current \TeX{} group.
% \end{function}
%
% \begin{function}[EXP]{\char_value_lccode:n}
%   \begin{syntax}
%     \cs{char_value_lccode:n} \Arg{integer expression}
%   \end{syntax}
%   Expands to the current lower case code of the \meta{character} with
%   character code given by the
%   \meta{integer expression}.
% \end{function}
%
% \begin{function}{\char_show_value_lccode:n}
%   \begin{syntax}
%     \cs{char_show_value_lccode:n} \Arg{integer expression}
%   \end{syntax}
%   Displays the current lower case code of the \meta{character} with
%   character code given by the  \meta{integer expression} on the
%   terminal.
% \end{function}
%
% \begin{function}{\char_set_uccode:nn}
%   \begin{syntax}
%     \cs{char_set_uccode:nn} \Arg{intexpr1} \Arg{intexpr2}
%   \end{syntax}
%   This function set up the behaviour of \meta{character} when
%   found inside \cs{tl_to_uppeercase:n}, such that \meta{character1}
%   will be converted into \meta{character2}. The two \meta{characters}
%   may be specified using an \meta{integer expression} for the character code
%   concerned. This may include the \TeX{} |`|\meta{character}
%   method for converting a single character into its character
%   code:
%   \begin{verbatim}
%     \char_set_uccode:nn { `\a } { `\A } % Standard behaviour
%     \char_set_uccode:nn { `\A } { `\A - 32 }
%     \char_set_uccode:nn { 60 } { 50 }
%   \end{verbatim}
%   The setting applies within the current \TeX{} group.
% \end{function}
%
% \begin{function}[EXP]{\char_value_uccode:n}
%   \begin{syntax}
%     \cs{char_value_uccode:n} \Arg{integer expression}
%   \end{syntax}
%   Expands to the current upper case code of the \meta{character} with
%   character code given by the
%   \meta{integer expression}.
% \end{function}
%
% \begin{function}{\char_show_value_uccode:n}
%   \begin{syntax}
%     \cs{char_show_value_uccode:n} \Arg{integer expression}
%   \end{syntax}
%   Displays the current upper case code of the \meta{character} with
%   character code given by the  \meta{integer expression} on the
%   terminal.
% \end{function}
%
% \begin{function}{\char_set_mathcode:nn}
%   \begin{syntax}
%     \cs{char_set_mathcode:nn} \Arg{intexpr1} \Arg{intexpr2}
%   \end{syntax}
%   This function sets up the math code of \meta{character}.
%   The \meta{character} is specified as
%   an \meta{integer expression} which will be used as the character
%   code of the relevant character. The setting applies within the
%   current \TeX{} group.
% \end{function}
%
% \begin{function}[EXP]{\char_value_mathcode:n}
%   \begin{syntax}
%     \cs{char_value_mathcode:n} \Arg{integer expression}
%   \end{syntax}
%   Expands to the current math code of the \meta{character} with
%   character code given by the
%   \meta{integer expression}.
% \end{function}
%
% \begin{function}{\char_show_value_mathcode:n}
%   \begin{syntax}
%     \cs{char_show_value_mathcode:n} \Arg{integer expression}
%   \end{syntax}
%   Displays the current math code of the \meta{character} with
%   character code given by the  \meta{integer expression} on the
%   terminal.
% \end{function}
%
% \begin{function}{\char_set_sfcode:nn}
%   \beginsfcode
%     \cs{char_set_mathcode:nn} \Arg{intexpr1} \Arg{intexpr2}
%   \end{syntax}
%   This function sets up the space factor for the \meta{character}.
%   The \meta{character} is specified as
%   an \meta{integer expression} which will be used as the character
%   code of the relevant character. The setting applies within the
%   current \TeX{} group.
% \end{function}
%
% \begin{function}[EXP]{\char_value_sfcode:n}
%   \begin{syntax}
%     \cs{char_value_sfcode:n} \Arg{integer expression}
%   \end{syntax}
%   Expands to the current space factor for the \meta{character} with
%   character code given by the
%   \meta{integer expression}.
% \end{function}
%
% \begin{function}{\char_show_value_sfcode:n}
%   \begin{syntax}
%     \cs{char_show_value_sfcode:n} \Arg{integer expression}
%   \end{syntax}
%   Displays the current space factor for the \meta{character} with
%   character code given by the  \meta{integer expression} on the
%   terminal.
% \end{function}
%
% \section{Generic tokens}
%
% \begin{function}{\token_new:Nn}
%   \begin{syntax}
%     \cs{token_new:Nn"} \Arg{token1} \Arg{token1}
%   \end{syntax}
%   Defines \meta{token1} to globally be a snapshot of \meta{token2}.
%   This will be an implicit representation of \meta{token2}.
% \end{function}
%
% \begin{variable}
%   {
%     \c_group_begin_token,
%     \c_group_end_token,
%     \c_math_shift_token,
%     \c_alignment_tab_token,
%     \c_parameter_token,
%     \c_math_superscript_token,
%     \c_math_subscript_token,
%     \c_space_token,
%     \c_letter_token,
%     \c_other_char_token,
%     \c_active_char_token
%   }
%   These are implicit tokens which have the category code described
%   by their name. They are used internally for test purposes but
%   are also available to the programmer for other uses.
% \end{variable}
%
% \section{Token conditionals}
%
% \begin{function}[EXP,pTF]{\token_if_group_begin:N}
%   \begin{syntax}
%     \cs{token_if_group_begin_p:N} \meta{token}
%     \cs{token_if_group_begin:NTF} \meta{token} \Arg{true code}
%     ~~\Arg{false code}
%   \end{syntax}
%   Tests if \meta{token} has the category code of a begin group token
%   (|{| when normal \TeX{} category codes are in ^^A }
%   force). The branching versions then leave either \meta{true code} or
%   \meta{false code} in the input stream, as appropriate to the truth of
%   the test and the variant of the function chosen. The logical truth of
%   the test is left in the input stream by the predicate version.
% \end{function}
%
% \begin{function}[EXP,pTF]{\token_if_group_end:N}
%   \begin{syntax}
%     \cs{token_if_group_end_p:N} \meta{token}
%     \cs{token_if_group_end:NTF} \meta{token} \Arg{true code}
%     ~~\Arg{false code}
%    \end{syntax}
%   Tests if \meta{token} has the category code of a end group token
%   (^^A {
%   |}| when normal \TeX{} category codes are in force). The branching
%   versions then leave either \meta{true code} or \meta{false code} in the
%   input stream, as appropriate to the truth of the test and the variant
%   of the function chosen. The logical truth of the test is left in the
%   input stream by the predicate version.
% \end{function}
%
% \begin{function}[EXP,pTF]{\token_if_math_shift:N}
%   \begin{syntax}
%     \cs{token_if_math_shift_p:N} \meta{token}
%     \cs{token_if_math_shift:NTF} \meta{token} \Arg{true code}
%     ~~\Arg{false code}
%   \end{syntax}
%   Tests if \meta{token} has the category code of a math shift token
%   (|$| when normal \TeX{} category codes are in force). The
%   branching versions then leave either \meta{true code} or
%   \meta{false code} in the input stream, as appropriate to the truth of
%   the test and the variant of the function chosen. The logical truth of
%   the test is left in the input stream by the predicate version.
% \end{function}
%
% \begin{function}[EXP,pTF]{\token_if_alignment_tab:N}
%   \begin{syntax}
%     \cs{token_if_alignment_tab_p:N} \meta{token}
%     \cs{token_if_alignment_tab:NTF} \meta{token} \Arg{true code}
%     ~~\Arg{false code}
%   \end{syntax}
%   Tests if \meta{token} has the category code of an alignment token
%   (|&| when normal \TeX{} category codes are in force). The
%   branching versions then leave either \meta{true code} or
%   \meta{false code} in the input stream, as appropriate to the truth of
%   the test and the variant of the function chosen. The logical truth of
%   the test is left in the input stream by the predicate version.
% \end{function}
%
% \begin{function}[EXP,pTF]{\token_if_parameter:N}
%   \begin{syntax}
%     \cs{token_if_parameter_p:N} \meta{token}
%     \cs{token_if_alignment:NTF} \meta{token} \Arg{true code}
%     ~~\Arg{false code}
%   \end{syntax}
%   Tests if \meta{token} has the category code of an alignment token
%   (|#| when normal \TeX{} category codes are in force). The
%   branching versions then leave either \meta{true code} or
%   \meta{false code} in the input stream, as appropriate to the truth of
%   the test and the variant of the function chosen. The logical truth of
%   the test is left in the input stream by the predicate version.
% \end{function}
%
% \begin{function}[EXP,pTF]{\token_if_math_superscript:N}
%   \begin{syntax}
%     \cs{token_if_math_superscript_p:N} \meta{token}
%     \cs{token_if_math_superscript:NTF} \meta{token} \Arg{true code}
%     ~~\Arg{false code}
%   \end{syntax}
%   Tests if \meta{token} has the category code of an superscript token
%   (|^| when normal \TeX{} category codes are in force). The
%   branching versions then leave either \meta{true code} or
%   \meta{false code} in the input stream, as appropriate to the truth of
%   the test and the variant of the function chosen. The logical truth of
%   the test is left in the input stream by the predicate version.
% \end{function}
%
% \begin{function}[EXP,pTF]{\token_if_math_subscript:N}
%   \begin{syntax}
%     \cs{token_if_math_subscript_p:N} \meta{token}
%     \cs{token_if_math_subscript:NTF} \meta{token} \Arg{true code}
%     ~~\Arg{false code}
%   \end{syntax}
%   Tests if \meta{token} has the category code of an subscript token
%   (|_| when normal \TeX{} category codes are in force). The
%   branching versions then leave either \meta{true code} or
%   \meta{false code} in the input stream, as appropriate to the truth of
%   the test and the variant of the function chosen. The logical truth of
%   the test is left in the input stream by the predicate version.
% \end{function}
%
% \begin{function}[EXP,pTF]{\token_if_space:N}
%   \begin{syntax}
%     \cs{token_if_space_p:N} \meta{token}
%     \cs{token_if_space:NTF} \meta{token} \Arg{true code}
%     ~~\Arg{false code}
%   \end{syntax}
%   Tests if \meta{token} has the category code of an space token. The
%   branching versions then leave either \meta{true code} or
%   \meta{false code} in the input stream, as appropriate to the truth of
%   the test and the variant of the function chosen. The logical truth of
%   the test is left in the input stream by the predicate version.
% \end{function}
%
% \begin{function}[EXP,pTF]{\token_if_letter:N}
%   \begin{syntax}
%     \cs{token_if_letter_p:N} \meta{token}
%     \cs{token_if_letter:NTF} \meta{token} \Arg{true code}
%     ~~\Arg{false code}
%   \end{syntax}
%   Tests if \meta{token} has the category code of an letter token. The
%   branching versions then leave either \meta{true code} or
%   \meta{false code} in the input stream, as appropriate to the truth of
%   the test and the variant of the function chosen. The logical truth of
%   the test is left in the input stream by the predicate version.
% \end{function}
%
% \begin{function}[EXP,pTF]{\token_if_other_char:N}
%   \begin{syntax}
%     \cs{token_if_other_char_p:N} \meta{token}
%     \cs{token_if_other_char:NTF} \meta{token} \Arg{true code}
%     ~~\Arg{false code}
%   \end{syntax}
%   Tests if \meta{token} has the category code of an \enquote{other}
%   token. The branching versions then leave either \meta{true code} or
%   \meta{false code} in the input stream, as appropriate to the truth of
%   the test and the variant of the function chosen. The logical truth of
%   the test is left in the input stream by the predicate version.
% \end{function}
%
% \begin{function}[EXP,pTF]{\token_if_active_char:N}
%   \begin{syntax}
%     \cs{token_if_active_char_p:N} \meta{token}
%     \cs{token_if_active_char:NTF} \meta{token} \Arg{true code}
%     ~~\Arg{false code}
%   \end{syntax}
%   Tests if \meta{token} has the category code of an active character.
%   The branching versions then leave either \meta{true code} or
%   \meta{false code} in the input stream, as appropriate to the truth of
%   the test and the variant of the function chosen. The logical truth of
%   the test is left in the input stream by the predicate version.
% \end{function}
%
% \begin{function}[EXP,pTF]{\token_if_eq_catcode:NN}
%   \begin{syntax}
%     \cs{token_if_eq_catcode_p:NN} \meta{token1} \meta{token2}
%     \cs{token_if_eq_catcode:NNTF} \meta{token1} \meta{token2}
%     ~~\Arg{true code} \Arg{false code}
%   \end{syntax}
%   Tests if the two \meta{tokens} have the same category code. The
%   branching versions then leave either \meta{true code} or
%   \meta{false code} in the input stream, as appropriate to the truth of
%   the test and the variant of the function chosen. The logical truth of
%   the test is left in the input stream by the predicate version.
% \end{function}
%
% \begin{function}[EXP,pTF]{\token_if_eq_charcode:NN}
%   \begin{syntax}
%     \cs{token_if_eq_charcode_p:NN} \meta{token1} \meta{token2}
%     \cs{token_if_eq_charcode:NNTF} \meta{token1} \meta{token2}
%     ~~\Arg{true code} \Arg{false code}
%   \end{syntax}
%   Tests if the two \meta{tokens} have the same character code. The
%   branching versions then leave either \meta{true code} or
%   \meta{false code} in the input stream, as appropriate to the truth of
%   the test and the variant of the function chosen. The logical truth of
%   the test is left in the input stream by the predicate version.
% \end{function}
%
% \begin{function}[EXP,pTF]{\token_if_eq_meaning:NN}
%   \begin{syntax}
%     \cs{token_if_eq_meaning_p:NN} \meta{token1} \meta{token2}
%     \cs{token_if_eq_meaning:NNTF} \meta{token1} \meta{token2}
%     ~~\Arg{true code} \Arg{false code}
%   \end{syntax}
%   Tests if the two \meta{tokens} have the same meaning when expanded.
%   The branching versions then leave either \meta{true code} or
%   \meta{false code} in the input stream, as appropriate to the truth of
%   the test and the variant of the function chosen. The logical truth of
%   the test is left in the input stream by the predicate version.
% \end{function}
%
% \begin{function}[EXP,pTF]{\token_if_macro:N}
%   \begin{syntax}
%     \cs{token_if_macro_p:N} \meta{token}
%     \cs{token_if_macro:NTF} \meta{token} \Arg{true code} \Arg{false code}
%   \end{syntax}
%   Tests if the \meta{token} is a \TeX{} macro.
%   The branching versions then leave either \meta{true code} or
%   \meta{false code} in the input stream, as appropriate to the truth of
%   the test and the variant of the function chosen. The logical truth of
%   the test is left in the input stream by the predicate version.
% \end{function}
%
% \begin{function}[EXP,pTF]{\token_if_cs:N}
%   \begin{syntax}
%     \cs{token_if_cs_p:N} \meta{token}
%     \cs{token_if_cs:NTF} \meta{token} \Arg{true code} \Arg{false code}
%   \end{syntax}
%   Tests if the \meta{token} is a control sequence.
%   The branching versions then leave either \meta{true code} or
%   \meta{false code} in the input stream, as appropriate to the truth of
%   the test and the variant of the function chosen. The logical truth of
%   the test is left in the input stream by the predicate version.
% \end{function}
%
% \begin{function}[EXP,pTF]{\token_if_expandable:N}
%   \begin{syntax}
%     \cs{token_if_expandable_p:N} \meta{token}
%     \cs{token_if_expandable:NTF} \meta{token}
%     ~~\Arg{true code} \Arg{false code}
%   \end{syntax}
%   Tests if the \meta{token} is expandable.
%   The branching versions then leave either \meta{true code} or
%   \meta{false code} in the input stream, as appropriate to the truth of
%   the test and the variant of the function chosen. The logical truth of
%   the test is left in the input stream by the predicate version.
% \end{function}
%
% \begin{function}[EXP,pTF]{\token_if_long_macro:N}
%   \begin{syntax}
%     \cs{token_if_long_macro_p:N} \meta{token}
%     \cs{token_if_long_macro:NTF} \meta{token}
%     ~~\Arg{true code} \Arg{false code}
%   \end{syntax}
%   Tests if the \meta{token} is a long macro.
%   The branching versions then leave either \meta{true code} or
%   \meta{false code} in the input stream, as appropriate to the truth of
%   the test and the variant of the function chosen. The logical truth of
%   the test is left in the input stream by the predicate version.
% \end{function}
%
% \begin{function}[EXP,pTF]{\token_if_protected_macro:N}
%   \begin{syntax}
%     \cs{token_if_protected_macro_p:N} \meta{token}
%     \cs{token_if_protected_macro:NTF} \meta{token}
%     ~~\Arg{true code} \Arg{false code}
%   \end{syntax}
%   Tests if the \meta{token} is a protected macro: a macro which
%   is both protected and long will return logical false.
%   The branching versions then leave either \meta{true code} or
%   \meta{false code} in the input stream, as appropriate to the truth of
%   the test and the variant of the function chosen. The logical truth of
%   the test is left in the input stream by the predicate version.
% \end{function}
%
%% \begin{function}[EXP,pTF]{\token_if_protected_long_macro:N}
%   \begin{syntax}
%     \cs{token_if_protected_long_macro_p:N} \meta{token}
%     \cs{token_if_protected_long_macro:NTF} \meta{token}
%     ~~\Arg{true code} \Arg{false code}
%   \end{syntax}
%   Tests if the \meta{token} is a protected long macro.
%   The branching versions then leave either \meta{true code} or
%   \meta{false code} in the input stream, as appropriate to the truth of
%   the test and the variant of the function chosen. The logical truth of
%   the test is left in the input stream by the predicate version.
% \end{function}
%
% \begin{function}[EXP,pTF]{\token_if_chardef:N}
%   \begin{syntax}
%     \cs{token_if_chardef_p:N} \meta{token}
%     \cs{token_if_chardef:NTF} \meta{token}
%     ~~\Arg{true code} \Arg{false code}
%   \end{syntax}
%   Tests if the \meta{token} is defined to be a chardef.
%   The branching versions then leave either \meta{true code} or
%   \meta{false code} in the input stream, as appropriate to the truth of
%   the test and the variant of the function chosen. The logical truth of
%   the test is left in the input stream by the predicate version.
% \end{function}
%
% \begin{function}[EXP,pTF]{\token_if_mathchardef:N}
%   \begin{syntax}
%     \cs{token_if_mathchardef_p:N} \meta{token}
%     \cs{token_if_mathchardef:NTF} \meta{token}
%     ~~\Arg{true code} \Arg{false code}
%   \end{syntax}
%   Tests if the \meta{token} is defined to be a mathchardef.
%   The branching versions then leave either \meta{true code} or
%   \meta{false code} in the input stream, as appropriate to the truth of
%   the test and the variant of the function chosen. The logical truth of
%   the test is left in the input stream by the predicate version.
% \end{function}
%
% \begin{function}[EXP,pTF]{\token_if_dim_register:N}
%   \begin{syntax}
%     \cs{token_if_dim_register_p:N} \meta{token}
%     \cs{token_if_dim_register:NTF} \meta{token}
%     ~~\Arg{true code} \Arg{false code}
%   \end{syntax}
%   Tests if the \meta{token} is defined to be a dimension register.
%   The branching versions then leave either \meta{true code} or
%   \meta{false code} in the input stream, as appropriate to the truth of
%   the test and the variant of the function chosen. The logical truth of
%   the test is left in the input stream by the predicate version.
% \end{function}
%
% \begin{function}[EXP,pTF]{\token_if_int_register:N}
%   \begin{syntax}
%     \cs{token_if_int_register_p:N} \meta{token}
%     \cs{token_if_int_register:NTF} \meta{token}
%     ~~\Arg{true code} \Arg{false code}
%   \end{syntax}
%   Tests if the \meta{token} is defined to be a integer register.
%   The branching versions then leave either \meta{true code} or
%   \meta{false code} in the input stream, as appropriate to the truth of
%   the test and the variant of the function chosen. The logical truth of
%   the test is left in the input stream by the predicate version.
% \end{function}
%
% \begin{function}[EXP,pTF]{\token_if_skip_register:N}
%   \begin{syntax}
%     \cs{token_if_skip_register_p:N} \meta{token}
%     \cs{token_if_skip_register:NTF} \meta{token}
%     ~~\Arg{true code} \Arg{false code}
%   \end{syntax}
%   Tests if the \meta{token} is defined to be a skip register.
%   The branching versions then leave either \meta{true code} or
%   \meta{false code} in the input stream, as appropriate to the truth of
%   the test and the variant of the function chosen. The logical truth of
%   the test is left in the input stream by the predicate version.
% \end{function}
%
%   \begin{syntax}
%     \cs{token_if_toks_register_p:N} \meta{token}
%     \cs{token_if_toks_register:NTF} \meta{token}
%     ~~\Arg{true code} \Arg{false code}
%   \end{syntax}
%   Tests if the \meta{token} is defined to be a toks register
%   (not used by\LaTeX3).
%   The branching versions then leave either \meta{true code} or
%   \meta{false code} in the input stream, as appropriate to the truth of
%   the test and the variant of the function chosen. The logical truth of
%   the test is left in the input stream by the predicate version.
% \end{function}
%
% \begin{function}[EXP,pTF]{\token_if_primitive:N}
%   \begin{syntax}
%     \cs{token_if_primitive_p:N} \meta{token}
%     \cs{token_if_primitive:NTF} \meta{token}
%     ~~\Arg{true code} \Arg{false code}
%   \end{syntax}
%   Tests if the \meta{token} is an engine primitive.
%   The branching versions then leave either \meta{true code} or
%   \meta{false code} in the input stream, as appropriate to the truth of
%   the test and the variant of the function chosen. The logical truth of
%   the test is left in the input stream by the predicate version.
% \end{function}
%
% \section{Peeking ahead at the next token}
%
% There is often a need to look ahead at the next token in the input
% stream while leaving it in place. This is handled using the
% \enquote{peek} functions. The generic \cs{peek_after:NN} is
% provided along with a family of predefined tests for common cases.
% As peeking ahead does \emph{not} skip spaces the predefined tests
% include both a space-respecting and space-skipping version.
%
% \begin{function}{\peek_after:Nw}
%   \begin{syntax}
%     \cs{peek_after:Nw} \meta{function} \meta{token}
%   \end{syntax}
%   Locally sets the test variable \cs{l_peek_token} equal to \meta{token}
%   (as an implicit token, \emph{not} as a token list), and then
%   expands the \meta{function}. The \meta{token} will remain in
%   the input stream as the next item after the \meta{function}.
%   The \meta{token} here may be \verb*| |, |{| or |}| (assuming
%   normal \TeX{} category codes), \emph{i.e.}~it is not necessarily the
%   next argument which would be grabbed by a normal function.
% \end{function}
% 
% \begin{function}{\peek_gafter:Nw}
%   \begin{syntax}
%     \cs{peek_gafter:Nw} \meta{function} \meta{token}
%   \end{syntax}
%   Globaly sets the test variable \cs{g_peek_token} equal to \meta{token}
%   (as an implicit token, \emph{not} as a token list), and then
%   expands the \meta{function}. The \meta{token} will remain in
%   the input stream as the next item after the \meta{function}.
%   The \meta{token} here may be \verb*| |, |{| or |}| (assuming
%   normal \TeX{} category codes), \emph{i.e.}~it is not necessarily the
%   next argument which would be grabbed by a normal function.
% \end{function}
%
% \begin{variable}{\l_peek_token}
%  Token set by \cs{peek_after:Nw} and available for testing
%  as described above.
% \end{variable}
% 
% \begin{variable}{\g_peek_token}
%  Token set by \cs{peek_gafter:Nw} and available for testing
%  as described above.
% \end{variable}
%
% \begin{function}[TF]{\peek_catcode:N}
%   \begin{syntax}
%     \cs{peek_catcode:NTF} \meta{test token} \Arg{true code} \Arg{false code}
%   \end{syntax}
%   Tests if the next \meta{token} in the input stream has the same
%   category code as the \meta{test token} (as defined by the test
%   \cs{token_if_eq_catcode:NNTF}). Spaces are respected by the test
%   and the \meta{token} will be left in the input stream after
%   the \meta{true code} or \meta{false code} (as appropriate to the
%   result of the test).
% \end{function}
%
% \begin{function}[TF]{\peek_catcode_ignore_spaces:N}
%   \begin{syntax}
%     \cs{peek_catcode_ignore_spaces:NTF} \meta{test token}
%     ~~\Arg{true code} \Arg{false code}
%   \end{syntax}
%   Tests if the next \meta{token} in the input stream has the same
%   category code as the \meta{test token} (as defined by the test
%   \cs{token_if_eq_catcode:NNTF}). Spaces are ignored by the test
%   and the \meta{token} will be left in the input stream after
%   the \meta{true code} or \meta{false code} (as appropriate to the
%   result of the test).
% \end{function}
%
% \begin{function}[TF]{\peek_catcode_remove:N}
%   \begin{syntax}
%     \cs{peek_catcode_remove:NTF} \meta{test token}
%     ~~\Arg{true code} \Arg{false code}
%   \end{syntax}
%   Tests if the next \meta{token} in the input stream has the same
%   category code as the \meta{test token} (as defined by the test
%   \cs{token_if_eq_catcode:NNTF}). Spaces are respected by the test
%   and the \meta{token} will be removed from the input stream if the
%   test is true. The function will then place either the
%   \meta{true code} or \meta{false code} in the input stream (as
%   appropriate to the result of the test).
% \end{function}
%
% \begin{function}[TF]{\peek_catcode_remove_ignore_spaces:N}
%   \begin{syntax}
%     \cs{peek_catcode_remove_ignore_spaces:NTF} \meta{test token}
%     ~~\Arg{true code} \Arg{false code}
%   \end{syntax}
%   Tests if the next \meta{token} in the input stream has the same
%   category code as the \meta{test token} (as defined by the test
%   \cs{token_if_eq_catcode:NNTF}). Spaces are ignored by the test
%   and the \meta{token} will be removed from the input stream if the
%   test is true. The function will then place either the
%   \meta{true code} or \meta{false code} in the input stream (as
%   appropriate to the result of the test).
% \end{function}
%
% \begin{function}[TF]{\peek_charcode:N}
%   \begin{syntax}
%     \cs{peek_charcode:NTF} \meta{test token}
%     ~~\Arg{true code} \Arg{false code}
%   \end{syntax}
%   Tests if the next \meta{token} in the input stream has the same
%   character code as the \meta{test token} (as defined by the test
%   \cs{token_if_eq_charcode:NNTF}). Spaces are respected by the test
%   and the \meta{token} will be left in the input stream after
%   the \meta{true code} or \meta{false code} (as appropriate to the
%   result of the test).
% \end{function}
%
% \begin{function}[TF]{\peek_charcode_ignore_spaces:N}
%   \begin{syntax}
%     \cs{peek_charcode_ignore_spaces:NTF} \meta{test token}
%     ~~\Arg{true code} \Arg{false code}
%   \end{syntax}
%   Tests if the next \meta{token} in the input stream has the same
%   character code as the \meta{test token} (as defined by the test
%   \cs{token_if_eq_charcode:NNTF}). Spaces are ignored by the test
%   and the \meta{token} will be left in the input stream after
%   the \meta{true code} or \meta{false code} (as appropriate to the
%   result of the test).
% \end{function}
%
% \begin{function}[TF]{\peek_charcode_remove:N}
%   \begin{syntax}
%     \cs{peek_charcode_remove:NTF} \meta{test token}
%     ~~\Arg{true code} \Arg{false code}
%   \end{syntax}
%   Tests if the next \meta{token} in the input stream has the same
%   character code as the \meta{test token} (as defined by the test
%   \cs{token_if_eq_charcode:NNTF}). Spaces are respected by the test
%   and the \meta{token} will be removed from the input stream if the
%   test is true. The function will then place either the
%   \meta{true code} or \meta{false code} in the input stream (as
%   appropriate to the result of the test).
% \end{function}
%
% \begin{function}[TF]{\peek_charcode_remove_ignore_spaces:N}
%   \begin{syntax}
%     \cs{peek_charcode_remove_ignore_spaces:NTF} \meta{test token}
%     ~~\Arg{true code} \Arg{false code}
%   \end{syntax}
%    Tests if the next \meta{token} in the input stream has the same
%   character code as the \meta{test token} (as defined by the test
%   \cs{token_if_eq_charcode:NNTF}). Spaces are ignored by the test
%   and the \meta{token} will be removed from the input stream if the
%   test is true. The function will then place either the
%   \meta{true code} or \meta{false code} in the input stream (as
%   appropriate to the result of the test).
% \end{function}
%
% \begin{function}[TF]{\peek_meaning:N}
%   \begin{syntax}
%     \cs{peek_meaning:NTF} \meta{test token} \Arg{true code} \Arg{false code}
%   \end{syntax}
%   Tests if the next \meta{token} in the input stream has the same
%   meaning as the \meta{test token} (as defined by the test
%   \cs{token_if_eq_meaning:NNTF}). Spaces are respected by the test
%   and the \meta{token} will be left in the input stream after
%   the \meta{true code} or \meta{false code} (as appropriate to the
%   result of the test).
% \end{function}
%
% \begin{function}[TF]{\peek_meaning_ignore_spaces:N}
%   \begin{syntax}
%     \cs{peek_meaning_ignore_spaces:NTF} \meta{test token}
%     ~~\Arg{true code} \Arg{false code}
%   \end{syntax}
%   Tests if the next \meta{token} in the input stream has the same
%   meaning as the \meta{test token} (as defined by the test
%   \cs{token_if_eq_meaning:NNTF}). Spaces are ignored by the test
%   and the \meta{token} will be left in the input stream after
%   the \meta{true code} or \meta{false code} (as appropriate to the
%   result of the test).
% \end{function}
%
% \begin{function}[TF]{\peek_meaning_remove:N}
%   \begin{syntax}
%     \cs{peek_meaning_remove:NTF} \meta{test token}
%     ~~\Arg{true code} \Arg{false code}
%   \end{syntax}
%   Tests if the next \meta{token} in the input stream has the same
%   meaning as the \meta{test token} (as defined by the test
%   \cs{token_if_eq_meaning:NNTF}). Spaces are respected by the test
%   and the \meta{token} will be removed from the input stream if the
%   test is true. The function will then place either the
%   \meta{true code} or \meta{false code} in the input stream (as
%   appropriate to the result of the test).
% \end{function}
%
% \begin{function}[TF]{\peek_meaning_remove_ignore_spaces:N}
%   \begin{syntax}
%     \cs{peek_meaning_remove_ignore_spaces:NTF} \meta{test token}
%     ~~\Arg{true code} \Arg{false code}
%   \end{syntax}
%   Tests if the next \meta{token} in the input stream has the same
%   meaning as the \meta{test token} (as defined by the test
%   \cs{token_if_eq_meaning:NNTF}). Spaces are ignored by the test
%   and the \meta{token} will be removed from the input stream if the
%   test is true. The function will then place either the
%   \meta{true code} or \meta{false code} in the input stream (as
%   appropriate to the result of the test).
% \end{function}
%
% \section{Decomposing a macro definition}
%
% These functions decompose \TeX{} macros into their constituent
% parts: if the \meta{token} passed is not a macro then no decomposition
% can occur. In the later case, all three functions leave \cs{scan_stop:}
% in the input stream.
%
% \begin{function}[EXP]{\token_get_arg_spec:N}
%   \begin{syntax}
%     \cs{token_get_arg_spec:N} \meta{token}
%   \end{syntax}
%   If the \meta{token} is a macro, this function will leave
%   the primitive \TeX{} argument specification in input stream as
%   a string of tokens of category code $12$ (with spaces having category
%   code $10$). Thus for example for a token \cs{next} defined by
%   \begin{verbatim}
%     \cs_set:Npn \next #1#2 { x #1 y #2 }
%   \end{verbatim}
%   will leave |#1#2| in the input stream. If the \meta{token} is
%   not a macro then \cs{scan_stop:} will be left in the input stream
%   \begin{texnote}
%     If the arg~spec. contains the string |->|, then the |spec| function
%     will produce incorrect results.
%   \end{texnote}
% \end{function}
%
% \begin{function}[EXP]{\token_get_replacement_text:N}
%   \begin{syntax}
%     \cs{token_get_replacement_text:N} \meta{token}
%   \end{syntax}
%   If the \meta{token} is a macro, this function will leave
%   the replacement text in input stream as
%   a string of tokens of category code $12$ (with spaces having category
%   code $10$). Thus for example for a token \cs{next} defined by
%   \begin{verbatim}
%     \cs_set:Npn \next #1#2 { x #1~y #2 }
%   \end{verbatim}
%   will leave \verb|x#1 y#2| in the input stream. If the \meta{token} is
%   not a macro then \cs{scan_stop:} will be left in the input stream
% \end{function}
%
% \begin{function}[EXP]{\token_get_prefix_spec:N}
%   \begin{syntax}
%     \cs{token_get_prefix_spec:N} \meta{token}
%   \end{syntax}
%   If the \meta{token} is a macro, this function will leave
%   the \TeX{} prefixes applicable in input stream as
%   a string of tokens of category code $12$ (with spaces having category
%   code $10$). Thus for example for a token \cs{next} defined by
%   \begin{verbatim}
%     \cs_set:Npn \next #1#2 { x #1~y #2 }
%   \end{verbatim}
%   will leave |\long| in the input stream. If the \meta{token} is
%   not a macro then \cs{scan_stop:} will be left in the input stream
% \end{function}
%
% \begin{implementation}
%
% \section{\pkg{l3token} implementation}
%
%    \begin{macrocode}
%<*initex|package>
%    \end{macrocode}
%
%    \begin{macrocode}
%<*package>
\ProvidesExplPackage
  {\filename}{\filedate}{\fileversion}{\filedescription}
\package_check_loaded_expl:
%</package>
%    \end{macrocode}
%
% \subsection{Character tokens}
%
% \begin{macro}{\char_set_catcode:nn}
% \begin{macro}{\char_value_catcode:n}
% \begin{macro}{\char_show_value_catcode:n}
%   Category code changes.
%    \begin{macrocode}
\cs_new_protected_nopar:Npn \char_set_catcode:nn #1#2
  { \tex_catcode:D #1 = \int_eval:w #2 \int_eval_end: }
\cs_new_nopar:Npn \char_value_catcode:n #1
  { \tex_the:D \tex_catcode:D \int_eval:w #1\int_eval_end: }
\cs_new_nopar:Npn \char_show_value_catcode:n #1
  { \tex_showthe:D \tex_catcode:D \int_eval:w #1 \int_eval_end: }
%    \end{macrocode}
% \end{macro}
% \end{macro}
% \end{macro}
%
% \begin{macro}{ \char_make_escape:N         , \char_make_begin_group:N      ,
%                \char_make_end_group:N      , \char_make_math_shift:N       ,
%                \char_make_alignment:N      , \char_make_end_line:N         ,
%                \char_make_parameter:N      , \char_make_math_superscript:N ,
%                \char_make_math_subscript:N , \char_make_ignore:N           ,
%                \char_make_space:N          , \char_make_letter:N           ,
%                \char_make_other:N          , \char_make_active:N           ,
%                \char_make_comment:N        , \char_make_invalid:N          }
%    \begin{macrocode}
\cs_new_protected_nopar:Npn \char_make_escape:N #1
  { \char_set_catcode:nn { `#1 } \c_zero }
\cs_new_protected_nopar:Npn \char_make_begin_group:N #1
  { \char_set_catcode:nn { `#1 } \c_one }
\cs_new_protected_nopar:Npn \char_make_end_group:N #1
  { \char_set_catcode:nn { `#1 } \c_two }
\cs_new_protected_nopar:Npn \char_make_math_shift:N #1
  { \char_set_catcode:nn { `#1 } \c_three }
\cs_new_protected_nopar:Npn \char_make_alignment:N #1
  { \char_set_catcode:nn { `#1 } \c_four }
\cs_new_protected_nopar:Npn \char_make_end_line:N #1
  { \char_set_catcode:nn { `#1 } \c_five }
\cs_new_protected_nopar:Npn \char_make_parameter:N #1
  { \char_set_catcode:nn { `#1 } \c_six }
\cs_new_protected_nopar:Npn \char_make_math_superscript:N #1
  { \char_set_catcode:nn { `#1 } \c_seven }
\cs_new_protected_nopar:Npn \char_make_math_subscript:N #1
  { \char_set_catcode:nn { `#1 } \c_eight }
\cs_new_protected_nopar:Npn \char_make_ignore:N #1
  { \char_set_catcode:nn { `#1 } \c_nine }
\cs_new_protected_nopar:Npn \char_make_space:N #1
  { \char_set_catcode:nn { `#1 } \c_ten }
\cs_new_protected_nopar:Npn \char_make_letter:N #1
  { \char_set_catcode:nn { `#1 } \c_eleven }
\cs_new_protected_nopar:Npn \char_make_other:N #1
  { \char_set_catcode:nn { `#1 } \c_twelve }
\cs_new_protected_nopar:Npn \char_make_active:N #1
  { \char_set_catcode:nn { `#1 } \c_thirteen }
\cs_new_protected_nopar:Npn \char_make_comment:N #1
  { \char_set_catcode:nn { `#1 } \c_fourteen }
\cs_new_protected_nopar:Npn \char_make_invalid:N #1
  { \char_set_catcode:nn { `#1 } \c_fifteen }
%    \end{macrocode}
% \end{macro}
%
% \begin{macro}{ \char_make_escape:n         , \char_make_begin_group:n      ,
%                \char_make_end_group:n      , \char_make_math_shift:n       ,
%                \char_make_alignment:n      , \char_make_end_line:n         ,
%                \char_make_parameter:n      , \char_make_math_superscript:n ,
%                \char_make_math_subscript:n , \char_make_ignore:n           ,
%                \char_make_space:n          , \char_make_letter:n           ,
%                \char_make_other:n          , \char_make_active:n           ,
%                \char_make_comment:n        , \char_make_invalid:n          }
%    \begin{macrocode}
\cs_new_protected_nopar:Npn \char_make_escape:n #1
  { \char_set_catcode:nn {#1} \c_zero }
\cs_new_protected_nopar:Npn \char_make_begin_group:n #1
  { \char_set_catcode:nn {#1} \c_one }
\cs_new_protected_nopar:Npn \char_make_end_group:n #1
  { \char_set_catcode:nn {#1} \c_two }
\cs_new_protected_nopar:Npn \char_make_math_shift:n #1
  { \char_set_catcode:nn {#1} \c_three }
\cs_new_protected_nopar:Npn \char_make_alignment:n #1
  { \char_set_catcode:nn {#1} \c_four }
\cs_new_protected_nopar:Npn \char_make_end_line:n #1
  { \char_set_catcode:nn {#1} \c_five }
\cs_new_protected_nopar:Npn \char_make_parameter:n #1
  { \char_set_catcode:nn {#1} \c_six }
\cs_new_protected_nopar:Npn \char_make_math_superscript:n #1
  { \char_set_catcode:nn {#1} \c_seven }
\cs_new_protected_nopar:Npn \char_make_math_subscript:n #1
  { \char_set_catcode:nn {#1} \c_eight }
\cs_new_protected_nopar:Npn \char_make_ignore:n #1
  { \char_set_catcode:nn {#1} \c_nine }
\cs_new_protected_nopar:Npn \char_make_space:n #1
  { \char_set_catcode:nn {#1} \c_ten }
\cs_new_protected_nopar:Npn \char_make_letter:n #1
  { \char_set_catcode:nn {#1} \c_eleven }
\cs_new_protected_nopar:Npn \char_make_other:n #1
  { \char_set_catcode:nn {#1} \c_twelve }
\cs_new_protected_nopar:Npn \char_make_active:n #1
  { \char_set_catcode:nn {#1} \c_thirteen }
\cs_new_protected_nopar:Npn \char_make_comment:n #1
  { \char_set_catcode:nn {#1} \c_fourteen }
\cs_new_protected_nopar:Npn \char_make_invalid:n #1
  { \char_set_catcode:nn {#1} \c_fifteen }
%    \end{macrocode}
% \end{macro}
%
% \begin{macro}{\char_set_mathcode:nn}
% \begin{macro}{\char_value_mathcode:n}
% \begin{macro}{\char_show_value_mathcode:n}
% \begin{macro}{\char_set_lccode:nn}
% \begin{macro}{\char_value_lccode:n}
% \begin{macro}{\char_show_value_lccode:n}
% \begin{macro}{\char_set_uccode:nn}
% \begin{macro}{\char_value_uccode:n}
% \begin{macro}{\char_show_value_uccode:n}
% \begin{macro}{\char_set_sfcode:nn}
% \begin{macro}{\char_value_sfcode:n}
% \begin{macro}{\char_show_value_sfcode:n}
%   Pretty repetitive, but necessary!
%    \begin{macrocode}
\cs_new_protected_nopar:Npn \char_set_mathcode:nn #1#2
  { \tex_mathcode:D #1 = \int_eval:w #2 \int_eval_end: }
\cs_new_nopar:Npn \char_value_mathcode:n #1
  { \tex_the:D \tex_mathcode:D \int_eval:w #1\int_eval_end: }
\cs_new_nopar:Npn \char_show_value_mathcode:n #1
  { \tex_showthe:D \tex_mathcode:D \int_eval:w #1 \int_eval_end: }
\cs_new_protected_nopar:Npn \char_set_lccode:nn #1#2
  { \tex_lccode:D #1 = \int_eval:w #2 \int_eval_end: }
\cs_new_nopar:Npn \char_value_lccode:n #1
  { \tex_the:D \tex_lccode:D \int_eval:w #1\int_eval_end: }
\cs_new_nopar:Npn \char_show_value_lccode:n #1
  { \tex_showthe:D \tex_lccode:D \int_eval:w #1 \int_eval_end: }
\cs_new_protected_nopar:Npn \char_set_uccode:nn #1#2
  { \tex_uccode:D #1 = \int_eval:w #2 \int_eval_end: }
\cs_new_nopar:Npn \char_value_uccode:n #1
  { \tex_the:D \tex_uccode:D \int_eval:w #1\int_eval_end: }
\cs_new_nopar:Npn \char_show_value_uccode:n #1
  { \tex_showthe:D \tex_uccode:D \int_eval:w #1 \int_eval_end: }
\cs_new_protected_nopar:Npn \char_set_sfcode:nn #1#2
  { \tex_sfcode:D #1 = \int_eval:w #2 \int_eval_end: }
\cs_new_nopar:Npn \char_value_sfcode:n #1
  { \tex_the:D \tex_sfcode:D \int_eval:w #1\int_eval_end: }
\cs_new_nopar:Npn \char_show_value_sfcode:n #1
  { \tex_showthe:D \tex_sfcode:D \int_eval:w #1 \int_eval_end: }
%    \end{macrocode}
% \end{macro}
% \end{macro}
% \end{macro}
% \end{macro}
% \end{macro}
% \end{macro}
% \end{macro}
% \end{macro}
% \end{macro}
% \end{macro}
% \end{macro}
% \end{macro}
%
% \subsection{Generic tokens}
%
% \begin{macro}{\token_new:Nn}
%  Creates a new token.
%    \begin{macrocode}
\cs_new_protected_nopar:Npn \token_new:Nn #1#2 { \cs_new_eq:NN #1 #2 }
%    \end{macrocode}
% \end{macro}
%
% \begin{macro}
%   {
%     \c_group_begin_token,
%     \c_group_end_token,
%     \c_math_shift_token,
%     \c_alignment_tab_token,
%     \c_parameter_token,
%     \c_math_superscript_token,
%     \c_math_subscript_token,
%     \c_space_token,
%     \c_letter_token,
%     \c_other_char_token,
%     \c_active_char_token
%   }
%    We define these useful tokens. We have to do it by hand with the
%    brace tokens for obvious reasons.
%    \begin{macrocode}
\cs_new_eq:NN \c_group_begin_token {
\cs_new_eq:NN \c_group_end_token }
\group_begin:
  \char_make_math_shift:N \*
  \token_new:Nn \c_math_shift_token { * }
  \char_make_alignment:N \*
  \token_new:Nn \c_alignment_tab_token { * }
  \token_new:Nn \c_parameter_token { # }
  \token_new:Nn \c_math_superscript_token { ^ }
  \char_make_math_subscript:N \*
  \token_new:Nn \c_math_subscript_token { * }
  \token_new:Nn \c_space_token { ~ }
  \token_new:Nn \c_letter_token { a }
  \token_new:Nn \c_other_char_token { 1 }
  \char_make_active:N \*
  \cs_new_nopar:Npn \c_active_char_token { \exp_not:N * }
\group_end:
%    \end{macrocode}
% \end{macro}
%
% \subsection{Token conditionals}
%
% \begin{macro}[pTF]{\token_if_group_begin:N}
%   Check if token is a begin group token. We use the constant
%   |\c_group_begin_token| for this.
%    \begin{macrocode}
\prg_new_conditional:Npnn \token_if_group_begin:N #1 { p , T ,  F , TF }
  {
    \if_catcode:w \exp_not:N #1 \c_group_begin_token
      \prg_return_true: \else: \prg_return_false: \fi:
  }
%    \end{macrocode}
% \end{macro}
%
% \begin{macro}[pTF]{\token_if_group_end:N}
%   Check if token is a end group token. We use the constant
%   |\c_group_end_token| for this.
%    \begin{macrocode}
\prg_new_conditional:Npnn \token_if_group_end:N #1 { p , T ,  F , TF }
  {
    \if_catcode:w \exp_not:N #1 \c_group_end_token
      \prg_return_true: \else: \prg_return_false: \fi:
  }
%    \end{macrocode}
% \end{macro}
%
% \begin{macro}[pTF]{\token_if_math_shift:N}
%   Check if token is a math shift token. We use the constant
%   |\c_math_shift_token| for this.
%    \begin{macrocode}
\prg_new_conditional:Npnn \token_if_math_shift:N #1 { p , T ,  F , TF }
  {
    \if_catcode:w \exp_not:N #1 \c_math_shift_token
      \prg_return_true: \else: \prg_return_false: \fi:
  }
%    \end{macrocode}
% \end{macro}
%
% \begin{macro}[pTF]{\token_if_alignment_tab:N}
%   Check if token is an alignment tab token. We use the constant
%   |\c_alignment_tab_token| for this.
%    \begin{macrocode}
\prg_new_conditional:Npnn \token_if_alignment_tab:N #1 { p , T ,  F , TF }
  {
    \if_catcode:w \exp_not:N #1 \c_alignment_tab_token
      \prg_return_true: \else: \prg_return_false: \fi:
  }
%    \end{macrocode}
% \end{macro}
%
% \begin{macro}[pTF]{\token_if_parameter:N}
%   Check if token is a parameter token. We use the constant
%   |\c_parameter_token| for this. We have to trick \TeX{} a bit to
%   avoid an error message.
%    \begin{macrocode}
\prg_new_conditional:Npnn \token_if_parameter:N #1 { p , T ,  F , TF }
  {
    \exp_after:wN \if_catcode:w \cs:w c_parameter_token \cs_end: \exp_not:N #1
      \prg_return_true: \else: \prg_return_false: \fi:
  }
%    \end{macrocode}
% \end{macro}
%
% \begin{macro}[pTF]{\token_if_math_superscript:N}
%   Check if token is a math superscript token. We use the constant
%   |\c_math_superscript_token| for this.
%    \begin{macrocode}
\prg_new_conditional:Npnn \token_if_math_superscript:N #1 { p , T ,  F , TF }
  {
    \if_catcode:w \exp_not:N #1 \c_math_superscript_token
      \prg_return_true: \else: \prg_return_false: \fi:
  }
%    \end{macrocode}
% \end{macro}
%
% \begin{macro}[pTF]{\token_if_math_subscript:N}
%   Check if token is a math subscript token. We use the constant
%   |\c_math_subscript_token| for this.
%    \begin{macrocode}
\prg_new_conditional:Npnn \token_if_math_subscript:N #1 { p , T ,  F , TF }
  {
    \if_catcode:w \exp_not:N #1 \c_math_subscript_token
      \prg_return_true: \else: \prg_return_false: \fi:
  }
%    \end{macrocode}
% \end{macro}
%
% \begin{macro}[TF]{\token_if_space:N}
%   Check if token is a space token. We use the constant
%   |\c_space_token| for this.
%    \begin{macrocode}
\prg_new_conditional:Npnn \token_if_space:N #1 { p , T ,  F , TF }
  {
    \if_catcode:w \exp_not:N #1 \c_space_token
      \prg_return_true: \else: \prg_return_false: \fi:
  }
%    \end{macrocode}
% \end{macro}
%
% \begin{macro}[pTF]{\token_if_letter:N}
%   Check if token is a letter token. We use the constant
%   |\c_letter_token| for this.
%    \begin{macrocode}
\prg_new_conditional:Npnn \token_if_letter:N #1 { p , T ,  F , TF }
  {
    \if_catcode:w \exp_not:N #1\c_letter_token
      \prg_return_true: \else: \prg_return_false: \fi:
  }
%    \end{macrocode}
% \end{macro}
%
% \begin{macro}[TF]{\token_if_other_char:N}
%   Check if token is an other char token. We use the constant
%   |\c_other_char_token| for this.
%    \begin{macrocode}
\prg_new_conditional:Npnn \token_if_other_char:N #1 { p , T ,  F , TF }
  {
    \if_catcode:w \exp_not:N #1 \c_other_char_token
      \prg_return_true: \else: \prg_return_false: \fi:
  }
%    \end{macrocode}
% \end{macro}
%
% \begin{macro}[pTF]{\token_if_active_char:N}
%   Check if token is an active char token. We use the constant
%   |\c_active_char_token| for this.
%    \begin{macrocode}
\prg_new_conditional:Npnn \token_if_active_char:N #1 { p , T ,  F , TF }
  {
    \if_catcode:w \exp_not:N #1 \c_active_char_token
      \prg_return_true: \else: \prg_return_false: \fi:
  }
%    \end{macrocode}
% \end{macro}
%
% \begin{macro}[pTF]{\token_if_eq_meaning:NN}
%   Check if the tokens |#1| and |#2| have same meaning.
%    \begin{macrocode}
\prg_new_conditional:Npnn \token_if_eq_meaning:NN #1#2 { p , T ,  F , TF }
  {
    \if_meaning:w  #1  #2
      \prg_return_true: \else: \prg_return_false: \fi:
  }
%    \end{macrocode}
% \end{macro}
%
% \begin{macro}[pTF]{\token_if_eq_catcode:NN}
%  Check if the tokens |#1| and |#2| have same category code.
%    \begin{macrocode}
\prg_new_conditional:Npnn \token_if_eq_catcode:NN #1#2 { p , T ,  F , TF }
  {
    \if_catcode:w \exp_not:N #1 \exp_not:N #2
      \prg_return_true: \else: \prg_return_false: \fi:
  }
%    \end{macrocode}
% \end{macro}
%
% \begin{macro}[pTF]{\token_if_eq_charcode:NN}
%  Check if the tokens |#1| and |#2| have same character code.
%    \begin{macrocode}
\prg_new_conditional:Npnn \token_if_eq_charcode:NN #1#2 { p , T ,  F , TF }
  {
    \if_charcode:w \exp_not:N #1 \exp_not:N #2
      \prg_return_true: \else: \prg_return_false: \fi:
  }
%    \end{macrocode}
% \end{macro}
%
% \begin{macro}[pTF]{\token_if_macro:N}
% \begin{macro}[aux]{\token_if_macro_p_aux:w}
%   When a token is a macro, |\token_to_meaning:N| will always output
%   something like |\long macro:#1->#1| so we simply check to see if
%   the meaning contains |->|. Argument |#2| in the code below will be
%   empty if the string |->| isn't present, proof that the token was
%   not a macro (which is why we reverse the emptiness test). However
%   this function will fail on its own auxiliary function (and a few
%   other private functions as well) but that should certainly never
%   be a problem!
%    \begin{macrocode}
\prg_new_conditional:Npnn \token_if_macro:N #1 { p , T ,  F , TF }
  { \exp_after:wN \token_if_macro_p_aux:w \token_to_meaning:N #1 -> \q_stop }
\cs_new_nopar:Npn \token_if_macro_p_aux:w #1 -> #2 \q_stop
  {
    \if_predicate:w \tl_if_empty_p:n {#2}
      \prg_return_false: \else: \prg_return_true: \fi:
  }
%    \end{macrocode}
%  \end{macro}
%  \end{macro}
%
% \begin{macro}[pTF]{\token_if_cs:N}
%   Check if token has same catcode as a control sequence. We use
%   |\scan_stop:| for this.
%    \begin{macrocode}
\prg_new_conditional:Npnn \token_if_cs:N #1 { p , T ,  F , TF }
  {
    \if_predicate:w \token_if_eq_catcode_p:NN \scan_stop: #1
      \prg_return_true: \else: \prg_return_false: \fi:
  }
%    \end{macrocode}
% \end{macro}
%
% \begin{macro}[pTF]{\token_if_expandable:N}
%   Check if token is expandable. We use the fact that \TeX{} will
%   temporarily convert |\exp_not:N| \meta{token} into |\scan_stop:| if
%   \meta{token} is expandable.
%    \begin{macrocode}
\prg_new_conditional:Npnn \token_if_expandable:N #1 { p , T ,  F , TF }
  {
    \cs_if_exist:NTF #1
      {
        \exp_after:wN \if_meaning:w \exp_not:N #1 #1
          \prg_return_false: \else: \prg_return_true: \fi:
      }
      { \prg_return_false: }
  }
%    \end{macrocode}
% \end{macro}
%
% \begin{macro}[pTF]{\token_if_chardef:N,\token_if_mathchardef:N,
%   \token_if_long_macro:N,             \token_if_protected_macro:N,
%   \token_if_protected_long_macro:N,   \token_if_dim_register:N,
%   \token_if_skip_register:N,          \token_if_int_register:N,
%   \token_if_toks_register:N}
% \begin{macro}[aux]{
%               \token_if_chardef_p_aux:w,
%               \token_if_mathchardef_p_aux:w,
%               \token_if_int_register_p_aux:w,
%               \token_if_skip_register_p_aux:w,
%               \token_if_dim_register_p_aux:w,
%               \token_if_toks_register_p_aux:w,
%               \token_if_protected_macro_p_aux:w,
%               \token_if_long_macro_p_aux:w,
%               \token_if_protected_long_macro_p_aux:w}
%   Most of these functions have to check the meaning of the token in
%   question so we need to do some checkups on which characters are
%   output by |\token_to_meaning:N|. As usual, these characters have
%   catcode 12 so we must do some serious substitutions in the code
%   below\dots
%    \begin{macrocode}
\group_begin:
  \char_set_lccode:nn { `\T } { `\T }
  \char_set_lccode:nn { `\F } { `\F }
  \char_set_lccode:nn { `\X } { `\n }
  \char_set_lccode:nn { `\Y } { `\t }
  \char_set_lccode:nn { `\Z } { `\d }
  \char_set_lccode:nn { `\? } { `\\ }
  \tl_map_inline:nn { \X \Y \Z \M \C \H \A \R \O \U \S \K \I \P \L \G \P \E }
    { \char_set_catcode:nn { `#1 } \c_twelve }
%    \end{macrocode}
%   We convert the token list to lower case and restore the catcode and
%   lowercase code changes.
%    \begin{macrocode}
\tl_to_lowercase:n
  {
    \group_end:
%    \end{macrocode}
%   First up is checking if something has been defined with
%   |\tex_chardef:D| or |\tex_mathchardef:D|. This is easy since \TeX{}
%   thinks of such tokens as hexadecimal so it stores them as
%   |\char"|\meta{hex~number} or |\mathchar"|\meta{hex~number}.
%    \begin{macrocode}
    \prg_new_conditional:Npnn \token_if_chardef:N #1 { p , T ,  F , TF }
      {
        \exp_after:wN \token_if_chardef_aux:w
          \token_to_meaning:N #1 ?CHAR" \q_stop
      }
    \cs_new_nopar:Npn \token_if_chardef_aux:w #1 ?CHAR" #2 \q_stop
      { \tl_if_empty:nTF {#1} { \prg_return_true: } { \prg_return_false: } }
%    \end{macrocode}
%
%    \begin{macrocode}
    \prg_new_conditional:Npnn \token_if_mathchardef:N #1 { p , T ,  F , TF }
      {
        \exp_after:wN \token_if_mathchardef_aux:w
          \token_to_meaning:N #1 ?MAYHCHAR" \q_stop
      }
    \cs_new_nopar:Npn \token_if_mathchardef_aux:w #1 ?MAYHCHAR" #2 \q_stop
      { \tl_if_empty:nTF {#1} { \prg_return_true: } { \prg_return_false: } }
%    \end{macrocode}
%   Integer registers are a little more difficult since they expand to
%   |\count|\meta{number} and there is also a primitive |\countdef|. So
%   we have to check for that primitive as well.
%    \begin{macrocode}
    \prg_new_conditional:Npnn \token_if_int_register:N #1 { p , T ,  F , TF }
      {
        \if_meaning:w \tex_countdef:D #1
          \prg_return_false:
        \else:
          \exp_after:wN \token_if_int_register_aux:w
            \token_to_meaning:N #1 ?COUXY \q_stop
        \fi:
      }
    \cs_new_nopar:Npn \token_if_int_register_aux:w #1 ?COUXY #2 \q_stop
      { \tl_if_empty:nTF {#1} { \prg_return_true: } { \prg_return_false: } }
%    \end{macrocode}
%   Skip registers are done the same way as the integer registers.
%    \begin{macrocode}
    \prg_new_conditional:Npnn \token_if_skip_register:N #1 { p , T ,  F , TF }
      {
        \if_meaning:w \tex_skipdef:D #1
          \prg_return_false:
        \else:
          \exp_after:wN \token_if_skip_register_aux:w
            \token_to_meaning:N #1?SKIP\q_stop
        \fi:
      }
    \cs_new_nopar:Npn \token_if_skip_register_aux:w #1 ?SKIP #2 \q_stop
      { \tl_if_empty:nTF {#1} { \prg_return_true: } { \prg_return_false: } }
%    \end{macrocode}
%   Dim registers. No news here
%    \begin{macrocode}
    \prg_new_conditional:Npnn \token_if_dim_register:N #1 { p , T ,  F , TF }
      {
        \if_meaning:w \tex_dimendef:D #1
          \c_false_bool
        \else:
          \exp_after:wN \token_if_dim_register_aux:w
            \token_to_meaning:N #1 ?ZIMEX \q_stop
        \fi:
      }
    \cs_new_nopar:Npn \token_if_dim_register_aux:w #1 ?ZIMEX #2 \q_stop
      { \tl_if_empty:nTF {#1} { \prg_return_true: } { \prg_return_false: } }
%    \end{macrocode}
%   Toks registers.
%    \begin{macrocode}
    \prg_new_conditional:Npnn \token_if_toks_register:N #1 { p , T ,  F , TF }
      {
        \if_meaning:w \tex_toksdef:D #1
          \prg_return_false:
        \else:
          \exp_after:wN \token_if_toks_register_aux:w
            \token_to_meaning:N #1 ?YOKS \q_stop
        \fi:
      }
     \cs_new_nopar:Npn \token_if_toks_register_aux:w #1 ?YOKS #2 \q_stop
       { \tl_if_empty:nTF {#1} { \prg_return_true: } { \prg_return_false: } }
%    \end{macrocode}
%   Protected macros.
%    \begin{macrocode}
    \prg_new_conditional:Npnn \token_if_protected_macro:N #1
      { p , T ,  F , TF }
      {
        \exp_after:wN \token_if_protected_macro_aux:w
          \token_to_meaning:N #1 ?PROYECYEZ~MACRO \q_stop
      }
    \cs_new_nopar:Npn \token_if_protected_macro_aux:w
      #1 ?PROYECYEZ~MACRO #2 \q_stop
      { \tl_if_empty:nTF {#1} { \prg_return_true: } { \prg_return_false: } }
%    \end{macrocode}
%   Long macros.
%    \begin{macrocode}
    \prg_new_conditional:Npnn \token_if_long_macro:N #1 { p , T ,  F , TF }
      {
        \exp_after:wN \token_if_long_macro_aux:w
          \token_to_meaning:N #1 ?LOXG~MACRO \q_stop
      }
    \cs_new_nopar:Npn \token_if_long_macro_aux:w #1 ?LOXG~MACRO #2 \q_stop
      { \tl_if_empty:nTF {#1} { \prg_return_true: } { \prg_return_false: } }
%    \end{macrocode}
%   Finally protected long macros where we for once don't have to add an
%   extra test since there is no primitive for the combined prefixes.
%    \begin{macrocode}
    \prg_new_conditional:Npnn \token_if_protected_long_macro:N #1
      { p , T ,  F , TF }
      {
        \exp_after:wN \token_if_protected_long_macro_aux:w
          \token_to_meaning:N #1 ?PROYECYEZ?LOXG~MACRO \q_stop
      }
    \cs_new_nopar:Npn \token_if_protected_long_macro_aux:w
      #1 ?PROYECYEZ?LOXG~MACRO #2 \q_stop
      { \tl_if_empty:nTF {#1} { \prg_return_true: } { \prg_return_false: } }
%    \end{macrocode}
% Finally the |\tl_to_lowercase:n| ends!
%    \begin{macrocode}
  }
%    \end{macrocode}
% \end{macro}
% \end{macro}
% \end{macro}
%
% \begin{macro}[pTF]{\token_if_primitive:N}
% \begin{macro}[aux]{\token_if_primitive_p_aux:N}
%   It is rather hard to determine if a token is a primitive. First we
%   can check if it is a control sequence or active character. If
%   either, we check if it is a macro. Then we can go through a
%   tedious process of testing for different register types\dots{} I
%   don't actually think this function is useful but you never know.
%    \begin{macrocode}
\prg_new_conditional:Npnn \token_if_primitive:N #1 { p , T , F , TF }
  {
    \if_predicate:w \token_if_cs_p:N #1
      \if_predicate:w \token_if_macro_p:N #1
        \prg_return_false:
      \else:
        \token_if_primitive_p_aux:N #1
      \fi:
    \else:
      \if_predicate:w \token_if_active_char_p:N #1
        \if_predicate:w \token_if_macro_p:N #1
          \prg_return_false:
        \else:
          \token_if_primitive_p_aux:N #1
        \fi:
      \else:
        \prg_return_false:
      \fi:
    \fi:
  }
\cs_new_nopar:Npn \token_if_primitive_p_aux:N #1
  {
    \if_predicate:w \token_if_chardef_p:N #1 \c_false_bool
    \else:
      \if_predicate:w \token_if_mathchardef_p:N #1 \prg_return_false:
      \else:
        \if_predicate:w \token_if_int_register_p:N #1 \prg_return_false:
        \else:
          \if_predicate:w \token_if_skip_register_p:N #1 \prg_return_false:
          \else:
            \if_predicate:w \token_if_dim_register_p:N #1 \prg_return_false:
            \else:
              \if_predicate:w \token_if_toks_register_p:N #1 \prg_return_false:
              \else:
%    \end{macrocode}
% We made it!
%    \begin{macrocode}
                \prg_return_true:
              \fi:
            \fi:
          \fi:
        \fi:
      \fi:
    \fi:
  }
%    \end{macrocode}
% \end{macro}
% \end{macro}
%
% \subsection{Peeking ahead at the next token}
%
% Peeking ahead is implemented using a two part mechanism. The
% outer level provides a defined interface to the lower level material.
% This allows a large amount of code to be shared. There are four
% cases:
% \begin{enumerate}
%   \item peek at the next token;
%   \item peek at the next non-space token;
%   \item peek at the next token and remove it;
%   \item peek at the next non-space token and remove it.
% \end{enumerate}
%
% \begin{variable}{\l_peek_token}
% \begin{variable}{\g_peek_token}
%   Storage tokens which are publicly documented: the token peeked.
%    \begin{macrocode}
\cs_new_eq:NN \l_peek_token ?
\cs_new_eq:NN \g_peek_token ?
%    \end{macrocode}
% \end{variable}
% \end{variable}
%
% \begin{variable}{\l_peek_search_token}
%   The token to search for as an implicit token:
%   \emph{cf.}~\cs{l_peek_search_tl}.
%    \begin{macrocode}
\cs_new_eq:NN \l_peek_search_token ?
%    \end{macrocode}
% \end{variable}
%
% \begin{variable}{\l_peek_search_tl}
%   The token to search for as an explicit token:
%   \emph{cf.}~\cs{l_peek_search_token}.
%    \begin{macrocode}
\cs_new_nopar:Npn \l_peek_search_tl { }
%    \end{macrocode}
% \end{variable}
%
% \begin{macro}[aux]
%   {\peek_true:w, \peek_true_aux:w, \peek_false:w, \peek_tmp:w}
%   Functions used by the branching and space-stripping code.
%    \begin{macrocode}
\cs_new_nopar:Npn \peek_true:w  { }
\cs_new_nopar:Npn \peek_true_aux:w  { }
\cs_new_nopar:Npn \peek_false:w { }
\cs_new:Npn \peek_tmp:w { }
%    \end{macrocode}
% \end{macro}
%
% \begin{macro}{\peek_after:Nw}
% \begin{macro}{\peek_after:Nw}
%   Simple wrappers for \cs{tex_futurelet:D}: no arguments absorbed
%   here.
%    \begin{macrocode}
\cs_new_protected_nopar:Npn \peek_after:Nw
  { \tex_futurelet:D \l_peek_token }
\cs_new_protected_nopar:Npn \peek_gafter:Nw
  { \pref_global:D \tex_futurelet:D \g_peek_token }
%    \end{macrocode}
% \end{macro}
% \end{macro}
%
% \begin{macro}[aux]{\peek_true_remove:w}
%   A function to remove the next token and then regain control.
%    \begin{macrocode}
\cs_new_protected:Npn \peek_true_remove:w
  {
    \group_align_safe_end:
    \tex_afterassignment:D \peek_true_aux:w
    \cs_set_eq:NN \peek_tmp:w
  }
%    \end{macrocode}
% \end{macro}
%
% \begin{macro}[TF]{\peek_token_generic:NN}
%   The generic function stores the test token in both implicit and
%   explicit modes, and the \texttt{true} and \texttt{false} code as
%   token lists, more or less. The two branches have to be absorbed here
%   as the input stream needs to be cleared for the peek function itself.
%    \begin{macrocode}
\cs_new_protected:Npn \peek_token_generic:NNTF #1#2#3#4
  {
    \cs_set_eq:NN \l_peek_search_token #2
    \tl_set:Nn \l_peek_search_tl {#2}
    \cs_set_nopar:Npx \peek_true:w
      {
        \exp_not:N \group_align_safe_end:
        \exp_not:n {#3}
      }
    \cs_set_nopar:Npx \peek_false:w
      {
        \exp_not:N \group_align_safe_end:
        \exp_not:n {#4}
      }
    \group_align_safe_begin:
      \peek_after:Nw #1
  }
\cs_new_protected:Npn \peek_token_generic:NNT #1#2#3
  { \peek_token_generic:NNTF #1 #2 {#3} { } }
\cs_new_protected:Npn \peek_token_generic:NNF #1#2#3
  { \peek_token_generic:NNTF #1 #2 { } {#3} }
%    \end{macrocode}
% \end{macro}
%
% \begin{macro}[TF]{\peek_token_remove_generic:NN}
%   For token removal there needs to be a call to the auxiliary
%   function which does the work.
%    \begin{macrocode}
\cs_new_protected:Npn \peek_token_remove_generic:NNTF #1#2#3#4
  {
    \cs_set_eq:NN \l_peek_search_token #2
    \tl_set:Nn \l_peek_search_tl {#2}
    \cs_set_eq:NN \peek_true:w \peek_true_remove:w
    \cs_set_nopar:Npx \peek_true_aux:w { \exp_not:n {#3} }
    \cs_set_nopar:Npx \peek_false:w
      {
        \group_align_safe_end:
        \exp_not:n {#4}
      }
    \group_align_safe_begin:
      \peek_after:Nw #1
  }
\cs_new_protected:Npn \peek_token_remove_generic:NNT #1#2#3
  { \peek_token_remove_generic:NNTF #1 #2 {#3} { } }
\cs_new_protected:Npn \peek_token_remove_generic:NNF #1#2#3
  { \peek_token_remove_generic:NNTF #1 #2 { } {#3} }
%    \end{macrocode}
% \end{macro}
%
% \begin{macro}
%   {\peek_execute_branches_catcode:, \peek_execute_branches_meaning:}
%   The category code and meaning tests are straight forward.
%    \begin{macrocode}
\cs_new_nopar:Npn \peek_execute_branches_catcode:
  {
    \if_catcode:
      \exp_not:N \l_peek_token \exp_not:N \l_peek_search_token
      \exp_after:wN \peek_true:w
    \else:
      \exp_after:wN \peek_false:w
    \fi:
  }
\cs_new_nopar:Npn \peek_execute_branches_meaning:
  {
    \if_meaning:w \l_peek_token \l_peek_search_token
      \exp_after:wN \peek_true:w
    \else:
      \exp_after:wN \peek_false:w
    \fi:
  }
%    \end{macrocode}
% \end{macro}
%
% \begin{macro}{\peek_execute_branches_charcode:}
% \begin{macro}[aux]{\peek_execute_branches_charcode:NN}
%   First the character code test there is a need to worry about \TeX{}
%   grabbing brace group or skipping spaces. These are all tested for
%   using a category code check before grabbing what must be a real
%   single token and doing the comparison.
%    \begin{macrocode}
\cs_new_nopar:Npn \peek_execute_branches_charcode:
  {
    \bool_if:nTF
      {
           \token_if_eq_catcode_p:NN \l_peek_token \c_group_begin_token
        || \token_if_eq_meaning_p:NN \l_peek_token \c_space_token
        || \token_if_eq_catcode_p:NN \l_peek_token \c_group_end_token
      }
      { \peek_false:w }
      {
        \exp_after:wN \peek_execute_branches_charcode_aux:NN
          \l_peek_search_tl
      }
  }
\cs_new:Npn \peek_execute_branches_charcode_aux:NN #1#2
  {
    \if: \exp_not:N #1 \exp_not:N #2
      \exp_after:wN \peek_true:w
    \else:
      \exp_after:wN \peek_false:w
    \fi:
    #2
  }
%    \end{macrocode}
% \end{macro}
% \end{macro}
%
% \begin{macro}{\peek_ignore_spaces_execute_branches:}
% \begin{macro}[aux]{\peek_ignore_spaces_execute_branches_aux:}
%   This function removes one token at a time with a mechanism that can
%   be applied to things other than spaces.
%    \begin{macrocode}
\cs_new_protected_nopar:Npn \peek_ignore_spaces_execute_branches:
  {
    \token_if_eq_meaning:NNTF \l_peek_token \c_space_token
      {
        \tex_afterassignment:D \peek_ignore_spaces_execute_branches_aux:
        \cs_set_eq:NN \peek_tmp:w
      }
      { \peek_execute_branches: }
  }
\cs_new_protected_nopar:Npn \peek_ignore_spaces_execute_branches_aux:
  { \peek_after:Nw \peek_ignore_spaces_execute_branches: }
%    \end{macrocode}
% \end{macro}
% \end{macro}
%
% \begin{macro}[aux]{\peek_def:nnnn}
% \begin{macro}[aux]{\peek_def_aux:nnnnn}
%   The public functions themselves cannot be defined using
%   \cs{prg_set_conditional:Npnn} and so a couple of auxiliary functions
%   are used. As a result, everything is done inside a group. As a result
%   things are a bit complicated.
%    \begin{macrocode}
\group_begin:
  \cs_set_nopar:Npn \peek_def:nnnn #1#2#3#4
    {
      \peek_def_aux:nnnnn {#1} {#2} {#3} {#4} { TF }
      \peek_def_aux:nnnnn {#1} {#2} {#3} {#4} { T }
      \peek_def_aux:nnnnn {#1} {#2} {#3} {#4} { F }
    }
  \cs_set_nopar:Npn \peek_def_aux:nnnnn #1#2#3#4#5
    {
      \cs_gset_nopar:cpx { #1 #5 }
        {
          \tl_if_empty:nF {#2}
            { \exp_not:n { \cs_set_eq:NN \peek_execute_branches: #2 } }
          \exp_not:c { #3 #5 }
          \exp_not:n {#4}
        }
    }
%    \end{macrocode}
% \end{macro}
% \end{macro}
% \begin{macro}[TF]
%   {
%     \peek_catcode:N, \peek_catcode_ignore_spaces:N,
%     \peek_catcode_remove:N, \peek_catcode_remove_ignore_spaces:N
%   }
%   With everything in place the definitions can take place. First for
%   category codes.
%    \begin{macrocode}
  \peek_def:nnnn { peek_catcode:N }
    { }
    { peek_token_generic:NN }
    { \peek_execute_branches_catcode: }
  \peek_def:nnnn { peek_catcode_ignore_spaces:N }
    { \peek_execute_branches_catcode: }
    { peek_token_generic:NN }
    { \peek_ignore_spaces_execute_branches: }
  \peek_def:nnnn { peek_catcode_remove:N }
    { }
    { peek_token_remove_generic:NN }
    { \peek_execute_branches_catcode: }
  \peek_def:nnnn { peek_catcode_remove_ignore_spaces:N }
    { \peek_execute_branches_catcode: }
    { peek_token_remove_generic:NN }
    { \peek_ignore_spaces_execute_branches: }
%    \end{macrocode}
% \end{macro}
% \begin{macro}[TF]
%   {
%     \peek_charcode:N, \peek_charcode_ignore_spaces:N,
%     \peek_charcode_remove:N, \peek_charcode_remove_ignore_spaces:N
%   }
%   Then for character codes.
%    \begin{macrocode}
  \peek_def:nnnn { peek_charcode:N }
    { }
    { peek_token_generic:NN }
    { \peek_execute_branches_charcode: }
  \peek_def:nnnn { peek_charcode_ignore_spaces:N }
    { \peek_execute_branches_charcode: }
    { peek_token_generic:NN }
    { \peek_ignore_spaces_execute_branches: }
  \peek_def:nnnn { peek_charcode_remove:N }
    { }
    { peek_token_remove_generic:NN }
    { \peek_execute_branches_charcode: }
  \peek_def:nnnn { peek_charcode_remove_ignore_spaces:N }
    { \peek_execute_branches_charcode: }
    { peek_token_remove_generic:NN }
    { \peek_ignore_spaces_execute_branches: }
%    \end{macrocode}
% \end{macro}
% \begin{macro}[TF]
%   {
%     \peek_meaning:N, \peek_meaning_ignore_spaces:N,
%     \peek_meaning_remove:N, \peek_meaning_remove_ignore_spaces:N
%   }
%   Finally for meaning, with the group closed to remove the temporary
%   definition functions.
%    \begin{macrocode}
  \peek_def:nnnn { peek_meaning:N }
    { }
    { peek_token_generic:NN }
    { \peek_execute_branches_meaning: }
  \peek_def:nnnn { peek_meaning_ignore_spaces:N }
    { \peek_execute_branches_meaning: }
    { peek_token_generic:NN }
    { \peek_ignore_spaces_execute_branches: }
  \peek_def:nnnn { peek_meaning_remove:N }
    { }
    { peek_token_remove_generic:NN }
    { \peek_execute_branches_meaning: }
  \peek_def:nnnn { peek_meaning_remove_ignore_spaces:N }
    { \peek_execute_branches_meaning: }
    { peek_token_remove_generic:NN }
    { \peek_ignore_spaces_execute_branches: }
\group_end:
%    \end{macrocode}
% \end{macro}
%
% \subsection{Decomposing a macro definition}
%
% \begin{macro}{\token_get_prefix_spec:N}
% \begin{macro}{\token_get_arg_spec:N}
% \begin{macro}{\token_get_replacement_spec:N}
% \begin{macro}[aux]{\token_get_prefix_arg_replacement_aux:wN}
%   We sometimes want to test if a
%   control sequence can be expanded to reveal a hidden
%   value. However, we cannot just expand the macro blindly as it may
%   have arguments and none might be present. Therefore we define
%   these functions to pick either the prefix(es), the argument
%   specification, or the replacement text from a macro. All of this
%   information is returned as characters with catcode~$12$. If the
%   token in question isn't a macro, the token |\scan_stop:| is
%   returned instead.
%    \begin{macrocode}
\group_begin:
  \char_set_lccode:nn { `\? } { `\: }
  \char_set_catcode:nn { `\M } \c_twelve
  \char_set_catcode:nn { `\A } \c_twelve
  \char_set_catcode:nn { `\C } \c_twelve
  \char_set_catcode:nn { `\R } \c_twelve
  \char_set_catcode:nn { `\O } \c_twelve
\tl_to_lowercase:n
  {
    \group_end:
    \cs_new_nopar:Npn \token_get_prefix_arg_replacement_aux:wN
      #1 MACRO? #2 -> #3 \q_stop #4
      { #4 {#1} {#2} {#3} }
    \cs_new:Npn \token_get_prefix_spec:N #1
      {
        \token_if_macro:NTF #1
          {
            \exp_after:wN \token_get_prefix_arg_replacement_aux:wN
              \token_to_meaning:N #1 \q_stop \use_i:nnn
          }
          { \scan_stop: }
      }
    \cs_new:Npn \token_get_arg_spec:N #1
      {
        \token_if_macro:NTF #1
          {
            \exp_after:wN \token_get_prefix_arg_replacement_aux:wN
              \token_to_meaning:N #1 \q_stop \use_ii:nnn
          }
          { \scan_stop: }
      }
    \cs_new:Npn \token_get_replacement_spec:N #1
      {
        \token_if_macro:NTF #1
          {
            \exp_after:wN \token_get_prefix_arg_replacement_aux:wN
              \token_to_meaning:N #1 \q_stop \use_iii:nnn
          }
          { \scan_stop: }
      }
  }
%    \end{macrocode}
% \end{macro}
% \end{macro}
% \end{macro}
% \end{macro}
%
% \subsection{Depreciated functions}
%
%  Depreciated on 0000-00-00, for removal on or before 0000-00-00.
%
% \begin{macro}{\char_set_catcode:w}
% \begin{macro}{\char_set_mathcode:w}
% \begin{macro}{\char_set_lccode:w}
% \begin{macro}{\char_set_uccode:w}
% \begin{macro}{\char_set_sfcode:w}
%   Primitives renamed.
%    \begin{macrocode}
\cs_new_eq:NN \char_set_catcode:w  \tex_catcode:D
\cs_new_eq:NN \char_set_mathcode:w \tex_mathcode:D
\cs_new_eq:NN \char_set_lccode:w   \tex_lccode:D
\cs_new_eq:NN \char_set_uccode:w   \tex_uccode:D
\cs_new_eq:NN \char_set_sfcode:w   \tex_sfcode:D
%    \end{macrocode}
% \end{macro}
% \end{macro}
% \end{macro}
% \end{macro}
% \end{macro}
%
% \begin{macro}{\char_value_catcode:w}
% \begin{macro}{\char_show_value_catcode:w}
% \begin{macro}{\char_value_mathcode:w}
% \begin{macro}{\char_show_value_mathcode:w}
% \begin{macro}{\char_value_lccode:w}
% \begin{macro}{\char_show_value_lccode:w}
% \begin{macro}{\char_value_uccode:w}
% \begin{macro}{\char_show_value_uccode:w}
% \begin{macro}{\char_value_sfcode:w}
% \begin{macro}{\char_show_value_sfcode:w}
%   More |w| functions we should not have.
%    \begin{macrocode}
\cs_new_nopar:Npn \char_value_catcode:w { \tex_the:D \char_set_catcode:w }
\cs_new_nopar:Npn \char_show_value_catcode:w
  { \tex_showthe:D \char_set_catcode:w }
\cs_new_nopar:Npn \char_value_mathcode:w { \tex_the:D \char_set_mathcode:w }
\cs_new_nopar:Npn \char_show_value_mathcode:w
  { \tex_showthe:D \char_set_mathcode:w }
\cs_new_nopar:Npn \char_value_lccode:w { \tex_the:D \char_set_lccode:w }
\cs_new_nopar:Npn \char_show_value_lccode:w
  { \tex_showthe:D \char_set_lccode:w }
\cs_new_nopar:Npn \char_value_uccode:w { \tex_the:D \char_set_uccode:w }
\cs_new_nopar:Npn \char_show_value_uccode:w
  { \tex_showthe:D \char_set_uccode:w }
\cs_new_nopar:Npn \char_value_sfcode:w { \tex_the:D \char_set_sfcode:w }
\cs_new_nopar:Npn \char_show_value_sfcode:w
  { \tex_showthe:D \char_set_sfcode:w }
%    \end{macrocode}
% \end{macro}
% \end{macro}
% \end{macro}
% \end{macro}
% \end{macro}
% \end{macro}
% \end{macro}
% \end{macro}
% \end{macro}
% \end{macro}
%
% \begin{macro}{\peek_after:NN}
% \begin{macro}{\peek_gafter:NN}
%   The second argument here must be |w|.
%    \begin{macrocode}
\cs_new_eq:NN \peek_after:NN  \peek_after:Nw
\cs_new_eq:NN \peek_gafter:NN \peek_gafter:Nw
%    \end{macrocode}
% \end{macro}
% \end{macro}
%
%    \begin{macrocode}
%</initex|package>
%    \end{macrocode}
%
% \end{implementation}
%
% \PrintIndex
