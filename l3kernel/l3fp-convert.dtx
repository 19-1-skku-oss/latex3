% \iffalse meta-comment
%
%% File: l3fp-convert.dtx
%
% Copyright (C) 2011-2019 The LaTeX3 Project
%
% It may be distributed and/or modified under the conditions of the
% LaTeX Project Public License (LPPL), either version 1.3c of this
% license or (at your option) any later version.  The latest version
% of this license is in the file
%
%    https://www.latex-project.org/lppl.txt
%
% This file is part of the "l3kernel bundle" (The Work in LPPL)
% and all files in that bundle must be distributed together.
%
% -----------------------------------------------------------------------
%
% The development version of the bundle can be found at
%
%    https://github.com/latex3/latex3
%
% for those people who are interested.
%
%<*driver>
\documentclass[full,kernel]{l3doc}
\begin{document}
  \DocInput{\jobname.dtx}
\end{document}
%</driver>
% \fi
%
% \title{^^A
%   The \textsf{l3fp-convert} package\\ Floating point conversion^^A
% }
%
% \author{^^A
%  The \LaTeX3 Project\thanks
%    {^^A
%      E-mail:
%        \href{mailto:latex-team@latex-project.org}
%          {latex-team@latex-project.org}^^A
%    }^^A
% }
%
% \date{Released 2019-04-06}
%
% \maketitle
%
% \begin{documentation}
%
% \end{documentation}
%
% \begin{implementation}
%
% \section{\texttt{l3fp-convert} implementation}
%
%    \begin{macrocode}
%<*initex|package>
%    \end{macrocode}
%
%    \begin{macrocode}
%<@@=fp>
%    \end{macrocode}
%
% \subsection{Dealing with tuples}
%
% \begin{macro}[EXP]
%   {\@@_tuple_convert:Nw, \@@_tuple_convert_loop:nNw, \@@_tuple_convert_end:w}
%   The first argument is for instance \cs{@@_to_tl_dispatch:w}, which
%   converts any floating point object to the appropriate
%   representation.  We loop through all items, putting |,~| between all
%   of them and making sure to remove the leading |,~|.
%    \begin{macrocode}
\cs_new:Npn \@@_tuple_convert:Nw #1 \s_@@_tuple \@@_tuple_chk:w #2 ;
  {
    \int_case:nnF { \@@_array_count:n {#2} }
      {
        { 0 } { ( ) }
        { 1 } { \@@_tuple_convert_end:w @ { #1 #2 , } }
      }
      {
        \@@_tuple_convert_loop:nNw { } #1
          #2 { ? \@@_tuple_convert_end:w } ;
          @ { \use_none:nn }
      }
  }
\cs_new:Npn \@@_tuple_convert_loop:nNw #1#2#3#4; #5 @ #6
  {
    \use_none:n #3
    \exp_args:Nf \@@_tuple_convert_loop:nNw { #2 #3#4 ; } #2 #5
      @ { #6 , ~ #1 }
  }
\cs_new:Npn \@@_tuple_convert_end:w #1 @ #2
  { \exp_after:wN ( \exp:w \exp_end_continue_f:w #2 ) }
%    \end{macrocode}
% \end{macro}
%
% \subsection{Trimming trailing zeros}
%
% \begin{macro}[EXP]{\@@_trim_zeros:w}
% \begin{macro}[EXP]
%   {\@@_trim_zeros_loop:w, \@@_trim_zeros_dot:w, \@@_trim_zeros_end:w}
%   If |#1| ends with a $0$, the \texttt{loop} auxiliary takes that zero
%   as an end-delimiter for its first argument, and the second argument
%   is the same \texttt{loop} auxiliary.  Once the last trailing zero is
%   reached, the second argument is the \texttt{dot} auxiliary,
%   which removes a trailing dot if any.  We then clean-up with the
%   \texttt{end} auxiliary, keeping only the number.
%    \begin{macrocode}
\cs_new:Npn \@@_trim_zeros:w #1 ;
  {
    \@@_trim_zeros_loop:w #1
      ; \@@_trim_zeros_loop:w 0; \@@_trim_zeros_dot:w .; \s_stop
  }
\cs_new:Npn \@@_trim_zeros_loop:w #1 0; #2 { #2 #1 ; #2 }
\cs_new:Npn \@@_trim_zeros_dot:w #1 .; { \@@_trim_zeros_end:w #1 ; }
\cs_new:Npn \@@_trim_zeros_end:w #1 ; #2 \s_stop { #1 }
%    \end{macrocode}
% \end{macro}
% \end{macro}
%
% \subsection{Scientific notation}
%
% \begin{macro}[EXP]
%   {\fp_to_scientific:N, \fp_to_scientific:c, \fp_to_scientific:n}
%   The three public functions evaluate their argument, then pass it to
%   \cs{@@_to_scientific_dispatch:w}.
%    \begin{macrocode}
\cs_new:Npn \fp_to_scientific:N #1
  { \exp_after:wN \@@_to_scientific_dispatch:w #1 }
\cs_generate_variant:Nn \fp_to_scientific:N { c }
\cs_new:Npn \fp_to_scientific:n
  {
    \exp_after:wN \@@_to_scientific_dispatch:w
    \exp:w \exp_end_continue_f:w \@@_parse:n
  }
%    \end{macrocode}
% \end{macro}
%
% \begin{macro}[EXP]
%   {\@@_to_scientific_dispatch:w, \@@_to_scientific_recover:w, \@@_tuple_to_scientific:w}
%   We allow tuples.
%    \begin{macrocode}
\cs_new:Npn \@@_to_scientific_dispatch:w #1
  {
    \@@_change_func_type:NNN
      #1 \@@_to_scientific:w \@@_to_scientific_recover:w
    #1
  }
\cs_new:Npn \@@_to_scientific_recover:w #1 #2 ;
  {
    \@@_error:nffn { fp-unknown-type } { \tl_to_str:n { #2 ; } } { } { }
    nan
  }
\cs_new:Npn \@@_tuple_to_scientific:w
  { \@@_tuple_convert:Nw \@@_to_scientific_dispatch:w }
%    \end{macrocode}
% \end{macro}
%
% \begin{macro}[EXP]
%   {
%     \@@_to_scientific:w,
%     \@@_to_scientific_normal:wnnnnn,
%     \@@_to_scientific_normal:wNw
%   }
%   Expressing an internal floating point number in scientific notation
%   is quite easy: no rounding, and the format is very well defined.
%   First cater for the sign: negative numbers ($|#2|=2$) start
%   with~|-|; we then only need to care about positive numbers and
%   \texttt{nan}.  Then filter the special cases: $\pm0$~are represented
%   as~|0|; infinities are converted to a number slightly larger than
%   the largest after an \enquote{invalid_operation} exception;
%   \texttt{nan} is represented as~|0| after an
%   \enquote{invalid_operation} exception.  In the normal case,
%   decrement the exponent and unbrace the $4$ brace groups, then in a
%   second step grab the first digit (previously hidden in braces) to
%   order the various parts correctly.
%    \begin{macrocode}
\cs_new:Npn \@@_to_scientific:w \s_@@ \@@_chk:w #1#2
  {
    \if_meaning:w 2 #2 \exp_after:wN - \exp:w \exp_end_continue_f:w \fi:
    \if_case:w #1 \exp_stop_f:
         \@@_case_return:nw { 0.000000000000000e0 }
    \or: \exp_after:wN \@@_to_scientific_normal:wnnnnn
    \or:
      \@@_case_use:nw
        {
          \@@_invalid_operation:nnw
            { \fp_to_scientific:N \c_@@_overflowing_fp }
            { fp_to_scientific }
        }
    \or:
      \@@_case_use:nw
        {
          \@@_invalid_operation:nnw
            { \fp_to_scientific:N \c_zero_fp }
            { fp_to_scientific }
        }
    \fi:
    \s_@@ \@@_chk:w #1 #2
  }
\cs_new:Npn \@@_to_scientific_normal:wnnnnn
  \s_@@ \@@_chk:w 1 #1 #2 #3#4#5#6 ;
  {
    \exp_after:wN \@@_to_scientific_normal:wNw
    \exp_after:wN e
    \int_value:w \@@_int_eval:w #2 - 1
    ; #3 #4 #5 #6 ;
  }
\cs_new:Npn \@@_to_scientific_normal:wNw #1 ; #2#3;
  { #2.#3 #1 }
%    \end{macrocode}
% \end{macro}
%
% \subsection{Decimal representation}
%
% \begin{macro}[EXP]
%   {\fp_to_decimal:N, \fp_to_decimal:c, \fp_to_decimal:n}
%   All three public variants are based on the same
%   \cs{@@_to_decimal_dispatch:w}
%   after evaluating their argument to an internal floating point.
%    \begin{macrocode}
\cs_new:Npn \fp_to_decimal:N #1
  { \exp_after:wN \@@_to_decimal_dispatch:w #1 }
\cs_generate_variant:Nn \fp_to_decimal:N { c }
\cs_new:Npn \fp_to_decimal:n
  {
    \exp_after:wN \@@_to_decimal_dispatch:w
    \exp:w \exp_end_continue_f:w \@@_parse:n
  }
%    \end{macrocode}
% \end{macro}
%
% \begin{macro}[EXP]
%   {\@@_to_decimal_dispatch:w, \@@_to_decimal_recover:w, \@@_tuple_to_decimal:w}
%   We allow tuples.
%    \begin{macrocode}
\cs_new:Npn \@@_to_decimal_dispatch:w #1
  {
    \@@_change_func_type:NNN
      #1 \@@_to_decimal:w \@@_to_decimal_recover:w
    #1
  }
\cs_new:Npn \@@_to_decimal_recover:w #1 #2 ;
  {
    \@@_error:nffn { fp-unknown-type } { \tl_to_str:n { #2 ; } } { } { }
    nan
  }
\cs_new:Npn \@@_tuple_to_decimal:w
  { \@@_tuple_convert:Nw \@@_to_decimal_dispatch:w }
%    \end{macrocode}
% \end{macro}
%
% \begin{macro}[EXP]
%   {
%     \@@_to_decimal:w,
%     \@@_to_decimal_normal:wnnnnn,
%     \@@_to_decimal_large:Nnnw,
%     \@@_to_decimal_huge:wnnnn,
%   }
%   The structure is similar to \cs{@@_to_scientific:w}.
%   Insert |-| for
%   negative numbers.  Zero gives $0$, $\pm\infty$ and \nan{} yield an
%   \enquote{invalid operation} exception; note that $\pm\infty$
%   produces a very large output, which we don't expand now since it
%   most likely won't be needed.  Normal numbers with an exponent in the
%   range $[1,15]$ have that number of digits before the decimal
%   separator: \enquote{decimate} them, and remove leading zeros with
%   \cs{int_value:w}, then trim trailing zeros and dot.  Normal
%   numbers with an exponent $16$ or larger have no decimal separator,
%   we only need to add trailing zeros.  When the exponent is
%   non-positive, the result should be $0.\meta{zeros}\meta{digits}$,
%   trimmed.
%    \begin{macrocode}
\cs_new:Npn \@@_to_decimal:w \s_@@ \@@_chk:w #1#2
  {
    \if_meaning:w 2 #2 \exp_after:wN - \exp:w \exp_end_continue_f:w \fi:
    \if_case:w #1 \exp_stop_f:
         \@@_case_return:nw { 0 }
    \or: \exp_after:wN \@@_to_decimal_normal:wnnnnn
    \or:
      \@@_case_use:nw
        {
          \@@_invalid_operation:nnw
            { \fp_to_decimal:N \c_@@_overflowing_fp }
            { fp_to_decimal }
        }
    \or:
      \@@_case_use:nw
        {
          \@@_invalid_operation:nnw
            { 0 }
            { fp_to_decimal }
        }
    \fi:
    \s_@@ \@@_chk:w #1 #2
  }
\cs_new:Npn \@@_to_decimal_normal:wnnnnn
    \s_@@ \@@_chk:w 1 #1 #2 #3#4#5#6 ;
  {
    \int_compare:nNnTF {#2} > 0
      {
        \int_compare:nNnTF {#2} < \c_@@_prec_int
          {
            \@@_decimate:nNnnnn { \c_@@_prec_int - #2 }
              \@@_to_decimal_large:Nnnw
          }
          {
            \exp_after:wN \exp_after:wN
            \exp_after:wN \@@_to_decimal_huge:wnnnn
            \prg_replicate:nn { #2 - \c_@@_prec_int } { 0 } ;
          }
        {#3} {#4} {#5} {#6}
      }
      {
        \exp_after:wN \@@_trim_zeros:w
        \exp_after:wN 0
        \exp_after:wN .
        \exp:w \exp_end_continue_f:w \prg_replicate:nn { - #2 } { 0 }
        #3#4#5#6 ;
      }
  }
\cs_new:Npn \@@_to_decimal_large:Nnnw #1#2#3#4;
  {
    \exp_after:wN \@@_trim_zeros:w \int_value:w
      \if_int_compare:w #2 > 0 \exp_stop_f:
        #2
      \fi:
      \exp_stop_f:
      #3.#4 ;
  }
\cs_new:Npn \@@_to_decimal_huge:wnnnn #1; #2#3#4#5 { #2#3#4#5 #1 }
%    \end{macrocode}
% \end{macro}
%
% \subsection{Token list representation}
%
% \begin{macro}[EXP]{\fp_to_tl:N, \fp_to_tl:c, \fp_to_tl:n}
%   These three public functions evaluate their argument, then pass it
%   to \cs{@@_to_tl_dispatch:w}.
%    \begin{macrocode}
\cs_new:Npn \fp_to_tl:N #1 { \exp_after:wN \@@_to_tl_dispatch:w #1 }
\cs_generate_variant:Nn \fp_to_tl:N { c }
\cs_new:Npn \fp_to_tl:n
  {
    \exp_after:wN \@@_to_tl_dispatch:w
    \exp:w \exp_end_continue_f:w \@@_parse:n
  }
%    \end{macrocode}
% \end{macro}
%
% \begin{macro}[EXP]{\@@_to_tl_dispatch:w, \@@_to_tl_recover:w, \@@_tuple_to_tl:w}
%   We allow tuples.
%    \begin{macrocode}
\cs_new:Npn \@@_to_tl_dispatch:w #1
  { \@@_change_func_type:NNN #1 \@@_to_tl:w \@@_to_tl_recover:w #1 }
\cs_new:Npn \@@_to_tl_recover:w #1 #2 ;
  {
    \@@_error:nffn { fp-unknown-type } { \tl_to_str:n { #2 ; } } { } { }
    nan
  }
\cs_new:Npn \@@_tuple_to_tl:w
  { \@@_tuple_convert:Nw \@@_to_tl_dispatch:w }
%    \end{macrocode}
% \end{macro}
%
% \begin{macro}[EXP]
%   {
%     \@@_to_tl:w, \@@_to_tl_normal:nnnnn,
%     \@@_to_tl_scientific:wnnnnn, \@@_to_tl_scientific:wNw
%   }
%   A structure similar to \cs{@@_to_scientific_dispatch:w} and
%   \cs{@@_to_decimal_dispatch:w}, but without the \enquote{invalid operation}
%   exception.  First filter special cases.  We express normal numbers
%   in decimal notation if the exponent is in the range $[-2,16]$, and
%   otherwise use scientific notation.
%    \begin{macrocode}
\cs_new:Npn \@@_to_tl:w \s_@@ \@@_chk:w #1#2
  {
    \if_meaning:w 2 #2 \exp_after:wN - \exp:w \exp_end_continue_f:w \fi:
    \if_case:w #1 \exp_stop_f:
           \@@_case_return:nw { 0 }
    \or:   \exp_after:wN \@@_to_tl_normal:nnnnn
    \or:   \@@_case_return:nw { inf }
    \else: \@@_case_return:nw { nan }
    \fi:
  }
\cs_new:Npn \@@_to_tl_normal:nnnnn #1
  {
    \int_compare:nTF
      { -2 <= #1 <= \c_@@_prec_int }
      { \@@_to_decimal_normal:wnnnnn }
      { \@@_to_tl_scientific:wnnnnn }
    \s_@@ \@@_chk:w 1 0 {#1}
  }
\cs_new:Npn \@@_to_tl_scientific:wnnnnn
  \s_@@ \@@_chk:w 1 #1 #2 #3#4#5#6 ;
  {
    \exp_after:wN \@@_to_tl_scientific:wNw
    \exp_after:wN e
    \int_value:w \@@_int_eval:w #2 - 1
    ; #3 #4 #5 #6 ;
  }
\cs_new:Npn \@@_to_tl_scientific:wNw #1 ; #2#3;
  { \@@_trim_zeros:w #2.#3 ; #1 }
%    \end{macrocode}
% \end{macro}
%
% \subsection{Formatting}
%
% This is not implemented yet, as it is not yet clear what a correct
% interface would be, for this kind of structured conversion from a
% floating point (or other types of variables) to a string.  Ideas
% welcome.
%
% \subsection{Convert to dimension or integer}
%
% \begin{macro}[EXP]{\fp_to_dim:N, \fp_to_dim:c, \fp_to_dim:n}
% \begin{macro}[EXP]{\@@_to_dim_dispatch:w, \@@_to_dim_recover:w, \@@_to_dim:w}
%   All three public variants are based on the same
%   \cs{@@_to_dim_dispatch:w} after evaluating their argument to an
%   internal floating point.
%   We only allow floating point numbers, not tuples.
%    \begin{macrocode}
\cs_new:Npn \fp_to_dim:N #1
  { \exp_after:wN \@@_to_dim_dispatch:w #1 }
\cs_generate_variant:Nn \fp_to_dim:N { c }
\cs_new:Npn \fp_to_dim:n
  {
    \exp_after:wN \@@_to_dim_dispatch:w
    \exp:w \exp_end_continue_f:w \@@_parse:n
  }
\cs_new:Npn \@@_to_dim_dispatch:w #1#2 ;
  {
    \@@_change_func_type:NNN #1 \@@_to_dim:w \@@_to_dim_recover:w
    #1 #2 ;
  }
\cs_new:Npn \@@_to_dim_recover:w #1
  { \@@_invalid_operation:nnw { 0pt } { fp_to_dim } }
\cs_new:Npn \@@_to_dim:w #1 ; { \@@_to_decimal:w #1 ; pt }
%    \end{macrocode}
% \end{macro}
% \end{macro}
%
% \begin{macro}[EXP]{\fp_to_int:N, \fp_to_int:c, \fp_to_int:n}
% \begin{macro}[EXP]{\@@_to_int_dispatch:w, \@@_to_int_recover:w}
%   For the most part identical to \cs{fp_to_dim:N} but without |pt|,
%   and where \cs{@@_to_int:w} does more work.
%   To convert to an integer, first round to $0$ places (to the nearest
%   integer), then express the result as a decimal number: the
%   definition of \cs{@@_to_decimal_dispatch:w} is such that there are no
%   trailing dot nor zero.
%    \begin{macrocode}
\cs_new:Npn \fp_to_int:N #1 { \exp_after:wN \@@_to_int_dispatch:w #1 }
\cs_generate_variant:Nn \fp_to_int:N { c }
\cs_new:Npn \fp_to_int:n
  {
    \exp_after:wN \@@_to_int_dispatch:w
    \exp:w \exp_end_continue_f:w \@@_parse:n
  }
\cs_new:Npn \@@_to_int_dispatch:w #1#2 ;
  {
    \@@_change_func_type:NNN #1 \@@_to_int:w \@@_to_int_recover:w
    #1 #2 ;
  }
\cs_new:Npn \@@_to_int_recover:w #1
  { \@@_invalid_operation:nnw { 0 } { fp_to_int } }
\cs_new:Npn \@@_to_int:w #1;
  {
    \exp_after:wN \@@_to_decimal:w \exp:w \exp_end_continue_f:w
    \@@_round:Nwn \@@_round_to_nearest:NNN #1; { 0 }
  }
%    \end{macrocode}
% \end{macro}
% \end{macro}
%
% \subsection{Convert from a dimension}
%
% \begin{macro}[EXP]{\dim_to_fp:n}
% \begin{macro}[EXP]
%   {
%     \@@_from_dim_test:ww,
%     \@@_from_dim:wNw,
%     \@@_from_dim:wNNnnnnnn,
%     \@@_from_dim:wnnnnwNw,
%   }
%   The dimension expression (which can in fact be a glue expression) is
%   evaluated, converted to a number (\emph{i.e.}, expressed in scaled
%   points), then multiplied by $2^{-16} = 0.0000152587890625$ to give a
%   value expressed in points.  The auxiliary \cs{@@_mul_npos_o:Nww}
%   expects the desired \meta{final sign} and two floating point
%   operands (of the form \cs{s_@@} \ldots{} |;|) as arguments.
%   This set of functions is also used to convert dimension registers to
%   floating points while parsing expressions: in this context there is
%   an additional exponent, which is the first argument of
%   \cs{@@_from_dim_test:ww}, and is combined with the exponent $-4$
%   of $2^{-16}$.  There is also a need to expand afterwards: this is
%   performed by \cs{@@_mul_npos_o:Nww}, and cancelled by
%   \cs{prg_do_nothing:} here.
%    \begin{macrocode}
\__kernel_patch_args:nNNpn { { (#1) } }
\cs_new:Npn \dim_to_fp:n #1
  {
    \exp_after:wN \@@_from_dim_test:ww
    \exp_after:wN 0
    \exp_after:wN ,
    \int_value:w \tex_glueexpr:D #1 ;
  }
\cs_new:Npn \@@_from_dim_test:ww #1, #2
  {
    \if_meaning:w 0 #2
      \@@_case_return:nw { \exp_after:wN \c_zero_fp }
    \else:
      \exp_after:wN \@@_from_dim:wNw
      \int_value:w \@@_int_eval:w #1 - 4
        \if_meaning:w - #2
          \exp_after:wN , \exp_after:wN 2 \int_value:w
        \else:
          \exp_after:wN , \exp_after:wN 0 \int_value:w #2
        \fi:
    \fi:
  }
\cs_new:Npn \@@_from_dim:wNw #1,#2#3;
  {
    \@@_pack_twice_four:wNNNNNNNN \@@_from_dim:wNNnnnnnn ;
    #3 000 0000 00 {10}987654321; #2 {#1}
  }
\cs_new:Npn \@@_from_dim:wNNnnnnnn #1; #2#3#4#5#6#7#8#9
  { \@@_from_dim:wnnnnwNn #1 {#2#300} {0000} ; }
\cs_new:Npn \@@_from_dim:wnnnnwNn #1; #2#3#4#5#6; #7#8
  {
    \@@_mul_npos_o:Nww #7
      \s_@@ \@@_chk:w 1 #7 {#5} #1 ;
      \s_@@ \@@_chk:w 1 0 {#8} {1525} {8789} {0625} {0000} ;
      \prg_do_nothing:
  }
%    \end{macrocode}
% \end{macro}
% \end{macro}
%
% \subsection{Use and eval}
%
% \begin{macro}[EXP]{\fp_use:N, \fp_use:c, \fp_eval:n}
%   Those public functions are simple copies of the decimal conversions.
%    \begin{macrocode}
\__kernel_if_debug:TF
  {
    \cs_new:Npn \fp_use:N #1
      {
        \__kernel_chk_exp_var_exist:NTF #1 { \fp_to_decimal:N #1 } {0}
      }
  }
  {
    \cs_new_eq:NN \fp_use:N \fp_to_decimal:N
  }
\cs_generate_variant:Nn \fp_use:N { c }
\cs_new_eq:NN \fp_eval:n \fp_to_decimal:n
%    \end{macrocode}
% \end{macro}
%
% \begin{macro}[EXP]{\fp_abs:n}
%   Trivial but useful.  See the implementation of \cs{fp_add:Nn} for an
%   explanation of why to use \cs{@@_parse:n}, namely, for better error
%   reporting.
%    \begin{macrocode}
\cs_new:Npn \fp_abs:n #1
  { \fp_to_decimal:n { abs \@@_parse:n {#1} } }
%    \end{macrocode}
% \end{macro}
%
% \begin{macro}[EXP]{\fp_max:nn, \fp_min:nn}
%   Similar to \cs{fp_abs:n}, for consistency with \cs{int_max:nn}, \emph{etc.}
%    \begin{macrocode}
\cs_new:Npn \fp_max:nn #1#2
  { \fp_to_decimal:n { max ( \@@_parse:n {#1} , \@@_parse:n {#2} ) } }
\cs_new:Npn \fp_min:nn #1#2
  { \fp_to_decimal:n { min ( \@@_parse:n {#1} , \@@_parse:n {#2} ) } }
%    \end{macrocode}
% \end{macro}
%
% \subsection{Convert an array of floating points to a comma list}
%
% \begin{macro}[EXP]{\@@_array_to_clist:n}
% \begin{macro}[EXP]{\@@_array_to_clist_loop:Nw}
%   Converts an array of floating point numbers to a comma-list.  If
%   speed here ends up irrelevant, we can simplify the code for the
%   auxiliary to become
%   \begin{verbatim}
%     \cs_new:Npn \__fp_array_to_clist_loop:Nw #1#2;
%       {
%         \use_none:n #1
%         { , ~ } \fp_to_tl:n { #1 #2 ; }
%         \__fp_array_to_clist_loop:Nw
%       }
%   \end{verbatim}
%   The \cs{use_ii:nn} function is expanded after \cs{@@_expand:n} is
%   done, and it removes |,~| from the start of the representation.
%    \begin{macrocode}
\cs_new:Npn \@@_array_to_clist:n #1
  {
    \tl_if_empty:nF {#1}
      {
        \exp_last_unbraced:Ne \use_ii:nn
          {
            \@@_array_to_clist_loop:Nw #1 { ? \prg_break: } ;
            \prg_break_point:
          }
      }
  }
\cs_new:Npn \@@_array_to_clist_loop:Nw #1#2;
  {
    \use_none:n #1
    , ~
    \exp_not:f { \@@_to_tl_dispatch:w #1 #2 ; }
    \@@_array_to_clist_loop:Nw
  }
%    \end{macrocode}
% \end{macro}
% \end{macro}
%
%    \begin{macrocode}
%</initex|package>
%    \end{macrocode}
%
% \end{implementation}
%
% \PrintChanges
%
% \PrintIndex
