% \iffalse meta-comment
%
%% File: l3doc.dtx Copyright (C) 1990-2015 The LaTeX3 project
%%
%% It may be distributed and/or modified under the conditions of the
%% LaTeX Project Public License (LPPL), either version 1.3c of this
%% license or (at your option) any later version.  The latest version
%% of this license is in the file
%%
%%    http://www.latex-project.org/lppl.txt
%%
%% This file is part of the "l3kernel bundle" (The Work in LPPL)
%% and all files in that bundle must be distributed together.
%%
%% The released version of this bundle is available from CTAN.
%%
%% -----------------------------------------------------------------------
%%
%% The development version of the bundle can be found at
%%
%%    http://www.latex-project.org/svnroot/experimental/trunk/
%%
%% for those people who are interested.
%%
%%%%%%%%%%%
%% NOTE: %%
%%%%%%%%%%%
%%
%%   Snapshots taken from the repository represent work in progress and may
%%   not work or may contain conflicting material!  We therefore ask
%%   people _not_ to put them into distributions, archives, etc. without
%%   prior consultation with the LaTeX3 Project.
%%
%% -----------------------------------------------------------------------
%
%<*driver>
\def\nameofplainTeX{plain}
\ifx\fmtname\nameofplainTeX\else
  \expandafter\begingroup
\fi
\input l3docstrip.tex
\askforoverwritefalse
\preamble


EXPERIMENTAL CODE

Do not distribute this file without also distributing the
source files specified above.

Do not distribute a modified version of this file.


\endpreamble
% stop docstrip adding \endinput
\postamble
\endpostamble
\generate{\file{l3doc.cls}{\from{l3doc.dtx}{class,cfg}}}
%\generate{\file{l3doc.ist}{\from{l3doc.dtx}{docist}}}
\ifx\fmtname\nameofplainTeX
  \expandafter\endbatchfile
\else
  \expandafter\endgroup
\fi
%</driver>
%
% Need to protect the file metadata for any modules that load
% \cls{l3doc}.  This is restored after \cs{ProvideExplClass} below.
%    \begin{macrocode}
%<class>\let        \filenameOld        \ExplFileName
%<class>\let        \filedateOld        \ExplFileDate
%<class>\let     \fileversionOld        \ExplFileVersion
%<class>\let \filedescriptionOld        \ExplFileDescription
%    \end{macrocode}
%
%<*driver|class>
\RequirePackage{expl3,xparse,calc}
\GetIdInfo$Id$
          {L3 Experimental documentation class}
%</driver|class>
%
%<*driver>
\ProvidesFile{\ExplFileName.dtx}
  [\ExplFileDate\space v\ExplFileVersion\space\ExplFileDescription]
\documentclass{l3doc}
\usepackage{framed,lipsum}
\begin{document}
  \DocInput{l3doc.dtx}
\end{document}
%</driver>
%
% This isn't included in the typeset documentation because it's a bit
% ugly:
%<*class>
\ProvidesExplClass
  {\ExplFileName}{\ExplFileDate}{\ExplFileVersion}{\ExplFileDescription}
\let        \ExplFileName        \filenameOld
\let        \ExplFileDate        \filedateOld
\let        \ExplFileVersion     \fileversionOld
\let        \ExplFileDescription \filedescriptionOld
%</class>
% \fi
%
% \title{The \cls{l3doc} class\thanks{This file
%         has version number v\ExplFileVersion, last
%         revised \ExplFileDate.}}
% \author{\Team}
% \date{\ExplFileDate}
% \maketitle
% \tableofcontents
%
% \begin{documentation}
%
%
% \section{Introduction}
%
% This is an ad-hoc class for documenting the \pkg{expl3} bundle, a
% collection of modules or packages that make up \LaTeX3's programming
% environment.  Eventually it will replace the \cls{ltxdoc} class for
% \LaTeX3, but not before the good ideas in \pkg{hypdoc}, \cls{xdoc2},
% \pkg{docmfp}, and \cls{gmdoc} are incorporated.
%
% \textbf{It is much less stable than the main \pkg{expl3} packages.
%   Use at own risk!}
%
% It is written as a \enquote{self-contained} docstrip file: executing
% |latex l3doc.dtx| will generate the \file{l3doc.cls} file and typeset
% this documentation; execute |tex l3doc.dtx| to only generate
% \file{l3doc.cls}.
%
% \section{Features of other packages}
%
% This class builds on the \pkg{ltxdoc} class and the \pkg{doc} package,
% but in the time since they were originally written some improvements
% and replacements have appeared that we would like to use as
% inspiration.
%
% These packages or classes are \pkg{hypdoc}, \pkg{docmfp}, \pkg{gmdoc},
% and \pkg{xdoc}.  I have summarised them below in order to work out
% what sort of features we should aim at a minimum for \pkg{l3doc}.
%
% \subsection{The \pkg{hypdoc} package}
%
% This package provides hyperlink support for the \pkg{doc} package.  I
% have included it in this list to remind me that cross-referencing
% between documentation and implementation of methods is not very
% good. (\emph{E.g.}, it would be nice to be able to automatically
% hyperlink the documentation for a function from its implementation and
% vice-versa.)
%
% \subsection{The \pkg{docmfp} package}
%
% \begin{itemize}
%   \item Provides \cs{DescribeRoutine} and the \env{routine}
%     environment (\emph{etc.}) for MetaFont and MetaPost code.
%   \item Provides \cs{DescribeVariable} and the \env{variable}
%     environment (\emph{etc.})  for more general code.
%   \item Provides \cs{Describe} and the \env{Code} environment
%     (\emph{etc.})  as a generalisation of the above two
%     instantiations.
%   \item Small tweaks to the DocStrip system to aid non-\LaTeX{} use.
% \end{itemize}
%
% \subsection{The \pkg{xdoc2} package}
%
% \begin{itemize}
%   \item Two-sided printing.
%   \item \cs{NewMacroEnvironment}, \cs{NewDescribeEnvironment}; similar
%     idea to \pkg{docmfp} but more comprehensive.
%   \item Tons of small improvements.
% \end{itemize}
%
% \subsection{The \pkg{gmdoc} package}
%
% Radical re-implementation of \pkg{doc} as a package or class.
% \begin{itemize}
%   \item Requires no |\begin{macrocode}| blocks!
%   \item Automatically inserts |\begin{macro}| blocks!
%   \item And a whole bunch of other little things.
% \end{itemize}
%
% \section{Problems \& Todo}
%
% Problems at the moment:
% (1)~not flexible in the types of things that can be documented;
% (2)~no obvious link between the |\begin{function}| environment for
%     documenting things to the |\begin{macro}| function that's used
%     analogously in the implementation.
%
% The \env{macro} should probably be renamed to \env{function} when it
% is used within an implementation section.  But they should have the
% same syntax before that happens!
%
% Furthermore, we need another \enquote{layer} of documentation commands
% to account for \enquote{user-macro} as opposed to
% \enquote{code-functions}; the \pkg{expl3} functions should be
% documented differently, probably, to the \pkg{xparse} user macros (at
% least in terms of indexing).
%
% In no particular order, a list of things to do:
% \begin{itemize}
%   \item Rename \env{function}/\env{macro} environments to better
%     describe their use.
%   \item Generalise \env{function}/\env{macro} for documenting
%     \enquote{other things}, such as environment names, package
%     options, even keyval options.
%   \item New function like \tn{part} but for files (remove awkward
%     \enquote{File} as \tn{partname}).
%   \item Something better to replace \cs{StopEventually}; I'm thinking
%     two environments \env{documentation} and \env{implementation} that
%     can conditionally typeset/ignore their material.  (This has been
%     implemented but needs further consideration.)
%   \item Hyperlink documentation and implementation of macros (see the
%     \textsc{dtx} file of \pkg{svn-multi} v2 as an example).  This is
%     partially done, now, but should be improved.
% \end{itemize}
%
% \section{Documentation}
%
% \subsection{Configuration}
%
% Before class options are processed, \pkg{l3doc} loads a configuration
% file \file{l3doc.cfg} if it exists, allowing you to customise the
% behaviour of the class without having to change the documentation
% source files.
%
% For example, to produce documentation on letter-sized paper instead of
% the default A4 size, create \file{l3doc.cfg} and include the line
% \begin{verbatim}
% \PassOptionsToClass{letterpaper}{l3doc}
% \end{verbatim}
%
% By default, \pkg{l3doc} selects the |T1| font encoding and loads the
% Latin Modern fonts.  To prevent this, use the class option
% |cm-default|.
%
% \subsection{Partitioning documentation and implementation}
%
% \pkg{doc} uses the \cs{OnlyDocumentation}/\cs{AlsoImplementation}
% macros to guide the use of \cs{StopEventually}|{}|, which is intended
% to be placed to partition the documentation and implementation within
% a single \file{.dtx} file.
%
% This isn't very flexible, since it assumes that we \emph{always} want
% to print the documentation.  For the \pkg{expl3} sources, I wanted to
% be be able to input \file{.dtx} files in two modes: only displaying
% the documentation, and only displaying the implementation.  For
% example:
% \begin{verbatim}
% \DisableImplementation
% \DocInput{l3basics,l3prg,...}
% \EnableImplementation
% \DisableDocumentation
% \DocInputAgain
% \end{verbatim}
%
% The idea being that the entire \pkg{expl3} bundle can be documented,
% with the implementation included at the back.  Now, this isn't
% perfect, but it's a start.
%
% Use |\begin{documentation}...\end{documentation}| around the
% documentation, and |\begin{implementation}...\end{implementation}|
% around the implementation.  The
% \cs{EnableDocumentation}/\cs{EnableImplementation} will cause them to
% be typeset when the \file{.dtx} file is \cs{DocInput}; use
% \cs{DisableDocumentation}/\cs{DisableImplementation} to omit the
% contents of those environments.
%
% Note that \cs{DocInput} now takes comma-separated arguments, and
% \cs{DocInputAgain} can be used to re-input all \file{.dtx} files
% previously input in this way.
%
% \subsection{General text markup}
%
% Many of the commands in this section come from \pkg{ltxdoc} with some
% improvements.
%
% \begin{function}{\cmd, \cs}
%   \begin{syntax}
%     \cmd{\cmd} \oarg{options} \meta{control sequence}\\
%     \cs{cs} \oarg{options} \marg{csname}
%   \end{syntax}
%   These commands are provided to typeset control sequences.
%   |\cmd\foo| produces \enquote{\cmd\foo} and |\cs{foo}| produces the
%   same (\enquote{\texttt{\string\foo|}}).  In general, \cs{cs} is more robust since
%   it doesn't rely on catcodes being \enquote{correct} and is therefore
%   recommended.
%
%   These commands are aware of the |@@| \pkg{l3docstrip} syntax and
%   will replace such instances correctly in the typeset documentation.
%   This only happens after a |%<@@=|\meta{module}|>| declaration.
%
%   Additionally, commands can be used in the argument of \cs{cs}.  For
%   instance, |\cs{\meta{name}:\meta{signature}}| produces
%   \cs{\meta{name}:\meta{signature}}.
%
%   The \meta{options} are a key--value list which can contain the
%   following keys:
%   \begin{itemize}
%     \item |index=|\meta{module}: the \meta{csname} will be indexed in
%       the list of commands from the \meta{module}; the \meta{module}
%       can in particular be |TeX| for \enquote{\TeX{} and \LaTeXe{}}
%       commands, or empty for commands which should be placed in the
%       main index.  By default, the \meta{module} is deduced
%       automatically from the command name.
%     \item |replace| is a boolean key (\texttt{true} by default) which
%       indicates whether to replace |@@| as \pkg{l3docstrip} does.
%   \end{itemize}
% \end{function}
%
% \begin{function}{\tn}
%   \begin{syntax}
%     \cs{tn} \oarg{options} \marg{csname}
%   \end{syntax}
%   Analoguous to \cs{cs} but intended for \enquote{traditional} \TeX{}
%   or \LaTeXe{} commands; they will be indexed accordingly.  This is in
%   fact equivalent to \cs{cs} |[index=TeX, replace=false,|
%   \meta{options}|]| \Arg{csname}.
% \end{function}
%
% \begin{function}{\meta}
%   \begin{syntax}
%     \cs{meta} \Arg{name}
%   \end{syntax}
%   \cs{meta} typesets the \meta{name} italicised in \meta{angle
%     brackets}.  Within a \env{function} environment or similar, angle
%   brackets |<...>| are set up to be a shorthand for |\meta{...}|.
%
%   This function has additional functionality over its \pkg{ltxdoc}
%   versions; underscores can be used to subscript material as in math
%   mode.  For example, |\meta{arg_{xy}}| produces
%   \enquote{\meta{arg_{xy}}}.
% \end{function}
%
% \begin{function}{\Arg, \marg, \oarg, \parg}
%   \begin{syntax}
%     |\Arg| \Arg{name}
%   \end{syntax}
%   Typesets the \meta{name} as for \cs{meta} and wraps it in braces.
%
%   The \cs{marg}/\cs{oarg}/\cs{parg} versions follow from \pkg{ltxdoc}
%   in being used for \enquote{mandatory} or \enquote{optional} or
%   \enquote{picture} brackets as per \LaTeXe{} syntax.
% \end{function}
%
% \begin{function}{\file, \env, \pkg, \cls}
%   \begin{syntax}
%     \cs{pkg} \Arg{name}
%   \end{syntax}
%   These all take one argument and are intended to be used as semantic
%   commands for representing files, environments, package names, and
%   class names, respectively.
% \end{function}
%
% \subsection{Describing functions in the documentation}
%
% \DescribeEnv{function}
% \DescribeEnv{syntax}
% Two heavily-used environments are defined to describe the syntax of
% \pkg{expl3} functions and variables.
% \begin{framed}
%   \vspace{-\baselineskip}
% \begin{verbatim}
% \begin{function}{\function_one:, \function_two:}
%   \begin{syntax}
%     |\foo_bar:| \Arg{meta} \meta{test_1}
%   \end{syntax}
% \meta{description}
% \end{function}
% \end{verbatim}
%   \hrulefill
%   \par
%   \hspace*{0.25\textwidth}
%   \begin{minipage}{0.5\textwidth}
%     \begin{function}{\function_one:, \function_two:}
%       \begin{syntax}
%         |\foo_bar:| \Arg{meta} \meta{test_1}
%       \end{syntax}
%       \meta{description}
%     \end{function}
%   \end{minipage}
% \end{framed}
%
% Function environments take an optional argument to indicate whether
% the function(s) it describes are expandable or restricted-expandable
% or defined in conditional forms. Use |EXP|, |rEXP|, |TF|, or |pTF| for
% this; note that |pTF| implies |EXP| since predicates must always be
% expandable.  As an example:
% \begin{framed}
%   \vspace{-\baselineskip}
% \begin{verbatim}
% \begin{function}[pTF]{\cs_if_exist:N}
%   \begin{syntax}
%     \cs{cs_if_exist_p:N} \meta{cs}
%   \end{syntax}
% \meta{description}
% \end{function}
% \end{verbatim}
%   \hrulefill
%   \par
%   \hspace*{0.25\textwidth}
%   \begin{minipage}{0.5\textwidth}
%     \begin{function}[pTF]{\cs_if_exist:N}
%       \begin{syntax}
%         \cs{cs_if_exist_p:N} \meta{cs}
%       \end{syntax}
%       \meta{description}
%     \end{function}
%   \end{minipage}
% \end{framed}
%
% \DescribeEnv{variable}
% If you are documenting a variable instead of a function, use the
% \env{variable} environment instead; it behaves identically to the
% \env{function} environment above.
%
% \DescribeEnv{texnote}
% This environment is used to call out sections within \env{function}
% and similar that are only of interest to seasoned \TeX{} developers.
%
% \subsection{Describing functions in the implementation}
%
% \DescribeEnv{macro}
% The well-used environment from \LaTeXe{} for marking up the
% implementation of macros/functions remains the \env{macro}
% environment.  Some changes in \pkg{l3doc}: it now accepts
% comma-separated lists of functions, to avoid a very large number of
% consecutive |\end{macro}| statements.
% \begin{verbatim}
% % \begin{macro}{\foo:N, \foo:c}
% %   \begin{macrocode}
% ... code for \foo:N and \foo:c ...
% %   \end{macrocode}
% % \end{macro}
% \end{verbatim}
% If you are documenting an auxiliary macro, it's generally not
% necessary to highlight it as much and you also don't need to check it
% for, say, having a test function and having a documentation chunk
% earlier in a \env{function} environment.  In this case, write
% |\begin{macro}[aux]| and it will be marked as such; its margin
% call-out will be printed in grey.
%
% Similarly, an internal package function still requires documentation
% but usually will not be documented for users to see; these can be
% marked as such with |\begin{macro}[int]|.
%
% For documenting \pkg{expl3}-type conditionals, you may also pass this
% environment a |TF| option (and omit it from the function name) to
% denote that the function is provided with |T|, |F|, and |TF| suffixes.
% A similar |pTF| option will print both |TF| and |_p| predicate forms.
%
%
% \DescribeMacro{\TestFiles}
% \cs{TestFiles}\marg{list of files} is used to indicate which test
% files are used for the current code; they will be printed in the
% documentation.
%
% \DescribeMacro{\UnitTested}
% Within a \env{macro} environment, it is a good idea to mark whether a
% unit test has been created for the commands it defines.  This is
% indicated by writing \cs{UnitTested} anywhere within |\begin{macro}|
%   \dots |\end{macro}|.
%
% If the class option |checktest| is enabled, then it is an \emph{error}
% to have a \env{macro} environment without a call to
% \file{Testfiles}.  This is intended for large packages such as
% \pkg{expl3} that should have absolutely comprehensive tests suites and
% whose authors may not always be as sharp at adding new tests with new
% code as they should be.
%
% \DescribeMacro{\TestMissing}
% If a function is missing a test, this may be flagged by writing (as
% many times as needed) \cs{TestMissing} \marg{explanation of test
%   required}.  These missing tests will be summarised in the listing
% printed at the end of the compilation run.
%
% \DescribeEnv{variable}
% When documenting variable definitions, use the \env{variable}
% environment instead.  It will, here, behave identically to the
% \env{macro} environment, except that if the class option |checktest|
% is enabled, variables will not be required to have a test file.
%
% \DescribeEnv{arguments}
% Within a \env{macro} environment, you may use the \env{arguments}
% environment to describe the arguments taken by the function(s).  It
% behaves like a modified enumerate environment.
% \begin{verbatim}
% % \begin{macro}{\foo:nn, \foo:VV}
% % \begin{arguments}
% %   \item Name of froozle to be frazzled
% %   \item Name of muble to be jubled
% % \end{arguments}
% %   \begin{macrocode}
% ... code for \foo:nn and \foo:VV ...
% %   \end{macrocode}
% % \end{macro}
% \end{verbatim}
%
%
% \subsection{Keeping things consistent}
%
% Whenever a function is either documented or defined with
% \env{function} and \env{macro} respectively, its name is stored in a
% sequence for later processing.
%
% At the end of the document (\emph{i.e.}, after the \file{.dtx} file
% has finished processing), the list of names is analysed to check
% whether all defined functions have been documented and vice versa. The
% results are printed in the console output.
%
% If you need to do more serious work with these lists of names, take a
% look at the implementation for the data structures and methods used to
% store and access them directly.
%
% \subsection{Documenting templates}
%
% The following macros are provided for documenting templates; might end
% up being something completely different but who knows.
% \begin{quote}\parskip=0pt\obeylines
%   |\begin{TemplateInterfaceDescription}| \Arg{template type name}
%   |  \TemplateArgument{none}{---}|
%   \textsc{or one or more of these:}
%   |  \TemplateArgument| \Arg{arg no} \Arg{meaning}
%   \textsc{and}
%   |\TemplateSemantics|
%   |  | \meta{text describing the template type semantics}
%   |\end{TemplateInterfaceDescription}|
% \end{quote}
%
% \begin{quote}\parskip=0pt\obeylines
%   |\begin{TemplateDescription}| \Arg{template type name} \Arg{name}
%   \textsc{one or more of these:}
%   |  \TemplateKey| \marg{key name} \marg{type of key}
%   |    |\marg{textual description of meaning}
%   |    |\marg{default value if any}
%   \textsc{and}
%   |\TemplateSemantics|
%   |  | \meta{text describing special additional semantics of the template}
%   |\end{TemplateDescription}|
% \end{quote}
%
% \begin{quote}\parskip=0pt\obeylines
%   |\begin{InstanceDescription}| \oarg{text to specify key column width (optional)}
%   \hfill\marg{template type name}\marg{instance name}\marg{template name}
%   \textsc{one or more of these:}
%   |  \InstanceKey| \marg{key name} \marg{value}
%   \textsc{and}
%   |\InstanceSemantics|
%   |  | \meta{text describing the result of this instance}
%   |\end{InstanceDescription}|
% \end{quote}
%
% \end{documentation}
%
% \begin{implementation}
%
% \section{\pkg{l3doc} implementation}
%
%    \begin{macrocode}
%<*class>
%    \end{macrocode}
%
%    \begin{macrocode}
%<@@=codedoc>
%    \end{macrocode}
%
% \subsection{Variants and helpers}
%
% \begin{macro}
%   {
%     \tl_count:f,
%     \tl_greplace_all:Nxn,
%     \tl_if_head_eq_charcode:oNTF,
%     \tl_if_head_eq_charcode:oNT,
%     \tl_if_head_eq_charcode:oNF,
%     \tl_if_in:NoTF,
%     \tl_if_in:NoT,
%     \tl_if_in:NoF,
%     \tl_remove_all:Nx,
%     \tl_replace_all:Nxn,
%     \tl_replace_all:Nnx,
%     \tl_replace_all:Non,
%     \tl_replace_all:Nno,
%     \tl_replace_once:Noo,
%     \prop_get:NxNTF,
%     \prop_put:Nxn,
%     \prop_gput:NVx,
%   }
%   A few missing variants.
%    \begin{macrocode}
\cs_generate_variant:Nn \tl_count:n { f }
\cs_generate_variant:Nn \tl_greplace_all:Nnn { Nx }
\cs_generate_variant:Nn \tl_if_head_eq_charcode:nNTF { o }
\cs_generate_variant:Nn \tl_if_head_eq_charcode:nNT  { o }
\cs_generate_variant:Nn \tl_if_head_eq_charcode:nNF  { o }
\cs_generate_variant:Nn \tl_if_in:NnTF { No }
\cs_generate_variant:Nn \tl_if_in:NnT  { No }
\cs_generate_variant:Nn \tl_if_in:NnF  { No }
\cs_generate_variant:Nn \tl_remove_all:Nn   { Nx }
\cs_generate_variant:Nn \tl_replace_all:Nnn { Nx , Nnx, No , Nno }
\cs_generate_variant:Nn \tl_replace_once:Nnn { Noo }
\cs_generate_variant:Nn \prop_get:NnNTF { Nx }
\cs_generate_variant:Nn \prop_put:Nnn { Nx }
\cs_generate_variant:Nn \prop_gput:Nnn { NVx }
%    \end{macrocode}
% \end{macro}
%
% \begin{macro}[int]{\@@_replace_at_at:N}
% \begin{macro}[aux]{\@@_replace_at_at_aux:Nn}
%   If there is no \meta{module~name}, do nothing.  Otherwise replace
%   all other-|@| by letter-|@| and all other-|_| by letter-|_|, then
%   replace |_@@| by |__|\meta{module~name} and |@@| by
%   |__|\meta{module~name} too.  The result contains |_| with category
%   code letter because this is what the |macrocode| environment
%   expects.  Other use cases can apply \cs{tl_to_str:n} if needed.
%   Note that we include spaces between the
%   |@| in the code below, since it is also processed through the same
%   replacement rules.
%    \begin{macrocode}
\cs_new_protected:Npn \@@_replace_at_at:N #1
  {
    \tl_if_empty:NF \g_@@_module_name_tl
      {
        \exp_args:NNo \@@_replace_at_at_aux:Nn
          #1 \g_@@_module_name_tl
      }
  }
\cs_new_protected:Npn \@@_replace_at_at_aux:Nn #1#2
  {
    \tl_replace_all:Non #1 { \token_to_str:N @ } { @ }
    \tl_replace_all:Non #1 { \token_to_str:N _ } { _ }
    \tl_replace_all:Nnn #1 { _ @ @ } { _ _ #2 }
    \tl_replace_all:Nnn #1 {   @ @ } { _ _ #2 }
  }
%    \end{macrocode}
% \end{macro}
% \end{macro}
%
% \begin{macro}[aux]{\@@_verb_get_seq:nN}
%   The argument~|#1| is given with catcodes $10$ (space), $12$ (other)
%   and $13$ (active).  Turn active characters to other.  Remove any
%   \enquote{\%} character at the beginning of a line.  Remove tabs and
%   newlines.  Finally, convert |_@@| and |@@| to |__|\meta{module name}
%   (if it is non-empty).  At this point, \cs{l_@@_tmpa_tl} contains a
%   comma-delimited list of names, where |@| and~|_| have category code
%   letter.  Turn it to a string, parse it as a comma-delimited list,
%   and turn the result into a sequence of function/macro names.
%    \begin{macrocode}
\cs_new_protected:Npn \@@_verb_get_seq:nN #1#2
  {
    \tl_set:Nx \l_@@_tmpa_tl { \tl_to_str:n {#1} }
    \tl_remove_all:Nx \l_@@_tmpa_tl
      { \iow_char:N \^^M \iow_char:N \% }
    \tl_remove_all:Nx \l_@@_tmpa_tl { \tl_to_str:n { ^ ^ A } }
    \tl_remove_all:Nx \l_@@_tmpa_tl { \iow_char:N \^^I }
    \tl_remove_all:Nx \l_@@_tmpa_tl { \iow_char:N \^^M }
    \@@_replace_at_at:N \l_@@_tmpa_tl
    \exp_args:NNx \seq_set_from_clist:Nn #2
      { \tl_to_str:N \l_@@_tmpa_tl }
  }
%    \end{macrocode}
% \end{macro}
%
% \begin{macro}[aux,rEXP]
%   {\@@_signature_base_form:n, \@@_signature_base_form_aux:n}
%   Expands to the \enquote{base form} of the signature.  For instance,
%   given |noxcfvV| it would obtain |nnnNnnn|, or given |ow| it would
%   obtain |nw|.  The loop stops at the first token that is not
%   recognized.
%    \begin{macrocode}
\cs_new:Npn \@@_signature_base_form:n #1
  { \@@_signature_base_form_aux:n #1 \c_empty_tl }
\cs_new:Npn \@@_signature_base_form_aux:n #1
  {
    \str_case:nnTF {#1}
      {
        { N } { N }
        { c } { N }
        { n } { n }
        { o } { n }
        { f } { n }
        { x } { n }
        { V } { n }
        { v } { n }
      }
      { \@@_signature_base_form_aux:n }
      {#1}
  }
%    \end{macrocode}
% \end{macro}
%
% \begin{macro}[aux]{\@@_predicate_from_base:N}
%   Get predicate from a function's base name.  This \enquote{works} for
%   functions with no signature too.
%    \begin{macrocode}
\cs_new:Npn \@@_predicate_from_base:N #1
  {
    \__cs_get_function_name:N #1 _p:
    \__cs_get_function_signature:N #1
  }
%    \end{macrocode}
% \end{macro}
%
% \begin{macro}[int]{\CodedocUseCs}
%   To implement commands which can be used in bookmarks and moving
%   arguments, it appears necessary to have an analogue of \cs{use:c}
%   using only user-level category codes, or to make all auxiliaries
%   user-level.
%    \begin{macrocode}
\cs_new_protected:Npn \CodedocUseCs #1 { \use:c { \tl_to_str:n {#1} } }
%    \end{macrocode}
% \end{macro}
%
% \begin{macro}[int]{\CodedocUnexpandedTokens}
% \begin{macro}[aux]{\CodedocUnexpandedTokens }
% \begin{macro}[aux]{\@@_exp_not:n}
%   Function used as \cs{CodedocUnexpandedTokens} \meta{junk}
%   \Arg{argument}.  The braces are necessary, and \meta{junk}, ignored,
%   must not contain braces.  When used in a typesetting context, the
%   function calls an auxiliary (whose name has a trailing space) which
%   leaves the \meta{argument} in the input stream.  When used in an
%   \texttt{x}-expansion context instead, the auxiliary does not expand,
%   and \cs{@@_exp_not:n} \Arg{argument} expands to itself.  If this is
%   written into a file and read back, the result has the original form
%   \cs{CodedocUnexpandedTokens} \meta{junk} \Arg{argument}.
%    \begin{macrocode}
\cs_new:Npn \CodedocUnexpandedTokens #1 #
  { \use:c { CodedocUnexpandedTokens ~ } \@@_exp_not:n }
\cs_new_protected:cpn { CodedocUnexpandedTokens ~ } #1#2 {#2}
\cs_new:Npn \@@_exp_not:n #1 { \exp_not:n { \@@_exp_not:n {#1} } }
%    \end{macrocode}
% \end{macro}
% \end{macro}
% \end{macro}
%
% \subsection{Variables}
%
% \begin{variable}{\g_docinput_clist}
%   The list of files which have been input through \cs{DocInput}.
%    \begin{macrocode}
\clist_new:N \g_docinput_clist
%    \end{macrocode}
% \end{variable}
%
% \begin{variable}{\g_doc_functions_seq, \g_doc_macros_seq}
%   All functions documented through \env{function}, and all macros
%   introduced through \env{macro}.  They can be compared to see what
%   documentation or code is missing.
%    \begin{macrocode}
\seq_new:N \g_doc_functions_seq
\seq_new:N \g_doc_macros_seq
%    \end{macrocode}
% \end{variable}
%
% \begin{variable}[int]{\l_@@_output_coffin}
%   The \env{function} environment is typeset by combining coffins
%   containing various pieces (function names, description, \emph{etc.})
%   into this coffin.
%    \begin{macrocode}
\coffin_new:N \l_@@_output_coffin
%    \end{macrocode}
% \end{variable}
%
% \begin{variable}[int]
%   {\l_@@_names_coffin, \l_@@_descr_coffin, \l_@@_syntax_coffin}
%   These coffins contain respectively the list of function names
%   (argument of the \env{function} environment), the text between
%   |\begin{function}| and |\end{function}|, and the syntax given in the
%   \env{syntax} environment.
%    \begin{macrocode}
\coffin_new:N \l_@@_names_coffin
\coffin_new:N \l_@@_descr_coffin
\coffin_new:N \l_@@_syntax_coffin
%    \end{macrocode}
% \end{variable}
%
% \begin{variable}[int]{\g_@@_syntax_box}
%   The contents of the \env{syntax} environment are typeset in this box
%   before being transferred to \cs{l_@@_syntax_coffin}.
%    \begin{macrocode}
\box_new:N \g_@@_syntax_box
%    \end{macrocode}
% \end{variable}
%
% \begin{variable}[int]{\l_@@_long_name_bool, \l_@@_trial_width_dim}
%   The boolean \cs{l_@@_long_name_bool} is \texttt{true} if the width
%   \cs{l_@@_trial_width_dim} of the coffin \cs{l_@@_names_coffin}
%   (containing the current function names) is bigger than the space
%   available in the margin.
%    \begin{macrocode}
\bool_new:N \l_@@_long_name_bool
\dim_new:N \l_@@_trial_width_dim
%    \end{macrocode}
% \end{variable}
%
% \begin{variable}[int]{\l_@@_nested_macro_int}
%   The nesting of \env{macro} environments (this is now~$0$ outside a
%   \env{macro} environment).
%    \begin{macrocode}
\int_new:N \l_@@_nested_macro_int
%    \end{macrocode}
% \end{variable}
%
% \begin{variable}[int]
%   {
%     \l_@@_macro_tested_bool,
%     \g_@@_missing_tests_prop,
%     \g_@@_not_tested_seq,
%     \g_@@_testfiles_seq,
%   }
%   A boolean describing whether the current macro has tests, and some
%   global structures which contain information about test files and
%   which tests are missing.
%    \begin{macrocode}
\bool_new:N \l_@@_macro_tested_bool
\prop_new:N \g_@@_missing_tests_prop
\seq_new:N \g_@@_not_tested_seq
\seq_new:N \g_@@_testfiles_seq
%    \end{macrocode}
% \end{variable}
%
% \begin{variable}[aux]
%   {
%     \l_@@_macro_internal_bool,
%     \l_@@_macro_aux_bool,
%     \l_@@_macro_TF_bool,
%     \l_@@_macro_pTF_bool,
%     \l_@@_macro_EXP_bool,
%     \l_@@_macro_rEXP_bool,
%     \l_@@_macro_var_bool,
%   }
%   Contain information about some options of function/macro
%   environments:
%    \begin{macrocode}
\bool_new:N \l_@@_macro_internal_bool
\bool_new:N \l_@@_macro_aux_bool
\bool_new:N \l_@@_macro_TF_bool
\bool_new:N \l_@@_macro_pTF_bool
\bool_new:N \l_@@_macro_EXP_bool
\bool_new:N \l_@@_macro_rEXP_bool
\bool_new:N \l_@@_macro_var_bool
%    \end{macrocode}
% \end{variable}
%
% \begin{variable}[aux]
%   {
%     \g_@@_lmodern_bool,
%     \g_@@_checkfunc_bool,
%     \g_@@_checktest_bool,
%   }
%   Information about package options.
%    \begin{macrocode}
\bool_new:N \g_@@_lmodern_bool
\bool_new:N \g_@@_checkfunc_bool
\bool_new:N \g_@@_checktest_bool
%    \end{macrocode}
% \end{variable}
%
% \begin{variable}[int]{\l_@@_tmpa_tl, \l_@@_tmpb_tl, \l_@@_tmpa_int}
%   Some temporary variables.
%    \begin{macrocode}
\tl_new:N \l_@@_tmpa_tl
\tl_new:N \l_@@_tmpb_tl
\int_new:N \l_@@_tmpa_int
%    \end{macrocode}
% \end{variable}
%
% \begin{variable}[int]{\l_@@_functions_block_prop}
%   Contains information about the functions to typeset and their
%   variants.
%    \begin{macrocode}
\prop_new:N \l_@@_functions_block_prop
%    \end{macrocode}
% \end{variable}
%
% \begin{variable}[int]{\l_@@_function_input_seq, \l_@@_macro_input_seq}
%   Both the \env{function} and the \env{macro} environments read their
%   argument verbatim, then remove percent signs (at the beginning of
%   lines), tabs and new lines, convert |_@@| and |@@| by
%   |__|\meta{module name} if appropriate, and interpret the result as a
%   comma list, which they store in these two sequences.
%    \begin{macrocode}
\seq_new:N \l_@@_function_input_seq
\seq_new:N \l_@@_macro_input_seq
%    \end{macrocode}
% \end{variable}
%
% \begin{variable}[int]{\c_@@_backslash_tl, \c_@@_backslash_token}
%   A single backslash, as a token list, or as an implicit character
%   token.
%    \begin{macrocode}
\tl_const:Nx \c_@@_backslash_tl { \iow_char:N \\ }
\exp_last_unbraced:NNo
  \cs_new_eq:NN \c_@@_backslash_token { \c_@@_backslash_tl }
%    \end{macrocode}
% \end{variable}
%
% \begin{variable}[int]{\g_@@_function_name_prefix_tl}
%   The \enquote{prefix} of the current function is a backslash if the
%   function starts with it, and otherwise is empty.
%    \begin{macrocode}
\tl_new:N \g_@@_function_name_prefix_tl
%    \end{macrocode}
% \end{variable}
%
% \begin{variable}
%   {\l_@@_index_macro_tl, \l_@@_index_key_tl, \l_@@_index_module_tl}
%   When analyzing a control sequence found within a \env{macrocode}
%   environment, \cs{l_@@_index_macro_tl} holds the control sequence
%   (partially a string), \cs{l_@@_index_key_tl} holds what will be used
%   as a sort key in the index, and \cs{l_@@_index_module_tl} is the
%   subindex in which the control sequence will be listed.
%    \begin{macrocode}
\tl_new:N \l_@@_index_macro_tl
\tl_new:N \l_@@_index_key_tl
\tl_new:N \l_@@_index_module_tl
%    \end{macrocode}
% \end{variable}
%
% \begin{variable}{\g_@@_module_name_tl}
%   The module name, set when reading a line |<@@=|\meta{module}|>|.
%    \begin{macrocode}
\tl_new:N \g_@@_module_name_tl
%    \end{macrocode}
% \end{variable}
%
% \begin{variable}{\c_@@_iow_rule_tl, \c_@@_iow_midrule_tl}
%   $40$~equal signs.
%    \begin{macrocode}
\tl_const:Nn \c_@@_iow_rule_tl
  { ======================================== }
\tl_const:Nn \c_@@_iow_mid_rule_tl
  { -------------------------------------- }
%    \end{macrocode}
% \end{variable}
%
% \begin{variable}
%   {\l_@@_macro_box, \l_@@_macro_index_box, \l_@@_macro_int}
%   A vertical box in which the names given to the macro environment are
%   typeset, a horizontal box in which we store the targets created by
%   indexing commands, and the number of macros so far (including those
%   from surrounding \env{macro} environments).
%    \begin{macrocode}
\box_new:N \l_@@_macro_index_box
\box_new:N \l_@@_macro_box
\int_new:N \l_@@_macro_int
%    \end{macrocode}
% \end{variable}
%
% \begin{variable}[int]
%   {
%     \l_@@_cmd_tl,
%     \l_@@_cmd_index_tl,
%     \l_@@_cmd_replace_bool,
%   }
%   Variables used to control the behaviour of \cs{cmd}, \cs{cs} and
%   \cs{tn}.
%    \begin{macrocode}
\tl_new:N \l_@@_cmd_tl
\tl_new:N \l_@@_cmd_index_tl
\bool_new:N \l_@@_cmd_replace_bool
%    \end{macrocode}
% \end{variable}
%
% \begin{variable}{\l_@@_in_implementation_bool}
%   This boolean is \texttt{true} within the \env{implementation}
%   environment, and \texttt{false} anywhere else.
%    \begin{macrocode}
\bool_new:N \l_@@_in_implementation_bool
%    \end{macrocode}
% \end{variable}
%
% \begin{variable}
%   {
%     \g_@@_typeset_documentation_bool,
%     \g_@@_typeset_implementation_bool
%   }
%   These booleans control whether the documentation/implementation
%   should be typeset.  By default both should be.
%    \begin{macrocode}
\bool_new:N \g_@@_typeset_documentation_bool
\bool_new:N \g_@@_typeset_implementation_bool
\bool_set_true:N \g_@@_typeset_documentation_bool
\bool_set_true:N \g_@@_typeset_implementation_bool
%    \end{macrocode}
% \end{variable}
%
% \begin{variable}{\g_@@_base_name_tl, \l_@@_variants_prop}
%   The name of the macro which is being documented (without its
%   signature), and a property list mapping base forms of variants to
%   all variants which have the same base form.
%    \begin{macrocode}
\tl_new:N \g_@@_base_name_tl
\prop_new:N \l_@@_variants_prop
%    \end{macrocode}
% \end{variable}
%
% \begin{variable}[int]{\l_@@_no_label_bool}
%   This boolean prevents the insertions of \tn{label}s in a
%   \env{function} environment.  This is only useful when a function's
%   documentation appears multiple times, for instance in
%   \file{source3body.tex} some examples repeat the documentation from
%   actual functions.
%    \begin{macrocode}
\bool_new:N \l_@@_no_label_bool
%    \end{macrocode}
% \end{variable}
%
% \begin{macro}[aux]{\@@_tmpa:w, \@@_tmpb:w}
%   Auxiliary macros for temporary use.
%    \begin{macrocode}
\cs_new_eq:NN \@@_tmpa:w ?
\cs_new_eq:NN \@@_tmpb:w ?
%    \end{macrocode}
% \end{macro}
%
% Bruno: can I delete this next line?
%    \begin{macrocode}
% \int_new:N \c@CodelineNo
%    \end{macrocode}
%
% \subsection{Options and configuration}
%
%    \begin{macrocode}
\DeclareOption { a5paper } { \@latexerr { Option~not~supported } { } }
%    \end{macrocode}
%
%    \begin{macrocode}
\DeclareOption { full }
  {
    \bool_gset_true:N \g_@@_typeset_documentation_bool
    \bool_gset_true:N \g_@@_typeset_implementation_bool
  }
\DeclareOption { onlydoc }
  {
    \bool_gset_true:N \g_@@_typeset_documentation_bool
    \bool_gset_false:N \g_@@_typeset_implementation_bool
  }
%    \end{macrocode}
%
%    \begin{macrocode}
\DeclareOption { check }
  { \bool_gset_true:N \g_@@_checkfunc_bool }
\DeclareOption { nocheck }
  { \bool_gset_false:N \g_@@_checkfunc_bool }
%    \end{macrocode}
%
%    \begin{macrocode}
\DeclareOption { checktest }
  { \bool_gset_true:N \g_@@_checktest_bool }
\DeclareOption { nochecktest }
  { \bool_gset_false:N \g_@@_checktest_bool }
%    \end{macrocode}
%
%    \begin{macrocode}
\DeclareOption { cm-default }
  { \bool_gset_false:N \g_@@_lmodern_bool }
\DeclareOption { lm-default }
  { \bool_gset_true:N \g_@@_lmodern_bool }
%    \end{macrocode}
%
%    \begin{macrocode}
\DeclareOption* { \PassOptionsToClass { \CurrentOption } { article } }
\ExecuteOptions { full, a4paper, nocheck, nochecktest, lm-default }
%    \end{macrocode}
%
% Input a local configuration file, if it exists, with a message to the
% console that this has happened. Since we distribute a \file{.cfg} file
% with the class, this should usually always be true. Therefore, check
% for \cs{ExplMakeTitle} (defined in \enquote{our} \file{.cfg} file) and
% only output the informational message if it's not found.
%
%    \begin{macrocode}
\msg_new:nnn { l3doc } { input-cfg }
  { Local~config~file~l3doc.cfg~loaded. }
\file_if_exist:nT { l3doc.cfg }
  {
    \file_input:nT { l3doc.cfg }
      {
        \cs_if_exist:cF { ExplMakeTitle }
          { \msg_info:nn { l3doc } { input-cfg } }
      }
  }
%    \end{macrocode}
%
%    \begin{macrocode}
\ProcessOptions
%    \end{macrocode}
%
%
% \subsection{Class and package loading}
%
%    \begin{macrocode}
\LoadClass{article}
\RequirePackage{doc}
\RequirePackage
  {
    array,
    alphalph,
    amsmath,
    amssymb,
    booktabs,
    color,
    colortbl,
    hologo,
    enumitem,
    pifont,
    textcomp,
    trace,
    underscore,
    csquotes,
    fancyvrb,
    verbatim
  }
\raggedbottom
%    \end{macrocode}
%
% Depending on the option, load the package \pkg{lmodern} to set the
% font.  Then replace the italic typewriter font with the oblique shape
% instead; the former makes my skin crawl. (Will, Aug 2011)
%    \begin{macrocode}
\bool_if:NT \g_@@_lmodern_bool
  {
    \RequirePackage[T1]{fontenc}
    \RequirePackage{lmodern}
    \group_begin:
      \ttfamily
      \DeclareFontShape{T1}{lmtt}{m}{it}{<->ec-lmtto10}{}
    \group_end:
  }
%    \end{macrocode}
%
% Must be last, as usual.
%    \begin{macrocode}
\RequirePackage{hypdoc}
%    \end{macrocode}
%
% \subsection{Configuration and tweaks}
%
% \begin{macro}{\MakePrivateLetters}
%   A few more letters are \enquote{private} in a \LaTeX3 programming
%   environment.
%    \begin{macrocode}
\cs_gset_nopar:Npn \MakePrivateLetters
  {
    \char_set_catcode_letter:N \@
    \char_set_catcode_letter:N \_
    \char_set_catcode_letter:N \:
  }
%    \end{macrocode}
% \end{macro}
%
% \begin{macro}{CodelineNo}
%   Some configurations which have to do with line numbering.
%    \begin{macrocode}
\setcounter{StandardModuleDepth}{1}
\@addtoreset{CodelineNo}{part}
\tl_replace_once:Nnn \theCodelineNo
  { \HDorg@theCodelineNo }
  { \textcolor[gray]{0.5} { \sffamily\tiny\arabic{CodelineNo} } }
%    \end{macrocode}
% \end{macro}
%
% \begin{macro}{\verbatim, \endverbatim}
%   In \file{.dtx} documents, the \env{verbatim} environment adds extra
%   space because it only removes the first \enquote{\%} sign, and not
%   the indentation (typically a space).  Fix it with \pkg{fancyvrb}:
%    \begin{macrocode}
\fvset{gobble=2}
\cs_gset_eq:NN \verbatim \Verbatim
\cs_gset_eq:NN \endverbatim \endVerbatim
%    \end{macrocode}
% \end{macro}
%
% \subsection{Design}
%
% Increase the text width slightly so that width the standard fonts
% 72~columns of code may appear in a \env{macrocode} environment.
% Increase the marginpar width slightly, for long command names.  And
% increase the left margin by a similar amount.
%    \begin{macrocode}
\setlength   \textwidth      { 385pt }
\addtolength \marginparwidth {  30pt }
\addtolength \oddsidemargin  {  20pt }
\addtolength \evensidemargin {  20pt }
%    \end{macrocode}
% (These were introduced when \cls{article} was the documentclass, but
% I've left them here for now to remind me to do something about them
% later.)
%
% \begin{macro}{\list}
% \begin{macro}[aux]{\@@_oldlist:nn}
%   Customise lists.
%    \begin{macrocode}
\cs_new_eq:NN \@@_oldlist:nn \list
\cs_gset_nopar:Npn \list #1 #2
  { \@@_oldlist:nn {#1} { #2 \dim_zero:N \listparindent } }
\setlength \parindent  { 2em }
\setlength \itemindent { 0pt }
\setlength \parskip    { 0pt plus 3pt minus 0pt }
%    \end{macrocode}
% \end{macro}
% \end{macro}
%
% \begin{macro}{\partname}
%   Use \enquote{File} as a name in Part titles.
%    \begin{macrocode}
\tl_gset:Nn \partname {File}
%    \end{macrocode}
% \end{macro}
%
% \begin{macro}{\l@section, \l@subsection}
%   Customise the table of contents (as we have so many sections).
%   Different design and/or structure is called for).
%    \begin{macrocode}
\@addtoreset{section}{part}
\cs_gset_nopar:Npn \l@section #1#2
  {
    \ifnum \c@tocdepth >\z@
      \addpenalty\@secpenalty
      \addvspace{1.0em \@plus\p@}
      \setlength\@tempdima{2.5em}  % was 1.5em
      \begingroup
        \parindent \z@ \rightskip \@pnumwidth
        \parfillskip -\@pnumwidth
        \leavevmode \bfseries
        \advance\leftskip\@tempdima
        \hskip -\leftskip
        #1\nobreak\hfil \nobreak\hb@xt@\@pnumwidth{\hss #2}\par
      \endgroup
    \fi
  }
\cs_gset_nopar:Npn \l@subsection
  { \@dottedtocline{2}{2.5em}{2.3em} }  % #2 = 1.5em
%    \end{macrocode}
% \end{macro}
%
% \subsection{Text markup}
%
% Make "|" and |"| be \enquote{short verb} characters, but not in the
% document preamble, where an active character may interfere with
% packages that are loaded.  Remove these short-hands at the end of the
% document before reading the \file{.aux} file, as they may appear in
% labels (for instance, \pkg{l3fp} documents an operation "||").
%    \begin{macrocode}
\AtBeginDocument
  {
    \MakeShortVerb \"
    \MakeShortVerb \|
  }
\AtEndDocument
  {
    \DeleteShortVerb \"
    \DeleteShortVerb \|
  }
%    \end{macrocode}
%
% \begin{macro}{\eTeX, \IniTeX, \Lua, \LuaTeX, \pdfTeX, \XeTeX,
%   \pTeX, \upTeX, \epTeX, \eupTeX}
%   Some commands for logos.
%    \begin{macrocode}
\providecommand*\eTeX{\hologo{eTeX}}
\providecommand*\IniTeX{\hologo{iniTeX}}
\providecommand*\Lua{Lua}
\providecommand*\LuaTeX{\hologo{LuaTeX}}
\providecommand*\pdfTeX{\hologo{pdfTeX}}
\providecommand*\XeTeX{\hologo{XeTeX}}
\providecommand*\pTeX{p\kern-.2em\hologo{TeX}}
\providecommand*\upTeX{up\kern-.2em\hologo{TeX}}
\providecommand*\epTeX{$\varepsilon$-\pTeX}
\providecommand*\eupTeX{$\varepsilon$-\upTeX}
%    \end{macrocode}
% \end{macro}
%
% \begin{macro}{\cmd, \cs, \tn}
%   To work within bookmarks, these commands must be expandable.  They
%   rely on a common auxiliary \cs{@@_cmd:nn} which receives as
%   arguments the options and some tokens whose string representation
%   starts with a backslash (to support cases such as |\cs{pkg_\ldots}|,
%   we do not turn the whole argument into a string).
%    \begin{macrocode}
\DeclareExpandableDocumentCommand \cmd { O{} m }
  { \@@_cmd:no {#1} { \token_to_str:N #2 } }
\DeclareExpandableDocumentCommand \cs  { O{} m }
  { \@@_cmd:no {#1} { \c_@@_backslash_tl #2 } }
\DeclareExpandableDocumentCommand \tn  { O{} m }
  {
    \@@_cmd:no
      { index = TeX , replace = false , #1 }
      { \c_@@_backslash_tl #2 }
  }
%    \end{macrocode}
% \end{macro}
%
% \begin{macro}{\meta}
%   To work within a bookmark, this command must be expandable.
%    \begin{macrocode}
\DeclareExpandableDocumentCommand { \meta } { m }
  { \@@_meta:n {#1} }
%    \end{macrocode}
% \end{macro}
%
% \begin{macro}{\Arg, \marg, \oarg, \parg}
%   |\marg{text}| prints \marg{text}, \enquote{mandatory argument}.\\
%   |\oarg{text}| prints \oarg{text}, \enquote{optional argument}.\\
%   |\parg{te,xt}| prints \parg{te,xt}, \enquote{picture mode argument}.
%   Finally, \cs{Arg} is the same as \cs{marg}.
%    \begin{macrocode}
\newcommand\Arg[1]
  { \texttt{\char`\{} \meta{#1} \texttt{\char`\}} }
\providecommand\marg[1]{ \Arg{#1} }
\providecommand\oarg[1]{ \texttt[ \meta{#1} \texttt] }
\providecommand\parg[1]{ \texttt( \meta{#1} \texttt) }
%    \end{macrocode}
% \end{macro}
%
% \begin{macro}{\file, \env, \pkg, \cls}
%   This list may change\dots this is just my preference for markup.
%    \begin{macrocode}
\DeclareRobustCommand \file {\nolinkurl}
\DeclareRobustCommand \env {\texttt}
\DeclareRobustCommand \pkg {\textsf}
\DeclareRobustCommand \cls {\textsf}
%    \end{macrocode}
% \end{macro}
%
% \begin{macro}{\EnableDocumentation, \EnableImplementation}
% \begin{macro}{\DisableDocumentation, \DisableImplementation}
%   Control whether to typeset the documentation/implementation or not.
%   These simply set two switches.
%    \begin{macrocode}
\NewDocumentCommand \EnableDocumentation { }
  { \bool_gset_true:N \g_@@_typeset_documentation_bool }
\NewDocumentCommand \EnableImplementation { }
  { \bool_gset_true:N \g_@@_typeset_implementation_bool }
\NewDocumentCommand \DisableDocumentation { }
  { \bool_gset_false:N \g_@@_typeset_documentation_bool }
\NewDocumentCommand \DisableImplementation { }
  { \bool_gset_false:N \g_@@_typeset_implementation_bool }
%    \end{macrocode}
% \end{macro}
% \end{macro}
%
% \begin{environment}{documentation}
% \begin{environment}{implementation}
%   If the documentation/implementation should be typeset, then simply
%   set the boolean \cs{l_@@_in_implementation_bool} which indicates
%   whether we are within the implementation section.  Otherwise use
%   \cs{comment} (and a paired \cs{endcomment}).
%    \begin{macrocode}
\NewDocumentEnvironment { documentation } { }
  {
    \bool_if:NTF \g_@@_typeset_documentation_bool
      { \bool_set_false:N \l_@@_in_implementation_bool }
      { \comment }
  }
  { \bool_if:NF \g_@@_typeset_documentation_bool { \endcomment } }
\NewDocumentEnvironment { implementation } { }
  {
    \bool_if:NTF \g_@@_typeset_implementation_bool
      { \bool_set_true:N \l_@@_in_implementation_bool }
      { \comment }
  }
  { \bool_if:NF \g_@@_typeset_implementation_bool { \endcomment } }
%    \end{macrocode}
% \end{environment}
% \end{environment}
%
% \begin{environment}{variable}
%   The \env{variable} environment behaves as a \env{function} or
%   \env{macro} environment depending on the part of the document.
%    \begin{macrocode}
\DeclareDocumentEnvironment { variable } { O{} +v }
  {
    \bool_if:NTF \l_@@_in_implementation_bool
      { \@@_macro:nnw { var , #1 } {#2} }
      { \@@_function:nnw {#1} {#2} }
  }
  {
    \bool_if:NTF \l_@@_in_implementation_bool
      { \@@_macro_end: }
      { \@@_function_end: }
  }
%    \end{macrocode}
% \end{environment}
%
% \begin{environment}{function}
% \begin{environment}{macro}
%   Environment for documenting function(s), and environment for
%   documenting the implementation of a macro.
%    \begin{macrocode}
\DeclareDocumentEnvironment { function } { O{} +v }
  { \@@_function:nnw {#1} {#2} }
  { \@@_function_end: }
\DeclareDocumentEnvironment { macro } { O{} +v }
  { \@@_macro:nnw {#1} {#2} }
  { \@@_macro_end: }
%    \end{macrocode}
% \end{environment}
% \end{environment}
%
% \begin{environment}{syntax}
%   Syntax block placed next to the list of functions to illustrate
%   their use.  TODO: test that the \env{syntax} environment is only
%   used inside the \env{function} environment, and that it only appears
%   once.
%    \begin{macrocode}
\NewDocumentEnvironment { syntax } { }
  { \@@_syntax:w }
  {
    \@@_syntax_end:
    \ignorespacesafterend
  }
%    \end{macrocode}
% \end{environment}
%
% \begin{environment}{texnote}
%   Used to describe information destined to \TeX{} experts only.
%    \begin{macrocode}
\NewDocumentEnvironment { texnote } { }
  {
    \endgraf
    \vspace{3mm}
    \small\textbf{\TeX~hackers~note:}
  }
  {
    \vspace{3mm}
  }
%    \end{macrocode}
% \end{environment}
%
% \begin{environment}{arguments}
%   This environment is designed to be used within a \env{macro}
%   environment to describe the arguments of the macro/function.
%    \begin{macrocode}
\NewDocumentEnvironment { arguments } { }
  {
    \enumerate [
      nolistsep ,
      label = \texttt{\#\arabic*} ~ : ,
      labelsep = * ,
    ]
  }
  {
    \endenumerate
  }
%    \end{macrocode}
% \end{environment}
%
% \subsubsection{Implementing text markup}
%
% Keys for \cs{cmd}, \cs{cs} and \cs{tn}.
%    \begin{macrocode}
\keys_define:nn { l3doc/cmd }
  {
    index     .tl_set:N     = \l_@@_cmd_index_tl        ,
    replace   .bool_set:N   = \l_@@_cmd_replace_bool    ,
  }
%    \end{macrocode}
%
% \begin{macro}[int]{\@@_cmd:nn, \@@_cmd:no}
% \begin{macro}[aux]{\@@_cmd_aux:nn}
%   Within a |pdfstring|, use the second argument directly.  Otherwise
%   call \cs{@@_cmd_aux:nn}: the indirection through
%   \cs{CodedocUnexpandedTokens} and \cs{CodedocUseCs} makes things
%   work when they pass through a
%   file.  Apply the key--value \meta{options}~|#1| after setting some
%   default values.  Then (unless |replace=false|) replace |@@| in~|#2|,
%   which is a bit tricky: the |_| must be given the catcode expected by
%   \cs{@@_replace_at_at:N}, but should be reverted to their original
%   catcode (normally active, needed for line-breaking) without
%   rescanning the whole argument.  Then typeset the command in
%   \tn{verbatim@font}, after turning it to harmless characters if
%   needed (and keeping the underscore breakable); in any case, spaces
%   must be turned into \tn{@xobeysp}.  Finally, produce an index entry.
%    \begin{macrocode}
\cs_new:Npn \@@_cmd:nn #1#2
  {
    \texorpdfstring
      {
        \CodedocUnexpandedTokens
          { \CodedocUseCs { @@_cmd_aux:nn } {#1} {#2} }
      }
      {#2}
  }
\cs_generate_variant:Nn \@@_cmd:nn { no }
\cs_new_protected:Npn \@@_cmd_aux:nn #1#2
  {
    \bool_set_true:N \l_@@_cmd_replace_bool
    \tl_set:Nn \l_@@_cmd_index_tl { \q_no_value }
    \keys_set:nn { l3doc/cmd } {#1}
    \tl_set:No \l_@@_cmd_tl { \token_to_str:N #2 }
    \bool_if:NT \l_@@_cmd_replace_bool
      {
        \tl_set_rescan:Nnn \l_@@_tmpb_tl { } { _ }
        \tl_replace_all:Non \l_@@_cmd_tl \l_@@_tmpb_tl { _ }
        \@@_replace_at_at:N \l_@@_cmd_tl
        \tl_replace_all:Nno \l_@@_cmd_tl { _ } \l_@@_tmpb_tl
      }
    \mode_if_math:T { \mbox }
    {
      \verbatim@font
      \int_compare:nNnF
        { \tl_count:N \l_@@_cmd_tl }
        < { \tl_count:f { \tl_to_str:N \l_@@_cmd_tl } }
        {
          \tl_set:Nx \l_@@_cmd_tl { \tl_to_str:N \l_@@_cmd_tl }
          \tl_replace_all:Non \l_@@_cmd_tl
            { \token_to_str:N _ } { \_ }
        }
      \tl_replace_all:Nnn \l_@@_cmd_tl { ~ } { \@xobeysp }
      \l_@@_cmd_tl
    }
    \exp_args:No \@@_key_get:n { \l_@@_cmd_tl }
    \quark_if_no_value:NF \l_@@_cmd_index_tl
      { \tl_set_eq:NN \l_@@_index_module_tl \l_@@_cmd_index_tl }
    \@@_special_index_module:ooon
      { \l_@@_index_key_tl }
      { \l_@@_index_macro_tl }
      { \l_@@_index_module_tl }
      { }
  }
%    \end{macrocode}
% \end{macro}
% \end{macro}
%
% \begin{macro}
%   {
%     \@@_meta:n,
%     \@@_meta_aux:n,
%     \@@_ensuremath_sb:n,
%     \@@_meta_original:n
%   }
%   Store |#1| in \cs{l_@@_tmpa_tl} and replaces every underscore,
%   regardless of its category (\enquote{math toggle},
%   \enquote{alignment}, \enquote{superscript}, \enquote{subscript},
%   \enquote{letter}, \enquote{other}, or \enquote{active}) by
%   \cs{@@_ensuremath_sb:n} (which creates math subscripts), then runs
%   the code used for \tn{meta} in \pkg{doc.sty}.
%    \begin{macrocode}
\cs_new:Npn \@@_meta:n #1
  {
    \texorpdfstring
      {
        \CodedocUnexpandedTokens
          { \CodedocUseCs { @@_meta_aux:n } {#1} }
      }
      { < #1 > }
  }
\cs_new_protected:Npn \@@_meta_aux:n #1
  {
    \tl_set:Nn \l_@@_tmpa_tl {#1}
    \tl_map_inline:nn
      { { 3 } { 4 } { 7 } { 8 } { 11 } { 12 } { 13 } }
      {
        \tl_set_rescan:Nnn \l_@@_tmpb_tl
          { \char_set_catcode:nn { `_ } {##1} } { _ }
        \tl_replace_all:Non \l_@@_tmpa_tl \l_@@_tmpb_tl
          { \@@_ensuremath_sb:n }
      }
    \exp_args:NV \@@_meta_original:n \l_@@_tmpa_tl
  }
\cs_new_protected:Npn \@@_ensuremath_sb:n #1
  { \ensuremath { \sb {#1} } }
\cs_new_protected:Npn \@@_meta_original:n #1
  {
    \ensuremath \langle
    \mode_if_math:T { \nfss@text }
    {
      \meta@font@select
      \edef \meta@hyphen@restore
        { \hyphenchar \the \font \the \hyphenchar \font }
      \hyphenchar \font \m@ne
      \language \l@nohyphenation
      #1 \/
      \meta@hyphen@restore
    }
    \ensuremath \rangle
  }
%    \end{macrocode}
% \end{macro}
%
% \subsubsection{The \env{function} environment}
%
%    \begin{macrocode}
\keys_define:nn { l3doc/function }
  {
    TF .code:n =
      {
        \bool_set_true:N \l_@@_macro_TF_bool
      } ,
    EXP .code:n =
      {
        \bool_set_true:N \l_@@_macro_EXP_bool
        \bool_set_false:N \l_@@_macro_rEXP_bool
      } ,
    rEXP .code:n =
      {
        \bool_set_false:N \l_@@_macro_EXP_bool
        \bool_set_true:N \l_@@_macro_rEXP_bool
      } ,
    pTF .code:n =
      {
        \bool_set_true:N \l_@@_macro_pTF_bool
        \bool_set_true:N \l_@@_macro_TF_bool
        \bool_set_true:N \l_@@_macro_EXP_bool
        \bool_set_false:N \l_@@_macro_rEXP_bool
      } ,
    added .tl_set:N = \l_@@_date_added_tl ,
    updated .tl_set:N = \l_@@_date_updated_tl ,
    tested .code:n = { } ,
    no-label .bool_set:N = \l_@@_no_label_bool ,
  }
%    \end{macrocode}
%
%
% \begin{macro}[int]{\@@_function:nnw}
%   \begin{arguments}
%     \item Key--value list.
%     \item Comma-separated list of functions; input has already been
%       sanitised by catcode changes before reading the argument.
%   \end{arguments}
% \begin{macro}[int]{\@@_function_end:}
%   Make sure any paragraph is finished, and similar safe practices at
%   the beginning of an environment which will typeset material.
%   Initialize some variables.  Parse the key--value list.  Clean up the
%   list of functions, then go through them to extract some data.  After
%   this, typeset the function names in the coffin
%   \cs{l_@@_names_coffin} and measure it to know if it fits in the
%   margin.  Finally, start a vertical coffin for the main part of the
%   environment.  This coffin stops when the environment ends, then all
%   the pieces are assembled into a single coffin, which is typeset.
%    \begin{macrocode}
\cs_new_protected:Npn \@@_function:nnw #1#2
  {
    \@@_function_typeset_start:
    \@@_function_init:
    \keys_set:nn { l3doc/function } {#1}
    \@@_verb_get_seq:nN {#2} \l_@@_function_input_seq
    \@@_function_parse:
    \@@_function_typeset:
    \@@_function_descr_start:w
  }
\cs_new_protected_nopar:Npn \@@_function_end:
  {
    \@@_function_descr_stop:
    \@@_function_assemble:
    \@@_function_typeset_stop:
  }
%    \end{macrocode}
% \end{macro}
% \end{macro}
%
% \begin{macro}[aux]
%   {\@@_function_typeset_start:, \@@_function_typeset_stop:}
%   At the start of the \env{function} environment, before performing
%   any assignment, close the last paragraph, and set up the typesetting
%   scene.  Further code typesets a coffin, so we end the paragraph and
%   allow a page break.
%    \begin{macrocode}
\cs_new_protected_nopar:Npn \@@_function_typeset_start:
  {
    \par \bigskip \noindent
    \phantomsection
  }
\cs_new_protected_nopar:Npn \@@_function_typeset_stop:
  {
    \par
    \allowbreak
  }
%    \end{macrocode}
% \end{macro}
%
% \begin{macro}[aux]{\@@_function_init:}
%   Allow |<...>| to be used as markup for |\meta{...}|.  Clear various
%   variables.
%    \begin{macrocode}
\group_begin:
  \char_set_catcode_active:N \<
  \cs_new_protected:Npn \@@_function_init:
    {
      \coffin_clear:N \l_@@_descr_coffin
      \box_gclear:N \g_@@_syntax_box
      \coffin_clear:N \l_@@_syntax_coffin
      \coffin_clear:N \l_@@_names_coffin
      \bool_set_false:N \l_@@_macro_TF_bool
      \bool_set_false:N \l_@@_macro_pTF_bool
      \bool_set_false:N \l_@@_macro_EXP_bool
      \bool_set_false:N \l_@@_macro_rEXP_bool
      \bool_set_false:N \l_@@_no_label_bool
      \char_set_catcode_active:N \<
      \cs_set_protected_nopar:Npn < ##1 > { \meta {##1} }
  }
\group_end:
%    \end{macrocode}
% \end{macro}
%
% \begin{macro}[aux]
%   {
%     \@@_function_parse:,
%     \@@_function_parse_one:n,
%     \@@_function_parse_cs_aux:nnN
%   }
%   The idea here is to populate \cs{l_@@_functions_block_prop} with
%   information about the functions being typeset and their variants.
%   The property list uses a specific format: the keys contained are the
%   function \enquote{names} (stripped of signature) with values
%   specified by a comma-separated list of (braced) signatures:
%   \begin{center}
%     |>  {my_function}  =>  {| \meta{sig_1} |,| \meta{sig_2} |, ... }|
%   \end{center}
%   In the case where the function has no signature (no colon), we use
%   a \meta{sig} equal to \cs{scan_stop:}.
%   This somewhat arcane syntax is chosen to distinguish between all
%   variants of control sequence that may be being documented
%   here.  (With or without \pkg{expl3} function syntax.)
%
%   If |#1| does not start with |\| then the \enquote{name prefix} is
%   empty and we store \cs{scan_stop:} as the value for the key |#1|, not
%   bothering with variants.  If instead |#1| starts with |\| then the
%   \enquote{name prefix} is |\| and we omit the first character of |#1|
%   subsequently.  If the next character is |:| then we have one of the
%   weird functions named |\::N| and so on, and we ignore variants,
%   putting \cs{scan_stop:} directly as the value for the key |::N|.
%   Otherwise, we have a regular control sequence, which we decompose
%   using \cs{@@_function_parse_cs_aux:nnN}.
%    \begin{macrocode}
\cs_new_protected_nopar:Npn \@@_function_parse:
  {
    \seq_map_function:NN
      \l_@@_function_input_seq
      \@@_function_parse_one:n
  }
\cs_new_protected:Npn \@@_function_parse_one:n #1
  {
    \tl_if_head_eq_charcode:nNTF {#1} \c_@@_backslash_token
      {
        \tl_gset_eq:NN \g_@@_function_name_prefix_tl \c_@@_backslash_tl
        \exp_args:No \tl_if_head_eq_charcode:nNTF { \use_none:n #1 } :
          {
            \prop_put:Nxn \l_@@_functions_block_prop
              { \use_none:n #1 } { \scan_stop: }
          }
          {
            \exp_args:Nc \__cs_split_function:NN { \use_none:n #1 }
              \@@_function_parse_cs_aux:nnN
          }
      }
      {
        \tl_gclear:N \g_@@_function_name_prefix_tl
        \prop_put:Nnn \l_@@_functions_block_prop {#1} { \scan_stop: }
      }
  }
\cs_new_protected:Npn \@@_function_parse_cs_aux:nnN #1#2#3
  {
    \prop_get:NnNF
      \l_@@_functions_block_prop {#1} \l_@@_tmpb_tl
      { \tl_clear:N \l_@@_tmpb_tl }
    \prop_put:Nnx \l_@@_functions_block_prop {#1}
      {
        \l_@@_tmpb_tl ,
        \bool_if:NTF #3 { {#2} } { \scan_stop: }
      }
  }
%    \end{macrocode}
% \end{macro}
%
% \begin{macro}[aux]{\@@_function_typeset:}
%   Typeset in the coffin \cs{l_@@_names_coffin} the functions listed in
%   \cs{l_@@_functions_block_prop} and the relevant dates, then set
%   \cs{l_@@_long_name_bool} to be \texttt{true} if this coffin is
%   larger than the available width in the margin.  The function
%   \cs{@@_typeset_names:} is quite involved hence given later.
%    \begin{macrocode}
\cs_new_protected_nopar:Npn \@@_function_typeset:
  {
    \dim_zero:N \l_@@_trial_width_dim
    \hcoffin_set:Nn \l_@@_names_coffin { \@@_typeset_names: }
    \dim_set:Nn \l_@@_trial_width_dim
      { \box_wd:N \l_@@_names_coffin }
    \bool_set:Nn \l_@@_long_name_bool
      { \dim_compare_p:nNn \l_@@_trial_width_dim > \marginparwidth }
  }
%    \end{macrocode}
% \end{macro}
%
% \begin{macro}[aux]
%   {\@@_function_descr_start:w, \@@_function_descr_stop:}
%   The last step in \cs{@@_function:nnw} (the beginning of a
%   \env{function} environment) is to open a coffin which will capture
%   the description of the function, namely the body of the
%   \env{function} environment.  This is closed by \cs{@@_function_end:}
%   (the end of a \env{function} environment).
%    \begin{macrocode}
\cs_new_protected_nopar:Npn \@@_function_descr_start:w
  {
    \vcoffin_set:Nnw \l_@@_descr_coffin { \textwidth }
      \noindent \ignorespaces
  }
\cs_new_protected_nopar:Npn \@@_function_descr_stop:
  { \vcoffin_set_end: }
%    \end{macrocode}
% \end{macro}
%
% \begin{macro}[aux]{\@@_function_assemble:}
%   The box \cs{g_@@_syntax_box} contains the contents of the syntax
%   environment if it was used.  Now that we have all the pieces, join
%   together the syntax coffin, the names coffin, and the description
%   coffin.  The relative positions depend on whether the names coffin
%   fits in the margin.  Then typeset the combination.
%    \begin{macrocode}
\cs_new_protected_nopar:Npn \@@_function_assemble:
  {
    \hcoffin_set:Nn  \l_@@_syntax_coffin
      { \box_use:N \g_@@_syntax_box }
    \bool_if:NTF \l_@@_long_name_bool
      {
        \coffin_join:NnnNnnnn
          \l_@@_output_coffin {hc} {vc}
          \l_@@_syntax_coffin {l} {T}
          {0pt} {0pt}
        \coffin_join:NnnNnnnn
          \l_@@_output_coffin {l} {t}
          \l_@@_names_coffin  {r} {t}
          {-\marginparsep} {0pt}
        \coffin_join:NnnNnnnn
          \l_@@_output_coffin {l} {b}
          \l_@@_descr_coffin  {l} {t}
          {0.75\marginparwidth + \marginparsep} {-\medskipamount}
        \coffin_typeset:Nnnnn \l_@@_output_coffin
          {\l_@@_descr_coffin-l} {\l_@@_descr_coffin-t}
          {0pt} {0pt}
      }
      {
        \coffin_join:NnnNnnnn
          \l_@@_output_coffin {hc} {vc}
          \l_@@_syntax_coffin {l} {t}
          {0pt} {0pt}
        \coffin_join:NnnNnnnn
          \l_@@_output_coffin {l} {b}
          \l_@@_descr_coffin  {l} {t}
          {0pt} {-\medskipamount}
        \coffin_join:NnnNnnnn
          \l_@@_output_coffin {l} {t}
          \l_@@_names_coffin  {r} {t}
          {-\marginparsep} {0pt}
        \coffin_typeset:Nnnnn \l_@@_output_coffin
          {\l_@@_syntax_coffin-l} {\l_@@_syntax_coffin-T}
          {0pt} {0pt}
      }
  }
%    \end{macrocode}
% \end{macro}
%
% \begin{macro}[aux]{\@@_typeset_names:}
%   This function builds the \cs{l_@@_names_coffin} by typesetting the
%   function names (with variants) and the relevant dates in a
%   \env{tabular} environment.  The use of rules \tn{toprule},
%   \tn{midrule} and \tn{bottomrule} requires whatever lies between the
%   last |\\| and the rule to be expandable, making our lives a bit
%   complicated.
%    \begin{macrocode}
\cs_new_protected_nopar:Npn \@@_typeset_names:
  {
    \small\ttfamily
    \begin{tabular} { @{} l @{} r @{} }
      \toprule
      \@@_typeset_functions:
      \@@_typeset_dates:
      \bottomrule
    \end{tabular}
    \normalfont\normalsize
  }
%    \end{macrocode}
% \end{macro}
%
% \begin{macro}[aux]{\@@_typeset_functions:}
% \begin{macro}[aux]
%   {\@@_typeset_functions_auxi:nn, \@@_typeset_functions_auxii:nn}
%   \begin{arguments}
%     \item Function/macro/variable name (stripped of signature in the
%       first case, if any)
%     \item Comma-list containing information about the signature
%       variants being documented (if any)
%   \end{arguments}
%   The mapping to \cs{l_@@_functions_block_prop} cannot use
%   \cs{prop_map_inline:Nn} because the code following |\\| would not be
%   expandable, thus breaking \tn{bottomrule}.  The loop goes as
%   follows: go through the list of variants |#2|, and collect them by
%   \enquote{base form} in a \cs{l_@@_variants_prop} whose keys are the
%   base form and whose values are the variants which can be obtained
%   from that base form using \cs{cs_generate_variant:Nn}.  Then start a
%   nested loop through these base forms, calling
%   \cs{@@_typeset_functions_auxii:nn}.  This auxiliary does not care
%   about the base form.  It stores the variants it receives in
%   \cs{g_@@_variants_clist}, remove the first, which will be displayed
%   more prominently, and reconstructs its actual name, passing it to
%   the \cs{@@_typeset_functions_auxiii:N} auxiliary.
%
%   Braces around |##1| are crucial since this item can be empty.
%    \begin{macrocode}
\cs_new_protected_nopar:Npn \@@_typeset_functions:
  {
    \prop_map_function:NN \l_@@_functions_block_prop
      \@@_typeset_functions_auxi:nn
  }
\cs_new_protected:Npn \@@_typeset_functions_auxi:nn #1#2
  {
    \tl_gset:Nn \g_@@_base_name_tl {#1}
    \prop_clear:N \l_@@_variants_prop
    \clist_map_inline:nn {#2}
      {
        \tl_set:Nx \l_@@_tmpa_tl
          { \@@_signature_base_form:n {##1} }
        \prop_get:NoNTF \l_@@_variants_prop
          \l_@@_tmpa_tl \l_@@_tmpb_tl
          { \tl_put_right:Nn \l_@@_tmpb_tl { , {##1} } }
          { \tl_set:Nn \l_@@_tmpb_tl { {##1} } }
        \prop_put:Noo \l_@@_variants_prop
          \l_@@_tmpa_tl \l_@@_tmpb_tl
      }
    \prop_map_function:NN \l_@@_variants_prop
      \@@_typeset_functions_auxii:nn
  }
\cs_new_protected:Npn \@@_typeset_functions_auxii:nn #1#2
  {
    \clist_gset:Nn \g_@@_variants_clist {#2}
    \clist_gpop:NN \g_@@_variants_clist \l_@@_tmpb_tl
    \exp_args:Nc \@@_typeset_functions_auxiii:N
      {
        \g_@@_base_name_tl
        \exp_last_unbraced:Nf \token_if_eq_meaning:NNF
          { \tl_head:f { \l_@@_tmpb_tl ? } }
          \scan_stop:
          { : \l_@@_tmpb_tl }
      }
  }
%    \end{macrocode}
% \end{macro}
% \end{macro}
%
% \begin{macro}[aux]{\@@_typeset_functions_auxiii:N}
%    \begin{macrocode}
\cs_new_protected:Npn \@@_typeset_functions_auxiii:N #1
  {
    \bool_if:NT \l_@@_macro_pTF_bool
      {
        \tl_set:Nx \l_@@_pTF_name_tl
          { \@@_predicate_from_base:N #1 }
        \@@_function_index:x { \l_@@_pTF_name_tl }
      }
    \@@_function_index:x
      {
        \cs_to_str:N #1
        \bool_if:NT \l_@@_macro_TF_bool { \tl_to_str:n {TF} }
      }
    \bool_if:NT \l_@@_macro_pTF_bool
      {
        \exp_args:Nc \@@_typeset_function_block:NN
          { \l_@@_pTF_name_tl } \c_false_bool
      }
    \@@_typeset_function_block:NN #1 \l_@@_macro_TF_bool
  }
\cs_new_protected:Npn \@@_function_index:x #1
  {
    \tl_set:Nx \l_@@_tmpa_tl
      { \g_@@_function_name_prefix_tl #1 }
    \seq_gput_right:No \g_doc_functions_seq { \l_@@_tmpa_tl }
    \@@_special_index:on { \l_@@_tmpa_tl } { usage }
  }
%    \end{macrocode}
% \end{macro}
%
% \begin{macro}[internal]{\@@_typeset_function_block:NN}
%   The first argument is a control sequence, the second a boolean which
%   controls whether to typeset |TF| or not.
%    \begin{macrocode}
\cs_new_protected:Npn \@@_typeset_function_block:NN #1#2
  {
    \@@_function_label:x
      { \g_@@_function_name_prefix_tl \cs_to_str:N #1 }
    \g_@@_function_name_prefix_tl \cs_to_str:N #1
    \bool_if:NT #2 { \@@_typeset_TF: }
    \@@_typeset_expandability:
    \clist_if_empty:NF \g_@@_variants_clist
      { \@@_typeset_variant_list:NN #1#2 }
    \\
  }
%    \end{macrocode}
%
%    \begin{macrocode}
\cs_new_protected_nopar:Npn \@@_typeset_expandability:
  {
    &
    \bool_if:NT \l_@@_macro_EXP_bool
      {
        \hspace{\tabcolsep}
        \hyperlink{expstar} {$\star$}
      }
    \bool_if:NT \l_@@_macro_rEXP_bool
      {
        \hspace{\tabcolsep}
        \hyperlink{rexpstar} {\ding{73}} % hollow star
      }
  }
%    \end{macrocode}
%
%    \begin{macrocode}
\cs_new_protected:Npn \@@_typeset_variant_list:NN #1#2
  {
    \\
    \@@_typeset_aux:n
      {
        \g_@@_function_name_prefix_tl
        \__cs_get_function_name:N #1
      }
    :
    \int_compare:nTF { \clist_count:N \g_@@_variants_clist == 1 }
      { \clist_use:Nn \g_@@_variants_clist { } }
      {
        \textrm(
          \clist_use:Nn \g_@@_variants_clist { \textrm| }
        \textrm)
      }
    \bool_if:NT #2 { \@@_typeset_TF: }
    \@@_typeset_expandability:
  }
%    \end{macrocode}
%
% TODO: understand whether we need one more boolean to only
% put labels if the whole document is typeset.
%    \begin{macrocode}
\cs_new_protected:Npn \@@_function_label:n #1
  {
    % \bool_if:NT \g_@@_typeset_implementation_bool
    %   {
    \bool_if:NF \l_@@_no_label_bool
      {
        \@@_get_hyper_target:nN {#1} \l_@@_tmpa_tl
        \exp_args:No \label { \l_@@_tmpa_tl }
      }
    %   }
  }
\cs_generate_variant:Nn \@@_function_label:n { x }
%    \end{macrocode}
% \end{macro}
%
% \begin{macro}[internal]{\@@_typeset_dates:}
%   To display metadata for when functions are added/modified.
%   This function must be expandable since it produces rules for use in
%   alignments.
%    \begin{macrocode}
\cs_new_nopar:Npn \@@_typeset_dates:
  {
    \bool_if:nF
      {
        \tl_if_empty_p:N \l_@@_date_added_tl &&
        \tl_if_empty_p:N \l_@@_date_updated_tl
      }
      { \midrule }
    \tl_if_empty:NF \l_@@_date_added_tl
      {
        \multicolumn { 2 } { @{} r @{} }
          { \scriptsize New: \, \l_@@_date_added_tl } \\
      }

    \tl_if_empty:NF \l_@@_date_updated_tl
      {
        \multicolumn { 2 } { @{} r @{} }
          { \scriptsize Updated: \, \l_@@_date_updated_tl } \\
      }
  }
%    \end{macrocode}
% \end{macro}
%
% \begin{macro}[aux]{\@@_syntax:w, \@@_syntax_end:}
%   Implement the \env{syntax} environment.
%    \begin{macrocode}
\dim_new:N \l_@@_syntax_dim
\cs_new_protected:Npn \@@_syntax:w
  {
    \dim_set:Nn \l_@@_syntax_dim
      {
        \textwidth
        \bool_if:NT \l_@@_long_name_bool
          { + 0.75 \marginparwidth - \l_@@_trial_width_dim }
      }
    \hbox_gset:Nw \g_@@_syntax_box
      \small \ttfamily
      \arrayrulecolor{white}
      \begin{tabular} { @{} l @{} }
        \toprule
        \begin{minipage}{\l_@@_syntax_dim}
          \raggedright
          \obeyspaces
          \obeylines
  }
\cs_new_protected:Npn \@@_syntax_end:
  {
        \end{minipage}
      \end{tabular}
      \arrayrulecolor{black}
    \hbox_gset_end:
  }
%    \end{macrocode}
% \end{macro}
%
% \subsubsection{The \env{macro} environment}
%
% Keyval for the \env{macro} environment.  TODO: use
% |.value_forbidden:|.  TODO: provide document command for talking about
% keys.
%    \begin{macrocode}
\keys_define:nn { l3doc/macro }
  {
    aux .code:n =
      { \bool_set_true:N \l_@@_macro_aux_bool } ,
    internal .code:n =
      { \bool_set_true:N \l_@@_macro_internal_bool } ,
    int .code:n =
      { \bool_set_true:N \l_@@_macro_internal_bool } ,
    var .code:n =
      { \bool_set_true:N \l_@@_macro_var_bool } ,
    TF .code:n =
      { \bool_set_true:N \l_@@_macro_TF_bool } ,
    pTF .code:n =
      {
        \bool_set_true:N \l_@@_macro_TF_bool
        \bool_set_true:N \l_@@_macro_pTF_bool
        \bool_set_true:N \l_@@_macro_EXP_bool
        \bool_set_false:N \l_@@_macro_rEXP_bool
      } ,
    EXP .code:n =
      {
        \bool_set_true:N \l_@@_macro_EXP_bool
        \bool_set_false:N \l_@@_macro_rEXP_bool
      } ,
    rEXP .code:n =
      {
        \bool_set_false:N \l_@@_macro_EXP_bool
        \bool_set_true:N \l_@@_macro_rEXP_bool
      } ,
    tested .code:n =
      {
        \bool_set_true:N \l_@@_macro_tested_bool
      } ,
    added .code:n = {} , % TODO
    updated .code:n = {} , % TODO
  }
%    \end{macrocode}
%
% \begin{macro}[int]{\@@_macro:nnw}
%   The arguments are a key--value list of \meta{options} and a
%   comma-list of \meta{names}, read verbatim by \pkg{xparse}.  First
%   initialize some variables before applying the \meta{options}, then
%   parse the \meta{names} to get a sequence of macro names, then apply
%   \cs{@@_macro_single:n} to each (this step is more subtle than
%   \cs{seq_map_function:NN} because of |TF|/|pTF|).  Finally typeset
%   the macro names in the margin.
%    \begin{macrocode}
\cs_new_protected:Npn \@@_macro:nnw #1#2
  {
    \@@_macro_init:
    \keys_set:nn { l3doc/macro } {#1}
    \@@_verb_get_seq:nN {#2} \l_@@_macro_input_seq
    \@@_macro_map:N \@@_macro_single:n
    \@@_macro_typeset:
  }
%    \end{macrocode}
% \end{macro}
%
% \begin{macro}[aux]{\@@_macro_init:}
%   The booleans hold various key--value options,
%   \cs{l_@@_nested_macro_int} counts the number of \env{macro}
%   environments around the current point (is $0$ outside).
%    \begin{macrocode}
\cs_new_protected_nopar:Npn \@@_macro_init:
  {
    \int_incr:N \l_@@_nested_macro_int
    \bool_set_false:N \l_@@_macro_aux_bool
    \bool_set_false:N \l_@@_macro_internal_bool
    \bool_set_false:N \l_@@_macro_TF_bool
    \bool_set_false:N \l_@@_macro_pTF_bool
    \bool_set_false:N \l_@@_macro_EXP_bool
    \bool_set_false:N \l_@@_macro_rEXP_bool
    \bool_set_false:N \l_@@_macro_var_bool
    \bool_set_false:N \l_@@_macro_tested_bool
    \cs_set_eq:NN \testfile \@@_print_testfile:n
    \box_clear:N \l_@@_macro_index_box
    \vbox_set:Nn \l_@@_macro_box
      {
        \hbox:n { \strut }
        \vskip \int_eval:n { \l_@@_macro_int - 1 } \baselineskip
      }
  }
%    \end{macrocode}
% \end{macro}
%
% \begin{macro}[aux]{\@@_macro_map:N}
%   Applies |#1| to all macros.  In the |pTF| case, this means going
%   twice through the sequence \cs{l_@@_macro_input_seq}, once for the
%   predicates and once for the |TF| variants.
%    \begin{macrocode}
\cs_new_protected:Npn \@@_macro_map:N #1
  {
    \bool_if:NT \l_@@_macro_pTF_bool
      {
        \bool_set_false:N \l_@@_macro_TF_bool
        \seq_map_inline:Nn \l_@@_macro_input_seq
          {
            \tl_set:Nn \l_@@_tmpa_tl {##1}
            \tl_replace_once:Noo \l_@@_tmpa_tl
              { \tl_to_str:n { : } } { \tl_to_str:n { _p: } }
            \exp_args:No #1 \l_@@_tmpa_tl
          }
        \bool_set_true:N \l_@@_macro_TF_bool
      }
    \seq_map_function:NN \l_@@_macro_input_seq #1
  }
%    \end{macrocode}
% \end{macro}
%
% \begin{macro}[aux]{\@@_macro_typeset:}
%   This calls |\makelabel{}|
%    \begin{macrocode}
\cs_new_protected_nopar:Npn \@@_macro_typeset:
  {
    \topsep\MacroTopsep
    \trivlist
    \cs_set:Npn \makelabel ##1
      {
        \llap
          {
            \hbox_unpack_clear:N \l_@@_macro_index_box
            \vtop to \baselineskip
              {
                \vbox_unpack_clear:N \l_@@_macro_box
                \vss
              }
          }
      }
    \item [ ]
  }
%    \end{macrocode}
% \end{macro}
%
% \begin{macro}{\@@_macro_single:n}
%   Let's start to mess around with \cls{doc}'s \env{macro} environment.
%   See \file{doc.dtx} for a full explanation of the original
%   environment.  It's rather \emph{enthusiastically} commented.
%   \begin{arguments}
%     \item Macro/function/whatever name; input has already been
%       sanitised.
%   \end{arguments}
%   The assignment to \cs{saved@macroname} is used by \pkg{doc}'s
%   \cs{changes} mechanism.
%    \begin{macrocode}
\cs_new_protected:Npn \@@_macro_single:n #1
  {
    \tl_set:Nn \saved@macroname {#1}

    \@@_macro_typeset_one:n {#1}
    \exp_args:Nx \@@_macro_index:n
      {
        #1
        \bool_if:NT \l_@@_macro_TF_bool { \tl_to_str:n { TF } }
      }
  }
\cs_new_protected:Npn \@@_macro_index:n #1
  {
    \bool_if:NF \l_@@_macro_aux_bool
      { \seq_gput_right:Nn \g_doc_macros_seq {#1} }
    \hbox_set:Nn \l_@@_macro_index_box
      {
        % This box only contains targets... it seems inefficient.
        \hbox_unpack_clear:N \l_@@_macro_index_box
        \int_gincr:N \c@CodelineNo
        \@@_special_index:nn {#1} { main }
        \DoNotIndex {#1}
        \int_gdecr:N \c@CodelineNo
      }
  }
%    \end{macrocode}
% \end{macro}
%
% \begin{macro}[aux]{\@@_macro_typeset_one:n}
%   For a long time, \cls{l3doc} collected the macro names as labels in
%   the first items of nested \tn{trivlist}, but these were not closed
%   properly with \tn{endtrivlist}.  Also, it interacted in surprising
%   ways with \pkg{hyperref} targets.  Now, we collect typeset macro
%   names by hand in the box \cs{l_@@_macro_box}.  Note the space |\ |.
%    \begin{macrocode}
\cs_new_protected:Npn \@@_macro_typeset_one:n #1
  {
    \vbox_set:Nn \l_@@_macro_box
      {
        \vbox_unpack_clear:N \l_@@_macro_box
        \hbox { \llap { \@@_print_macroname:n {#1} \ } }
      }
    \int_incr:N \l_@@_macro_int
  }
%    \end{macrocode}
% \end{macro}
%
% \begin{macro}{\@@_print_macroname:n}
%    \begin{macrocode}
\cs_new_protected:Npn \@@_print_macroname:n #1
  {
    \strut
    \HD@target

    % TODO: INEFFICIENT(!)
    \exp_args:NNx \seq_if_in:NnTF \g_doc_functions_seq
      { #1 \bool_if:NT \l_@@_macro_TF_bool { \tl_to_str:n {TF} } }
      {
        \@@_get_hyper_target:nN {#1} \l_@@_tmpa_tl
        \exp_last_unbraced:NNo \hyperref [ \l_@@_tmpa_tl ]
      }
      { \use:n }
      {
        \int_compare:nTF { \tl_count:n {#1} <= 28 }
          { \MacroFont } { \MacroLongFont }
        \@@_macroname_prefix:n {#1} \@@_macroname_suffix:
      }
  }
\cs_new_protected:Npn \@@_macroname_prefix:n #1
  {
    \bool_if:NTF \l_@@_macro_aux_bool
      { \@@_typeset_aux:n {#1} } {#1}
  }
\cs_new_protected_nopar:Npn \@@_macroname_suffix:
  { \bool_if:NTF \l_@@_macro_TF_bool { \@@_typeset_TF: } { } }
%    \end{macrocode}
% \end{macro}
%
% \begin{macro}{\MacroLongFont}
%    \begin{macrocode}
\providecommand \MacroLongFont
  {
    \fontfamily{lmtt}\fontseries{lc}\small
  }
%    \end{macrocode}
% \end{macro}
%
% \begin{macro}{\@@_print_testfile:n, \@@_print_testfile_aux:n}
%   Used to show that a macro has a test, somewhere.
%    \begin{macrocode}
\cs_new_protected:Npn \@@_print_testfile:n #1
  {
    \bool_set_true:N \l_@@_macro_tested_bool
    \tl_if_eq:nnF {#1} {*}
      {
        \seq_if_in:NnF \g_@@_testfiles_seq {#1}
          {
            \seq_gput_right:Nn \g_@@_testfiles_seq {#1}
            \par
            \@@_print_testfile_aux:n {#1}
          }
      }
  }
\cs_new_protected:Npn \@@_print_testfile_aux:n #1
  {
    \footnotesize
    (
    \textit
      {
        The~ test~ suite~ for~ this~ command,~
        and~ others~ in~ this~ file,~ is~ \textsf{#1}
      }.
    )\par
  }
%    \end{macrocode}
% \end{macro}
%
% \begin{macro}{\TestFiles}
%    \begin{macrocode}
\DeclareDocumentCommand \TestFiles {m}
  {
    \par
    \textit
      {
        The~ following~ test~ files~ are~
        used~ for~ this~ code:~ \textsf{#1}.
      }
    \par \ignorespaces
  }
%    \end{macrocode}
% \end{macro}
%
% \begin{macro}{\UnitTested}
%    \begin{macrocode}
\DeclareDocumentCommand \UnitTested { } { \testfile* }
%    \end{macrocode}
% \end{macro}
%
% \begin{macro}{\TestMissing}
%    \begin{macrocode}
\DeclareDocumentCommand \TestMissing { m }
  { \@@_test_missing:n {#1} }
%    \end{macrocode}
% \end{macro}
%
% \begin{macro}[aux]{\@@_test_missing:n}
%   Keys in \cs{g_@@_missing_tests_prop} are lists of macros given as
%   arguments of one \env{macro} environment.  Values are comma lists of
%   filenames (Bruno thinks).
%    \begin{macrocode}
\cs_new_protected:Npn \@@_test_missing:n #1
  {
    \tl_set:Nx \l_@@_tmpb_tl
      { \seq_use:Nn \l_@@_macro_input_seq { , } }
    \prop_if_in:NVTF \g_@@_missing_tests_prop \l_@@_tmpb_tl
      {
        \prop_get:NVN \g_@@_missing_tests_prop \l_@@_tmpb_tl
          \l_@@_tmpa_tl
      }
      { \tl_clear:N \l_@@_tmpa_tl }
    \clist_set:Nx \l_@@_tmpa_clist { \l_@@_tmpa_tl , #1 }
    \prop_gput:NVV \g_@@_missing_tests_prop \l_@@_tmpb_tl
      \l_@@_tmpa_clist
  }
%    \end{macrocode}
% \end{macro}
%
% \begin{macro}[aux]{\@@_macro_end:}
%   It is too late for anyone to declare a test file for this macro, so
%   we can check now whether the macro is tested.  If the \env{macro}
%   environment which is being ended is the outermost one, then wrap
%   each macro in \tn{texttt} (with the addition of |TF| if relevant)
%   and typeset two informations: that this ends the definition of some
%   macros, and that they are documented on some page.
%    \begin{macrocode}
\cs_new_protected:Npn \@@_macro_end:
  {
    \endtrivlist
    \@@_macro_end_check_tested:
    \int_compare:nNnT \l_@@_nested_macro_int = 1
      {
        \@@_macro_end_style:n
          {
            \@@_print_end_definition:
            \@@_print_documented:
          }
      }
  }
%    \end{macrocode}
% \end{macro}
%
% \begin{macro}[aux]{\@@_macro_end_check_tested:}
%   If the |checktest| option was issued and the macro is not an
%   auxiliary nor a variable (and it does not have a test), then add it
%   to the sequence of non-tested macros.
%    \begin{macrocode}
\cs_new_protected_nopar:Npn \@@_macro_end_check_tested:
  {
    \bool_if:nT
     {
       \g_@@_checktest_bool &&
       ! \l_@@_macro_aux_bool &&
       ! \l_@@_macro_var_bool &&
       ! \l_@@_macro_tested_bool
     }
     {
       \seq_gput_right:Nx \g_@@_not_tested_seq
         {
           \seq_use:Nn \l_@@_macro_input_seq { , }
           \bool_if:NTF \l_@@_macro_pTF_bool {~(pTF)}
             { \bool_if:NT \l_@@_macro_TF_bool {~(TF)} }
         }
     }
  }
%    \end{macrocode}
% \end{macro}
%
% \begin{macro}[aux]{\@@_macro_end_style:n}
%   Style for the extra information at the end of a top-level
%   \env{macro} environment.
%    \begin{macrocode}
\cs_new_protected:Npn \@@_macro_end_style:n #1
  {
    \nobreak \noindent
    { \footnotesize ( \emph{#1} ) \par }
  }
%    \end{macrocode}
% \end{macro}
%
% \begin{macro}[aux]
%   {
%     \@@_print_end_definition:,
%     \@@_macro_end_wrap_items:N,
%     \@@_print_documented:
%   }
%   Surround each item by \tn{texttt} and |TF| if needed, replacing |_|
%   by \tn{_} as well.  Then list the
%   macro names through \cs{seq_use:Nnnn}, unless there are too many.
%   Finally, if the macro is neither auxiliary nor internal, add a link
%   to where it is documented.
%    \begin{macrocode}
\cs_new_protected:Npn \@@_macro_end_wrap_items:N #1
  {
    \bool_if:NT \l_@@_macro_TF_bool
      { \seq_set_map:NNn #1 #1 { ##1 TF } }
    \seq_set_map:NNn #1 #1
      {
        \exp_not:n
          {
            \tl_set:Nn \l_@@_tmpa_tl {##1}
            \tl_replace_all:Non \l_@@_tmpa_tl
              { \token_to_str:N _ } { \_ }
            \texttt { \l_@@_tmpa_tl }
          }
      }
  }
\cs_new_protected_nopar:Npn \@@_print_end_definition:
  {
    \group_begin:
    \@@_macro_end_wrap_items:N \l_@@_macro_input_seq
    End~ definition~ for~
    \int_compare:nTF { \seq_count:N \l_@@_macro_input_seq <= 3 }
      {
        \seq_use:Nnnn \l_@@_macro_input_seq
          { \,~and~ } { \,,~ } { \,,~and~ }
        \@.
      }
      { \seq_item:Nn \l_@@_macro_input_seq {1}\,~and~others. }
    \group_end:
  }
\cs_new_protected_nopar:Npn \@@_print_documented:
  {
    \bool_if:nT
      {
        ! \l_@@_macro_aux_bool &&
        ! \l_@@_macro_internal_bool
      }
      {
        \int_set:Nn \l_@@_tmpa_int
          { \seq_count:N \l_@@_macro_input_seq }
        \int_compare:nNnTF \l_@@_tmpa_int = 1 {~This~} {~These~}
        \bool_if:NTF \l_@@_macro_var_bool {variable} {function}
        \int_compare:nNnTF \l_@@_tmpa_int = 1 {~is~} {s~are~}
        documented~on~page~
        \exp_args:Nx \@@_get_hyper_target:nN
          { \seq_item:Nn \l_@@_macro_input_seq { 1 } }
          \l_@@_tmpa_tl
        \exp_args:Nx \pageref { \l_@@_tmpa_tl } .
      }
  }
%    \end{macrocode}
% \end{macro}
%
% \subsubsection{Common between \env{macro} and \env{function}}
%
% \begin{macro}{\@@_typeset_TF:, \@@_typeset_aux:n}
%   Used by \cs{@@_macro_single:n} and in the \env{function} environment
%   to typeset conditionals and auxiliary functions.
%    \begin{macrocode}
\cs_new_protected_nopar:Npn \@@_typeset_TF:
  {
    \hyperlink{explTF}
      {
        \color{black}
        \itshape TF
        \makebox[0pt][r]
          {
            \color{red}
            \underline { \phantom{\itshape TF} \kern-0.1em }
          }
      }
  }
\cs_new_protected:Npn \@@_typeset_aux:n #1
  {
    { \color[gray]{0.7} #1 }
  }
%    \end{macrocode}
% \end{macro}
%
% \begin{macro}{\@@_get_hyper_target:nN}
%   Create a \pkg{hyperref} anchor from a macro name~|#1| and stores it
%   in the token list variable~|#2|.  For instance, |\prg_replicate:nn|
%   gives |doc/function//prg/replicate:nn|.
%    \begin{macrocode}
\cs_new_protected:Npn \@@_get_hyper_target:nN #1#2
  {
    \tl_set:Nx #2
      {
        \tl_to_str:n {#1}
        \bool_if:NT \l_@@_macro_TF_bool { \tl_to_str:n {TF} }
      }
    \tl_replace_all:Nxn #2 { \iow_char:N \_ } { / }
    \tl_remove_all:Nx   #2 { \iow_char:N \\ }
    \tl_put_left:Nn #2 { doc/function// }
  }
%    \end{macrocode}
% \end{macro}
%
% \subsubsection{Misc}
%
% \begin{macro}{\DescribeOption}
%   For describing package options.  Due to Joseph Wright.  Name/usage
%   might change soon.
%    \begin{macrocode}
\newcommand*{\DescribeOption}
  {
    \leavevmode
    \@bsphack
    \begingroup
      \MakePrivateLetters
      \Describe@Option
  }
%    \end{macrocode}
%
%    \begin{macrocode}
\newcommand*{\Describe@Option}[1]
  {
    \endgroup
    \marginpar{
      \raggedleft
      \PrintDescribeEnv{#1}
    }
    \SpecialOptionIndex{#1}
    \@esphack
    \ignorespaces
  }
%    \end{macrocode}
%
%    \begin{macrocode}
\newcommand*{\SpecialOptionIndex}[1]
  {
    \@bsphack
    \begingroup
      \HD@target
      \let\HDorg@encapchar\encapchar
      \edef\encapchar usage
        {
          \HDorg@encapchar hdclindex{\the\c@HD@hypercount}{usage}
        }
      \index
        {
          #1\actualchar{\protect\ttfamily#1}~(option)
          \encapchar usage
        }
      \index
        {
          options:\levelchar#1\actualchar{\protect\ttfamily#1}
          \encapchar usage
        }
    \endgroup
    \@esphack
  }
%    \end{macrocode}
% \end{macro}
%
% Here are some definitions for additional markup that will help to
% structure your documentation.
%
% \begin{environment}{danger}
% \begin{environment}{ddanger}
%   \begin{syntax}
%     |\begin{[d]danger}|\\
%       dangerous code\\
%     |\end{[d]danger}|
%   \end{syntax}
%   \begin{danger}
%     Provides a danger bend, as known from the \TeX{}book.
%   \end{danger}
%   The actual character from the font |manfnt|:
%    \begin{macrocode}
\font \manual = manfnt \scan_stop:
\cs_gset_nopar:Npn \dbend { {\manual\char127} }
%    \end{macrocode}
%
% Defines the single danger bend. Use it whenever there is a feature in
% your package that might be tricky to use.  FIXME: Has to be fixed when
% in combination with a macro-definition.
%    \begin{macrocode}
\newenvironment {danger}
  {
    \begin{trivlist}\item[]\noindent
    \begingroup\hangindent=2pc\hangafter=-2
    \cs_set_nopar:Npn \par{\endgraf\endgroup}
    \hbox to0pt{\hskip-\hangindent\dbend\hfill}\ignorespaces
  }
  {
    \par\end{trivlist}
  }
%    \end{macrocode}
%
% \begin{ddanger}
%   Use the double danger bend if there is something which could cause
%   serious problems when used in a wrong way. Better the normal user
%   does not know about such things.
% \end{ddanger}
%    \begin{macrocode}
\newenvironment {ddanger}
  {
    \begin{trivlist}\item[]\noindent
    \begingroup\hangindent=3.5pc\hangafter=-2
    \cs_set_nopar:Npn \par{\endgraf\endgroup}
    \hbox to0pt{\hskip-\hangindent\dbend\kern2pt\dbend\hfill}\ignorespaces
  }{
      \par\end{trivlist}
  }
%    \end{macrocode}
% \end{environment}
% \end{environment}
%
% \subsection{Documenting templates}
%
%    \begin{macrocode}
\newenvironment{TemplateInterfaceDescription}[1]
  {
    \subsection{The~object~type~`#1'}
    \begingroup
    \@beginparpenalty\@M
    \description
    \def\TemplateArgument##1##2{\item[Arg:~##1]##2\par}
    \def\TemplateSemantics
      {
        \enddescription\endgroup
        \subsubsection*{Semantics:}
      }
  }
  {
    \par\bigskip
  }
%    \end{macrocode}
%
%    \begin{macrocode}
\newenvironment{TemplateDescription}[2]
  {
    \subsection{The~template~`#2'~(object~type~#1)}
    \subsubsection*{Attributes:}
    \begingroup
    \@beginparpenalty\@M
    \description
    \def\TemplateKey##1##2##3##4
      {
        \item[##1~(##2)]##3%
        \ifx\TemplateKey##4\TemplateKey\else
%         \hskip0ptplus3em\penalty-500\hskip 0pt plus 1filll Default:~##4%
          \hfill\penalty500\hbox{}\hfill Default:~##4%
          \nobreak\hskip-\parfillskip\hskip0pt\relax
        \fi
        \par
      }
    \def\TemplateSemantics
      {
        \enddescription\endgroup
        \subsubsection*{Semantics~\&~Comments:}
      }
  }
  { \par \bigskip }
%    \end{macrocode}
%
%    \begin{macrocode}
\newenvironment{InstanceDescription}[4][xxxxxxxxxxxxxxx]
  {
    \subsubsection{The~instance~`#3'~(template~#2/#4)}
    \subsubsection*{Attribute~values:}
    \begingroup
    \@beginparpenalty\@M
    \def\InstanceKey##1##2{\>\textbf{##1}\>##2\\}
    \def\InstanceSemantics{\endtabbing\endgroup
      \vskip-30pt\vskip0pt
      \subsubsection*{Layout~description~\&~Comments:}}
    \tabbing
    xxxx\=#1\=\kill
  }
  { \par \bigskip }
%    \end{macrocode}
%
% \subsection{Inheriting doc}
%
% Code here is taken from \pkg{doc}, stripped of comments and translated
% into \pkg{expl3} syntax. New features are added in various places.
%
% \begin{macro}
%   {\StopEventually, \Finale, \AlsoImplementation, \OnlyDescription}
% \begin{variable}[aux]{\g_@@_finale_tl}
%   TODO: remove these four commands altogether, document that it is
%   better to use the \env{documentation} and \env{implementation}
%   environments.
%    \begin{macrocode}
\DeclareDocumentCommand \OnlyDescription { }
  { \bool_gset_false:N \g_@@_typeset_implementation_bool }
\DeclareDocumentCommand \AlsoImplementation { }
  { \bool_gset_true:N \g_@@_typeset_implementation_bool }
\DeclareDocumentCommand \StopEventually { m }
  {
    \bool_if:NTF \g_@@_typeset_implementation_bool
      {
        \@bsphack
        \tl_gset:Nn \g_@@_finale_tl { #1 \check@checksum }
        \init@checksum
        \@esphack
      }
      { #1 \endinput }
  }
\DeclareDocumentCommand \Finale { }
  { \tl_use:N \g_@@_finale_tl }
\tl_new:N \g_@@_finale_tl
%    \end{macrocode}
% \end{variable}
% \end{macro}
%
% \begin{macro}[aux]{\@@_input:n}
%   Inputting a file, with some setup: the module name should be empty
%   before the first |<@@=|\meta{module}|>| line in the file.
%    \begin{macrocode}
\cs_new_protected:Npn \@@_input:n #1
  {
    \tl_gclear:N \g_@@_module_name_tl
    \MakePercentIgnore
    \input{#1}
    \MakePercentComment
  }
%    \end{macrocode}
% \end{macro}
%
% \begin{macro}{\DocInput}
%   Modified from \pkg{doc} to accept comma-list input (who has commas
%   in filenames?).
%    \begin{macrocode}
\DeclareDocumentCommand \DocInput { m }
  {
    \clist_map_inline:nn {#1}
      {
        \clist_put_right:Nn \g_docinput_clist {##1}
        \@@_input:n {##1}
      }
  }
%    \end{macrocode}
% \end{macro}
%
% \begin{macro}{\DocInputAgain}
%   Uses \cs{g_docinput_clist} to re-input whatever's already been
%   \tn{DocInput}-ed until now.  May be used multiple times.
%    \begin{macrocode}
\DeclareDocumentCommand \DocInputAgain { }
  { \clist_map_function:NN \g_docinput_clist \@@_input:n }
%    \end{macrocode}
% \end{macro}
%
% \begin{macro}{\DocInclude}
%   More or less exactly the same as \tn{include}, but uses
%   \tn{DocInput} on a \file{.dtx} file, not \tn{input} on a \file{.tex}
%   file.
%
%    \begin{macrocode}
\NewDocumentCommand \DocInclude { m }
  {
    \relax\clearpage
    \docincludeaux
    \IfFileExists{#1.fdd}
      { \cs_set_nopar:Npn \currentfile{#1.fdd} }
      { \cs_set_nopar:Npn \currentfile{#1.dtx} }
    \int_compare:nNnTF \@auxout = \@partaux
      { \@latexerr{\string\include\space cannot~be~nested}\@eha }
      { \@docinclude #1 }
  }
%    \end{macrocode}
%
%    \begin{macrocode}
\cs_gset:Npn \@docinclude #1
  {
    \clearpage
    \immediate\write\@mainaux{\string\@input{#1.aux}}
    \@tempswatrue
    \if@partsw
      \@tempswafalse
      \cs_set_nopar:Npx \@tempb{#1}
      \@for\@tempa:=\@partlist\do
        {
          \ifx\@tempa\@tempb\@tempswatrue\fi
        }
    \fi
    \if@tempswa
      \cs_set_eq:NN \@auxout                 \@partaux
      \immediate\openout\@partaux #1.aux
      \immediate\write\@partaux{\relax}
      \cs_set_eq:NN \@ltxdoc@PrintIndex      \PrintIndex
      \cs_set_eq:NN \PrintIndex              \relax
      \cs_set_eq:NN \@ltxdoc@PrintChanges    \PrintChanges
      \cs_set_eq:NN \PrintChanges            \relax
      \cs_set_eq:NN \@ltxdoc@theglossary     \theglossary
      \cs_set_eq:NN \@ltxdoc@endtheglossary  \endtheglossary
      \part{\currentfile}
      {
        \cs_set_eq:NN \ttfamily\relax
        \cs_gset_nopar:Npx \filekey
          { \filekey, \thepart = { \ttfamily \currentfile } }
      }
      \DocInput{\currentfile}
      \cs_set_eq:NN \PrintIndex              \@ltxdoc@PrintIndex
      \cs_set_eq:NN \PrintChanges            \@ltxdoc@PrintChanges
      \cs_set_eq:NN \theglossary             \@ltxdoc@theglossary
      \cs_set_eq:NN \endtheglossary          \@ltxdoc@endtheglossary
      \clearpage
      \@writeckpt{#1}
      \immediate \closeout \@partaux
    \else
      \@nameuse{cp@#1}
    \fi
    \cs_set_eq:NN \@auxout \@mainaux
  }
%    \end{macrocode}
%
%    \begin{macrocode}
\cs_gset:Npn \codeline@wrindex #1
  {
    \immediate\write\@indexfile
      {
        \string\indexentry{#1}
          { \filesep \int_use:N \c@CodelineNo }
      }
  }
\tl_gclear:N \filesep
%    \end{macrocode}
% \end{macro}
%
% \begin{macro}{\docincludeaux}
%    \begin{macrocode}
\cs_gset_nopar:Npn \docincludeaux
  {
    \tl_set:Nn \thepart { \alphalph { part } }
    \tl_set:Nn \filesep { \thepart - }
    \cs_set_eq:NN \filekey \use_none:n
    \tl_gput_right:Nn \index@prologue
      {
        \cs_gset_nopar:Npn \@oddfoot
          {
            \parbox { \textwidth }
              {
                \strut \footnotesize
                \raggedright { \bfseries File~Key: } ~ \filekey
              }
          }
        \cs_set_eq:NN \@evenfoot \@oddfoot
      }
    \cs_gset_eq:NN \docincludeaux \relax
    \cs_gset_nopar:Npn \@oddfoot
      {
        \cs_if_exist:cTF { ver @ \currentfile }
          { File~\thepart :~{\ttfamily\currentfile}~ }
          {
            \GetFileInfo{\currentfile}
            File~\thepart :~{\ttfamily\filename}~
            Date:~\ExplFileDate\ % space
            Version~\ExplFileVersion
          }
        \hfill \thepage
      }
    \cs_set_eq:NN \@evenfoot \@oddfoot
  }
%    \end{macrocode}
% \end{macro}
%
% \subsubsection{The \env{macrocode} environment}
%
% \begin{macro}[aux]{\xmacro@code, \@@_xmacro_code:n, \@@_xmacro_code:w}
%   Hook into the \texttt{macrocode} environment in a dirty way:
%   \tn{xmacro@code} is responsible for grabbing (and tokenizing) the
%   body of the environment.  Redefine it to pass what it grabs to
%   \cs{@@_xmacro_code:n}.  This new macro replaces all |@@| by the
%   appropriate module name.  One exceptional case is the
%   |<@@=|\meta{module}|>| lines themselves, where |@@| should not be
%   modified.  Actually, we search for such lines, to set the module
%   name automatically.  We need to be careful: no |<@@=| should appear
%   as such in the code below since \pkg{l3doc} is also typeset using
%   this code.
%   TODO: right now, in a line containing |<@@=|\meta{module}|>|, the
%   |@@| are replaced (using different values of the \meta{module}
%   before and after the assignment).  Is this a waste?
%    \begin{macrocode}
\group_begin:
  \char_set_catcode_escape:N \/
  \char_set_catcode_other:N \^^A
  \char_set_catcode_active:N \^^S
  \char_set_catcode_active:N \^^B
  \char_set_catcode_other:N \^^L
  \char_set_catcode_other:N \^^R
  \char_set_lccode:nn { `\^^A } { `\% }
  \char_set_lccode:nn { `\^^S } { `\  }
  \char_set_lccode:nn { `\^^B } { `\\ }
  \char_set_lccode:nn { `\^^L } { `\{ }
  \char_set_lccode:nn { `\^^R } { `\} }
  \tl_to_lowercase:n
    {
      \group_end:
      \cs_set_protected:Npn \xmacro@code
          #1 ^^A ^^S^^S^^S^^S ^^Bend ^^Lmacrocode^^R
        { \@@_xmacro_code:n {#1} /end{macrocode} }
    }
\group_begin:
  \char_set_catcode_active:N \<
  \char_set_catcode_active:N \>
  \cs_new_protected:Npn \@@_xmacro_code:n #1
    {
      \tl_if_in:nnTF {#1} { < @ @ = }
        { \@@_xmacro_code:w #1 \q_stop }
        {
          \tl_set:Nn \l_@@_tmpa_tl {#1}
          \@@_replace_at_at:N \l_@@_tmpa_tl
          \tl_use:N \l_@@_tmpa_tl
        }
    }
  \cs_new_protected:Npn \@@_xmacro_code:w #1 < @ @ = #2 > #3 \q_stop
    {
      \tl_set:Nn \l_@@_tmpa_tl {#1}
      \@@_replace_at_at:N \l_@@_tmpa_tl

      \tl_gset:Nn \g_@@_module_name_tl {#2}
      \tl_put_right:Nn \l_@@_tmpa_tl { < @ @ = #2 > }

      \tl_set:Nn \l_@@_tmpb_tl {#3}
      \@@_replace_at_at:N \l_@@_tmpb_tl
      \tl_put_right:No \l_@@_tmpa_tl { \l_@@_tmpb_tl }

      \tl_use:N \l_@@_tmpa_tl
    }
\group_end:
%    \end{macrocode}
% \end{macro}
%
% \subsection{At end document}
%
% Print all defined and documented macros/functions.
%
%    \begin{macrocode}
\iow_new:N \g_@@_func_iow
%    \end{macrocode}
%
%    \begin{macrocode}
\tl_new:N \l_@@_doc_def_tl
\tl_new:N \l_@@_doc_undef_tl
\tl_new:N \l_@@_undoc_def_tl
%    \end{macrocode}
%
%    \begin{macrocode}
\cs_new_protected_nopar:Npn \@@_show_functions_defined:
  {
    \bool_if:nT
      { \g_@@_typeset_implementation_bool && \g_@@_checkfunc_bool }
      {
        \iow_term:x { \c_@@_iow_separator_tl \iow_newline: }
        \iow_open:Nn \g_@@_func_iow { \c_job_name_tl .cmds }

        \tl_clear:N \l_@@_doc_def_tl
        \tl_clear:N \l_@@_doc_undef_tl
        \tl_clear:N \l_@@_undoc_def_tl
        \seq_map_inline:Nn \g_doc_functions_seq
          {
            \seq_if_in:NnTF \g_doc_macros_seq {##1}
              {
                \tl_put_right:Nx \l_@@_doc_def_tl
                  { ##1 \iow_newline: }
                \iow_now:Nn \g_@@_func_iow { > ~ ##1 }
              }
              {
                \tl_put_right:Nx \l_@@_doc_undef_tl
                  { ##1 \iow_newline: }
                \iow_now:Nn \g_@@_func_iow { ! ~ ##1 }
              }
          }
        \seq_map_inline:Nn \g_doc_macros_seq
          {
            \seq_if_in:NnF \g_doc_functions_seq {##1}
              {
                \tl_put_right:Nx \l_@@_undoc_def_tl
                  { ##1 \iow_newline: }
                \iow_now:Nn \g_@@_func_iow { ? ~ ##1 }
              }
          }
        \@@_functions_typeout:nN
          {
            Functions~both~documented~and~defined: \iow_newline:
            (In~order~of~being~documented)
          }
          \l_@@_doc_def_tl
        \@@_functions_typeout:nN
          { Functions~documented~but~not~defined: }
          \l_@@_doc_undef_tl
        \@@_functions_typeout:nN
          { Functions~defined~but~not~documented: }
          \l_@@_undoc_def_tl

        \iow_close:N \g_@@_func_iow
        \iow_term:x { \c_@@_iow_separator_tl }
      }
  }
\AtEndDocument { \@@_show_functions_defined: }
%    \end{macrocode}
%
% TODO: use \cs{iow_term:x}.
%    \begin{macrocode}
\cs_new_protected:Npn \@@_functions_typeout:nN #1#2
  {
    \tl_if_empty:NF #2
      {
        \typeout
          {
            \c_@@_iow_midrule_tl \iow_newline:
            #1 \iow_newline:
            \c_@@_iow_midrule_tl \iow_newline:
            #2
          }
        \tl_clear:N #2
      }
  }
%    \end{macrocode}
%
%    \begin{macrocode}
\cs_new_protected_nopar:Npn \@@_show_not_tested:
  {
    \bool_if:NT \g_@@_checktest_bool
      {
        \tl_clear:N \l_@@_tmpa_tl
        \prop_if_empty:NF \g_@@_missing_tests_prop
          {
            \cs_set:Npn \@@_tmpa:w ##1##2
              {
                \iow_newline:
                \space\space\space\space \exp_not:n {##1}
                \clist_map_function:nN {##2} \@@_tmpb:w
              }
            \cs_set:Npn \@@_tmpb:w ##1
              {
                \iow_newline:
                \space\space\space\space\space\space * ~ ##1
              }
            \tl_put_right:Nx \l_@@_tmpa_tl
              {
                \iow_newline: \iow_newline:
                The~ following~ macro(s)~ have~ incomplete~ tests:
                \iow_newline:
                \prop_map_function:NN
                  \g_@@_missing_tests_prop \@@_tmpa:w
              }
          }
        \seq_if_empty:NF \g_@@_not_tested_seq
          {
            \cs_set:Npn \@@_tmpa:w ##1
              { \clist_map_function:nN {##1} \@@_tmpb:w }
            \cs_set:Npn \@@_tmpb:w ##1
              {
                \iow_newline:
                \space\space\space\space ##1
              }
            \tl_put_right:Nx \l_@@_tmpa_tl
              {
                \iow_newline:
                \iow_newline:
                The~ following~ macro(s)~ do~ not~ have~ any~ tests:
                \iow_newline:
                \seq_map_function:NN
                  \g_@@_not_tested_seq \@@_tmpa:w
              }
          }
        \tl_if_empty:NF \l_@@_tmpa_tl
          {
            \int_set:Nn \l_@@_tmpa_int { \etex_interactionmode:D }
            \errorstopmode
            \ClassError { l3doc } { \l_@@_tmpa_tl } { }
            \int_set:Nn \etex_interactionmode:D { \l_@@_tmpa_int }
          }
      }
  }
\AtEndDocument { \@@_show_not_tested: }
%    \end{macrocode}
%
% \subsection{Indexing}
%
% \subsubsection{Userspace commands}
%
% Fix index (for now):
%    \begin{macrocode}
\g@addto@macro \theindex { \MakePrivateLetters }
\cs_gset:Npn \verbatimchar {&}
%    \end{macrocode}
%
%    \begin{macrocode}
\setcounter { IndexColumns } { 2 }
%    \end{macrocode}
%
% Set up the Index to use \tn{part}
%    \begin{macrocode}
\IndexPrologue
  {
    \part*{Index}
    \markboth{Index}{Index}
    \addcontentsline{toc}{part}{Index}
    The~italic~numbers~denote~the~pages~where~the~
    corresponding~entry~is~described,~
    numbers~underlined~point~to~the~definition,~
    all~others~indicate~the~places~where~it~is~used.
  }
%    \end{macrocode}
%
% \begin{macro}{\SpecialIndex}
%   An attempt at affecting how commands which appear within the
%   \env{macrocode} environment are treated in the index.
%    \begin{macrocode}
\cs_gset_protected:Npn \SpecialIndex #1
  {
    \@bsphack
    \@@_special_index:nn {#1} { }
    \@esphack
  }
%    \end{macrocode}
% \end{macro}
%
%    \begin{macrocode}
\msg_new:nnn { l3doc } { print-index-howto }
  {
    Generate~the~index~by~executing\\
    \iow_indent:n
      { makeindex~-s~gind.ist~-o~\c_job_name_tl.ind~\c_job_name_tl.idx }
  }
\tl_gput_right:Nn \PrintIndex
  { \AtEndDocument { \msg_info:nn { l3doc } { print-index-howto } } }
%    \end{macrocode}
%
% \subsubsection{Internal index commands}
%
% \begin{macro}[int]{\it@is@a}
%   The index of one-character commands within the \env{macrocode}
%   environment is produced using \tn{it@is@a} \meta{char}.  Alter that
%   command.
%    \begin{macrocode}
\cs_gset_protected:Npn \it@is@a #1
  {
    \use:x
      {
        \@@_special_index_module:nnnn
          { \quotechar #1 }
          { \quotechar \bslash \quotechar #1 }
          { }
          { }
      }
  }
%    \end{macrocode}
% \end{macro}
%
% \begin{macro}[aux]{\@@_special_index:nn}
%    \begin{macrocode}
\cs_new_protected:Npn \@@_special_index:nn #1#2
  {
    \@@_key_get:n {#1}
    \@@_special_index_module:ooon
      { \l_@@_index_key_tl }
      { \l_@@_index_macro_tl }
      { \l_@@_index_module_tl }
      {#2}
  }
\cs_generate_variant:Nn \@@_special_index:nn { o }
%    \end{macrocode}
% \end{macro}
%
% \begin{macro}
%   {
%     \@@_special_index_module:nnnn,
%     \@@_special_index_module:ooon,
%     \@@_special_index_aux:nnnnn,
%     \@@_special_index_aux:onnnn,
%     \@@_special_index_set:Nn,
%   }
%   Remotely based on Heiko's replacement to play nicely with
%   \pkg{hypdoc}.  We use \tn{verb} or a \tn{verbatim@font} construction
%   depending on whether the number of tokens in |#2| is equal to its
%   number of characters.
%    \begin{macrocode}
\tl_new:N \l_@@_index_escaped_macro_tl
\cs_new_protected:Npn \@@_special_index_module:nnnn #1#2#3#4
  {
    \use:x
      {
        \exp_not:n { \@@_special_index_aux:nnnnn {#1} {#2} }
          \str_case_x:nnF {#3}
            {
              { } { { } { } }
              { TeX }
                {
                  { TeX~and~LaTeX2e~commands }
                  { \string\TeX{}~and~\string\LaTeXe{}~commands: }
                }
            }
            { { #3~commands } { \string\pkg{#3}~commands: } }
      }
          {#4}
  }
\cs_generate_variant:Nn \@@_special_index_module:nnnn { ooo }
\cs_new_protected:Npn \@@_special_index_aux:nnnnn #1#2#3#4#5
  {
    \HD@target
    \@@_special_index_set:Nn \l_@@_index_escaped_macro_tl {#2}
    \index
      {
        \tl_if_empty:nF { #3 #4 }
          { #3 \actualchar #4 \levelchar }
        #1
        \actualchar
        {
          \token_to_str:N \verbatim@font \c_space_tl
          \l_@@_index_escaped_macro_tl
        }
        \encapchar
        hdclindex{\the\c@HD@hypercount}{#5}
      }
  }
\cs_generate_variant:Nn \@@_special_index_aux:nnnnn { o }
\cs_new_protected:Npn \@@_special_index_set:Nn #1#2
  {
    \tl_set:Nx #1 { \tl_to_str:n {#2} }
    \int_compare:nNnTF { \tl_count:n {#2} } < { \tl_count:N #1 }
      {
        \tl_set:Nn #1 {#2}
        \tl_replace_all:Non #1
          { \c_@@_backslash_tl }
          { \token_to_str:N \bslash \c_space_tl }
      }
      {
        \exp_args:Nx \tl_map_inline:nn
          { \tl_to_str:N \verbatimchar \token_to_str:N _ }
          {
            \tl_replace_all:Nnn #1 {##1}
              {
                \verbatimchar \c_@@_backslash_tl ##1
                \token_to_str:N \verb \quotechar * \verbatimchar
              }
          }
        \tl_map_inline:nn { \actualchar \encapchar \levelchar }
          {
            \tl_replace_all:Nxn #1
              { \tl_to_str:N ##1 } { \quotechar \tl_to_str:N ##1 }
          }
        \tl_set:Nx #1
          {
            \token_to_str:N \verb \quotechar * \verbatimchar
            #1 \verbatimchar
          }
      }
  }
%    \end{macrocode}
% \end{macro}
%
% \subsubsection{Finding sort-key and module}
%
% \begin{macro}[aux]{\@@_key_get:n}
%   Sets \cs{l_@@_index_macro_tl}, \cs{l_@@_index_key_tl}, and
%   \cs{l_@@_index_module_tl} from |#1|.  The \cs{l_@@_index_macro_tl}
%   contains~|#1| directly.  The starting point for the \meta{key} is
%   |#1| as a string.  If it the first character is a backslash, remove
%   it.  Then recognize |expl| functions and variables by the presence
%   of |:| or~|_| and \TeX{}/\LaTeXe{} commands by the presence of~|@|.
%   For |expl| names, we call \cs{@@_key_expl:}, which is responsible
%   for removing some characters and finding the module name, while for
%   \TeX{}/\LaTeXe{} commands the module name is |TeX|, and others have
%   an empty module name.  The function \cs{@@_key_expl:} detects a
%   leading~|.| and removes it (key property), and otherwise does the
%   following: if the second character is~|_| and the first is as well,
%   then remove both (internal |expl| function); if the second character
%   is~|_| but the first is not (the case of an |expl| variable), then
%   remove both and remove a subsequent~|_| character (for internal
%   variables).  In all cases, call \cs{@@_key_get_module:}, which sets
%   \cs{l_@@_index_module_tl} by removing anything after |_| or~|:| in
%   \cs{l_@@_index_key_tl}.
%    \begin{macrocode}
\cs_new_protected:Npn \@@_key_get:n #1
  {
    \tl_set:Nn \l_@@_index_macro_tl {#1}
    \tl_set:Nx \l_@@_index_key_tl { \tl_to_str:n {#1} }
    \tl_clear:N \l_@@_index_module_tl
    \tl_if_head_eq_charcode:oNT
      { \l_@@_index_key_tl } \c_@@_backslash_token
      { \@@_key_pop: }
    \tl_if_in:NoTF \l_@@_index_key_tl { \token_to_str:N : }
      { \@@_key_expl: }
      {
        \tl_if_in:NoTF \l_@@_index_key_tl { \token_to_str:N _ }
          { \@@_key_expl: }
          {
            \tl_if_in:NoT \l_@@_index_key_tl { \token_to_str:N @ }
              { \tl_set:Nn \l_@@_index_module_tl { TeX } }
          }
      }
  }
\cs_new_protected_nopar:Npn \@@_key_pop:
  {
    \tl_set:Nx \l_@@_index_key_tl
      { \tl_tail:N \l_@@_index_key_tl }
  }
\cs_new_protected_nopar:Npn \@@_key_expl:
  {
    \tl_if_head_eq_charcode:oNTF { \l_@@_index_key_tl } .
      { \@@_key_pop: }
      {
        \exp_args:Nx \tl_if_head_eq_charcode:nNT
          { \exp_args:No \str_tail:n \l_@@_index_key_tl } _
          {
            \tl_if_head_eq_charcode:oNTF { \l_@@_index_key_tl } _
              {
                \@@_key_pop:
                \@@_key_pop:
              }
              {
                \@@_key_pop:
                \@@_key_pop:
                \tl_if_head_eq_charcode:oNT { \l_@@_index_key_tl } _
                  { \@@_key_pop: }
              }
          }
      }
    \@@_key_get_module:
  }
%    \end{macrocode}
% \end{macro}
%
% \begin{macro}[aux]{\@@_key_get_module:, \@@_key_get_module_aux:n}
%   Sets \cs{l_@@_index_module_tl} from \cs{l_@@_index_key_tl}, by
%   removing anything after |:| or |_|, taking care of not expanding the
%   rest for cases such as |\cs{\meta{name}:\meta{signature}}|.
%    \begin{macrocode}
\cs_new_protected_nopar:Npn \@@_key_get_module:
  {
    \tl_set_eq:NN \l_@@_index_module_tl \l_@@_index_key_tl
    \exp_args:No \@@_key_get_module_aux:n { \token_to_str:N : }
    \exp_args:No \@@_key_get_module_aux:n { \token_to_str:N _ }
  }
\cs_new_protected:Npn \@@_key_get_module_aux:n #1
  {
    \cs_set:Npn \@@_tmpa:w ##1 #1 ##2 \q_stop { \exp_not:n {##1} }
    \tl_set:Nx \l_@@_index_module_tl
      { \exp_after:wN \@@_tmpa:w \l_@@_index_module_tl #1 \q_stop }
  }
%    \end{macrocode}
% \end{macro}
%
% \subsection{Change history}
%
% Set the change history to use \tn{part}.
% Allow control names to be hyphenated in here\dots
%    \begin{macrocode}
\GlossaryPrologue
  {
    \part*{Change~History}
    {\GlossaryParms\ttfamily\hyphenchar\font=`\-}
    \markboth{Change~History}{Change~History}
    \addcontentsline{toc}{part}{Change~History}
  }
%    \end{macrocode}
%
%    \begin{macrocode}
\msg_new:nnn { l3doc } { print-changes-howto }
  {
    Generate~the~change~list~by~executing\\
    \iow_indent:n
      { makeindex~-s~gglo.ist~-o~\c_job_name_tl.gls~\c_job_name_tl.glo }
  }
\tl_gput_right:Nn \PrintChanges
  { \AtEndDocument { \msg_info:nn { l3doc } { print-changes-howto } } }
%    \end{macrocode}
%
%^^A The standard \changes command modified slightly to better cope with
%^^A this multiple file document.
%^^A\def\changes@#1#2#3{%
%^^A  \let\protect\@unexpandable@protect
%^^A  \edef\@tempa{\noexpand\glossary{#2\space\currentfile\space#1\levelchar
%^^A                                 \ifx\saved@macroname\@empty
%^^A                                   \space
%^^A                                   \actualchar
%^^A                                   \generalname
%^^A                                 \else
%^^A                                   \expandafter\@gobble
%^^A                                   \saved@macroname
%^^A                                   \actualchar
%^^A                                   \string\verb\quotechar*%
%^^A                                   \verbatimchar\saved@macroname
%^^A                                   \verbatimchar
%^^A                                 \fi
%^^A                                 :\levelchar #3}}%
%^^A  \@tempa\endgroup\@esphack}
%
% \subsection{Default configuration}
%
%    \begin{macrocode}
\bool_if:NTF \g_@@_typeset_implementation_bool
  {
    \RecordChanges
    \CodelineIndex
    \EnableCrossrefs
    \AlsoImplementation
  }
  {
    \CodelineNumbered
    \DisableCrossrefs
    \OnlyDescription
  }
%    \end{macrocode}
%
%
%    \begin{macrocode}
%</class>
%    \end{macrocode}
%
% \subsection{Internal macros for \LaTeX3 sources}
%
% These definitions are only used by the \LaTeX3 documentation; they are
% not necessary for third-party users of \cls{l3doc}.  In time this will
% be broken into a separate package that is specifically loaded in the
% various \pkg{expl3} modules, \emph{etc.}
%
%    \begin{macrocode}
%<*cfg>
%    \end{macrocode}
%
% The Guilty Parties.
%    \begin{macrocode}
\tl_const:Nn \Team
  {
    The~\LaTeX3~Project\thanks
      {
       Frank~Mittelbach,~Denys~Duchier,~Chris~Rowley,~
       Rainer~Sch\"opf,~Johannes~Braams,~Michael~Downes,~
       David~Carlisle,~Alan~Jeffrey,~Morten~H\o{}gholm,~Thomas~Lotze,~
       Javier~Bezos,~Will~Robertson,~Joseph~Wright,~Bruno~Le~Floch
      }
  }
%    \end{macrocode}
%
%    \begin{macrocode}
\NewDocumentCommand{\ExplMakeTitle}{mm}
  {
    \title
      {
       The~\pkg{#1}~package \\ #2
       \thanks
        {
         This~file~describes~v\ExplFileVersion,~
         last~revised~\ExplFileDate.
        }
      }
    \author
      {
       The~\LaTeX3~Project\thanks{E-mail:~
       \href{mailto:latex-l@listserv.uni-heidelberg.de}
            {latex-l@listserv.uni-heidelberg.de}}
      }
    \date{Released~\ExplFileDate}
    \maketitle
  }
%    \end{macrocode}
%
% \subsection{Text extras}
%
%    \begin{macrocode}
\DeclareDocumentCommand \ie { } { \emph{i.e.} }
\DeclareDocumentCommand \eg { } { \emph{e.g.} }
\DeclareDocumentCommand \Ie { } { \emph{I.e.} }
\DeclareDocumentCommand \Eg { } { \emph{E.g.} }
%    \end{macrocode}
%
% \subsection{Math extras}
%
% For \pkg{l3fp}.
%
%    \begin{macrocode}
\AtBeginDocument
  {
    \clist_map_inline:nn
      {
       asin, acos, atan, acot,
       asinh, acosh, atanh, acoth, round, floor, ceil
      }
      { \exp_args:Nc \DeclareMathOperator{#1}{#1} }
  }
%    \end{macrocode}
%
% \begin{macro}{\nan}
%    \begin{macrocode}
\NewDocumentCommand { \nan } { } { \text { \texttt { nan } } }
%    \end{macrocode}
% \end{macro}
%
%    \begin{macrocode}
%</cfg>
%    \end{macrocode}
%
%
% \subsection{Makeindex configuration}
%
% The makeindex style \file{l3doc.ist} is used in place of the usual
% \file{gind.ist} to ensure that |I| is used in the sequence |I J K| not
% |I II II|, which would be the default makeindex behaviour.
%
% Will: Do we need this?
%
% Frank: at the moment we do not distribute or generate this file.
%        \file{gind.ist} is used instead.
%
%    \begin{macrocode}
%<*docist>
actual '='
quote '!'
level '>'
preamble
"\n \\begin{theindex} \n \\makeatletter\\scan@allowedfalse\n"
postamble
"\n\n \\end{theindex}\n"
item_x1   "\\efill \n \\subitem "
item_x2   "\\efill \n \\subsubitem "
delim_0   "\\pfill "
delim_1   "\\pfill "
delim_2   "\\pfill "
% The next lines will produce some warnings when
% running Makeindex as they try to cover two different
% versions of the program:
lethead_prefix   "{\\bfseries\\hfil "
lethead_suffix   "\\hfil}\\nopagebreak\n"
lethead_flag       1
heading_prefix   "{\\bfseries\\hfil "
heading_suffix   "\\hfil}\\nopagebreak\n"
headings_flag       1

% and just for source3:
% Remove R so I is treated in sequence I J K not I II III
page_precedence "rnaA"
%</docist>
%    \end{macrocode}
%
% \section{Testing}
%
% \begin{function}{foo}
%   \begin{syntax}
%     |\foo| \meta{something_n} \Arg{something_1}
%   \end{syntax}
%   Does \meta{something_1} with \meta{something_n}
% \end{function}
%
% \begin{function}{.bool_set:N}
%   blah
% \end{function}
%
% \begin{variable}
%   {
%     \c_alignment_token,
%   }
% \end{variable}
%
% \begin{function}[added=2011-09-06]{\example_foo:N, \example_foo:c}
%   \begin{syntax}
%     |\example_foo:N| <arg_1>
%     |\example_foo:c| \Arg{arg_1}
%   \end{syntax}
%   <0123456789>
% \end{function}
%
% \begin{function}[added=2011-09-06]
%   {\foo, \fooo:, \foooo:n, \foooo:x, \fooooo:n, \fooooo:c, \fooooo:x}
%   \begin{syntax}
%     |\example_foo:N| \meta{arg_1}
%   \end{syntax}
%   \meta{0123456789}
% \end{function}
%
% \begin{function}[TF]{\foo:N, \foo_if:c}
%   Test.
% \end{function}
% \begin{function}[TF,EXP]{\foo:N, \foo_if:c}
%   Test.
% \end{function}
%
% \begin{function}[added=2011-09-06,EXP]{\fffoo:N}
%   Test.
% \end{function}
% \begin{function}[added=2011-09-06,updated=2011-09-07,EXP]{\fffoo:N}
%   Test.
% \end{function}
% \begin{function}[updated=2011-09-06,EXP]{\fffoo:N}
%   Test.
% \end{function}
% \begin{function}[TF]{\ffffoo:N}
%   Test.
% \end{function}
% \begin{function}[pTF]{\ffoo:N}
%   \lipsum[6]
% \end{function}
% \begin{function}[pTF]{\ffoo:N, \ffoo:c, \ffoo:V}
%   \lipsum[6]
% \end{function}
% \begin{function}[pTF]{\ffoo:N, \ffoo:c}
%   \lipsum[6]
% \end{function}
%
%
% \begin{function}[TF]{\ffffoo_with_very_very_very_long_name:N}
%   \lipsum[1]
% \end{function}
%
% \begin{function}[TF]
%   {
%     \ffffoo_with_very_very_very_long_name:N,
%     \ffffoo_with_very_very_very_long_name:c,
%     \ffffoo_with_very_very_very_long_name:V
%   }
%   \lipsum[1]
% \end{function}
%
% \begin{function}[TF]
%   {
%     \ffffoo_with_very_very_very_long_name:N,
%     \ffffoo_with_very_very_very_long_name:c,
%     \ffffoo_with_very_very_very_long_name:V
%   }
%   \begin{syntax}
%     this is how you use it
%   \end{syntax}
%   \lipsum[1]
% \end{function}
%
% \begin{function}[TF]
%   {
%     \ffffoo_with_very_very_very_long_name:N,
%     \ffffoo_with_very_very_very_long_name:V
%   }
%   \begin{syntax}
%     this is how you use it
%   \end{syntax}
%   \lipsum[1]
% \end{function}
%
% \bigskip\bigskip
%
% \begin{macro}[aux]{\foo_aux:}
%   Testing the \enquote{aux} option.
% \end{macro}
%
% \begin{macro}[TF]{\foo_if:c}
%   Testing the \enquote{TF} option.
% \end{macro}
% \begin{macro}[TF]{\foo_if:c, \fooo_if:n}
%   Testing the \enquote{TF} option.
% \end{macro}
%
% \begin{macro}[pTF]{ \foo_if:d }
%   Testing the \enquote{pTF} option.
% \end{macro}
%
% \begin{macro}[internal]{\test_internal:}
%   Testing the \enquote{internal} option.
% \end{macro}
%
% \bigskip\bigskip
%
% \begin{macro}{\aaaa_bbbb_cccc_dddd_eeee_ffff_gggg_hhhh}
%   Long macro names need to be printed in a shorter font.
%    \begin{macrocode}
%    \end{macrocode}
% \end{macro}
%
% \begin{function}{\::N}
%   This is (no longer) weird.
% \end{function}
%
% \begin{macro}{\::N}
%   This is (no longer) weird.
% \end{macro}
%
% \begin{function}[EXP]{\foo}\end{function}
% \begin{function}[rEXP]{\foo}\end{function}
%
% Here is some verbatim text:
% \begin{verbatim}
% a & B # c
% \end{verbatim}
% without overriding this with \pkg{fancyvrb} there would be extraneous
% whitespace.
%
% \begin{macro}
%   {
%     \c_minus_one,
%     \c_zero,
%     \c_one,
%     \c_two,
%     \c_three,
%     \c_four,
%     \c_five,
%     \c_six,
%     \c_seven,
%     \c_eight,
%     \c_nine,
%     \c_ten,
%     \c_eleven,
%     \c_sixteen,
%     \c_thirty_two,
%     \c_hundred_one,
%     \c_twohundred_fifty_five,
%     \c_twohundred_fifty_six,
%     \c_thousand,
%     \c_ten_thousand,
%     \c_ten_thousand_one
%   }
%   \begin{arguments}
%     \item name
%     \item parameters
%   \end{arguments}
%   Another test.
% \end{macro}
%
%
% \subsection{Macros}
% \raggedright
% \ExplSyntaxOn
% \seq_map_inline:Nn \g_doc_macros_seq { \texttt{\enquote{#1}} \quad }
% \ExplSyntaxOff
%
% \subsection{Functions}
% \ExplSyntaxOn
% \seq_map_inline:Nn \g_doc_functions_seq { \texttt{\enquote{#1}} \quad }
% \ExplSyntaxOff
%
% \end{implementation}
%
% \PrintIndex
%
% \endinput
