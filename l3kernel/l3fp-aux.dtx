% \iffalse meta-comment
%
%% File: l3fp-aux.dtx Copyright(C) 2011-2012 The LaTeX3 Project
%%
%% It may be distributed and/or modified under the conditions of the
%% LaTeX Project Public License (LPPL), either version 1.3c of this
%% license or (at your option) any later version.  The latest version
%% of this license is in the file
%%
%%    http://www.latex-project.org/lppl.txt
%%
%% This file is part of the "l3kernel bundle" (The Work in LPPL)
%% and all files in that bundle must be distributed together.
%%
%% The released version of this bundle is available from CTAN.
%%
%% -----------------------------------------------------------------------
%%
%% The development version of the bundle can be found at
%%
%%    http://www.latex-project.org/svnroot/experimental/trunk/
%%
%% for those people who are interested.
%%
%%%%%%%%%%%
%% NOTE: %%
%%%%%%%%%%%
%%
%%   Snapshots taken from the repository represent work in progress and may
%%   not work or may contain conflicting material!  We therefore ask
%%   people _not_ to put them into distributions, archives, etc. without
%%   prior consultation with the LaTeX Project Team.
%%
%% -----------------------------------------------------------------------
%%
%
%<*driver>
\RequirePackage{l3bootstrap}
\GetIdInfo$Id$
  {L3 Floating-point support functions}
\documentclass[full]{l3doc}
\begin{document}
  \DocInput{\jobname.dtx}
\end{document}
%</driver>
% \fi
%
% \title{^^A
%   The \textsf{l3fp-aux} package\\ Support for floating points^^A
%   \thanks{This file describes v\ExplFileVersion,
%     last revised \ExplFileDate.}^^A
% }
%
% \author{^^A
%  The \LaTeX3 Project\thanks
%    {^^A
%      E-mail:
%        \href{mailto:latex-team@latex-project.org}
%          {latex-team@latex-project.org}^^A
%    }^^A
% }
%
% \date{Released \ExplFileDate}
%
% \maketitle
%
% \begin{documentation}
%
% \end{documentation}
%
% \begin{implementation}
%
% \section{\pkg{l3fp-aux} implementation}
%
%    \begin{macrocode}
%<*initex|package>
%    \end{macrocode}
%
%    \begin{macrocode}
%<@@=fp>
%    \end{macrocode}
%
% ^^A todo: make sanitize and pack more homogeneous between modules.
%
% ^^A begin[todo]: move
% \section{Internal storage of floating points numbers}
%
% A floating point number \meta{X} is stored as
% \begin{quote}
%   \cs{s_@@} \cs{@@_chk:w} \meta{case} \meta{sign} \meta{body} |;|
% \end{quote}
% Here, \meta{case} is 0 for $\pm 0$, 1 for normal numbers, 2 for $\pm
% \infty$, and 3 for \texttt{nan}, and \meta{sign} is $0$ for positive
% numbers, $1$ for \texttt{nan}s, and $2$ for negative numbers. The
% \meta{body} of normal numbers is \Arg{exponent} \Arg{X_1} \Arg{X_2}
% \Arg{X_3} \Arg{X_4}, with
% \[
% \meta{X} = (-1)^{\meta{sign}} 10^{-\meta{exponent}} \sum_i
% \meta{X_i} 10^{-4i}.
% \]
% Calculations are done in base $10000$, \emph{i.e.} one myriad.  The
% \meta{exponent} lies between $\pm\cs{c_@@_max_exponent_int} = \pm
% \the\csname\detokenize{c__fp_max_exponent_int}\endcsname$ inclusive.
%
% Additionally, positive and negative floating point numbers may only be
% stored with $1000\leq\meta{X_1}<10000$. This requirement is necessary
% in order to preserve accuracy and speed.
%
% ^^A end[todo]
%
% \subsection{Using arguments and semicolons}
%
% \begin{macro}[int, EXP]{\@@_use_none_stop_f:n}
%   This function removes an argument (typically a digit) and replaces
%   it by \cs{exp_stop_f:}, a marker which stops \texttt{f}-type
%   expansion.
%    \begin{macrocode}
\cs_new:Npn \@@_use_none_stop_f:n #1 { \exp_stop_f: }
%    \end{macrocode}
% \end{macro}
%
% \begin{macro}[int, EXP]{\@@_use_s:n, \@@_use_s:nn}
%   Those functions place a semicolon after one or two arguments
%   (typically digits).
%    \begin{macrocode}
\cs_new:Npn \@@_use_s:n #1 { #1; }
\cs_new:Npn \@@_use_s:nn #1#2 { #1#2; }
%    \end{macrocode}
% \end{macro}
%
% \begin{macro}[int, EXP]
%   {\@@_use_none_until_s:w, \@@_use_i_until_s:nw, \@@_use_ii_until_s:nnw}
%   Those functions select specific arguments among a set of arguments
%   delimited by a semicolon.
%    \begin{macrocode}
\cs_new:Npn \@@_use_none_until_s:w #1; { }
\cs_new:Npn \@@_use_i_until_s:nw #1#2; {#1}
\cs_new:Npn \@@_use_ii_until_s:nnw #1#2#3; {#2}
%    \end{macrocode}
% \end{macro}
%
% \begin{macro}[int, EXP]{\@@_reverse_args:Nww}
%   Many internal functions take arguments delimited by semicolons, and
%   it is occasionally useful to swap two such arguments.
%    \begin{macrocode}
\cs_new:Npn \@@_reverse_args:Nww #1 #2; #3; { #1 #3; #2; }
%    \end{macrocode}
% \end{macro}
%
% \subsection{Constants, and structure of floating points}
%
% \begin{macro}[int]{\s_@@, \@@_chk:w}
%   Floating points numbers all start with \cs{s_@@} \cs{@@_chk:w},
%   where \cs{s_@@} is equal to the \TeX{} primitive \tn{relax}, and
%   \cs{@@_chk:w} is protected.  The rest of the floating point number
%   is made of characters (or \tn{relax}).  This ensures that nothing
%   expands under \texttt{f}-expansion, nor under \texttt{x}-expansion.
%   However, when typeset, \cs{s_@@} does nothing, and \cs{@@_chk:w} is
%   expanded.  We define \cs{@@_chk:w} to produce an error.
%    \begin{macrocode}
\__scan_new:N \s_@@
\cs_new_protected:Npn \@@_chk:w #1 ;
  {
    \__msg_kernel_error:nnx { kernel } { misused-fp }
      { \@@_to_tl:w \s_@@ \@@_chk:w #1 ; }
  }
%    \end{macrocode}
% \end{macro}
%
% \begin{macro}[int]{\s_@@_mark, \s_@@_stop}
%   Aliases of \cs{tex_relax:D}, used to terminate expressions.
%    \begin{macrocode}
\__scan_new:N \s_@@_mark
\__scan_new:N \s_@@_stop
%    \end{macrocode}
% \end{macro}
%
% \begin{macro}[int]
%   {
%     \s_@@_invalid, \s_@@_underflow, \s_@@_overflow,
%     \s_@@_division, \s_@@_exact
%   }
%   A couple of scan marks used to indicate where special floating point
%   numbers come from.
%    \begin{macrocode}
\__scan_new:N \s_@@_invalid
\__scan_new:N \s_@@_underflow
\__scan_new:N \s_@@_overflow
\__scan_new:N \s_@@_division
\__scan_new:N \s_@@_exact
%    \end{macrocode}
% \end{macro}
%
% \begin{variable}
%   {\c_zero_fp, \c_minus_zero_fp, \c_inf_fp, \c_minus_inf_fp, \c_nan_fp}
%   The special floating points. All of them have the form
%   \begin{quote}
%     \cs{s_@@} \cs{@@_chk:w} \meta{case} \meta{sign} \cs{s_@@_...} |;|
%   \end{quote}
%   where the dots in \cs{s_@@_...} are one of \texttt{invalid},
%   \texttt{underflow}, \texttt{overflow}, \texttt{division},
%   \texttt{exact}, describing how the floating point was created.  We
%   define the floating points here as \enquote{exact}.
%    \begin{macrocode}
\tl_const:Nn \c_zero_fp       { \s_@@ \@@_chk:w 0 0 \s_@@_exact ; }
\tl_const:Nn \c_minus_zero_fp { \s_@@ \@@_chk:w 0 2 \s_@@_exact ; }
\tl_const:Nn \c_inf_fp        { \s_@@ \@@_chk:w 2 0 \s_@@_exact ; }
\tl_const:Nn \c_minus_inf_fp  { \s_@@ \@@_chk:w 2 2 \s_@@_exact ; }
\tl_const:Nn \c_nan_fp        { \s_@@ \@@_chk:w 3 1 \s_@@_exact ; }
%    \end{macrocode}
% \end{variable}
%
% \begin{variable}{\c_@@_max_exponent_int}
%   Normal floating point numbers have an exponent at most
%   \texttt{max_exponent} in absolute value.  Larger numbers are rounded
%   to $\pm\infty$.  Smaller numbers are subnormal (not implemented yet),
%   and digits beyond
%   $10^{-\text{\texttt{max_exponent}}}$ are rounded away, hence the
%   true minimum exponent is $-\text{\texttt{max_exponent}}-16$;
%   beyond this, numbers are rounded to zero.  Why this choice of
%   limits?  When computing $(a\cdot 10^n)^(b\cdot 10^p)$, we need to
%   evaluate $\log(a\cdot 10^n) = \log(a) + n \log(10)$ as a fixed point
%   number, which we manipulate as blocks of $4$ digits.  Multiplying
%   such a fixed point number by $n<10000$ is much cheaper than larger
%   $n$, because we can multiply $n$ with each block safely.
%    \begin{macrocode}
\int_const:Nn \c_@@_max_exponent_int { 10000 }
%    \end{macrocode}
% \end{variable}
%
% \begin{macro}[int, EXP]{\@@_zero_fp:N, \@@_inf_fp:N}
%   In case of overflow or underflow, we have to output
%   a zero or infinity with a given sign.
%    \begin{macrocode}
\cs_new:Npn \@@_zero_fp:N #1 { \s_@@ \@@_chk:w 0 #1 \s_@@_underflow ; }
\cs_new:Npn \@@_inf_fp:N #1  { \s_@@ \@@_chk:w 2 #1 \s_@@_overflow ; }
%    \end{macrocode}
% \end{macro}
%
%^^A todo: currently unused.
% \begin{macro}[int, EXP]{\@@_max_fp:N, \@@_min_fp:N}
%   In some cases, we need to output the smallest or biggest positive or
%   negative finite numbers.
%    \begin{macrocode}
\cs_new:Npn \@@_min_fp:N #1
  {
    \s_@@ \@@_chk:w 1 #1
      { \int_eval:n { - \c_@@_max_exponent_int } }
      {1000} {0000} {0000} {0000} ;
  }
\cs_new:Npn \@@_max_fp:N #1
  {
    \s_@@ \@@_chk:w 1 #1
      { \int_use:N \c_@@_max_exponent_int }
      {9999} {9999} {9999} {9999} ;
  }
%    \end{macrocode}
% \end{macro}
%
% \begin{macro}[int, EXP]{\@@_exponent:w}
%   For normal numbers, the function expands to the exponent, otherwise
%   to $0$.
%    \begin{macrocode}
\cs_new:Npn \@@_exponent:w \s_@@ \@@_chk:w #1
  {
    \if_meaning:w 1 #1
      \exp_after:wN \@@_use_ii_until_s:nnw
    \else:
      \exp_after:wN \@@_use_i_until_s:nw
      \exp_after:wN 0
    \fi:
  }
%    \end{macrocode}
% \end{macro}
%
% \subsection{Overflow, underflow, and exact zero}
%
%^^A todo: the sign of exact zeros should depend on the rounding mode.
%
% \begin{macro}[int, EXP]{\@@_sanitize:Nw, \@@_sanitize:wN}
% \begin{macro}[aux, EXP]{\@@_sanitize_zero:w}
%   Expects the sign and the exponent in some order, then the
%   significand (which we don't touch).  Outputs the corresponding
%   floating point number, possibly underflowed to $\pm 0$ or overflowed
%   to $\pm\infty$.  The functions \cs{@@_underflow:w} and
%   \cs{@@_overflow:w} are defined in \pkg{l3fp-traps}.
%    \begin{macrocode}
\cs_new:Npn \@@_sanitize:Nw #1 #2;
  {
    \if_case:w \if_int_compare:w #2 > \c_@@_max_exponent_int \c_one \else:
               \if_int_compare:w #2 < - \c_@@_max_exponent_int \c_two \else:
               \if_meaning:w 1 #1 \c_three \else: \c_zero \fi: \fi: \fi:
    \or: \exp_after:wN \@@_overflow:w
    \or: \exp_after:wN \@@_underflow:w
    \or: \exp_after:wN \@@_sanitize_zero:w
    \fi:
    \s_@@ \@@_chk:w 1 #1 {#2}
  }
\cs_new:Npn \@@_sanitize:wN #1; #2 { \@@_sanitize:Nw #2 #1; }
\cs_new:Npn \@@_sanitize_zero:w \s_@@ \@@_chk:w #1 #2 #3; { \c_zero_fp }
%    \end{macrocode}
% \end{macro}
% \end{macro}
%
% \subsection{Expanding after a floating point number}
%
% \begin{macro}[int, EXP]{\@@_exp_after_o:w}
% \begin{macro}[int, EXP]{\@@_exp_after_o:nw, \@@_exp_after_f:nw}
%   \begin{syntax}
%     \cs{@@_exp_after_o:nw} \Arg{tokens} \meta{floating point} \meta{more tokens}
%   \end{syntax}
%   Places \meta{tokens} (empty in the case of \cs{@@_exp_after_o:w})
%   between the \meta{floating point} and the \meta{more tokens}, then
%   hits those tokens with either \texttt{o}-expansion (one
%   \cs{exp_after:wN}) or \texttt{f}-expansion, and leaves the floating
%   point number unchanged.
%
%   We first distinguish normal floating points, which have a mantissa,
%   from the much simpler special floating points.
%    \begin{macrocode}
\cs_new:Npn \@@_exp_after_o:w \s_@@ \@@_chk:w #1
  {
    \if_meaning:w 1 #1
      \exp_after:wN \@@_exp_after_normal:nNNw
    \else:
      \exp_after:wN \@@_exp_after_special:nNNw
    \fi:
    { }
    #1
  }
\cs_new:Npn \@@_exp_after_o:nw #1 \s_@@ \@@_chk:w #2
  {
    \if_meaning:w 1 #2
      \exp_after:wN \@@_exp_after_normal:nNNw
    \else:
      \exp_after:wN \@@_exp_after_special:nNNw
    \fi:
    { #1 }
    #2
  }
\cs_new:Npn \@@_exp_after_f:nw #1 \s_@@ \@@_chk:w #2
  {
    \if_meaning:w 1 #2
      \exp_after:wN \@@_exp_after_normal:nNNw
    \else:
      \exp_after:wN \@@_exp_after_special:nNNw
    \fi:
    { \tex_romannumeral:D -`0 #1 }
    #2
  }
%    \end{macrocode}
% \end{macro}
% \end{macro}
%
% \begin{macro}[aux, EXP]{\@@_exp_after_special:nNNw}
%   \begin{syntax}
%     \cs{@@_exp_after_special:nNNw} \Arg{after} \meta{case} \meta{sign} \meta{scan mark} |;|
%   \end{syntax}
%   Special floating point numbers are easy to jump over since they
%   contain few tokens.
%    \begin{macrocode}
\cs_new:Npn \@@_exp_after_special:nNNw #1#2#3#4;
  {
    \exp_after:wN \s_@@
    \exp_after:wN \@@_chk:w
    \exp_after:wN #2
    \exp_after:wN #3
    \exp_after:wN #4
    \exp_after:wN ;
    #1
  }
%    \end{macrocode}
% \end{macro}
%
% \begin{macro}[aux, EXP]{\@@_exp_after_normal:nNNw}
%   For normal floating point numbers, life is slightly harder, since we
%   have many tokens to jump over.  Here it would be slightly better if
%   the digits were not braced but instead were delimited arguments (for
%   instance delimited by |,|).  That may be changed some day.
%    \begin{macrocode}
\cs_new:Npn \@@_exp_after_normal:nNNw #1 1 #2 #3 #4#5#6#7;
  {
    \exp_after:wN \@@_exp_after_normal:Nwwwww
    \exp_after:wN #2
    \__int_value:w #3   \exp_after:wN ;
    \__int_value:w 1 #4 \exp_after:wN ;
    \__int_value:w 1 #5 \exp_after:wN ;
    \__int_value:w 1 #6 \exp_after:wN ;
    \__int_value:w 1 #7 \exp_after:wN ; #1
  }
\cs_new:Npn \@@_exp_after_normal:Nwwwww
    #1 #2; 1 #3 ; 1 #4 ; 1 #5 ; 1 #6 ;
  { \s_@@ \@@_chk:w 1 #1 {#2} {#3} {#4} {#5} {#6} ; }
%    \end{macrocode}
% \end{macro}
%
% \subsection{Packing digits}
%
% When a positive integer |#1| is known to be less than $10^8$, the
% following trick will split it into two blocks of $4$ digits, padding
% with zeros on the left.
% \begin{verbatim}
%   \cs_new:Npn \pack:NNNNNw #1 #2#3#4#5 #6; { {#2#3#4#5} {#6} }
%   \exp_after:wN \pack:NNNNNw
%     \int_use:N \__int_eval:w 1 0000 0000 + #1 ;
% \end{verbatim}
% The idea is that adding $10^8$ to the number ensures that it has
% exactly $9$ digits, and can then easily find which digits correspond
% to what position in the number. Of course, this can be modified
% for any number of digits less or equal to~$9$ (we are limited by
% \TeX{}'s integers). This method is very heavily relied upon in
% \texttt{l3fp-basics}.
%
% More specifically, the auxiliary inserts |+ #1#2#3#4#5 ; {#6}|, which
% allows us to compute several blocks of $4$ digits in a nested manner,
% performing carries on the fly.  Say we want to compute $1\,2345 \times
% 6677\,8899$.  With simplified names, we would do
% \begin{verbatim}
%   \exp_after:wN \post_processing:w
%   \int_use:N \__int_eval:w - 5 0000
%     \exp_after:wN \pack:NNNNNw
%     \int_use:N \__int_eval:w 4 9995 0000
%       + 12345 * 6677
%       \exp_after:wN \pack:NNNNNw
%       \int_use:N \__int_eval:w 5 0000 0000
%         + 12345 * 8899 ;
% \end{verbatim}
% The \cs{exp_after:wN} triggers |\int_use:N \__int_eval:w|, which
% starts a first computation, whose initial value is $- 5\,0000$ (the
% \enquote{leading shift}).  In that computation appears an
% \cs{exp_after:wN}, which triggers the nested computation
% |\int_use:N \__int_eval:w| with starting value $4\,9995\,0000$ (the
% \enquote{middle shift}).  That, in turn, expands \cs{exp_after:wN}
% which triggers the third computation.  The third computation's value
% is $5\,0000\,0000 + 12345 \times 8899$, which has $9$ digits. Adding
% $5\cdot 10^{8}$ to the product allowed us to know how many digits to
% expect as long as the numbers to multiply are not too big; it will
% also work to some extent with negative results.  The \texttt{pack}
% function puts the last $4$ of those $9$ digits into a brace group,
% moves the semi-colon delimiter, and inserts a |+|, which combines the
% carry with the previous computation.  The shifts nicely combine into
% $5\,0000\,0000 / 10^{4} + 4\,9995\,0000 = 5\,0000\,0000$.  As long as
% the operands are in some range, the result of this second computation
% will have $9$ digits.  The corresponding \texttt{pack} function,
% expanded after the result is computed, braces the last $4$ digits, and
% leaves |+| \meta{5 digits} for the initial computation.  The
% \enquote{leading shift} cancels the combination of the other shifts,
% and the |\post_processing:w| takes care of packing the last few
% digits.
%
% Admittedly, this is quite intricate.  It is probably the key in making
% \pkg{l3fp} as fast as other pure \TeX{} floating point units despite
% its increased precision.  In fact, this is used so much that we
% provide different sets of packing functions and shifts, depending on
% ranges of input.
%
% \begin{macro}[int, EXP]{\@@_pack:NNNNNw}
% \begin{variable}
%   {
%     \c_@@_trailing_shift_int ,
%     \c_@@_middle_shift_int   ,
%     \c_@@_leading_shift_int  ,
%   }
%   This set of shifts allows for computations involving results in the
%   range $[-4\cdot 10^{8}, 5\cdot 10^{8}-1]$.  Shifted values all have
%   exactly $9$ digits.
%    \begin{macrocode}
\int_const:Nn \c_@@_leading_shift_int  { - 5 0000 }
\int_const:Nn \c_@@_middle_shift_int   { 5 0000 *  9999 }
\int_const:Nn \c_@@_trailing_shift_int { 5 0000 * 10000 }
\cs_new:Npn \@@_pack:NNNNNw #1 #2#3#4#5 #6; { + #1#2#3#4#5 ; {#6} }
%    \end{macrocode}
% \end{variable}
% \end{macro}
%
% \begin{macro}[int, EXP]{\@@_pack_big:NNNNNNw}
% \begin{variable}
%   {
%     \c_@@_big_trailing_shift_int ,
%     \c_@@_big_middle_shift_int   ,
%     \c_@@_big_leading_shift_int  ,
%   }
%   This set of shifts allows for computations involving results in the
%   range $[-5\cdot 10^{8}, 6\cdot 10^{8}-1]$ (actually a bit more).
%   Shifted values all have exactly $10$ digits.  Note that the upper
%   bound is due to \TeX{}'s limit of $2^{31}-1$ on integers.  The
%   shifts are chosen to be roughly the mid-point of $10^{9}$ and
%   $2^{31}$, the two bounds on $10$-digit integers in \TeX{}.
%    \begin{macrocode}
\int_const:Nn \c_@@_big_leading_shift_int  { - 15 2374 }
\int_const:Nn \c_@@_big_middle_shift_int   { 15 2374 *  9999 }
\int_const:Nn \c_@@_big_trailing_shift_int { 15 2374 * 10000 }
\cs_new:Npn \@@_pack_big:NNNNNNw #1#2 #3#4#5#6 #7;
  { + #1#2#3#4#5#6 ; {#7} }
%    \end{macrocode}
% \end{variable}
% \end{macro}
%
% \begin{macro}[int, EXP]{\@@_pack_twice_four:wNNNNNNNN}
%   \begin{syntax}
%     \cs{@@_pack_twice_four:wNNNNNNNN} \meta{tokens} |;| \meta{$\geq 8$ digits}
%   \end{syntax}
%   Grabs two sets of $4$ digits and places them before the semi-colon
%   delimiter.  Putting several copies of this function before a
%   semicolon will pack more digits since each will take the digits
%   packed by the others in its first argument.
%    \begin{macrocode}
\cs_new:Npn \@@_pack_twice_four:wNNNNNNNN #1; #2#3#4#5 #6#7#8#9
  { #1 {#2#3#4#5} {#6#7#8#9} ; }
%    \end{macrocode}
% \end{macro}
%
% \subsection{Decimate (dividing by a power of 10)}
%
% ^^A begin[todo]
% \begin{macro}[int, EXP]{\@@_decimate:nNnnnn}
%   \begin{syntax}
%     \cs{@@_decimate:nNnnnn} \Arg{shift} \Arg{f_1}
%     ~~\Arg{X_1} \Arg{X_2} \Arg{X_3} \Arg{X_4}
%   \end{syntax}
%   Each \meta{X_i} consists in $4$ digits exactly,
%   and $1000\leq\meta{X_1}<9999$. The first argument determines
%   by how much we shift the digits. \meta{f_1} is called as follows:
%   \begin{syntax}
%     \meta{f_1} \meta{rounding} \Arg{X'_1} \Arg{X'_2} \meta{extra-digits} |;|
%   \end{syntax}
%   where $0\leq\meta{X'_i}<10^{8}-1$ are $8$ digit numbers,
%   forming the truncation of our number. In other words,
%   \[
%   \left(
%     \sum_{i=1}^{4} \meta{X_i} \cdot 10^{-4i} \cdot 10^{-\meta{shift}}
%     - \meta{X'_1} \cdot 10^{-8} + \meta{X'_2} \cdot 10^{-16}
%   \right)
%   \in [0,10^{-16}).
%   \]
%   To round properly later, we need to remember some information
%   about the difference. The \meta{rounding} digit is $0$ if and
%   only if the difference is exactly $0$, and $5$ if and only if
%   the difference is exactly $0.5\cdot 10^{-16}$. Otherwise, it
%   is the (non-$0$, non-$5$) digit closest to $10^{17}$ times the
%   difference.  In particular, if the shift is $17$ or more, all
%   the digits are dropped, \meta{rounding} is $1$ (not $0$), and
%   \meta{X'_1} \meta{X'_2} are both zero.
%
%   If the shift is $1$, the \meta{rounding} digit is simply the
%   only digit that was pushed out of the brace groups (this is
%   important for subtraction). It would be more natural for the
%   \meta{rounding} digit to be placed after the \meta{X_i},
%   but the choice we make involves less reshuffling.
%
%   Note that this function fails for negative \meta{shift}.
%    \begin{macrocode}
\cs_new:Npn \@@_decimate:nNnnnn #1
  {
    \cs:w
      @@_decimate_
      \if_int_compare:w \__int_eval:w #1 > \c_sixteen
        tiny
      \else:
        \tex_romannumeral:D \__int_eval:w #1
      \fi:
      :Nnnnn
    \cs_end:
  }
%    \end{macrocode}
%   Each of the auxiliaries see the function \meta{f_1},
%   followed by $4$ blocks of $4$ digits.
% \end{macro}
%
% \begin{macro}[aux, EXP]{\@@_decimate_:Nnnnn, \@@_decimate_tiny:Nnnnn}
%   If the \meta{shift} is zero, or too big, life is very easy.
%    \begin{macrocode}
\cs_new:Npn \@@_decimate_:Nnnnn #1 #2#3#4#5
  { #1 0 {#2#3} {#4#5} ; }
\cs_new:Npn \@@_decimate_tiny:Nnnnn #1 #2#3#4#5
  { #1 1 { 0000 0000 } { 0000 0000 } 0 #2#3#4#5 ; }
%    \end{macrocode}
% \end{macro}
%
% \begin{macro}[aux, EXP]
%   {
%     \@@_decimate_i:Nnnnn,    \@@_decimate_ii:Nnnnn,
%     \@@_decimate_iii:Nnnnn,  \@@_decimate_iv:Nnnnn,
%     \@@_decimate_v:Nnnnn,    \@@_decimate_vi:Nnnnn,
%     \@@_decimate_vii:Nnnnn,  \@@_decimate_viii:Nnnnn,
%     \@@_decimate_ix:Nnnnn,   \@@_decimate_x:Nnnnn,
%     \@@_decimate_xi:Nnnnn,   \@@_decimate_xii:Nnnnn,
%     \@@_decimate_xiii:Nnnnn, \@@_decimate_xiv:Nnnnn,
%     \@@_decimate_xv:Nnnnn,   \@@_decimate_xvi:Nnnnn
%   }
%   \begin{syntax}
%     \cs{@@_decimate_i:Nnnnn} \meta{f_1} \Arg{X_1} \Arg{X_2} \Arg{X_3} \Arg{X_4}
%   \end{syntax}
%   Shifting happens in two steps: compute the \meta{rounding} digit,
%   and repack digits into two blocks of $8$. The sixteen functions
%   are very similar, and defined through \cs{@@_tmp:w}.
%   The arguments are as follows: |#1| indicates which function is
%   being defined; after one step of expansion, |#2| yields the
%   \enquote{extra digits} which are then converted by
%   \cs{@@_decimate_round:Nw} to the \meta{rounding} digit.
%   This triggers the \texttt{f}-expansion of
%   \cs{@@_decimate_pack:nnnnnnnnnnw},\footnote{No, the argument
%     spec is not a mistake: the function calls an auxiliary to
%     do half of the job.} responsible for building two blocks of
%   $8$ digits, and removing the rest. For this to work, |#3|
%   alternates between braced and unbraced blocks of $4$ digits,
%   in such a way that the $5$ first and $5$ next token groups
%   yield the correct blocks of $8$ digits.
%    \begin{macrocode}
\cs_new:Npn \@@_tmp:w #1 #2 #3
  {
    \cs_new:cpn { @@_decimate_ #1 :Nnnnn } ##1 ##2##3##4##5
      {
        \exp_after:wN ##1
        \__int_value:w
          \exp_after:wN \@@_decimate_round:Nw #2 ;
        \@@_decimate_pack:nnnnnnnnnnw #3 ;
      }
  }
\@@_tmp:w {i}   {\use_none:nnn      #50} {    0{#2}#3{#4}#5                }
\@@_tmp:w {ii}  {\use_none:nn       #5 } {    00{#2}#3{#4}#5               }
\@@_tmp:w {iii} {\use_none:n        #5 } {    000{#2}#3{#4}#5              }
\@@_tmp:w {iv}  {                   #5 } {   {0000}#2{#3}#4 #5             }
\@@_tmp:w {v}   {\use_none:nnn    #4#5 } {   0{0000}#2{#3}#4 #5            }
\@@_tmp:w {vi}  {\use_none:nn     #4#5 } {   00{0000}#2{#3}#4 #5           }
\@@_tmp:w {vii} {\use_none:n      #4#5 } {   000{0000}#2{#3}#4 #5          }
\@@_tmp:w {viii}{                 #4#5 } {  {0000}0000{#2}#3 #4 #5         }
\@@_tmp:w {ix}  {\use_none:nnn  #3#4+#5} {  0{0000}0000{#2}#3 #4 #5        }
\@@_tmp:w {x}   {\use_none:nn   #3#4+#5} {  00{0000}0000{#2}#3 #4 #5       }
\@@_tmp:w {xi}  {\use_none:n    #3#4+#5} {  000{0000}0000{#2}#3 #4 #5      }
\@@_tmp:w {xii} {               #3#4+#5} { {0000}0000{0000}#2 #3 #4 #5     }
\@@_tmp:w {xiii}{\use_none:nnn#2#3+#4#5} { 0{0000}0000{0000}#2 #3 #4 #5    }
\@@_tmp:w {xiv} {\use_none:nn #2#3+#4#5} { 00{0000}0000{0000}#2 #3 #4 #5   }
\@@_tmp:w {xv}  {\use_none:n  #2#3+#4#5} { 000{0000}0000{0000}#2 #3 #4 #5  }
\@@_tmp:w {xvi} {             #2#3+#4#5} {{0000}0000{0000}0000 #2 #3 #4 #5 }
%    \end{macrocode}
% \end{macro}
%
% \begin{macro}[EXP, aux]
%   {\@@_decimate_round:Nw, \@@_decimate_pack:nnnnnnnnnnw}
%   \cs{@@_decimate_round:Nw} will receive the \enquote{extra digits}
%   as its argument, and its expansion is triggered by \cs{__int_value:w}.
%   If the first digit is neither $0$ nor $5$, then it is the \meta{rounding}
%   digit.  Otherwise, if the remaining digits are not all zero, we need
%   to add $1$ to that leading digit to get the rounding digit. Some caution
%   is required, though, because there may be more than $10$
%   \enquote{extra digits}, and this may overflow \TeX{}'s integers.
%   Instead of feeding the digits directly to \cs{@@_decimate_round:Nw},
%   they come split into several blocks, separated by $+$. Hence the first
%   \cs{__int_eval:w} here.
%    \begin{macrocode}
\cs_new:Npn \@@_decimate_round:Nw #1 #2;
  {
    \if_int_odd:w \if_meaning:w 0 #1 \c_one \else:
                  \if_meaning:w 5 #1 \c_one \else:
                  \c_zero \fi: \fi:
      \if_int_compare:w \__int_eval:w #2 > \c_zero
        \__int_eval:w 1 +
      \fi:
    \fi:
    #1
  }
%    \end{macrocode}
%   The computation of the \meta{rounding} digit leaves an unfinished
%   \cs{__int_value:w}, which expands the following functions. This
%   allows us to repack nicely the digits we keep. Those digits come
%   as an alternation of unbraced and braced blocks of $4$ digits,
%   such that the first $5$ groups of token consist in $4$ single digits,
%   and one brace group (in some order), and the next $5$ have the same
%   structure. This is followed by some digits and a semicolon.
%    \begin{macrocode}
\cs_new:Npn \@@_decimate_pack:nnnnnnnnnnw #1#2#3#4#5
  { \@@_decimate_pack_ii:nnnnnnw { #1#2#3#4#5 } }
\cs_new:Npn \@@_decimate_pack_ii:nnnnnnw #1 #2#3#4#5#6
  { {#1} {#2#3#4#5#6} }
%    \end{macrocode}
% \end{macro}
% ^^A end[todo]
%
% \subsection{Functions for use within primitive conditional branches}
%
% The functions described in this section are not pretty and can easily
% be misused.  When correctly used, each of them removes one \cs{fi:} as
% part of its parameter text, and puts one back as part of its
% replacement text.
%
% Many computation functions in \pkg{l3fp} must perform tests on the
% type of floating points that they receive.  This is often done in an
% \cs{if_case:w} statement or another conditional statement, and only a
% few cases lead to actual computations: most of the special cases are
% treated using a few standard functions which we define now.  A typical
% use context for those functions would be
% \begin{syntax}
%   |\if_case:w| \meta{integer} |\exp_stop_f:|
%   |     \@@_case_return_o:Nw| \meta{fp var}
%   |\or: \@@_case_use:nw| \Arg{some computation}
%   |\or: \@@_case_return_same_o:w|
%   |\or: \@@_case_return:nw| \Arg{something}
%   |\fi:|
%   \meta{junk}
%   \meta{floating point}
% \end{syntax}
% In this example, the case $0$ will return the floating point
% \meta{fp~var}, expanding once after that floating point.  Case $1$
% will do \meta{some computation} using the \meta{floating point}
% (presumably compute the operation requested by the user in that
% non-trivial case).  Case $2$ will return the \meta{floating point}
% without modifying it, removing the \meta{junk} and expanding once
% after.  Case $3$ will close the conditional, remove the \meta{junk}
% and the \meta{floating point}, and expand \meta{something} next.  In
% other cases, the \enquote{\meta{junk}} is expanded, performing some
% other operation on the \meta{floating point}.  We provide similar
% functions with two trailing \meta{floating points}.
%
% \begin{macro}[int, EXP]{\@@_case_use:nw}
%   This function ends a \TeX{} conditional, removes junk until the next
%   floating point, and places its first argument before that floating
%   point, to perform some operation on the floating point.
%    \begin{macrocode}
\cs_new:Npn \@@_case_use:nw #1#2 \fi: #3 \s_@@ { \fi: #1 \s_@@ }
%    \end{macrocode}
% \end{macro}
%
% \begin{macro}[int, EXP]{\@@_case_return:nw}
%   This function ends a \TeX{} conditional, removes junk and a floating
%   point, and places its first argument in the input stream.  A quirk
%   is that we don't define this function requiring a floating point to
%   follow, simply anything ending in a semicolon.  This, in turn, means
%   that the \meta{junk} may not contain semicolons.
%    \begin{macrocode}
\cs_new:Npn \@@_case_return:nw #1#2 \fi: #3 ; { \fi: #1 }
%    \end{macrocode}
% \end{macro}
%
% \begin{macro}[int, EXP]{\@@_case_return_o:Nw}
%   This function ends a \TeX{} conditional, removes junk and a floating
%   point, and returns its first argument, a \meta{fp~var}, expanding
%   once after it.
%    \begin{macrocode}
\cs_new:Npn \@@_case_return_o:Nw #1#2 \fi: #3 \s_@@ #4 ;
  { \fi: \exp_after:wN #1 }
%    \end{macrocode}
% \end{macro}
%
% \begin{macro}[int, EXP]{\@@_case_return_same_o:w}
%   This function ends a \TeX{} conditional, removes junk, and returns
%   the following floating point, expanding once after it.
%    \begin{macrocode}
\cs_new:Npn \@@_case_return_same_o:w #1 \fi: #2 \s_@@
  { \fi: \@@_exp_after_o:w \s_@@ }
%    \end{macrocode}
% \end{macro}
%
% \begin{macro}[int, EXP]{\@@_case_return_o:Nww}
%   Same as \cs{@@_case_return_o:Nw} but with two trailing floating
%   points.
%    \begin{macrocode}
\cs_new:Npn \@@_case_return_o:Nww #1#2 \fi: #3 \s_@@ #4 ; #5 ;
  { \fi: \exp_after:wN #1 }
%    \end{macrocode}
% \end{macro}
%
% \begin{macro}[int, EXP]{\@@_case_return_ii_o:ww}
%   Similar to \cs{@@_case_return_same_o:w}, but this returns the second
%   of two trailing floating point numbers, expanding once after it.
%    \begin{macrocode}
\cs_new:Npn \@@_case_return_ii_o:ww #1 \fi: #2 \s_@@ #3 ;
  { \fi: \@@_exp_after_o:w }
%    \end{macrocode}
% \end{macro}
%
% \subsection{Small integer floating points}
%
% \begin{macro}[int, EXP]{\@@_small_int:wTF}
% \begin{macro}[aux, EXP]
%   {
%     \@@_small_int_true:wTF,
%     \@@_small_int_normal:NnwTF,
%     \@@_small_int_test:NnnwNTF
%   }
%   This function tests if its floating point argument is an integer in
%   the range $[-99999999,99999999]$.  If it is, the result of the
%   conversion is fed as a braced argument to the \meta{true code}.
%   Otherwise, the \meta{false code} is performed.  First filter special
%   cases: neither \texttt{nan} nor infinities are integers.  Normal
%   numbers with a non-positive exponent are never integers.  When the
%   exponent is greater than $8$, the number is too large for the range.
%   Otherwise, decimate, and test the digits after the decimal
%   separator.  The \cs{use_iii:nnn} remove a trailing |;| and the true
%   branch, leaving only the false branch.  The \cs{__int_value:w}
%   appearing in the case where the normal floating point is an integer
%   takes care of expanding all the conditionals until the trailing |;|.
%   That integer is fed to \cs{@@_small_int_true:wTF} which places it as
%   a braced argument of the true branch.
%    \begin{macrocode}
\cs_new:Npn \@@_small_int:wTF \s_@@ \@@_chk:w #1
  {
    \if_case:w #1 \exp_stop_f:
           \@@_case_return:nw { \@@_small_int_true:wTF 0 ; }
    \or:   \exp_after:wN \@@_small_int_normal:NnwTF
    \else: \@@_case_return:nw \use_ii:nn
    \fi:
  }
\cs_new:Npn \@@_small_int_true:wTF #1; #2#3 { #2 {#1} }
\cs_new:Npn \@@_small_int_normal:NnwTF #1#2#3;
  {
    \if_int_compare:w #2 > \c_zero
      \if_int_compare:w #2 > \c_eight
        \exp_after:wN \exp_after:wN
        \exp_after:wN \use_iii:nnn
      \else:
        \@@_decimate:nNnnnn { \c_sixteen - #2 }
          \@@_small_int_test:NnnwNTF
          #3 #1
      \fi:
    \else:
      \exp_after:wN \use_iii:nnn
    \fi:
    ;
  }
\cs_new:Npn \@@_small_int_test:NnnwNTF #1#2#3#4; #5
  {
    \if_meaning:w 0 #1
      \exp_after:wN \@@_small_int_true:wTF
      \__int_value:w \if_meaning:w 2 #5 - \fi: #3
    \else:
      \exp_after:wN \use_iii:nnn
    \fi:
  }
%    \end{macrocode}
% \end{macro}
% \end{macro}
%
% \subsection{Length of a floating point array}
%
% \begin{macro}[int, EXP]{\@@_array_count:w}
% \begin{macro}[aux, EXP]{\@@_array_count_loop:Nw}
%   Count the number of items in an array of floating points.  The
%   technique is very similar to \cs{tl_count:n}, but with the loop
%   built-in.  Checking for the end of the loop is done with the
%   |\use_none:n #1| construction.
%    \begin{macrocode}
\cs_new:Npn \@@_array_count:w #1 @
  {
    \int_use:N \__int_eval:w \c_zero
      \@@_array_count_loop:Nw #1 { ? \__prg_break: } ;
      \__prg_break_point:
    \__int_eval_end:
  }
\cs_new:Npn \@@_array_count_loop:Nw #1#2;
  { \use_none:n #1 + \c_one \@@_array_count_loop:Nw }
%    \end{macrocode}
% \end{macro}
% \end{macro}
%
% \subsection{Messages}
%
% Using a floating point directly is an error.
%    \begin{macrocode}
\__msg_kernel_new:nnnn { kernel } { misused-fp }
  { A~floating~point~with~value~'#1'~was~misused. }
  {
    To~obtain~the~value~of~a~floating~point~variable,~use~
    '\token_to_str:N \fp_to_decimal:N',~
    '\token_to_str:N \fp_to_scientific:N',~or~other~
    conversion~functions.
  }
%    \end{macrocode}
%
%    \begin{macrocode}
%</initex|package>
%    \end{macrocode}
%
% \end{implementation}
%
% \PrintChanges
%
% \PrintIndex
