% \iffalse meta-comment
%
%% File: l3bootstrap.dtx Copyright (C) 2011-2017 The LaTeX3 project
%
% It may be distributed and/or modified under the conditions of the
% LaTeX Project Public License (LPPL), either version 1.3c of this
% license or (at your option) any later version.  The latest version
% of this license is in the file
%
%    https://www.latex-project.org/lppl.txt
%
% This file is part of the "l3kernel bundle" (The Work in LPPL)
% and all files in that bundle must be distributed together.
%
% -----------------------------------------------------------------------
%
% The development version of the bundle can be found at
%
%    https://github.com/latex3/latex3
%
% for those people who are interested.
%
%<*driver|package>
% \begin{macro}{\GetIdInfo}
% \begin{macro}{\GetIdInfoAuxI, \GetIdInfoAuxII, \GetIdInfoAuxIII}
%   The idea here is to extract out the information needed from a standard
%   \textsc{svn} \texttt{Id} line, but to avoid a line that would get
%   changed when the file is checked in. Hence the fact that none of the
%   lines here include both a dollar sign and the \texttt{Id} keyword!
%
%   At this stage, no test has taken place for the \eTeX{} extensions, and
%   so using \tn{protected} could give an error. To avoid that, it is used
%   by csname: if it's not available, the bootstrap code will bail out at
%   the point of testing for the necessary primitives, while if it is available
%   then \cs{GetIdInfo} and so on will be properly protected. All of this is
%   then done in a group to avoid leaving \tn{protected} about as equivalent
%   to \tn{relax} if the extensions are unavailable.
%    \begin{macrocode}
\begingroup
  \csname protected\endcsname\gdef\GetIdInfo
    {%
      \begingroup
        \catcode 32 = 10 %
        \GetIdInfoAuxI
    }%
%    \end{macrocode}
%   A first check for a completely empty \textsc{svn} field. If that is
%   not the case, there is a second case when a file created using
%   \texttt{svn cp} but has not been checked in. That leaves a special
%   marker \texttt{-1} version, which has no further data. Dealing correctly
%   with that is the reason for the space in the line to use
%   \cs{GetIdInforAuxII}.
%    \begin{macrocode}
  \csname protected\endcsname\gdef\GetIdInfoAuxI$#1$#2%
    {%
      \def\tempa{#1}%
      \def\tempb{Id}%
      \ifx\tempa\tempb
        \def\tempa
          {%
            \endgroup
            \def\ExplFileDate{0000/00/00}%
            \def\ExplFileDescription{#2}%
            \def\ExplFileName{[unknown]}%
            \def\ExplFileExtension{[unknown extension]}%
            \def\ExplFileVersion{-1}%
          }%
      \else
        \def\tempa
          {%
            \endgroup
            \def\ExplFileDescription{#2}%
            \GetIdInfoAuxII$#1 $%
          }%
      \fi
      \tempa
    }%
%    \end{macrocode}
%   Here, |#1| is |Id|, |#2| is the file name, |#3| is the extension,
%   |#4| is the version, |#5| is the check in date and |#6| is the check in
%   time and user, plus some trailing spaces. If |#4| is the marker |-1| value
%   then |#5| and |#6| are empty.
%    \begin{macrocode}
  \csname protected\endcsname\gdef\GetIdInfoAuxII$#1 #2.#3 #4 #5 #6$%
    {%
      \def\ExplFileName{#2}%
      \def\ExplFileExtension{#3}%
      \def\ExplFileVersion{#4}%
      \begingroup
        \def\tempa{#4}%
        \def\tempb{-1}%
        \ifx\tempa\tempb
          \def\tempa
            {%
              \endgroup
              \def\ExplFileDate{0000/00/00}%
            }%
        \else
          \def\tempa
            {%
              \endgroup
              \GetIdInfoAuxIII$#5$%
            }%
        \fi
        \tempa
    }%
%    \end{macrocode}
%   Convert an \textsc{svn}-style date into a \LaTeX{}-style one.
%    \begin{macrocode}
  \csname protected\endcsname\gdef\GetIdInfoAuxIII$#1-#2-#3$%
    {%
      \def\ExplFileDate{#1/#2/#3}%
    }%
\endgroup
%    \end{macrocode}
% \end{macro}
% \end{macro}
% \end{macro}
%</driver|package>
%<*driver>
\documentclass[full,kernel]{l3doc}
\begin{document}
  \DocInput{\jobname.dtx}
\end{document}
%</driver>
% \fi
%
% \title{^^A
%   The \pkg{l3bootstrap} package\\ Bootstrap code^^A
% }
%
% \author{^^A
%  The \LaTeX3 Project\thanks
%    {^^A
%      E-mail:
%        \href{mailto:latex-team@latex-project.org}
%          {latex-team@latex-project.org}^^A
%    }^^A
% }
%
% \date{Released 2017/12/16}
%
% \maketitle
%
% \begin{documentation}
%
% \section{Using the \LaTeX3 modules}
%
% The modules documented in \file{source3} are designed to be used on top of
% \LaTeXe{} and are loaded all as one with the usual |\usepackage{expl3}| or
% |\RequirePackage{expl3}| instructions. These modules will also form the
% basis of the \LaTeX3 format, but work in this area is incomplete and not
% included in this documentation at present.
%
% As the modules use a coding syntax different from standard
% \LaTeXe{} it provides a few functions for setting it up.
%
% \begin{function}[updated = 2011-08-13]{\ExplSyntaxOn, \ExplSyntaxOff}
%   \begin{syntax}
%     \cs{ExplSyntaxOn} \meta{code} \cs{ExplSyntaxOff}
%   \end{syntax}
%   The \cs{ExplSyntaxOn} function switches to a category code
%   r\'{e}gime in which spaces are ignored and in which the colon (|:|)
%   and underscore (|_|) are treated as \enquote{letters}, thus allowing
%   access to the names of code functions and variables. Within this
%   environment, |~| is used to input a space. The \cs{ExplSyntaxOff}
%   reverts to the document category code r\'{e}gime.
% \end{function}
%
% \begin{function}[updated = 2017-03-19]
%   {\ProvidesExplPackage, \ProvidesExplClass, \ProvidesExplFile}
%   \begin{syntax}
%     |\RequirePackage{expl3}| \\
%     \cs{ProvidesExplPackage} \Arg{package} \Arg{date} \Arg{version} \Arg{description}
%   \end{syntax}
%   These functions act broadly in the same way as the corresponding
%   \LaTeXe{} kernel
%   functions \tn{ProvidesPackage}, \tn{ProvidesClass} and
%   \tn{ProvidesFile}. However, they also implicitly switch
%   \cs{ExplSyntaxOn} for the remainder of the code with the file. At the
%   end of the file, \cs{ExplSyntaxOff} will be called to reverse this.
%   (This is the same concept as \LaTeXe{} provides in turning on
%   \tn{makeatletter} within package and class code.) The \meta{date} should
%   be given in the format \meta{year}/\meta{month}/\meta{day}. If the
%   \meta{version} is given then it will be prefixed with \texttt{v} in
%   the package identifier line.
% \end{function}
%
% \begin{function}[updated = 2012-06-04]{\GetIdInfo}
%   \begin{syntax}
%     |\RequirePackage{l3bootstrap}|
%     \cs{GetIdInfo} |$Id:| \meta{SVN info field} |$| \Arg{description}
%   \end{syntax}
%   Extracts all information from a SVN field. Spaces are not
%   ignored in these fields. The information pieces are stored in
%   separate control sequences with \cs{ExplFileName} for the part of the
%   file name leading up to the period, \cs{ExplFileDate} for date,
%   \cs{ExplFileVersion} for version and \cs{ExplFileDescription} for the
%   description.
% \end{function}
%
% To summarize: Every single package using this syntax should identify
% itself using one of the above methods. Special care is taken so that
% every package or class file loaded with \tn{RequirePackage} or similar
% are loaded with usual \LaTeXe{} category codes and the \LaTeX3 category code
% scheme is reloaded when needed afterwards. See implementation for
% details. If you use the \cs{GetIdInfo} command you can use the
% information when loading a package with
% \begin{verbatim}
%   \ProvidesExplPackage{\ExplFileName}
%     {\ExplFileDate}{\ExplFileVersion}{\ExplFileDescription}
% \end{verbatim}
%
% \subsection{Internal functions and variables}
%
% \begin{variable}{\l__kernel_expl_bool}
%   A boolean which records the current code syntax status: \texttt{true} if
%   currently inside a code environment. This variable should only be
%   set by \cs{ExplSyntaxOn}/\cs{ExplSyntaxOff}.
% \end{variable}
%
% \end{documentation}
%
% \begin{implementation}
%
% \section{\pkg{l3bootstrap} implementation}
%
%    \begin{macrocode}
%<*initex|package>
%<@@=kernel>
%    \end{macrocode}
%
% \subsection{Format-specific code}
%
% The very first thing to do is to bootstrap the \IniTeX{} system so
% that everything else will actually work. \TeX{} does not start with
% some pretty basic character codes set up.
%    \begin{macrocode}
%<*initex>
\catcode `\{ = 1 %
\catcode `\} = 2 %
\catcode `\# = 6 %
\catcode `\^ = 7 %
%</initex>
%    \end{macrocode}
%
% Tab characters should not show up in the code, but to be on the
% safe side.
%    \begin{macrocode}
%<*initex>
\catcode `\^^I = 10 %
%</initex>
%    \end{macrocode}
%
% For \LuaTeX{}, the extra primitives need to be enabled. This is not needed
% in package mode: common formats have the primitives enabled.
%    \begin{macrocode}
%<*initex>
\begingroup\expandafter\expandafter\expandafter\endgroup
\expandafter\ifx\csname directlua\endcsname\relax
\else
  \directlua{tex.enableprimitives("", tex.extraprimitives())}%
\fi
%</initex>
%    \end{macrocode}
%
% Depending on the versions available, the \LaTeX{} format may not have
% the raw |\Umath| primitive names available. We fix that globally:
% it should cause no issues. Older \LuaTeX{} versions do not have
% a pre-built table of the primitive names here so sort one out
% ourselves. These end up globally-defined but at that is true with
% a newer format anyway and as they all start |\U| this should be
% reasonably safe.
%    \begin{macrocode}
%<*package>
\begingroup
  \expandafter\ifx\csname directlua\endcsname\relax
  \else
    \directlua{%
      local i
      local t = { }
      for _,i in pairs(tex.extraprimitives("luatex")) do
        if string.match(i,"^U") then
          if not string.match(i,"^Uchar$") then %$
            table.insert(t,i)
          end
        end
      end
      tex.enableprimitives("", t)
    }%
  \fi
\endgroup
%</package>
%    \end{macrocode}
%
% \subsection{The \tn{pdfstrcmp} primitive in \XeTeX{}}
%
% Only \pdfTeX{} has a primitive called \tn{pdfstrcmp}. The \XeTeX{}
% version is just \tn{strcmp}, so there is some shuffling to do. As
% this is still a real primitive, using the \pdfTeX{} name is \enquote{safe}.
%    \begin{macrocode}
\begingroup\expandafter\expandafter\expandafter\endgroup
  \expandafter\ifx\csname pdfstrcmp\endcsname\relax
  \let\pdfstrcmp\strcmp
\fi
%    \end{macrocode}
%
% \subsection{Loading support \Lua{} code}
%
% When \LuaTeX{} is used there are various pieces of \Lua{} code which need to
% be loaded. The code itself is defined in \pkg{l3luatex} and is extracted into
% a separate file. Thus here the task is to load the \Lua{} code both now and
% (if required) at the start of each job.
%    \begin{macrocode}
\begingroup\expandafter\expandafter\expandafter\endgroup
\expandafter\ifx\csname directlua\endcsname\relax
\else
  \ifnum\luatexversion<70 %
  \else
%    \end{macrocode}
%   In package mode a category code table is needed: either use a pre-loaded
%   allocator or provide one using the \LaTeXe{}-based generic code. In format
%   mode the table used here can be hard-coded into the \Lua{}.
%    \begin{macrocode}
%<*package>
    \begingroup\expandafter\expandafter\expandafter\endgroup
    \expandafter\ifx\csname newcatcodetable\endcsname\relax
      \input{ltluatex}%
    \fi
    \newcatcodetable\ucharcat@table
    \directlua{
      l3kernel = l3kernel or { }
      local charcat_table = \number\ucharcat@table\space
      l3kernel.charcat_table = charcat_table
    }%
%</package>
    \directlua{require("expl3")}%
%    \end{macrocode}
%   As the user might be making a custom format, no assumption is made about
%   matching package mode with only loading the \Lua{} code once. Instead, a
%   query to \Lua{} reveals what mode is in operation.
%    \begin{macrocode}
    \ifnum 0%
      \directlua{
        if status.ini_version then
          tex.write("1")
        end
      }>0 %
      \everyjob\expandafter{%
        \the\expandafter\everyjob
        \csname\detokenize{lua_now_x:n}\endcsname{require("expl3")}%
      }%
    \fi
  \fi
\fi
%    \end{macrocode}
%
% \subsection{Engine requirements}
%
% The code currently requires \eTeX{} and functionality equivalent to
% \tn{pdfstrcmp}, and also driver and Unicode character support. This is
% available in a reasonably-wide range of engines.
%    \begin{macrocode}
\begingroup
  \def\next{\endgroup}%
  \def\ShortText{Required primitives not found}%
  \def\LongText%
    {%
      LaTeX3 requires the e-TeX primitives and additional functionality as
      described in the README file.
      \LineBreak
      These are available in the engines\LineBreak
      - pdfTeX v1.40\LineBreak
      - XeTeX v0.9994\LineBreak
      - LuaTeX v0.70\LineBreak
      - e-(u)pTeX mid-2012\LineBreak
      or later.\LineBreak
      \LineBreak
    }%
  \ifnum0%
    \expandafter\ifx\csname pdfstrcmp\endcsname\relax
    \else
      \expandafter\ifx\csname pdftexversion\endcsname\relax
        1%
      \else
        \ifnum\pdftexversion<140 \else 1\fi
      \fi
    \fi
    \expandafter\ifx\csname directlua\endcsname\relax
    \else
      \ifnum\luatexversion<70 \else 1\fi
    \fi
    =0 %
      \newlinechar`\^^J %
%<*initex>
      \def\LineBreak{^^J}%
      \edef\next
        {%
          \errhelp
            {%
              \LongText
              For pdfTeX and XeTeX the '-etex' command-line switch is also
              needed.\LineBreak
              \LineBreak
              Format building will abort!\LineBreak
            }%
          \errmessage{\ShortText}%
          \endgroup
          \noexpand\end
        }%
%</initex>
%<*package>
      \def\LineBreak{\noexpand\MessageBreak}%
      \expandafter\ifx\csname PackageError\endcsname\relax
        \def\LineBreak{^^J}%
        \def\PackageError#1#2#3%
          {%
            \errhelp{#3}%
            \errmessage{#1 Error: #2}%
          }%
      \fi
      \edef\next
        {%
          \noexpand\PackageError{expl3}{\ShortText}
            {\LongText Loading of expl3 will abort!}%
          \endgroup
          \noexpand\endinput
        }%
%</package>
  \fi
\next
%    \end{macrocode}
%
% \subsection{Extending allocators}
%
% In format mode, allocating registers is handled by \pkg{l3alloc}. However, in
% package mode it's much safer to rely on more general code. For example,
% the ability to extend \TeX{}'s allocation routine to allow for \eTeX{} has
% been around since 1997 in the \pkg{etex} package.
%
% Loading this support is delayed until here as we are now sure that the
% \eTeX{} extensions and \tn{pdfstrcmp} or equivalent are available. Thus
% there is no danger of an \enquote{uncontrolled} error if the engine
% requirements are not met.
%
% For \LaTeXe{} we need to make sure that the extended pool is being used:
% \pkg{expl3} uses a lot of registers. For formats from 2015 onward there is
% nothing to do as this is automatic. For older formats, the \pkg{etex}
% package needs to be loaded to do the job. In that case, some inserts are
% reserved also as these have to be from the standard pool. Note that
% \tn{reserveinserts} is \tn{outer} and so is accessed here by csname. In
% earlier versions, loading \pkg{etex} was done directly and so
% \tn{reserveinserts} appeared in the code: this then required a \tn{relax}
% after \tn{RequirePackage} to prevent an error with \enquote{unsafe}
% definitions as seen for example with \pkg{capoptions}. The optional loading
% here is done using a group and \tn{ifx} test as we are not quite in the
% position to have a single name for \tn{pdfstrcmp} just yet.
%    \begin{macrocode}
%<*package>
\begingroup
  \def\@tempa{LaTeX2e}%
  \def\next{}%
  \ifx\fmtname\@tempa
    \expandafter\ifx\csname extrafloats\endcsname\relax
      \def\next
        {%
          \RequirePackage{etex}%
          \csname reserveinserts\endcsname{32}%
        }%
    \fi
  \fi
\expandafter\endgroup
\next
%</package>
%    \end{macrocode}
%
% \subsection{Character data}
%
% \TeX{} needs various pieces of data to be set about characters, in particular
% which ones to treat as letters and which \tn{lccode} values apply as these
% affect hyphenation. It makes most sense to set this and related information
% up in one place. Whilst for \LuaTeX{} hyphenation patterns can be read
% anywhere, other engines have to build them into the format and so we
% \emph{must} do this set up before reading the patterns. For the Unicode
% engines, there are shared loaders available to obtain the relevant
% information directly from the Unicode Consortium data files. These need
% standard (Ini)\TeX{} category codes and primitive availability and must
% therefore loaded \emph{very} early. This has a knock-on effect on the
% $8$-bit set up: it makes sense to do the definitions for those here as
% well so it is all in one place.
%
% For \XeTeX{} and \LuaTeX{}, which are natively Unicode engines, simply
% load the Unicode data.
%    \begin{macrocode}
%<*initex>
\ifdefined\Umathcode
  \input load-unicode-data %
  \input load-unicode-math-classes %
\else
%    \end{macrocode}
% For the $8$-bit engines a font encoding scheme must be chosen. At present,
% this is the EC (|T1|) scheme, with the assumption that languages for which
% this is not appropriate will be used with one of the Unicode engines.
%    \begin{macrocode}
  \begingroup
%    \end{macrocode}
% Lower case chars: map to themselves when lower casing and down by |"20| when
% upper casing. (The characters |a|--|z| are set up correctly by Ini\TeX{}.)
%    \begin{macrocode}
    \def\temp{%
      \ifnum\count0>\count2 %
      \else
        \global\lccode\count0 = \count0 %
        \global\uccode\count0 = \numexpr\count0 - "20\relax
        \advance\count0 by 1 %
        \expandafter\temp
      \fi
    }
    \count0 = "A0 %
    \count2 = "BC %
    \temp
    \count0 = "E0 %
    \count2 = "FF %
    \temp
%    \end{macrocode}
% Upper case chars: map up by |"20| when lower casing, to themselves when upper
% casing and require an \tn{sfcode} of $999$. (The characters |A|--|Z| are set
% up correctly by Ini\TeX{}.)
%    \begin{macrocode}
    \def\temp{%
      \ifnum\count0>\count2 %
      \else
        \global\lccode\count0 = \numexpr\count0 + "20\relax
        \global\uccode\count0 = \count0 %
        \global\sfcode\count0 = 999 %
        \advance\count0 by 1 %
        \expandafter\temp
      \fi
    }
    \count0 = "80 %
    \count2 = "9C %
    \temp
    \count0 = "C0 %
    \count2 = "DF %
    \temp
%    \end{macrocode}
% A few special cases where things are not as one might expect using the above
% pattern: dotless-I, dotless-J, dotted-I and d-bar.
%    \begin{macrocode}
    \global\lccode`\^^Y = `\^^Y %
    \global\uccode`\^^Y = `\I %
    \global\lccode`\^^Z = `\^^Z %
    \global\uccode`\^^Y = `\J %
    \global\lccode"9D = `\i %
    \global\uccode"9D = "9D %
    \global\lccode"9E = "9E %
    \global\uccode"9E = "D0 %
%    \end{macrocode}
% Allow hyphenation at a zero-width glyph (used to break up ligatures or
% to place accents between characters).
%    \begin{macrocode}
    \global\lccode23 = 23 %
  \endgroup
\fi
%    \end{macrocode}
%</initex>
%    \end{macrocode}
%
% \subsection{The \LaTeX3 code environment}
%
% The code environment is now set up.
%
% \begin{macro}{\ExplSyntaxOff}
%   Before changing any category codes, in package mode we need to save
%   the situation before loading. Note the set up here means that once applied
%   \cs{ExplSyntaxOff} becomes a \enquote{do nothing} command until
%   \cs{ExplSyntaxOn} is used. For format mode, there is no need to save
%   category codes so that step is skipped.
%    \begin{macrocode}
\protected\def\ExplSyntaxOff{}%
%<*package>
\protected\edef\ExplSyntaxOff
  {%
    \protected\def\ExplSyntaxOff{}%
    \catcode   9 = \the\catcode   9\relax
    \catcode  32 = \the\catcode  32\relax
    \catcode  34 = \the\catcode  34\relax
    \catcode  38 = \the\catcode  38\relax
    \catcode  58 = \the\catcode  58\relax
    \catcode  94 = \the\catcode  94\relax
    \catcode  95 = \the\catcode  95\relax
    \catcode 124 = \the\catcode 124\relax
    \catcode 126 = \the\catcode 126\relax
    \endlinechar = \the\endlinechar\relax
    \chardef\csname\detokenize{l_@@_expl_bool}\endcsname = 0\relax
  }%
%</package>
%    \end{macrocode}
% \end{macro}
%
% The code environment is now set up.
%    \begin{macrocode}
\catcode 9   = 9\relax
\catcode 32  = 9\relax
\catcode 34  = 12\relax
\catcode 38 =  4\relax
\catcode 58  = 11\relax
\catcode 94  = 7\relax
\catcode 95  = 11\relax
\catcode 124 = 12\relax
\catcode 126 = 10\relax
\endlinechar = 32\relax
%    \end{macrocode}
%
% \begin{variable}{\l_@@_expl_bool}
%   The status for experimental code syntax: this is on at present.
%    \begin{macrocode}
\chardef\l_@@_expl_bool = 1\relax
%    \end{macrocode}
%\end{variable}
%
% \begin{macro}{\ExplSyntaxOn}
%  The idea here is that multiple \cs{ExplSyntaxOn} calls are not
%  going to mess up category codes, and that multiple calls to
%  \cs{ExplSyntaxOff} are also not wasting time. Applying
%  \cs{ExplSyntaxOn} alters the definition of \cs{ExplSyntaxOff}
%  and so in package mode this function should not be used until after
%  the end of the loading process!
%    \begin{macrocode}
\protected \def \ExplSyntaxOn
  {
    \bool_if:NF \l_@@_expl_bool
      {
        \cs_set_protected:Npx \ExplSyntaxOff
          {
            \char_set_catcode:nn { 9 }   { \char_value_catcode:n { 9 } }
            \char_set_catcode:nn { 32 }  { \char_value_catcode:n { 32 } }
            \char_set_catcode:nn { 34 }  { \char_value_catcode:n { 34 } }
            \char_set_catcode:nn { 38 }  { \char_value_catcode:n { 38 } }
            \char_set_catcode:nn { 58 }  { \char_value_catcode:n { 58 } }
            \char_set_catcode:nn { 94 }  { \char_value_catcode:n { 94 } }
            \char_set_catcode:nn { 95 }  { \char_value_catcode:n { 95 } }
            \char_set_catcode:nn { 124 } { \char_value_catcode:n { 124 } }
            \char_set_catcode:nn { 126 } { \char_value_catcode:n { 126 } }
            \tex_endlinechar:D =
              \tex_the:D \tex_endlinechar:D \scan_stop:
            \bool_set_false:N \l_@@_expl_bool
            \cs_set_protected:Npn \ExplSyntaxOff { }
          }
      }
    \char_set_catcode_ignore:n           { 9 }   % tab
    \char_set_catcode_ignore:n           { 32 }  % space
    \char_set_catcode_other:n            { 34 }  % double quote
    \char_set_catcode_alignment:n        { 38 }  % ampersand
    \char_set_catcode_letter:n           { 58 }  % colon
    \char_set_catcode_math_superscript:n { 94 }  % circumflex
    \char_set_catcode_letter:n           { 95 }  % underscore
    \char_set_catcode_other:n            { 124 } % pipe
    \char_set_catcode_space:n            { 126 } % tilde
    \tex_endlinechar:D = 32 \scan_stop:
    \bool_set_true:N \l_@@_expl_bool
  }
%    \end{macrocode}
% \end{macro}
%
%    \begin{macrocode}
%</initex|package>
%    \end{macrocode}
%
% \end{implementation}
%
% \PrintIndex
