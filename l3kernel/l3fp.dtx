% \iffalse meta-comment
%
%% File: l3fp.dtx Copyright (C) 2010-2011 by The LaTeX3 Project
%%
%% It may be distributed and/or modified under the conditions of the
%% LaTeX Project Public License (LPPL), either version 1.3c of this
%% license or (at your option) any later version.  The latest version
%% of this license is in the file
%%
%%    http://www.latex-project.org/lppl.txt
%%
%% This file is part of the "expl3 bundle" (The Work in LPPL)
%% and all files in that bundle must be distributed together.
%%
%% The released version of this bundle is available from CTAN.
%%
%% -----------------------------------------------------------------------
%%
%% The development version of the bundle can be found at
%%
%%    http://www.latex-project.org/svnroot/experimental/trunk/
%%
%% for those people who are interested.
%%
%%%%%%%%%%%
%% NOTE: %%
%%%%%%%%%%%
%%
%%   Snapshots taken from the repository represent work in progress and may
%%   not work or may contain conflicting material!  We therefore ask
%%   people _not_ to put them into distributions, archives, etc. without
%%   prior consultation with the LaTeX3 Project.
%%
%% -----------------------------------------------------------------------
%
%<*driver|package>
\RequirePackage{l3names}
\GetIdInfo$Id$
  {L3 Experimental floating-point operations}
%</driver|package>
%<*driver>
\documentclass[full]{l3doc}
\begin{document}
  \DocInput{\jobname.dtx}
\end{document}
%</driver>
% \fi
%
% \title{^^A
%   The \pkg{l3fp} package: Floating-point operations
%   \thanks{This file describes v\fileversion, last revised \filedate.}^^A
% }
%
% \author{^^A
%  The \LaTeX3 Project\thanks
%    {^^A
%      E-mail:
%        \href{mailto:latex-team@latex-project.org}
%          {latex-team@latex-project.org}^^A
%    }^^A
% }
%
% \date{Released \filedate}
%
% \maketitle
%
% \begin{documentation}
%
% A floating point number is one which is stored as a mantissa and
% a separate exponent. This module implements arithmetic using radix
% $10$ floating point numbers. This means that the mantissa should
% be a real number in the range $1 \le \expandafter\mathopen\string|
% x \expandafter\mathclose\string| < 10$, with the
% exponent given as an integer between $-99$ and $99$. In the
% input, the exponent part is represented starting with an \texttt{e}.
% As this is a low-level module, error-checking is minimal. Numbers
% which are too large for the floating point unit to handle will result
% in errors, either from \TeX{} or from \LaTeX{}. The \LaTeX{} code does not
% check that the input will not overflow, hence the possibility of a
% \TeX{} error. On the other hand, numbers which are too small will be
% dropped, which will mean that extra decimal digits will simply be
% lost.
%
% When parsing numbers, any missing parts will be interpreted as
% zero. So for example
%\begin{verbatim}
%  \fp_set:Nn \l_my_fp { }
%  \fp_set:Nn \l_my_fp { . }
%  \fp_set:Nn \l_my_fp { - }
% \end{verbatim}
% will all be interpreted as zero values without raising an error.
%
% Operations which give an undefined result (such as division by
% $0$) will not lead to errors. Instead special marker values are
% returned, which can be tested for using fr example
% \cs{fp_if_undefined:N(TF)}. In this way it is possible to work with
% asymptotic functions without first checking the input. If these
% special values are carried forward in calculations they will be
% treated as $0$.
%
% Floating point numbers are stored in the \texttt{fp} floating point
% variable type. This has a standard range of functions for
% variable management.
%
% \subsection{Floating-point variables}
%
% \begin{function}{\fp_new:N, \fp_new:c}
%   \begin{syntax}
%     \cs{fp_new:N} \meta{floating point variable}
%   \end{syntax}
%   Creates a new \meta{floating point variable} or raises an error if
%   the name is already taken. The declaration global. The
%   \meta{floating point} will initially be set to  |+0.000000000e0|
%   (the zero floating point).
% \end{function}
%
% \begin{function}{\fp_const:Nn, \fp_const:cn}
%   \begin{syntax}
%     \cs{fp_const:Nn} \meta{floating point variable} \Arg{value}
%   \end{syntax}
%   Creates a new constant \meta{floating point variable} or raises an
%   error  if the name is already taken. The value of the
%   \meta{floating point variable} will be set globally to the
%   \meta{value}.
% \end{function}
%
% \begin{function}{\fp_set_eq:NN, \fp_set_eq:cN, \fp_set_eq:Nc, \fp_set_eq:cc}
%   \begin{syntax}
%     \cs{fp_set_eq:NN} \meta{fp var1} \meta{fp var2}
%   \end{syntax}
%   Sets the value of \meta{floating point variable1} equal to that of
%   \meta{floating point variable2}. This assignment is restricted to the
%   current \TeX{} group level.
% \end{function}
%
% \begin{function}
%   {\fp_gset_eq:NN, \fp_gset_eq:cN, \fp_gset_eq:Nc, \fp_gset_eq:cc}
%   \begin{syntax}
%     \cs{fp_gset_eq:NN} \meta{fp var1} \meta{fp var2}
%   \end{syntax}
%   Sets the value of \meta{floating point variable1} equal to that of
%   \meta{floating point variable2}. This assignment is global and so is
%   not limited by the current \TeX{} group level.
% \end{function}
%
% \begin{function}{\fp_zero:N, \fp_zero:c}
%   \begin{syntax}
%     \cs{fp_zero:N} \meta{floating point variable}
%   \end{syntax}
%   Sets the \meta{floating point variable} to |+0.000000000e0| within
%   the current scope.
% \end{function}
%
% \begin{function}{\fp_gzero:N, \fp_gzero:c}
%   \begin{syntax}
%     \cs{fp_gzero:N} \meta{floating point variable}
%   \end{syntax}
%   Sets the \meta{floating point variable} to |+0.000000000e0| globally.
% \end{function}
%
% \begin{function}{\fp_set:Nn, \fp_set:cn}
%   \begin{syntax}
%     \cs{fp_set:Nn} \meta{floating point variable} \Arg{value}
%   \end{syntax}
%   Sets the \meta{floating point variable} variable to \meta{value}
%   within the scope of the current \TeX{} group.
% \end{function}
%
% \begin{function}{\fp_gset:Nn, \fp_gset:cn}
%   \begin{syntax}
%     \cs{fp_gset:Nn} \meta{floating point variable} \Arg{value}
%   \end{syntax}
%   Sets the \meta{floating point variable} variable to \meta{value}
%   globally.
% \end{function}
%
% \begin{function}{\fp_set_from_dim:Nn, \fp_set_from_dim:cn}
%   \begin{syntax}
%     \cs{fp_set_from_dim:Nn} \meta{floating point variable} \Arg{dimexpr}
%   \end{syntax}
%   Sets the \meta{floating point variable} to the distance represented
%   by the \meta{dimension expression} in the units points. This means
%   that distances given in other units are first converted to points
%   before being assigned to the \meta{floating point variable}. The
%   assignment is local.
% \end{function}
%
% \begin{function}{\fp_gset_from_dim:Nn, \fp_gset_from_dim:cn}
%   \begin{syntax}
%     \cs{fp_gset_from_dim:Nn} \meta{floating point variable} \Arg{dimexpr}
%   \end{syntax}
%   Sets the \meta{floating point variable} to the distance represented
%   by the \meta{dimension expression} in the units points. This means
%   that distances given in other units are first converted to points
%   before being assigned to the \meta{floating point variable}. The
%   assignment is global.
% \end{function}
%
% \begin{function}[EXP]{\fp_use:N, \fp_use:c}
%   \begin{syntax}
%     \cs{fp_use:N} \meta{floating point variable}
%   \end{syntax}
%   Inserts the value of the \meta{floating point variable} into the
%   input stream. The value will be given as a real number without any
%   exponent part, and will always include a decimal point. For example,
%   \begin{verbatim}
%     \fp_new:Nn \test
%     \fp_set:Nn \test { 1.234 e 5 }
%     \fp_use:N \test
%   \end{verbatim}
%   will insert |12345.00000| into the input stream.
%   As illustrated, a floating point will always be inserted with ten
%   significant digits given. Very large and very small values will
%   include additional zeros for place value.
% \end{function}
%
% \begin{function}{\fp_show:N, \fp_show:c}
%   \begin{syntax}
%     \cs{fp_show:N} \meta{floating point variable}
%   \end{syntax}
%   Displays the content of the \meta{floating point variable} on the
%   terminal.
% \end{function}
%
% \section{Conversion of floating point values to other formats}
%
% It is useful to be able to convert floating point variables to
% other forms. These functions are expandable, so that the material
% can be used in a variety of contexts. The \cs{fp_use:N} function
% should also be consulted in this context, as it will insert the
% value of the floating point variable as a real number.
%
% \begin{function}[EXP]{\fp_to_dim:N, \fp_to_dim:c}
%   \begin{syntax}
%     \cs{fp_to_dim:N} \meta{floating point variable}
%   \end{syntax}
%   Inserts the value of the \meta{floating point variable}
%   into the input stream converted into a dimension in points.
% \end{function}
%
% \begin{function}[EXP]{\fp_to_int:N, \fp_to_int:c}
%   \begin{syntax}
%     \cs{fp_to_int:N} \meta{floating point variable}
%   \end{syntax}
%   Inserts the integer value of the \meta{floating point variable}
%   into the input stream. The decimal part of the number will not be
%   included, but will be used to round the integer.
% \end{function}
%
% \begin{function}[EXP]{\fp_to_tl:N, \fp_to_tl:c}
%   \begin{syntax}
%     \cs{fp_to_tl:N} \meta{floating point variable}
%   \end{syntax}
%   Inserts a representation of the \meta{floating point variable} into
%   the input stream as a token list. The representation follows the
%   conventions of a pocket calculator:
%   \begin{center}
%     \ttfamily
%     \begin{tabular}{r@{.}lr@{.}l}
%       \toprule
%         \multicolumn{2}{l}{\rmfamily{Floating point value}} &
%         \multicolumn{2}{l}{\rmfamily{Representation}} \\
%       \midrule
%          1 & 234000000000e0  &  1 & 234    \\
%         -1 & 234000000000e0  & -1 & 234    \\
%          1 & 234000000000e3  &  \multicolumn{2}{l}{1234} \\
%          1 & 234000000000e13 &  \multicolumn{2}{l}{1234e13} \\
%          1 & 234000000000e-1 & 0 & 1234   \\
%          1 & 234000000000e-2 & 0 & 01234  \\
%          1 & 234000000000e-3 & 1 & 234e-3 \\
%       \bottomrule
%     \end{tabular}
%   \end{center}
%   Notice that trailing zeros are removed in this process, and that
%   numbers which do not require a decimal part do \emph{not} include
%   a decimal marker.
% \end{function}
%
% \section{Rounding floating point values}
%
% The module can round floating point values to either decimal places
% or significant figures using the usual method in which exact halves
% are rounded up.
%
% \begin{function}{\fp_round_figures:Nn, \fp_round_figures:cn}
%   \begin{syntax}
%     \cs{fp_round_figures:Nn} \meta{floating point variable} \Arg{target}
%   \end{syntax}
%   Rounds the \meta{floating point variable} to the \meta{target} number
%   of significant figures (an integer expression). The rounding is
%   carried out locally.
% \end{function}
%
% \begin{function}{\fp_ground_figures:Nn, \fp_ground_figures:cn}
%   \begin{syntax}
%     \cs{fp_ground_figures:Nn} \meta{floating point variable} \Arg{target}
%   \end{syntax}
%   Rounds the \meta{floating point variable} to the \meta{target} number
%   of significant figures (an integer expression). The rounding is
%   carried out globally.
% \end{function}
%
% \begin{function}{\fp_round_places:Nn, \fp_round_places:cn}
%   \begin{syntax}
%     \cs{fp_round_places:Nn} \meta{floating point variable} \Arg{target}
%   \end{syntax}
%   Rounds the \meta{floating point variable} to the \meta{target} number
%   of decimal places (an integer expression). The rounding is
%   carried out locally.
% \end{function}
%
% \begin{function}{\fp_ground_places:Nn, \fp_ground_places:cn}
%   \begin{syntax}
%     \cs{fp_ground_places:Nn} \meta{floating point variable} \Arg{target}
%   \end{syntax}
%   Rounds the \meta{floating point variable} to the \meta{target} number
%   of decimal places (an integer expression). The rounding is
%   carried out globally.
% \end{function}
%
% \section{Floating-point conditionals}
%
% \begin{function}[EXP,pTF]{\fp_if_undefined:N}
%   \begin{syntax}
%     \cs{fp_if_undefined_p:N} \meta{fixed-point}
%     \cs{fp_if_undefined:NTF} \meta{fixed-point}
%     ~~\Arg{true code} \Arg{false code}
%   \end{syntax}
%   Tests if \meta{floating point} is undefined (\emph{i.e.}~equal to the
%   special \cs{c_undefined_fp} variable). The branching versions then
%   leave either \meta{true code} or \meta{false code} in the input
%   stream, as appropriate to the truth of the test and the variant of
%   the function chosen. The logical truth of the test is left in the
%   input stream by the predicate version.
% \end{function}
%
% \begin{function}[EXP]{\fp_if_zero:N}
%   \begin{syntax}
%     \cs{fp_if_zero_p:N} \meta{fixed-point}
%     \cs{fp_if_zero:NTF} \meta{fixed-point} \Arg{true code} \Arg{false code}
%   \end{syntax}
%   Tests if \meta{floating point} is equal to zero (\emph{i.e.}~equal to
%   the special \cs{c_zero_fp} variable). The branching versions then
%   leave either \meta{true code} or \meta{false code} in the input
%   stream, as appropriate to the truth of the test and the variant of
%   the function chosen. The logical truth of the test is left in the
%   input stream by the predicate version.
% \end{function}
%
% \begin{function}[TF]{\fp_compare:nNn}
%   \begin{syntax}
%     \cs{fp_compare:nNnTF}
%     ~~\Arg{floating point1} \meta{relation} \Arg{floating point2}
%     ~~\Arg{true code} \Arg{false code}
%   \end{syntax}
%   This function compared the two \meta{floating point} values, which
%   may be stored as \texttt{fp} variables, using the \meta{relation}:
%   \begin{center}
%     \begin{tabular}{ll}
%       Equal                 & |=| \\
%       Greater than          & |>| \\
%       Less than             & |<| \\
%     \end{tabular}
%   \end{center}
%   Either \meta{true code} or \meta{false code} is then left in the
%   input stream, as appropriate to the truth of the test and the variant
%   of the function chosen. The tests treat undefined floating points as
%   zero as the comparison is intended for real numbers only.
% \end{function}
% 
% \begin{function}[TF]{\fp_compare:n}
%   \begin{syntax}
%     \cs{fp_compare:nTF} 
%     ~~\{ \meta{floating point1} \meta{relation} \meta{floating point2} \}
%     ~~\Arg{true code} \Arg{false code}
%   \end{syntax}
%   This function compared the two \meta{floating point} values, which
%   may be stored as \texttt{fp} variables, using the \meta{relation}:
%   \begin{center}
%     \begin{tabular}{ll}
%       Equal                 & |=| or |==| \\
%       Greater than          & |>|  \\
%       Greater than or equal & |>=| \\
%       Less than             & |<|  \\
%       Less than or equal    & |<=| \\
%       Not equal             & |!=| \\
%     \end{tabular}
%   \end{center}
%   Either \meta{true code} or \meta{false code} is then left in the 
%   input stream, as appropriate to the truth of the test and the variant
%   of the function chosen. The tests treat undefined floating points as
%   zero as the comparison is intended for real numbers only.
% \end{function}
%
% \section{Unary floating-point operations}
%
% The unary operations alter the value stored within an \texttt{fp}
% variable.
%
% \begin{function}{\fp_abs:N, \fp_abs:c}
%   \begin{syntax}
%     \cs{fp_abs:N} \meta{floating point variable}
%   \end{syntax}
%   Converts the \meta{floating point variable} to its absolute value,
%   assigning the result within the current \TeX\ group.
% \end{function}
%
% \begin{function}{\fp_gabs:N, \fp_gabs:c}
%   \begin{syntax}
%     \cs{fp_gabs:N} \meta{floating point variable}
%   \end{syntax}
%   Converts the \meta{floating point variable} to its absolute value,
%   assigning the result globally.
% \end{function}
%
% \begin{function}{\fp_neg:N, \fp_neg:c}
%   \begin{syntax}
%     \cs{fp_neg:N} \meta{floating point variable}
%   \end{syntax}
%   Reverse the sign of the \meta{floating point variable}, assigning the
%   result within the current \TeX\ group.
% \end{function}
%
% \begin{function}{\fp_gneg:N, \fp_gneg:c}
%   \begin{syntax}
%     \cs{fp_gneg:N} \meta{floating point variable}
%   \end{syntax}
%   Reverse the sign of the \meta{floating point variable}, assigning the
%   result globally.
% \end{function}
%
% \section{Floating-point arithmetic}
%
% Binary arithmetic operations act on the value stored in an
% \texttt{fp}, so for example
% \begin{verbatim}
%   \fp_set:Nn \l_my_fp { 1.234 }
%   \fp_sub:Nn \l_my_fp { 5.678 }
% \end{verbatim}
%  sets \cs{l_my_fp} to the result of $1.234 - 5.678$
%  (\emph{i.e.}~$-4.444$).
%
% \begin{function}{\fp_add:Nn, \fp_add:cn}
%   \begin{syntax}
%     \cs{fp_add:Nn} \meta{floating point} \Arg{value}
%   \end{syntax}
%   Adds the \meta{value} to the \meta{floating point}, making the
%   assignment within the current \TeX{} group level.
% \end{function}
%
% \begin{function}{\fp_gadd:Nn, \fp_gadd:cn}
%   \begin{syntax}
%     \cs{fp_gadd:Nn} \meta{floating point} \Arg{value}
%   \end{syntax}
%   Adds the \meta{value} to the \meta{floating point}, making the
%   assignment globally.
% \end{function}
%
% \begin{function}{\fp_sub:Nn, \fp_sub:cn}
%   \begin{syntax}
%     \cs{fp_sub:Nn} \meta{floating point} \Arg{value}
%   \end{syntax}
%   Subtracts the \meta{value} from the \meta{floating point}, making the
%   assignment within the current \TeX{} group level.
% \end{function}
%
% \begin{function}{\fp_gsub:Nn, \fp_gsub:cn}
%   \begin{syntax}
%     \cs{fp_gsub:Nn} \meta{floating point} \Arg{value}
%   \end{syntax}
%   Subtracts the \meta{value} from the \meta{floating point}, making the
%   assignment globally.
% \end{function}
%
% \begin{function}{\fp_mul:Nn, \fp_mul:cn}
%   \begin{syntax}
%     \cs{fp_mul:Nn} \meta{floating point} \Arg{value}
%   \end{syntax}
%   Multiples the \meta{floating point} by the \meta{value}, making the
%   assignment within the current \TeX{} group level.
% \end{function}
%
% \begin{function}{\fp_gmul:Nn, \fp_gmul:cn}
%   \begin{syntax}
%     \cs{fp_gmul:Nn} \meta{floating point} \Arg{value}
%   \end{syntax}
%   Multiples the \meta{floating point} by the \meta{value}, making the
%   assignment globally.
% \end{function}
%
% \begin{function}{\fp_div:Nn, \fp_div:cn}
%   \begin{syntax}
%     \cs{fp_div:Nn} \meta{floating point} \Arg{value}
%   \end{syntax}
%   Divides the \meta{floating point} by the \meta{value}, making the
%   assignment within the current \TeX{} group level. If the \meta{value}
%   is zero, the \meta{floating point} will be set to
%   \cs{c_undefined_fp}. The assignment is local.
% \end{function}
%
% \begin{function}{\fp_gdiv:Nn, \fp_gdiv:cn}
%   \begin{syntax}
%     \cs{fp_gdiv:Nn} \meta{floating point} \Arg{value}
%   \end{syntax}
%   Divides the \meta{floating point} by the \meta{value}, making the
%   assignment globally. If the \meta{value} is zero, the
%   \meta{floating point} will be set to \cs{c_undefined_fp}.
%   The assignment is global.
% \end{function}
%
% \section{Floating-point power operations}
%
% \begin{function}{\fp_pow:Nn, \fp_pow:cn}
%   \begin{syntax}
%     \cs{fp_pow:Nn} \meta{floating point} \Arg{value}
%   \end{syntax}
%   Raises the \meta{floating point} to the given \meta{value}. If the
%   \meta{floating point} is negative, then the \meta{value} should be
%   either a positive real number or a negative integer. If the
%   \meta{floating point} is positive, then the \meta{value} may be any
%   real value. Mathematically invalid operations such as $0^{0}$
%   will give set the \meta{floating point} to  to \cs{c_undefined_fp}.
%   The assignment is local.
% \end{function}
%
% \begin{function}{\fp_gpow:Nn, \fp_gpow:cn}
%   \begin{syntax}
%     \cs{fp_gpow:Nn} \meta{floating point} \Arg{value}
%   \end{syntax}
%   Raises the \meta{floating point} to the given \meta{value}. If the
%   \meta{floating point} is negative, then the \meta{value} should be
%   either a positive real number or a negative integer. If the
%   \meta{floating point} is positive, then the \meta{value} may be any
%   real value. Mathematically invalid operations such as $0^{0}$
%   will give set the \meta{floating point} to  to \cs{c_undefined_fp}.
%   The assignment is global.
% \end{function}
%
% \subsection{Exponential and logarithm functions}
%
% \begin{function}{\fp_exp:Nn, \fp_exp:cn}
%   \begin{syntax}
%     \cs{fp_exp:Nn} \meta{floating point} \Arg{value}
%   \end{syntax}
%   Calculates the exponential of the \meta{value} and assigns this
%   to the \meta{floating point}. The assignment is local.
% \end{function}
%
% \begin{function}{\fp_gexp:Nn, \fp_gexp:cn}
%   \begin{syntax}
%     \cs{fp_gexp:Nn} \meta{floating point} \Arg{value}
%   \end{syntax}
%   Calculates the exponential of the \meta{value} and assigns this
%   to the \meta{floating point}. The assignment is global.
% \end{function}
%
% \begin{function}{\fp_ln:Nn, \fp_ln:cn}
%   \begin{syntax}
%     \cs{fp_ln:Nn} \meta{floating point} \Arg{value}
%   \end{syntax}
%   Calculates the natural logarithm of the \meta{value} and assigns
%   this to the \meta{floating point}. The assignment is local.
% \end{function}
%
% \begin{function}{\fp_gln:Nn, \fp_gln:cn}
%   \begin{syntax}
%     \cs{fp_gln:Nn} \meta{floating point} \Arg{value}
%   \end{syntax}
%   Calculates the natural logarithm of the \meta{value} and assigns
%   this to the \meta{floating point}. The assignment is global.
% \end{function}
%
% \section{Trigonometric functions}
%
% The trigonometric functions all work in radians. They accept a maximum
% input value of $100\,000\,000$, as there are issues with range
% reduction and very large input values.
%
% \begin{function}{\fp_sin:Nn, \fp_sin:cn}
%   \begin{syntax}
%     \cs{fp_sin:Nn} \meta{floating point} \Arg{value}
%   \end{syntax}
%   Assigns the sine of the \meta{value} to the \meta{floating point}.
%   The \meta{value} should be given in radians. The assignment is
%   local.
% \end{function}
%
% \begin{function}{\fp_gsin:Nn, \fp_gsin:cn}
%   \begin{syntax}
%     \cs{fp_gsin:Nn} \meta{floating point} \Arg{value}
%   \end{syntax}
%   Assigns the sine of the \meta{value} to the \meta{floating point}.
%   The \meta{value} should be given in radians. The assignment is
%   global.
% \end{function}
%
% \begin{function}{\fp_cos:Nn, \fp_cos:cn}
%   \begin{syntax}
%     \cs{fp_cos:Nn} \meta{floating point} \Arg{value}
%   \end{syntax}
%   Assigns the cosine of the \meta{value} to the \meta{floating point}.
%   The \meta{value} should be given in radians. The assignment is
%   local.
% \end{function}
%
% \begin{function}{\fp_gcos:Nn, \fp_gcos:cn}
%   \begin{syntax}
%     \cs{fp_gcos:Nn} \meta{floating point} \Arg{value}
%   \end{syntax}
%   Assigns the cosine of the \meta{value} to the \meta{floating point}.
%   The \meta{value} should be given in radians. The assignment is
%   global.
% \end{function}
%
% \begin{function}{\fp_tan:Nn, \fp_tan:cn}
%   \begin{syntax}
%     \cs{fp_tan:Nn} \meta{floating point} \Arg{value}
%   \end{syntax}
%   Assigns the tangent of the \meta{value} to the \meta{floating point}.
%   The \meta{value} should be given in radians. The assignment is
%   local.
% \end{function}
%
% \begin{function}{\fp_gtan:Nn, \fp_gtan:cn}
%   \begin{syntax}
%     \cs{fp_gtan:Nn} \meta{floating point} \Arg{value}
%   \end{syntax}
%   Assigns the tangent of the \meta{value} to the \meta{floating point}.
%   The \meta{value} should be given in radians. The assignment is
%   global.
% \end{function}
%
% \section{Constant floating point values}
%
% \begin{variable}{\c_e_fp}
%   The value of the base of natural numbers, $\mathrm{e}$.
% \end{variable}
%
% \begin{variable}{\c_one_fp}
%   A floating point variable with permanent value $1$: used for
%   speeding up some comparisons.
% \end{variable}
%
% \begin{variable}{\c_pi_fp}
%   The value of $\pi$.
% \end{variable}
%
% \begin{variable}{\c_undefined_fp}
%   A special marker floating point variable representing the result of
%   an operation which does not give a defined result (such as division
%   by $0$).
% \end{variable}
%
% \begin{variable}{\c_zero_fp}
%   A permanently zero floating point variable.
% \end{variable}
%
% \section{Notes on the floating point unit}
%
% As calculation of the elemental transcendental functions is
% computationally expensive compared to storage of results, after
% calculating a trigonometric function, exponent, \emph{etc.}~the module
% stored the result for reuse. Thus the performance of the module for
% repeated operations, most probably trigonometric functions, should be
% much higher than if the values were re-calculated every time they
% were needed.
%
% Anyone with experience of programming floating point calculations will
% know that this is a complex area. The aim of the unit is to be
% accurate enough for the likely applications in a typesetting context.
% The arithmetic operations are therefore intended to provide ten digit
% accuracy with the last digit accurate to $\pm 1$. The elemental
% transcendental functions may not provide such high accuracy in every
% case, although the design aim has been to provide $10$ digit
% accuracy for cases likely to be relevant in typesetting situations.
% A good overview of the challenges in this area can be found in
% J.-M.~Muller, \emph{Elementary functions: algorithms and
% implementation}, 2nd edition, Birkh{\"a}uer Boston, New York, USA,
% 2006.
%
% The internal representation of numbers is tuned to the needs of the
% underlying \TeX{} system. This means that the format is somewhat
% different from that used in, for example, computer floating point
% units. Programming in \TeX{} makes it most convenient to use a
% radix $10$ system, using \TeX{} \texttt{count} registers for
% storage and taking advantage where possible of delimited arguments.
%
% \end{documentation}
%
% \begin{implementation}
%
% \section{\pkg{l3fp} Implementation}
%
% \TestFiles{m3fp003.lvt}
%
%    \begin{macrocode}
%<*initex|package>
%    \end{macrocode}
%
%    \begin{macrocode}
%<*package>
\ProvidesExplPackage
  {\filename}{\filedate}{\fileversion}{\filedescription}
\package_check_loaded_expl:
%</package>
%    \end{macrocode}
%
% \subsection{Constants}
%
% \begin{variable}{\c_forty_four}
% \begin{variable}{\c_one_million}
% \begin{variable}{\c_one_hundred_million}
% \begin{variable}{\c_five_hundred_million}
% \begin{variable}{\c_one_thousand_million}
%   There is some speed to gain by moving numbers into fixed positions.
%    \begin{macrocode}
\int_const:Nn \c_forty_four { 44 }
\int_const:Nn \c_one_million { 1 000 000 }
\int_const:Nn \c_one_hundred_million { 100 000 000 }
\int_const:Nn \c_five_hundred_million { 500 000 000 }
\int_const:Nn \c_one_thousand_million { 1 000 000 000 }
%    \end{macrocode}
% \end{variable}
% \end{variable}
% \end{variable}
% \end{variable}
% \end{variable}
%
% \begin{variable}{\c_fp_pi_by_four_decimal_int}
% \begin{variable}{\c_fp_pi_by_four_extended_int}
% \begin{variable}{\c_fp_pi_decimal_int}
% \begin{variable}{\c_fp_pi_extended_int}
% \begin{variable}{\c_fp_two_pi_decimal_int}
% \begin{variable}{\c_fp_two_pi_extended_int}
%   Parts of $\pi$ for trigonometric range reduction, implemented
%   as \texttt{int} variables for speed.
%    \begin{macrocode}
\int_new:N  \c_fp_pi_by_four_decimal_int
\int_set:Nn \c_fp_pi_by_four_decimal_int { 785 398 158 }
\int_new:N  \c_fp_pi_by_four_extended_int
\int_set:Nn \c_fp_pi_by_four_extended_int { 897 448 310 }
\int_new:N  \c_fp_pi_decimal_int
\int_set:Nn \c_fp_pi_decimal_int { 141 592 653 }
\int_new:N  \c_fp_pi_extended_int
\int_set:Nn \c_fp_pi_extended_int { 589 793 238 }
\int_new:N  \c_fp_two_pi_decimal_int
\int_set:Nn \c_fp_two_pi_decimal_int { 283 185 307 }
\int_new:N  \c_fp_two_pi_extended_int
\int_set:Nn \c_fp_two_pi_extended_int { 179 586 477 }
%    \end{macrocode}
% \end{variable}
% \end{variable}
% \end{variable}
% \end{variable}
% \end{variable}
% \end{variable}
%
% \begin{variable}{\c_e_fp}
%   The value $\mathrm{e}$ as a \enquote{machine number}.
%    \begin{macrocode}
\tl_const:Nn \c_e_fp { + 2.718281828 e 0 }
%    \end{macrocode}
% \end{variable}
%
% \begin{variable}{\c_one_fp}
%   The constant value $1$: used for fast comparisons.
%    \begin{macrocode}
\tl_const:Nn \c_one_fp { + 1.000000000 e 0 }
%    \end{macrocode}
% \end{variable}
%
% \begin{variable}{\c_pi_fp}
%   The value $\pi$ as a \enquote{machine number}.
%    \begin{macrocode}
\tl_const:Nn \c_pi_fp { + 3.141592654 e 0 }
%    \end{macrocode}
% \end{variable}
%
% \begin{variable}{\c_undefined_fp}
%   A marker for undefined values.
%    \begin{macrocode}
\tl_const:Nn \c_undefined_fp { X 0.000000000 e 0 }
%    \end{macrocode}
% \end{variable}
%
% \begin{variable}{\c_zero_fp}
%   The constant zero value.
%    \begin{macrocode}
\tl_const:Nn \c_zero_fp { + 0.000000000 e 0 }
%    \end{macrocode}
% \end{variable}
%
% \subsection{Variables}
%
% \begin{variable}{\l_fp_arg_tl}
%   A token list to store the formalised representation of the input
%   for transcendental functions.
%    \begin{macrocode}
\tl_new:N \l_fp_arg_tl
%    \end{macrocode}
% \end{variable}
%
% \begin{variable}{\l_fp_count_int}
%   A counter for things like the number of divisions possible.
%    \begin{macrocode}
\int_new:N \l_fp_count_int
%    \end{macrocode}
% \end{variable}
%
% \begin{variable}{\l_fp_div_offset_int}
%   When carrying out division, an offset is used for the results to
%   get the decimal part correct.
%    \begin{macrocode}
\int_new:N \l_fp_div_offset_int
%    \end{macrocode}
% \end{variable}
%
% \begin{variable}{\l_fp_exp_integer_int}
% \begin{variable}{\l_fp_exp_decimal_int}
% \begin{variable}{\l_fp_exp_extended_int}
% \begin{variable}{\l_fp_exp_exponent_int}
%   Used for the calculation of exponent values.
%    \begin{macrocode}
\int_new:N \l_fp_exp_integer_int
\int_new:N \l_fp_exp_decimal_int
\int_new:N \l_fp_exp_extended_int
\int_new:N \l_fp_exp_exponent_int
%    \end{macrocode}
% \end{variable}
% \end{variable}
% \end{variable}
% \end{variable}
%
% \begin{variable}{\l_fp_input_a_sign_int}
% \begin{variable}{\l_fp_input_a_integer_int}
% \begin{variable}{\l_fp_input_a_decimal_int}
% \begin{variable}{\l_fp_input_a_exponent_int}
% \begin{variable}{\l_fp_input_b_sign_int}
% \begin{variable}{\l_fp_input_b_integer_int}
% \begin{variable}{\l_fp_input_b_decimal_int}
% \begin{variable}{\l_fp_input_b_exponent_int}
%   Storage for the input: two storage areas as there are at most two
%   inputs.
%    \begin{macrocode}
\int_new:N \l_fp_input_a_sign_int
\int_new:N \l_fp_input_a_integer_int
\int_new:N \l_fp_input_a_decimal_int
\int_new:N \l_fp_input_a_exponent_int
\int_new:N \l_fp_input_b_sign_int
\int_new:N \l_fp_input_b_integer_int
\int_new:N \l_fp_input_b_decimal_int
\int_new:N \l_fp_input_b_exponent_int
%    \end{macrocode}
% \end{variable}
% \end{variable}
% \end{variable}
% \end{variable}
% \end{variable}
% \end{variable}
% \end{variable}
% \end{variable}
%
% \begin{variable}{\l_fp_input_a_extended_int}
% \begin{variable}{\l_fp_input_b_extended_int}
%   For internal use, \enquote{extended} floating point numbers are
%   needed.
%    \begin{macrocode}
\int_new:N \l_fp_input_a_extended_int
\int_new:N \l_fp_input_b_extended_int
%    \end{macrocode}
% \end{variable}
% \end{variable}
%
% \begin{variable}{\l_fp_mul_a_i_int}
% \begin{variable}{\l_fp_mul_a_ii_int}
% \begin{variable}{\l_fp_mul_a_iii_int}
% \begin{variable}{\l_fp_mul_a_iv_int}
% \begin{variable}{\l_fp_mul_a_v_int}
% \begin{variable}{\l_fp_mul_a_vi_int}
% \begin{variable}{\l_fp_mul_b_i_int}
% \begin{variable}{\l_fp_mul_b_ii_int}
% \begin{variable}{\l_fp_mul_b_iii_int}
% \begin{variable}{\l_fp_mul_b_iv_int}
% \begin{variable}{\l_fp_mul_b_v_int}
% \begin{variable}{\l_fp_mul_b_vi_int}
%   Multiplication requires that the decimal part is split into parts
%   so that there are no overflows.
%    \begin{macrocode}
\int_new:N \l_fp_mul_a_i_int
\int_new:N \l_fp_mul_a_ii_int
\int_new:N \l_fp_mul_a_iii_int
\int_new:N \l_fp_mul_a_iv_int
\int_new:N \l_fp_mul_a_v_int
\int_new:N \l_fp_mul_a_vi_int
\int_new:N \l_fp_mul_b_i_int
\int_new:N \l_fp_mul_b_ii_int
\int_new:N \l_fp_mul_b_iii_int
\int_new:N \l_fp_mul_b_iv_int
\int_new:N \l_fp_mul_b_v_int
\int_new:N \l_fp_mul_b_vi_int
%    \end{macrocode}
% \end{variable}
% \end{variable}
% \end{variable}
% \end{variable}
% \end{variable}
% \end{variable}
% \end{variable}
% \end{variable}
% \end{variable}
% \end{variable}
% \end{variable}
% \end{variable}
%
% \begin{variable}{\l_fp_mul_output_int}
% \begin{variable}{\l_fp_mul_output_tl}
%   Space for multiplication results.
%    \begin{macrocode}
\int_new:N \l_fp_mul_output_int
\tl_new:N  \l_fp_mul_output_tl
%    \end{macrocode}
% \end{variable}
% \end{variable}
%
% \begin{variable}{\l_fp_output_sign_int}
% \begin{variable}{\l_fp_output_integer_int}
% \begin{variable}{\l_fp_output_decimal_int}
% \begin{variable}{\l_fp_output_exponent_int}
%   Output is stored in the same way as input.
%    \begin{macrocode}
\int_new:N \l_fp_output_sign_int
\int_new:N \l_fp_output_integer_int
\int_new:N \l_fp_output_decimal_int
\int_new:N \l_fp_output_exponent_int
%    \end{macrocode}
% \end{variable}
% \end{variable}
% \end{variable}
% \end{variable}
%
% \begin{variable}{\l_fp_output_extended_int}
%   Again, for calculations an extended part.
%    \begin{macrocode}
\int_new:N \l_fp_output_extended_int
%    \end{macrocode}
% \end{variable}
%
% \begin{variable}{\l_fp_round_carry_bool}
%   To indicate that a digit needs to be carried forward.
%    \begin{macrocode}
\bool_new:N \l_fp_round_carry_bool
%    \end{macrocode}
% \end{variable}
%
% \begin{variable}{\l_fp_round_decimal_tl}
%   A temporary store when rounding, to build up the decimal part without
%   needing to do any maths.
%    \begin{macrocode}
\tl_new:N \l_fp_round_decimal_tl
%    \end{macrocode}
% \end{variable}
%
% \begin{variable}{\l_fp_round_position_int}
% \begin{variable}{\l_fp_round_target_int}
%   Used to check the position for rounding.
%    \begin{macrocode}
\int_new:N \l_fp_round_position_int
\int_new:N \l_fp_round_target_int
%    \end{macrocode}
% \end{variable}
% \end{variable}
%
% \begin{variable}{\l_fp_sign_tl}
%   There are places where the sign needs to be set up \enquote{early},
%   so that the registers can be re-used.
%    \begin{macrocode}
\tl_new:N \l_fp_sign_tl
%    \end{macrocode}
% \end{variable}
%
% \begin{variable}{\l_fp_split_sign_int}
%   When splitting the input it is fastest to use a fixed name for the
%   sign part, and to transfer it after the split is complete.
%    \begin{macrocode}
\int_new:N \l_fp_split_sign_int
%    \end{macrocode}
% \end{variable}
%
% \begin{variable}{\l_fp_tmp_int}
%   A scratch \texttt{int}: used only where the value is not carried
%   forward.
%    \begin{macrocode}
\int_new:N \l_fp_tmp_int
%    \end{macrocode}
% \end{variable}
%
% \begin{variable}{\l_fp_tmp_tl}
%   A scratch token list variable for expanding material.
%    \begin{macrocode}
\tl_new:N \l_fp_tmp_tl
%    \end{macrocode}
% \end{variable}
%
% \begin{variable}{\l_fp_trig_octant_int}
%   To track which octant the trigonometric input is in.
%    \begin{macrocode}
\int_new:N \l_fp_trig_octant_int
%    \end{macrocode}
% \end{variable}
%
% \begin{variable}{\l_fp_trig_sign_int}
% \begin{variable}{\l_fp_trig_decimal_int}
% \begin{variable}{\l_fp_trig_extended_int}
%   Used for the calculation of trigonometric values.
%    \begin{macrocode}
\int_new:N \l_fp_trig_sign_int
\int_new:N \l_fp_trig_decimal_int
\int_new:N \l_fp_trig_extended_int
%    \end{macrocode}
% \end{variable}
% \end{variable}
% \end{variable}
%
% \subsection{Parsing numbers}
%
% \begin{macro}{\fp_read:N}
% \begin{macro}[aux]{\fp_read_aux:w}
%   Reading a stored value is made easier as the format is designed to
%   match the delimited function. This is always used to read the first
%   value (register |a|).
%    \begin{macrocode}
\cs_new_protected_nopar:Npn \fp_read:N #1
  { \exp_after:wN \fp_read_aux:w #1 \q_stop }
\cs_new_protected_nopar:Npn \fp_read_aux:w #1#2 . #3 e #4 \q_stop
  {
    \if:w #1 -
      \l_fp_input_a_sign_int \c_minus_one
    \else:
      \l_fp_input_a_sign_int \c_one
    \fi:
    \l_fp_input_a_integer_int  #2 \scan_stop:
    \l_fp_input_a_decimal_int  #3 \scan_stop:
    \l_fp_input_a_exponent_int #4 \scan_stop:
  }
%    \end{macrocode}
% \end{macro}
% \end{macro}
%
% \begin{macro}{\fp_split:Nn}
% \begin{macro}[aux]{\fp_split_sign:}
% \begin{macro}[aux]{\fp_split_exponent:}
% \begin{macro}[aux]{\fp_split_aux_i:w}
% \begin{macro}[aux]{\fp_split_aux_ii:w}
% \begin{macro}[aux]{\fp_split_aux_iii:w}
% \begin{macro}[aux]{\fp_split_decimal:w}
% \begin{macro}[aux]{\fp_split_decimal_aux:w}
%   The aim here is to use as much of \TeX{}'s mechanism as possible to pick
%   up the numerical input without any mistakes. In particular, negative
%   numbers have to be filtered out first in case the integer part is
%   $0$ (in which case \TeX{} would drop the |-| sign). That process
%   has to be done in a loop for cases where the sign is repeated.
%   Finding an exponent is relatively easy, after which the next phase is
%   to find the integer part, which will terminate with a |.|, and trigger
%   the decimal-finding code. The later will allow the decimal to be too
%   long, truncating the result.
%    \begin{macrocode}
\cs_new_protected_nopar:Npn \fp_split:Nn #1#2
  {
    \tl_set:Nx \l_fp_tmp_tl {#2}
    \tl_set_rescan:Nno \l_fp_tmp_tl { \char_make_ignore:n { 32 } }
      { \l_fp_tmp_tl }
    \l_fp_split_sign_int \c_one
    \fp_split_sign:
    \use:c { l_fp_input_ #1 _sign_int } \l_fp_split_sign_int
    \exp_after:wN \fp_split_exponent:w \l_fp_tmp_tl e e \q_stop #1
  }
\cs_new_protected_nopar:Npn \fp_split_sign:
  {
    \if_int_compare:w \pdf_strcmp:D
      { \exp_after:wN \tl_head:w \l_fp_tmp_tl ? \q_stop } { - }
        = \c_zero
      \tl_set:Nx \l_fp_tmp_tl
        {
          \exp_after:wN
            \tl_tail:w \l_fp_tmp_tl \prg_do_nothing: \q_stop
        }
      \l_fp_split_sign_int -\l_fp_split_sign_int
      \exp_after:wN \fp_split_sign:
    \else:
      \if_int_compare:w \pdf_strcmp:D
        { \exp_after:wN \tl_head:w \l_fp_tmp_tl ? \q_stop } { + }
          = \c_zero
        \tl_set:Nx \l_fp_tmp_tl
          {
            \exp_after:wN
              \tl_tail:w \l_fp_tmp_tl \prg_do_nothing: \q_stop
          }
        \exp_after:wN \exp_after:wN \exp_after:wN \fp_split_sign:
       \fi:
    \fi:
  }
\cs_new_protected_nopar:Npn \fp_split_exponent:w #1 e #2 e #3 \q_stop #4
  {
    \use:c { l_fp_input_ #4 _exponent_int }
      \int_eval:w 0 #2 \scan_stop:
    \tex_afterassignment:D \fp_split_aux_i:w
    \use:c { l_fp_input_ #4 _integer_int }
      \int_eval:w 0 #1 . . \q_stop #4
  }
\cs_new_protected_nopar:Npn \fp_split_aux_i:w #1 . #2 . #3 \q_stop
  { \fp_split_aux_ii:w #2 000000000 \q_stop }
\cs_new_protected_nopar:Npn \fp_split_aux_ii:w #1#2#3#4#5#6#7#8#9
  { \fp_split_aux_iii:w {#1#2#3#4#5#6#7#8#9} }
\cs_new_protected_nopar:Npn \fp_split_aux_iii:w #1#2 \q_stop
  {
    \l_fp_tmp_int 1 #1 \scan_stop:
    \exp_after:wN \fp_split_decimal:w
      \int_use:N \l_fp_tmp_int 000000000 \q_stop
  }
\cs_new_protected_nopar:Npn \fp_split_decimal:w #1#2#3#4#5#6#7#8#9
  { \fp_split_decimal_aux:w {#2#3#4#5#6#7#8#9} }
\cs_new_protected_nopar:Npn \fp_split_decimal_aux:w #1#2#3 \q_stop #4
  {
    \use:c { l_fp_input_ #4 _decimal_int } #1#2 \scan_stop:
    \if_int_compare:w
      \int_eval:w
        \use:c { l_fp_input_ #4 _integer_int } +
        \use:c { l_fp_input_ #4 _decimal_int }
      \scan_stop:
        = \c_zero
      \use:c { l_fp_input_ #4 _sign_int } \c_one
    \fi:
    \if_int_compare:w
      \use:c { l_fp_input_ #4 _integer_int } < \c_one_thousand_million
    \else:
      \exp_after:wN \fp_overflow_msg:
    \fi:
  }
%    \end{macrocode}
% \end{macro}
% \end{macro}
% \end{macro}
% \end{macro}
% \end{macro}
% \end{macro}
% \end{macro}
% \end{macro}
%
% \begin{macro}{\fp_standardise:NNNN}
% \begin{macro}[aux]{\fp_standardise_aux:NNNN}
% \begin{macro}[aux]{\fp_standardise_aux:}
% \begin{macro}[aux]{\fp_standardise_aux:w}
%   The idea here is to shift the input into a known exponent range. This
%   is done using \TeX{} tokens where possible, as this is faster than
%   arithmetic.
%    \begin{macrocode}
\cs_new_protected_nopar:Npn \fp_standardise:NNNN #1#2#3#4
  {
    \if_int_compare:w
      \int_eval:w #2 + #3 = \c_zero
      #1 \c_one
      #4 \c_zero
      \exp_after:wN \use_none:nnnn
    \else:
      \exp_after:wN \fp_standardise_aux:NNNN
    \fi:
    #1#2#3#4
  }
\cs_new_protected_nopar:Npn \fp_standardise_aux:NNNN #1#2#3#4
  {
    \cs_set_protected_nopar:Npn \fp_standardise_aux:
      {
        \if_int_compare:w #2 = \c_zero
          \int_advance:w #3 \c_one_thousand_million
          \exp_after:wN \fp_standardise_aux:w
            \int_use:N #3 \q_stop
           \exp_after:wN \fp_standardise_aux:
         \fi:
      }
    \cs_set_protected_nopar:Npn
      \fp_standardise_aux:w ##1##2##3##4##5##6##7##8##9 \q_stop
      {
        #2 ##2 \scan_stop:
        #3 ##3##4##5##6##7##8##9 0 \scan_stop:
        \int_advance:w #4 \c_minus_one
      }
    \fp_standardise_aux:
    \cs_set_protected_nopar:Npn \fp_standardise_aux:
      {
        \if_int_compare:w #2 > \c_nine
          \int_advance:w #2 \c_one_thousand_million
          \exp_after:wN \use_i:nn \exp_after:wN
            \fp_standardise_aux:w \int_use:N #2
           \exp_after:wN \fp_standardise_aux:
         \fi:
      }
    \cs_set_protected_nopar:Npn
      \fp_standardise_aux:w ##1##2##3##4##5##6##7##8##9
      {
        #2 ##1##2##3##4##5##6##7##8 \scan_stop:
        \int_advance:w #3 \c_one_thousand_million
        \tex_divide:D #3 \c_ten
        \tl_set:Nx \l_fp_tmp_tl
          {
            ##9
            \exp_after:wN \use_none:n \int_use:N #3
          }
        #3 \l_fp_tmp_tl \scan_stop:
        \int_advance:w #4 \c_one
      }
    \fp_standardise_aux:
    \if_int_compare:w #4 < \c_one_hundred
      \if_int_compare:w #4 > -\c_one_hundred
      \else:
        #1 \c_one
        #2 \c_zero
        #3 \c_zero
        #4 \c_zero
      \fi:
    \else:
      \exp_after:wN \fp_overflow_msg:
    \fi:
  }
\cs_new_protected_nopar:Npn \fp_standardise_aux: { }
\cs_new_protected_nopar:Npn \fp_standardise_aux:w { }
%    \end{macrocode}
% \end{macro}
% \end{macro}
% \end{macro}
% \end{macro}
%
% \subsection{Internal utilities}
%
% \begin{macro}{\fp_level_input_exponents:}
% \begin{macro}[aux]{\fp_level_input_exponents_a:}
% \begin{macro}[aux]{\fp_level_input_exponents_a:NNNNNNNNN}
% \begin{macro}[aux]{\fp_level_input_exponents_b:}
% \begin{macro}[aux]{\fp_level_input_exponents_b:NNNNNNNNN}
%   The routines here are similar to those used to standardise the
%   exponent. However, the aim here is different: the two exponents need
%   to end up the same.
%    \begin{macrocode}
\cs_new_protected_nopar:Npn \fp_level_input_exponents:
  {
    \if_int_compare:w \l_fp_input_a_exponent_int > \l_fp_input_b_exponent_int
      \exp_after:wN \fp_level_input_exponents_a:
    \else:
      \exp_after:wN \fp_level_input_exponents_b:
    \fi:
  }
\cs_new_protected_nopar:Npn \fp_level_input_exponents_a:
  {
    \if_int_compare:w \l_fp_input_a_exponent_int > \l_fp_input_b_exponent_int
      \int_advance:w \l_fp_input_b_integer_int \c_one_thousand_million
      \exp_after:wN \use_i:nn \exp_after:wN
        \fp_level_input_exponents_a:NNNNNNNNN
          \int_use:N \l_fp_input_b_integer_int
      \exp_after:wN \fp_level_input_exponents_a:
    \fi:
  }
\cs_new_protected_nopar:Npn \fp_level_input_exponents_a:NNNNNNNNN
  #1#2#3#4#5#6#7#8#9
  {
    \l_fp_input_b_integer_int #1#2#3#4#5#6#7#8 \scan_stop:
    \int_advance:w \l_fp_input_b_decimal_int \c_one_thousand_million
    \tex_divide:D \l_fp_input_b_decimal_int \c_ten
    \tl_set:Nx \l_fp_tmp_tl
      {
        #9
        \exp_after:wN \use_none:n
          \int_use:N \l_fp_input_b_decimal_int
      }
    \l_fp_input_b_decimal_int \l_fp_tmp_tl \scan_stop:
    \int_advance:w \l_fp_input_b_exponent_int \c_one
  }
\cs_new_protected_nopar:Npn \fp_level_input_exponents_b:
  {
    \if_int_compare:w \l_fp_input_b_exponent_int > \l_fp_input_a_exponent_int
      \int_advance:w \l_fp_input_a_integer_int \c_one_thousand_million
      \exp_after:wN \use_i:nn \exp_after:wN
        \fp_level_input_exponents_b:NNNNNNNNN
          \int_use:N \l_fp_input_a_integer_int
      \exp_after:wN \fp_level_input_exponents_b:
    \fi:
  }
\cs_new_protected_nopar:Npn \fp_level_input_exponents_b:NNNNNNNNN
  #1#2#3#4#5#6#7#8#9
  {
    \l_fp_input_a_integer_int #1#2#3#4#5#6#7#8 \scan_stop:
    \int_advance:w \l_fp_input_a_decimal_int \c_one_thousand_million
    \tex_divide:D \l_fp_input_a_decimal_int \c_ten
    \tl_set:Nx \l_fp_tmp_tl
      {
        #9
        \exp_after:wN \use_none:n
          \int_use:N \l_fp_input_a_decimal_int
      }
    \l_fp_input_a_decimal_int \l_fp_tmp_tl \scan_stop:
    \int_advance:w \l_fp_input_a_exponent_int \c_one
  }
%    \end{macrocode}
% \end{macro}
% \end{macro}
% \end{macro}
% \end{macro}
% \end{macro}
%
% \begin{macro}[aux]{\fp_tmp:w}
%   Used for output of results, cutting down on \cs{exp_after:wN}.
%   This is just a place holder definition.
%    \begin{macrocode}
\cs_new_protected_nopar:Npn \fp_tmp:w #1#2 { }
%    \end{macrocode}
% \end{macro}
%
% \subsection{Operations for \texttt{fp} variables}
%
% The format of \texttt{fp} variables is tightly defined, so that
% they can be read quickly by the internal code. The format is a single
% sign token, a single number, the decimal point, nine decimal numbers,
% an |e| and finally the exponent. This final part may vary in length.
% When stored, floating points will always be stored with a value in
% the integer position unless the number is zero.
%
% \begin{macro}{\fp_new:N, \fp_new:c}
% \UnitTested
%   Fixed-points always have a value, and of course this has to be
%   initialised globally.
%    \begin{macrocode}
\cs_new_protected_nopar:Npn \fp_new:N #1
  {
    \tl_new:N #1
    \tl_gset_eq:NN #1 \c_zero_fp
  }
\cs_generate_variant:Nn \fp_new:N { c }
%    \end{macrocode}
% \end{macro}
%
% \begin{macro}{\fp_const:Nn, \fp_const:cn}
%   A simple wrapper.
%    \begin{macrocode}
\cs_new_protected_nopar:Npn \fp_const:Nn #1#2
  {
    \fp_new:N #1
    \fp_gset:Nn #1 {#2}
  }
\cs_generate_variant:Nn \fp_const:Nn { c }
%    \end{macrocode}
% \end{macro}
%
% \begin{macro}{\fp_zero:N, \fp_zero:c}
% \UnitTested
% \begin{macro}{\fp_gzero:N, \fp_gzero:c}
% \UnitTested
%   Zeroing fixed-points is pretty obvious.
%    \begin{macrocode}
\cs_new_protected_nopar:Npn \fp_zero:N #1
  { \tl_set_eq:NN #1 \c_zero_fp }
\cs_new_protected_nopar:Npn \fp_gzero:N #1
  { \tl_gset_eq:NN #1 \c_zero_fp }
\cs_generate_variant:Nn \fp_zero:N { c }
\cs_generate_variant:Nn \fp_gzero:N { c }
%    \end{macrocode}
% \end{macro}
% \end{macro}
%
% \begin{macro}{\fp_set:Nn, \fp_set:cn}
% \UnitTested
% \begin{macro}{\fp_gset:Nn, \fp_gset:cn}
% \UnitTested
% \begin{macro}[aux]{\fp_set_aux:NNn}
%   To trap any input errors, a very simple version of the parser is run
%   here. This will pick up any invalid characters at this stage, saving
%   issues later. The splitting approach is the same as the more
%   advanced function later.
%    \begin{macrocode}
\cs_new_protected_nopar:Npn \fp_set:Nn  { \fp_set_aux:NNn \tl_set:Nn }
\cs_new_protected_nopar:Npn \fp_gset:Nn { \fp_set_aux:NNn \tl_gset:Nn }
\cs_new_protected_nopar:Npn \fp_set_aux:NNn #1#2#3
  {
    \group_begin:
      \fp_split:Nn a {#3}
      \fp_standardise:NNNN
        \l_fp_input_a_sign_int
        \l_fp_input_a_integer_int
        \l_fp_input_a_decimal_int
        \l_fp_input_a_exponent_int
      \int_advance:w \l_fp_input_a_decimal_int \c_one_thousand_million
      \cs_set_protected_nopar:Npx \fp_tmp:w
        {
          \group_end:
          #1 \exp_not:N #2
            {
              \if_int_compare:w \l_fp_input_a_sign_int < \c_zero
                -
              \else:
                +
              \fi:
              \int_use:N \l_fp_input_a_integer_int
              .
              \exp_after:wN \use_none:n
                \int_use:N \l_fp_input_a_decimal_int
              e
              \int_use:N \l_fp_input_a_exponent_int
            }
        }
    \fp_tmp:w
  }
\cs_generate_variant:Nn \fp_set:Nn  { c }
\cs_generate_variant:Nn \fp_gset:Nn { c }
%    \end{macrocode}
% \end{macro}
% \end{macro}
% \end{macro}
%
%
%
% \begin{macro}{\fp_set_from_dim:Nn,  \fp_set_from_dim:cn}
% \UnitTested
% \begin{macro}{\fp_gset_from_dim:Nn, \fp_gset_from_dim:cn}
% \UnitTested
% \begin{macro}[aux]{\fp_set_from_dim_aux:NNn}
% \begin{macro}[aux]{\fp_set_from_dim_aux:w}
% \begin{variable}{\l_fp_tmp_dim}
% \begin{variable}{\l_fp_tmp_skip}
%   Here, dimensions are converted to fixed-points \emph{via} a
%   temporary variable. This ensures that they always convert as points.
%   The code is then essentially the same as for \cs{fp_set:Nn}, but with
%   the dimension passed so that it will be striped of the |pt| on the
%   way through. The passage through a skip is used to remove any rubber
%   part.
%    \begin{macrocode}
\cs_new_protected_nopar:Npn \fp_set_from_dim:Nn
  { \fp_set_from_dim_aux:NNn \tl_set:Nx }
\cs_new_protected_nopar:Npn \fp_gset_from_dim:Nn
  { \fp_set_from_dim_aux:NNn \tl_gset:Nx }
\cs_new_protected_nopar:Npn \fp_set_from_dim_aux:NNn #1#2#3
  {
    \group_begin:
      \l_fp_tmp_skip \etex_glueexpr:D #3 \scan_stop:
      \l_fp_tmp_dim \l_fp_tmp_skip
      \fp_split:Nn a
        {
          \exp_after:wN \fp_set_from_dim_aux:w
            \dim_use:N \l_fp_tmp_dim
        }
      \fp_standardise:NNNN
        \l_fp_input_a_sign_int
        \l_fp_input_a_integer_int
        \l_fp_input_a_decimal_int
        \l_fp_input_a_exponent_int
      \int_advance:w \l_fp_input_a_decimal_int \c_one_thousand_million
      \cs_set_protected_nopar:Npx \fp_tmp:w
        {
          \group_end:
          #1 \exp_not:N #2
            {
              \if_int_compare:w \l_fp_input_a_sign_int < \c_zero
                -
              \else:
                +
              \fi:
              \int_use:N \l_fp_input_a_integer_int
              .
              \exp_after:wN \use_none:n
                \int_use:N \l_fp_input_a_decimal_int
              e
              \int_use:N \l_fp_input_a_exponent_int
            }
        }
    \fp_tmp:w
  }
\cs_set_protected_nopar:Npx \fp_set_from_dim_aux:w
  {
    \cs_set_nopar:Npn \exp_not:N \fp_set_from_dim_aux:w
      ##1 \tl_to_str:n { pt } {##1}
  }
\fp_set_from_dim_aux:w
\cs_generate_variant:Nn \fp_set_from_dim:Nn  { c }
\cs_generate_variant:Nn \fp_gset_from_dim:Nn { c }
\dim_new:N \l_fp_tmp_dim
\skip_new:N \l_fp_tmp_skip
%    \end{macrocode}
% \end{variable}
% \end{variable}
% \end{macro}
% \end{macro}
% \end{macro}
% \end{macro}
%
% \begin{macro}{\fp_set_eq:NN, \fp_set_eq:cN, \fp_set_eq:Nc, \fp_set_eq:cc}
% \UnitTested
% \begin{macro}{\fp_gset_eq:NN, \fp_gset_eq:cN, \fp_gset_eq:Nc, \fp_gset_eq:cc}
% \UnitTested
%   Pretty simple, really.
%    \begin{macrocode}
\cs_new_eq:NN \fp_set_eq:NN  \tl_set_eq:NN
\cs_new_eq:NN \fp_set_eq:cN  \tl_set_eq:cN
\cs_new_eq:NN \fp_set_eq:Nc  \tl_set_eq:Nc
\cs_new_eq:NN \fp_set_eq:cc  \tl_set_eq:cc
\cs_new_eq:NN \fp_gset_eq:NN \tl_gset_eq:NN
\cs_new_eq:NN \fp_gset_eq:cN \tl_gset_eq:cN
\cs_new_eq:NN \fp_gset_eq:Nc \tl_gset_eq:Nc
\cs_new_eq:NN \fp_gset_eq:cc \tl_gset_eq:cc
%    \end{macrocode}
% \end{macro}
% \end{macro}
%
% \begin{macro}{\fp_show:N, \fp_show:c}
% \UnitTested
%   Simple showing of the underlying variable.
%    \begin{macrocode}
\cs_new_eq:NN \fp_show:N \tl_show:N
\cs_new_eq:NN \fp_show:c \tl_show:c
%    \end{macrocode}
% \end{macro}
%
% \begin{macro}{\fp_use:N, \fp_use:c}
% \UnitTested
% \begin{macro}[aux]{\fp_use_aux:w}
% \begin{macro}[aux]{\fp_use_none:w}
% \begin{macro}[aux]{\fp_use_small:w}
% \begin{macro}[aux]{\fp_use_large:w}
% \begin{macro}[aux]{\fp_use_large_aux_i:w}
% \begin{macro}[aux]{\fp_use_large_aux_1:w}
% \begin{macro}[aux]{\fp_use_large_aux_2:w}
% \begin{macro}[aux]{\fp_use_large_aux_3:w}
% \begin{macro}[aux]{\fp_use_large_aux_4:w}
% \begin{macro}[aux]{\fp_use_large_aux_5:w}
% \begin{macro}[aux]{\fp_use_large_aux_6:w}
% \begin{macro}[aux]{\fp_use_large_aux_7:w}
% \begin{macro}[aux]{\fp_use_large_aux_8:w}
% \begin{macro}[aux]{\fp_use_large_aux_i:w}
% \begin{macro}[aux]{\fp_use_large_aux_ii:w}
%   The idea of the \cs{fp_use:N} function to convert the stored
%   value into something suitable for \TeX{} to use as a number in an
%   expandable manner. The first step is to deal with the sign, then
%   work out how big the input is.
%    \begin{macrocode}
\cs_new_nopar:Npn \fp_use:N #1
  { \exp_after:wN \fp_use_aux:w #1 \q_stop }
\cs_generate_variant:Nn \fp_use:N { c }
\cs_new_nopar:Npn \fp_use_aux:w #1#2 e #3 \q_stop
  {
    \if:w #1 -
      -
    \fi:
    \if_int_compare:w #3 > \c_zero
      \exp_after:wN \fp_use_large:w
    \else:
      \if_int_compare:w #3 < \c_zero
        \exp_after:wN \exp_after:wN \exp_after:wN
          \fp_use_small:w
      \else:
        \exp_after:wN \exp_after:wN \exp_after:wN \fp_use_none:w
      \fi:
    \fi:
    #2 e #3 \q_stop
  }
%    \end{macrocode}
%   When the exponent is zero, the input is simply returned as output.
%    \begin{macrocode}
\cs_new_nopar:Npn \fp_use_none:w #1 e #2 \q_stop {#1}
%    \end{macrocode}
%   For small numbers (less than $1$) the correct number of zeros
%   have to be inserted, but the decimal point is easy.
%    \begin{macrocode}
\cs_new_nopar:Npn \fp_use_small:w #1 . #2 e #3 \q_stop
  {
    0 .
    \prg_replicate:nn { -#3 - 1 } { 0 }
    #1#2
  }
%    \end{macrocode}
%   Life is more complex for large numbers. The decimal point needs to
%   be shuffled, with potentially some zero-filling for very large values.
%    \begin{macrocode}
\cs_new_nopar:Npn \fp_use_large:w #1 . #2 e #3 \q_stop
  {
    \if_int_compare:w #3 < \c_ten
      \exp_after:wN \fp_use_large_aux_i:w
    \else:
      \exp_after:wN \fp_use_large_aux_ii:w
    \fi:
    #1#2 e #3 \q_stop
  }
\cs_new_nopar:Npn \fp_use_large_aux_i:w #1#2 e #3 \q_stop
  {
    #1
    \use:c { fp_use_large_aux_ #3 :w } #2 \q_stop
  }
\cs_new_nopar:cpn { fp_use_large_aux_1:w } #1#2 \q_stop { #1 . #2 }
\cs_new_nopar:cpn { fp_use_large_aux_2:w } #1#2#3 \q_stop
  { #1#2 . #3 }
\cs_new_nopar:cpn { fp_use_large_aux_3:w } #1#2#3#4 \q_stop
  { #1#2#3 . #4 }
\cs_new_nopar:cpn { fp_use_large_aux_4:w } #1#2#3#4#5 \q_stop
  { #1#2#3#4 . #5 }
\cs_new_nopar:cpn { fp_use_large_aux_5:w } #1#2#3#4#5#6 \q_stop
  { #1#2#3#4#5 . #6 }
\cs_new_nopar:cpn { fp_use_large_aux_6:w } #1#2#3#4#5#6#7 \q_stop
  { #1#2#3#4#5#6 . #7 }
\cs_new_nopar:cpn { fp_use_large_aux_7:w } #1#2#3#4#5#6#7#8 \q_stop
  { #1#2#3#4#6#7 . #8 }
\cs_new_nopar:cpn { fp_use_large_aux_8:w } #1#2#3#4#5#6#7#8#9 \q_stop
  { #1#2#3#4#5#6#7#8 . #9 }
\cs_new_nopar:cpn { fp_use_large_aux_9:w } #1 \q_stop { #1 . }
\cs_new_nopar:Npn \fp_use_large_aux_ii:w #1 e #2 \q_stop
  {
    #1
    \prg_replicate:nn { #2 - 9 } { 0 }
    .
  }
%    \end{macrocode}
% \end{macro}
% \end{macro}
% \end{macro}
% \end{macro}
% \end{macro}
% \end{macro}
% \end{macro}
% \end{macro}
% \end{macro}
% \end{macro}
% \end{macro}
% \end{macro}
% \end{macro}
% \end{macro}
% \end{macro}
% \end{macro}
%
% \subsection{Transferring to other types}
%
% The \cs{fp_use:N} function converts a floating point variable to
% a form that can be used by \TeX{}. Here, the functions are slightly
% different, as some information may be discarded.
%
% \begin{macro}{\fp_to_dim:N, \fp_to_dim:c}
%   A very simple wrapper.
%    \begin{macrocode}
\cs_new_nopar:Npn \fp_to_dim:N #1 { \fp_use:N #1 pt }
\cs_generate_variant:Nn \fp_to_dim:N { c }
%    \end{macrocode}
% \end{macro}
%
%
% \begin{macro}{\fp_to_int:N, \fp_to_int:c}
% \UnitTested
% \begin{macro}[aux]{\fp_to_int_aux:w}
% \begin{macro}[aux]{\fp_to_int_none:w}
% \begin{macro}[aux]{\fp_to_int_small:w}
% \begin{macro}[aux]{\fp_to_int_large:w}
% \begin{macro}[aux]{\fp_to_int_large_aux_i:w}
% \begin{macro}[aux]{\fp_to_int_large_aux_1:w}
% \begin{macro}[aux]{\fp_to_int_large_aux_2:w}
% \begin{macro}[aux]{\fp_to_int_large_aux_3:w}
% \begin{macro}[aux]{\fp_to_int_large_aux_4:w}
% \begin{macro}[aux]{\fp_to_int_large_aux_5:w}
% \begin{macro}[aux]{\fp_to_int_large_aux_6:w}
% \begin{macro}[aux]{\fp_to_int_large_aux_7:w}
% \begin{macro}[aux]{\fp_to_int_large_aux_8:w}
% \begin{macro}[aux]{\fp_to_int_large_aux_i:w}
% \begin{macro}[aux]{\fp_to_int_large_aux:nnn}
% \begin{macro}[aux]{\fp_to_int_large_aux_ii:w}
%   Converting to integers in an expandable manner is very similar to
%   simply using floating point variables, particularly in the lead-off.
%    \begin{macrocode}
\cs_new_nopar:Npn \fp_to_int:N #1
  { \exp_after:wN \fp_to_int_aux:w #1 \q_stop }
\cs_generate_variant:Nn \fp_to_int:N { c }
\cs_new_nopar:Npn \fp_to_int_aux:w #1#2 e #3 \q_stop
  {
    \if:w #1 -
      -
    \fi:
    \if_int_compare:w #3 < \c_zero
      \exp_after:wN \fp_to_int_small:w
    \else:
      \exp_after:wN \fp_to_int_large:w
    \fi:
    #2 e #3 \q_stop
  }
%    \end{macrocode}
%   For small numbers, if the decimal part is greater than a half then
%   there is rounding up to do.
%    \begin{macrocode}
\cs_new_nopar:Npn \fp_to_int_small:w #1 . #2 e #3 \q_stop
  {
    \if_int_compare:w #3 > \c_one
    \else:
      \if_int_compare:w #1 < \c_five
        0
      \else:
        1
      \fi:
    \fi:
  }
%    \end{macrocode}
% For large numbers, the idea is to split off the part for rounding,
% do the rounding and fill if needed.
%    \begin{macrocode}
\cs_new_nopar:Npn \fp_to_int_large:w #1 . #2 e #3 \q_stop
  {
    \if_int_compare:w #3 < \c_ten
      \exp_after:wN \fp_to_int_large_aux_i:w
    \else:
      \exp_after:wN \fp_to_int_large_aux_ii:w
    \fi:
    #1#2 e #3 \q_stop
  }
\cs_new_nopar:Npn \fp_to_int_large_aux_i:w #1#2 e #3 \q_stop
  { \use:c { fp_to_int_large_aux_ #3 :w } #2 \q_stop {#1} }
\cs_new_nopar:cpn { fp_to_int_large_aux_1:w } #1#2 \q_stop
  { \fp_to_int_large_aux:nnn { #2 0 } {#1} }
\cs_new_nopar:cpn { fp_to_int_large_aux_2:w } #1#2#3 \q_stop
  { \fp_to_int_large_aux:nnn { #3 00 } {#1#2} }
\cs_new_nopar:cpn { fp_to_int_large_aux_3:w } #1#2#3#4 \q_stop
  { \fp_to_int_large_aux:nnn { #4 000 } {#1#2#3} }
\cs_new_nopar:cpn { fp_to_int_large_aux_4:w } #1#2#3#4#5 \q_stop
  { \fp_to_int_large_aux:nnn { #5 0000 } {#1#2#3#4} }
\cs_new_nopar:cpn { fp_to_int_large_aux_5:w } #1#2#3#4#5#6 \q_stop
  { \fp_to_int_large_aux:nnn { #6 00000 } {#1#2#3#4#5} }
\cs_new_nopar:cpn { fp_to_int_large_aux_6:w } #1#2#3#4#5#6#7 \q_stop
  { \fp_to_int_large_aux:nnn { #7 000000 } {#1#2#3#4#5#6} }
\cs_new_nopar:cpn { fp_to_int_large_aux_7:w } #1#2#3#4#5#6#7#8 \q_stop
  { \fp_to_int_large_aux:nnn { #8 0000000 } {#1#2#3#4#5#6#7} }
\cs_new_nopar:cpn { fp_to_int_large_aux_8:w } #1#2#3#4#5#6#7#8#9 \q_stop
  { \fp_to_int_large_aux:nnn { #9 00000000 } {#1#2#3#4#5#6#7#8} }
\cs_new_nopar:cpn { fp_to_int_large_aux_9:w } #1 \q_stop {#1}
\cs_new_nopar:Npn \fp_to_int_large_aux:nnn #1#2#3
  {
    \if_int_compare:w #1 < \c_five_hundred_million
      #3#2
    \else:
      \int_value:w \int_eval:w #3#2 + 1 \int_eval_end:
    \fi:
  }
\cs_new_nopar:Npn \fp_to_int_large_aux_ii:w #1 e #2 \q_stop
  {
    #1
    \prg_replicate:nn { #2 - 9 } { 0 }
  }
%    \end{macrocode}
% \end{macro}
% \end{macro}
% \end{macro}
% \end{macro}
% \end{macro}
% \end{macro}
% \end{macro}
% \end{macro}
% \end{macro}
% \end{macro}
% \end{macro}
% \end{macro}
% \end{macro}
% \end{macro}
% \end{macro}
% \end{macro}
% \end{macro}
%
% \begin{macro}{\fp_to_tl:N, \fp_to_tl:c}
% \UnitTested
% \begin{macro}[aux]{\fp_to_tl_aux:w}
% \begin{macro}[aux]{\fp_to_tl_large:w}
% \begin{macro}[aux]{\fp_to_tl_large_aux_i:w}
% \begin{macro}[aux]{\fp_to_tl_large_aux_ii:w}
% \begin{macro}[aux]{\fp_to_tl_large_0:w}
% \begin{macro}[aux]{\fp_to_tl_large_1:w}
% \begin{macro}[aux]{\fp_to_tl_large_2:w}
% \begin{macro}[aux]{\fp_to_tl_large_3:w}
% \begin{macro}[aux]{\fp_to_tl_large_4:w}
% \begin{macro}[aux]{\fp_to_tl_large_5:w}
% \begin{macro}[aux]{\fp_to_tl_large_6:w}
% \begin{macro}[aux]{\fp_to_tl_large_7:w}
% \begin{macro}[aux]{\fp_to_tl_large_8:w}
% \begin{macro}[aux]{\fp_to_tl_large_8_aux:w}
% \begin{macro}[aux]{\fp_to_tl_large_9:w}
% \begin{macro}[aux]{\fp_to_tl_small:w}
% \begin{macro}[aux]{\fp_to_tl_small_one:w}
% \begin{macro}[aux]{\fp_to_tl_small_two:w}
% \begin{macro}[aux]{\fp_to_tl_small_aux:w}
% \begin{macro}[aux]{\fp_to_tl_large_zeros:NNNNNNNNN}
% \begin{macro}[aux]{\fp_to_tl_small_zeros:NNNNNNNNN}
% \begin{macro}[aux]{\fp_use_iix_ix:NNNNNNNNN}
% \begin{macro}[aux]{\fp_use_ix:NNNNNNNNN}
% \begin{macro}[aux]{\fp_use_i_to_vii:NNNNNNNNN}
% \begin{macro}[aux]{\fp_use_i_to_iix:NNNNNNNNN}
%   Converting to integers in an expandable manner is very similar to
%   simply using floating point variables, particularly in the lead-off.
%    \begin{macrocode}
\cs_new_nopar:Npn \fp_to_tl:N #1
  { \exp_after:wN \fp_to_tl_aux:w #1 \q_stop }
\cs_generate_variant:Nn \fp_to_tl:N { c }
\cs_new_nopar:Npn \fp_to_tl_aux:w #1#2 e #3 \q_stop
  {
    \if:w #1 -
      -
    \fi:
    \if_int_compare:w #3 < \c_zero
      \exp_after:wN \fp_to_tl_small:w
    \else:
      \exp_after:wN \fp_to_tl_large:w
    \fi:
    #2 e #3 \q_stop
  }
%    \end{macrocode}
%   For \enquote{large} numbers (exponent $\ge 0$) there are two
%   cases. For very large exponents ($ \ge 10 $) life is easy: apart
%   from dropping extra zeros there is no work to do. On the other hand,
%   for intermediate exponent values the decimal needs to be moved, then
%   zeros can be dropped.
%    \begin{macrocode}
\cs_new_nopar:Npn \fp_to_tl_large:w #1 e #2 \q_stop
  {
    \if_int_compare:w #2 < \c_ten
      \exp_after:wN \fp_to_tl_large_aux_i:w
    \else:
      \exp_after:wN \fp_to_tl_large_aux_ii:w
    \fi:
    #1 e #2 \q_stop
  }
\cs_new_nopar:Npn \fp_to_tl_large_aux_i:w #1 e #2 \q_stop
  { \use:c { fp_to_tl_large_ #2 :w } #1 \q_stop }
\cs_new_nopar:Npn \fp_to_tl_large_aux_ii:w #1 . #2 e #3 \q_stop
  {
    #1
    \fp_to_tl_large_zeros:NNNNNNNNN #2
    e #3
  }
\cs_new_nopar:cpn { fp_to_tl_large_0:w } #1 . #2 \q_stop
  {
    #1
    \fp_to_tl_large_zeros:NNNNNNNNN #2
  }
\cs_new_nopar:cpn { fp_to_tl_large_1:w } #1 . #2#3 \q_stop
  {
    #1#2
    \fp_to_tl_large_zeros:NNNNNNNNN #3 0
  }
\cs_new_nopar:cpn { fp_to_tl_large_2:w } #1 . #2#3#4 \q_stop
  {
    #1#2#3
    \fp_to_tl_large_zeros:NNNNNNNNN #4 00
  }
\cs_new_nopar:cpn { fp_to_tl_large_3:w } #1 . #2#3#4#5 \q_stop
  {
    #1#2#3#4
    \fp_to_tl_large_zeros:NNNNNNNNN #5 000
  }
\cs_new_nopar:cpn { fp_to_tl_large_4:w } #1 . #2#3#4#5#6 \q_stop
  {
    #1#2#3#4#5
    \fp_to_tl_large_zeros:NNNNNNNNN #6 0000
  }
\cs_new_nopar:cpn { fp_to_tl_large_5:w } #1 . #2#3#4#5#6#7 \q_stop
  {
    #1#2#3#4#5#6
    \fp_to_tl_large_zeros:NNNNNNNNN #7 00000
  }
\cs_new_nopar:cpn { fp_to_tl_large_6:w } #1 . #2#3#4#5#6#7#8 \q_stop
  {
    #1#2#3#4#5#6#7
    \fp_to_tl_large_zeros:NNNNNNNNN #8 000000
  }
\cs_new_nopar:cpn { fp_to_tl_large_7:w } #1 . #2#3#4#5#6#7#8#9 \q_stop
  {
    #1#2#3#4#5#6#7#8
    \fp_to_tl_large_zeros:NNNNNNNNN #9 0000000
  }
\cs_new_nopar:cpn { fp_to_tl_large_8:w } #1 .
  {
    #1
    \use:c { fp_to_tl_large_8_aux:w }
  }
\cs_new_nopar:cpn { fp_to_tl_large_8_aux:w } #1#2#3#4#5#6#7#8#9 \q_stop
  {
    #1#2#3#4#5#6#7#8
    \fp_to_tl_large_zeros:NNNNNNNNN #9 00000000
  }
\cs_new_nopar:cpn { fp_to_tl_large_9:w } #1 . #2 \q_stop {#1#2}
%    \end{macrocode}
%   Dealing with small numbers is a bit more complex as there has to be
%   rounding. This makes life rather awkward, as there need to be a series
%   of tests and calculations, as things cannot be stored in an
%   expandable system.
%    \begin{macrocode}
\cs_new_nopar:Npn \fp_to_tl_small:w #1 e #2 \q_stop
  {
    \if_int_compare:w #2 = \c_minus_one
      \exp_after:wN \fp_to_tl_small_one:w
    \else:
      \if_int_compare:w #2 = -\c_two
        \exp_after:wN \exp_after:wN \exp_after:wN \fp_to_tl_small_two:w
      \else:
        \exp_after:wN \exp_after:wN \exp_after:wN \fp_to_tl_small_aux:w
      \fi:
    \fi:
    #1 e #2 \q_stop
  }
\cs_new_nopar:Npn \fp_to_tl_small_one:w #1 . #2 e #3 \q_stop
  {
    \if_int_compare:w \fp_use_ix:NNNNNNNNN #2 > \c_four
      \if_int_compare:w
        \int_eval:w #1 \fp_use_i_to_iix:NNNNNNNNN #2 + 1
          < \c_one_thousand_million
        0.
        \exp_after:wN \fp_to_tl_small_zeros:NNNNNNNNN
          \int_value:w \int_eval:w
              #1 \fp_use_i_to_iix:NNNNNNNNN #2 + 1
            \int_eval_end:
      \else:
        1
      \fi:
    \else:
      0. #1
      \fp_to_tl_small_zeros:NNNNNNNNN #2
    \fi:
  }
\cs_new_nopar:Npn \fp_to_tl_small_two:w #1 . #2 e #3 \q_stop
  {
    \if_int_compare:w \fp_use_iix_ix:NNNNNNNNN #2 > \c_forty_four
      \if_int_compare:w
        \int_eval:w #1 \fp_use_i_to_vii:NNNNNNNNN #2 0 + \c_ten
          < \c_one_thousand_million
        0.0
        \exp_after:wN \fp_to_tl_small_zeros:NNNNNNNNN
          \int_value:w \int_eval:w
              #1 \fp_use_i_to_vii:NNNNNNNNN #2 0 + \c_ten
            \int_eval_end:
      \else:
        0.1
      \fi:
    \else:
      0.0
      #1
      \fp_to_tl_small_zeros:NNNNNNNNN #2
    \fi:
  }
\cs_new_nopar:Npn \fp_to_tl_small_aux:w #1 . #2 e #3 \q_stop
  {
    #1
    \fp_to_tl_large_zeros:NNNNNNNNN #2
    e #3
  }
%    \end{macrocode}
%   Rather than a complex recursion, the tests for finding trailing zeros
%   are written out long-hand.  The difference between the two is only the
%   need for a decimal marker.
%    \begin{macrocode}
\cs_new_nopar:Npn \fp_to_tl_large_zeros:NNNNNNNNN #1#2#3#4#5#6#7#8#9
  {
    \if_int_compare:w #9 = \c_zero
      \if_int_compare:w #8 = \c_zero
        \if_int_compare:w #7 = \c_zero
          \if_int_compare:w #6 = \c_zero
            \if_int_compare:w #5 = \c_zero
              \if_int_compare:w #4 = \c_zero
                \if_int_compare:w #3 = \c_zero
                  \if_int_compare:w #2 = \c_zero
                    \if_int_compare:w #1 = \c_zero
                    \else:
                      . #1
                    \fi:
                  \else:
                    . #1#2
                  \fi:
                \else:
                  . #1#2#3
                \fi:
              \else:
                . #1#2#3#4
              \fi:
            \else:
              . #1#2#3#4#5
            \fi:
          \else:
            . #1#2#3#4#5#6
          \fi:
        \else:
          . #1#2#3#4#5#6#7
        \fi:
      \else:
         . #1#2#3#4#5#6#7#8
      \fi:
    \else:
      . #1#2#3#4#5#6#7#8#9
    \fi:
  }
\cs_new_nopar:Npn \fp_to_tl_small_zeros:NNNNNNNNN #1#2#3#4#5#6#7#8#9
  {
    \if_int_compare:w #9 = \c_zero
      \if_int_compare:w #8 = \c_zero
        \if_int_compare:w #7 = \c_zero
          \if_int_compare:w #6 = \c_zero
            \if_int_compare:w #5 = \c_zero
              \if_int_compare:w #4 = \c_zero
                \if_int_compare:w #3 = \c_zero
                  \if_int_compare:w #2 = \c_zero
                    \if_int_compare:w #1 = \c_zero
                    \else:
                      #1
                    \fi:
                  \else:
                    #1#2
                  \fi:
                \else:
                  #1#2#3
                \fi:
              \else:
                #1#2#3#4
              \fi:
            \else:
              #1#2#3#4#5
            \fi:
          \else:
            #1#2#3#4#5#6
          \fi:
        \else:
          #1#2#3#4#5#6#7
        \fi:
      \else:
         #1#2#3#4#5#6#7#8
      \fi:
    \else:
      #1#2#3#4#5#6#7#8#9
    \fi:
  }
%    \end{macrocode}
%   Some quick \enquote{return a few} functions.
%    \begin{macrocode}
\cs_new_nopar:Npn \fp_use_iix_ix:NNNNNNNNN #1#2#3#4#5#6#7#8#9 {#8#9}
\cs_new_nopar:Npn \fp_use_ix:NNNNNNNNN #1#2#3#4#5#6#7#8#9 {#9}
\cs_new_nopar:Npn \fp_use_i_to_vii:NNNNNNNNN #1#2#3#4#5#6#7#8#9
  {#1#2#3#4#5#6#7}
\cs_new_nopar:Npn \fp_use_i_to_iix:NNNNNNNNN #1#2#3#4#5#6#7#8#9
  {#1#2#3#4#5#6#7#8}
%    \end{macrocode}
% \end{macro}
% \end{macro}
% \end{macro}
% \end{macro}
% \end{macro}
% \end{macro}
% \end{macro}
% \end{macro}
% \end{macro}
% \end{macro}
% \end{macro}
% \end{macro}
% \end{macro}
% \end{macro}
% \end{macro}
% \end{macro}
% \end{macro}
% \end{macro}
% \end{macro}
% \end{macro}
% \end{macro}
% \end{macro}
% \end{macro}
% \end{macro}
% \end{macro}
% \end{macro}
%
% \subsection{Rounding numbers}
%
% The results may well need to be rounded. A couple of related functions
% to do this for a stored value.
%
% \begin{macro}{\fp_round_figures:Nn,  \fp_round_figures:cn}
% \UnitTested
% \begin{macro}{\fp_ground_figures:Nn, \fp_ground_figures:cn}
% \UnitTested
% \begin{macro}[aux]{\fp_round_figures_aux:NNn}
%   Rounding to figures needs only an adjustment to the target by one
%   (as the target is in decimal places).
%    \begin{macrocode}
\cs_new_protected_nopar:Npn \fp_round_figures:Nn
  { \fp_round_figures_aux:NNn \tl_set:Nn }
\cs_generate_variant:Nn \fp_round_figures:Nn { c }
\cs_new_protected_nopar:Npn \fp_ground_figures:Nn
  { \fp_round_figures_aux:NNn \tl_gset:Nn }
\cs_generate_variant:Nn \fp_ground_figures:Nn { c }
\cs_new_protected_nopar:Npn \fp_round_figures_aux:NNn #1#2#3
  {
    \group_begin:
      \fp_read:N #2
      \int_set:Nn \l_fp_round_target_int { #3 - 1 }
      \if_int_compare:w \l_fp_round_target_int < \c_ten
        \exp_after:wN \fp_round:
      \fi:
      \int_advance:w \l_fp_input_a_decimal_int \c_one_thousand_million
      \cs_set_protected_nopar:Npx \fp_tmp:w
        {
          \group_end:
          #1 \exp_not:N #2
            {
              \if_int_compare:w \l_fp_input_a_sign_int < \c_zero
                -
              \else:
                +
              \fi:
              \int_use:N \l_fp_input_a_integer_int
              .
              \exp_after:wN \use_none:n
                \int_use:N \l_fp_input_a_decimal_int
              e
              \int_use:N \l_fp_input_a_exponent_int
            }
        }
    \fp_tmp:w
  }
%    \end{macrocode}
% \end{macro}
% \end{macro}
% \end{macro}
%
% \begin{macro}{\fp_round_places:Nn,  \fp_round_places:cn}
% \UnitTested
% \begin{macro}{\fp_ground_places:Nn, \fp_ground_places:cn}
% \UnitTested
% \begin{macro}[aux]{\fp_round_places_aux:NNn}
%   Rounding to places needs an adjustment for the exponent value, which
%   will mean that everything should be correct.
%    \begin{macrocode}
\cs_new_protected_nopar:Npn \fp_round_places:Nn
  { \fp_round_places_aux:NNn \tl_set:Nn }
\cs_generate_variant:Nn \fp_round_places:Nn { c }
\cs_new_protected_nopar:Npn \fp_ground_places:Nn
  { \fp_round_places_aux:NNn \tl_gset:Nn }
\cs_generate_variant:Nn \fp_ground_places:Nn { c }
\cs_new_protected_nopar:Npn \fp_round_places_aux:NNn #1#2#3
  {
    \group_begin:
      \fp_read:N #2
      \int_set:Nn \l_fp_round_target_int
        { #3 + \l_fp_input_a_exponent_int }
      \if_int_compare:w \l_fp_round_target_int < \c_ten
        \exp_after:wN \fp_round:
      \fi:
      \int_advance:w \l_fp_input_a_decimal_int \c_one_thousand_million
      \cs_set_protected_nopar:Npx \fp_tmp:w
        {
          \group_end:
          #1 \exp_not:N #2
            {
              \if_int_compare:w \l_fp_input_a_sign_int < \c_zero
                -
              \else:
                +
              \fi:
              \int_use:N \l_fp_input_a_integer_int
              .
              \exp_after:wN \use_none:n
                \int_use:N \l_fp_input_a_decimal_int
              e
              \int_use:N \l_fp_input_a_exponent_int
            }
        }
    \fp_tmp:w
  }
%    \end{macrocode}
% \end{macro}
% \end{macro}
% \end{macro}
%
% \begin{macro}{\fp_round:}
% \begin{macro}[aux]{\fp_round_aux:NNNNNNNNN}
% \begin{macro}{\fp_round_loop:N}
%   The rounding approach is the same for decimal places and significant
%   figures. There are always nine decimal digits to round, so the code
%   can be written to account for this. The basic logic is simply to
%   find the rounding, track any carry digit and move along. At the end
%   of the loop there is a possible shuffle if the integer part has
%   become $10$.
%    \begin{macrocode}
\cs_new_protected_nopar:Npn \fp_round:
  {
    \bool_set_false:N \l_fp_round_carry_bool
    \l_fp_round_position_int \c_eight
    \tl_clear:N \l_fp_round_decimal_tl
    \int_advance:w \l_fp_input_a_decimal_int \c_one_thousand_million
    \exp_after:wN \use_i:nn \exp_after:wN
      \fp_round_aux:NNNNNNNNN \int_use:N \l_fp_input_a_decimal_int
  }
\cs_new_protected_nopar:Npn \fp_round_aux:NNNNNNNNN #1#2#3#4#5#6#7#8#9
  {
    \fp_round_loop:N #9#8#7#6#5#4#3#2#1
    \bool_if:NT \l_fp_round_carry_bool
      { \int_advance:w \l_fp_input_a_integer_int \c_one }
    \l_fp_input_a_decimal_int \l_fp_round_decimal_tl \scan_stop:
    \if_int_compare:w \l_fp_input_a_integer_int < \c_ten
    \else:
      \l_fp_input_a_integer_int \c_one
      \tex_divide:D \l_fp_input_a_decimal_int \c_ten
      \int_advance:w \l_fp_input_a_exponent_int \c_one
    \fi:
  }
\cs_new_protected_nopar:Npn \fp_round_loop:N #1
  {
    \if_int_compare:w \l_fp_round_position_int < \l_fp_round_target_int
      \bool_if:NTF \l_fp_round_carry_bool
        { \l_fp_tmp_int \int_eval:w #1 + \c_one \scan_stop: }
        { \l_fp_tmp_int \int_eval:w #1 \scan_stop: }
      \if_int_compare:w \l_fp_tmp_int = \c_ten
        \l_fp_tmp_int \c_zero
      \else:
        \bool_set_false:N \l_fp_round_carry_bool
      \fi:
      \tl_set:Nx \l_fp_round_decimal_tl
        { \int_use:N \l_fp_tmp_int \l_fp_round_decimal_tl }
    \else:
      \tl_set:Nx \l_fp_round_decimal_tl { 0 \l_fp_round_decimal_tl }
      \if_int_compare:w \l_fp_round_position_int = \l_fp_round_target_int
        \if_int_compare:w #1 > \c_four
          \bool_set_true:N \l_fp_round_carry_bool
        \fi:
      \fi:
    \fi:
    \int_advance:w \l_fp_round_position_int \c_minus_one
    \if_int_compare:w \l_fp_round_position_int > \c_minus_one
      \exp_after:wN \fp_round_loop:N
    \fi:
  }
%    \end{macrocode}
% \end{macro}
% \end{macro}
% \end{macro}
%
% \subsection{Unary functions}
%
% \begin{macro}{\fp_abs:N,  \fp_abs:c}
% \UnitTested
% \begin{macro}{\fp_gabs:N, \fp_gabs:c}
% \UnitTested
% \begin{macro}[aux]{\fp_abs_aux:NN}
%   Setting the absolute value is easy: read the value, ignore the sign,
%   return the result.
%    \begin{macrocode}
\cs_new_protected_nopar:Npn \fp_abs:N  { \fp_abs_aux:NN \tl_set:Nn }
\cs_new_protected_nopar:Npn \fp_gabs:N { \fp_abs_aux:NN \tl_gset:Nn }
\cs_generate_variant:Nn \fp_abs:N  { c }
\cs_generate_variant:Nn \fp_gabs:N { c }
\cs_new_protected_nopar:Npn \fp_abs_aux:NN #1#2
  {
    \group_begin:
      \fp_read:N #2
      \int_advance:w \l_fp_input_a_decimal_int \c_one_thousand_million
      \cs_set_protected_nopar:Npx \fp_tmp:w
        {
          \group_end:
          #1 \exp_not:N #2
            {
              +
              \int_use:N \l_fp_input_a_integer_int
              .
              \exp_after:wN \use_none:n
                \int_use:N \l_fp_input_a_decimal_int
              e
              \int_use:N \l_fp_input_a_exponent_int
            }
        }
    \fp_tmp:w
  }
%    \end{macrocode}
% \end{macro}
% \end{macro}
% \end{macro}
%
% \begin{macro}{\fp_neg:N,  \fp_neg:c}
% \UnitTested
% \begin{macro}{\fp_gneg:N, \fp_gneg:c}
% \UnitTested
% \begin{macro}[aux]{\fp_neg:NN}
% Just a bit more complex: read the input, reverse the sign and
% output the result.
%    \begin{macrocode}
\cs_new_protected_nopar:Npn \fp_neg:N  { \fp_neg_aux:NN \tl_set:Nn }
\cs_new_protected_nopar:Npn \fp_gneg:N { \fp_neg_aux:NN \tl_gset:Nn }
\cs_generate_variant:Nn \fp_neg:N  { c }
\cs_generate_variant:Nn \fp_gneg:N { c }
\cs_new_protected_nopar:Npn \fp_neg_aux:NN #1#2
  {
    \group_begin:
      \fp_read:N #2
      \int_advance:w \l_fp_input_a_decimal_int \c_one_thousand_million
      \tl_set:Nx \l_fp_tmp_tl
        {
          \if_int_compare:w \l_fp_input_a_sign_int < \c_zero
            +
          \else:
            -
          \fi:
          \int_use:N \l_fp_input_a_integer_int
          .
          \exp_after:wN \use_none:n
            \int_use:N \l_fp_input_a_decimal_int
          e
          \int_use:N \l_fp_input_a_exponent_int
        }
    \exp_after:wN \group_end: \exp_after:wN
    #1 \exp_after:wN #2 \exp_after:wN { \l_fp_tmp_tl }
  }
%    \end{macrocode}
% \end{macro}
% \end{macro}
% \end{macro}
%
% \subsection{Basic arithmetic}
%
% \begin{macro}{\fp_add:Nn, \fp_add:cn}
% \UnitTested
% \begin{macro}{\fp_gadd:Nn,\fp_gadd:cn}
% \UnitTested
% \begin{macro}[aux]{\fp_add_aux:NNn}
% \begin{macro}[aux]{\fp_add_core:}
% \begin{macro}[aux]{\fp_add_sum:}
% \begin{macro}[aux]{\fp_add_difference:}
%   The various addition functions are simply different ways to call the
%   single master function below. This pattern is repeated for the
%   other arithmetic functions.
%    \begin{macrocode}
\cs_new_protected_nopar:Npn \fp_add:Nn  { \fp_add_aux:NNn \tl_set:Nn }
\cs_new_protected_nopar:Npn \fp_gadd:Nn { \fp_add_aux:NNn \tl_gset:Nn }
\cs_generate_variant:Nn \fp_add:Nn   { c }
\cs_generate_variant:Nn \fp_gadd:Nn  { c }
%    \end{macrocode}
% Addition takes place using one of two paths. If the signs of the
% two parts are the same, they are simply combined. On the other
% hand, if the signs are different the calculation finds this
% difference.
%    \begin{macrocode}
\cs_new_protected_nopar:Npn \fp_add_aux:NNn #1#2#3
  {
    \group_begin:
      \fp_read:N #2
      \fp_split:Nn b {#3}
      \fp_standardise:NNNN
        \l_fp_input_b_sign_int
        \l_fp_input_b_integer_int
        \l_fp_input_b_decimal_int
        \l_fp_input_b_exponent_int
      \fp_add_core:
    \fp_tmp:w #1#2
  }
\cs_new_protected_nopar:Npn \fp_add_core:
  {
    \fp_level_input_exponents:
    \if_int_compare:w
      \int_eval:w
        \l_fp_input_a_sign_int * \l_fp_input_b_sign_int
        > \c_zero
      \exp_after:wN \fp_add_sum:
    \else:
      \exp_after:wN \fp_add_difference:
    \fi:
    \l_fp_output_exponent_int \l_fp_input_a_exponent_int
    \fp_standardise:NNNN
      \l_fp_output_sign_int
      \l_fp_output_integer_int
      \l_fp_output_decimal_int
      \l_fp_output_exponent_int
    \cs_set_protected_nopar:Npx \fp_tmp:w ##1##2
      {
        \group_end:
        ##1 ##2
          {
             \if_int_compare:w \l_fp_output_sign_int < \c_zero
              -
            \else:
              +
            \fi:
            \int_use:N \l_fp_output_integer_int
            .
            \exp_after:wN \use_none:n
              \int_value:w \int_eval:w
                 \l_fp_output_decimal_int + \c_one_thousand_million
            e
            \int_use:N \l_fp_output_exponent_int
          }
      }
  }
%    \end{macrocode}
%   Finding the sum of two numbers is trivially easy.
%    \begin{macrocode}
\cs_new_protected_nopar:Npn \fp_add_sum:
  {
    \l_fp_output_sign_int \l_fp_input_a_sign_int
    \l_fp_output_integer_int
      \int_eval:w
        \l_fp_input_a_integer_int + \l_fp_input_b_integer_int
      \scan_stop:
    \l_fp_output_decimal_int
      \int_eval:w
        \l_fp_input_a_decimal_int + \l_fp_input_b_decimal_int
      \scan_stop:
    \if_int_compare:w \l_fp_output_decimal_int < \c_one_thousand_million
    \else:
      \int_advance:w \l_fp_output_integer_int \c_one
      \int_advance:w \l_fp_output_decimal_int -\c_one_thousand_million
    \fi:
  }
%    \end{macrocode}
%   When the signs of the two parts of the input are different, the
%   absolute difference is worked out first. There is then a calculation
%   to see which way around everything has worked out, so that the final
%   sign is correct. The difference might also give a zero result with
%   a negative sign, which is reversed as zero is regarded as positive.
%    \begin{macrocode}
\cs_new_protected_nopar:Npn \fp_add_difference:
  {
    \l_fp_output_integer_int
      \int_eval:w
        \l_fp_input_a_integer_int - \l_fp_input_b_integer_int
      \scan_stop:
    \l_fp_output_decimal_int
      \int_eval:w
        \l_fp_input_a_decimal_int - \l_fp_input_b_decimal_int
      \scan_stop:
    \if_int_compare:w \l_fp_output_decimal_int < \c_zero
      \int_advance:w \l_fp_output_integer_int \c_minus_one
      \int_advance:w \l_fp_output_decimal_int \c_one_thousand_million
    \fi:
    \if_int_compare:w \l_fp_output_integer_int < \c_zero
      \l_fp_output_sign_int \l_fp_input_b_sign_int
      \if_int_compare:w \l_fp_output_decimal_int = \c_zero
        \l_fp_output_integer_int -\l_fp_output_integer_int
      \else:
        \l_fp_output_decimal_int
          \int_eval:w
            \c_one_thousand_million - \l_fp_output_decimal_int
          \scan_stop:
        \l_fp_output_integer_int
           \int_eval:w
             - \l_fp_output_integer_int - \c_one
           \scan_stop:
      \fi:
    \else:
      \l_fp_output_sign_int \l_fp_input_a_sign_int
    \fi:
  }
%    \end{macrocode}
% \end{macro}
% \end{macro}
% \end{macro}
% \end{macro}
% \end{macro}
% \end{macro}
%
% \begin{macro}{\fp_sub:Nn, \fp_sub:cn}
% \UnitTested
% \begin{macro}{\fp_gsub:Nn,\fp_gsub:cn}
% \UnitTested
% \begin{macro}[aux]{\fp_sub_aux:NNn}
%   Subtraction is essentially the same as addition, but with the sign
%   of the second component reversed. Thus the core of the two function
%   groups is the same, with just a little set up here.
%    \begin{macrocode}
\cs_new_protected_nopar:Npn \fp_sub:Nn  { \fp_sub_aux:NNn \tl_set:Nn }
\cs_new_protected_nopar:Npn \fp_gsub:Nn { \fp_sub_aux:NNn \tl_gset:Nn }
\cs_generate_variant:Nn \fp_sub:Nn   { c }
\cs_generate_variant:Nn \fp_gsub:Nn  { c }
\cs_new_protected_nopar:Npn \fp_sub_aux:NNn #1#2#3
  {
    \group_begin:
      \fp_read:N #2
      \fp_split:Nn b {#3}
      \fp_standardise:NNNN
        \l_fp_input_b_sign_int
        \l_fp_input_b_integer_int
        \l_fp_input_b_decimal_int
        \l_fp_input_b_exponent_int
      \tex_multiply:D \l_fp_input_b_sign_int \c_minus_one
      \fp_add_core:
    \fp_tmp:w #1#2
  }
%    \end{macrocode}
% \end{macro}
% \end{macro}
% \end{macro}
%
% \begin{macro}{\fp_mul:Nn, \fp_mul:cn}
% \UnitTested
% \begin{macro}{\fp_gmul:Nn,\fp_gmul:cn}
% \UnitTested
% \begin{macro}[aux]{\fp_mul_aux:NNn}
% \begin{macro}[aux]{\fp_mul_internal:}
% \begin{macro}[aux]{\fp_mul_split:NNNN}
% \begin{macro}[aux]{\fp_mul_split:w}
% \begin{macro}[aux]{\fp_mul_end_level:}
% \begin{macro}[aux]{\fp_mul_end_level:NNNNNNNNN}
%   The pattern is much the same for multiplication.
%    \begin{macrocode}
\cs_new_protected_nopar:Npn \fp_mul:Nn  { \fp_mul_aux:NNn \tl_set:Nn }
\cs_new_protected_nopar:Npn \fp_gmul:Nn { \fp_mul_aux:NNn \tl_gset:Nn }
\cs_generate_variant:Nn \fp_mul:Nn  { c }
\cs_generate_variant:Nn \fp_gmul:Nn { c }
%    \end{macrocode}
%   The approach to multiplication is as follows. First, the two numbers
%   are split into blocks of three digits. These are then multiplied
%   together to find products for each group of three output digits. This
%   is al written out in full for speed reasons. Between each block of
%   three digits in the output, there is a carry step. The very lowest
%   digits are not calculated, while
%    \begin{macrocode}
\cs_new_protected_nopar:Npn \fp_mul_aux:NNn #1#2#3
  {
    \group_begin:
      \fp_read:N #2
      \fp_split:Nn b {#3}
      \fp_standardise:NNNN
        \l_fp_input_b_sign_int
        \l_fp_input_b_integer_int
        \l_fp_input_b_decimal_int
        \l_fp_input_b_exponent_int
      \fp_mul_internal:
      \l_fp_output_exponent_int
        \int_eval:w
          \l_fp_input_a_exponent_int + \l_fp_input_b_exponent_int
        \scan_stop:
      \fp_standardise:NNNN
        \l_fp_output_sign_int
        \l_fp_output_integer_int
        \l_fp_output_decimal_int
        \l_fp_output_exponent_int
      \cs_set_protected_nopar:Npx \fp_tmp:w
        {
          \group_end:
          #1 \exp_not:N #2
            {
              \if_int_compare:w
                \int_eval:w
                  \l_fp_input_a_sign_int * \l_fp_input_b_sign_int
                  < \c_zero
                \if_int_compare:w
                  \int_eval:w
                    \l_fp_output_integer_int + \l_fp_output_decimal_int
                    = \c_zero
                  +
                \else:
                  -
                \fi:
              \else:
                +
             \fi:
              \int_use:N \l_fp_output_integer_int
              .
              \exp_after:wN \use_none:n
                \int_value:w \int_eval:w
                   \l_fp_output_decimal_int + \c_one_thousand_million
              e
              \int_use:N \l_fp_output_exponent_int
            }
        }
    \fp_tmp:w
  }
%    \end{macrocode}
%   Done separately so that the internal use is a bit easier.
%    \begin{macrocode}
\cs_new_protected_nopar:Npn \fp_mul_internal:
  {
    \fp_mul_split:NNNN \l_fp_input_a_decimal_int
      \l_fp_mul_a_i_int \l_fp_mul_a_ii_int \l_fp_mul_a_iii_int
    \fp_mul_split:NNNN \l_fp_input_b_decimal_int
      \l_fp_mul_b_i_int \l_fp_mul_b_ii_int \l_fp_mul_b_iii_int
    \l_fp_mul_output_int \c_zero
    \tl_clear:N \l_fp_mul_output_tl
    \fp_mul_product:NN \l_fp_mul_a_i_int         \l_fp_mul_b_iii_int
    \fp_mul_product:NN \l_fp_mul_a_ii_int        \l_fp_mul_b_ii_int
    \fp_mul_product:NN \l_fp_mul_a_iii_int       \l_fp_mul_b_i_int
    \tex_divide:D \l_fp_mul_output_int \c_one_thousand
    \fp_mul_product:NN \l_fp_input_a_integer_int \l_fp_mul_b_iii_int
    \fp_mul_product:NN \l_fp_mul_a_i_int         \l_fp_mul_b_ii_int
    \fp_mul_product:NN \l_fp_mul_a_ii_int        \l_fp_mul_b_i_int
    \fp_mul_product:NN \l_fp_mul_a_iii_int       \l_fp_input_b_integer_int
    \fp_mul_end_level:
    \fp_mul_product:NN \l_fp_input_a_integer_int \l_fp_mul_b_ii_int
    \fp_mul_product:NN \l_fp_mul_a_i_int         \l_fp_mul_b_i_int
    \fp_mul_product:NN \l_fp_mul_a_ii_int        \l_fp_input_b_integer_int
    \fp_mul_end_level:
    \fp_mul_product:NN \l_fp_input_a_integer_int \l_fp_mul_b_i_int
    \fp_mul_product:NN \l_fp_mul_a_i_int         \l_fp_input_b_integer_int
    \fp_mul_end_level:
    \l_fp_output_decimal_int 0 \l_fp_mul_output_tl \scan_stop:
    \tl_clear:N \l_fp_mul_output_tl
    \fp_mul_product:NN \l_fp_input_a_integer_int \l_fp_input_b_integer_int
    \fp_mul_end_level:
    \l_fp_output_integer_int 0 \l_fp_mul_output_tl \scan_stop:
  }
%    \end{macrocode}
%   The split works by making a $10$ digit number, from which
%   the first digit can then be dropped using a delimited argument. The
%   groups of three digits are then assigned to the various parts of
%   the input: notice that |##9| contains the last two digits of the
%   smallest part of the input.
%    \begin{macrocode}
\cs_new_protected_nopar:Npn \fp_mul_split:NNNN #1#2#3#4
  {
    \int_advance:w #1 \c_one_thousand_million
    \cs_set_protected_nopar:Npn \fp_mul_split_aux:w
       ##1##2##3##4##5##6##7##8##9 \q_stop {
        #2 ##2##3##4 \scan_stop:
        #3 ##5##6##7 \scan_stop:
        #4 ##8##9    \scan_stop:
      }
    \exp_after:wN \fp_mul_split_aux:w \int_use:N #1 \q_stop
    \int_advance:w #1 -\c_one_thousand_million
  }
\cs_new_protected_nopar:Npn \fp_mul_product:NN #1#2
  {
    \l_fp_mul_output_int
      \int_eval:w \l_fp_mul_output_int + #1 * #2 \scan_stop:
  }
%    \end{macrocode}
%   At the end of each output group of three, there is a transfer of
%   information so that there is no danger of an overflow. This is done by
%   expansion to keep the number of calculations down.
%    \begin{macrocode}
\cs_new_protected_nopar:Npn \fp_mul_end_level:
  {
    \int_advance:w \l_fp_mul_output_int \c_one_thousand_million
    \exp_after:wN \use_i:nn \exp_after:wN
      \fp_mul_end_level:NNNNNNNNN \int_use:N \l_fp_mul_output_int
  }
\cs_new_protected_nopar:Npn \fp_mul_end_level:NNNNNNNNN #1#2#3#4#5#6#7#8#9
  {
    \tl_set:Nx \l_fp_mul_output_tl { #7#8#9 \l_fp_mul_output_tl }
    \l_fp_mul_output_int #1#2#3#4#5#6 \scan_stop:
  }
%    \end{macrocode}
% \end{macro}
% \end{macro}
% \end{macro}
% \end{macro}
% \end{macro}
% \end{macro}
% \end{macro}
% \end{macro}
%
% \begin{macro}{\fp_div:Nn, \fp_div:cn}
% \UnitTested
% \begin{macro}{\fp_gdiv:Nn,\fp_gdiv:cn}
% \UnitTested
% \begin{macro}[aux]{\fp_div_aux:NNn}
% \begin{macro}{\fp_div_internal:}
% \begin{macro}[aux]{\fp_div_loop:}
% \begin{macro}[aux]{\fp_div_divide:}
% \begin{macro}[aux]{\fp_div_divide_aux:}
% \begin{macro}[aux]{\fp_div_store:}
% \begin{macro}[aux]{\fp_div_store_integer:}
% \begin{macro}[aux]{\fp_div_store_decimal:}
%   The pattern is much the same for multiplication.
%    \begin{macrocode}
\cs_new_protected_nopar:Npn \fp_div:Nn  { \fp_div_aux:NNn \tl_set:Nn }
\cs_new_protected_nopar:Npn \fp_gdiv:Nn { \fp_div_aux:NNn \tl_gset:Nn }
\cs_generate_variant:Nn \fp_div:Nn  { c }
\cs_generate_variant:Nn \fp_gdiv:Nn { c }
%    \end{macrocode}
%   Division proper starts with a couple of tests. If the denominator is
%   zero then a error is issued. On the other hand, if the numerator is
%   zero then the result must be $0.0$ and can be given with no
%   further work.
%    \begin{macrocode}
\cs_new_protected_nopar:Npn \fp_div_aux:NNn #1#2#3
  {
    \group_begin:
      \fp_read:N #2
      \fp_split:Nn b {#3}
      \fp_standardise:NNNN
        \l_fp_input_b_sign_int
        \l_fp_input_b_integer_int
        \l_fp_input_b_decimal_int
        \l_fp_input_b_exponent_int
      \if_int_compare:w
        \int_eval:w
          \l_fp_input_b_integer_int + \l_fp_input_b_decimal_int
          = \c_zero
        \cs_set_protected_nopar:Npx \fp_tmp:w ##1##2
          {
            \group_end:
            #1 \exp_not:N #2 { \c_undefined_fp }
          }
      \else:
        \if_int_compare:w
          \int_eval:w
            \l_fp_input_a_integer_int + \l_fp_input_a_decimal_int
            = \c_zero
          \cs_set_protected_nopar:Npx \fp_tmp:w ##1##2
            {
              \group_end:
              #1 \exp_not:N #2 { \c_zero_fp }
            }
        \else:
          \exp_after:wN \exp_after:wN \exp_after:wN \fp_div_internal:
        \fi:
      \fi:
    \fp_tmp:w #1#2
  }
%    \end{macrocode}
%   The main division algorithm works by finding how many times |b| can
%   be removed from |a|, storing the result and doing the subtraction.
%   Input |a| is then multiplied by $10$, and the process is repeated.
%   The looping ends either when there is nothing left of |a|
%   (\emph{i.e.}~an exact result) or when the code reaches the ninth
%   decimal place. Most of the process takes place in the loop function
%   below.
%    \begin{macrocode}
\cs_new_protected_nopar:Npn \fp_div_internal: {
  \l_fp_output_integer_int \c_zero
  \l_fp_output_decimal_int \c_zero
  \cs_set_eq:NN \fp_div_store: \fp_div_store_integer:
  \l_fp_div_offset_int \c_one_hundred_million
  \fp_div_loop:
  \l_fp_output_exponent_int
    \int_eval:w
      \l_fp_input_a_exponent_int - \l_fp_input_b_exponent_int
    \scan_stop:
  \fp_standardise:NNNN
    \l_fp_output_sign_int
    \l_fp_output_integer_int
    \l_fp_output_decimal_int
    \l_fp_output_exponent_int
  \cs_set_protected_nopar:Npx \fp_tmp:w ##1##2
    {
      \group_end:
      ##1 ##2
        {
          \if_int_compare:w
            \int_eval:w
              \l_fp_input_a_sign_int * \l_fp_input_b_sign_int
              < \c_zero
            \if_int_compare:w
              \int_eval:w
                \l_fp_output_integer_int + \l_fp_output_decimal_int
                = \c_zero
              +
            \else:
              -
            \fi:
          \else:
            +
          \fi:
          \int_use:N \l_fp_output_integer_int
          .
          \exp_after:wN \use_none:n
            \int_value:w \int_eval:w
               \l_fp_output_decimal_int + \c_one_thousand_million
             \int_eval_end:
          e
          \int_use:N \l_fp_output_exponent_int
        }
    }
}
%    \end{macrocode}
%   The main loop implements the approach described above. The storing
%   function is done as a function so that the integer and decimal parts
%   can be done separately but rapidly.
%    \begin{macrocode}
\cs_new_protected_nopar:Npn \fp_div_loop:
  {
    \l_fp_count_int \c_zero
    \fp_div_divide:
    \fp_div_store:
    \tex_multiply:D \l_fp_input_a_integer_int \c_ten
    \int_advance:w \l_fp_input_a_decimal_int \c_one_thousand_million
    \exp_after:wN \fp_div_loop_step:w
      \int_use:N \l_fp_input_a_decimal_int \q_stop
    \if_int_compare:w
      \int_eval:w \l_fp_input_a_integer_int + \l_fp_input_a_decimal_int
        > \c_zero
        \if_int_compare:w \l_fp_div_offset_int > \c_zero
          \exp_after:wN \exp_after:wN \exp_after:wN
            \fp_div_loop:
        \fi:
    \fi:
  }
%    \end{macrocode}
%   Checking to see if the numerator can be divides needs quite an
%   involved check. Either the integer part has to be bigger for the
%   numerator or, if it is not smaller then the decimal part of the
%   numerator must not be smaller than that of the denominator. Once
%   the test is right the rest is much as elsewhere.
%    \begin{macrocode}
\cs_new_protected_nopar:Npn \fp_div_divide:
  {
    \if_int_compare:w \l_fp_input_a_integer_int > \l_fp_input_b_integer_int
      \exp_after:wN \fp_div_divide_aux:
    \else:
      \if_int_compare:w \l_fp_input_a_integer_int < \l_fp_input_b_integer_int
      \else:
        \if_int_compare:w
          \l_fp_input_a_decimal_int < \l_fp_input_b_decimal_int
        \else:
          \exp_after:wN \exp_after:wN \exp_after:wN
            \exp_after:wN \exp_after:wN \exp_after:wN
            \exp_after:wN \fp_div_divide_aux:
        \fi:
      \fi:
    \fi:
  }
\cs_new_protected_nopar:Npn \fp_div_divide_aux:
  {
    \int_advance:w \l_fp_count_int \c_one
    \int_advance:w \l_fp_input_a_integer_int -\l_fp_input_b_integer_int
    \int_advance:w \l_fp_input_a_decimal_int -\l_fp_input_b_decimal_int
    \if_int_compare:w \l_fp_input_a_decimal_int < \c_zero
      \int_advance:w \l_fp_input_a_integer_int \c_minus_one
      \int_advance:w \l_fp_input_a_decimal_int \c_one_thousand_million
    \fi:
    \fp_div_divide:
  }
%    \end{macrocode}
%   Storing the number of each division is done differently for the
%   integer and decimal. The integer is easy and a one-off, while the
%   decimal also needs to account for the position of the digit to store.
%    \begin{macrocode}
\cs_new_protected_nopar:Npn \fp_div_store: { }
\cs_new_protected_nopar:Npn \fp_div_store_integer:
  {
    \l_fp_output_integer_int \l_fp_count_int
    \cs_set_eq:NN \fp_div_store: \fp_div_store_decimal:
  }
\cs_new_protected_nopar:Npn \fp_div_store_decimal:
  {
    \l_fp_output_decimal_int
      \int_eval:w
        \l_fp_output_decimal_int +
        \l_fp_count_int * \l_fp_div_offset_int
      \int_eval_end:
    \tex_divide:D \l_fp_div_offset_int \c_ten
  }
\cs_new_protected_nopar:Npn \fp_div_loop_step:w #1#2#3#4#5#6#7#8#9 \q_stop
  {
    \l_fp_input_a_integer_int
      \int_eval:w #2 + \l_fp_input_a_integer_int  \int_eval_end:
    \l_fp_input_a_decimal_int #3#4#5#6#7#8#9 0 \scan_stop:
  }
%    \end{macrocode}
% \end{macro}
% \end{macro}
% \end{macro}
% \end{macro}
% \end{macro}
% \end{macro}
% \end{macro}
% \end{macro}
% \end{macro}
% \end{macro}
%
% \subsection{Arithmetic for internal use}
%
% For the more complex functions, it is only possible to deliver
% reliable $10$ digit accuracy if the internal calculations are
% carried out to a higher degree of precision. This is done using a
% second set of functions so that the `user' versions are not
% slowed down. These versions are also focussed on the needs of internal
% calculations. No error checking, sign checking or exponent levelling
% is done. For addition and subtraction, the arguments are:
% \begin{itemize}
%   \item Integer part of input |a|.
%   \item Decimal part of input |a|.
%   \item Additional decimal part of input |a|.
%   \item Integer part of input |b|.
%   \item Decimal part of input |b|.
%   \item Additional decimal part of input |b|.
%   \item Integer part of output.
%   \item Decimal part of output.
%   \item Additional decimal part of output.
% \end{itemize}
% The situation for multiplication and division is a little different as
% they only deal with the decimal part.
%
% \begin{macro}{\fp_add:NNNNNNNNN}
%   The internal sum is always exactly that: it is always a sum and there
%   is no sign check.
%    \begin{macrocode}
\cs_new_protected_nopar:Npn \fp_add:NNNNNNNNN #1#2#3#4#5#6#7#8#9
  {
    #7 \int_eval:w #1 + #4 \int_eval_end:
    #8 \int_eval:w #2 + #5 \int_eval_end:
    #9 \int_eval:w #3 + #6 \int_eval_end:
    \if_int_compare:w #9 < \c_one_thousand_million
    \else:
      \int_advance:w #8 \c_one
      \int_advance:w #9 -\c_one_thousand_million
    \fi:
    \if_int_compare:w #8 < \c_one_thousand_million
    \else:
      \int_advance:w #7 \c_one
      \int_advance:w #8 -\c_one_thousand_million
    \fi:
  }
%    \end{macrocode}
% \end{macro}
%
% \begin{macro}{\fp_sub:NNNNNNNNN}
%   Internal subtraction is needed only when the first number is bigger
%   than the second, so there is no need to worry about the sign. This is
%   a good job as there are no arguments left. The flipping flag is
%   used in the rare case where a sign change is possible.
%    \begin{macrocode}
\cs_new_protected_nopar:Npn \fp_sub:NNNNNNNNN #1#2#3#4#5#6#7#8#9
  {
    #7 \int_eval:w #1 - #4 \int_eval_end:
    #8 \int_eval:w #2 - #5 \int_eval_end:
    #9 \int_eval:w #3 - #6 \int_eval_end:
    \if_int_compare:w #9 < \c_zero
      \int_advance:w #8 \c_minus_one
      \int_advance:w #9 \c_one_thousand_million
    \fi:
    \if_int_compare:w #8 < \c_zero
      \int_advance:w #7 \c_minus_one
      \int_advance:w #8 \c_one_thousand_million
    \fi:
    \if_int_compare:w #7 < \c_zero
      \if_int_compare:w \int_eval:w #8 + #9 = \c_zero
        #7 -#7
      \else:
        \int_advance:w #7 \c_one
        #8 \int_eval:w \c_one_thousand_million - #8 \int_eval_end:
        #9 \int_eval:w \c_one_thousand_million - #9 \int_eval_end:
      \fi:
    \fi:
  }
%    \end{macrocode}
% \end{macro}
%
% \begin{macro}{\fp_mul:NNNNNN}
% Decimal-part only multiplication but with higher accuracy than the
% user version.
%    \begin{macrocode}
\cs_new_protected_nopar:Npn \fp_mul:NNNNNN #1#2#3#4#5#6
  {
    \fp_mul_split:NNNN #1
      \l_fp_mul_a_i_int \l_fp_mul_a_ii_int \l_fp_mul_a_iii_int
    \fp_mul_split:NNNN #2
      \l_fp_mul_a_iv_int \l_fp_mul_a_v_int \l_fp_mul_a_vi_int
    \fp_mul_split:NNNN #3
      \l_fp_mul_b_i_int \l_fp_mul_b_ii_int \l_fp_mul_b_iii_int
    \fp_mul_split:NNNN #4
      \l_fp_mul_b_iv_int \l_fp_mul_b_v_int \l_fp_mul_b_vi_int
    \l_fp_mul_output_int \c_zero
    \tl_clear:N \l_fp_mul_output_tl
    \fp_mul_product:NN \l_fp_mul_a_i_int         \l_fp_mul_b_vi_int
    \fp_mul_product:NN \l_fp_mul_a_ii_int        \l_fp_mul_b_v_int
    \fp_mul_product:NN \l_fp_mul_a_iii_int       \l_fp_mul_b_iv_int
    \fp_mul_product:NN \l_fp_mul_a_iv_int        \l_fp_mul_b_iii_int
    \fp_mul_product:NN \l_fp_mul_a_v_int         \l_fp_mul_b_ii_int
    \fp_mul_product:NN \l_fp_mul_a_vi_int        \l_fp_mul_b_i_int
    \tex_divide:D \l_fp_mul_output_int \c_one_thousand
    \fp_mul_product:NN \l_fp_mul_a_i_int        \l_fp_mul_b_v_int
    \fp_mul_product:NN \l_fp_mul_a_ii_int       \l_fp_mul_b_iv_int
    \fp_mul_product:NN \l_fp_mul_a_iii_int      \l_fp_mul_b_iii_int
    \fp_mul_product:NN \l_fp_mul_a_iv_int       \l_fp_mul_b_ii_int
    \fp_mul_product:NN \l_fp_mul_a_v_int        \l_fp_mul_b_i_int
    \fp_mul_end_level:
    \fp_mul_product:NN \l_fp_mul_a_i_int        \l_fp_mul_b_iv_int
    \fp_mul_product:NN \l_fp_mul_a_ii_int       \l_fp_mul_b_iii_int
    \fp_mul_product:NN \l_fp_mul_a_iii_int      \l_fp_mul_b_ii_int
    \fp_mul_product:NN \l_fp_mul_a_iv_int       \l_fp_mul_b_i_int
    \fp_mul_end_level:
    \fp_mul_product:NN \l_fp_mul_a_i_int        \l_fp_mul_b_iii_int
    \fp_mul_product:NN \l_fp_mul_a_ii_int       \l_fp_mul_b_ii_int
    \fp_mul_product:NN \l_fp_mul_a_iii_int      \l_fp_mul_b_i_int
    \fp_mul_end_level:
    #6 0 \l_fp_mul_output_tl \scan_stop:
    \tl_clear:N \l_fp_mul_output_tl
    \fp_mul_product:NN \l_fp_mul_a_i_int        \l_fp_mul_b_ii_int
    \fp_mul_product:NN \l_fp_mul_a_ii_int       \l_fp_mul_b_i_int
    \fp_mul_end_level:
    \fp_mul_product:NN \l_fp_mul_a_i_int        \l_fp_mul_b_i_int
    \fp_mul_end_level:
    \fp_mul_end_level:
    #5 0 \l_fp_mul_output_tl \scan_stop:
  }
%    \end{macrocode}
% \end{macro}
%
% \begin{macro}{\fp_mul:NNNNNNNNN}
%   For internal multiplication where the integer does need to be
%   retained. This means of course that this code is quite slow, and so
%   is only used when necessary.
%    \begin{macrocode}
\cs_new_protected_nopar:Npn \fp_mul:NNNNNNNNN #1#2#3#4#5#6#7#8#9
  {
    \fp_mul_split:NNNN #2
      \l_fp_mul_a_i_int \l_fp_mul_a_ii_int \l_fp_mul_a_iii_int
    \fp_mul_split:NNNN #3
      \l_fp_mul_a_iv_int \l_fp_mul_a_v_int \l_fp_mul_a_vi_int
    \fp_mul_split:NNNN #5
      \l_fp_mul_b_i_int \l_fp_mul_b_ii_int \l_fp_mul_b_iii_int
    \fp_mul_split:NNNN #6
      \l_fp_mul_b_iv_int \l_fp_mul_b_v_int \l_fp_mul_b_vi_int
    \l_fp_mul_output_int \c_zero
    \tl_clear:N \l_fp_mul_output_tl
    \fp_mul_product:NN \l_fp_mul_a_i_int         \l_fp_mul_b_vi_int
    \fp_mul_product:NN \l_fp_mul_a_ii_int        \l_fp_mul_b_v_int
    \fp_mul_product:NN \l_fp_mul_a_iii_int       \l_fp_mul_b_iv_int
    \fp_mul_product:NN \l_fp_mul_a_iv_int        \l_fp_mul_b_iii_int
    \fp_mul_product:NN \l_fp_mul_a_v_int         \l_fp_mul_b_ii_int
    \fp_mul_product:NN \l_fp_mul_a_vi_int        \l_fp_mul_b_i_int
    \tex_divide:D \l_fp_mul_output_int \c_one_thousand
    \fp_mul_product:NN #1                       \l_fp_mul_b_vi_int
    \fp_mul_product:NN \l_fp_mul_a_i_int        \l_fp_mul_b_v_int
    \fp_mul_product:NN \l_fp_mul_a_ii_int       \l_fp_mul_b_iv_int
    \fp_mul_product:NN \l_fp_mul_a_iii_int      \l_fp_mul_b_iii_int
    \fp_mul_product:NN \l_fp_mul_a_iv_int       \l_fp_mul_b_ii_int
    \fp_mul_product:NN \l_fp_mul_a_v_int        \l_fp_mul_b_i_int
    \fp_mul_product:NN \l_fp_mul_a_vi_int       #4
    \fp_mul_end_level:
    \fp_mul_product:NN #1                       \l_fp_mul_b_v_int
    \fp_mul_product:NN \l_fp_mul_a_i_int        \l_fp_mul_b_iv_int
    \fp_mul_product:NN \l_fp_mul_a_ii_int       \l_fp_mul_b_iii_int
    \fp_mul_product:NN \l_fp_mul_a_iii_int      \l_fp_mul_b_ii_int
    \fp_mul_product:NN \l_fp_mul_a_iv_int       \l_fp_mul_b_i_int
    \fp_mul_product:NN \l_fp_mul_a_v_int        #4
    \fp_mul_end_level:
    \fp_mul_product:NN #1                       \l_fp_mul_b_iv_int
    \fp_mul_product:NN \l_fp_mul_a_i_int        \l_fp_mul_b_iii_int
    \fp_mul_product:NN \l_fp_mul_a_ii_int       \l_fp_mul_b_ii_int
    \fp_mul_product:NN \l_fp_mul_a_iii_int      \l_fp_mul_b_i_int
    \fp_mul_product:NN \l_fp_mul_a_iv_int       #4
    \fp_mul_end_level:
    #9 0 \l_fp_mul_output_tl \scan_stop:
    \tl_clear:N \l_fp_mul_output_tl
    \fp_mul_product:NN #1                       \l_fp_mul_b_iii_int
    \fp_mul_product:NN \l_fp_mul_a_i_int        \l_fp_mul_b_ii_int
    \fp_mul_product:NN \l_fp_mul_a_ii_int       \l_fp_mul_b_i_int
    \fp_mul_product:NN \l_fp_mul_a_iii_int      #4
    \fp_mul_end_level:
    \fp_mul_product:NN #1                       \l_fp_mul_b_ii_int
    \fp_mul_product:NN \l_fp_mul_a_i_int        \l_fp_mul_b_i_int
    \fp_mul_product:NN \l_fp_mul_a_ii_int       #4
    \fp_mul_end_level:
    \fp_mul_product:NN #1                       \l_fp_mul_b_i_int
    \fp_mul_product:NN \l_fp_mul_a_i_int        #4
    \fp_mul_end_level:
    #8 0 \l_fp_mul_output_tl \scan_stop:
    \tl_clear:N \l_fp_mul_output_tl
    \fp_mul_product:NN #1 #4
    \fp_mul_end_level:
    #7 0 \l_fp_mul_output_tl \scan_stop:
  }
%    \end{macrocode}
% \end{macro}
%
% \begin{macro}{\fp_div_integer:NNNNN}
%   Here, division is always by an integer, and so it is possible to
%   use \TeX{}'s native calculations rather than doing it in macros.
%   The idea here is to divide the decimal part, find any remainder,
%   then do the real division of the two parts before adding in what
%   is needed for the remainder.
%    \begin{macrocode}
\cs_new_protected_nopar:Npn \fp_div_integer:NNNNN #1#2#3#4#5
  {
    \l_fp_tmp_int #1
    \tex_divide:D \l_fp_tmp_int #3
    \l_fp_tmp_int \int_eval:w #1 - \l_fp_tmp_int * #3 \int_eval_end:
    #4 #1
    \tex_divide:D #4 #3
    #5 #2
    \tex_divide:D #5 #3
    \tex_multiply:D \l_fp_tmp_int \c_one_thousand
    \tex_divide:D \l_fp_tmp_int #3
    #5 \int_eval:w #5 + \l_fp_tmp_int * \c_one_million \int_eval_end:
    \if_int_compare:w #5 > \c_one_thousand_million
      \int_advance:w #4 \c_one
      \int_advance:w #5 -\c_one_thousand_million
    \fi:
  }
%    \end{macrocode}
% \end{macro}
%
% \begin{macro}{\fp_extended_normalise:}
% \begin{macro}[aux]{\fp_extended_normalise_aux_i:}
% \begin{macro}[aux]{\fp_extended_normalise_aux_i:w}
% \begin{macro}[aux]{\fp_extended_normalise_aux_ii:w}
% \begin{macro}[aux]{\fp_extended_normalise_aux_ii:}
% \begin{macro}[aux]{\fp_extended_normalise_aux:NNNNNNNNN}
%   The \enquote{extended} integers for internal use are mainly used in
%   fixed-point mode. This comes up in a few places, so a generalised
%   utility is made available to carry out the change. This function
%   simply calls the two loops to shift the input to the point of
%   having a zero exponent.
%    \begin{macrocode}
\cs_new_protected_nopar:Npn \fp_extended_normalise:
  {
    \fp_extended_normalise_aux_i:
    \fp_extended_normalise_aux_ii:
  }
\cs_new_protected_nopar:Npn \fp_extended_normalise_aux_i:
  {
    \if_int_compare:w \l_fp_input_a_exponent_int > \c_zero
      \tex_multiply:D \l_fp_input_a_integer_int \c_ten
      \int_advance:w \l_fp_input_a_decimal_int \c_one_thousand_million
      \exp_after:wN \fp_extended_normalise_aux_i:w
        \int_use:N \l_fp_input_a_decimal_int \q_stop
       \exp_after:wN \fp_extended_normalise_aux_i:
     \fi:
  }
\cs_new_protected_nopar:Npn \fp_extended_normalise_aux_i:w
  #1#2#3#4#5#6#7#8#9 \q_stop
  {
    \l_fp_input_a_integer_int
      \int_eval:w \l_fp_input_a_integer_int + #2 \scan_stop:
    \l_fp_input_a_decimal_int #3#4#5#6#7#8#9 0 \scan_stop:
    \int_advance:w \l_fp_input_a_extended_int \c_one_thousand_million
    \exp_after:wN \fp_extended_normalise_aux_ii:w
      \int_use:N \l_fp_input_a_extended_int \q_stop
  }
\cs_new_protected_nopar:Npn \fp_extended_normalise_aux_ii:w
  #1#2#3#4#5#6#7#8#9 \q_stop
  {
    \l_fp_input_a_decimal_int
      \int_eval:w \l_fp_input_a_decimal_int + #2 \scan_stop:
    \l_fp_input_a_extended_int #3#4#5#6#7#8#9 0 \scan_stop:
    \int_advance:w \l_fp_input_a_exponent_int \c_minus_one
  }
\cs_new_protected_nopar:Npn \fp_extended_normalise_aux_ii:
  {
    \if_int_compare:w \l_fp_input_a_exponent_int < \c_zero
      \int_advance:w \l_fp_input_a_decimal_int \c_one_thousand_million
      \exp_after:wN \use_i:nn \exp_after:wN
        \fp_extended_normalise_ii_aux:NNNNNNNNN
        \int_use:N \l_fp_input_a_decimal_int
       \exp_after:wN \fp_extended_normalise_aux_ii:
     \fi:
  }
\cs_new_protected_nopar:Npn \fp_extended_normalise_ii_aux:NNNNNNNNN
  #1#2#3#4#5#6#7#8#9
  {
    \if_int_compare:w \l_fp_input_a_integer_int = \c_zero
      \l_fp_input_a_decimal_int #1#2#3#4#5#6#7#8 \scan_stop:
    \else:
      \tl_set:Nx \l_fp_tmp_tl
        {
          \int_use:N \l_fp_input_a_integer_int
          #1#2#3#4#5#6#7#8
        }
      \l_fp_input_a_integer_int \c_zero
      \l_fp_input_a_decimal_int \l_fp_tmp_tl \scan_stop:
    \fi:
    \tex_divide:D \l_fp_input_a_extended_int \c_ten
    \tl_set:Nx \l_fp_tmp_tl
      {
        #9
        \int_use:N \l_fp_input_a_extended_int
      }
    \l_fp_input_a_extended_int \l_fp_tmp_tl \scan_stop:
    \int_advance:w \l_fp_input_a_exponent_int \c_one
  }
%    \end{macrocode}
% \end{macro}
% \end{macro}
% \end{macro}
% \end{macro}
% \end{macro}
% \end{macro}
%
% \begin{macro}{\fp_extended_normalise_output:}
% \begin{macro}[aux]{\fp_extended_normalise_output_aux_i:NNNNNNNNN}
% \begin{macro}[aux]{\fp_extended_normalise_output_aux_ii:NNNNNNNNN}
% \begin{macro}[aux]{\fp_extended_normalise_output_aux:N}
%   At some stages in working out extended output, it is possible for the
%   value to need shifting to keep the integer part in range. This only
%   ever happens such that the integer needs to be made smaller.
%    \begin{macrocode}
\cs_new_protected_nopar:Npn \fp_extended_normalise_output:
  {
    \if_int_compare:w \l_fp_output_integer_int > \c_nine
      \int_advance:w \l_fp_output_integer_int \c_one_thousand_million
      \exp_after:wN \use_i:nn \exp_after:wN
        \fp_extended_normalise_output_aux_i:NNNNNNNNN
        \int_use:N \l_fp_output_integer_int
      \exp_after:wN \fp_extended_normalise_output:
    \fi:
  }
\cs_new_protected_nopar:Npn \fp_extended_normalise_output_aux_i:NNNNNNNNN
  #1#2#3#4#5#6#7#8#9
  {
    \l_fp_output_integer_int #1#2#3#4#5#6#7#8 \scan_stop:
    \int_advance:w \l_fp_output_decimal_int \c_one_thousand_million
    \tl_set:Nx \l_fp_tmp_tl
      {
        #9
        \exp_after:wN \use_none:n
        \int_use:N \l_fp_output_decimal_int
      }
    \exp_after:wN \fp_extended_normalise_output_aux_ii:NNNNNNNNN
      \l_fp_tmp_tl
  }
\cs_new_protected_nopar:Npn \fp_extended_normalise_output_aux_ii:NNNNNNNNN
  #1#2#3#4#5#6#7#8#9
  {
    \l_fp_output_decimal_int #1#2#3#4#5#6#7#8#9 \scan_stop:
    \fp_extended_normalise_output_aux:N
  }
\cs_new_protected_nopar:Npn \fp_extended_normalise_output_aux:N #1
  {
    \int_advance:w \l_fp_output_extended_int \c_one_thousand_million
    \tex_divide:D \l_fp_output_extended_int \c_ten
    \tl_set:Nx \l_fp_tmp_tl
      {
        #1
        \exp_after:wN \use_none:n
          \int_use:N \l_fp_output_extended_int
      }
    \l_fp_output_extended_int \l_fp_tmp_tl \scan_stop:
    \int_advance:w \l_fp_output_exponent_int \c_one
  }
%    \end{macrocode}
% \end{macro}
% \end{macro}
% \end{macro}
% \end{macro}
%
% \subsection{Trigonometric functions}
%
% \begin{macro}{\fp_trig_normalise:}
% \begin{macro}[aux]{\fp_trig_normalise_aux:}
% \begin{macro}[aux]{\fp_trig_sub:NNN}
%   For normalisation, the code essentially switches to fixed-point
%   arithmetic. There is a shift of the exponent, then repeated
%   subtractions. The end result is a number in the range
%   $ -\pi < x \le \pi $.
%    \begin{macrocode}
\cs_new_protected_nopar:Npn \fp_trig_normalise:
  {
    \if_int_compare:w \l_fp_input_a_exponent_int < \c_ten
      \l_fp_input_a_extended_int \c_zero
      \fp_extended_normalise:
      \fp_trig_normalise_aux:
      \if_int_compare:w \l_fp_input_a_integer_int < \c_zero
        \l_fp_input_a_sign_int -\l_fp_input_a_sign_int
        \l_fp_input_a_integer_int -\l_fp_input_a_integer_int
      \fi:
       \exp_after:wN \fp_trig_octant:
    \else:
      \l_fp_input_a_sign_int    \c_one
      \l_fp_output_integer_int  \c_zero
      \l_fp_output_decimal_int  \c_zero
      \l_fp_output_exponent_int \c_zero
      \exp_after:wN \fp_trig_overflow_msg:
    \fi:
  }
\cs_new_protected_nopar:Npn \fp_trig_normalise_aux:
  {
    \if_int_compare:w \l_fp_input_a_integer_int > \c_three
      \fp_trig_sub:NNN
        \c_six \c_fp_two_pi_decimal_int \c_fp_two_pi_extended_int
      \exp_after:wN \fp_trig_normalise_aux:
    \else:
      \if_int_compare:w \l_fp_input_a_integer_int > \c_two
        \if_int_compare:w \l_fp_input_a_decimal_int > \c_fp_pi_decimal_int
          \fp_trig_sub:NNN
            \c_six \c_fp_two_pi_decimal_int \c_fp_two_pi_extended_int
          \exp_after:wN \exp_after:wN \exp_after:wN
          \exp_after:wN \exp_after:wN \exp_after:wN
            \exp_after:wN \fp_trig_normalise_aux:
          \fi:
      \fi:
    \fi:
  }
%    \end{macrocode}
%   Here, there may be a sign change but there will never be any
%   variation in the input. So a dedicated function can be used.
%    \begin{macrocode}
\cs_new_protected_nopar:Npn \fp_trig_sub:NNN #1#2#3
  {
    \l_fp_input_a_integer_int
      \int_eval:w \l_fp_input_a_integer_int - #1 \int_eval_end:
    \l_fp_input_a_decimal_int
      \int_eval:w \l_fp_input_a_decimal_int - #2 \int_eval_end:
    \l_fp_input_a_extended_int
      \int_eval:w \l_fp_input_a_extended_int - #3 \int_eval_end:
    \if_int_compare:w \l_fp_input_a_extended_int < \c_zero
      \int_advance:w \l_fp_input_a_decimal_int \c_minus_one
      \int_advance:w \l_fp_input_a_extended_int \c_one_thousand_million
    \fi:
    \if_int_compare:w \l_fp_input_a_decimal_int < \c_zero
      \int_advance:w \l_fp_input_a_integer_int \c_minus_one
      \int_advance:w \l_fp_input_a_decimal_int \c_one_thousand_million
    \fi:
    \if_int_compare:w \l_fp_input_a_integer_int < \c_zero
      \l_fp_input_a_sign_int -\l_fp_input_a_sign_int
      \if_int_compare:w
        \int_eval:w
          \l_fp_input_a_decimal_int + \l_fp_input_a_extended_int
        = \c_zero
        \l_fp_input_a_integer_int -\l_fp_input_a_integer_int
      \else:
        \l_fp_input_a_integer_int
           \int_eval:w
             - \l_fp_input_a_integer_int - \c_one
           \int_eval_end:
        \l_fp_input_a_decimal_int
          \int_eval:w
            \c_one_thousand_million - \l_fp_input_a_decimal_int
          \int_eval_end:
        \l_fp_input_a_extended_int
          \int_eval:w
            \c_one_thousand_million - \l_fp_input_a_extended_int
          \int_eval_end:
      \fi:
    \fi:
  }
%    \end{macrocode}
% \end{macro}
% \end{macro}
% \end{macro}
%
% \begin{macro}{\fp_trig_octant:}
% \begin{macro}[aux]{\fp_trig_octant_aux:}
%   Here, the input is further reduced into the range
%   $ 0 \le x < \pi / 4 $. This is pretty simple: check if
%   $ \pi / 4 $ can be taken off and if it can do it and loop. The
%   check at the end is to \enquote{mop up} values which are so close to
%   $ \pi / 4 $ that they should be treated as such.  The test for
%   an even octant is needed as the `remainder' needed is from
%   the nearest $ \pi / 2 $.
%    \begin{macrocode}
\cs_new_protected_nopar:Npn \fp_trig_octant:
  {
    \l_fp_trig_octant_int \c_one
    \fp_trig_octant_aux:
    \if_int_compare:w \l_fp_input_a_decimal_int < \c_ten
      \l_fp_input_a_decimal_int  \c_zero
      \l_fp_input_a_extended_int \c_zero
    \fi:
    \if_int_odd:w \l_fp_trig_octant_int
    \else:
      \fp_sub:NNNNNNNNN
        \c_zero \c_fp_pi_by_four_decimal_int \c_fp_pi_by_four_extended_int
        \l_fp_input_a_integer_int \l_fp_input_a_decimal_int
          \l_fp_input_a_extended_int
        \l_fp_input_a_integer_int \l_fp_input_a_decimal_int
          \l_fp_input_a_extended_int
    \fi:
  }
\cs_new_protected_nopar:Npn \fp_trig_octant_aux:
  {
    \if_int_compare:w \l_fp_input_a_integer_int > \c_zero
      \fp_sub:NNNNNNNNN
        \l_fp_input_a_integer_int \l_fp_input_a_decimal_int
          \l_fp_input_a_extended_int
        \c_zero \c_fp_pi_by_four_decimal_int \c_fp_pi_by_four_extended_int
        \l_fp_input_a_integer_int \l_fp_input_a_decimal_int
          \l_fp_input_a_extended_int
      \int_advance:w \l_fp_trig_octant_int \c_one
      \exp_after:wN \fp_trig_octant_aux:
    \else:
      \if_int_compare:w
        \l_fp_input_a_decimal_int > \c_fp_pi_by_four_decimal_int
        \fp_sub:NNNNNNNNN
          \l_fp_input_a_integer_int \l_fp_input_a_decimal_int
            \l_fp_input_a_extended_int
          \c_zero \c_fp_pi_by_four_decimal_int
            \c_fp_pi_by_four_extended_int
          \l_fp_input_a_integer_int \l_fp_input_a_decimal_int
            \l_fp_input_a_extended_int
        \int_advance:w \l_fp_trig_octant_int \c_one
        \exp_after:wN \exp_after:wN \exp_after:wN
          \fp_trig_octant_aux:
      \fi:
    \fi:
  }
%    \end{macrocode}
% \end{macro}
% \end{macro}
%
% \begin{macro}{\fp_sin:Nn, \fp_sin:cn}
% \UnitTested
% \begin{macro}{\fp_gsin:Nn,\fp_gsin:cn}
% \UnitTested
% \begin{macro}[aux]{\fp_sin_aux:NNn}
% \begin{macro}[aux]{\fp_sin_aux_i:}
% \begin{macro}[aux]{\fp_sin_aux_ii:}
%   Calculating the sine starts off in the usual way. There is a check
%   to see if the value has already been worked out before proceeding
%   further.
%    \begin{macrocode}
\cs_new_protected_nopar:Npn \fp_sin:Nn  { \fp_sin_aux:NNn \tl_set:Nn }
\cs_new_protected_nopar:Npn \fp_gsin:Nn { \fp_sin_aux:NNn \tl_gset:Nn }
\cs_generate_variant:Nn \fp_sin:Nn   { c }
\cs_generate_variant:Nn \fp_gsin:Nn  { c }
%    \end{macrocode}
%   The internal routine for sines does a check to see if the value is
%   already known. This saves a lot of repetition when doing rotations.
%   For very small values it is best to simply return the input as the
%   sine: the cut-off is $ 1 \times 10^{-5} $.
%    \begin{macrocode}
\cs_new_protected_nopar:Npn \fp_sin_aux:NNn #1#2#3
  {
    \group_begin:
      \fp_split:Nn a {#3}
      \fp_standardise:NNNN
        \l_fp_input_a_sign_int
        \l_fp_input_a_integer_int
        \l_fp_input_a_decimal_int
        \l_fp_input_a_exponent_int
      \tl_set:Nx \l_fp_arg_tl
        {
          \if_int_compare:w \l_fp_input_a_sign_int < \c_zero
            -
          \else:
            +
          \fi:
          \int_use:N \l_fp_input_a_integer_int
          .
          \exp_after:wN \use_none:n
            \int_value:w \int_eval:w
              \l_fp_input_a_decimal_int + \c_one_thousand_million
          e
          \int_use:N \l_fp_input_a_exponent_int
        }
      \if_int_compare:w \l_fp_input_a_exponent_int < -\c_five
        \cs_set_protected_nopar:Npx \fp_tmp:w
        {
          \group_end:
          #1 \exp_not:N #2 { \l_fp_arg_tl }
        }
      \else:
        \if_cs_exist:w
          c_fp_sin ( \l_fp_arg_tl ) _fp
        \cs_end:
        \else:
          \exp_after:wN \exp_after:wN \exp_after:wN
            \fp_sin_aux_i:
        \fi:
        \cs_set_protected_nopar:Npx \fp_tmp:w
          {
            \group_end:
            #1 \exp_not:N #2
              { \use:c { c_fp_sin ( \l_fp_arg_tl ) _fp } }
          }
      \fi:
    \fp_tmp:w
  }
%    \end{macrocode}
%   The internals for sine first normalise the input into an octant, then
%   choose the correct set up for the Taylor series. The sign for the sine
%   function is easy, so there is no worry about it. So the only thing to
%   do is to get the output standardised.
%    \begin{macrocode}
\cs_new_protected_nopar:Npn \fp_sin_aux_i:
  {
    \fp_trig_normalise:
    \fp_sin_aux_ii:
    \if_int_compare:w \l_fp_output_integer_int = \c_one
      \l_fp_output_exponent_int \c_zero
    \else:
      \l_fp_output_integer_int \l_fp_output_decimal_int
      \l_fp_output_decimal_int \l_fp_output_extended_int
      \l_fp_output_exponent_int -\c_nine
    \fi:
    \fp_standardise:NNNN
      \l_fp_input_a_sign_int
      \l_fp_output_integer_int
      \l_fp_output_decimal_int
      \l_fp_output_exponent_int
    \tl_new:c { c_fp_sin ( \l_fp_arg_tl ) _fp }
    \tl_gset:cx { c_fp_sin ( \l_fp_arg_tl ) _fp }
      {
        \if_int_compare:w \l_fp_input_a_sign_int > \c_zero
          +
        \else:
          -
        \fi:
        \int_use:N \l_fp_output_integer_int
        .
        \exp_after:wN \use_none:n
          \int_value:w \int_eval:w
             \l_fp_output_decimal_int + \c_one_thousand_million
        e
        \int_use:N \l_fp_output_exponent_int
      }
  }
\cs_new_protected_nopar:Npn \fp_sin_aux_ii:
  {
    \if_case:w \l_fp_trig_octant_int
    \or:
      \exp_after:wN \fp_trig_calc_sin:
    \or:
      \exp_after:wN \fp_trig_calc_cos:
    \or:
      \exp_after:wN \fp_trig_calc_cos:
    \or:
      \exp_after:wN \fp_trig_calc_sin:
    \fi:
  }
%    \end{macrocode}
% \end{macro}
% \end{macro}
% \end{macro}
% \end{macro}
% \end{macro}
%
% \begin{macro}{\fp_cos:Nn, \fp_cos:cn}
% \UnitTested
% \begin{macro}{\fp_gcos:Nn,\fp_gcos:cn}
% \UnitTested
% \begin{macro}[aux]{\fp_cos_aux:NNn}
% \begin{macro}[aux]{\fp_cos_aux_i:}
% \begin{macro}[aux]{\fp_cos_aux_ii:}
%   Cosine is almost identical, but there is no short cut code here.
%    \begin{macrocode}
\cs_new_protected_nopar:Npn \fp_cos:Nn  { \fp_cos_aux:NNn \tl_set:Nn }
\cs_new_protected_nopar:Npn \fp_gcos:Nn { \fp_cos_aux:NNn \tl_gset:Nn }
\cs_generate_variant:Nn \fp_cos:Nn   { c }
\cs_generate_variant:Nn \fp_gcos:Nn  { c }
\cs_new_protected_nopar:Npn \fp_cos_aux:NNn #1#2#3
  {
    \group_begin:
      \fp_split:Nn a {#3}
      \fp_standardise:NNNN
        \l_fp_input_a_sign_int
        \l_fp_input_a_integer_int
        \l_fp_input_a_decimal_int
        \l_fp_input_a_exponent_int
      \tl_set:Nx \l_fp_arg_tl
        {
          \if_int_compare:w \l_fp_input_a_sign_int < \c_zero
            -
          \else:
            +
          \fi:
          \int_use:N \l_fp_input_a_integer_int
          .
        \exp_after:wN \use_none:n
          \int_value:w \int_eval:w
             \l_fp_input_a_decimal_int + \c_one_thousand_million
          e
          \int_use:N \l_fp_input_a_exponent_int
        }
      \if_cs_exist:w c_fp_cos ( \l_fp_arg_tl ) _fp \cs_end:
      \else:
        \exp_after:wN \fp_cos_aux_i:
      \fi:
      \cs_set_protected_nopar:Npx \fp_tmp:w
        {
          \group_end:
          #1 \exp_not:N #2
            { \use:c { c_fp_cos ( \l_fp_arg_tl ) _fp } }
        }
    \fp_tmp:w
  }
%    \end{macrocode}
% Almost the same as for sine: just a bit of correction for the sign
% of the output.
%    \begin{macrocode}
\cs_new_protected_nopar:Npn \fp_cos_aux_i:
  {
    \fp_trig_normalise:
    \fp_cos_aux_ii:
    \if_int_compare:w \l_fp_output_integer_int = \c_one
      \l_fp_output_exponent_int \c_zero
    \else:
      \l_fp_output_integer_int \l_fp_output_decimal_int
      \l_fp_output_decimal_int \l_fp_output_extended_int
      \l_fp_output_exponent_int -\c_nine
    \fi:
    \fp_standardise:NNNN
      \l_fp_input_a_sign_int
      \l_fp_output_integer_int
      \l_fp_output_decimal_int
      \l_fp_output_exponent_int
    \tl_new:c { c_fp_cos ( \l_fp_arg_tl ) _fp }
    \tl_gset:cx { c_fp_cos ( \l_fp_arg_tl ) _fp }
      {
        \if_int_compare:w \l_fp_input_a_sign_int > \c_zero
          +
        \else:
          -
        \fi:
        \int_use:N \l_fp_output_integer_int
        .
        \exp_after:wN \use_none:n
          \int_value:w \int_eval:w
             \l_fp_output_decimal_int + \c_one_thousand_million
        e
        \int_use:N \l_fp_output_exponent_int
      }
  }
\cs_new_protected_nopar:Npn \fp_cos_aux_ii:
  {
    \if_case:w \l_fp_trig_octant_int
    \or:
      \exp_after:wN \fp_trig_calc_cos:
    \or:
      \exp_after:wN \fp_trig_calc_sin:
    \or:
      \exp_after:wN \fp_trig_calc_sin:
    \or:
      \exp_after:wN \fp_trig_calc_cos:
    \fi:
    \if_int_compare:w \l_fp_input_a_sign_int > \c_zero
      \if_int_compare:w \l_fp_trig_octant_int > \c_two
        \l_fp_input_a_sign_int \c_minus_one
      \fi:
    \else:
      \if_int_compare:w \l_fp_trig_octant_int > \c_two
      \else:
        \l_fp_input_a_sign_int \c_one
      \fi:
    \fi:
  }
%    \end{macrocode}
% \end{macro}
% \end{macro}
% \end{macro}
% \end{macro}
% \end{macro}
%
% \begin{macro}{\fp_trig_calc_cos:}
% \begin{macro}{\fp_trig_calc_sin:}
% \begin{macro}[aux]{\fp_trig_calc_Taylor:}
%   These functions actually do the calculation for sine and cosine.
%    \begin{macrocode}
\cs_new_protected_nopar:Npn \fp_trig_calc_cos:
  {
    \if_int_compare:w \l_fp_input_a_decimal_int = \c_zero
      \l_fp_output_integer_int \c_one
      \l_fp_output_decimal_int \c_zero
    \else:
      \l_fp_trig_sign_int \c_minus_one
      \fp_mul:NNNNNN
        \l_fp_input_a_decimal_int \l_fp_input_a_extended_int
        \l_fp_input_a_decimal_int \l_fp_input_a_extended_int
        \l_fp_trig_decimal_int \l_fp_trig_extended_int
      \fp_div_integer:NNNNN
        \l_fp_trig_decimal_int \l_fp_trig_extended_int
        \c_two
        \l_fp_trig_decimal_int \l_fp_trig_extended_int
      \l_fp_count_int \c_three
      \if_int_compare:w \l_fp_trig_extended_int = \c_zero
        \if_int_compare:w \l_fp_trig_decimal_int = \c_zero
          \l_fp_output_integer_int \c_one
          \l_fp_output_decimal_int \c_zero
          \l_fp_output_extended_int \c_zero
        \else:
          \l_fp_output_integer_int \c_zero
          \l_fp_output_decimal_int \c_one_thousand_million
          \l_fp_output_extended_int \c_zero
        \fi:
      \else:
        \l_fp_output_integer_int \c_zero
        \l_fp_output_decimal_int 999999999 \scan_stop:
        \l_fp_output_extended_int \c_one_thousand_million
      \fi:
      \int_advance:w \l_fp_output_extended_int -\l_fp_trig_extended_int
      \int_advance:w \l_fp_output_decimal_int -\l_fp_trig_decimal_int
      \exp_after:wN \fp_trig_calc_Taylor:
    \fi:
  }
\cs_new_protected_nopar:Npn \fp_trig_calc_sin:
  {
    \l_fp_output_integer_int \c_zero
    \if_int_compare:w \l_fp_input_a_decimal_int = \c_zero
      \l_fp_output_decimal_int \c_zero
    \else:
      \l_fp_output_decimal_int \l_fp_input_a_decimal_int
      \l_fp_output_extended_int \l_fp_input_a_extended_int
      \l_fp_trig_sign_int \c_one
      \l_fp_trig_decimal_int \l_fp_input_a_decimal_int
      \l_fp_trig_extended_int \l_fp_input_a_extended_int
      \l_fp_count_int \c_two
      \exp_after:wN \fp_trig_calc_Taylor:
    \fi:
  }
%    \end{macrocode}
%   This implements a Taylor series calculation for the trigonometric
%   functions. Lots of shuffling about as \TeX\ is not exactly a natural
%   choice for this sort of thing.
%    \begin{macrocode}
\cs_new_protected_nopar:Npn \fp_trig_calc_Taylor:
  {
    \l_fp_trig_sign_int -\l_fp_trig_sign_int
    \fp_mul:NNNNNN
      \l_fp_trig_decimal_int \l_fp_trig_extended_int
      \l_fp_input_a_decimal_int \l_fp_input_a_extended_int
      \l_fp_trig_decimal_int \l_fp_trig_extended_int
    \fp_mul:NNNNNN
      \l_fp_trig_decimal_int \l_fp_trig_extended_int
      \l_fp_input_a_decimal_int \l_fp_input_a_extended_int
      \l_fp_trig_decimal_int \l_fp_trig_extended_int
    \fp_div_integer:NNNNN
      \l_fp_trig_decimal_int \l_fp_trig_extended_int
      \l_fp_count_int
      \l_fp_trig_decimal_int \l_fp_trig_extended_int
    \int_advance:w \l_fp_count_int \c_one
    \fp_div_integer:NNNNN
      \l_fp_trig_decimal_int \l_fp_trig_extended_int
      \l_fp_count_int
      \l_fp_trig_decimal_int \l_fp_trig_extended_int
    \int_advance:w \l_fp_count_int \c_one
    \if_int_compare:w \l_fp_trig_decimal_int > \c_zero
      \if_int_compare:w \l_fp_trig_sign_int > \c_zero
        \int_advance:w \l_fp_output_decimal_int \l_fp_trig_decimal_int
        \int_advance:w \l_fp_output_extended_int
          \l_fp_trig_extended_int
        \if_int_compare:w \l_fp_output_extended_int < \c_one_thousand_million
        \else:
          \int_advance:w \l_fp_output_decimal_int \c_one
          \int_advance:w \l_fp_output_extended_int
            -\c_one_thousand_million
        \fi:
        \if_int_compare:w \l_fp_output_decimal_int < \c_one_thousand_million
        \else:
          \int_advance:w \l_fp_output_integer_int \c_one
          \int_advance:w \l_fp_output_decimal_int
            -\c_one_thousand_million
        \fi:
      \else:
        \int_advance:w \l_fp_output_decimal_int -\l_fp_trig_decimal_int
        \int_advance:w \l_fp_output_extended_int
          -\l_fp_input_a_extended_int
        \if_int_compare:w \l_fp_output_extended_int < \c_zero
          \int_advance:w \l_fp_output_decimal_int \c_minus_one
          \int_advance:w \l_fp_output_extended_int \c_one_thousand_million
        \fi:
        \if_int_compare:w \l_fp_output_decimal_int < \c_zero
          \int_advance:w \l_fp_output_integer_int \c_minus_one
          \int_advance:w \l_fp_output_decimal_int \c_one_thousand_million
        \fi:
      \fi:
      \exp_after:wN \fp_trig_calc_Taylor:
    \fi:
  }
%    \end{macrocode}
% \end{macro}
% \end{macro}
% \end{macro}
%
% \begin{macro}{\fp_tan:Nn, \fp_tan:cn}
% \UnitTested
% \begin{macro}{\fp_gtan:Nn,\fp_gtan:cn}
% \UnitTested
% \begin{macro}[aux]{\fp_tan_aux:NNn}
% \begin{macro}[aux]{\fp_tan_aux_i:}
% \begin{macro}[aux]{\fp_tan_aux_ii:}
% \begin{macro}[aux]{\fp_tan_aux_iii:}
% \begin{macro}[aux]{\fp_tan_aux_iv:}
%   As might be expected, tangents are calculated from the sine and cosine
%   by division. So there is a bit of set up, the two subsidiary pieces
%   of work are done and then a division takes place. For small numbers,
%   the same approach is used as for sines, with the input value simply
%   returned as is.
%    \begin{macrocode}
\cs_new_protected_nopar:Npn \fp_tan:Nn  { \fp_tan_aux:NNn \tl_set:Nn }
\cs_new_protected_nopar:Npn \fp_gtan:Nn { \fp_tan_aux:NNn \tl_gset:Nn }
\cs_generate_variant:Nn \fp_tan:Nn   { c }
\cs_generate_variant:Nn \fp_gtan:Nn  { c }
\cs_new_protected_nopar:Npn \fp_tan_aux:NNn #1#2#3
  {
    \group_begin:
      \fp_split:Nn a {#3}
      \fp_standardise:NNNN
        \l_fp_input_a_sign_int
        \l_fp_input_a_integer_int
        \l_fp_input_a_decimal_int
        \l_fp_input_a_exponent_int
      \tl_set:Nx \l_fp_arg_tl
        {
          \if_int_compare:w \l_fp_input_a_sign_int < \c_zero
            -
          \else:
            +
          \fi:
          \int_use:N \l_fp_input_a_integer_int
          .
          \exp_after:wN \use_none:n
            \int_value:w \int_eval:w
              \l_fp_input_a_decimal_int + \c_one_thousand_million
          e
          \int_use:N \l_fp_input_a_exponent_int
        }
      \if_int_compare:w \l_fp_input_a_exponent_int < -\c_five
        \cs_set_protected_nopar:Npx \fp_tmp:w
        {
          \group_end:
          #1 \exp_not:N #2 { \l_fp_arg_tl }
        }
      \else:
        \if_cs_exist:w
          c_fp_tan ( \l_fp_arg_tl ) _fp
        \cs_end:
        \else:
          \exp_after:wN \exp_after:wN \exp_after:wN
            \fp_tan_aux_i:
        \fi:
        \cs_set_protected_nopar:Npx \fp_tmp:w
          {
            \group_end:
            #1 \exp_not:N #2
              { \use:c { c_fp_tan ( \l_fp_arg_tl ) _fp } }
          }
      \fi:
    \fp_tmp:w
  }
%    \end{macrocode}
%   The business of the calculation does not check for stored sines or
%   cosines as there would then be an overhead to reading them back in.
%   There is also no need to worry about \enquote{small} sine values as
%   these will have been dealt with earlier. There is a two-step lead off
%   so that undefined division is not even attempted.
%    \begin{macrocode}
\cs_new_protected_nopar:Npn \fp_tan_aux_i:
  {
    \if_int_compare:w \l_fp_input_a_exponent_int < \c_ten
      \exp_after:wN \fp_tan_aux_ii:
    \else:
      \cs_new_eq:cN { c_fp_tan ( \l_fp_arg_tl ) _fp }
        \c_zero_fp
      \exp_after:wN \fp_trig_overflow_msg:
    \fi:
  }
\cs_new_protected_nopar:Npn \fp_tan_aux_ii:
  {
    \fp_trig_normalise:
    \if_int_compare:w \l_fp_input_a_sign_int > \c_zero
      \if_int_compare:w \l_fp_trig_octant_int > \c_two
        \l_fp_output_sign_int \c_minus_one
      \else:
        \l_fp_output_sign_int \c_one
      \fi:
    \else:
      \if_int_compare:w \l_fp_trig_octant_int > \c_two
        \l_fp_output_sign_int \c_one
      \else:
        \l_fp_output_sign_int \c_minus_one
      \fi:
    \fi:
    \fp_cos_aux_ii:
    \if_int_compare:w \l_fp_input_a_decimal_int = \c_zero
      \if_int_compare:w \l_fp_input_a_integer_int = \c_zero
        \cs_new_eq:cN { c_fp_tan ( \l_fp_arg_tl ) _fp }
          \c_undefined_fp
      \else:
        \exp_after:wN \exp_after:wN \exp_after:wN
          \fp_tan_aux_iii:
      \fi:
    \else:
      \exp_after:wN \fp_tan_aux_iii:
    \fi:
  }
%    \end{macrocode}
%   The division is done here using the same code as the standard division
%   unit, shifting the digits in the calculated sine and cosine to
%   maintain accuracy.
%    \begin{macrocode}
\cs_new_protected_nopar:Npn \fp_tan_aux_iii:
  {
    \l_fp_input_b_integer_int \l_fp_output_decimal_int
    \l_fp_input_b_decimal_int \l_fp_output_extended_int
    \l_fp_input_b_exponent_int -\c_nine
    \fp_standardise:NNNN
      \l_fp_input_b_sign_int
      \l_fp_input_b_integer_int
      \l_fp_input_b_decimal_int
      \l_fp_input_b_exponent_int
    \fp_sin_aux_ii:
    \l_fp_input_a_integer_int \l_fp_output_decimal_int
    \l_fp_input_a_decimal_int \l_fp_output_extended_int
    \l_fp_input_a_exponent_int -\c_nine
    \fp_standardise:NNNN
      \l_fp_input_a_sign_int
      \l_fp_input_a_integer_int
      \l_fp_input_a_decimal_int
      \l_fp_input_a_exponent_int
    \if_int_compare:w \l_fp_input_a_decimal_int = \c_zero
      \if_int_compare:w \l_fp_input_a_integer_int = \c_zero
        \cs_new_eq:cN { c_fp_tan ( \l_fp_arg_tl ) _fp }
          \c_zero_fp
      \else:
        \exp_after:wN \exp_after:wN \exp_after:wN \fp_tan_aux_iv:
      \fi:
    \else:
      \exp_after:wN \fp_tan_aux_iv:
    \fi:
  }
\cs_new_protected_nopar:Npn \fp_tan_aux_iv:
  {
    \l_fp_output_integer_int \c_zero
    \l_fp_output_decimal_int \c_zero
    \cs_set_eq:NN \fp_div_store: \fp_div_store_integer:
    \l_fp_div_offset_int \c_one_hundred_million
    \fp_div_loop:
    \l_fp_output_exponent_int
      \int_eval:w
        \l_fp_input_a_exponent_int - \l_fp_input_b_exponent_int
      \int_eval_end:
    \fp_standardise:NNNN
      \l_fp_output_sign_int
      \l_fp_output_integer_int
      \l_fp_output_decimal_int
      \l_fp_output_exponent_int
    \tl_new:c { c_fp_tan ( \l_fp_arg_tl ) _fp }
    \tl_gset:cx { c_fp_tan ( \l_fp_arg_tl ) _fp }
      {
        \if_int_compare:w \l_fp_output_sign_int > \c_zero
          +
        \else:
          -
        \fi:
        \int_use:N \l_fp_output_integer_int
        .
        \exp_after:wN \use_none:n
          \int_value:w \int_eval:w
             \l_fp_output_decimal_int + \c_one_thousand_million
        e
        \int_use:N \l_fp_output_exponent_int
      }
  }
%    \end{macrocode}
% \end{macro}
% \end{macro}
% \end{macro}
% \end{macro}
% \end{macro}
% \end{macro}
% \end{macro}
%
% \subsection{Exponent and logarithm functions}
%
% \begin{variable}{\c_fp_exp_1_tl}
% \begin{variable}{\c_fp_exp_2_tl}
% \begin{variable}{\c_fp_exp_3_tl}
% \begin{variable}{\c_fp_exp_4_tl}
% \begin{variable}{\c_fp_exp_5_tl}
% \begin{variable}{\c_fp_exp_6_tl}
% \begin{variable}{\c_fp_exp_7_tl}
% \begin{variable}{\c_fp_exp_8_tl}
% \begin{variable}{\c_fp_exp_9_tl}
% \begin{variable}{\c_fp_exp_10_tl}
% \begin{variable}{\c_fp_exp_20_tl}
% \begin{variable}{\c_fp_exp_30_tl}
% \begin{variable}{\c_fp_exp_40_tl}
% \begin{variable}{\c_fp_exp_50_tl}
% \begin{variable}{\c_fp_exp_60_tl}
% \begin{variable}{\c_fp_exp_70_tl}
% \begin{variable}{\c_fp_exp_80_tl}
% \begin{variable}{\c_fp_exp_90_tl}
% \begin{variable}{\c_fp_exp_100_tl}
% \begin{variable}{\c_fp_exp_200_tl}
%   Calculation of exponentials requires a number of precomputed values:
%   first the positive integers.
%    \begin{macrocode}
\tl_const:cn { c_fp_exp_1_tl }   { { 2 } { 718281828 } { 459045235 } { 0 } }
\tl_const:cn { c_fp_exp_2_tl }   { { 7 } { 389056098 } { 930650227 } { 0 } }
\tl_const:cn { c_fp_exp_3_tl }   { { 2 } { 008553692 } { 318766774 } { 1 } }
\tl_const:cn { c_fp_exp_4_tl }   { { 5 } { 459815003 } { 314423908 } { 1 } }
\tl_const:cn { c_fp_exp_5_tl }   { { 1 } { 484131591 } { 025766034 } { 2 } }
\tl_const:cn { c_fp_exp_6_tl }   { { 4 } { 034287934 } { 927351226 } { 2 } }
\tl_const:cn { c_fp_exp_7_tl }   { { 1 } { 096633158 } { 428458599 } { 3 } }
\tl_const:cn { c_fp_exp_8_tl }   { { 2 } { 980957987 } { 041728275 } { 3 } }
\tl_const:cn { c_fp_exp_9_tl }   { { 8 } { 103083927 } { 575384008 } { 3 } }
\tl_const:cn { c_fp_exp_10_tl }  { { 2 } { 202646579 } { 480671652 } { 4 } }
\tl_const:cn { c_fp_exp_20_tl }  { { 4 } { 851651954 } { 097902280 } { 8 } }
\tl_const:cn { c_fp_exp_30_tl }  { { 1 } { 068647458 } { 152446215 } { 13 } }
\tl_const:cn { c_fp_exp_40_tl }  { { 2 } { 353852668 } { 370199854 } { 17 } }
\tl_const:cn { c_fp_exp_50_tl }  { { 5 } { 184705528 } { 587072464 } { 21 } }
\tl_const:cn { c_fp_exp_60_tl }  { { 1 } { 142007389 } { 815684284 } { 26 } }
\tl_const:cn { c_fp_exp_70_tl }  { { 2 } { 515438670 } { 919167006 } { 30 } }
\tl_const:cn { c_fp_exp_80_tl }  { { 5 } { 540622384 } { 393510053 } { 34 } }
\tl_const:cn { c_fp_exp_90_tl }  { { 1 } { 220403294 } { 317840802 } { 39 } }
\tl_const:cn { c_fp_exp_100_tl } { { 2 } { 688117141 } { 816135448 } { 43 } }
\tl_const:cn { c_fp_exp_200_tl } { { 7 } { 225973768 } { 125749258 } { 86 } }
%    \end{macrocode}
% \end{variable}
% \end{variable}
% \end{variable}
% \end{variable}
% \end{variable}
% \end{variable}
% \end{variable}
% \end{variable}
% \end{variable}
% \end{variable}
% \end{variable}
% \end{variable}
% \end{variable}
% \end{variable}
% \end{variable}
% \end{variable}
% \end{variable}
% \end{variable}
% \end{variable}
% \end{variable}
%
% \begin{variable}{\c_fp_exp_-1_tl}
% \begin{variable}{\c_fp_exp_-2_tl}
% \begin{variable}{\c_fp_exp_-3_tl}
% \begin{variable}{\c_fp_exp_-4_tl}
% \begin{variable}{\c_fp_exp_-5_tl}
% \begin{variable}{\c_fp_exp_-6_tl}
% \begin{variable}{\c_fp_exp_-7_tl}
% \begin{variable}{\c_fp_exp_-8_tl}
% \begin{variable}{\c_fp_exp_-9_tl}
% \begin{variable}{\c_fp_exp_-10_tl}
% \begin{variable}{\c_fp_exp_-20_tl}
% \begin{variable}{\c_fp_exp_-30_tl}
% \begin{variable}{\c_fp_exp_-40_tl}
% \begin{variable}{\c_fp_exp_-50_tl}
% \begin{variable}{\c_fp_exp_-60_tl}
% \begin{variable}{\c_fp_exp_-70_tl}
% \begin{variable}{\c_fp_exp_-80_tl}
% \begin{variable}{\c_fp_exp_-90_tl}
% \begin{variable}{\c_fp_exp_-100_tl}
% \begin{variable}{\c_fp_exp_-200_tl}
% Now the negative integers.
%    \begin{macrocode}
\tl_const:cn { c_fp_exp_-1_tl }   { { 3 } { 678794411 } { 71442322 }  { -1 } }
\tl_const:cn { c_fp_exp_-2_tl }   { { 1 } { 353352832 } { 366132692 } { -1 } }
\tl_const:cn { c_fp_exp_-3_tl }   { { 4 } { 978706836 } { 786394298 } { -2 } }
\tl_const:cn { c_fp_exp_-4_tl }   { { 1 } { 831563888 } { 873418029 } { -2 } }
\tl_const:cn { c_fp_exp_-5_tl }   { { 6 } { 737946999 } { 085467097 } { -3 } }
\tl_const:cn { c_fp_exp_-6_tl }   { { 2 } { 478752176 } { 666358423 } { -3 } }
\tl_const:cn { c_fp_exp_-7_tl }   { { 9 } { 118819655 } { 545162080 } { -4 } }
\tl_const:cn { c_fp_exp_-8_tl }   { { 3 } { 354626279 } { 025118388 } { -4 } }
\tl_const:cn { c_fp_exp_-9_tl }   { { 1 } { 234098040 } { 866795495 } { -4 } }
\tl_const:cn { c_fp_exp_-10_tl }  { { 4 } { 539992976 } { 248451536 } { -5 } }
\tl_const:cn { c_fp_exp_-20_tl }  { { 2 } { 061153622 } { 438557828 } { -9 } }
\tl_const:cn { c_fp_exp_-30_tl }  { { 9 } { 357622968 } { 840174605 } { -14 } }
\tl_const:cn { c_fp_exp_-40_tl }  { { 4 } { 248354255 } { 291588995 } { -18 } }
\tl_const:cn { c_fp_exp_-50_tl }  { { 1 } { 928749847 } { 963917783 } { -22 } }
\tl_const:cn { c_fp_exp_-60_tl }  { { 8 } { 756510762 } { 696520338 } { -27 } }
\tl_const:cn { c_fp_exp_-70_tl }  { { 3 } { 975449735 } { 908646808 } { -31 } }
\tl_const:cn { c_fp_exp_-80_tl }  { { 1 } { 804851387 } { 845415172 } { -35 } }
\tl_const:cn { c_fp_exp_-90_tl }  { { 8 } { 194012623 } { 990515430 } { -40 } }
\tl_const:cn { c_fp_exp_-100_tl } { { 3 } { 720075976 } { 020835963 } { -44 } }
\tl_const:cn { c_fp_exp_-200_tl } { { 1 } { 383896526 } { 736737530 } { -87 } }
%    \end{macrocode}
% \end{variable}
% \end{variable}
% \end{variable}
% \end{variable}
% \end{variable}
% \end{variable}
% \end{variable}
% \end{variable}
% \end{variable}
% \end{variable}
% \end{variable}
% \end{variable}
% \end{variable}
% \end{variable}
% \end{variable}
% \end{variable}
% \end{variable}
% \end{variable}
% \end{variable}
% \end{variable}
%
% \begin{macro}{\fp_exp:Nn, \fp_exp:cn}
% \UnitTested
% \begin{macro}{\fp_gexp:Nn,\fp_gexp:cn}
% \UnitTested
% \begin{macro}[aux]{\fp_exp_aux:NNn}
% \begin{macro}[aux]{\fp_exp_internal:}
% \begin{macro}[aux]{\fp_exp_aux:}
% \begin{macro}[aux]{\fp_exp_integer:}
% \begin{macro}[aux]{\fp_exp_integer_tens:}
% \begin{macro}[aux]{\fp_exp_integer_units:}
% \begin{macro}[aux]{\fp_exp_integer_const:n}
% \begin{macro}[aux]{\fp_exp_integer_const:nnnn}
% \begin{macro}[aux]{\fp_exp_decimal:}
% \begin{macro}[aux]{\fp_exp_Taylor:}
% \begin{macro}[aux]{\fp_exp_const:Nx}
% \begin{macro}[aux]{\fp_exp_const:cx}
%   The calculation of an exponent starts off starts in much the same
%   way as the trigonometric functions: normalise the input, look for
%   a pre-defined value and if one is not found hand off to the real
%   workhorse function. The test for a definition of the result is used
%   so that overflows do not result in any outcome being defined.
%    \begin{macrocode}
\cs_new_protected_nopar:Npn \fp_exp:Nn  { \fp_exp_aux:NNn \tl_set:Nn }
\cs_new_protected_nopar:Npn \fp_gexp:Nn { \fp_exp_aux:NNn \tl_gset:Nn }
\cs_generate_variant:Nn \fp_exp:Nn  { c }
\cs_generate_variant:Nn \fp_gexp:Nn { c }
\cs_new_protected_nopar:Npn \fp_exp_aux:NNn #1#2#3
  {
    \group_begin:
      \fp_split:Nn a {#3}
      \fp_standardise:NNNN
        \l_fp_input_a_sign_int
        \l_fp_input_a_integer_int
        \l_fp_input_a_decimal_int
        \l_fp_input_a_exponent_int
      \l_fp_input_a_extended_int \c_zero
      \tl_set:Nx \l_fp_arg_tl
        {
          \if_int_compare:w \l_fp_input_a_sign_int < \c_zero
            -
          \else:
            +
          \fi:
          \int_use:N \l_fp_input_a_integer_int
          .
          \exp_after:wN \use_none:n
            \int_value:w \int_eval:w
               \l_fp_input_a_decimal_int + \c_one_thousand_million
          e
          \int_use:N \l_fp_input_a_exponent_int
        }
      \if_cs_exist:w c_fp_exp ( \l_fp_arg_tl ) _fp \cs_end:
      \else:
        \exp_after:wN \fp_exp_internal:
      \fi:
      \cs_set_protected_nopar:Npx \fp_tmp:w
        {
          \group_end:
          #1 \exp_not:N #2
            {
              \if_cs_exist:w c_fp_exp ( \l_fp_arg_tl ) _fp
                \cs_end:
                \use:c { c_fp_exp ( \l_fp_arg_tl ) _fp }
              \else:
                \c_zero_fp
              \fi:
            }
        }
    \fp_tmp:w
  }
%    \end{macrocode}
%   The first real step is to convert the input into a fixed-point
%   representation for further calculation: anything which is dropped
%   here as too small would not influence the output in any case. There
%   are a couple of overflow tests: the maximum
%    \begin{macrocode}
\cs_new_protected_nopar:Npn \fp_exp_internal:
  {
    \if_int_compare:w \l_fp_input_a_exponent_int < \c_three
      \fp_extended_normalise:
      \if_int_compare:w \l_fp_input_a_sign_int > \c_zero
        \if_int_compare:w \l_fp_input_a_integer_int < 230 \scan_stop:
          \exp_after:wN \exp_after:wN \exp_after:wN
          \exp_after:wN \exp_after:wN \exp_after:wN
            \exp_after:wN \fp_exp_aux:
        \else:
          \exp_after:wN \exp_after:wN \exp_after:wN
          \exp_after:wN \exp_after:wN \exp_after:wN
            \exp_after:wN \fp_exp_overflow_msg:
          \fi:
      \else:
        \if_int_compare:w \l_fp_input_a_integer_int < 230 \scan_stop:
          \exp_after:wN \exp_after:wN \exp_after:wN
          \exp_after:wN \exp_after:wN \exp_after:wN
            \exp_after:wN  \fp_exp_aux:
        \else:
          \fp_exp_const:cx { c_fp_exp ( \l_fp_arg_tl ) _fp }
            { \c_zero_fp }
        \fi:
      \fi:
    \else:
      \exp_after:wN \fp_exp_overflow_msg:
    \fi:
  }
%    \end{macrocode}
%   The main algorithm makes use of the fact that
%   \[
%     \mathrm{e}^{nmp.q} =
%       \mathrm{e}^{n}
%       \mathrm{e}^{m}
%       \mathrm{e}^{p}
%       \mathrm{e}^{0.q}
%   \]
%   and that there is a Taylor series that can be used to calculate
%   $ \mathrm{e}^{0.q} $. Thus the approach needed is in three parts.
%   First, the exponent of the integer part of the input is found
%   using the pre-calculated constants. Second, the Taylor series is
%   used to find the exponent for the decimal part of the input. Finally,
%   the two parts are multiplied together to give the result. As the
%   normalisation code will already have dealt with any overflowing
%   values, there are no further checks needed.
%    \begin{macrocode}
\cs_new_protected_nopar:Npn \fp_exp_aux:
  {
    \if_int_compare:w \l_fp_input_a_integer_int > \c_zero
      \exp_after:wN \fp_exp_integer:
    \else:
      \l_fp_output_integer_int  \c_one
      \l_fp_output_decimal_int  \c_zero
      \l_fp_output_extended_int \c_zero
      \l_fp_output_exponent_int \c_zero
      \exp_after:wN \fp_exp_decimal:
    \fi:
  }
%    \end{macrocode}
%   The integer part calculation starts with the hundreds. This is
%   set up such that very large negative numbers can short-cut the entire
%   procedure and simply return zero. In other cases, the code either
%   recovers the exponent of the hundreds value or sets the appropriate
%   storage to one (so that multiplication works correctly).
%    \begin{macrocode}
\cs_new_protected_nopar:Npn \fp_exp_integer:
  {
    \if_int_compare:w \l_fp_input_a_integer_int < \c_one_hundred
      \l_fp_exp_integer_int  \c_one
      \l_fp_exp_decimal_int  \c_zero
      \l_fp_exp_extended_int \c_zero
      \l_fp_exp_exponent_int \c_zero
      \exp_after:wN \fp_exp_integer_tens:
    \else:
      \tl_set:Nx \l_fp_tmp_tl
        {
          \exp_after:wN \use_i:nnn
            \int_use:N \l_fp_input_a_integer_int
        }
      \l_fp_input_a_integer_int
        \int_eval:w
          \l_fp_input_a_integer_int - \l_fp_tmp_tl 00
        \int_eval_end:
      \if_int_compare:w \l_fp_input_a_sign_int < \c_zero
        \if_int_compare:w \l_fp_output_integer_int > 200 \scan_stop:
          \fp_exp_const:cx { c_fp_exp ( \l_fp_arg_tl ) _fp }
            { \c_zero_fp }
        \else:
          \fp_exp_integer_const:n { - \l_fp_tmp_tl 00 }
          \exp_after:wN \exp_after:wN \exp_after:wN
            \exp_after:wN \exp_after:wN \exp_after:wN
            \exp_after:wN \fp_exp_integer_tens:
        \fi:
      \else:
        \fp_exp_integer_const:n { \l_fp_tmp_tl 00 }
        \exp_after:wN \exp_after:wN \exp_after:wN
          \exp_after:wN \fp_exp_integer_tens:
      \fi:
    \fi:
  }
%    \end{macrocode}
%   The tens and units parts are handled in a similar way, with a
%   multiplication step to build up the final value. That also includes a
%   correction step to avoid an overflow of the integer part.
%    \begin{macrocode}
\cs_new_protected_nopar:Npn \fp_exp_integer_tens:
  {
    \l_fp_output_integer_int  \l_fp_exp_integer_int
    \l_fp_output_decimal_int  \l_fp_exp_decimal_int
    \l_fp_output_extended_int \l_fp_exp_extended_int
    \l_fp_output_exponent_int \l_fp_exp_exponent_int
    \if_int_compare:w \l_fp_input_a_integer_int > \c_nine
      \tl_set:Nx \l_fp_tmp_tl
        {
          \exp_after:wN \use_i:nn
            \int_use:N \l_fp_input_a_integer_int
        }
      \l_fp_input_a_integer_int
        \int_eval:w
          \l_fp_input_a_integer_int - \l_fp_tmp_tl 0
        \int_eval_end:
      \if_int_compare:w \l_fp_input_a_sign_int > \c_zero
        \fp_exp_integer_const:n { \l_fp_tmp_tl 0 }
      \else:
        \fp_exp_integer_const:n { - \l_fp_tmp_tl 0 }
      \fi:
      \fp_mul:NNNNNNNNN
        \l_fp_exp_integer_int \l_fp_exp_decimal_int \l_fp_exp_extended_int
        \l_fp_output_integer_int \l_fp_output_decimal_int
          \l_fp_output_extended_int
        \l_fp_output_integer_int \l_fp_output_decimal_int
          \l_fp_output_extended_int
      \int_advance:w \l_fp_output_exponent_int \l_fp_exp_exponent_int
      \fp_extended_normalise_output:
    \fi:
    \fp_exp_integer_units:
  }
\cs_new_protected_nopar:Npn \fp_exp_integer_units:
  {
    \if_int_compare:w \l_fp_input_a_integer_int > \c_zero
      \if_int_compare:w \l_fp_input_a_sign_int > \c_zero
        \fp_exp_integer_const:n { \int_use:N \l_fp_input_a_integer_int }
      \else:
        \fp_exp_integer_const:n
          { - \int_use:N \l_fp_input_a_integer_int }
      \fi:
      \fp_mul:NNNNNNNNN
        \l_fp_exp_integer_int \l_fp_exp_decimal_int \l_fp_exp_extended_int
        \l_fp_output_integer_int \l_fp_output_decimal_int
          \l_fp_output_extended_int
        \l_fp_output_integer_int \l_fp_output_decimal_int
          \l_fp_output_extended_int
      \int_advance:w \l_fp_output_exponent_int \l_fp_exp_exponent_int
      \fp_extended_normalise_output:
    \fi:
    \fp_exp_decimal:
  }
%    \end{macrocode}
%   Recovery of the stored constant values into the separate registers
%   is done with a simple expansion then assignment.
%    \begin{macrocode}
\cs_new_protected_nopar:Npn \fp_exp_integer_const:n #1
  {
    \exp_after:wN \exp_after:wN \exp_after:wN
      \fp_exp_integer_const:nnnn
      \cs:w c_fp_exp_ #1 _tl \cs_end:
  }
\cs_new_protected_nopar:Npn \fp_exp_integer_const:nnnn #1#2#3#4
  {
    \l_fp_exp_integer_int  #1 \scan_stop:
    \l_fp_exp_decimal_int  #2 \scan_stop:
    \l_fp_exp_extended_int #3 \scan_stop:
    \l_fp_exp_exponent_int #4 \scan_stop:
  }
%    \end{macrocode}
%   Finding the exponential for the decimal part of the number requires
%   a Taylor series calculation. The set up is done here with the loop
%   itself a separate function. Once the decimal part is available this
%   is multiplied by the integer part already worked out to give
%   the final result.
%    \begin{macrocode}
\cs_new_protected_nopar:Npn \fp_exp_decimal:
  {
    \if_int_compare:w \l_fp_input_a_decimal_int > \c_zero
      \if_int_compare:w \l_fp_input_a_sign_int > \c_zero
        \l_fp_exp_integer_int  \c_one
        \l_fp_exp_decimal_int  \l_fp_input_a_decimal_int
        \l_fp_exp_extended_int \l_fp_input_a_extended_int
      \else:
        \l_fp_exp_integer_int \c_zero
        \if_int_compare:w \l_fp_exp_extended_int = \c_zero
          \l_fp_exp_decimal_int
            \int_eval:w
              \c_one_thousand_million - \l_fp_input_a_decimal_int
            \int_eval_end:
          \l_fp_exp_extended_int \c_zero
        \else:
          \l_fp_exp_decimal_int
            \int_eval:w
              999999999 - \l_fp_input_a_decimal_int
            \scan_stop:
          \l_fp_exp_extended_int
            \int_eval:w
              \c_one_thousand_million - \l_fp_input_a_extended_int
            \int_eval_end:
        \fi:
      \fi:
      \l_fp_input_b_sign_int     \l_fp_input_a_sign_int
      \l_fp_input_b_decimal_int  \l_fp_input_a_decimal_int
      \l_fp_input_b_extended_int \l_fp_input_a_extended_int
      \l_fp_count_int \c_one
      \fp_exp_Taylor:
      \fp_mul:NNNNNNNNN
        \l_fp_exp_integer_int \l_fp_exp_decimal_int \l_fp_exp_extended_int
        \l_fp_output_integer_int \l_fp_output_decimal_int
          \l_fp_output_extended_int
        \l_fp_output_integer_int \l_fp_output_decimal_int
          \l_fp_output_extended_int
    \fi:
    \if_int_compare:w \l_fp_output_extended_int < \c_five_hundred_million
    \else:
      \int_advance:w \l_fp_output_decimal_int \c_one
      \if_int_compare:w \l_fp_output_decimal_int < \c_one_thousand_million
      \else:
        \l_fp_output_decimal_int \c_zero
        \int_advance:w \l_fp_output_integer_int \c_one
      \fi:
    \fi:
    \fp_standardise:NNNN
      \l_fp_output_sign_int
      \l_fp_output_integer_int
      \l_fp_output_decimal_int
      \l_fp_output_exponent_int
    \fp_exp_const:cx { c_fp_exp ( \l_fp_arg_tl ) _fp }
      {
        +
        \int_use:N \l_fp_output_integer_int
        .
        \exp_after:wN \use_none:n
          \int_value:w \int_eval:w
             \l_fp_output_decimal_int + \c_one_thousand_million
        e
        \int_use:N \l_fp_output_exponent_int
      }
  }
%    \end{macrocode}
%   The Taylor series for $ \exp(x) $ is
%   \[
%     1 + x + \frac{x^2}{2!} + \frac{x^3}{3!} + \frac{x^4}{4!} + \cdots
%   \]
%   which converges for $ -1 < x < 1 $. The code above sets up
%   the $ x $ part, leaving the loop to multiply the running
%   value by $ x / n $ and add it onto the sum. The way that this is
%   done is that the running total is stored in the \texttt{exp} set of
%   registers, while the current item is stored as \texttt{input_b}.
%    \begin{macrocode}
\cs_new_protected_nopar:Npn \fp_exp_Taylor:
  {
    \int_advance:w \l_fp_count_int \c_one
    \tex_multiply:D \l_fp_input_b_sign_int \l_fp_input_a_sign_int
    \fp_mul:NNNNNN
      \l_fp_input_a_decimal_int \l_fp_input_a_extended_int
      \l_fp_input_b_decimal_int \l_fp_input_b_extended_int
      \l_fp_input_b_decimal_int \l_fp_input_b_extended_int
    \fp_div_integer:NNNNN
      \l_fp_input_b_decimal_int \l_fp_input_b_extended_int
      \l_fp_count_int
      \l_fp_input_b_decimal_int \l_fp_input_b_extended_int
    \if_int_compare:w
      \int_eval:w
        \l_fp_input_b_decimal_int + \l_fp_input_b_extended_int
        > \c_zero
      \if_int_compare:w \l_fp_input_b_sign_int > \c_zero
        \int_advance:w \l_fp_exp_decimal_int \l_fp_input_b_decimal_int
        \int_advance:w \l_fp_exp_extended_int
          \l_fp_input_b_extended_int
        \if_int_compare:w \l_fp_exp_extended_int < \c_one_thousand_million
      \else:
          \int_advance:w \l_fp_exp_decimal_int \c_one
          \int_advance:w \l_fp_exp_extended_int
            -\c_one_thousand_million
        \fi:
        \if_int_compare:w \l_fp_exp_decimal_int < \c_one_thousand_million
        \else:
          \int_advance:w \l_fp_exp_integer_int \c_one
          \int_advance:w \l_fp_exp_decimal_int
            -\c_one_thousand_million
        \fi:
      \else:
        \int_advance:w \l_fp_exp_decimal_int -\l_fp_input_b_decimal_int
        \int_advance:w \l_fp_exp_extended_int
          -\l_fp_input_a_extended_int
        \if_int_compare:w \l_fp_exp_extended_int < \c_zero
          \int_advance:w \l_fp_exp_decimal_int \c_minus_one
          \int_advance:w \l_fp_exp_extended_int \c_one_thousand_million
        \fi:
        \if_int_compare:w \l_fp_exp_decimal_int < \c_zero
          \int_advance:w \l_fp_exp_integer_int \c_minus_one
          \int_advance:w \l_fp_exp_decimal_int \c_one_thousand_million
        \fi:
      \fi:
      \exp_after:wN \fp_exp_Taylor:
    \fi:
  }
%    \end{macrocode}
%   This is set up as a function so that the power code can redirect
%   the effect.
%    \begin{macrocode}
\cs_new_protected_nopar:Npn \fp_exp_const:Nx #1#2
  {
    \tl_new:N #1
    \tl_gset:Nx #1 {#2}
  }
\cs_generate_variant:Nn \fp_exp_const:Nx { c }
%    \end{macrocode}
% \end{macro}
% \end{macro}
% \end{macro}
% \end{macro}
% \end{macro}
% \end{macro}
% \end{macro}
% \end{macro}
% \end{macro}
% \end{macro}
% \end{macro}
% \end{macro}
% \end{macro}
% \end{macro}
%
% \begin{variable}{\c_fp_ln_10_1_tl}
% \begin{variable}{\c_fp_ln_10_2_tl}
% \begin{variable}{\c_fp_ln_10_3_tl}
% \begin{variable}{\c_fp_ln_10_4_tl}
% \begin{variable}{\c_fp_ln_10_5_tl}
% \begin{variable}{\c_fp_ln_10_6_tl}
% \begin{variable}{\c_fp_ln_10_7_tl}
% \begin{variable}{\c_fp_ln_10_8_tl}
% \begin{variable}{\c_fp_ln_10_9_tl}
%   Constants for working out logarithms: first those for the powers of
%   ten.
%    \begin{macrocode}
\tl_const:cn { c_fp_ln_10_1_tl } { { 2 } { 302585092 } { 994045684 } { 0 } }
\tl_const:cn { c_fp_ln_10_2_tl } { { 4 } { 605170185 } { 988091368 } { 0 } }
\tl_const:cn { c_fp_ln_10_3_tl } { { 6 } { 907755278 } { 982137052 } { 0 } }
\tl_const:cn { c_fp_ln_10_4_tl } { { 9 } { 210340371 } { 976182736 } { 0 } }
\tl_const:cn { c_fp_ln_10_5_tl } { { 1 } { 151292546 } { 497022842 } { 1 } }
\tl_const:cn { c_fp_ln_10_6_tl } { { 1 } { 381551055 } { 796427410 } { 1 } }
\tl_const:cn { c_fp_ln_10_7_tl } { { 1 } { 611809565 } { 095831979 } { 1 } }
\tl_const:cn { c_fp_ln_10_8_tl } { { 1 } { 842068074 } { 395226547 } { 1 } }
\tl_const:cn { c_fp_ln_10_9_tl } { { 2 } { 072326583 } { 694641116 } { 1 } }
%    \end{macrocode}
% \end{variable}
% \end{variable}
% \end{variable}
% \end{variable}
% \end{variable}
% \end{variable}
% \end{variable}
% \end{variable}
% \end{variable}
%
% \begin{variable}{\c_fp_ln_2_1_tl }
% \begin{variable}{\c_fp_ln_2_2_tl }
% \begin{variable}{\c_fp_ln_2_3_tl }
% The smaller set for powers of two.
%    \begin{macrocode}
\tl_const:cn { c_fp_ln_2_1_tl } { { 0 } { 693147180 } { 559945309 } { 0 } }
\tl_const:cn { c_fp_ln_2_2_tl } { { 1 } { 386294361 } { 119890618 } { 0 } }
\tl_const:cn { c_fp_ln_2_3_tl } { { 2 } { 079441541 } { 679835928 } { 0 } }
%    \end{macrocode}
% \end{variable}
% \end{variable}
% \end{variable}
%
% \begin{macro}{\fp_ln:Nn, \fp_ln:cn}
% \UnitTested
% \begin{macro}{\fp_gln:Nn,\fp_gln:cn}
% \UnitTested
% \begin{macro}[aux]{\fp_ln_aux:NNn}
% \begin{macro}[aux]{\fp_ln_aux:}
% \begin{macro}[aux]{\fp_ln_exponent:}
% \begin{macro}[aux]{\fp_ln_internal:}
% \begin{macro}[aux]{\fp_ln_exponent_units:}
% \begin{macro}[aux]{\fp_ln_normalise:}
% \begin{macro}[aux]{\fp_ln_nornalise_aux:NNNNNNNNN}
% \begin{macro}[aux]{\fp_ln_mantissa:}
% \begin{macro}[aux]{\fp_ln_mantissa_aux:}
% \begin{macro}[aux]{\fp_ln_mantissa_divide_two:}
% \begin{macro}[aux]{\fp_ln_integer_const:nn}
% \begin{macro}[aux]{\fp_ln_Taylor:}
% \begin{macro}[aux]{\fp_ln_fixed:}
% \begin{macro}[aux]{\fp_ln_fixed_aux:NNNNNNNNN}
% \begin{macro}[aux]{\fp_ln_Taylor_aux:}
%   The approach for logarithms is again based on a mix of tables and
%   Taylor series. Here, the initial validation is a bit easier and so it
%   is set up earlier, meaning less need to escape later on.
%    \begin{macrocode}
\cs_new_protected_nopar:Npn \fp_ln:Nn  { \fp_ln_aux:NNn \tl_set:Nn  }
\cs_new_protected_nopar:Npn \fp_gln:Nn { \fp_ln_aux:NNn \tl_gset:Nn }
\cs_generate_variant:Nn \fp_ln:Nn  { c }
\cs_generate_variant:Nn \fp_gln:Nn { c }
\cs_new_protected_nopar:Npn \fp_ln_aux:NNn #1#2#3
  {
    \group_begin:
      \fp_split:Nn a {#3}
      \fp_standardise:NNNN
        \l_fp_input_a_sign_int
        \l_fp_input_a_integer_int
        \l_fp_input_a_decimal_int
        \l_fp_input_a_exponent_int
      \if_int_compare:w \l_fp_input_a_sign_int > \c_zero
        \if_int_compare:w
          \int_eval:w
            \l_fp_input_a_integer_int + \l_fp_input_a_decimal_int
            > \c_zero
          \exp_after:wN \exp_after:wN \exp_after:wN \fp_ln_aux:
        \else:
          \cs_set_protected_nopar:Npx \fp_tmp:w ##1##2
            {
              \group_end:
              ##1 \exp_not:N ##2 { \c_zero_fp }
            }
          \exp_after:wN \exp_after:wN \exp_after:wN \fp_ln_error_msg:
        \fi:
      \else:
        \cs_set_protected_nopar:Npx \fp_tmp:w ##1##2
          {
            \group_end:
            ##1 \exp_not:N ##2 { \c_zero_fp }
          }
        \exp_after:wN \fp_ln_error_msg:
      \fi:
    \fp_tmp:w #1 #2
  }
%    \end{macrocode}
%   As the input at this stage meets the validity criteria above, the
%   argument can now be saved for further processing. There is no need
%   to look at the sign of the input as it must be positive. The function
%   here simply sets up to either do the full calculation or recover
%   the stored value, as appropriate.
%    \begin{macrocode}
\cs_new_protected_nopar:Npn \fp_ln_aux:
  {
    \tl_set:Nx \l_fp_arg_tl
      {
        +
        \int_use:N \l_fp_input_a_integer_int
        .
        \exp_after:wN \use_none:n
          \int_value:w \int_eval:w
             \l_fp_input_a_decimal_int + \c_one_thousand_million
        e
        \int_use:N \l_fp_input_a_exponent_int
      }
    \if_cs_exist:w c_fp_ln ( \l_fp_arg_tl ) _fp \cs_end:
    \else:
      \exp_after:wN \fp_ln_exponent:
    \fi:
    \cs_set_protected_nopar:Npx \fp_tmp:w ##1##2
      {
        \group_end:
        ##1 \exp_not:N ##2
          { \use:c { c_fp_ln ( \l_fp_arg_tl ) _fp } }
      }
  }
%    \end{macrocode}
%   The main algorithm here uses the fact the logarithm can be divided
%   up, first taking out the powers of ten, then powers of two and finally
%   using a Taylor series for the remainder.
%   \[
%     \ln ( 10^{n} \times 2^{m} \times x )
%       = \ln ( 10^{n} ) \times \ln ( 2^{m} ) \times \ln ( x )
%   \]
%   The second point to remember is that
%   \[
%     \ln ( x^{-1} ) = - \ln ( x )
%   \]
%   which means that for the powers of $ 10 $ and $ 2 $ constants
%   are only needed for positive powers.
%
%   The first step is to set up the sign for the output functions and
%   work out the powers of ten in the exponent. First the larger powers
%   are sorted out. The values for the constants are the same as those
%   for the smaller ones, just with a shift in the exponent.
%    \begin{macrocode}
\cs_new_protected_nopar:Npn \fp_ln_exponent:
  {
    \fp_ln_internal:
    \if_int_compare:w \l_fp_output_extended_int < \c_five_hundred_million
    \else:
      \int_advance:w \l_fp_output_decimal_int \c_one
      \if_int_compare:w \l_fp_output_decimal_int < \c_one_thousand_million
      \else:
        \l_fp_output_decimal_int \c_zero
        \int_advance:w \l_fp_output_integer_int \c_one
      \fi:
    \fi:
    \fp_standardise:NNNN
      \l_fp_output_sign_int
      \l_fp_output_integer_int
      \l_fp_output_decimal_int
      \l_fp_output_exponent_int
    \tl_const:cx { c_fp_ln ( \l_fp_arg_tl ) _fp }
      {
        \if_int_compare:w \l_fp_output_sign_int > \c_zero
          +
        \else:
          -
        \fi:
        \int_use:N \l_fp_output_integer_int
        .
        \exp_after:wN \use_none:n
          \int_value:w \int_eval:w
             \l_fp_output_decimal_int + \c_one_thousand_million
           \scan_stop:
        e
        \int_use:N \l_fp_output_exponent_int
      }
  }
\cs_new_protected_nopar:Npn \fp_ln_internal:
  {
    \if_int_compare:w \l_fp_input_a_exponent_int < \c_zero
      \l_fp_input_a_exponent_int -\l_fp_input_a_exponent_int
      \l_fp_output_sign_int \c_minus_one
    \else:
      \l_fp_output_sign_int \c_one
    \fi:
    \if_int_compare:w \l_fp_input_a_exponent_int > \c_nine
      \tl_set:Nx \l_fp_tmp_tl
        {
          \exp_after:wN \use_i:nn
            \int_use:N \l_fp_input_a_exponent_int
        }
      \l_fp_input_a_exponent_int
        \int_eval:w
          \l_fp_input_a_exponent_int - \l_fp_tmp_tl 0
        \int_eval_end:
      \fp_ln_const:nn { 10 } { \l_fp_tmp_tl }
      \int_advance:w \l_fp_exp_exponent_int \c_one
      \l_fp_output_integer_int  \l_fp_exp_integer_int
      \l_fp_output_decimal_int  \l_fp_exp_decimal_int
      \l_fp_output_extended_int \l_fp_exp_extended_int
      \l_fp_output_exponent_int \l_fp_exp_exponent_int
    \else:
      \l_fp_output_integer_int  \c_zero
      \l_fp_output_decimal_int  \c_zero
      \l_fp_output_extended_int \c_zero
      \l_fp_output_exponent_int \c_zero
    \fi:
    \fp_ln_exponent_units:
  }
%    \end{macrocode}
%   Next the smaller powers of ten, which will need to be combined
%   with the above: always an additive process.
%    \begin{macrocode}
\cs_new_protected_nopar:Npn \fp_ln_exponent_units:
  {
    \if_int_compare:w \l_fp_input_a_exponent_int > \c_zero
      \fp_ln_const:nn { 10 } { \int_use:N \l_fp_input_a_exponent_int }
      \fp_ln_normalise:
      \fp_add:NNNNNNNNN
        \l_fp_exp_integer_int \l_fp_exp_decimal_int \l_fp_exp_extended_int
        \l_fp_output_integer_int \l_fp_output_decimal_int
          \l_fp_output_extended_int
        \l_fp_output_integer_int \l_fp_output_decimal_int
          \l_fp_output_extended_int
    \fi:
    \fp_ln_mantissa:
  }
%    \end{macrocode}
%   The smaller table-based parts may need to be exponent shifted so that
%   they stay in line with the larger parts. This is similar to the
%   approach in other places, but here there is a need to watch the
%   extended part of the number.
%    \begin{macrocode}
\cs_new_protected_nopar:Npn \fp_ln_normalise:
  {
    \if_int_compare:w \l_fp_exp_exponent_int < \l_fp_output_exponent_int
      \int_advance:w \l_fp_exp_decimal_int \c_one_thousand_million
      \exp_after:wN \use_i:nn \exp_after:wN
        \fp_ln_normalise_aux:NNNNNNNNN
        \int_use:N \l_fp_exp_decimal_int
       \exp_after:wN \fp_ln_normalise:
     \fi:
  }
\cs_new_protected_nopar:Npn \fp_ln_normalise_aux:NNNNNNNNN #1#2#3#4#5#6#7#8#9
  {
    \if_int_compare:w \l_fp_exp_integer_int = \c_zero
      \l_fp_exp_decimal_int #1#2#3#4#5#6#7#8 \scan_stop:
    \else:
      \tl_set:Nx \l_fp_tmp_tl
        {
          \int_use:N \l_fp_exp_integer_int
          #1#2#3#4#5#6#7#8
        }
      \l_fp_exp_integer_int \c_zero
      \l_fp_exp_decimal_int \l_fp_tmp_tl \scan_stop:
    \fi:
    \tex_divide:D \l_fp_exp_extended_int \c_ten
    \tl_set:Nx \l_fp_tmp_tl
      {
        #9
        \int_use:N \l_fp_exp_extended_int
      }
    \l_fp_exp_extended_int \l_fp_tmp_tl \scan_stop:
    \int_advance:w \l_fp_exp_exponent_int \c_one
  }
%    \end{macrocode}
%   The next phase is to decompose the mantissa by division by two to
%   leave a value which is in the range $ 1 \le x < 2 $. The sum of the
%   two powers needs to take account of the sign of the output: if it
%   is negative then the result gets \emph{smaller} as the mantissa gets
% \emph{bigger}.
%    \begin{macrocode}
\cs_new_protected_nopar:Npn \fp_ln_mantissa:
  {
    \l_fp_count_int \c_zero
    \l_fp_input_a_extended_int \c_zero
    \fp_ln_mantissa_aux:
    \if_int_compare:w \l_fp_count_int > \c_zero
      \fp_ln_const:nn { 2 } { \int_use:N \l_fp_count_int }
      \fp_ln_normalise:
      \if_int_compare:w \l_fp_output_sign_int > \c_zero
        \exp_after:wN \fp_add:NNNNNNNNN
      \else:
        \exp_after:wN \fp_sub:NNNNNNNNN
      \fi:
      \l_fp_output_integer_int \l_fp_output_decimal_int
        \l_fp_output_extended_int
      \l_fp_exp_integer_int \l_fp_exp_decimal_int \l_fp_exp_extended_int
      \l_fp_output_integer_int \l_fp_output_decimal_int
        \l_fp_output_extended_int
    \fi:
    \if_int_compare:w
      \int_eval:w
        \l_fp_input_a_integer_int + \l_fp_input_a_decimal_int > \c_one
      \exp_after:wN \fp_ln_Taylor:
    \fi:
  }
\cs_new_protected_nopar:Npn \fp_ln_mantissa_aux:
  {
    \if_int_compare:w \l_fp_input_a_integer_int > \c_one
      \int_advance:w \l_fp_count_int \c_one
      \fp_ln_mantissa_divide_two:
      \exp_after:wN \fp_ln_mantissa_aux:
    \fi:
  }
%    \end{macrocode}
%   A fast one-shot division by two.
%    \begin{macrocode}
\cs_new_protected_nopar:Npn \fp_ln_mantissa_divide_two:
  {
    \if_int_odd:w \l_fp_input_a_decimal_int
      \int_advance:w \l_fp_input_a_extended_int \c_one_thousand_million
    \fi:
    \if_int_odd:w \l_fp_input_a_integer_int
      \int_advance:w \l_fp_input_a_decimal_int \c_one_thousand_million
    \fi:
    \tex_divide:D \l_fp_input_a_integer_int  \c_two
    \tex_divide:D \l_fp_input_a_decimal_int  \c_two
    \tex_divide:D \l_fp_input_a_extended_int \c_two
  }
%    \end{macrocode}
%   Recovering constants makes use of the same auxiliary code as for
%   exponents.
%    \begin{macrocode}
\cs_new_protected_nopar:Npn \fp_ln_const:nn #1#2
  {
    \exp_after:wN \exp_after:wN \exp_after:wN
      \fp_exp_integer_const:nnnn
      \cs:w c_fp_ln_ #1 _ #2 _tl \cs_end:
  }
%    \end{macrocode}
%   The Taylor series for the logarithm function is best implemented using
%   the identity
%   \[
%     \ln(x) = \ln\left( \frac{y + 1}{y - 1} \right)
%   \]
%   with
%   \[
%     y = \frac{x - 1}{x + 1}
%   \]
%   This leads to the series
%   \[
%     \ln(x)
%       = 2y
%         \left(
%           1 + y^{2}
%             \left(
%               \frac{1}{3} + y^{2}
%                 \left(
%                   \frac{1}{5} + y^{2}
%                     \left(
%                       \frac{1}{7} + y^{2}
%                         \left(
%                           \frac{1}{9} + \cdots
%                         \right)
%                     \right)
%                 \right)
%             \right)
%         \right)
%   \]
%   This expansion has the advantage that a lot of the work can be
%   loaded up early by finding $ y^{2} $ before the loop itself starts.
%   (In practice, the implementation does the multiplication by two at the
%   end of the loop, and expands out the brackets as this is an overall
%   more efficient approach.)
%
%   At the implementation level, the code starts by calculating $ y $
%   and storing that in input \texttt{a} (which is no longer needed
%   for other purposes). That is done using the full division system
%   avoiding the parsing step. The value is then switched to a fixed-point
%   representation. There is then some shuffling to get all of the working
%   space set up. At this stage, a lot of registers are in use and so
%   the Taylor series is calculated within a group so that the
%   \texttt{output} variables can be used to hold the result. The value
%   of $ y^{2} $ is held in input \texttt{b} (there are a few
%   assignments saved by choosing this over \texttt{a}), while input
%   \texttt{a} is used for the \enquote{loop value}.
%    \begin{macrocode}
\cs_new_protected_nopar:Npn \fp_ln_Taylor:
  {
    \group_begin:
      \l_fp_input_a_integer_int \c_zero
      \l_fp_input_a_exponent_int \c_zero
      \l_fp_input_b_integer_int \c_two
      \l_fp_input_b_decimal_int \l_fp_input_a_decimal_int
      \l_fp_input_b_exponent_int \c_zero
      \fp_div_internal:
      \fp_ln_fixed:
      \l_fp_input_a_integer_int  \l_fp_output_integer_int
      \l_fp_input_a_decimal_int  \l_fp_output_decimal_int
      \l_fp_input_a_exponent_int \l_fp_output_exponent_int
      \l_fp_input_a_extended_int \c_zero
      \l_fp_output_decimal_int \c_zero
      \l_fp_output_decimal_int  \l_fp_input_a_decimal_int
      \l_fp_output_extended_int \l_fp_input_a_extended_int
      \fp_mul:NNNNNN
        \l_fp_input_a_decimal_int \l_fp_input_a_extended_int
        \l_fp_input_a_decimal_int \l_fp_input_a_extended_int
        \l_fp_input_b_decimal_int \l_fp_input_b_extended_int
      \l_fp_count_int \c_one
      \fp_ln_Taylor_aux:
      \cs_set_protected_nopar:Npx \fp_tmp:w
        {
          \group_end:
          \exp_not:N \l_fp_exp_decimal_int
            \int_use:N \l_fp_output_decimal_int \scan_stop:
          \exp_not:N \l_fp_exp_extended_int
            \int_use:N \l_fp_output_extended_int \scan_stop:
          \exp_not:N \l_fp_exp_exponent_int
            \int_use:N \l_fp_output_exponent_int \scan_stop:
        }
    \fp_tmp:w
%    \end{macrocode}
%   After the loop part of the Taylor series, the factor of $ 2 $ needs
%   to be included. The total for the result can then be constructed.
%    \begin{macrocode}
    \int_advance:w \l_fp_exp_decimal_int \l_fp_exp_decimal_int
    \if_int_compare:w \l_fp_exp_extended_int < \c_five_hundred_million
    \else:
      \int_advance:w \l_fp_exp_extended_int -\c_five_hundred_million
      \int_advance:w \l_fp_exp_decimal_int \c_one
    \fi:
    \int_advance:w \l_fp_exp_extended_int \l_fp_exp_extended_int
    \if_int_compare:w \l_fp_output_sign_int > \c_zero
      \exp_after:wN \fp_add:NNNNNNNNN
    \else:
      \exp_after:wN \fp_sub:NNNNNNNNN
    \fi:
    \l_fp_output_integer_int \l_fp_output_decimal_int
      \l_fp_output_extended_int
    \c_zero \l_fp_exp_decimal_int \l_fp_exp_extended_int
    \l_fp_output_integer_int \l_fp_output_decimal_int
      \l_fp_output_extended_int
  }
%    \end{macrocode}
% The usual shifts to move to fixed-point working. This is done using
% the \texttt{output} registers as this saves a reassignment here.
%    \begin{macrocode}
\cs_new_protected_nopar:Npn \fp_ln_fixed:
  {
    \if_int_compare:w \l_fp_output_exponent_int < \c_zero
      \int_advance:w \l_fp_output_decimal_int \c_one_thousand_million
      \exp_after:wN \use_i:nn \exp_after:wN
        \fp_ln_fixed_aux:NNNNNNNNN
        \int_use:N \l_fp_output_decimal_int
       \exp_after:wN \fp_ln_fixed:
     \fi:
  }
\cs_new_protected_nopar:Npn \fp_ln_fixed_aux:NNNNNNNNN #1#2#3#4#5#6#7#8#9
  {
    \if_int_compare:w \l_fp_output_integer_int = \c_zero
      \l_fp_output_decimal_int #1#2#3#4#5#6#7#8 \scan_stop:
    \else:
      \tl_set:Nx \l_fp_tmp_tl
        {
          \int_use:N \l_fp_output_integer_int
          #1#2#3#4#5#6#7#8
        }
      \l_fp_output_integer_int \c_zero
      \l_fp_output_decimal_int \l_fp_tmp_tl \scan_stop:
    \fi:
    \int_advance:w \l_fp_output_exponent_int \c_one
  }
%    \end{macrocode}
%   The main loop for the Taylor series: unlike some of the other similar
%   functions, the result here is not the final value and is therefore
%   subject to further manipulation outside of the loop.
%    \begin{macrocode}
\cs_new_protected_nopar:Npn \fp_ln_Taylor_aux:
  {
    \int_advance:w \l_fp_count_int \c_two
    \fp_mul:NNNNNN
      \l_fp_input_a_decimal_int \l_fp_input_a_extended_int
      \l_fp_input_b_decimal_int \l_fp_input_b_extended_int
      \l_fp_input_a_decimal_int \l_fp_input_a_extended_int
    \if_int_compare:w
      \int_eval:w
        \l_fp_input_a_decimal_int + \l_fp_input_a_extended_int
        > \c_zero
      \fp_div_integer:NNNNN
        \l_fp_input_a_decimal_int \l_fp_input_a_extended_int
        \l_fp_count_int
        \l_fp_exp_decimal_int \l_fp_exp_extended_int
        \int_advance:w \l_fp_output_decimal_int \l_fp_exp_decimal_int
        \int_advance:w \l_fp_output_extended_int \l_fp_exp_extended_int
        \if_int_compare:w \l_fp_output_extended_int < \c_one_thousand_million
        \else:
          \int_advance:w \l_fp_output_decimal_int \c_one
          \int_advance:w \l_fp_output_extended_int
            -\c_one_thousand_million
        \fi:
        \if_int_compare:w \l_fp_output_decimal_int < \c_one_thousand_million
        \else:
          \int_advance:w \l_fp_output_integer_int \c_one
          \int_advance:w \l_fp_output_decimal_int
            -\c_one_thousand_million
        \fi:
      \exp_after:wN \fp_ln_Taylor_aux:
    \fi:
  }
%    \end{macrocode}
% \end{macro}
% \end{macro}
% \end{macro}
% \end{macro}
% \end{macro}
% \end{macro}
% \end{macro}
% \end{macro}
% \end{macro}
% \end{macro}
% \end{macro}
% \end{macro}
% \end{macro}
% \end{macro}
% \end{macro}
% \end{macro}
% \end{macro}
%
% \begin{macro}{\fp_pow:Nn, \fp_pow:cn}
% \UnitTested
% \begin{macro}{\fp_gpow:Nn,\fp_gpow:cn}
% \UnitTested
% \begin{macro}[aux]{\fp_pow_aux:NNn}
% \begin{macro}[aux]{\fp_pow_aux_i:}
% \begin{macro}[aux]{\fp_pow_positive:}
% \begin{macro}[aux]{\fp_pow_negative:}
% \begin{macro}[aux]{\fp_pow_aux_ii:}
% \begin{macro}[aux]{\fp_pow_aux_iii:}
% \begin{macro}[aux]{\fp_pow_aux_iv:}
%   The approach used for working out powers is to first filter out the
%   various special cases and then do most of the work using the
%   logarithm and exponent functions. The two storage areas are used
%   in the reverse of the `natural' logic as this avoids some
%   re-assignment in the sanity checking code.
%    \begin{macrocode}
\cs_new_protected_nopar:Npn \fp_pow:Nn  { \fp_pow_aux:NNn \tl_set:Nn }
\cs_new_protected_nopar:Npn \fp_gpow:Nn { \fp_pow_aux:NNn \tl_gset:Nn }
\cs_generate_variant:Nn \fp_pow:Nn  { c }
\cs_generate_variant:Nn \fp_gpow:Nn { c }
\cs_new_protected_nopar:Npn \fp_pow_aux:NNn #1#2#3
  {
    \group_begin:
      \fp_read:N #2
      \l_fp_input_b_sign_int     \l_fp_input_a_sign_int
      \l_fp_input_b_integer_int  \l_fp_input_a_integer_int
      \l_fp_input_b_decimal_int  \l_fp_input_a_decimal_int
      \l_fp_input_b_exponent_int \l_fp_input_a_exponent_int
      \fp_split:Nn a {#3}
      \fp_standardise:NNNN
        \l_fp_input_a_sign_int
        \l_fp_input_a_integer_int
        \l_fp_input_a_decimal_int
        \l_fp_input_a_exponent_int
      \if_int_compare:w
        \int_eval:w
          \l_fp_input_b_integer_int + \l_fp_input_b_decimal_int
          = \c_zero
         \if_int_compare:w
           \int_eval:w
             \l_fp_input_a_integer_int + \l_fp_input_a_decimal_int
             = \c_zero
             \cs_set_protected_nopar:Npx \fp_tmp:w ##1##2
               {
                 \group_end:
                 ##1 ##2 { \c_undefined_fp }
               }
           \else:
             \cs_set_protected_nopar:Npx \fp_tmp:w ##1##2
               {
                 \group_end:
                 ##1 ##2 { \c_zero_fp }
               }
          \fi:
       \else:
         \if_int_compare:w
           \int_eval:w
             \l_fp_input_a_integer_int + \l_fp_input_a_decimal_int
             = \c_zero
             \cs_set_protected_nopar:Npx \fp_tmp:w ##1##2
               {
                 \group_end:
                 ##1 ##2 { \c_one_fp }
               }
           \else:
             \exp_after:wN \exp_after:wN \exp_after:wN
               \fp_pow_aux_i:
          \fi:
       \fi:
    \fp_tmp:w #1 #2
}
%    \end{macrocode}
%   Simply using the logarithm function directly will fail when negative
%   numbers are raised to integer powers, which is a mathematically valid
%   operation. So there are some more tests to make, after forcing the
%   power into an integer and decimal parts, if necessary.
%    \begin{macrocode}
\cs_new_protected_nopar:Npn \fp_pow_aux_i:
  {
    \if_int_compare:w \l_fp_input_b_sign_int > \c_zero
      \tl_set:Nn \l_fp_sign_tl { + }
      \exp_after:wN \fp_pow_aux_ii:
    \else:
      \l_fp_input_a_extended_int \c_zero
      \if_int_compare:w \l_fp_input_a_exponent_int < \c_ten
        \group_begin:
        \fp_extended_normalise:
        \if_int_compare:w
          \int_eval:w
            \l_fp_input_a_decimal_int + \l_fp_input_a_extended_int
            = \c_zero
           \group_end:
          \tl_set:Nn \l_fp_sign_tl { - }
          \exp_after:wN \exp_after:wN \exp_after:wN
          \exp_after:wN \exp_after:wN \exp_after:wN
          \exp_after:wN \fp_pow_aux_ii:
        \else:
          \group_end:
          \cs_set_protected_nopar:Npx \fp_tmp:w ##1##2
            {
              \group_end:
              ##1 ##2 { \c_undefined_fp }
           }
        \fi:
      \else:
        \cs_set_protected_nopar:Npx \fp_tmp:w ##1##2
          {
            \group_end:
            ##1 ##2 { \c_undefined_fp }
         }
      \fi:
    \fi:
  }
%    \end{macrocode}
%   The approach used here for powers works well in most cases but gives
%   poorer results for negative integer powers, which often have exact
%   values.  So there is some filtering to do. For negative powers where
%   the power is small, an alternative approach is used in which the
%   positive value is worked out and the reciprocal is then taken. The
%   filtering is unfortunately rather long.
%    \begin{macrocode}
\cs_new_protected_nopar:Npn \fp_pow_aux_ii:
  {
    \if_int_compare:w \l_fp_input_a_sign_int > \c_zero
      \exp_after:wN \fp_pow_aux_iv:
    \else:
      \if_int_compare:w \l_fp_input_a_exponent_int < \c_ten
        \group_begin:
        \l_fp_input_a_extended_int \c_zero
        \fp_extended_normalise:
        \if_int_compare:w \l_fp_input_a_decimal_int = \c_zero
          \if_int_compare:w \l_fp_input_a_integer_int > \c_ten
            \group_end:
            \exp_after:wN \exp_after:wN \exp_after:wN
            \exp_after:wN \exp_after:wN \exp_after:wN
            \exp_after:wN \exp_after:wN \exp_after:wN
            \exp_after:wN \exp_after:wN \exp_after:wN
            \exp_after:wN \exp_after:wN \exp_after:wN
              \fp_pow_aux_iv:
          \else:
            \group_end:
            \exp_after:wN \exp_after:wN \exp_after:wN
            \exp_after:wN \exp_after:wN \exp_after:wN
            \exp_after:wN \exp_after:wN \exp_after:wN
            \exp_after:wN \exp_after:wN \exp_after:wN
            \exp_after:wN \exp_after:wN \exp_after:wN
              \exp_after:wN \fp_pow_aux_iii:
          \fi:
        \else:
          \group_end:
            \exp_after:wN \exp_after:wN \exp_after:wN
            \exp_after:wN \exp_after:wN \exp_after:wN
              \exp_after:wN \fp_pow_aux_iv:
        \fi:
      \else:
        \exp_after:wN \exp_after:wN \exp_after:wN
          \fp_pow_aux_iv:
      \fi:
    \fi:
    \cs_set_protected_nopar:Npx \fp_tmp:w ##1##2
      {
        \group_end:
        ##1 ##2
          {
            \l_fp_sign_tl
            \int_use:N \l_fp_output_integer_int
            .
            \exp_after:wN \use_none:n
              \int_value:w \int_eval:w
                 \l_fp_output_decimal_int + \c_one_thousand_million
            e
            \int_use:N \l_fp_output_exponent_int
          }
      }
  }
%    \end{macrocode}
%   For the small negative integer powers, the calculation is done for
%   the positive power and the reciprocal is then taken.
%    \begin{macrocode}
\cs_new_protected_nopar:Npn \fp_pow_aux_iii:
  {
    \l_fp_input_a_sign_int \c_one
    \fp_pow_aux_iv:
    \l_fp_input_a_integer_int  \c_one
    \l_fp_input_a_decimal_int  \c_zero
    \l_fp_input_a_exponent_int \c_zero
    \l_fp_input_b_integer_int  \l_fp_output_integer_int
    \l_fp_input_b_decimal_int  \l_fp_output_decimal_int
    \l_fp_input_b_exponent_int \l_fp_output_exponent_int
    \fp_div_internal:
  }
%    \end{macrocode}
%   The business end of the code starts by finding the logarithm of the
%   given base. There is a bit of a shuffle so that this does not have
%   to be re-parsed and so that the output ends up in the correct place.
%   There is also a need to enable using the short-cut for a
%   pre-calculated result. The internal part of the multiplication
%   function can then be used to do the second part of the calculation
%   directly. There is some more set up before doing the exponential:
%   the idea here is to deactivate some internals so that everything works
%   smoothly.
%    \begin{macrocode}
\cs_new_protected_nopar:Npn \fp_pow_aux_iv:
  {
    \group_begin:
      \l_fp_input_a_integer_int  \l_fp_input_b_integer_int
      \l_fp_input_a_decimal_int  \l_fp_input_b_decimal_int
      \l_fp_input_a_exponent_int \l_fp_input_b_exponent_int
      \fp_ln_internal:
      \cs_set_protected_nopar:Npx \fp_tmp:w
        {
          \group_end:
          \exp_not:N \l_fp_input_b_sign_int
            \int_use:N \l_fp_output_sign_int \scan_stop:
          \exp_not:N \l_fp_input_b_integer_int
            \int_use:N \l_fp_output_integer_int \scan_stop:
          \exp_not:N \l_fp_input_b_decimal_int
            \int_use:N \l_fp_output_decimal_int \scan_stop:
          \exp_not:N \l_fp_input_b_extended_int
            \int_use:N \l_fp_output_extended_int \scan_stop:
          \exp_not:N \l_fp_input_b_exponent_int
            \int_use:N \l_fp_output_exponent_int \scan_stop:
        }
    \fp_tmp:w
    \l_fp_input_a_extended_int  \c_zero
    \fp_mul:NNNNNNNNN
      \l_fp_input_a_integer_int \l_fp_input_a_decimal_int
        \l_fp_input_a_extended_int
      \l_fp_input_b_integer_int \l_fp_input_b_decimal_int
        \l_fp_input_b_extended_int
      \l_fp_input_a_integer_int \l_fp_input_a_decimal_int
        \l_fp_input_a_extended_int
    \int_advance:w \l_fp_input_a_exponent_int \l_fp_input_b_exponent_int
    \l_fp_output_integer_int  \c_zero
    \l_fp_output_decimal_int  \c_zero
    \l_fp_output_exponent_int \c_zero
    \cs_set_eq:NN \fp_exp_const:Nx \use_none:nn
    \fp_exp_internal:
  }
%    \end{macrocode}
% \end{macro}
% \end{macro}
% \end{macro}
% \end{macro}
% \end{macro}
% \end{macro}
% \end{macro}
% \end{macro}
% \end{macro}
%
% \subsection{Tests for special values}
%
% \begin{macro}[pTF]{\fp_if_undefined:N}
% \UnitTested
%   Testing for an undefined value is easy.
%    \begin{macrocode}
\prg_new_conditional:Npnn \fp_if_undefined:N #1 { p , T , F , TF }
  {
    \if_meaning:w #1 \c_undefined_fp
      \prg_return_true:
    \else:
      \prg_return_false:
    \fi:
  }
%    \end{macrocode}
% \end{macro}
%
% \begin{macro}[pTF]{\fp_if_zero:N}
% \UnitTested
%   Testing for a zero fixed-point is also easy.
%    \begin{macrocode}
\prg_new_conditional:Npnn \fp_if_zero:N #1 { p , T , F , TF }
  {
    \if_meaning:w #1 \c_zero_fp
      \prg_return_true:
    \else:
      \prg_return_false:
    \fi:
  }
%    \end{macrocode}
% \end{macro}
%
% \subsection{Floating-point conditionals}
%
% \begin{macro}[TF]{\fp_compare:nNn}
% \begin{macro}[TF]{\fp_compare:NNN}
% \UnitTested
% \begin{macro}[aux]{\fp_compare_aux:N}
% \begin{macro}[aux]{\fp_compare_=:}
% \begin{macro}[aux]{\fp_compare_<:}
% \begin{macro}[aux]{\fp_compare_<_aux:}
% \begin{macro}[aux]{\fp_compare_absolute_a>b:}
% \begin{macro}[aux]{\fp_compare_absolute_a<b:}
% \begin{macro}[aux]{\fp_compare_>:}
%   The idea for the comparisons is to provide two versions: slower and
%   faster. The lead off for both is the same: get the two numbers
%   read and then look for a function to handle the comparison.
%    \begin{macrocode}
\prg_new_protected_conditional:Npnn \fp_compare:nNn #1#2#3 { T , F , TF }
  {
    \group_begin:
      \fp_split:Nn a {#1}
      \fp_standardise:NNNN
        \l_fp_input_a_sign_int
        \l_fp_input_a_integer_int
        \l_fp_input_a_decimal_int
        \l_fp_input_a_exponent_int
      \fp_split:Nn b {#3}
      \fp_standardise:NNNN
        \l_fp_input_b_sign_int
        \l_fp_input_b_integer_int
        \l_fp_input_b_decimal_int
        \l_fp_input_b_exponent_int
      \fp_compare_aux:N #2
  }
\prg_new_protected_conditional:Npnn \fp_compare:NNN #1#2#3 { T , F , TF }
  {
    \group_begin:
      \fp_read:N #3
      \l_fp_input_b_sign_int     \l_fp_input_a_sign_int
      \l_fp_input_b_integer_int  \l_fp_input_a_integer_int
      \l_fp_input_b_decimal_int  \l_fp_input_a_decimal_int
      \l_fp_input_b_exponent_int \l_fp_input_a_exponent_int
      \fp_read:N #1
      \fp_compare_aux:N #2
  }
\cs_new_protected_nopar:Npn \fp_compare_aux:N #1
  {
    \cs_if_exist:cTF { fp_compare_#1: }
      { \use:c { fp_compare_#1: } }
      {
        \group_end:
        \prg_return_false:
      }
  }
%    \end{macrocode}
%   For equality, the test is pretty easy as things are either equal or
%   they are not.
%    \begin{macrocode}
\cs_new_protected_nopar:cpn { fp_compare_=: }
  {
    \if_int_compare:w \l_fp_input_a_sign_int = \l_fp_input_b_sign_int
      \if_int_compare:w \l_fp_input_a_integer_int = \l_fp_input_b_integer_int
        \if_int_compare:w \l_fp_input_a_decimal_int = \l_fp_input_b_decimal_int
          \if_int_compare:w
            \l_fp_input_a_exponent_int = \l_fp_input_b_exponent_int
            \group_end:
            \prg_return_true:
          \else:
            \group_end:
            \prg_return_false:
          \fi:
        \else:
          \group_end:
          \prg_return_false:
        \fi:
      \else:
        \group_end:
        \prg_return_false:
      \fi:
    \else:
      \group_end:
      \prg_return_false:
    \fi:
  }
%    \end{macrocode}
%   Comparing two values is quite complex. First, there is a filter step
%   to check if one or other of the given values is zero. If it is then
%   the result is relatively easy to determine.
%    \begin{macrocode}
\cs_new_protected_nopar:cpn { fp_compare_>: }
  {
    \if_int_compare:w \int_eval:w
      \l_fp_input_a_integer_int + \l_fp_input_a_decimal_int
      = \c_zero
      \if_int_compare:w \int_eval:w
        \l_fp_input_b_integer_int + \l_fp_input_b_decimal_int
        = \c_zero
        \group_end:
        \prg_return_false:
      \else:
        \if_int_compare:w \l_fp_input_b_sign_int > \c_zero
          \group_end:
          \prg_return_false:
        \else:
          \group_end:
          \prg_return_true:
        \fi:
      \fi:
    \else:
      \if_int_compare:w \int_eval:w
        \l_fp_input_b_integer_int + \l_fp_input_b_decimal_int
        = \c_zero
        \if_int_compare:w \l_fp_input_a_sign_int > \c_zero
          \group_end:
          \prg_return_true:
        \else:
          \group_end:
          \prg_return_false:
        \fi:
      \else:
        \use:c { fp_compare_>_aux: }
      \fi:
    \fi:
  }
%    \end{macrocode}
%   Next, check the sign of the input: this again may give an obvious
%   result. If both signs are the same, then hand off to comparing the
%   absolute values.
%    \begin{macrocode}
\cs_new_protected_nopar:cpn { fp_compare_>_aux: }
  {
    \if_int_compare:w \l_fp_input_a_sign_int > \l_fp_input_b_sign_int
      \group_end:
      \prg_return_true:
    \else:
      \if_int_compare:w \l_fp_input_a_sign_int < \l_fp_input_b_sign_int
        \group_end:
        \prg_return_false:
      \else:
        \if_int_compare:w \l_fp_input_a_sign_int > \c_zero
          \use:c { fp_compare_absolute_a>b: }
        \else:
          \use:c { fp_compare_absolute_a<b: }
        \fi:
      \fi:
    \fi:
  }
%    \end{macrocode}
%   Rather long runs of checks, as there is the need to go through each
%   layer of the input and do the comparison. There is also the need to
%   avoid messing up with equal inputs at each stage.
%    \begin{macrocode}
\cs_new_protected_nopar:cpn { fp_compare_absolute_a>b: }
  {
    \if_int_compare:w \l_fp_input_a_exponent_int > \l_fp_input_b_exponent_int
      \group_end:
      \prg_return_true:
    \else:
      \if_int_compare:w \l_fp_input_a_exponent_int < \l_fp_input_b_exponent_int
        \group_end:
        \prg_return_false:
      \else:
        \if_int_compare:w \l_fp_input_a_integer_int > \l_fp_input_b_integer_int
          \group_end:
          \prg_return_true:
        \else:
          \if_int_compare:w
            \l_fp_input_a_integer_int < \l_fp_input_b_integer_int
            \group_end:
            \prg_return_false:
          \else:
            \if_int_compare:w
              \l_fp_input_a_decimal_int > \l_fp_input_b_decimal_int
              \group_end:
              \prg_return_true:
            \else:
              \group_end:
              \prg_return_false:
            \fi:
          \fi:
        \fi:
      \fi:
    \fi:
  }
\cs_new_protected_nopar:cpn { fp_compare_absolute_a<b: }
  {
    \if_int_compare:w \l_fp_input_b_exponent_int > \l_fp_input_a_exponent_int
      \group_end:
      \prg_return_true:
    \else:
      \if_int_compare:w \l_fp_input_b_exponent_int < \l_fp_input_a_exponent_int
        \group_end:
        \prg_return_false:
      \else:
        \if_int_compare:w \l_fp_input_b_integer_int > \l_fp_input_a_integer_int
          \group_end:
          \prg_return_true:
        \else:
          \if_int_compare:w
            \l_fp_input_b_integer_int < \l_fp_input_a_integer_int
            \group_end:
            \prg_return_false:
          \else:
            \if_int_compare:w
              \l_fp_input_b_decimal_int > \l_fp_input_a_decimal_int
              \group_end:
              \prg_return_true:
            \else:
              \group_end:
              \prg_return_false:
            \fi:
          \fi:
        \fi:
      \fi:
    \fi:
  }
%    \end{macrocode}
%   This is just a case of reversing the two input values and then
%   running the tests already defined.
%    \begin{macrocode}
\cs_new_protected_nopar:cpn { fp_compare_<: }
  {
    \tl_set:Nx \l_fp_tmp_tl
      {
        \int_set:Nn \exp_not:N \l_fp_input_a_sign_int
          { \int_use:N \l_fp_input_b_sign_int }
        \int_set:Nn \exp_not:N \l_fp_input_a_integer_int
          { \int_use:N \l_fp_input_b_integer_int }
        \int_set:Nn \exp_not:N \l_fp_input_a_decimal_int
          { \int_use:N \l_fp_input_b_decimal_int }
        \int_set:Nn \exp_not:N \l_fp_input_a_exponent_int
          { \int_use:N \l_fp_input_b_exponent_int }
        \int_set:Nn \exp_not:N \l_fp_input_b_sign_int
          { \int_use:N \l_fp_input_a_sign_int }
        \int_set:Nn \exp_not:N \l_fp_input_b_integer_int
          { \int_use:N \l_fp_input_a_integer_int }
        \int_set:Nn \exp_not:N \l_fp_input_b_decimal_int
          { \int_use:N \l_fp_input_a_decimal_int }
        \int_set:Nn \exp_not:N \l_fp_input_b_exponent_int
          { \int_use:N \l_fp_input_a_exponent_int }
      }
    \l_fp_tmp_tl
    \use:c { fp_compare_>: }
  }
%    \end{macrocode}
% \end{macro}
% \end{macro}
% \end{macro}
% \end{macro}
% \end{macro}
% \end{macro}
% \end{macro}
% \end{macro}
% \end{macro}
% 
% \begin{macro}[TF]{\fp_compare:n}
% \begin{macro}[aux]
%   {
%     \fp_compare_aux_i:w,  \fp_compare_aux_ii:w, \fp_compare_aux_iii:w,
%     \fp_compare_aux_iv:w, \fp_compare_aux_v:w,  \fp_compare_aux_vi:w,
%     \fp_compare_aux_vii:w
%   }
%   As \TeX{} cannot help out here, a daisy-chain of delimited functions
%   are used. This is very much a first-generation approach: revision will
%   be needed if these functions are really useful.
%    \begin{macrocode}
\prg_new_protected_conditional:Npnn \fp_compare:n #1 { T , F , TF }
  {
    \group_begin:
      \tl_set:Nx \l_fp_tmp_tl
        {
          \group_end:
          \fp_compare_aux_i:w #1 \exp_not:n { == \q_nil == \q_stop }
        }
     \l_fp_tmp_tl
  }
\cs_new_protected_nopar:Npn \fp_compare_aux_i:w #1 == #2 == #3 \q_stop
  {
    \quark_if_nil:nTF {#2}
      { \fp_compare_aux_ii:w #1 != \q_nil != \q_stop }
      { \fp_compare:nNnTF {#1} = {#2} \prg_return_true: \prg_return_false: }
  }
\cs_new_protected_nopar:Npn \fp_compare_aux_ii:w #1 != #2 != #3 \q_stop
  {
    \quark_if_nil:nTF {#2}
      { \fp_compare_aux_iii:w #1 <= \q_nil <= \q_stop }
      { \fp_compare:nNnTF {#1} = {#2} \prg_return_false: \prg_return_true: }
  }
\cs_new_protected_nopar:Npn \fp_compare_aux_iii:w #1 <= #2 <= #3 \q_stop
  {
    \quark_if_nil:nTF {#2}
      { \fp_compare_aux_iv:w #1 >= \q_nil >= \q_stop }
      { \fp_compare:nNnTF {#1} > {#2} \prg_return_false: \prg_return_true: }
  }
\cs_new_protected_nopar:Npn \fp_compare_aux_iv:w #1 >= #2 >= #3 \q_stop
  {
    \quark_if_nil:nTF {#2}
      { \fp_compare_aux_v:w #1 = \q_nil  \q_stop }
      { \fp_compare:nNnTF {#1} < {#2} \prg_return_false: \prg_return_true: }
  }
\cs_new_protected_nopar:Npn \fp_compare_aux_v:w #1 = #2 = #3 \q_stop
  {
    \quark_if_nil:nTF {#2}
      { \fp_compare_aux_vi:w #1 < \q_nil < \q_stop }
      { \fp_compare:nNnTF {#1} = {#2} \prg_return_true: \prg_return_false: }
  }
\cs_new_protected_nopar:Npn \fp_compare_aux_vi:w #1 < #2 < #3 \q_stop
  {
    \quark_if_nil:nTF {#2}
      { \fp_compare_aux_vii:w #1 > \q_nil > \q_stop }
      { \fp_compare:nNnTF {#1} < {#2} \prg_return_true: \prg_return_false: }
  }
\cs_new_protected_nopar:Npn \fp_compare_aux_vii:w #1 > #2 > #3 \q_stop
  {
    \quark_if_nil:nTF {#2}
      { \prg_return_false: }
      { \fp_compare:nNnTF {#1} > {#2} \prg_return_true: \prg_return_false: }
  }
%    \end{macrocode}
% \end{macro}
% \end{macro}
%
% \subsection{Messages}
%
% \begin{macro}{\fp_overflow_msg:}
% A generic overflow message, used whenever there is a possible
% overflow.
%    \begin{macrocode}
\msg_kernel_new:nnnn { fpu } { overflow }
  { Number~too~big. }
  {
    The~input~given~is~too~big~for~the~LaTeX~floating~point~unit. \\
    Further~errors~may~well~occur!
  }
\cs_new_protected_nopar:Npn \fp_overflow_msg:
  { \msg_kernel_error:nn { fpu } { overflow } }
%    \end{macrocode}
% \end{macro}
%
% \begin{macro}{\fp_exp_overflow_msg:}
% A slightly more helpful message for exponent overflows.
%    \begin{macrocode}
\msg_kernel_new:nnnn { fpu } { exponent-overflow }
  { Number~too~big~for~exponent~unit. }
  {
    The~exponent~of~the~input~given~is~too~big~for~the~floating~point~
    unit:~the~maximum~input~value~for~an~exponent~is~230.
  }
\cs_new_protected_nopar:Npn \fp_exp_overflow_msg:
  { \msg_kernel_error:nn { fpu } { exponent-overflow } }
%    \end{macrocode}
% \end{macro}
%
% \begin{macro}{\fp_ln_error_msg:}
% Logarithms are only valid for positive number
%    \begin{macrocode}
\msg_kernel_new:nnnn { fpu } { logarithm-input-error }
  { Invalid~input~to~ln~function. }
  { Logarithms~can~only~be~calculated~for~positive~numbers. }
\cs_new_protected_nopar:Npn \fp_ln_error_msg: {
  \msg_kernel_error:nn { fpu } { logarithm-input-error }
}
%    \end{macrocode}
% \end{macro}
%
% \begin{macro}{\fp_trig_overflow_msg:}
% A slightly more helpful message for trigonometric overflows.
%    \begin{macrocode}
\msg_kernel_new:nnnn { fpu } { trigonometric-overflow }
  { Number~too~big~for~trigonometry~unit. }
  {
    The~trigonometry~code~can~only~work~with~numbers~smaller~
    than~1000000000.
  }
\cs_new_protected_nopar:Npn \fp_trig_overflow_msg:
  { \msg_kernel_error:nn { fpu } { trigonometric-overflow } }
%    \end{macrocode}
% \end{macro}
%
%    \begin{macrocode}
%</initex|package>
%    \end{macrocode}
%
% \end{implementation}
%
%\PrintIndex
