% \iffalse meta-comment
%
%% File: l3unicode-data.dtx Copyright(C) 2014-2015 The LaTeX3 Project
%%
%% It may be distributed and/or modified under the conditions of the
%% LaTeX Project Public License (LPPL), either version 1.3c of this
%% license or (at your option) any later version.  The latest version
%% of this license is in the file
%%
%%    http://www.latex-project.org/lppl.txt
%%
%% This file is part of the "l3kernel bundle" (The Work in LPPL)
%% and all files in that bundle must be distributed together.
%%
%% The released version of this bundle is available from CTAN.
%%
%% -----------------------------------------------------------------------
%%
%% The development version of the bundle can be found at
%%
%%    http://www.latex-project.org/svnroot/experimental/trunk/
%%
%% for those people who are interested.
%%
%%%%%%%%%%%
%% NOTE: %%
%%%%%%%%%%%
%%
%%   Snapshots taken from the repository represent work in progress and may
%%   not work or may contain conflicting material!  We therefore ask
%%   people _not_ to put them into distributions, archives, etc. without
%%   prior consultation with the LaTeX Project Team.
%%
%% -----------------------------------------------------------------------
%%
%
% Both the driver and the script need \pkg{expl3}: as the script runs with
% plain \TeX{}, set up in generic mode.
%<*driver|script>
\input expl3-generic\relax
\GetIdInfo$Id$
  {L3 Case data script}
%</driver|script>
%
% The same approach as used in \pkg{DocStrip}: if \cs{documentclass}
% is undefined then skip the driver, allowing the file to be used directly.
% This works as the \cs{fi} is only seen if \LaTeX{} is not in use. The odd
% \cs{jobname} business allows the extraction to work with \LaTeX{} provided
% an appropriate \texttt{.ins} file is set up.
%<*gobble>
\ifx\jobname\relax
  \let\documentclass\undefined
\fi
\begingroup\expandafter\expandafter\expandafter\endgroup
\expandafter\ifx\csname documentclass\endcsname\relax
\else
  \csname fi\endcsname
%</gobble>
%
%<*driver>
  \documentclass[full]{l3doc}
  \begin{document}
    \DocInput{\jobname.dtx}
  \end{document}
%<*gobble>
\fi
%</gobble>
%</driver>
% \fi
%
% \title{^^A
%   The \textsf{l3unicode-data} script\\Unicode data script^^A
%   \thanks{This file describes v\ExplFileVersion,
%     last revised \ExplFileDate.}^^A
% }
%
% \author{^^A
%  The \LaTeX3 Project\thanks
%    {^^A
%      E-mail:
%        \href{mailto:latex-team@latex-project.org}
%          {latex-team@latex-project.org}^^A
%    }^^A
% }
%
% \date{Released \ExplFileDate}
%
% \maketitle
%
% \begin{documentation}
%
% The Unicode Consortium provide comprehensive data on the standard
% mapping of characters (or more formally codepoints) when carrying
% out various different case-changing functions:
% \begin{itemize}
%    \item Uppercasing
%    \item Lowercasing
%    \item Titlecasing (used for the first codepoint of a word:
%      may be subtly different to uppercasing)
%    \item Folding (removing case for comparison purposes: close
%      but not identical to lowercasing)
% \end{itemize}
% This data is available in machine readable format, such that many of
% the basics of case changing can be set up on an automated basis.
%
% This file provides a script which will read the raw Unicode files
% and convert the material to a form which can be used by \pkg{expl3}.
% As the conversions here cover the entire Unicode range, this cannot
% be carried out by \pdfTeX{}: at present, the script works only
% with \LuaTeX{}.
%
% Note that this file is designed such that running \LaTeX{} will typeset
% the documentation using any engine: the script will be run if the file
% is processed by plain \TeX{}, specifically the |luatex| command.
% This process requires the files |CaseFolding.txt|, |SpecialCasing.txt|
% and |UnicodeData.txt| from the \href{http://www.unicode.org/}^^A
% {Unicode Consortium website}.
%
% The file produced by this script, |l3unicode-data.def|, contains
% appropriate definitions for all of the data structures used by \pkg{expl3}
% for Unicode transformations. It also provides appropriate alternative
% definitions for use with \pdfTeX{}.
%
% \end{documentation}
%
% \begin{implementation}
%
% \section{\pkg{l3unicode-data} Implementation}
%
%    \begin{macrocode}
%<*script>
%    \end{macrocode}
%
% The driver part has loaded \pkg{expl3}: turn on the syntax environment.
%    \begin{macrocode}
\ExplSyntaxOn
%    \end{macrocode}
%
% \subsection{Setup}
% 
% \begin{variable}{\l__unicode_tmp_tl}
%   Scratch space.
%    \begin{macrocode}
\tl_new:N \l__unicode_tmp_tl
%    \end{macrocode}
% \end{variable}
%
% The first step is to generate a series of temporary variables to
% contain the data as it's extracted. This requires a nested loop
% to give a total of $100$ token lists.
%    \begin{macrocode}
\tl_map_inline:nn { 0123456789 }
  {
    \tl_map_inline:nn { 0123456789 }
      {
        \tl_new:c { l__unicode_lower_ #1 _X_ ##1 _tl }
        \tl_new:c { l__unicode_upper_ #1 _X_ ##1 _tl }
      }
  }
%    \end{macrocode}
%
% \begin{variable}{\l__unicode_compat_seq}
%   A sequence to hold the list of compatibility chars currently defined by
%   Unicode. This is needed for both case mapping and case folding (it's
%   defined by information in the master file |UnicodeData.txt|).
%    \begin{macrocode}
\seq_new:N \l__unicode_compat_seq
%    \end{macrocode}
% \end{variable}
%
% \begin{variable}^^A
%   {
%     \l__unicode_lower_exceptions_tl ,
%     \l__unicode_title_exceptions_tl ,
%     \l__unicode_upper_exceptions_tl
%   }
%   Exceptions to the usual mappings.
%    \begin{macrocode}
\tl_new:N \l__unicode_lower_exceptions_tl
\tl_new:N \l__unicode_title_exceptions_tl
\tl_new:N \l__unicode_upper_exceptions_tl
%    \end{macrocode}
% \end{variable}
%
% \begin{macro}{\__unicode_file_map:nn}
%   A utility function for handling file reading.
%    \begin{macrocode}
\cs_new_protected:Npn \__unicode_file_map:nn #1#2
  {
    \ior_open:Nn \g__unicode_data_ior {#1}
    \ior_if_eof:NTF \g__unicode_data_ior
      { \__msg_kernel_error:nnn { unicode } { file-not-found } {#1} }
      {
        \ior_str_map_inline:Nn \g__unicode_data_ior {#2}
        \ior_close:N \g__unicode_data_ior
      }
  }
\__msg_kernel_new:nnn { unicode } { file-not-found }
  { Could~not~find~data~file~'#1'. }
%    \end{macrocode}
% \end{macro}
%
% \begin{variable}{\g__unicode_data_ior}
% \begin{variable}{\g__unicode_result_iow}
%   Streams for reading and writing the data.
%    \begin{macrocode}
\ior_new:N \g__unicode_data_ior
\iow_new:N \g__unicode_result_iow
%    \end{macrocode}
% \end{variable}
% \end{variable}
%
% Open the data file for writing.
%    \begin{macrocode}
\iow_open:Nn \g__unicode_result_iow { l3unicode-data.def }
%    \end{macrocode}
%
% \subsection{Verbatim copying}
%
% \begin{macro}[int]{\__unicode_verb:}
% \begin{macro}[aux]{\__unicode_verb_auxi:w, \__unicode_verb_auxii:w}
% \begin{macro}[int]{\__unicode_verb_end:}
%   There are various bits of code which need to be transferred into the data
%   file from the source. This has to take place as part of the general writing
%   process so needs to be done without using DocStrip. That is achieved by
%   having a verbatim-copy mechanism available: this is all set up here.
%   As the line containing the \cs{__unicode_verb:} function will end up with a
%   (category code $12$) space at the start, there is a dedicated function to
%   clear this part up.
%    \begin{macrocode}
\group_begin:
  \char_set_catcode_other:n { `\^^M }%
  \cs_new_protected:Npn \__unicode_verb:%
    {%
      \group_begin:%
        \char_set_catcode_other:n { `\^^M }%
        \tex_endlinechar:D = `\^^M%
        \clist_map_inline:nn%
          { \\ , \{ , \} , \# , \^ , \% , \  }%
          { \char_set_catcode_other:n { `##1 } }%
        \__unicode_verb_auxi:w%
    }%
  \cs_new_protected:Npn \__unicode_verb_auxi:w#1^^M%
    {%
      \exp_after:wN \__unicode_verb_auxii:w \use_none:n #1 ^^M
    }%
  \cs_new_protected:Npn \__unicode_verb_auxii:w#1^^M%
    {%
      \str_if_eq_x:nnTF {#1} { \token_to_str:N \__unicode_verb_end: }%
        { \group_end: }%
        {%
          \iow_now:Nn \g__unicode_result_iow {#1}%
          \__unicode_verb_auxii:w%
        }%
    }%
\group_end:%
%    \end{macrocode}
% \end{macro}
% \end{macro}
% \end{macro}
%
% \subsection{Shared data}
%
%    \end{macrocode}
% There are various lines that now need to go at the start of the file.
% First, there is some header information.
%    \begin{macrocode}
\__unicode_verb:
%% This is the file l3unicode-data.def
%% generated using the script l3unicode-data.dtx.
%%
%% The data here are derived from the files
\__unicode_verb_end:
%    \end{macrocode}
% \begin{macro}{\__unicode_parse_line:w}
% \begin{macro}[aux]{\__unicode_parse_line_auxi:w}
%   Extract version information from the source files so this is
%   recorded. The main Unicode databse file doesn't have a header
%   but the other files do.
%    \begin{macrocode}
\use:x
  {
    \cs_new_protected:Npn
      \exp_not:N \__unicode_parse_line:w ##1 - ##2 \tl_to_str:n { .txt } 
  }
    #3 \scan_stop: #4
  {
    \tl_if_empty:nF {#2}
      {
        \ior_get_str:NN \g__unicode_data_ior \l__unicode_tmp_tl
        \exp_after:wN \__unicode_parse_line_auxi:w
          \l__unicode_tmp_tl \scan_stop:
        \iow_now:Nx \g__unicode_result_iow
          {
            \iow_char:N \%
            \iow_char:N \%
            \c_space_tl \c_space_tl \c_space_tl
            Version ~ #2 ~ dated \l__unicode_tmp_tl
          }
      }
  }
\group_begin:
\char_set_lccode:nn { `\* } { `\: }
\tex_lowercase:D
  {
    \group_end:
    \cs_new_protected:Npn \__unicode_parse_line_auxi:w #1 *
  }
       ~ #2 , ~ #3 ~ #4 \scan_stop:
  { \tl_set:Nn \l__unicode_tmp_tl { #2 , ~ #3 } }
\clist_map_inline:nn
  { UnicodeData , SpecialCasing , CaseFolding }
  {
    \ior_open:Nn \g__unicode_data_ior { #1 .txt }
    \ior_if_eof:NTF \g__unicode_data_ior
      { \__msg_kernel_error:nnn { unicode } { file-not-found } {#1} }
      {
        \iow_now:Nx \g__unicode_result_iow
          {
            \iow_char:N \%
            \iow_char:N \%
            \c_space_tl
            - ~ #1 .txt
          }
        \ior_get_str:NN \g__unicode_data_ior \l__unicode_tmp_tl
        \tl_set:Nx \l__unicode_tmp_tl
          {
            \l__unicode_tmp_tl
            \tl_to_str:n { - .txt }
          }
       \exp_after:wN \__unicode_parse_line:w
         \l__unicode_tmp_tl \scan_stop: {#1}
        \ior_close:N \g__unicode_data_ior
      }
  }
%    \end{macrocode}
% \end{macro}
% \end{macro}
% Finish off this part of the header.
%    \begin{macrocode}
\__unicode_verb:
%% which are maintained by the Unicode Consortium.
%%
\__unicode_verb_end:
%    \end{macrocode}
%    
% \begin{macro}[EXP]{\__unicode_zero_fill:N}
%   A short piece of reusable code for the header.
%    \begin{macrocode}
\cs_new:Npn \__unicode_zero_fill:N #1
  {
    \int_compare:nNnT #1 < \c_ten { 0 }
    \int_use:N #1
  }
%    \end{macrocode}
% \end{macro}
%
% Automatically include the current date.
%    \begin{macrocode}
\iow_now:Nx \g__unicode_result_iow
  {
    \iow_char:N \%
    \iow_char:N \%
    \c_space_tl
    Generated~on~
    \int_use:N \tex_year:D -
    \__unicode_zero_fill:N \tex_month:D -
    \__unicode_zero_fill:N \tex_day:D .
  }
\iow_now:Nx \g__unicode_result_iow
  {
    \iow_char:N \%
    \iow_char:N \%
  }
%    \end{macrocode}
% Write an identification line to the file: the file data here can't be set
% automatically and so will need to be edited by hand. As such, the data here
% the standard SVN filler.
%    \begin{macrocode}
\iow_now:Nx \g__unicode_result_iow
  {
    \exp_not:N \ProvidesExplFile
      { l3unicode-data.def } ~
      {
        \int_use:N \tex_year:D /
        \__unicode_zero_fill:N \tex_month:D / 
        \__unicode_zero_fill:N \tex_day:D 
      } ~
      { -1 } ~
      { L3~Unicode~data }
  }
%    \end{macrocode}
%
% \subsection{\pdfTeX{} support}
%
% As \pdfTeX{} does not support Unicode input natively, most of the data
% here will not be useful. Rather than use two separate mechanisms for
% each function depending on the engine, the system is designed such that
% \enquote{truncated} data structures are provided for \pdfTeX{}. These
% are coded here for direct transfer to the |.def| file, which can then
% abort loading when \pdfTeX{} is in use.
%
% The idea here is simple: map over all of the letters of the Latin
% alphabet and create appropriate token lists, then add all of the rest
% of the data structures.
%
% After the mapping, the small number of fixed data structures that are
% used for the special case conversions are created. These are mainly empty,
% but for cases where a match is possible (as the test char is in the \pdfTeX{}
% range), no-op data is included (as the \emph{output} would be out-of-range).
%    \begin{macrocode}
\__unicode_verb:
\pdftex_if_engine:T
  {
    \group_begin:
      \cs_set_protected:Npn \__unicode_tmp:NN #1#2
        {
          \quark_if_recursion_tail_stop:N #1
          \exp_after:wN \__unicode_tmp:NNNNNNN
            \tex_number:D \__int_eval:w `#1 \exp_after:wN \__int_eval_end:
            \tex_number:D \__int_eval:w 100 + `#2 \__int_eval_end:
            #1 #2
          \__unicode_tmp:NN
        }
       \cs_set_protected:Npn \__unicode_tmp:NNNNNNN #1#2#3#4#5#6#7
         {
           \tl_const:cx { c__unicode_fold_  #1 _X_ #2 _ tl } { #6#7 }
           \tl_const:cn { c__unicode_lower_ #1 _X_ #2 _ tl } { #6#7 }
           \tl_const:cn { c__unicode_upper_ #4 _X_ #5 _ tl } { #7#6 }
         }
       \__unicode_tmp:NN
         AaBbCcDdEeFfGgHhIiJjKkLlMmNnOoPpQqRrSsTtUuVvWwXxYyZz
        \q_recursion_tail ? \q_recursion_stop
    \group_end:
    \int_step_inline:nnnn { 0 } { 1 } { 9 }
      {
        \int_step_inline:nnnn { 0 } { 1 } { 9 }
          {
            \clist_map_inline:nn { fold , lower , upper }
              {
                \tl_if_exist:cF { c__unicode_ ####1 _ #1 _X_ ##1 _ tl }
                  { \tl_const:cn { c__unicode_ ####1 _ #1 _X_ ##1 _ tl } { } }
              }
          }
      }
    \tl_const:Nn \c__unicode_lower_exceptions_tl { }
    \tl_const:Nn \c__unicode_mixed_exceptions_tl { }
    \tl_const:Nn \c__unicode_upper_exceptions_tl { }
    \tl_const:Nn \c__unicode_std_sigma_tl   { }
    \tl_const:Nn \c__unicode_final_sigma_tl { }
    \tl_const:Nn \c__unicode_accents_lt_tl  { }
    \tl_const:Nn \c__unicode_dot_above_tl   { }
    \tl_const:Nn \c__unicode_dotless_i_tl   { I }
    \tl_const:Nn \c__unicode_dotted_I_tl    { i }
    \tl_const:Nn \c__unicode_i_ogonek_tl    { }
    \tl_const:Nn \c__unicode_I_ogonek_tl    { }
    \tex_endinput:D
  }
\__unicode_verb_end:
%    \end{macrocode}
%
% \subsection{Upper/lower/title casing}
%
% \begin{macro}{\__unicode_parse_line:w}
% \begin{macro}[aux]{\__unicode_parse_line_auxi:w}
% \begin{macro}[aux]{\__unicode_parse_line_auxii:nw}
% \begin{macro}[aux]{\__unicode_parse_line_auxiii:nw}
% \begin{macro}[aux]{\__unicode_parse_line_auxiv:nnn}
% \begin{macro}[aux]{\__unicode_parse_line_auxv:wnnn}
% \begin{macro}[aux]{\__unicode_parse_line_auxvi:nnNNn}
% \begin{macro}[aux]{\__unicode_brace:n}
%   When parsing |UnicodeData.txt| there are no comments or blank lines. Case
%   mappings when present here are one-to-one and so are easy to deal with. The
%   slight complication here is that the lines are rather long, so a multi-part
%   approach is needed to grab the correct parts of the line as arguments. Of
%   the first set of arguments, the two that needed are |#1| (the code point)
%   and |#6| (details about the code point which may include the fact it's a
%   compatibility char).
%    \begin{macrocode}
\cs_set_protected:Npn \__unicode_parse_line:w
  #1 ; #2 ; #3 ; #4 ; #5 ; #6 ; #7 ; #8 ; #9 ;
  {
    \__unicode_parse_line_auxi:w #1 ; #6 ;
  }
%    \end{macrocode}
%   With some data items removed, at this stage the hexadecimal
%   representation of the char is |#1|, any compatibility char information is
%   in |#2|, the upper case code point is |#6|, the lower case one |#7| and
%   the title case one |#8|. These may or may not be present and the upper and
%   title case values may be identical. The compatibility data is first
%   extracted into a sequence, then the main information is processed. Where
%   there are values for upper/lower case, they are saved into the appropriate
%   token lists. For title case, since the number of exceptions is small and so
%   they are saved in a single dedicated space. Note that there is a space at
%   the end of |#8| as we are reading the data in with spaces not ignored: that
%   has to be allowed for to get the equality test right. The
%   \enquote{business end} of the code here is inside a rescan block so the
%   later parts of the code do not need to be concerned with string
%   \emph{versus} standard category codes.
%    \begin{macrocode}
\cs_set_protected:Npn \__unicode_parse_line_auxi:w
  #1 ; #2 ; #3 ; #4 ; #5 ; #6 ; #7 ; #8 \q_stop
  {
    \use:x
      {
        \__unicode_parse_line_auxii:nw {#1} #2 \tl_to_str:n { <compat> }
          \c_space_tl \exp_not:N \q_stop
      }
    \tl_rescan:nn { }
      {
        \__unicode_parse_line_auxiv:nnn {#1} {#6} { upper }
        \__unicode_parse_line_auxiv:nnn {#1} {#7} { lower }
        \bool_if:nF
          {
               \tl_if_empty_p:n {#6}
            || \str_if_eq_p:nn {#6} {#8}
          }
          {
            \seq_if_in:NnTF \l__unicode_compat_seq {#1}
              { \cs_set:Npn \__unicode_brace:n ##1 { { ##1 } } }
              { \cs_set_eq:NN \__unicode_brace:n \use:n }
            \tl_put_right:Nx \l__unicode_title_exceptions_tl
              {
                \__unicode_brace:n { \luatex_Uchar:D "#1 \c_space_tl }
                \__unicode_brace:n { \luatex_Uchar:D "#8 \c_space_tl }
              }
          }
      }
  }
%    \end{macrocode}
%   Compatibility chars have information as the marker |<compat>| then a list
%   of one to three resulting code points. The one-to-one cases are not an
%   issue for dealing the the data, so it's only the more complex versions that
%   need to be recorded.
%    \begin{macrocode}
\use:x
  {
    \cs_new_protected:Npn \exp_not:N \__unicode_parse_line_auxii:nw
      ##1##2 \tl_to_str:n { <compat> } ~ ##3 \exp_not:N \q_stop
  }
  {
    \tl_if_blank:nF {#3}
      {
        \__unicode_parse_line_auxiii:nw {#1} #3 ~ \q_stop
      }
  }
\cs_new_protected:Npn \__unicode_parse_line_auxiii:nw #1#2 ~ #3 \q_stop
  {
    \tl_if_blank:nF {#3}
      {
        \seq_put_right:Nn \l__unicode_compat_seq {#1}
      }
  }
%    \end{macrocode}
%   The array structure here is in two parts, one for upper and one
%   for lower case mappings.
%    \begin{macrocode}
\cs_new_protected:Npn \__unicode_parse_line_auxiv:nnn #1#2
  {
    \exp_last_unbraced:Nf \__unicode_parse_line_auxv:wnnn
      { \int_eval:n { 1000000 + "#1 } } \q_stop
      {#1} {#2}
  }
\cs_new_protected:Npn \__unicode_parse_line_auxv:wnnn
  #1#2#3#4#5#6#7 \q_stop #8#9
  { \__unicode_parse_line_auxvi:nnNNn {#8} {#9} #6 #7 }
%    \end{macrocode}
%   Store data as long as there is something to actually do in a mapping.
%   For entries in the the compatibility list there is a need to add braces
%   around the chars in case there is any normalisation during file reading.
%    \begin{macrocode}
\cs_new_protected:Npn \__unicode_parse_line_auxvi:nnNNn #1#2#3#4#5
  {
    \tl_if_empty:nF {#2}
      {
        \seq_if_in:NnTF \l__unicode_compat_seq {#1}
          { \cs_set:Npn \__unicode_brace:n ##1 { { ##1 } } }
          { \cs_set_eq:NN \__unicode_brace:n \use:n }
        \tl_put_right:cx { l__unicode_ #5 _ #3 _X_ #4 _tl }
          {
            \__unicode_brace:n
              { \luatex_Uchar:D "#1 \c_space_tl }
            \__unicode_brace:n
              { \luatex_Uchar:D "#2 \c_space_tl }
          }
      }
  }
\cs_new_eq:NN \__unicode_brace:n \use:n
%    \end{macrocode}
% \end{macro}
% \end{macro}
% \end{macro}
% \end{macro}
% \end{macro}
% \end{macro}
% \end{macro}
% \end{macro}
%
% Everything is set up and so the read loop can take place.
%    \begin{macrocode}
\__unicode_file_map:nn
  { UnicodeData.txt }
  { \__unicode_parse_line:w #1 \q_stop }
%    \end{macrocode}
%
% \begin{macro}{\__unicode_parse_line:w}
% \begin{macro}[aux]{\__unicode_parse_line_auxii:w}
% \begin{macro}[aux]{\__unicode_parse_line_auxiii:nnn}
% \begin{macro}[aux]{\__unicode_parse_line_auxiv:nwn}
%   The file |SpecialCasing.txt| uses C-style comments and may contain
%   blank lines: those two awkward situations need to be filtered out before
%   parsing the real data in the line.
%    \begin{macrocode}
\cs_set_protected:Npn \__unicode_parse_line:w #1 \q_stop
  {
    \tl_if_blank:nF {#1}
      {
        \str_if_eq_x:nnF { \tl_head:n {#1} } { \cs_to_str:N \# }
          { \__unicode_parse_line_auxii:w #1 \q_stop }
      }
  }
%    \end{macrocode}
%   Here, |#1| is the code point for the input, |#2| is the lower case mapping,
%   |#3| the title case mapping and |#4| the upper case mapping: all three
%   mappings are always given even if they are also in |UnicodeData.txt|.  As
%   most of the title case exceptions are also upper case exceptions, a test is
%   made so that we are only storing truly useful exceptions for title case.
%    \begin{macrocode}
\cs_new_protected:Npn \__unicode_parse_line_auxii:w
  #1 ;~ #2 ;~ #3 ;~ #4 ; #5 \q_stop
  {
    \__unicode_parse_line_auxiii:nnn {#1} {#2} { lower }
    \str_if_eq:nnF {#3} {#4}
      { \__unicode_parse_line_auxiii:nnn {#1} {#3} { title } }
    \__unicode_parse_line_auxiii:nnn {#1} {#4} { upper }
  }
%    \end{macrocode}
%   For each mapping there may be one, two or three code points in the
%   output. After a bit of a trick to allow for ease of parsing, we check if
%   there are at least two numbers for the case-changed char. If there are,
%   then save the exception. If not, then the value will also be in the main
%   table and we can ignore it here. There is also a test to see if the
%   current value is a title case exception: they don't need extra braces
%   for those.
%    \begin{macrocode}
\cs_new_protected:Npn \__unicode_parse_line_auxiii:nnn #1#2#3
  { \use:n { \__unicode_parse_line_auxiv:nwn {#1} #2 ~ } ~ \q_stop {#3} }
\cs_new_protected:Npn \__unicode_parse_line_auxiv:nwn #1#2 ~ #3 ~ #4 \q_stop #5
  {
    \tl_if_empty:nF {#3}
      {
        \seq_if_in:NnTF \l__unicode_compat_seq {#1}
          { \cs_set:Npn \__unicode_brace:n ##1 { { ##1 } } }
          { \cs_set_eq:NN \__unicode_brace:n \use:n }
        \tl_rescan:nn
          { }
          {
            \tl_put_right:cx { l__unicode_ #5 _exceptions_tl }
              {
                \__unicode_brace:n { \luatex_Uchar:D "#1 \c_space_tl }
                {
                  \luatex_Uchar:D "#2 \c_space_tl
                  \luatex_Uchar:D "#3 \c_space_tl
                  \tl_if_empty:nF {#4}
                    { \luatex_Uchar:D "#4 \c_space_tl }
                }
              }
          }
      }
  }
%    \end{macrocode}
% \end{macro}
% \end{macro}
% \end{macro}
% \end{macro}
%
% Parsing set up, read the special cases file. The input contains both
% general special cases and ones dependent on context. We only want to read
% the former, so there is a check for the line that splits the two:
% at that point, simply stop parsing.
%    \begin{macrocode}
\__unicode_file_map:nn
  { SpecialCasing.txt }
  {
    \str_if_eq_x:nnTF {#1} { \cs_to_str:N \# \c_space_tl Conditional~Mappings }
      { \ior_map_break: }
      { \__unicode_parse_line:w #1 \q_stop }
  }
%    \end{macrocode}
%
% Saving the data uses a single file, with the two arrays followed by the
% exceptions. 
%    \begin{macrocode}
\tl_map_inline:nn { 0123456789 }
  {
    \tl_map_inline:nn { 0123456789 }
      {
        \iow_now:Nx \g__unicode_result_iow
          {
            \tl_const:cn
              { ~ c__unicode_lower_ #1 _X_ ##1 _tl ~ } ~
              { ~ \exp_not:v { l__unicode_lower_ #1 _X_ ##1 _tl } ~ }
          }
        \tl_clear:c { l__unicode_lower_ #1 _X_ ##1 _tl }
      }
  }
\tl_map_inline:nn { 0123456789 }
  {
    \tl_map_inline:nn { 0123456789 }
      {
        \iow_now:Nx \g__unicode_result_iow
          {
            \tl_const:cn
              { ~ c__unicode_upper_ #1 _X_ ##1 _tl ~ } ~
              { ~ \exp_not:v { l__unicode_upper_ #1 _X_ ##1 _tl } ~ }
          }
      }
  }
\clist_map_inline:nn
  { upper , lower , title } 
  {
    \iow_now:Nx \g__unicode_result_iow
      {
        \tl_const:Nn
        \exp_not:c
          {
            c__unicode_
            \str_if_eq:nnTF {#1} { title } { mixed } {#1}
            _exceptions_tl 
          }
         { \exp_not:v { l__unicode_ #1 _ exceptions_tl } }
      }
  }
%    \end{macrocode}
%
% Data for the special cases is now stored. This is mainly a series of simple
% token lists with appropriate names and content, but there is also one place
% where a small mapping list is required.
%    \begin{macrocode}
\cs_new_protected:Npn \__unicode_special_case:nn #1#2
  {
    \quark_if_recursion_tail_stop:n {#1}
    \iow_now:Nx \g__unicode_result_iow
      {
        \tl_const:Nn \exp_not:c { c__unicode_ #1 _tl }
          { ~ \luatex_Uchar:D "#2 \c_space_tl \c_space_tl }
      }
    \__unicode_special_case:nn
  }
\__unicode_special_case:nn
  { std_sigma }     { 03C3 }
  { final_sigma }   { 03C2 }
  { dotless_i }     { 0131 }
  { dot_above }     { 0307 }
  { dotted_I }      { 0130 }
  { i_ogonek }      { 012F }
  { I_ogonek }      { 012E }
  \q_recursion_tail {      }
  \q_recursion_stop
\iow_now:Nx \g__unicode_result_iow
  {
    \tl_const:Nn \exp_not:N \c__unicode_accents_lt_tl
      {
        \luatex_Uchar:D "00CC
          { \luatex_Uchar:D "0069 \luatex_Uchar:D "0307 \luatex_Uchar:D "0300 }
        \luatex_Uchar:D "00CD
          { \luatex_Uchar:D "0069 \luatex_Uchar:D "0307 \luatex_Uchar:D "0301 }
        \luatex_Uchar:D "0128
          { \luatex_Uchar:D "0069 \luatex_Uchar:D "0307 \luatex_Uchar:D "0303 }
      }
  }
%    \end{macrocode}
%
% \subsection{Case folding}
%
% \begin{macro}{\__unicode_parse_line:w}
% \begin{macro}[aux]{\__unicode_parse_line_auxi:Nw}
% \begin{macro}[aux]{\__unicode_parse_line_auxii:w}
% \begin{macro}[aux]{\__unicode_parse_line_auxiii:nw}
% \begin{macro}[aux]{\__unicode_parse_line_auxiv:nn}
% \begin{macro}[aux]{\__unicode_parse_line_auxv:wnn}
%   As for |SpecialCasing.txt|, the format of |CaseFolding.txt| allows both
%   blank lines and C-style comments starting with |#|.
%    \begin{macrocode}
\cs_set_protected:Npn \__unicode_parse_line:w #1 \q_stop
  {
    \tl_if_blank:nF {#1}
      { \__unicode_parse_line_auxi:Nw #1 \q_stop }
  }
\cs_set_protected:Npn \__unicode_parse_line_auxi:Nw #1#2 \q_stop
  {
    \str_if_eq_x:nnF { \exp_not:n {#1} } { \cs_to_str:N \# }
      { \__unicode_parse_line_auxii:w #1#2 \q_stop }
  }
%    \end{macrocode}
%   For lines actually containing data, there will be four entries separated by
%   |;| tokens: the hex code for the char itself, which folding regim\'{e}s
%   the line applies to, the hex code(s) for the folded char and a
%   description. Of these, we need all but the last one. In the simple
%   case of core foldings, the mapping is one--one and this information
%   can be passed directly to the next stage. We also handle the full
%   mappings (dropping simple ones plus any Turkic variation): an additional
%   step is needed to parse this case.
%    \begin{macrocode}
\cs_set_protected:Npn \__unicode_parse_line_auxii:w #1 ;~ #2 ; #3 ; #4 \q_stop
  {
    \str_if_eq:nnTF {#2} { C }
      {
        \__unicode_parse_line_auxiv:nn
          {#1} { \luatex_Uchar:D "#3 \c_space_tl }
      }
      {
        \str_if_eq:nnT {#2} { F }
          { \__unicode_parse_line_auxiii:nw {#1} #3 ~ \q_stop }
      }
  }
%    \end{macrocode}
%   Full folding produces two or three Unicode code points from a single
%   input char. To deal with this, we split the relevant part of the input
%   and check how many chars to generate. The entire folding output is
%   braced so that when read back \TeX{} will see this as a group in our
%   replacement code: the only exceptions occur when the input char is on
%   the compatibility list, as that would lead to an extra set of braces.
%    \begin{macrocode}
\cs_set_protected:Npn \__unicode_parse_line_auxiii:nw #1 ~ #2 ~ #3 ~ #4 \q_stop
  {
    \seq_if_in:NnTF \l__unicode_compat_seq {#1}
      { \cs_set_eq:NN \__unicode_brace:n \use:n }
      { \cs_set:Npn \__unicode_brace:n ##1 { { ##1 } } }
    \exp_args:Nno \__unicode_parse_line_auxiv:nn
      {#1}
      {
        \__unicode_brace:n
          {
            \luatex_Uchar:D "#2 \c_space_tl
            \luatex_Uchar:D "#3 \c_space_tl
            \tl_if_empty:nF {#4}
              { \luatex_Uchar:D "#4 \c_space_tl }
        }
      }
  }
%    \end{macrocode}
%   The final stage of extracting the mapping is to split the various cases
%   up such that comparison and replacement does not need to check every
%   character. That is done by taking the charcode modulo $100$: this splits
%   the list of chars into $100$ much shorter lists. With that done, the
%   input and output chars are added to the appropriate token lists.
%    \begin{macrocode}
\cs_new_protected:Npn \__unicode_parse_line_auxiv:nn #1#2
  {
    \exp_last_unbraced:Nf \__unicode_parse_line_auxv:wnn
      { \int_eval:n { 1000000 + "#1 } } \q_stop
      {#1} {#2}
  }
%    \end{macrocode}
%   As the input is read in string mode, there is a need for a rescan
%   here since \tn{Uchar} requires letters for hexadecimal digits
%   beyond~$9$.
%    \begin{macrocode}
\cs_new_protected:Npn \__unicode_parse_line_auxv:wnn
  #1#2#3#4#5#6#7 \q_stop #8#9
  {
    \seq_if_in:NnTF \l__unicode_compat_seq {#8}
      { \cs_set:Npn \__unicode_brace:n ##1 { { ##1 } } }
      { \cs_set_eq:NN \__unicode_brace:n \use:n }
    \tl_rescan:nn
      { }
      {
        \tl_put_right:cx { l__unicode_lower_ #6 _X_ #7 _tl }
          {
             \__unicode_brace:n { \luatex_Uchar:D "#8 \c_space_tl }
             \__unicode_brace:n { #9 }
          }
      }
  }
%    \end{macrocode}
% \end{macro}
% \end{macro}
% \end{macro}
% \end{macro}
% \end{macro}
% \end{macro}
%
% The main loop can now take place, reading the source data and saving all of
% the information in the token list array.
%    \begin{macrocode}
\__unicode_file_map:nn
  { CaseFolding.txt }
  { \__unicode_parse_line:w #1 \q_stop }
%    \end{macrocode}
%
% Write the processed data to the |.def| file.
%    \begin{macrocode}
\tl_map_inline:nn { 0123456789 }
  {
    \tl_map_inline:nn { 0123456789 }
      {
        \iow_now:Nx \g__unicode_result_iow
          {
            \tl_const:cn
              { ~ c__unicode_fold_ #1 _X_ ##1 _tl ~ } ~
              { ~ \exp_not:v { l__unicode_lower_ #1 _X_ ##1 _tl } ~ }
          }
        \tl_clear:c { l__unicode_lower_ #1 _X_ ##1 _tl }
      }
  }
%    \end{macrocode}
%
% Job done, end the \TeX{} run.
%    \begin{macrocode}
\iow_close:N \g__unicode_result_iow
\tex_end:D
%    \end{macrocode}
%
%    \begin{macrocode}
%</script>
%    \end{macrocode}
%
% \end{implementation}
%
% \PrintIndex