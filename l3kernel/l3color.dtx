% \iffalse meta-comment
% 
%% File: l3color.dtx Copyright(C) 2011 The LaTeX3 Project
%%
%% It may be distributed and/or modified under the conditions of the
%% LaTeX Project Public License (LPPL), either version 1.3c of this
%% license or (at your option) any later version.  The latest version
%% of this license is in the file
%%
%%    http://www.latex-project.org/lppl.txt
%%
%% This file is part of the "l3kernel bundle" (The Work in LPPL)
%% and all files in that bundle must be distributed together.
%%
%% The released version of this bundle is available from CTAN.
%%
%% -----------------------------------------------------------------------
%%
%% The development version of the bundle can be found at
%%
%%    http://www.latex-project.org/svnroot/experimental/trunk/
%%
%% for those people who are interested.
%%
%%%%%%%%%%%
%% NOTE: %%
%%%%%%%%%%%
%%
%%   Snapshots taken from the repository represent work in progress and may
%%   not work or may contain conflicting material!  We therefore ask
%%   people _not_ to put them into distributions, archives, etc. without
%%   prior consultation with the LaTeX Project Team.
%%
%% -----------------------------------------------------------------------
%%
%
%<*driver|package>
\RequirePackage{l3names}
\GetIdInfo$Id$
  {L3 Experimental colour support}
%</driver|package>
%<*driver>
\documentclass[full]{l3doc}
\begin{document}
  \DocInput{\jobname.dtx}
\end{document}
%</driver>
% \fi
% 
% \title{^^A
%   The \textsf{l3color} package\\ Colour support^^A
%   \thanks{This file describes v\ExplFileVersion,
%     last revised \ExplFileDate.}^^A
% }
%         
% \author{^^A
%  The \LaTeX3 Project\thanks
%    {^^A
%      E-mail:
%        \href{mailto:latex-team@latex-project.org}
%          {latex-team@latex-project.org}^^A
%    }^^A
% }
%
% \date{Released \ExplFileDate}
%
% \maketitle
%
% \begin{documentation}
% 
% This module provides support for colour in \LaTeX3{}. At present, the
% material here is mainly intended to support a small number of low-level
% requirements in other \pkg{l3kernel} modules.
% 
% \section{Colour in boxes}
% 
% Controlling the colour of text in boxes requires a small number of control
% functions, so that the boxed material uses the colour at the point where
% it is set, rather than where it is used.
% 
% \begin{function}[added = 2011-09-04]{\color_group_begin:, \color_group_end:}
%   \begin{syntax}
%     \cs{color_group_begin:}
%       \ldots
%     \cs{color_group_end:}
%   \end{syntax}
%   Creates a colour group: one used to \enquote{trap} colour settings.
% \end{function}
% 
% \begin{function}[added = 2011-09-04]{\color_ensure_current:}
%   \begin{syntax}
%     \cs{color_ensure_current:}
%   \end{syntax}
%   Ensures that material inside a box will use the foreground colour
%   at the point where the box is set, rather than that in force when the
%   box is used. This function should usually be used within a
%   \cs{color_group_begin:} \ldots \cs{color_group_end:} group.
% \end{function}
% 
% \end{documentation}
%
% \begin{implementation}
%
% \section{\pkg{l3color} Implementation}
% 
%    \begin{macrocode}
%<*initex|package>
%    \end{macrocode}
%    
%    \begin{macrocode}
%<*package>
\ProvidesExplPackage
  {\ExplFileName}{\ExplFileDate}{\ExplFileVersion}{\ExplFileDescription}
\package_check_loaded_expl:
%</package>
%    \end{macrocode}
%    
% \begin{macro}{\color_group_begin:, \color_group_end:}
%   Grouping for colour is almost the same as using the basic \cs{group_begin:}
%   and \cs{group_end:} functions.  However, in vertical mode the end-of-group
%   needs a \cs{tex_par:D}, which in horizontal mode does nothing.
%    \begin{macrocode}
\cs_new_eq:NN \color_group_begin: \group_begin:
\cs_new_protected_nopar:Npn \color_group_end:
  {
      \tex_par:D
    \group_end:
  }
%    \end{macrocode}
% \end{macro}
% 
% \begin{macro}{\color_ensure_current:}
%   A driver-independent wrapper for setting the foreground colour to the
%   current colour \enquote{now}.
%    \begin{macrocode}
%<*initex>
\cs_new_protected_nopar:Npn \color_ensure_current:
  { \driver_color_ensure_current: }
%</initex>
%<*package>
\cs_new_protected_nopar:Npn \color_ensure_current: { \set@color }
%</package>
%    \end{macrocode}
% \end{macro}
%
%    \begin{macrocode}
%</initex|package>
%    \end{macrocode}
%
% \end{implementation}
%
% \PrintIndex