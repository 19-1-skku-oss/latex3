% \iffalse meta-comment
%
%% File: l3fp-trig.dtx Copyright (C) 2011-2012 The LaTeX3 Project
%%
%% It may be distributed and/or modified under the conditions of the
%% LaTeX Project Public License (LPPL), either version 1.3c of this
%% license or (at your option) any later version.  The latest version
%% of this license is in the file
%%
%%    http://www.latex-project.org/lppl.txt
%%
%% This file is part of the "l3kernel bundle" (The Work in LPPL)
%% and all files in that bundle must be distributed together.
%%
%% The released version of this bundle is available from CTAN.
%%
%% -----------------------------------------------------------------------
%%
%% The development version of the bundle can be found at
%%
%%    http://www.latex-project.org/svnroot/experimental/trunk/
%%
%% for those people who are interested.
%%
%%%%%%%%%%%
%% NOTE: %%
%%%%%%%%%%%
%%
%%   Snapshots taken from the repository represent work in progress and may
%%   not work or may contain conflicting material!  We therefore ask
%%   people _not_ to put them into distributions, archives, etc. without
%%   prior consultation with the LaTeX Project Team.
%%
%% -----------------------------------------------------------------------
%%
%
%<*driver>
\RequirePackage{l3bootstrap}
\GetIdInfo$Id: l3fp-trig.dtx 3514 2012-03-08 06:14:48Z bruno $
  {L3 Floating-point trigonometric functions}
\documentclass[full]{l3doc}
\begin{document}
  \DocInput{\jobname.dtx}
\end{document}
%</driver>
% \fi
%
% \title{The \textsf{l3fp-trig} package\thanks{This file
%         has version number \ExplFileVersion, last
%         revised \ExplFileDate.}\\
% Floating point trigonometric functions}
% \author{^^A
%  The \LaTeX3 Project\thanks
%    {^^A
%      E-mail:
%        \href{mailto:latex-team@latex-project.org}
%          {latex-team@latex-project.org}^^A
%    }^^A
% }
% \date{Released \ExplFileDate}
%
% \maketitle
%
% \begin{documentation}
%
% \end{documentation}
%
% \begin{implementation}
%
% \section{Implementation}
%
%    \begin{macrocode}
%<*initex|package>
%    \end{macrocode}
%
%    \begin{macrocode}
%<@@=fp>
%    \end{macrocode}
%
%^^A todo: check EXP/rEXP everywhere.
%
% \subsection{Inverting a floating point number}
%
% \begin{macro}[int, EXP]{\@@_one_over:w}
%   Expects a floating point of the form \cs{s_@@} \ldots{} |;| and
%   computes its multiplicative inverse.  This is used to compute the
%   cotangent function very near $0$.
%    \begin{macrocode}
\cs_new_nopar:Npx \@@_one_over:w
  {
    \exp_not:N \exp_after:wN
    \exp_not:c { @@_/_o:ww }
    \exp_not:N \c_one_fp
  }
%    \end{macrocode}
% \end{macro}
%
% \subsection{Direct trigonometric functions}
%
% The approach for all trigonometric functions (sine, cosine, tangent,
% and cotangent) is the same.
% \begin{itemize}
% \item Filter out special cases ($\pm 0$, $\pm\inf$ and \nan{}).
% \item Keep the sign for later, and work with the absolute value $|x|$
%   of the argument.
% \item For numbers less than $1$, shift the mantissa to convert them to
%   fixed point numbers.  Very small numbers take a slightly different
%   route.
% \item For numbers $\geq 1$, subtract a multiple of $\pi/2$ to bring
%   them to the range to $[0, \pi/2]$.
% \item Reduce further to $[0, \pi/4]$ using $\sin x = \cos (\pi/2-x)$.
% \item Use the appropriate power series depending on the octant
%   $\lfloor\frac{|x|}{\pi/4}\rfloor \mod 8$, the sign, and the function
%   to compute.
% \end{itemize}
%
% \subsubsection{Sign and special numbers}
%
% \begin{macro}[int, EXP]{\@@_sin:w}
%   The sine of $\pm 0$ or \nan{} is the same floating point number.
%   The sine of $\pm\infty$ raises an invalid operation exception.
%   Otherwise, check the exponent, preparing to use
%   \cs{@@_sin_series:NNwww} for the calculation, with a sign |#2|, and
%   an initial octant of $0$.  The question mark is an argument which is
%   not used in this case.
%    \begin{macrocode}
\cs_new:Npn \@@_sin:w \s_@@ \@@_chk:w #1#2
  {
    \if_case:w #1 \exp_stop_f:
           \@@_case_return_same_o:w
    \or:
      \exp_after:wN \@@_trig_exponent:NNNNwn
      \exp_after:wN \@@_sin_series:NNwww
      \exp_after:wN ?
      \exp_after:wN #2
      \exp_after:wN \c_zero
    \or:
      \@@_case_use:nw { \@@_invalid_operation_o:nw { sin } }
    \else: \@@_case_return_same_o:w
    \fi:
    \s_@@ \@@_chk:w #1#2
  }
%    \end{macrocode}
% \end{macro}
%
% \begin{macro}[int, EXP]{\@@_cos:w}
%   The cosine of $\pm 0$ is $1$.  The cosine of $\pm\infty$ raises an
%   invalid operation exception.  The cosine of \nan{} is itself.
%   Otherwise, check the exponent, preparing to use
%   \cs{@@_sin_series:NNwww} for the calculation, with a positive sign
%   ($0$), and an initial octant of $2$, because $\cos x = \sin ( \pi/2
%   + |x|)$.  The question mark is an argument which is not used in this
%   case.
%    \begin{macrocode}
\cs_new:Npn \@@_cos:w \s_@@ \@@_chk:w #1#2
  {
    \if_case:w #1 \exp_stop_f:
           \@@_case_return_o:Nw \c_one_fp
    \or:
      \@@_case_use:nw %^^A todo: is that faster than the exp_after route?
        {
          \@@_trig_exponent:NNNNwn
          \@@_sin_series:NNwww
          ?
          0
          \c_two
        }
    \or:
      \@@_case_use:nw { \@@_invalid_operation_o:nw { cos } }
    \else: \@@_case_return_same_o:w
    \fi:
    \s_@@ \@@_chk:w #1#2
  }
%    \end{macrocode}
% \end{macro}
%
% \begin{macro}[int, EXP]{\@@_tan:w}
%   The tangent of $\pm 0$ or \nan{} is the same floating point number.
%   The tangent of $\pm\infty$ raises an invalid operation exception.
%   Otherwise, check the exponent, preparing to use
%   \cs{@@_tan_series:NNwww} for the calculation, with a positive sign
%   ($0$), and an initial octant of $1$, chosen to be distinct from the
%   octants for sine and cosine.  See \cs{@@_cot:w} for an
%   explanation of the $0$ argument.
%    \begin{macrocode}
\cs_new:Npn \@@_tan:w \s_@@ \@@_chk:w #1#2
  {
    \if_case:w #1 \exp_stop_f:
           \@@_case_return_same_o:w
    \or:
      \exp_after:wN \@@_trig_exponent:NNNNwn
      \exp_after:wN \@@_tan_series:NNwww
      \exp_after:wN 0
      \exp_after:wN #2
      \exp_after:wN \c_one
    \or:
      \@@_case_use:nw { \@@_invalid_operation_o:nw { tan } }
    \else: \@@_case_return_same_o:w
    \fi:
    \s_@@ \@@_chk:w #1#2
  }
%    \end{macrocode}
% \end{macro}
%
% \begin{macro}[int, EXP]{\@@_cot:w}
%   The cotangent of $\pm 0$ is $\pm \infty$ with the same sign,
%   produced by \cs{@@_one_over:w}.  The cotangent of $\pm\infty$ raises
%   an invalid operation exception.  The cotangent of \nan{} is itself.
%   We use $\cot x = - \tan (\pi/2 + x)$, and the initial octant for the
%   tangent was chosen to be $1$, so the octant here starts at $3$.  The
%   change in sign is obtained by feeding \cs{@@_tan_series:NNwww} two
%   signs rather than just the sign of the argument: the first of those
%   indicates whether we compute tangent or cotangent.  Those signs are
%   eventually combined.
%    \begin{macrocode}
\cs_new:Npn \@@_cot:w \s_@@ \@@_chk:w #1#2
  {
    \if_case:w #1 \exp_stop_f:
           \exp_after:wN \@@_one_over:w
    \or:
      \exp_after:wN \@@_trig_exponent:NNNNwn
      \exp_after:wN \@@_tan_series:NNwww
      \exp_after:wN 2
      \exp_after:wN #2
      \exp_after:wN \c_three
    \or:
      \@@_case_use:nw { \@@_invalid_operation_o:nw { cot } }
    \else: \@@_case_return_same_o:w
    \fi:
    \s_@@ \@@_chk:w #1#2
  }
%    \end{macrocode}
% \end{macro}
%
% \subsubsection{Small and tiny arguments}
%
% \begin{macro}[aux, EXP]{\@@_trig_exponent:NNNNwn}
%   The first four arguments control what trigonometric function we
%   compute, then follows a normal floating point number.  If the
%   floating point is smaller than $10^{-8}$, then call the appropriate
%   \texttt{_epsilon} auxiliary.  Otherwise, call the function |#1|,
%   with arguments |#2|, |#3|, the octant, computed in an integer
%   expression starting with |#4|, and a fixed point number obtained
%   from the floating point number by argument reduction.  Numbers less
%   than $1$ are converted using \cs{@@_trig_small:w} which simply
%   shifts the mantissa, while large numbers need argument reduction.
%    \begin{macrocode}
\cs_new:Npn \@@_trig_exponent:NNNNwn #1#2#3#4 \s_@@ \@@_chk:w 1#5#6
  {
    \if_int_compare:w #6 > - \c_eight
      \exp_after:wN #1
      \exp_after:wN #2
      \exp_after:wN #3
      \int_use:N \__int_eval:w #4
        \if_int_compare:w #6 > \c_zero
          \exp_after:wN \@@_trig_large:w \__int_value:w
        \else:
          \exp_after:wN \@@_trig_small:w \__int_value:w
        \fi:
    \else:
      \if_case:w #4
             \@@_sin_epsilon:w
      \or:   \@@_sin_epsilon:w
      \or:   \@@_cos_epsilon:w
      \else: \@@_cot_epsilon:w
      \fi:
      #5
    \fi:
    #6 ;
  }
%    \end{macrocode}
% \end{macro}
%
% \begin{macro}[aux, EXP]
%   {\@@_sin_epsilon:w, \@@_cos_epsilon:w, \@@_cot_epsilon:w}
%   Sine and tangent of tiny numbers give the number itself: the
%   relative error is less than $5 \cdot 10^{-17}$, which is
%   appropriate.  Cosine simply gives $1$.  Cotangent computes the
%   inverse.  This is actually slightly wrong because further terms in
%   the power series could affect the rounding for cotangent.
%    \begin{macrocode}
\cs_new:Npn \@@_sin_epsilon:w #1 \fi: #2 \fi: #3 ;
  { \fi: \fi: \@@_exp_after_o:w \s_@@ \@@_chk:w 1 #2 {#3} }
\cs_new:Npn \@@_cos_epsilon:w #1 \fi: #2 \fi: #3 ; #4 ;
  { \fi: \fi: \exp_after:wN \c_one_fp }
\cs_new:Npn \@@_cot_epsilon:w \fi: #1 \fi: #2 ;
  { \fi: \fi: \@@_one_over:w \s_@@ \@@_chk:w 1 #1 {#2} }
%    \end{macrocode}
% \end{macro}
%
% \begin{macro}[aux, EXP]{\@@_trig_small:w, \@@_trig_small_aux:wwNN}
%   Floating point numbers less than $1$ are converted to fixed point
%   numbers by shifting the mantissa.  Since we have already filtered
%   out numbers less than $10^{-8}$, no digit is lost in converting to
%   a fixed point number.
%    \begin{macrocode}
\cs_new:Npn \@@_trig_small:w #1;
  {
    \exp_after:wN \exp_after:wN \exp_after:wN \@@_trig_small_aux:wwNN
    \prg_replicate:nn { - #1 } { 0 } ;
  }
\cs_new:Npn \@@_trig_small_aux:wwNN #1; #2#3#4#5;
  {
    \@@_pack_twice_four:wNNNNNNNN
    \@@_pack_twice_four:wNNNNNNNN
    \@@_pack_twice_four:wNNNNNNNN
    .
    ;
    #1#2#3#4#5 0000 0000;
  }
%    \end{macrocode}
% \end{macro}
%
% \subsubsection{Reduction of large arguments}
%
% In the case of a floating point argument greater or equal to $1$, we
% need to perform argument reduction.
%
% \begin{macro}[aux, rEXP]
%   {
%     \@@_trig_large:w, \@@_trig_large_i:www,
%     \@@_trig_large_ii:wnnnnnn, \@@_trig_large_break:w
%   }
%   We shift the mantissa by one digit at a time, subtracting a multiple
%   of $2\pi$ at each step.  We use a value of $2\pi$ rounded up,
%   consistent with the choice of \cs{c_pi_fp}.  This is not quite
%   correct from an accuracy perspective, but has the nice property that
%   $\sin(180\mathrm{deg}) = 0$ exactly.  The arguments of
%   \cs{@@_trig_large_i:www} are a leading block of up to $5$ digits,
%   three brace groups of $4$ digits each, and the exponent, decremented
%   at each step.  The multiple of $2\pi$ to subtract is estimated as
%   $\lfloor |#1| / 6283\rfloor$ (the formula chosen always gives a
%   non-negative integer).  The subtraction has a form similar to our
%   usual multiplications (see \pkg{l3fp-basics} or
%   \pkg{l3fp-extended}).  Once the exponent reaches $0$, we are done
%   subtracting $2\pi$, and we call \cs{@@_trig_octant_loop:nw} to do
%   the reduction by $\pi/2$.
%    \begin{macrocode}
\cs_new:Npn \@@_trig_large:w #1; #2#3;
  { \@@_trig_large_i:www #2; #3 ;  #1; }
\cs_new:Npn \@@_trig_large_i:www #1; #2; #3;
  {
    \if_meaning:w 0 #3 \@@_trig_large_break:w \fi:
    \exp_after:wN \@@_trig_large_ii:wnnnnnn
    \int_use:N \__int_eval:w ( #1 - 3141 ) / 6283 ;
    {#1} #2;
    \int_use:N \__int_eval:w \c_minus_one + #3;
  }
\cs_new:Npn \@@_trig_large_ii:wnnnnnn #1; #2#3#4#5;
  {
    \exp_after:wN \@@_trig_large_i:www
    \int_use:N \__int_eval:w -5 0000 + #20 - #1*62831
      \exp_after:wN \@@_fixed_mul_pack:NNNNNw
      \int_use:N \__int_eval:w 4 9995 0000 + #30 - #1*8530
        \exp_after:wN \@@_fixed_mul_pack:NNNNNw
        \int_use:N \__int_eval:w 4 9995 0000 + #40 - #1*7179
          \exp_after:wN \@@_fixed_mul_pack:NNNNNw
          \int_use:N \__int_eval:w 5 0000 0000 + #50 - #1*5880
    \exp_after:wN ;
    \exp_after:wN ;
  }
\cs_new:Npn \@@_trig_large_break:w \fi: #1; #2;
  { \fi: \@@_trig_octant_loop:nw #2 {0000} {0000} ; }
%    \end{macrocode}
% \end{macro}
%
%^^A todo: optimize: we don't need 6x4 digits here, only 4x4.
%
% \begin{macro}[aux, rEXP]
%   {
%     \@@_trig_octant_loop:nw, \@@_trig_octant_break:w,
%     \@@_trig_octant_neg:w
%   }
%   We receive a fixed point number as argument.  As long as it is
%   greater than $1.5707$ (a slight underestimate of $\pi/2$), subtract
%   $\pi/2$, and leave |+ \c_two| in the integer expression for the
%   octant.  Once it becomes smaller, if it is greater than $0.7854$
%   (overestimate of $\pi/4$), then compute $\pi/2 - x$ and increment
%   the octant.  If it is negative, correct this by changing the sign
%   and decrementing the octant (by adding $7$).  The result is in all
%   cases in the range $[0, 0.7854]$, appropriate for a series
%   expansion.
%    \begin{macrocode}
\cs_new:Npn \@@_trig_octant_loop:nw #1#2;
  {
    \if_int_compare:w #1 < 15707 \exp_stop_f:
      \@@_trig_octant_break:w
    \fi:
    + \c_two
    \@@_fixed_sub_back:wwN
      {15707} {9632} {6794} {8970} {0000} {0000} ;
      {#1} #2;
    \@@_trig_octant_loop:nw
  }
\cs_new:Npn \@@_trig_octant_break:w #1 \fi: + #2#3 #4; #5#6; #7;
  {
    \fi:
    \if_int_compare:w #5 < 7854 \exp_stop_f:
      \if_int_compare:w #5 < \c_zero
        \exp_after:wN \@@_trig_octant_neg:w
      \fi:
      \exp_after:wN \@@_use_i_until_s:nw
      \exp_after:wN .
    \fi:
    + \c_one
    \@@_fixed_sub:wwN
    {15707} {9632} {6794} {8970} {0000} {0000} ;
    {#5} #6 ; . ;
  }
\cs_new:Npn \@@_trig_octant_neg:w #1\fi: #2; #3#4#5#6#7#8; #9
  {
    \fi:
    + \c_seven
    \exp_after:wN \@@_fixed_add_after:NNNNNwN
    \int_use:N \__int_eval:w 1 9999 9998 - #30000 - #4
      \exp_after:wN \@@_fixed_add_pack:NNNNNwN
      \int_use:N \__int_eval:w 1 9999 9998 - #5#6
        \exp_after:wN \@@_fixed_add_pack:NNNNNwN
        \int_use:N \__int_eval:w 2 0000 0000 - #7#8 ; {#9} ;
  }
%    \end{macrocode}
% \end{macro}
%
% \subsection{Computing the power series}
%
% \begin{macro}[aux, EXP]{\@@_sin_series:NNwww, \@@_sin_series_aux:Nnww}
%   Here we receive an unused |?|, a \meta{sign} ($0$ or $2$), a
%   (non-negative) \meta{octant} delimited by a dot, a \meta{fixed
%     point} number, and junk delimited by a semicolon.  The auxiliary
%   receives:
%   \begin{itemize}
%   \item The final sign, which depends on the octant |#3| and the
%     original sign |#2|,
%   \item The octant |#3|, which will control the series we use.
%   \item The square |#4 * #4| of the argument, computed with
%     \cs{@@_fixed_mul:wwn}.
%   \item The number itself.
%   \end{itemize}
%   If the octant is in $\{1,2,5,6,\ldots{}\}$, we are near an extremum
%   of the function and we use the series
%   \[
%   \cos(x) = 1 - x^2 \bigg( \frac{1}{2!} - x^2 \bigg( \frac{1}{4!}
%   - x^2 \bigg( \cdots \bigg) \bigg) \bigg) .
%   \]
%   Otherwise, the series
%   \[
%   \sin(x) = x \bigg( 1 - x^2 \bigg( \frac{1}{3!} - x^2 \bigg(
%   \frac{1}{5!} - x^2 \bigg( \cdots \bigg) \bigg) \bigg) \bigg)
%   \]
%   is used.  Finally, the fixed point number is converted to a floating
%   point number with the given sign, and we check for overflow or
%   underflow. %^^A todo: can over/underflow really happen??
%    \begin{macrocode}
\cs_new:Npn \@@_sin_series:NNwww #1#2#3 . #4; #5;
  {
    \@@_fixed_mul:wwn #4; #4;
    {
      \exp_after:wN \@@_sin_series_aux:Nnww
      \__int_value:w
        \if_int_odd:w \__int_eval:w ( #3 + \c_two ) / \c_four \__int_eval_end:
          #2
        \else:
          \if_meaning:w #2 0 2 \else: 0 \fi:
        \fi:
      {#3}
    }
    #4 ;
  }
\cs_new:Npn \@@_sin_series_aux:Nnww #1#2 #3; #4;
  {
    \if_int_odd:w \__int_eval:w #2 / \c_two \__int_eval_end:
      \exp_after:wN \use_i:nn
    \else:
      \exp_after:wN \use_ii:nn
    \fi:
    {
      \@@_fixed_continue:wn {0000}{0000}{0000}{0001}{5619}{2070}; % 1/18!
      \@@_fixed_mul_sub_back:wwwn #3; {0000}{0000}{0000}{0477}{9477}{3324};
      \@@_fixed_mul_sub_back:wwwn #3; {0000}{0000}{0011}{4707}{4559}{7730};
      \@@_fixed_mul_sub_back:wwwn #3; {0000}{0000}{2087}{6756}{9878}{6810};
      \@@_fixed_mul_sub_back:wwwn #3; {0000}{0027}{5573}{1922}{3985}{8907};
      \@@_fixed_mul_sub_back:wwwn #3; {0000}{2480}{1587}{3015}{8730}{1587};
      \@@_fixed_mul_sub_back:wwwn #3; {0013}{8888}{8888}{8888}{8888}{8889};
      \@@_fixed_mul_sub_back:wwwn #3; {0416}{6666}{6666}{6666}{6666}{6667};
      \@@_fixed_mul_sub_back:wwwn #3; {5000}{0000}{0000}{0000}{0000}{0000};
      \@@_fixed_mul_sub_back:wwwn #3;{10000}{0000}{0000}{0000}{0000}{0000};
    }
    {
      \@@_fixed_continue:wn {0000}{0000}{0000}{0028}{1145}{7254}; % 1/17!
      \@@_fixed_mul_sub_back:wwwn #3; {0000}{0000}{0000}{7647}{1637}{3182};
      \@@_fixed_mul_sub_back:wwwn #3; {0000}{0000}{0160}{5904}{3836}{8216};
      \@@_fixed_mul_sub_back:wwwn #3; {0000}{0002}{5052}{1083}{8544}{1719};
      \@@_fixed_mul_sub_back:wwwn #3; {0000}{0275}{5731}{9223}{9858}{9065};
      \@@_fixed_mul_sub_back:wwwn #3; {0001}{9841}{2698}{4126}{9841}{2698};
      \@@_fixed_mul_sub_back:wwwn #3; {0083}{3333}{3333}{3333}{3333}{3333};
      \@@_fixed_mul_sub_back:wwwn #3; {1666}{6666}{6666}{6666}{6666}{6667};
      \@@_fixed_mul_sub_back:wwwn #3;{10000}{0000}{0000}{0000}{0000}{0000};
      \@@_fixed_mul:wwn #4;
    }
    {
      \exp_after:wN \@@_sanitize:Nw
      \exp_after:wN #1
      \int_use:N \__int_eval:w \@@_fixed_to_float:wN
    }
    #1
  }
%    \end{macrocode}
% \end{macro}
%
% \begin{macro}[aux, EXP]{\@@_tan_series:NNwww, \@@_tan_series_aux:Nnww}
%   Similar to \cs{@@_sin_series:NNwww}, but with slightly different
%   rules to find the sign.  The result is expressed as a ratio of
%   polynomials, of the form
%   \[
%   \tan(x) \simeq
%   \frac{x (1 - x^2 (a_1 - x^2 (a_2 - x^2 (a_3 - x^2 (a_4 - x^2 a_5)))))}
%     {1 - x^2 (b_1 - x^2 (b_2 - x^2 (b_3 - x^2 (b_4 - x^2 b_5))))} .
%   \]
%   The ratio of the two fixed point numbers is converted to a floating
%   point number directly to avoid rounding issues.  The two fixed
%   points may be exchanged before computing the ratio, depending on the
%   quadrant.
%    \begin{macrocode}
\cs_new:Npn \@@_tan_series:NNwww #1#2#3. #4; #5;
  {
    \@@_fixed_mul:wwn #4; #4;
    {
      \exp_after:wN \@@_tan_series_aux:Nnww
      \__int_value:w
        \if_int_odd:w \__int_eval:w #3 / \c_two \__int_eval_end:
          \exp_after:wN \reverse_if:N
        \fi:
        \if_meaning:w #1#2 2 \else: 0 \fi:
      {#3}
    }
    #4 ;
  }
\cs_new:Npn \@@_tan_series_aux:Nnww #1 #2 #3; #4;
  {
    \@@_fixed_continue:wn {0000}{0000}{1527}{3493}{0856}{7059};
    \@@_fixed_mul_sub_back:wwwn #3; {0000}{0159}{6080}{0274}{5257}{6472};
    \@@_fixed_mul_sub_back:wwwn #3; {0002}{4571}{2320}{0157}{2558}{8481};
    \@@_fixed_mul_sub_back:wwwn #3; {0115}{5830}{7533}{5397}{3168}{2147};
    \@@_fixed_mul_sub_back:wwwn #3; {1929}{8245}{6140}{3508}{7719}{2982};
    \@@_fixed_mul_sub_back:wwwn #3;{10000}{0000}{0000}{0000}{0000}{0000};
    \@@_fixed_mul:wwn #4;
      {
        \@@_fixed_continue:wn {0000}{0007}{0258}{0681}{9408}{4706};
        \@@_fixed_mul_sub_back:wwwn #3; {0000}{2343}{7175}{1399}{6151}{7670};
        \@@_fixed_mul_sub_back:wwwn #3; {0019}{2638}{4588}{9232}{8861}{3691};
        \@@_fixed_mul_sub_back:wwwn #3; {0536}{6357}{0691}{4344}{6852}{4252};
        \@@_fixed_mul_sub_back:wwwn #3; {5263}{1578}{9473}{6842}{1052}{6315};
        \@@_fixed_mul_sub_back:wwwn #3;{10000}{0000}{0000}{0000}{0000}{0000};
          {
            \exp_after:wN \@@_sanitize:Nw
            \exp_after:wN #1
            \int_use:N \__int_eval:w
              \reverse_if:N \if_int_odd:w
                  \__int_eval:w (#2 - \c_one) / \c_two \__int_eval_end:
                \exp_after:wN \@@_reverse_args:Nww
              \fi:
              \@@_fixed_div_to_float:ww
          }
      }
  }
%    \end{macrocode}
% \end{macro}
%
%    \begin{macrocode}
%</initex|package>
%    \end{macrocode}
%
% \end{implementation}
%
% \PrintChanges
%
% \PrintIndex 
