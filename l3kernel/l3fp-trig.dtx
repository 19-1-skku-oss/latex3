% \iffalse meta-comment
%
%% File: l3fp-trig.dtx Copyright (C) 2011-2014 The LaTeX3 Project
%%
%% It may be distributed and/or modified under the conditions of the
%% LaTeX Project Public License (LPPL), either version 1.3c of this
%% license or (at your option) any later version.  The latest version
%% of this license is in the file
%%
%%    http://www.latex-project.org/lppl.txt
%%
%% This file is part of the "l3kernel bundle" (The Work in LPPL)
%% and all files in that bundle must be distributed together.
%%
%% The released version of this bundle is available from CTAN.
%%
%% -----------------------------------------------------------------------
%%
%% The development version of the bundle can be found at
%%
%%    http://www.latex-project.org/svnroot/experimental/trunk/
%%
%% for those people who are interested.
%%
%%%%%%%%%%%
%% NOTE: %%
%%%%%%%%%%%
%%
%%   Snapshots taken from the repository represent work in progress and may
%%   not work or may contain conflicting material!  We therefore ask
%%   people _not_ to put them into distributions, archives, etc. without
%%   prior consultation with the LaTeX Project Team.
%%
%% -----------------------------------------------------------------------
%%
%
%<*driver>
\documentclass[full]{l3doc}
\GetIdInfo$Id$
  {L3 Floating-point trigonometric functions}
\begin{document}
  \DocInput{\jobname.dtx}
\end{document}
%</driver>
% \fi
%
% \title{The \textsf{l3fp-trig} package\thanks{This file
%         has version number \ExplFileVersion, last
%         revised \ExplFileDate.}\\
% Floating point trigonometric functions}
% \author{^^A
%  The \LaTeX3 Project\thanks
%    {^^A
%      E-mail:
%        \href{mailto:latex-team@latex-project.org}
%          {latex-team@latex-project.org}^^A
%    }^^A
% }
% \date{Released \ExplFileDate}
%
% \maketitle
%
% \begin{documentation}
%
% \end{documentation}
%
% \begin{implementation}
%
% \section{\pkg{l3fp-trig} Implementation}
%
%    \begin{macrocode}
%<*initex|package>
%    \end{macrocode}
%
%    \begin{macrocode}
%<@@=fp>
%    \end{macrocode}
%
%^^A todo: check EXP/rEXP everywhere.
%
% \subsection{Direct trigonometric functions}
%
% The approach for all trigonometric functions (sine, cosine, tangent,
% cotangent, cosecant, and secant), with arguments given in radians or
% in degrees, is the same.
% \begin{itemize}
%   \item Filter out special cases ($\pm 0$, $\pm\inf$ and \nan{}).
%   \item Keep the sign for later, and work with the absolute value
%     $\lvert x\rvert$ of the argument.
%   \item Small numbers ($\lvert x\rvert<1$ in radians, $\lvert
%     x\rvert<10$ in degrees) are converted to fixed point numbers (and
%     to radians if $\lvert x\rvert$ is in degrees).
%   \item For larger numbers, we need argument reduction.  Subtract a
%     multiple of $\pi/2$ (in degrees,~$90$) to bring the number to the
%     range to $[0, \pi/2)$ (in degrees, $[0,90)$).
%   \item Reduce further to $[0, \pi/4]$ (in degrees, $[0,45]$) using
%     $\sin x = \cos (\pi/2-x)$, and when working in degrees, convert to
%     radians.
%   \item Use the appropriate power series depending on the octant
%     $\lfloor\frac{|x|}{\pi/4}\rfloor \mod 8$ (in degrees, the same
%     formula with $\pi/4\to 45$), the sign, and the function to
%     compute.
% \end{itemize}
%
% \subsubsection{Filtering special cases}
%
% \begin{macro}[int, EXP]{\@@_sin_o:w}
%   This function, and its analogs for \texttt{cos}, \texttt{csc},
%   \texttt{sec}, \texttt{tan}, and \texttt{cot} instead of
%   \texttt{sin}, are followed either by \cs{use_i:nn} and a float in
%   radians or by \cs{use_ii:nn} and a float in degrees.  The sine of
%   $\pm 0$ or \nan{} is the same float.  The sine of $\pm\infty$ raises
%   an invalid operation exception with the appropriate function name.
%   Otherwise, call the \texttt{trig} function to perform argument
%   reduction and if necessary convert the reduced argument to radians.
%   Then, \cs{@@_sin_series_o:NNwwww} will be called to compute the
%   Taylor series: this function receives a sign~|#3|, an initial octant
%   of~$0$, and the function \cs{@@_ep_to_float:wwN} which converts the
%   result of the series to a floating point directly rather than taking
%   its inverse, since $\sin(x) = \#3 \sin\lvert x\rvert$.
%    \begin{macrocode}
\cs_new:Npn \@@_sin_o:w #1 \s_@@ \@@_chk:w #2#3#4; @
  {
    \if_case:w #2 \exp_stop_f:
           \@@_case_return_same_o:w
    \or:   \@@_case_use:nw
             {
               \@@_trig:NNNNNwn #1 \@@_sin_series_o:NNwwww
               \@@_ep_to_float:wwN #3 \c_zero
             }
    \or:   \@@_case_use:nw
             { \@@_invalid_operation_o:fw { #1 { sin } { sind } } }
    \else: \@@_case_return_same_o:w
    \fi:
    \s_@@ \@@_chk:w #2 #3 #4;
  }
%    \end{macrocode}
% \end{macro}
%
% \begin{macro}[int, EXP]{\@@_cos_o:w}
%   The cosine of $\pm 0$ is $1$.  The cosine of $\pm\infty$ raises an
%   invalid operation exception.  The cosine of \nan{} is itself.
%   Otherwise, the \texttt{trig} function reduces the argument to at
%   most half a right-angle and converts if necessary to radians.  We
%   will then call the same series as for sine, but using a positive
%   sign~|0| regardless of the sign of~$x$, and with an initial octant
%   of~$2$, because $\cos(x) = + \sin(\pi/2 + \lvert x\rvert)$.
%    \begin{macrocode}
\cs_new:Npn \@@_cos_o:w #1 \s_@@ \@@_chk:w #2#3; @
  {
    \if_case:w #2 \exp_stop_f:
           \@@_case_return_o:Nw \c_one_fp
    \or:   \@@_case_use:nw
             {
               \@@_trig:NNNNNwn #1 \@@_sin_series_o:NNwwww
               \@@_ep_to_float:wwN 0 \c_two
             }
    \or:   \@@_case_use:nw
             { \@@_invalid_operation_o:fw { #1 { cos } { cosd } } }
    \else: \@@_case_return_same_o:w
    \fi:
    \s_@@ \@@_chk:w #2 #3;
  }
%    \end{macrocode}
% \end{macro}
%
% \begin{macro}[int, EXP]{\@@_csc_o:w}
%   The cosecant of $\pm 0$ is $\pm \infty$ with the same sign, with a
%   division by zero exception (see \cs{@@_cot_zero_o:Nfw} defined
%   below), which requires the function name.  The cosecant of
%   $\pm\infty$ raises an invalid operation exception.  The cosecant of
%   \nan{} is itself.  Otherwise, the \texttt{trig} function performs
%   the argument reduction, and converts if necessary to radians before
%   calling the same series as for sine, using the sign~|#3|, a starting
%   octant of~$0$, and inverting during the conversion from the fixed
%   point sine to the floating point result, because $\csc(x) = \#3
%   \big( \sin\lvert x\rvert\big)^{-1}$.
%    \begin{macrocode}
\cs_new:Npn \@@_csc_o:w #1 \s_@@ \@@_chk:w #2#3#4; @
  {
    \if_case:w #2 \exp_stop_f:
           \@@_cot_zero_o:Nfw #3 { #1 { csc } { cscd } }
    \or:   \@@_case_use:nw
             {
               \@@_trig:NNNNNwn #1 \@@_sin_series_o:NNwwww
               \@@_ep_inv_to_float:wwN #3 \c_zero
             }
    \or:   \@@_case_use:nw
             { \@@_invalid_operation_o:fw { #1 { csc } { cscd } } }
    \else: \@@_case_return_same_o:w
    \fi:
    \s_@@ \@@_chk:w #2 #3 #4;
  }
%    \end{macrocode}
% \end{macro}
%
% \begin{macro}[int, EXP]{\@@_sec_o:w}
%   The secant of $\pm 0$ is $1$.  The secant of $\pm \infty$ raises an
%   invalid operation exception.  The secant of \nan{} is itself.
%   Otherwise, the \texttt{trig} function reduces the argument and turns
%   it to radians before calling the same series as for sine, using a
%   positive sign~$0$, a starting octant of~$2$, and inverting upon
%   conversion, because $\sec(x) = + 1 / \sin(\pi/2 + \lvert x\rvert)$.
%    \begin{macrocode}
\cs_new:Npn \@@_sec_o:w #1 \s_@@ \@@_chk:w #2#3; @
  {
    \if_case:w #2 \exp_stop_f:
           \@@_case_return_o:Nw \c_one_fp
    \or:   \@@_case_use:nw
             {
               \@@_trig:NNNNNwn #1 \@@_sin_series_o:NNwwww
               \@@_ep_inv_to_float:wwN 0 \c_two
             }
    \or:   \@@_case_use:nw
             { \@@_invalid_operation_o:fw { #1 { sec } { secd } } }
    \else: \@@_case_return_same_o:w
    \fi:
    \s_@@ \@@_chk:w #2 #3;
  }
%    \end{macrocode}
% \end{macro}
%
% \begin{macro}[int, EXP]{\@@_tan_o:w}
%   The tangent of $\pm 0$ or \nan{} is the same floating point number.
%   The tangent of $\pm\infty$ raises an invalid operation exception.
%   Once more, the \texttt{trig} function does the argument reduction
%   step and conversion to radians before calling
%   \cs{@@_tan_series_o:NNwwww}, with a sign~|#3| and an initial octant
%   of~$1$ (this shift is somewhat arbitrary).  See \cs{@@_cot_o:w} for
%   an explanation of the $0$~argument.
%    \begin{macrocode}
\cs_new:Npn \@@_tan_o:w #1 \s_@@ \@@_chk:w #2#3#4; @
  {
    \if_case:w #2 \exp_stop_f:
           \@@_case_return_same_o:w
    \or:   \@@_case_use:nw
             {
               \@@_trig:NNNNNwn #1
                 \@@_tan_series_o:NNwwww 0 #3 \c_one
             }
    \or:   \@@_case_use:nw
             { \@@_invalid_operation_o:fw { #1 { tan } { tand } } }
    \else: \@@_case_return_same_o:w
    \fi:
    \s_@@ \@@_chk:w #2 #3 #4;
  }
%    \end{macrocode}
% \end{macro}
%
% \begin{macro}[int, EXP]{\@@_cot_o:w}
% \begin{macro}[aux, EXP]{\@@_cot_zero_o:Nfw}
%   The cotangent of $\pm 0$ is $\pm \infty$ with the same sign, with a
%   division by zero exception (see \cs{@@_cot_zero_o:Nfw}.  The
%   cotangent of $\pm\infty$ raises an invalid operation exception.  The
%   cotangent of \nan{} is itself.  We use $\cot x = - \tan (\pi/2 +
%   x)$, and the initial octant for the tangent was chosen to be $1$, so
%   the octant here starts at $3$.  The change in sign is obtained by
%   feeding \cs{@@_tan_series_o:NNwwww} two signs rather than just the
%   sign of the argument: the first of those indicates whether we
%   compute tangent or cotangent.  Those signs are eventually combined.
%    \begin{macrocode}
\cs_new:Npn \@@_cot_o:w #1 \s_@@ \@@_chk:w #2#3#4; @
  {
    \if_case:w #2 \exp_stop_f:
           \@@_cot_zero_o:Nfw #3 { #1 { cot } { cotd } }
    \or:   \@@_case_use:nw
             {
               \@@_trig:NNNNNwn #1
                 \@@_tan_series_o:NNwwww 2 #3 \c_three
             }
    \or:   \@@_case_use:nw
             { \@@_invalid_operation_o:fw { #1 { cot } { cotd } } }
    \else: \@@_case_return_same_o:w
    \fi:
    \s_@@ \@@_chk:w #2 #3 #4;
  }
\cs_new:Npn \@@_cot_zero_o:Nfw #1#2#3 \fi:
  {
    \fi:
    \token_if_eq_meaning:NNTF 0 #1
      { \exp_args:NNf \@@_division_by_zero_o:Nnw \c_inf_fp }
      { \exp_args:NNf \@@_division_by_zero_o:Nnw \c_minus_inf_fp }
    {#2}
  }
%    \end{macrocode}
% \end{macro}
% \end{macro}
%
% \subsubsection{Distinguishing small and large arguments}
%
% \begin{macro}[aux, EXP]{\@@_trig:NNNNNwn}
%   The first argument is \cs{use_i:nn} if the operand is in radians and
%   \cs{use_ii:nn} if it is in degrees.  Arguments |#2| to~|#5| control
%   what trigonometric function we compute, and |#6| to~|#8| are pieces
%   of a normal floating point number.  Call the \texttt{_series}
%   function~|#2|, with arguments |#3|, either a conversion function
%   (\cs{@@_ep_to_float:wN} or \cs{@@_ep_inv_to_float:wN}) or a sign $0$
%   or~$2$ when computing tangent or cotangent; |#4|, a sign $0$ or~$2$;
%   the octant, computed in an integer expression starting with~|#5| and
%   stopped by a period; and a fixed point number obtained from the
%   floating point number by argument reduction (if necessary) and
%   conversion to radians (if necessary).  Any argument reduction
%   adjusts the octant accordingly by leaving a (positive) shift into
%   its integer expression.  Let us explain the integer comparison.  Two
%   of the four \cs{exp_after:wN} are expanded, the expansion hits the
%   test, which is true if the float is at least~$1$ when working in
%   radians, and at least $10$ when working in degrees.  Then one of the
%   remaining \cs{exp_after:wN} hits |#1|, which picks the \texttt{trig}
%   or \texttt{trigd} function in whichever branch of the conditional
%   was taken.  The final \cs{exp_after:wN} closes the conditional.  At
%   the end of the day, a number is \texttt{large} if it is $\geq 1$ in
%   radians or $\geq 10$ in degrees, and \texttt{small} otherwise.  All
%   four \texttt{trig}/\texttt{trigd} auxiliaries receive the operand as
%   an extended-precision number.
%    \begin{macrocode}
\cs_new:Npn \@@_trig:NNNNNwn #1#2#3#4#5 \s_@@ \@@_chk:w 1#6#7#8;
  {
    \exp_after:wN #2
    \exp_after:wN #3
    \exp_after:wN #4
    \int_use:N \__int_eval:w #5
      \exp_after:wN \exp_after:wN \exp_after:wN \exp_after:wN
      \if_int_compare:w #7 > #1 \c_zero \c_one
        #1 \@@_trig_large:ww \@@_trigd_large:ww
      \else:
        #1 \@@_trig_small:ww \@@_trigd_small:ww
      \fi:
    #7,#8{0000}{0000};
  }
%    \end{macrocode}
% \end{macro}
%
% \subsubsection{Small arguments}
%
% \begin{macro}[aux, EXP]{\@@_trig_small:ww}
%   This receives a small extended-precision number in radians and
%   converts it to a fixed point number.  Some trailing digits may be
%   lost in the conversion, so we keep the original floating point
%   number around: when computing sine or tangent (or their inverses),
%   the last step will be to multiply by the floating point number (as
%   an extended-precision number) rather than the fixed point number.
%   The period serves to end the integer expression for the octant.
%    \begin{macrocode}
\cs_new:Npn \@@_trig_small:ww #1,#2;
  { \@@_ep_to_fixed:wwn #1,#2; . #1,#2; }
%    \end{macrocode}
% \end{macro}
%
% \begin{macro}[aux, EXP]{\@@_trigd_small:ww}
%   Convert the extended-precision number to radians, then call
%   \cs{@@_trig_small:ww} to massage it in the form appropriate for the
%   \texttt{_series} auxiliary.
%    \begin{macrocode}
\cs_new:Npn \@@_trigd_small:ww #1,#2;
  {
    \@@_ep_mul_raw:wwwwN
      -1,{1745}{3292}{5199}{4329}{5769}{2369}; #1,#2;
    \@@_trig_small:ww
  }
%    \end{macrocode}
% \end{macro}
%
% \subsubsection{Argument reduction in degrees}
%
% \begin{macro}[aux, rEXP]
%   {
%     \@@_trigd_large:ww, \@@_trigd_large_auxi:nnnnwNNNN,
%     \@@_trigd_large_auxii:wNw, \@@_trigd_large_auxiii:www
%   }
%   Note that $25\times 360 = 9000$, so $10^{k+1} \equiv 10^{k}
%   \pmod{360}$ for $k\geq 3$.  When the exponent~|#1| is very large, we
%   can thus safely replace it by~$22$ (or even~$19$).  We turn the
%   floating point number into a fixed point number with two blocks of
%   $8$~digits followed by five blocks of $4$~digits.  The original
%   float is $100\times\meta{block_1}\cdots\meta{block_3}.
%   \meta{block_4}\cdots\meta{block_7}$, or is equal to it modulo~$360$
%   if the exponent~|#1| is very large.  The first auxiliary finds
%   $\meta{block_1} + \meta{block_2} \pmod{9}$, a single digit, and
%   prepends it to the $4$~digits of \meta{block_3}.  It also unpacks
%   \meta{block_4} and grabs the $4$~digits of \meta{block_7}.  The
%   second auxiliary grabs the \meta{block_3} plus any contribution from
%   the first two blocks as~|#1|, the first digit of \meta{block_4}
%   (just after the decimal point in hundreds of degrees) as~|#2|, and
%   the three other digits as~|#3|.  It finds the quotient and remainder
%   of |#1#2| modulo~$9$, adds twice the quotient to the integer
%   expression for the octant, and places the remainder (between $0$
%   and~$8$) before |#3| to form a new \meta{block_4}.  The resulting
%   fixed point number is $x\in [0, 0.9]$.  If $x\geq 0.45$, we add~$1$
%   to the octant and feed $0.9-x$ with an exponent of~$2$ (to
%   compensate the fact that we are working in units of hundreds of
%   degrees rather than degrees) to \cs{@@_trigd_small:ww}.  Otherwise,
%   we feed it~$x$ with an exponent of~$2$.  The third auxiliary also
%   discards digits which were not packed into the various
%   \meta{blocks}.  Since the original exponent~|#1| is at least~$2$,
%   those are all~$0$ and no precision is lost (|#6| and~|#7| are
%   four~$0$ each).
%    \begin{macrocode}
\cs_new:Npn \@@_trigd_large:ww #1, #2#3#4#5#6#7;
  {
    \exp_after:wN \@@_pack_eight:wNNNNNNNN
    \exp_after:wN \@@_pack_eight:wNNNNNNNN
    \exp_after:wN \@@_pack_twice_four:wNNNNNNNN
    \exp_after:wN \@@_pack_twice_four:wNNNNNNNN
    \exp_after:wN \@@_trigd_large_auxi:nnnnwNNNN
    \exp_after:wN ;
    \tex_romannumeral:D -`0
    \prg_replicate:nn { \int_max:nn { 22 - #1 } { 0 } } { 0 }
    #2#3#4#5#6#7 0000 0000 0000 !
  }
\cs_new:Npn \@@_trigd_large_auxi:nnnnwNNNN #1#2#3#4#5; #6#7#8#9
  {
    \exp_after:wN \@@_trigd_large_auxii:wNw
    \int_use:N \__int_eval:w #1 + #2
      - (#1 + #2 - \c_four) / \c_nine * \c_nine \__int_eval_end:
    #3;
    #4; #5{#6#7#8#9};
  }
\cs_new:Npn \@@_trigd_large_auxii:wNw #1; #2#3;
  {
    + (#1#2 - \c_four) / \c_nine * \c_two
    \exp_after:wN \@@_trigd_large_auxiii:www
    \int_use:N \__int_eval:w #1#2
      - (#1#2 - \c_four) / \c_nine * \c_nine \__int_eval_end: #3 ;
  }
\cs_new:Npn \@@_trigd_large_auxiii:www #1; #2; #3!
  {
    \if_int_compare:w #1 < 4500 \exp_stop_f:
      \exp_after:wN \@@_use_i_until_s:nw
      \exp_after:wN \@@_fixed_continue:wn
    \else:
      + \c_one
    \fi:
    \@@_fixed_sub:wwn {9000}{0000}{0000}{0000}{0000}{0000};
      {#1}#2{0000}{0000};
    { \@@_trigd_small:ww 2, }
  }
%    \end{macrocode}
% \end{macro}
%
% \subsubsection{Argument reduction in radians}
%
% Arguments greater or equal to~$1$ need to be reduced to a range where
% we only need a few terms of the Taylor series.  We reduce to the range
% $[0,2\pi]$ by subtracting multiples of~$2\pi$, then to the smaller
% range $[0,\pi/2]$ by subtracting multiples of~$\pi/2$ (keeping track
% of how many times~$\pi/2$ is subtracted), then to $[0,\pi/4]$ by
% mapping $x\to \pi/2 - x$ if appropriate.  When the argument is very
% large, say, $10^{100}$, an equally large multiple of~$2\pi$ must be
% subtracted, hence we must work with a very good approximation
% of~$2\pi$ in order to get a sensible remainder modulo~$2\pi$.
%
% Specifically, we multiply the argument by an approximation
% of~$1/(2\pi)$ with $\ExplSyntaxOn\int_eval:n { \c__fp_max_exponent_int
%   + 48 }\ExplSyntaxOff$~digits, then discard the integer part of the
% result, keeping $52$~digits of the fractional part.  From the
% fractional part of $x/(2\pi)$ we deduce the octant (quotient of the
% first three digits by~$125$).  We then multiply by $8$ or~$-8$ (the
% latter when the octant is odd), ignore any integer part (related to
% the octant), and convert the fractional part to an extended precision
% number, before multiplying by~$\pi/4$ to convert back to a value in
% radians in $[0,\pi/4]$.
%
% It is possible to prove that given the precision of floating points
% and their range of exponents, the $52$~digits may start at most with
% $24$~zeros.  The $5$~last digits are affected by carries from
% computations which are not done, hence we are left with at least $52 -
% 24 - 5 = 23$ significant digits, enough to round correctly up to
% $0.6\cdot\text{ulp}$ in all cases.
%
% ^^A todo: if the exponent range is reduced, store 1/2pi as a simple tl
% \begin{variable}[aux, EXP]{\@@_trig_inverse_two_pi:}
%   This macro expands to |,,!| or~|,!| followed by $10112$~decimals of
%   $10^{-16}/(2\pi)$.  The number of decimals we really need is the
%   maximum exponent plus the number of digits we will need later,~$52$,
%   plus~$12$ ($4-1$~groups of $4$~digits).  We store the decimals as a
%   control sequence name, and convert it to a token list when required:
%   strings take up less memory than their token list representation.
%    \begin{macrocode}
\cs_new_nopar:Npx \@@_trig_inverse_two_pi:
  {
    \exp_not:n { \exp_after:wN \use_none:n \token_to_str:N }
    \cs:w , , !
    0000000000000000159154943091895335768883763372514362034459645740 ~
    4564487476673440588967976342265350901138027662530859560728427267 ~
    5795803689291184611457865287796741073169983922923996693740907757 ~
    3077746396925307688717392896217397661693362390241723629011832380 ~
    1142226997557159404618900869026739561204894109369378440855287230 ~
    9994644340024867234773945961089832309678307490616698646280469944 ~
    8652187881574786566964241038995874139348609983868099199962442875 ~
    5851711788584311175187671605465475369880097394603647593337680593 ~
    0249449663530532715677550322032477781639716602294674811959816584 ~
    0606016803035998133911987498832786654435279755070016240677564388 ~
    8495713108801221993761476813777647378906330680464579784817613124 ~
    2731406996077502450029775985708905690279678513152521001631774602 ~
    0924811606240561456203146484089248459191435211575407556200871526 ~
    6068022171591407574745827225977462853998751553293908139817724093 ~
    5825479707332871904069997590765770784934703935898280871734256403 ~
    6689511662545705943327631268650026122717971153211259950438667945 ~
    0376255608363171169525975812822494162333431451061235368785631136 ~
    3669216714206974696012925057833605311960859450983955671870995474 ~
    6510431623815517580839442979970999505254387566129445883306846050 ~
    7852915151410404892988506388160776196993073410389995786918905980 ~
    9373777206187543222718930136625526123878038753888110681406765434 ~
    0828278526933426799556070790386060352738996245125995749276297023 ~
    5940955843011648296411855777124057544494570217897697924094903272 ~
    9477021664960356531815354400384068987471769158876319096650696440 ~
    4776970687683656778104779795450353395758301881838687937766124814 ~
    9530599655802190835987510351271290432315804987196868777594656634 ~
    6221034204440855497850379273869429353661937782928735937843470323 ~
    0237145837923557118636341929460183182291964165008783079331353497 ~
    7909974586492902674506098936890945883050337030538054731232158094 ~
    3197676032283131418980974982243833517435698984750103950068388003 ~
    9786723599608024002739010874954854787923568261139948903268997427 ~
    0834961149208289037767847430355045684560836714793084567233270354 ~
    8539255620208683932409956221175331839402097079357077496549880868 ~
    6066360968661967037474542102831219251846224834991161149566556037 ~
    9696761399312829960776082779901007830360023382729879085402387615 ~
    5744543092601191005433799838904654921248295160707285300522721023 ~
    6017523313173179759311050328155109373913639645305792607180083617 ~
    9548767246459804739772924481092009371257869183328958862839904358 ~
    6866663975673445140950363732719174311388066383072592302759734506 ~
    0548212778037065337783032170987734966568490800326988506741791464 ~
    6835082816168533143361607309951498531198197337584442098416559541 ~
    5225064339431286444038388356150879771645017064706751877456059160 ~
    8716857857939226234756331711132998655941596890719850688744230057 ~
    5191977056900382183925622033874235362568083541565172971088117217 ~
    9593683256488518749974870855311659830610139214454460161488452770 ~
    2511411070248521739745103866736403872860099674893173561812071174 ~
    0478899368886556923078485023057057144063638632023685201074100574 ~
    8592281115721968003978247595300166958522123034641877365043546764 ~
    6456565971901123084767099309708591283646669191776938791433315566 ~
    5066981321641521008957117286238426070678451760111345080069947684 ~
    2235698962488051577598095339708085475059753626564903439445420581 ~
    7886435683042000315095594743439252544850674914290864751442303321 ~
    3324569511634945677539394240360905438335528292434220349484366151 ~
    4663228602477666660495314065734357553014090827988091478669343492 ~
    2737602634997829957018161964321233140475762897484082891174097478 ~
    2637899181699939487497715198981872666294601830539583275209236350 ~
    6853889228468247259972528300766856937583659722919824429747406163 ~
    8183113958306744348516928597383237392662402434501997809940402189 ~
    6134834273613676449913827154166063424829363741850612261086132119 ~
    9863346284709941839942742955915628333990480382117501161211667205 ~
    1912579303552929241134403116134112495318385926958490443846807849 ~
    0973982808855297045153053991400988698840883654836652224668624087 ~
    2540140400911787421220452307533473972538149403884190586842311594 ~
    6322744339066125162393106283195323883392131534556381511752035108 ~
    7459558201123754359768155340187407394340363397803881721004531691 ~
    8295194879591767395417787924352761740724605939160273228287946819 ~
    3649128949714953432552723591659298072479985806126900733218844526 ~
    7943350455801952492566306204876616134365339920287545208555344144 ~
    0990512982727454659118132223284051166615650709837557433729548631 ~
    2041121716380915606161165732000083306114606181280326258695951602 ~
    4632166138576614804719932707771316441201594960110632830520759583 ~
    4850305079095584982982186740289838551383239570208076397550429225 ~
    9847647071016426974384504309165864528360324933604354657237557916 ~
    1366324120457809969715663402215880545794313282780055246132088901 ~
    8742121092448910410052154968097113720754005710963406643135745439 ~
    9159769435788920793425617783022237011486424925239248728713132021 ~
    7667360756645598272609574156602343787436291321097485897150713073 ~
    9104072643541417970572226547980381512759579124002534468048220261 ~
    7342299001020483062463033796474678190501811830375153802879523433 ~
    4195502135689770912905614317878792086205744999257897569018492103 ~
    2420647138519113881475640209760554895793785141404145305151583964 ~
    2823265406020603311891586570272086250269916393751527887360608114 ~
    5569484210322407772727421651364234366992716340309405307480652685 ~
    0930165892136921414312937134106157153714062039784761842650297807 ~
    8606266969960809184223476335047746719017450451446166382846208240 ~
    8673595102371302904443779408535034454426334130626307459513830310 ~
    2293146934466832851766328241515210179422644395718121717021756492 ~
    1964449396532222187658488244511909401340504432139858628621083179 ~
    3939608443898019147873897723310286310131486955212620518278063494 ~
    5711866277825659883100535155231665984394090221806314454521212978 ~
    9734471488741258268223860236027109981191520568823472398358013366 ~
    0683786328867928619732367253606685216856320119489780733958419190 ~
    6659583867852941241871821727987506103946064819585745620060892122 ~
    8416394373846549589932028481236433466119707324309545859073361878 ~
    6290631850165106267576851216357588696307451999220010776676830946 ~
    9814975622682434793671310841210219520899481912444048751171059184 ~
    4139907889455775184621619041530934543802808938628073237578615267 ~
    7971143323241969857805637630180884386640607175368321362629671224 ~
    2609428540110963218262765120117022552929289655594608204938409069 ~
    0760692003954646191640021567336017909631872891998634341086903200 ~
    5796637103128612356988817640364252540837098108148351903121318624 ~
    7228181050845123690190646632235938872454630737272808789830041018 ~
    9485913673742589418124056729191238003306344998219631580386381054 ~
    2457893450084553280313511884341007373060595654437362488771292628 ~
    9807423539074061786905784443105274262641767830058221486462289361 ~
    9296692992033046693328438158053564864073184440599549689353773183 ~
    6726613130108623588021288043289344562140479789454233736058506327 ~
    0439981932635916687341943656783901281912202816229500333012236091 ~
    8587559201959081224153679499095448881099758919890811581163538891 ~
    6339402923722049848375224236209100834097566791710084167957022331 ~
    7897107102928884897013099533995424415335060625843921452433864640 ~
    3432440657317477553405404481006177612569084746461432976543900008 ~
    3826521145210162366431119798731902751191441213616962045693602633 ~
    6102355962140467029012156796418735746835873172331004745963339773 ~
    2477044918885134415363760091537564267438450166221393719306748706 ~
    2881595464819775192207710236743289062690709117919412776212245117 ~
    2354677115640433357720616661564674474627305622913332030953340551 ~
    3841718194605321501426328000879551813296754972846701883657425342 ~
    5016994231069156343106626043412205213831587971115075454063290657 ~
    0248488648697402872037259869281149360627403842332874942332178578 ~
    7750735571857043787379693402336902911446961448649769719434527467 ~
    4429603089437192540526658890710662062575509930379976658367936112 ~
    8137451104971506153783743579555867972129358764463093757203221320 ~
    2460565661129971310275869112846043251843432691552928458573495971 ~
    5042565399302112184947232132380516549802909919676815118022483192 ~
    5127372199792134331067642187484426215985121676396779352982985195 ~
    8545392106957880586853123277545433229161989053189053725391582222 ~
    9232597278133427818256064882333760719681014481453198336237910767 ~
    1255017528826351836492103572587410356573894694875444694018175923 ~
    0609370828146501857425324969212764624247832210765473750568198834 ~
    5641035458027261252285503154325039591848918982630498759115406321 ~
    0354263890012837426155187877318375862355175378506956599570028011 ~
    5841258870150030170259167463020842412449128392380525772514737141 ~
    2310230172563968305553583262840383638157686828464330456805994018 ~
    7001071952092970177990583216417579868116586547147748964716547948 ~
    8312140431836079844314055731179349677763739898930227765607058530 ~
    4083747752640947435070395214524701683884070908706147194437225650 ~
    2823145872995869738316897126851939042297110721350756978037262545 ~
    8141095038270388987364516284820180468288205829135339013835649144 ~
    3004015706509887926715417450706686888783438055583501196745862340 ~
    8059532724727843829259395771584036885940989939255241688378793572 ~
    7967951654076673927031256418760962190243046993485989199060012977 ~
    7469214532970421677817261517850653008552559997940209969455431545 ~
    2745856704403686680428648404512881182309793496962721836492935516 ~
    2029872469583299481932978335803459023227052612542114437084359584 ~
    9443383638388317751841160881711251279233374577219339820819005406 ~
    3292937775306906607415304997682647124407768817248673421685881509 ~
    9133422075930947173855159340808957124410634720893194912880783576 ~
    3115829400549708918023366596077070927599010527028150868897828549 ~
    4340372642729262103487013992868853550062061514343078665396085995 ~
    0058714939141652065302070085265624074703660736605333805263766757 ~
    2018839497277047222153633851135483463624619855425993871933367482 ~
    0422097449956672702505446423243957506869591330193746919142980999 ~
    3424230550172665212092414559625960554427590951996824313084279693 ~
    7113207021049823238195747175985519501864630940297594363194450091 ~
    9150616049228764323192129703446093584259267276386814363309856853 ~
    2786024332141052330760658841495858718197071242995959226781172796 ~
    4438853796763139274314227953114500064922126500133268623021550837
    \cs_end:
  }
%    \end{macrocode}
% \end{variable}
%
% \begin{macro}[aux, rEXP]
%   {
%     \@@_trig_large:ww,
%     \@@_trig_large_auxi:wwwwww,
%     \@@_trig_large_auxii:ww,
%     \@@_trig_large_auxiii:wNNNNNNNN,
%     \@@_trig_large_auxiv:wN
%   }
%   The exponent~|#1| is between $1$ and~$\ExplSyntaxOn \int_use:N
%   \c__fp_max_exponent_int$.  We discard the integer part of
%   $10^{\text{\texttt{\#1}}-16}/(2\pi)$, that is, the first |#1|~digits
%   of $10^{-16}/(2\pi)$, because it yields an integer contribution to
%   $x/(2\pi)$.  The \texttt{auxii} auxiliary discards~$64$ digits at a
%   time thanks to spaces inserted in the result of
%   \cs{@@_trig_inverse_two_pi:}, while \texttt{auxiii} discards~$8$
%   digits at a time, and \texttt{auxiv} discards digits one at a time.
%   Then $64$~digits are packed into groups of~$4$ and the \texttt{auxv}
%   auxiliary is called.
%    \begin{macrocode}
\cs_new:Npn \@@_trig_large:ww #1, #2#3#4#5#6;
  {
    \exp_after:wN \@@_trig_large_auxi:wwwwww
    \int_use:N \__int_eval:w (#1 - 32) / 64 \exp_after:wN ,
    \int_use:N \__int_eval:w (#1 - 4) / 8 \exp_after:wN ,
    \__int_value:w #1 \@@_trig_inverse_two_pi: ;
    {#2}{#3}{#4}{#5} ;
  }
\cs_new:Npn \@@_trig_large_auxi:wwwwww #1, #2, #3, #4!
  {
    \prg_replicate:nn {#1} { \@@_trig_large_auxii:ww }
    \prg_replicate:nn { #2 - #1 * \c_eight }
      { \@@_trig_large_auxiii:wNNNNNNNN }
    \prg_replicate:nn { #3 - #2 * \c_eight }
      { \@@_trig_large_auxiv:wN }
    \prg_replicate:nn { \c_eight } { \@@_pack_twice_four:wNNNNNNNN }
    \@@_trig_large_auxv:www
    ;
  }
\cs_new:Npn \@@_trig_large_auxii:ww #1; #2 ~ { #1; }
\cs_new:Npn \@@_trig_large_auxiii:wNNNNNNNN
  #1; #2#3#4#5#6#7#8#9 { #1; }
\cs_new:Npn \@@_trig_large_auxiv:wN #1; #2 { #1; }
%    \end{macrocode}
% \end{macro}
%
% \begin{macro}[aux, rEXP]
%   {
%     \@@_trig_large_auxv:www,
%     \@@_trig_large_auxvi:wnnnnnnnn,
%     \@@_trig_large_pack:NNNNNw
%   }
%   First come the first $64$~digits of the fractional part of
%   $10^{\text{\texttt{\#1}}-16}/(2\pi)$, arranged in $16$~blocks
%   of~$4$, and ending with a semicolon.  Then some more digits of the
%   same fractional part, ending with a semicolon, then $4$~blocks of
%   $4$~digits holding the significand of the original argument.
%   Multiply the $16$-digit significand with the $64$-digit fractional
%   part: the \texttt{auxvi} auxiliary receives the significand
%   as~|#2#3#4#5| and $16$~digits of the fractional part as~|#6#7#8#9|,
%   and computes one step of the usual ladder of \texttt{pack} functions
%   we use for multiplication (see \emph{e.g.,} \cs{@@_fixed_mul:wwn}),
%   then discards one block of the fractional part to set things up for
%   the next step of the ladder.  We perform $13$~such steps, replacing
%   the last \texttt{middle} shift by the appropriate \texttt{trailing}
%   shift, then discard the significand and remaining $3$~blocks from
%   the fractional part, as there are not enough digits to compute any
%   more step in the ladder.  The last semicolon closes the ladder, and
%   we return control to the \texttt{auxvii} auxiliary.
%    \begin{macrocode}
\cs_new:Npn \@@_trig_large_auxv:www #1; #2; #3;
  {
    \exp_after:wN \@@_use_i_until_s:nw
    \exp_after:wN \@@_trig_large_auxvii:w
    \int_use:N \__int_eval:w \c_@@_leading_shift_int
      \prg_replicate:nn { \c_thirteen }
        { \@@_trig_large_auxvi:wnnnnnnnn }
      + \c_@@_trailing_shift_int - \c_@@_middle_shift_int
      \@@_use_i_until_s:nw
      ; #3 #1 ; ;
  }
\cs_new:Npn \@@_trig_large_auxvi:wnnnnnnnn #1; #2#3#4#5#6#7#8#9
  {
    \exp_after:wN \@@_trig_large_pack:NNNNNw
    \int_use:N \__int_eval:w \c_@@_middle_shift_int
      + #2*#9 + #3*#8 + #4*#7 + #5*#6
      #1; {#2}{#3}{#4}{#5} {#7}{#8}{#9}
  }
\cs_new:Npn \@@_trig_large_pack:NNNNNw #1#2#3#4#5#6;
  { + #1#2#3#4#5 ; #6 }
%    \end{macrocode}
% \end{macro}
%
% \begin{macro}[aux, rEXP]
%   {
%     \@@_trig_large_auxvii:w,
%     \@@_trig_large_auxviii:w,
%   }
% \begin{macro}[aux, EXP]
%   {
%     \@@_trig_large_auxix:Nw,
%     \@@_trig_large_auxx:wNNNNN,
%     \@@_trig_large_auxxi:w
%   }
%   The \texttt{auxvii} auxiliary is followed by $52$~digits and a
%   semicolon.  We find the octant as the integer part of $8$~times what
%   follows, or equivalently as the integer part of $|#1#2#3|/125$, and
%   add it to the surrounding integer expression for the octant.  We
%   then compute $8$~times the $52$-digit number, with a minus sign if
%   the octant is odd.  Again, the last \texttt{middle} shift is
%   converted to a \texttt{trailing} shift.  Any integer part (including
%   negative values which come up when the octant is odd) is discarded
%   by \cs{@@_use_i_until_s:nw}.  The resulting fractional part should
%   then be converted to radians by multiplying by~$2\pi/8$, but first,
%   build an extended precision number by abusing
%   \cs{@@_ep_to_ep_loop:N} with the appropriate trailing markers.
%   Finally, \cs{@@_trig_small:ww} sets up the argument for the
%   functions which compute the Taylor series.
%    \begin{macrocode}
\cs_new:Npn \@@_trig_large_auxvii:w #1#2#3
  {
    \exp_after:wN \@@_trig_large_auxviii:ww
    \int_use:N \__int_eval:w (#1#2#3 - 62) / 125 ;
    #1#2#3
  }
\cs_new:Npn \@@_trig_large_auxviii:ww #1;
  {
    + #1
    \if_int_odd:w #1 \exp_stop_f:
      \exp_after:wN \@@_trig_large_auxix:Nw
      \exp_after:wN -
    \else:
      \exp_after:wN \@@_trig_large_auxix:Nw
      \exp_after:wN +
    \fi:
  }
\cs_new_nopar:Npn \@@_trig_large_auxix:Nw
  {
    \exp_after:wN \@@_use_i_until_s:nw
    \exp_after:wN \@@_trig_large_auxxi:w
    \int_use:N \__int_eval:w \c_@@_leading_shift_int
      \prg_replicate:nn { \c_thirteen }
        { \@@_trig_large_auxx:wNNNNN }
      + \c_@@_trailing_shift_int - \c_@@_middle_shift_int
      ;
  }
\cs_new:Npn \@@_trig_large_auxx:wNNNNN #1; #2 #3#4#5#6
  {
    \exp_after:wN \@@_trig_large_pack:NNNNNw
    \int_use:N \__int_eval:w \c_@@_middle_shift_int
      #2 \c_eight * #3#4#5#6
      #1; #2
  }
\cs_new:Npn \@@_trig_large_auxxi:w #1;
  {
    \exp_after:wN \@@_ep_mul_raw:wwwwN
    \int_use:N \__int_eval:w \c_zero \@@_ep_to_ep_loop:N #1 ; ; !
    0,{7853}{9816}{3397}{4483}{0961}{5661};
    \@@_trig_small:ww
  }
%    \end{macrocode}
% \end{macro}
% \end{macro}
%
% \subsubsection{Computing the power series}
%
% \begin{macro}[aux, EXP]
%   {\@@_sin_series_o:NNwwww, \@@_sin_series_aux_o:NNnwww}
%   Here we receive a conversion function \cs{@@_ep_to_float:wwN} or
%   \cs{@@_ep_inv_to_float:wwN}, a \meta{sign} ($0$ or~$2$), a
%   (non-negative) \meta{octant} delimited by a dot, a \meta{fixed
%     point} number delimited by a semicolon, and an extended-precision
%   number.  The auxiliary receives:
%   \begin{itemize}
%   \item the conversion function~|#1|;
%   \item the final sign, which depends on the octant~|#3| and the
%     sign~|#2|;
%   \item the octant~|#3|, which will control the series we use;
%   \item the square |#4 * #4| of the argument as a fixed point number,
%     computed with \cs{@@_fixed_mul:wwn};
%   \item the number itself as an extended-precision number.
%   \end{itemize}
%   If the octant is in $\{1,2,5,6,\ldots{}\}$, we are near an extremum
%   of the function and we use the series
%   \[
%   \cos(x) = 1 - x^2 \bigg( \frac{1}{2!} - x^2 \bigg( \frac{1}{4!}
%   - x^2 \bigg( \cdots \bigg) \bigg) \bigg) .
%   \]
%   Otherwise, the series
%   \[
%   \sin(x) = x \bigg( 1 - x^2 \bigg( \frac{1}{3!} - x^2 \bigg(
%   \frac{1}{5!} - x^2 \bigg( \cdots \bigg) \bigg) \bigg) \bigg)
%   \]
%   is used.  Finally, the extended-precision number is converted to a
%   floating point number with the given sign, and \cs{@@_sanitize:Nw}
%   checks for overflow and underflow.
%    \begin{macrocode}
\cs_new:Npn \@@_sin_series_o:NNwwww #1#2#3. #4;
  {
    \@@_fixed_mul:wwn #4; #4;
    {
      \exp_after:wN \@@_sin_series_aux_o:NNnwww
      \exp_after:wN #1
      \__int_value:w
        \if_int_odd:w \__int_eval:w ( #3 + \c_two ) / \c_four \__int_eval_end:
          #2
        \else:
          \if_meaning:w #2 0 2 \else: 0 \fi:
        \fi:
      {#3}
    }
  }
\cs_new:Npn \@@_sin_series_aux_o:NNnwww #1#2#3 #4; #5,#6;
  {
    \if_int_odd:w \__int_eval:w #3 / \c_two \__int_eval_end:
      \exp_after:wN \use_i:nn
    \else:
      \exp_after:wN \use_ii:nn
    \fi:
    { % 1/18!
      \@@_fixed_mul_sub_back:wwwn     {0000}{0000}{0000}{0001}{5619}{2070};
                                  #4; {0000}{0000}{0000}{0477}{9477}{3324};
      \@@_fixed_mul_sub_back:wwwn #4; {0000}{0000}{0011}{4707}{4559}{7730};
      \@@_fixed_mul_sub_back:wwwn #4; {0000}{0000}{2087}{6756}{9878}{6810};
      \@@_fixed_mul_sub_back:wwwn #4; {0000}{0027}{5573}{1922}{3985}{8907};
      \@@_fixed_mul_sub_back:wwwn #4; {0000}{2480}{1587}{3015}{8730}{1587};
      \@@_fixed_mul_sub_back:wwwn #4; {0013}{8888}{8888}{8888}{8888}{8889};
      \@@_fixed_mul_sub_back:wwwn #4; {0416}{6666}{6666}{6666}{6666}{6667};
      \@@_fixed_mul_sub_back:wwwn #4; {5000}{0000}{0000}{0000}{0000}{0000};
      \@@_fixed_mul_sub_back:wwwn #4;{10000}{0000}{0000}{0000}{0000}{0000};
      { \@@_fixed_continue:wn 0, }
    }
    { % 1/17!
      \@@_fixed_mul_sub_back:wwwn     {0000}{0000}{0000}{0028}{1145}{7254};
                                  #4; {0000}{0000}{0000}{7647}{1637}{3182};
      \@@_fixed_mul_sub_back:wwwn #4; {0000}{0000}{0160}{5904}{3836}{8216};
      \@@_fixed_mul_sub_back:wwwn #4; {0000}{0002}{5052}{1083}{8544}{1719};
      \@@_fixed_mul_sub_back:wwwn #4; {0000}{0275}{5731}{9223}{9858}{9065};
      \@@_fixed_mul_sub_back:wwwn #4; {0001}{9841}{2698}{4126}{9841}{2698};
      \@@_fixed_mul_sub_back:wwwn #4; {0083}{3333}{3333}{3333}{3333}{3333};
      \@@_fixed_mul_sub_back:wwwn #4; {1666}{6666}{6666}{6666}{6666}{6667};
      \@@_fixed_mul_sub_back:wwwn #4;{10000}{0000}{0000}{0000}{0000}{0000};
      { \@@_ep_mul:wwwwn 0, } #5,#6;
    }
    {
      \exp_after:wN \@@_sanitize:Nw
      \exp_after:wN #2
      \int_use:N \__int_eval:w #1
    }
    #2
  }
%    \end{macrocode}
% \end{macro}
%
% \begin{macro}[aux, EXP]
%   {\@@_tan_series_o:NNwwww, \@@_tan_series_aux_o:Nnwww}
%   Contrarily to \cs{@@_sin_series_o:NNwwww} which received a
%   conversion auxiliary as~|#1|, here, |#1| is $0$ for tangent
%   and $2$ for
%   cotangent.  Consider first the case of the tangent.  The octant |#3|
%   starts at $1$, which means that it is $1$ or $2$ for $\lvert
%   x\rvert\in[0,\pi/2]$, it is $3$ or $4$ for $\lvert
%   x\rvert\in[\pi/2,\pi]$, and so on: the intervals on which
%   $\tan\lvert x\rvert\geq 0$ coincide with those for which $\lfloor
%   (|#3| + 1) / 2\rfloor$ is odd.  We also have to take into account
%   the original sign of $x$ to get the sign of the final result; it is
%   straightforward to check that the first \cs{__int_value:w} expansion
%   produces $0$ for a positive final result, and $2$ otherwise.  A
%   similar story holds for $\cot(x)$.
%
%   The auxiliary receives the sign, the octant, the square of the
%   (reduced) input, and the (reduced) input (an extended-precision
%   number) as arguments.  It then
%   computes the numerator and denominator of
%   \[
%   \tan(x) \simeq
%   \frac{x (1 - x^2 (a_1 - x^2 (a_2 - x^2 (a_3 - x^2 (a_4 - x^2 a_5)))))}
%     {1 - x^2 (b_1 - x^2 (b_2 - x^2 (b_3 - x^2 (b_4 - x^2 b_5))))} .
%   \]
%   The ratio is computed by \cs{@@_ep_div:wwwwn}, then converted to a
%   floating point number.  For octants~|#3| (really, quadrants) next to
%   a pole of the
%   functions, the fixed point numerator and denominator are exchanged
%   before computing the ratio.  Note that this \cs{if_int_odd:w} test
%   relies on the fact that the octant is at least~$1$.
%    \begin{macrocode}
\cs_new:Npn \@@_tan_series_o:NNwwww #1#2#3. #4;
  {
    \@@_fixed_mul:wwn #4; #4;
    {
      \exp_after:wN \@@_tan_series_aux_o:Nnwww
      \__int_value:w
        \if_int_odd:w \__int_eval:w #3 / \c_two \__int_eval_end:
          \exp_after:wN \reverse_if:N
        \fi:
        \if_meaning:w #1#2 2 \else: 0 \fi:
      {#3}
    }
  }
\cs_new:Npn \@@_tan_series_aux_o:Nnwww #1 #2 #3; #4,#5;
  {
    \@@_fixed_mul_sub_back:wwwn     {0000}{0000}{1527}{3493}{0856}{7059};
                                #3; {0000}{0159}{6080}{0274}{5257}{6472};
    \@@_fixed_mul_sub_back:wwwn #3; {0002}{4571}{2320}{0157}{2558}{8481};
    \@@_fixed_mul_sub_back:wwwn #3; {0115}{5830}{7533}{5397}{3168}{2147};
    \@@_fixed_mul_sub_back:wwwn #3; {1929}{8245}{6140}{3508}{7719}{2982};
    \@@_fixed_mul_sub_back:wwwn #3;{10000}{0000}{0000}{0000}{0000}{0000};
    { \@@_ep_mul:wwwwn 0, } #4,#5;
    {
      \@@_fixed_mul_sub_back:wwwn     {0000}{0007}{0258}{0681}{9408}{4706};
                                  #3; {0000}{2343}{7175}{1399}{6151}{7670};
      \@@_fixed_mul_sub_back:wwwn #3; {0019}{2638}{4588}{9232}{8861}{3691};
      \@@_fixed_mul_sub_back:wwwn #3; {0536}{6357}{0691}{4344}{6852}{4252};
      \@@_fixed_mul_sub_back:wwwn #3; {5263}{1578}{9473}{6842}{1052}{6315};
      \@@_fixed_mul_sub_back:wwwn #3;{10000}{0000}{0000}{0000}{0000}{0000};
      {
        \reverse_if:N \if_int_odd:w
            \__int_eval:w (#2 - \c_one) / \c_two \__int_eval_end:
          \exp_after:wN \@@_reverse_args:Nww
        \fi:
        \@@_ep_div:wwwwn 0,
      }
    }
    {
      \exp_after:wN \@@_sanitize:Nw
      \exp_after:wN #1
      \int_use:N \__int_eval:w \@@_ep_to_float:wwN
    }
    #1
  }
%    \end{macrocode}
% \end{macro}
%
% \subsection{Inverse trigonometric functions}
%
% All inverse trigonometric functions (arcsine, arccosine, arctangent,
% arccotangent, arccosecant, and arcsecant) are based on a function
% often denoted \texttt{atan2}.  This function is accessed directly by
% feeding two arguments to arctangent, and is defined by \(\operatorname{atan}(y, x) =
% \operatorname{atan}(y/x)\) for generic \(y\) and~\(x\).  Its advantages over the
% conventional arctangent is that it takes values in $[-\pi,\pi]$ rather
% than $[-\pi/2,\pi/2]$, and that it is better behaved in boundary
% cases.  Other inverse trigonometric functions are expressed in terms
% of \(\operatorname{atan}\) as
% \begin{align}
%   \operatorname{acos} x & = \operatorname{atan}(\sqrt{1-x^2}, x) \\
%   \operatorname{asin} x & = \operatorname{atan}(x, \sqrt{1-x^2}) \\
%   \operatorname{asec} x & = \operatorname{atan}(\sqrt{x^2-1}, 1) \\
%   \operatorname{acsc} x & = \operatorname{atan}(1, \sqrt{x^2-1}) \\
%   \operatorname{atan} x & = \operatorname{atan}(x, 1) \\
%   \operatorname{acot} x & = \operatorname{atan}(1, x) .
% \end{align}
% Rather than introducing a new function, \texttt{atan2}, the arctangent
% function \texttt{atan} is overloaded: it can take one or two
% arguments.  In the comments below, following many texts, we call the
% first argument~$y$ and the second~$x$, because $\operatorname{atan}(y, x) = \operatorname{atan}(y
% / x)$ is the angular coordinate of the point $(x, y)$.
%
% As for direct trigonometric functions, the first step in computing
% $\operatorname{atan}(y, x)$ is argument reduction.  The sign of~$y$ will give that
% of the result.  We distinguish eight regions where the point $(x,
% \lvert y\rvert)$ can lie, of angular size roughly $\pi/8$,
% characterized by their ``octant'', between $0$ and~$7$ included.  In
% each region, we compute an arctangent as a Taylor series, then shift
% this arctangent by the appropriate multiple of $\pi/4$ and sign to get
% the result.  Here is a list of octants, and how we compute the
% arctangent (we assume $y>0$: otherwise replace $y$ by~$-y$ below):
% \begin{itemize}
% \item[0] $0 < \lvert y\rvert < 0.41421 x$, then
%   $\operatorname{atan}\frac{\lvert y\rvert}{x}$
%   is given by a nicely convergent Taylor series;
% \item[1] $0 < 0.41421 x < \lvert y\rvert < x$, then
%   $\operatorname{atan}\frac{\lvert y\rvert}{x}
%   = \frac{\pi}{4}-\operatorname{atan}\frac{x-\lvert y\rvert}{x+\lvert y\rvert}$;
% \item[2] $0 < 0.41421 \lvert y\rvert < x < \lvert y\rvert$, then
%   $\operatorname{atan}\frac{\lvert y\rvert}{x}
%   = \frac{\pi}{4}+\operatorname{atan}\frac{-x+\lvert y\rvert}{x+\lvert y\rvert}$;
% \item[3] $0 < x < 0.41421 \lvert y\rvert$, then
%   $\operatorname{atan}\frac{\lvert y\rvert}{x}
%   = \frac{\pi}{2}-\operatorname{atan}\frac{x}{\lvert y\rvert}$;
% \item[4] $0 < -x < 0.41421 \lvert y\rvert$, then
%   $\operatorname{atan}\frac{\lvert y\rvert}{x}
%   = \frac{\pi}{2}+\operatorname{atan}\frac{-x}{\lvert y\rvert}$;
% \item[5] $0 < 0.41421 \lvert y\rvert < -x < \lvert y\rvert$, then
%   $\operatorname{atan}\frac{\lvert y\rvert}{x}
%   =\frac{3\pi}{4}-\operatorname{atan}\frac{x+\lvert y\rvert}{-x+\lvert y\rvert}$;
% \item[6] $0 < -0.41421 x < \lvert y\rvert < -x$, then
%   $\operatorname{atan}\frac{\lvert y\rvert}{x}
%   =\frac{3\pi}{4}+\operatorname{atan}\frac{-x-\lvert y\rvert}{-x+\lvert y\rvert}$;
% \item[7] $0 < \lvert y\rvert < -0.41421 x$, then
%   $\operatorname{atan}\frac{\lvert y\rvert}{x}
%   = \pi-\operatorname{atan}\frac{\lvert y\rvert}{-x}$.
% \end{itemize}
% In the following, we will denote by~$z$ the ratio among
% $\lvert\frac{y}{x}\rvert$, $\lvert\frac{x}{y}\rvert$,
% $\lvert\frac{x+y}{x-y}\rvert$, $\lvert\frac{x-y}{x+y}\rvert$ which
% appears in the right-hand side above.
%
% \subsubsection{Arctangent and arccotangent}
%
% \begin{macro}[int, EXP]{\@@_atan_o:Nw, \@@_acot_o:Nw}
% \begin{macro}[aux, EXP]{\@@_atan_dispatch_o:NNnNw}
%   The parsing step manipulates \texttt{atan} and \texttt{acot} like
%   \texttt{min} and \texttt{max}, reading in an array of operands, but
%   also leaves \cs{use_i:nn} or \cs{use_ii:nn} depending on whether the
%   result should be given in radians or in degrees.  Here, we dispatch
%   according to the number of arguments.  The one-argument versions of
%   arctangent and arccotangent are special cases of the two-argument
%   ones: $\operatorname{atan}(y) = \operatorname{atan}(y, 1) = \operatorname{acot}(1, y)$ and
%   $\operatorname{acot}(x) = \operatorname{atan}(1, x) = \operatorname{acot}(x, 1)$.
%    \begin{macrocode}
\cs_new_nopar:Npn \@@_atan_o:Nw
  {
    \@@_atan_dispatch_o:NNnNw
      \@@_acotii_o:Nww \@@_atanii_o:Nww { atan }
  }
\cs_new_nopar:Npn \@@_acot_o:Nw
  {
    \@@_atan_dispatch_o:NNnNw
      \@@_atanii_o:Nww \@@_acotii_o:Nww { acot }
  }
\cs_new:Npn \@@_atan_dispatch_o:NNnNw #1#2#3#4#5@
  {
    \if_case:w
      \__int_eval:w \@@_array_count:n {#5} - \c_one \__int_eval_end:
         \exp_after:wN #1 \exp_after:wN #4 \c_one_fp #5
         \tex_romannumeral:D
    \or: #2 #4 #5 \tex_romannumeral:D
    \else:
      \__msg_kernel_expandable_error:nnnnn
        { kernel } { fp-num-args } { #3() } { 1 } { 2 }
      \exp_after:wN \c_nan_fp \tex_romannumeral:D
    \fi:
    \exp_after:wN \c_zero
  }
%    \end{macrocode}
% \end{macro}
% \end{macro}
%
% \begin{macro}[int, EXP]{\@@_atanii_o:Nww, \@@_acotii_o:Nww}
%   If either operand is \texttt{nan}, we return it.  If both are
%   normal, we call \cs{@@_atan_normal_o:NNnwNnw}.  If both are zero or
%   both infinity, we call \cs{@@_atan_inf_o:NNNw} with argument~$2$,
%   leading to a result among $\{\pm\pi/4, \pm 3\pi/4\}$ (in degrees,
%   $\{\pm 45, \pm 135\}$).  Otherwise, one is much bigger than the
%   other, and we call \cs{@@_atan_inf_o:NNNw} with either an argument
%   of~$4$, leading to the values $\pm\pi/2$ (in degrees,~$\pm 90$),
%   or~$0$, leading to $\{\pm 0, \pm\pi\}$ (in degrees, $\{\pm 0,\pm
%   180\}$).  Since $\operatorname{acot}(x, y) = \operatorname{atan}(y, x)$,
%   \cs{@@_acotii_o:ww} simply reverses its two arguments.
%    \begin{macrocode}
\cs_new:Npn \@@_atanii_o:Nww
    #1 \s_@@ \@@_chk:w #2#3#4; \s_@@ \@@_chk:w #5
  {
    \if_meaning:w 3 #2 \@@_case_return_i_o:ww \fi:
    \if_meaning:w 3 #5 \@@_case_return_ii_o:ww \fi:
    \if_case:w
      \if_meaning:w #2 #5
        \if_meaning:w 1 #2 \c_ten \else: \c_zero \fi:
      \else:
        \if_int_compare:w #2 > #5 \c_one \else: \c_two \fi:
      \fi:
         \@@_case_return:nw { \@@_atan_inf_o:NNNw #1 #3 \c_two }
    \or: \@@_case_return:nw { \@@_atan_inf_o:NNNw #1 #3 \c_four }
    \or: \@@_case_return:nw { \@@_atan_inf_o:NNNw #1 #3 \c_zero }
    \fi:
    \@@_atan_normal_o:NNnwNnw #1
    \s_@@ \@@_chk:w #2#3#4;
    \s_@@ \@@_chk:w #5
  }
\cs_new:Npn \@@_acotii_o:Nww #1#2; #3;
  { \@@_atanii_o:Nww #1#3; #2; }
%    \end{macrocode}
% \end{macro}
%
% \begin{macro}[aux, EXP]{\@@_atan_inf_o:NNNw}
%   This auxiliary is called whenever one number is $\pm 0$ or
%   $\pm\infty$ (and neither is \nan{}).  Then the result only depends
%   on the signs, and its value is a multiple of $\pi/4$.  We use the
%   same auxiliary as for normal numbers,
%   \cs{@@_atan_combine_o:NwwwwwN}, with arguments the final sign~|#2|;
%   the octant~|#3|; $\operatorname{atan} z/z=1$ as a fixed point number; $z=0$~as a
%   fixed point number; and $z=0$~as an extended-precision number.
%   Given the values we provide, $\operatorname{atan} z$ will be computed to be~$0$,
%   and the result will be $[|#3|/2]\cdot\pi/4$ if the sign~|#5| of~$x$
%   is positive, and $[(7-|#3|)/2]\cdot\pi/4$ for negative~$x$, where
%   the divisions are rounded up.
%    \begin{macrocode}
\cs_new:Npn \@@_atan_inf_o:NNNw #1#2#3 \s_@@ \@@_chk:w #4#5#6;
  {
    \exp_after:wN \@@_atan_combine_o:NwwwwwN
    \exp_after:wN #2
    \int_use:N \__int_eval:w
      \if_meaning:w 2 #5 \c_seven - \fi: #3 \exp_after:wN ;
    \c_@@_one_fixed_tl ;
    {0000}{0000}{0000}{0000}{0000}{0000};
    0,{0000}{0000}{0000}{0000}{0000}{0000}; #1
  }
%    \end{macrocode}
% \end{macro}
%
% \begin{macro}[aux, EXP]{\@@_atan_normal_o:NNnwNnw}
%   Here we simply reorder the floating point data into a pair of signed
%   extended-precision numbers, that is, a sign, an exponent ending with
%   a comma, and a six-block mantissa ending with a semi-colon.  This
%   extended precision is required by other inverse trigonometric
%   functions, to compute things like $\operatorname{atan}(x,\sqrt{1-x^2})$ without
%   intermediate rounding errors.
%    \begin{macrocode}
\cs_new_protected:Npn \@@_atan_normal_o:NNnwNnw
    #1 \s_@@ \@@_chk:w 1#2#3#4; \s_@@ \@@_chk:w 1#5#6#7;
  {
    \@@_atan_test_o:NwwNwwN
      #2 #3, #4{0000}{0000};
      #5 #6, #7{0000}{0000}; #1
  }
%    \end{macrocode}
% \end{macro}
%
% \begin{macro}[aux, EXP]{\@@_atan_test_o:NwwNwwN}
%   This receives: the sign~|#1| of~$y$, its exponent~|#2|, its $24$
%   digits~|#3| in groups of~$4$, and similarly for~$x$.  We prepare to
%   call \cs{@@_atan_combine_o:NwwwwwN} which expects the sign~|#1|, the
%   octant, the ratio $(\operatorname{atan} z)/z = 1 - \cdots$, and the value of~$z$,
%   both as a fixed point number and as an extended-precision floating
%   point number with a mantissa in $[0.01,1)$.  For now, we place |#1|
%   as a first argument, and start an integer expression for the octant.
%   The sign of $x$ does not affect what~$z$ will be, so we simply leave
%   a contribution to the octant: $\meta{octant} \to 7 - \meta{octant}$
%   for negative~$x$.  Then we order $\lvert y\rvert$ and $\lvert
%   x\rvert$ in a non-decreasing order: if $\lvert y\rvert > \lvert
%   x\rvert$, insert $3-$ in the expression for the octant, and swap the
%   two numbers.  The finer test with $0.41421$ is done by
%   \cs{@@_atan_div:wnwwnw} after the operands have been ordered.
%    \begin{macrocode}
\cs_new:Npn \@@_atan_test_o:NwwNwwN #1#2,#3; #4#5,#6;
  {
    \exp_after:wN \@@_atan_combine_o:NwwwwwN
    \exp_after:wN #1
    \int_use:N \__int_eval:w
      \if_meaning:w 2 #4
        \c_seven - \__int_eval:w
      \fi:
      \if_int_compare:w
          \@@_ep_compare:wwww #2,#3; #5,#6; > \c_zero
        \c_three -
        \exp_after:wN \@@_reverse_args:Nww
      \fi:
      \@@_atan_div:wnwwnw #2,#3; #5,#6;
  }
%    \end{macrocode}
% \end{macro}
%
% \begin{macro}[aux, rEXP]{\@@_atan_div:wnwwnw, \@@_atan_near:wwwn}
% \begin{macro}[aux, EXP]{\@@_atan_near_aux:wwn}
%   This receives two positive numbers $a$ and~$b$ (equal to $\lvert
%   x\rvert$ and~$\lvert y\rvert$ in some order), each as an exponent
%   and $6$~blocks of $4$~digits, such that $0<a<b$.  If $0.41421b<a$,
%   the two numbers are ``near'', hence the point $(y,x)$ that we
%   started with is closer to the diagonals $\{\lvert y\rvert = \lvert
%   x\rvert\}$ than to the axes $\{xy = 0\}$.  In that case, the octant
%   is~$1$ (possibly combined with the $7-$ and $3-$ inserted earlier)
%   and we wish to compute $\operatorname{atan}\frac{b-a}{a+b}$.  Otherwise, the
%   octant is~$0$ (again, combined with earlier terms) and we wish to
%   compute $\operatorname{atan}\frac{a}{b}$.  In any case, call \cs{@@_atan_auxi:ww}
%   followed by~$z$, as a comma-delimited exponent and a fixed point
%   number.
%    \begin{macrocode}
\cs_new:Npn \@@_atan_div:wnwwnw #1,#2#3; #4,#5#6;
  {
    \if_int_compare:w
      \__int_eval:w 41421 * #5 < #2 000
        \if_case:w \__int_eval:w #4 - #1 \__int_eval_end: 00 \or: 0 \fi:
      \exp_stop_f:
      \exp_after:wN \@@_atan_near:wwwn
    \fi:
    \c_zero
    \@@_ep_div:wwwwn #1,{#2}#3; #4,{#5}#6;
    \@@_atan_auxi:ww
  }
\cs_new:Npn \@@_atan_near:wwwn
    \c_zero \@@_ep_div:wwwwn #1,#2; #3,
  {
    \c_one
    \@@_ep_to_fixed:wwn #1 - #3, #2;
    \@@_atan_near_aux:wwn
  }
\cs_new:Npn \@@_atan_near_aux:wwn #1; #2;
  {
    \@@_fixed_add:wwn #1; #2;
    { \@@_fixed_sub:wwn #2; #1; { \@@_ep_div:wwwwn 0, } 0, }
  }
%    \end{macrocode}
% \end{macro}
% \end{macro}
%
% \begin{macro}[aux, EXP]{\@@_atan_auxi:ww, \@@_atan_auxii:w}
%   Convert~$z$ from a representation as an exponent and a fixed point
%   number in $[0.01,1)$ to a fixed point number only, then set up the
%   call to \cs{@@_atan_Taylor_loop:www}, followed by the fixed point
%   representation of~$z$ and the old representation.
%    \begin{macrocode}
\cs_new:Npn \@@_atan_auxi:ww #1,#2;
  { \@@_ep_to_fixed:wwn #1,#2; \@@_atan_auxii:w #1,#2; }
\cs_new:Npn \@@_atan_auxii:w #1;
  {
    \@@_fixed_mul:wwn #1; #1;
    {
      \@@_atan_Taylor_loop:www 39 ;
        {0000}{0000}{0000}{0000}{0000}{0000} ;
    }
    ! #1;
  }
%    \end{macrocode}
% \end{macro}
%
% \begin{macro}[aux, EXP]
%   {\@@_atan_Taylor_loop:www, \@@_atan_Taylor_break:w}
%   We compute the series of $(\operatorname{atan} z)/z$.  A typical intermediate
%   stage has $|#1|=2k-1$, $|#2| =
%   \frac{1}{2k+1}-z^2(\frac{1}{2k+3}-z^2(\cdots-z^2\frac{1}{39}))$, and
%   $|#3|=z^2$.  To go to the next step $k\to k-1$, we compute
%   $\frac{1}{2k-1}$, then subtract from it $z^2$ times |#2|.  The loop
%   stops when $k=0$: then |#2| is $(\operatorname{atan} z)/z$, and there is a need to
%   clean up all the unnecessary data, end the integer expression
%   computing the octant with a semicolon, and leave the result~|#2|
%   afterwards.
%    \begin{macrocode}
\cs_new:Npn \@@_atan_Taylor_loop:www #1; #2; #3;
  {
    \if_int_compare:w #1 = \c_minus_one
      \@@_atan_Taylor_break:w
    \fi:
    \exp_after:wN \@@_fixed_div_int:wwN \c_@@_one_fixed_tl ; #1;
    \@@_rrot:www \@@_fixed_mul_sub_back:wwwn #2; #3;
    {
      \exp_after:wN \@@_atan_Taylor_loop:www
      \int_use:N \__int_eval:w #1 - \c_two ;
    }
    #3;
  }
\cs_new:Npn \@@_atan_Taylor_break:w
    \fi: #1 \@@_fixed_mul_sub_back:wwwn #2; #3 !
  { \fi: ; #2 ; }
%    \end{macrocode}
% \end{macro}
%
% \begin{macro}[aux, EXP]
%   {\@@_atan_combine_o:NwwwwwN, \@@_atan_combine_aux:ww}
%   This receives a \meta{sign}, an \meta{octant}, a fixed point value
%   of $(\operatorname{atan} z)/z$, a fixed point number~$z$, and another
%   representation of~$z$, as an \meta{exponent} and the fixed point
%   number $10^{-\meta{exponent}} z$, followed by either \cs{use_i:nn}
%   (when working in radians) or \cs{use_ii:nn} (when working in
%   degrees).  The function computes the floating point result
%   \begin{equation}
%     \meta{sign} \left(
%       \left\lceil\frac{\meta{octant}}{2}\right\rceil
%       \frac{\pi}{4}
%       + (-1)^{\meta{octant}} \frac{\operatorname{atan} z}{z} \cdot z\right) \,,
%   \end{equation}
%   multiplied by $180/\pi$ if working in degrees, and using in any case
%   the most appropriate representation of~$z$.  The floating point
%   result is passed to \cs{@@_sanitize:Nw}, which checks for overflow
%   or underflow.  If the octant is~$0$, leave the exponent~|#5| for
%   \cs{@@_sanitize:Nw}, and multiply $|#3|=\frac{\operatorname{atan} z}{z}$
%   with~|#6|, the adjusted~$z$.  Otherwise, multiply $|#3|=\frac{\operatorname{atan}
%     z}{z}$ with $|#4|=z$, then compute the appropriate multiple of
%   $\frac{\pi}{4}$ and add or subtract the product $|#3|\cdot|#4|$.  In
%   both cases, convert to a floating point with
%   \cs{@@_fixed_to_float:wN}.
%    \begin{macrocode}
\cs_new:Npn \@@_atan_combine_o:NwwwwwN #1 #2; #3; #4; #5,#6; #7
  {
    \exp_after:wN \@@_sanitize:Nw
    \exp_after:wN #1
    \int_use:N \__int_eval:w
      \if_meaning:w 0 #2
        \exp_after:wN \use_i:nn
      \else:
        \exp_after:wN \use_ii:nn
      \fi:
      { #5 \@@_fixed_mul:wwn #3; #6; }
      {
        \@@_fixed_mul:wwn #3; #4;
        {
          \exp_after:wN \@@_atan_combine_aux:ww
          \int_use:N \__int_eval:w #2 / \c_two ; #2;
        }
      }
      { #7 \@@_fixed_to_float:wN \@@_fixed_to_float_rad:wN }
      #1
  }
\cs_new:Npn \@@_atan_combine_aux:ww #1; #2;
  {
    \@@_fixed_mul_short:wwn
      {7853}{9816}{3397}{4483}{0961}{5661};
      {#1}{0000}{0000};
    {
      \if_int_odd:w #2 \exp_stop_f:
        \exp_after:wN \@@_fixed_sub:wwn
      \else:
        \exp_after:wN \@@_fixed_add:wwn
      \fi:
    }
  }
%    \end{macrocode}
% \end{macro}
%
% \subsubsection{Arcsine and arccosine}
%
% \begin{macro}[int, EXP]{\@@_asin_o:w}
%   Again, the first argument provided by \pkg{l3fp-parse} is
%   \cs{use_i:nn} if we are to work in radians and \cs{use_ii:nn} for
%   degrees.  Then comes a floating point number.  The arcsine of $\pm
%   0$ or \nan{} is the same floating point number.  The arcsine of
%   $\pm\infty$ raises an invalid operation exception.  Otherwise, call
%   an auxiliary common with \cs{@@_acos_o:w}, feeding it information
%   about what function is being performed (for ``invalid operation''
%   exceptions).
%    \begin{macrocode}
\cs_new:Npn \@@_asin_o:w #1 \s_@@ \@@_chk:w #2#3; @
  {
    \if_case:w #2 \exp_stop_f:
      \@@_case_return_same_o:w
    \or:
      \@@_case_use:nw
        { \@@_asin_normal_o:NfwNnnnnw #1 { #1 { asin } { asind } } }
    \or:
      \@@_case_use:nw
        { \@@_invalid_operation_o:fw { #1 { asin } { asind } } }
    \else:
      \@@_case_return_same_o:w
    \fi:
    \s_@@ \@@_chk:w #2 #3;
  }
%    \end{macrocode}
% \end{macro}
%
% \begin{macro}[int, EXP]{\@@_acos_o:w}
%   The arccosine of $\pm 0$ is $\pi / 2$ (in degrees,~$90$).  The
%   arccosine of $\pm\infty$ raises an invalid operation exception.  The
%   arccosine of \nan{} is itself.  Otherwise, call an auxiliary common
%   with \cs{@@_sin_o:w}, informing it that it was called by
%   \texttt{acos} or \texttt{acosd}, and preparing to swap some
%   arguments down the line.
%    \begin{macrocode}
\cs_new:Npn \@@_acos_o:w #1 \s_@@ \@@_chk:w #2#3; @
  {
    \if_case:w #2 \exp_stop_f:
      \@@_case_use:nw { \@@_atan_inf_o:NNNw #1 0 \c_four }
    \or:
      \@@_case_use:nw
        {
          \@@_asin_normal_o:NfwNnnnnw #1 { #1 { acos } { acosd } }
            \@@_reverse_args:Nww
        }
    \or:
      \@@_case_use:nw
        { \@@_invalid_operation_o:fw { #1 { acos } { acosd } } }
    \else:
      \@@_case_return_same_o:w
    \fi:
    \s_@@ \@@_chk:w #2 #3;
  }
%    \end{macrocode}
% \end{macro}
%
% \begin{macro}[aux, EXP]{\@@_asin_normal_o:NfwNnnnnw}
%   If the exponent~|#5| is strictly less than~$1$, the operand lies
%   within $(-1,1)$ and the operation is permitted: call
%   \cs{@@_asin_auxi_o:nNww} with the appropriate arguments.  If the
%   number is exactly~$\pm 1$ (the test works because we know that
%   $|#5|\geq 1$, $|#6#7|\geq 10000000$, $|#8#9|\geq 0$, with equality
%   only for $\pm 1$), we also call \cs{@@_asin_auxi_o:nNww}.
%   Otherwise, \cs{@@_use_i:ww} gets rid of the \texttt{asin} auxiliary,
%   and raises instead an invalid operation, because the operand is
%   outside the domain of arcsine or arccosine.
%    \begin{macrocode}
\cs_new:Npn \@@_asin_normal_o:NfwNnnnnw
    #1#2#3 \s_@@ \@@_chk:w 1#4#5#6#7#8#9;
  {
    \if_int_compare:w #5 < \c_one
      \exp_after:wN \@@_use_none_until_s:w
    \fi:
    \if_int_compare:w \__int_eval:w #5 + #6#7 + #8#9 = 1000 0001 ~
      \exp_after:wN \@@_use_none_until_s:w
    \fi:
    \@@_use_i:ww
    \@@_invalid_operation_o:fw {#2}
      \s_@@ \@@_chk:w 1#4{#5}{#6}{#7}{#8}{#9};
    \@@_asin_auxi_o:NnNww
      #1 {#3} #4 #5,{#6}{#7}{#8}{#9}{0000}{0000};
  }
%    \end{macrocode}
% \end{macro}
%
% \begin{macro}[aux, EXP]{\@@_asin_auxi_o:NnNww, \@@_asin_isqrt:wn}
%   We compute $x/\sqrt{1-x^2}$.  This function is used by \texttt{asin}
%   and \texttt{acos}, but also by \texttt{acsc} and \texttt{asec} after
%   inverting the operand, thus it must manipulate extended-precision
%   numbers.  First evaluate $1-x^2$ as $(1+x)(1-x)$: this behaves
%   better near~$x=1$.  We do the addition/subtraction with fixed point
%   numbers (they are not implemented for extended-precision floats),
%   but go back to extended-precision floats to multiply and compute the
%   inverse square root $1/\sqrt{1-x^2}$.  Finally, multiply by the
%   (positive) extended-precision float $\lvert x\rvert$, and feed the
%   (signed) result, and the number~$+1$, as arguments to the arctangent
%   function.  When computing the arccosine, the arguments
%   $x/\sqrt{1-x^2}$ and~$+1$ are swapped by~|#2|
%   (\cs{@@_reverse_args:Nww} in that case) before
%   \cs{@@_atan_test_o:NwwNwwN} is evaluated.  Note that the arctangent
%   function requires normalized arguments, hence the need for
%   \texttt{ep_to_ep} and \texttt{continue} after \texttt{ep_mul}.
%    \begin{macrocode}
\cs_new:Npn \@@_asin_auxi_o:NnNww #1#2#3#4,#5;
  {
    \@@_ep_to_fixed:wwn #4,#5;
    \@@_asin_isqrt:wn
    \@@_ep_mul:wwwwn #4,#5;
    \@@_ep_to_ep:wwN
    \@@_fixed_continue:wn
    { #2 \@@_atan_test_o:NwwNwwN #3 }
    0 1,{1000}{0000}{0000}{0000}{0000}{0000}; #1
  }
\cs_new:Npn \@@_asin_isqrt:wn #1;
  {
    \exp_after:wN \@@_fixed_sub:wwn \c_@@_one_fixed_tl ; #1;
    {
      \@@_fixed_add_one:wN #1;
      \@@_fixed_continue:wn { \@@_ep_mul:wwwwn 0, } 0,
    }
    \@@_ep_isqrt:wwn
  }
%    \end{macrocode}
% \end{macro}
%
% \subsubsection{Arccosecant and arcsecant}
%
% \begin{macro}[int, EXP]{\@@_acsc_o:w}
%   Cases are mostly labelled by~|#2|, except when |#2| is~$2$: then we
%   use |#3#2|, which is $02=2$ when the number is $+\infty$ and
%   $22$~when the number is $-\infty$.  The arccosecant of $\pm 0$
%   raises an invalid operation exception.  The arccosecant of
%   $\pm\infty$ is $\pm 0$ with the same sign.  The arcosecant of \nan{}
%   is itself.  Otherwise, \cs{@@_acsc_normal_o:NfwNnw} does some more
%   tests, keeping the function name (\texttt{acsc} or \texttt{acscd})
%   as an argument for invalid operation exceptions.
%    \begin{macrocode}
\cs_new:Npn \@@_acsc_o:w #1 \s_@@ \@@_chk:w #2#3#4; @
  {
    \if_case:w \if_meaning:w 2 #2 #3 \fi: #2 \exp_stop_f:
           \@@_case_use:nw
             { \@@_invalid_operation_o:fw { #1 { acsc } { acscd } } }
    \or:   \@@_case_use:nw
             { \@@_acsc_normal_o:NfwNnw #1 { #1 { acsc } { acscd } } }
    \or:   \@@_case_return_o:Nw \c_zero_fp
    \or:   \@@_case_return_same_o:w
    \else: \@@_case_return_o:Nw \c_minus_zero_fp
    \fi:
    \s_@@ \@@_chk:w #2 #3 #4;
  }
%    \end{macrocode}
% \end{macro}
%
% \begin{macro}[int, EXP]{\@@_asec_o:w}
%   The arcsecant of $\pm 0$ raises an invalid operation exception.  The
%   arcsecant of $\pm\infty$ is $\pi / 2$ (in degrees,~$90$).  The
%   arcosecant of \nan{} is itself.  Otherwise, do some more tests,
%   keeping the function name \texttt{asec} (or \texttt{asecd}) as an
%   argument for invalid operation exceptions, and a
%   \cs{@@_reverse_args:Nww} following precisely that appearing in
%   \cs{@@_acos_o:w}.
%    \begin{macrocode}
\cs_new:Npn \@@_asec_o:w #1 \s_@@ \@@_chk:w #2#3; @
  {
    \if_case:w #2 \exp_stop_f:
      \@@_case_use:nw
        { \@@_invalid_operation_o:fw { #1 { asec } { asecd } } }
    \or:
      \@@_case_use:nw
        {
          \@@_acsc_normal_o:NfwNnw #1 { #1 { asec } { asecd } }
            \@@_reverse_args:Nww
        }
    \or:   \@@_case_use:nw { \@@_atan_inf_o:NNNw #1 0 \c_four }
    \else: \@@_case_return_same_o:w
    \fi:
    \s_@@ \@@_chk:w #2 #3;
  }
%    \end{macrocode}
% \end{macro}
%
% \begin{macro}[aux, EXP]{\@@_acsc_normal_o:NfwNnw}
%   If the exponent is non-positive, the operand is less than~$1$ in
%   absolute value, which is always an invalid operation: complain.
%   Otherwise, compute the inverse of the operand, and feed it to
%   \cs{@@_asin_auxi_o:nNww} (with all the appropriate arguments).  This
%   computes what we want thanks to
%   $\operatorname{acsc}(x)=\operatorname{asin}(1/x)$ and
%   $\operatorname{asec}(x)=\operatorname{acos}(1/x)$.
%    \begin{macrocode}
\cs_new:Npn \@@_acsc_normal_o:NfwNnw #1#2#3 \s_@@ \@@_chk:w 1#4#5#6;
  {
    \int_compare:nNnTF {#5} < \c_one
      {
        \@@_invalid_operation_o:fw {#2}
          \s_@@ \@@_chk:w 1#4{#5}#6;
      }
      {
        \@@_ep_div:wwwwn
          1,{1000}{0000}{0000}{0000}{0000}{0000};
          #5,#6{0000}{0000};
        { \@@_asin_auxi_o:NnNww #1 {#3} #4 }
      }
  }
%    \end{macrocode}
% \end{macro}
%
%    \begin{macrocode}
%</initex|package>
%    \end{macrocode}
%
% \end{implementation}
%
% \PrintChanges
%
% \PrintIndex
