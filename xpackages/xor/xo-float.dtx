% \iffalse
%% File xo-float.dtx (C) Copyright 1999-2000 Frank Mittelbach, David Carlisle, Chris Rowley
%%                   (C) Copyright 2004-2009 Frank Mittelbach, LaTeX3 Project
%%
%% It may be distributed and/or modified under the conditions of the
%% LaTeX Project Public License (LPPL), either version 1.3c of this
%% license or (at your option) any later version.  The latest version
%% of this license is in the file
%%
%%    http://www.latex-project.org/lppl.txt
%%
%% This file is part of the ``xor bundle'' (The Work in LPPL)
%% and all files in that bundle must be distributed together.
%%
%% The released version of this bundle is available from CTAN.
%%
%% -----------------------------------------------------------------------
%%
%% The development version of the bundle can be found at
%%
%%    http://www.latex-project.org/svnroot/experimental/trunk/
%%
%% for those people who are interested.
%%
%%%%%%%%%%%
%% NOTE: %%
%%%%%%%%%%%
%%
%%   Snapshots taken from the repository represent work in progress and may
%%   not work or may contain conflicting material!  We therefore ask
%%   people _not_ to put them into distributions, archives, etc. without
%%   prior consultation with the LaTeX Project Team.
%%
%% -----------------------------------------------------------------------
%%
\RequirePackage{l3names}
\GetIdInfo $Id$
          {xo-float (part of xor)}
\ProvidesExplPackage{\filename}
  {\filedate}{\fileversion}{\filedescription}
% \fi
%
%
%  \begin{macro}{\l_xor_curr_float_sequence_class_tl}
%    \begin{macrocode}
\tl_new:N\l_xor_curr_float_sequence_class_tl
%    \end{macrocode}
%  \end{macro}
%
%  \begin{macro}{\l_xor_curr_float_type_tl}
%    \begin{macrocode}
\tl_new:N\l_xor_curr_float_type_tl
%    \end{macrocode}
%  \end{macro}
%
%
% \begin{macro}{\@xfloat}
% The start of a float environment.
% Unlike \LaTeXe, now save the float as a horizontal list.
%    \begin{macrocode}
\cs_set_nopar:Npn \@xfloat #1[#2]{%
%<*trace>
\trace_push:n{float~environment}
%</trace>
  \@nodocument
% #2 is either user supplied or |\g_xor_type_#1_allowed_areas_tl|
  \tl_set:Nn \l_xor_this_allowed_areas_tl {#2}
  \tl_set:Nn \l_xor_curr_float_type_tl {#1}
%    \end{macrocode}
%    Next line could be inline since it is used only once
%    now.\footnote{Change?}
%    \begin{macrocode}
  \cs_set_eq:Nc \l_xor_curr_float_sequence_class_tl {g_xor_type_ #1 _class_tl}
%    \end{macrocode}
%
%    \begin{macrocode}
  \cs_set_eq:Nc \float@counter {g_xor_type_ #1  _counter_tl}
%    \end{macrocode}
%    Floats (not caption) increment counter:
%    \begin{macrocode}
  \refstepcounter\float@counter
  \mode_if_horizontal:T { \@bsphack }
  \mode_if_inner:TF
%FMi penalty leftover?
  {   \@parmoderr\@floatpenalty\c_zero }
  {
    \@allocating@next \g_xor_curr_float_box_tl \g_xor_floats_free_seq
%      {
      \int_gincr:N \g_xor_flseq_int
      \exp_args:NNo
        \PutMark\l_xor_curr_float_sequence_class_tl{\int_use:N \g_xor_flseq_int}
      \global\toks \g_xor_curr_float_box_tl {}
%      }
  }
%    \end{macrocode}
%    
%  Save |\g_xor_curr_float_box_tl| as an |\hbox|.
%    \begin{macrocode}
  \tl_gset_eq:NN \g_saved_label_tl \c_empty_tl
  \hbox_gset_inline_begin:N \g_xor_curr_float_box_tl
%    \end{macrocode}
%    The float body is typeset in a normalised environment, i.e., we
%    reset colour and font. This is followed by a call to
%    |\g_xor_type_TYPE_body_tl| to allow the designer to set up special
%    conventions for each float type, e.g., a special font or font
%    size or a special colour, etc.
%    \begin{macrocode}
      \color@begingroup
       \normalcolor
       \normalfont
       \normalsize
       \use:c {g_xor_type_ \l_xor_curr_float_type_tl _body_tl}
%    \end{macrocode}
%    And inside floats we don't want any alignment points since they
%    will never be seen in the galley---this may not be enough as
%    something inside might turn them on again so this needs further
%    thought\footnote{check/FIX}
%    \begin{macrocode}
       \IgnoreAlignToGrid
       \ignorespaces}
%    \end{macrocode}
% \end{macro}
%
% \begin{macro}{\g_saved_label_tl}
%    \begin{macrocode}
\tl_new:N \g_saved_label_tl
%    \end{macrocode}
% \end{macro}
%
% \begin{macro}{\@endfloatbox}
% The end of a float environment.
% First finish off the box assignment. Then make a copy in a second box.
% The current code makes use of the fact that the allocated boxes
% are in two continuous runs, |\sbx@offset| apart.
%    \begin{macrocode}
\cs_set_nopar:Npn  \@endfloatbox{%
  \unskip
      \outer@nobreak
  \color@endgroup
  \hbox_gset_inline_end:
%    \end{macrocode}
%    Next line calculates the span count for this float. This should change to
%    simply store the width of the float and calculate the span count when the
%    float is placed since the current column width may change over time. For
%    another rainy day.\footnote{fix}
%    \begin{macrocode}
  \num_set:Nn \SPANCNT {1}
  \dim_set:Nn \l_tmpa_dim {\wd \g_xor_curr_float_box_tl }
  \dim_sub:Nn \l_tmpa_dim \columnwidth 

  \dim_while_do:nNnn \l_tmpa_dim > \c_zero_dim
    {
     \num_incr:N \SPANCNT
     \dim_sub:Nn \l_tmpa_dim \columnwidth
     \dim_sub:Nn \l_tmpa_dim \columnsep 
    }
  \intexpr_compare:nNnT \SPANCNT > \g_xor_cols_int
    {
%<*trace>
     \trace:n{WARNING:~ float~ spans~ more~ cols~ than~ available:~ 
              (\SPANCNT > \int_use:N\g_xor_cols_int)}
%</trace>
     \num_set:Nn \SPANCNT {\int_use:N\g_xor_cols_int}
    }
%    \end{macrocode}
%    
%    \begin{macrocode}
  \edef\@tempa{
    \global\toks\g_xor_curr_float_box_tl{
      {\the\toks\g_xor_curr_float_box_tl}{\SPANCNT}
      {\g_saved_label_tl}
      {\csname the\float@counter \endcsname}
      {\l_xor_this_allowed_areas_tl}
      {\l_xor_curr_float_type_tl}
      {\int_use:N \g_xor_flseq_int}
    }}
 \@tempa
%<*trace>
\trace:n{\the\toks\g_xor_curr_float_box_tl}
%</trace>
%    \end{macrocode}
%    Simple temporary interface to get here floats for testing. Needs
%    replacment!\footnote{FIX!!!}
%    \begin{macrocode}
   \tl_set:Nn \l_tmpa_tl {h}
   \tl_if_eq:NNTF \l_xor_this_allowed_areas_tl \l_tmpa_tl
     \xor_setup_here_float:
     { \seq_gput_right:No \g_xor_floats_active_seq \g_xor_curr_float_box_tl }
%<*trace>
\trace_pop:n{float~environment}
%</trace>
}
%    \end{macrocode}
% \end{macro}
%

% \begin{macro}{\SPANCNT}
%    Missing interface\ldots\ right now we have to manually set the
%    span count if we want floats spanning more than one column
%    \begin{macrocode}
\cs_set_nopar:Npn \SPANCNT{1} %%% do properly one day
%    \end{macrocode}
% \end{macro}
%
% \begin{macro}{\g_xor_flseq_int}
%    |\g_xor_flseq_int|  is the counter used to uniquely  refer
%    to the floats, it is incremented whenever we encounter a new float.
%    \begin{macrocode}
\int_new:N \g_xor_flseq_int
%    \end{macrocode}
% \end{macro}
%
% \begin{macro}{\label}
%    Modify to save the label string.\footnote{Make this work with
%    \textsf{xlang} and the new cross reference model. MH}
%    \begin{macrocode}
\cs_set_nopar:Npn \label#1{\@bsphack
  \tl_gset:Nn \g_saved_label_tl {#1}
  \protected@write\@auxout{}%
         {\string\newlabel{#1}{{\@currentlabel}{\thepage}}}%
  \@esphack}
%    \end{macrocode}
% \end{macro}
%
%
%
%
%
%
% \begin{macro}{\DeclareFloatType}
%    A float type is an identifier like ``figure'' the defines a class
%    (?) of floats of floats which have certain attributes in common,
%    e.g., they have the same fixed caption text (e.g. \texttt{Fig}),
%    they use the same counter to number the floats, etc.
%
%    Setting up a float type is done by declaring an instance of type
%    \texttt{floattypesetup} and then executing it.
%    \begin{macrocode}
\cs_new_nopar:Npn \DeclareFloatType#1#2{
 \DeclareInstance{floattypesetup}{#1}{std}
   { type-id = #1,  #2 }
 \UseInstance{floattypesetup}{#1}
}
%    \end{macrocode}
% \end{macro}
%
%    For each float type \texttt{TYPE} the following utility macros
%    hold important static data.
%
% \DescribeMacro{\g_xor_type_TYPE_class_tl}
%    The |\g_xor_type_TYPE_class_tl| holds the name of the float sequence
%    class the float type |TYPE| belongs to. A sequence class contains
%    all types which should be placed in order of their call-outs,
%    e.g., if type ``figure'' and ``table'' are in the same sequence
%    class a table whose call-out is later than that of a figure will
%    not be placed before it. If on the other hand these types are in
%    different sequence classes such reordering will be undertaken by
%    the placement algorithm if feasible.
%
%
% \DescribeMacro\g_xor_type_TYPE_extension_tl
%    The |\g_xor_type_TYPE_extension_tl| holds the file extension to which
%    caption information for this type is written, e.g., |lot| for
%    tables, or |lof| for figures. If there is the desire to produce
%    combined listing one can use the same extension for different
%    float types.\footnote{This functionality can alternatively be
%    provided by putting all captions into a single external file and
%    then ignore certain entries when reading the data back, so it
%    might vanish again.}
%
%
% \DescribeMacro{\g_xor_type_TYPE_allowed_areas_tl}
%    The |\g_xor_type_TYPE_allowed_areas_tl| holds the default float areas in which this float
%    type is allowed to go. By default this is the full list of all
%    known areas. By restricting it to a subset of areas it is
%    possible to provide special placement arrangements, e.g., tables
%    always at the bottom only.
%
%
% \DescribeMacro{\g_xor_type_TYPE_caption_tl}
%    The |\g_xor_type_TYPE_caption_tl| holds the fixed caption text which is
%    displayed together with the caption number, e.g., |Figure|,
%    etc. This is passed to the caption formatting instances.
%
%
% \DescribeMacro{\g_xor_type_TYPE_counter_tl}
%    The |\g_xor_type_TYPE_counter_tl| holds the name of the counter to use for
%    numbering the floats of type |TYPE|. It is possible that
%    different types share the same counter, i.e., are numbered as a
%    single sequence. However in that case one should probably ensure
%    that the types belong to a single float sequence class since
%    otherwise the numbers within the document might get out of
%    sequence. On the other hand it is not required to have floats in
%    the same sequence class share a counter; they can be numbered
%    independently without problems.
%
% \DescribeMacro{\g_xor_type_TYPE_body_tl}
%    The |\g_xor_type_TYPE_body_tl| holds the formatting instructions that should be
%    applied to the body of each float of type |TYPE|.
%
%    \begin{macrocode}
\DeclareTemplateType{floattypesetup}{0}

\DeclareTemplate{floattypesetup}{std}{0}{
   type-id            =n  \l_floattypesetup_id_tl,
   sequence-class-id  =n  \l_floattypesetup_class_id_tl,
   toc-extension      =n  \l_floattypesetup_toc_ext_tl,
   default-area-list  =n  \l_floattypesetup_area_clist,
   caption-text       =n  \l_floattypesetup_caption_text_tl,
   numbered-boolean   =b  @test,
   numbered-id        =n  \l_floattypesetup_numbered_id_tl,
   numbered-within-id =n  \l_floattypesetup_numbered_within_id_tl,
   numbered-action    =f0 \floattypesetup_numbered_action:,
   body-decls         =f0 \floattypesetup_body_decls:,

 }
 {
%    \end{macrocode}
%    Setting up defaults is done (at least in parts) in two steps:
%    first we set certain key commands to |\relax| then we run
%    |\DoParameterAssignments| and then check if they are still
%    |\relax|. If so we set the commands that we really want to
%    set. This is necessary because several of them have names that
%    contain the float type as part of the name and this type name is
%    one of the keys.
%    \begin{macrocode}
    \TP_unset_key:N \l_floattypesetup_class_id_tl 
    \TP_unset_key:N \l_floattypesetup_numbered_id_tl
    \TP_unset_key:N \floattypesetup_numbered_action:
    \TP_unset_key:N \floattypesetup_body_decls:
%    \end{macrocode}
%    Now for the defaults we can set directly:
%    the |@test| switch is locally used to denote that we want to
%    number the floats.
%    \begin{macrocode}
    \@testtrue % number by default

    \clist_clear:N \l_floattypesetup_area_clist 
%    \end{macrocode}
%    Default for the fixed caption text is empty.
%    \begin{macrocode}
    \tl_clear:N \l_floattypesetup_caption_text_tl
%    \end{macrocode}
%    By default we don't run the float counter within another
%    counter. We use |\c_empty_tl| to denote this since this is also what
%    would be returned from |numbered-within-id=,| which might be
%    specified by a user.
%    \begin{macrocode}
    \tl_clear:N \l_floattypesetup_numbered_within_id_tl
%    \end{macrocode}
%    Everything goes by default to the |.toc| file (which isn't too
%    bad if the extended \texttt{xcontents} package is used).
%    \begin{macrocode}
    \tl_set:Nn \l_floattypesetup_toc_ext_tl {toc}  % everything in here by default
%    \end{macrocode}
%    Now we are ready to look at the designer specifications:
%    \begin{macrocode}
    \DoParameterAssignments
%    \end{macrocode}
%    Now for the second step: evaluate what we have: If the
%    |sequence-class-id| is not set up, we try to default it to the
%    |type-id|:
%    \begin{macrocode}
    \if_TP_key_unset:NT \l_floattypesetup_class_id_tl
    {
      \tl_set_eq:NN \l_floattypesetup_class_id_tl \l_floattypesetup_id_tl
    }
%    \end{macrocode}
%    Float numbers are only set up if requested (which is the default).
%    \begin{macrocode}
    \if@test
%    \end{macrocode}
%    If the |numbered-id| is not set up, we try to default it to
%    the |type-id|:\footnote{This needs fixing since part of the float
%    handling assumes there is some counter to step etc. FIX!}
%    \begin{macrocode}
      \if_TP_key_unset:NT \l_floattypesetup_numbered_id_tl
        {
         \tl_set_eq:NN \l_floattypesetup_numbered_id_tl 
                        \l_floattypesetup_id_tl
        }
%    \end{macrocode}
%    Set up the new counter, if it already exists skip this part:
%    \begin{macrocode}
      \cs_if_free:cT {c@\l_floattypesetup_numbered_id_tl}
      {
        \tl_if_empty:NTF \l_floattypesetup_numbered_within_id_tl
        {
          \newcounter\l_floattypesetup_numbered_id_tl
        }
        {
          \newcounter\l_floattypesetup_numbered_id_tl
                    [\l_floattypesetup_numbered_within_id_tl]
        }
%    \end{macrocode}
%    The default display for the float number is ``arabic'', we
%    overwrite this only if requested:
%    \begin{macrocode}
        \if_TP_key_unset:NF \floattypesetup_numbered_action:
        {
          \cs_gset_eq:cN {the\l_floattypesetup_numbered_id_tl}
                   \floattypesetup_numbered_action:
        }
      }
    \fi:
%    \end{macrocode}
%
%    Low-level environment declaration (allows overwriting since
%    figure etc already exists)\footnote{Handle better?}
%    \begin{macrocode}
    \cs_gset_nopar:cpx {\l_floattypesetup_id_tl}
          {\exp_not:N\@float{\l_floattypesetup_id_tl}}
    \cs_gset_eq:cN {end\l_floattypesetup_id_tl} \end@float
%    \end{macrocode}
%    The remaining bits of the code set up the data structures that
%    belong to each float type by linking the macro name containing
%    the type in its name (i.e., |\l_floattypesetup_id_tl|) to the
%    corresponding key macro. They are otherwise straight forward
%    |\cs_set_eq:NN|s.
%
%    First for |\g_xor_type_|\meta{type}|_class_tl|:
%    \begin{macrocode}
    \tl_gset_eq:cN
      {g_xor_type_ \l_floattypesetup_id_tl _class_tl}
      \l_floattypesetup_class_id_tl
%    \end{macrocode}
%    Then |\g_xor_type_TYPE_extension_tl|:
%    \begin{macrocode}
    \tl_gset_eq:cN {g_xor_type_ \l_floattypesetup_id_tl 
                     _extension_tl}
                    \l_floattypesetup_toc_ext_tl
%    \end{macrocode}
%    Then |\g_xor_type_|\meta{type}|_allowed_areas_tl|:
%
%    The areas which should be tried for the current type are by
%    default all areas, i.e., |\g_xor_areas_known_clist|; we have to
%    defer that to the beginning of the document since further areas
%    might become known.
%    to allow this list to grow afterwards.
%    \begin{macrocode}
    \clist_if_empty:NTF \l_floattypesetup_area_clist
    {
      \cs_set_nopar:Npx \@tempa{
      \exp_not:N\AtBeginDocument{
        \exp_not:N \clist_gset_eq:cN
          {g_xor_type_ \l_floattypesetup_id_tl _allowed_areas_tl}
          \exp_not:N \g_xor_areas_known_clist
%    \end{macrocode}
%    currently we misuse the old 2e|\fps@TYPE| interface so above is
%     really not that useful and we also have to set up:\footnote{FIX}
%    \begin{macrocode}
        \exp_not:N \tl_gset_eq:cN
          {fps@ \l_floattypesetup_id_tl}
          \exp_not:N \g_xor_areas_known_clist
        }
      }
      \@tempa
    }
    {
      \clist_gset_eq:cN
          {g_xor_type_ \l_floattypesetup_id_tl 
           _allowed_areas_tl}
          \l_floattypesetup_area_clist
      \clist_gset_eq:cN
          {fps@ \l_floattypesetup_id_tl} 
          \l_floattypesetup_area_clist
    }
%    \end{macrocode}
%    Then |\g_xor_type_TYPE_caption_tl|:
%    \begin{macrocode}
    \tl_gset_eq:cN {g_xor_type_ \l_floattypesetup_id_tl
                     _caption_tl}
                     \l_floattypesetup_caption_text_tl
%    \end{macrocode}
%    Then |\g_xor_type_TYPE_counter_tl|:
%    \begin{macrocode}
    \tl_gset_eq:cN {g_xor_type_ \l_floattypesetup_id_tl
                     _counter_tl}
                    \l_floattypesetup_numbered_id_tl
%    \end{macrocode}
%    Then |\g_xor_type_TYPE_body_tl|:
%    \begin{macrocode}
    \tl_gset_eq:cN {g_xor_type_ \l_floattypesetup_id_tl
                     _body_tl}
                    \floattypesetup_body_decls:
  }
%    \end{macrocode}
%
%
%
% \begin{macro}{\DeclareFloatSequenceClass}
%    A float sequence class is a list of float types for which the
%    placement algorithm should preserve call-out order even between
%    objects with different float type.
%
%    For each class we need a corresponding ``mark'' (needed by the
%    algorithm) and we need to record the class name in an |\@elt|
%    list.
%
%    Perhaps this declaration should be implicitly handled by
%    |\DeclareFloatType|. It would mean a bit more work for the latter
%    but less declarations in the document design!\footnote{Change one
%    day!}
%    \begin{macrocode}
\cs_new_nopar:Npn \DeclareFloatSequenceClass#1{
%    \end{macrocode}
%    We probably have to check if that class is already
%    declared since in that case we have to leave out a couple of
%    actions. We assume that this is the case if a mark class is
%    already defined for it.\footnote{Do it properly!}
%    \begin{macrocode}
  \cs_if_free:cTF{mark@#1}
    {
     \DeclareMarkType{#1}
%    \end{macrocode}
%    Add this to the list of known classes.
%    \begin{macrocode}
     \seq_gput_right:Nn \g_xor_float_classes_seq{#1}
    }
    \ErrorAlreadyDefined
}
%    \end{macrocode}
% \end{macro}
%
%
% \begin{macro}{\g_xor_float_classes_seq}
%    The list of float classes known in the system.
%    \begin{macrocode}
\seq_new:N \g_xor_float_classes_seq
%    \end{macrocode}
% \end{macro}
%
%
%
% \begin{macro}{\c@figure}
% \begin{macro}{\c@table}
%    Throw away good registers\ldots this is only for testing to allow
%    us using the OR together with classes like article that sets up
%    figure and table already\footnote{Clean up!}
%    \begin{macrocode}
\cs_if_free:cF{c@figure}
{
  \cs_gundefine:N \c@figure % thus new counter will be defined
  \cs_gundefine:N \c@table
}
%    \end{macrocode}
% \end{macro}
% \end{macro}
%
%
%
%
%
%
%
%
% \subsection{Float Placement Control Files}
%
% Code in this section is pretty old and so far only with minimal conversion
% to expl3 syntax. Most of it needs a complete rewrite at a later stage.
%
% \begin{macro}{\xor_perhaps_write_placements_to_fpl_file:}
%    |\xor_perhaps_write_placements_to_fpl_file:| writes all float
%    placements (except for ``here'' floats) into the |fpl| in case we
%    are writing to this file.
%
%    By default, don't write fpl files.
%    \begin{macrocode}
\cs_new_eq:NN \xor_perhaps_write_placements_to_fpl_file: \prg_do_nothing:
%    \end{macrocode}
% \end{macro}
%
% \begin{macro}{\xor_perhaps_write_to_fpl_file:n}
%    |\xor_perhaps_write_to_fpl_file:n| writes its first argument to the
%    |fpl| in case we are writing to it.
%
%    By default, don't write fpl files.
%    \begin{macrocode}
\cs_new_eq:NN \xor_perhaps_write_to_fpl_file:n \use_none:n
%    \end{macrocode}
% \end{macro}
%
% \begin{macro}{\xor_gen_float_placement_fpl_info:nnnnnnn}
%    The format for a Float: line in an |fpl| file.  Only the sequence
%    number (|#7|) is currently used, the rest just helps a human
%    reader identify the float.
%    \begin{macrocode}
\cs_new_nopar:Npn \xor_gen_float_placement_fpl_info:nnnnnnn#1#2#3#4#5#6#7{
  \@spaces\@spaces Float:~#7~(#6~#4)~[#3]^^J}
%    \end{macrocode}
% \end{macro}
%
% \begin{macro}{\savefloatplacements}
% Declaration to turn on writing of FPL files.
%    \begin{macrocode}
\cs_new_nopar:Npn \savefloatplacements{
  \newwrite\fpl@file
  \immediate\openout\fpl@file\jobname.fpl
  \cs_set_nopar:Npn \xor_perhaps_write_to_fpl_file:n{
     \immediate\write\fpl@file
     }
  \cs_set_nopar:Npn \xor_perhaps_write_placements_to_fpl_file:{
    \xor_perhaps_write_to_fpl_file:n{
    ^^JPage:~~
      \num_use:N \g_xor_page_absolute_num \space (\the\c@page)}
    \xor_perhaps_write_to_fpl_file:n{
      \@spaces Style:~~ \g_xor_used_pagestyle_tl }
%    \end{macrocode}
%
%    \begin{macrocode}
    \xor_forall_areas:n
    {
      \xor_perhaps_write_to_fpl_file:n{\@spaces Area:~ \curr@area \space - }
      \seq_map_inline:cn {g_xor_area_ \curr@area _float_seq}
      {
        \xor_perhaps_write_to_fpl_file:n
        { \exp_after:wN
          \xor_gen_float_placement_fpl_info:nnnnnnn
          \the\toks########1
        }% yes, all these are needed.
      }
    }
  }
}
%    \end{macrocode}
% \end{macro}
%
%
%
%
% \begin{macro}{\readfloatplacements}
% Similar declaration to read FPC files. The file is read at this
%  point up to the first Page: line. and the output routine command
% |\xor_trial_place_float_loop:| is redefined to read the FPC info for each page.
%    \begin{macrocode}
\cs_new_nopar:Npn \readfloatplacements{
  \newread\fpc@file
  \openin\fpc@file\jobname.fpc\scan_stop:
  \ifeof\fpc@file
    \PackageWarningNoLine{xor}
        {No~\jobname.fpc:~using~automatic~float~placement}
  \else:
    \cs_set_eq:NN \xor_maybe_get_fpc_page_data: \xor_get_fpc_line_and_recurse:
    \xor_maybe_get_fpc_page_data:

% fix next line

    \cs_gset_eq:NN \xor_try_float_pages: \prg_do_nothing:

%    \end{macrocode}
%    Next line should go once we implement moving floats in and out of
%    ``here'' as then we need to read the file at a different
%    place!\footnote{Reimplement!}
%    \begin{macrocode}
    \cs_gset_eq:NN \xor_trial_place_float_loop: \xor_fpc_place_float_loop:
  \fi:
}
%    \end{macrocode}
% \end{macro}
%
%
% \begin{macro}{\xor_fpc_place_float_loop:}
%    Version of |\xor_trial_place_float_loop:| to use with float placement
%    control file.  Should be moved to xo-or once it works.
%    \begin{macrocode}
\cs_new_nopar:Npn \xor_fpc_place_float_loop:{
%<*trace>
  \trace_push:n{xor_fpc_place_float_loop:}
%</trace>
%    \end{macrocode}
%
%    \begin{macrocode}
   \xor_get_fpc_line_and_recurse:
%    \end{macrocode}
%    Adding the floats changes the column heights but the values in
%    |\g_xor_best_trial_col_hts_tl| are not updated by this process so when
%    finally typesetting the pages we would retrieve the wrong values. So we
%    either have to update this tlp or (what is currently done) clear it
%    totally. The latter is probably okay as wehaven't done trials and we are
%    not backing up so \ldots{} cross fingers and \ldots
%
%    Eventually this need to change once we allow partially set up fpc
%    files.\footnote{Check/fix!}
%    \begin{macrocode}
   \tl_gclear:N \g_xor_best_trial_col_hts_tl
%    \end{macrocode}
%
%    Do we have to set |\if_xor_trial_with_floats:|?\footnote{Check!}
%    \begin{macrocode}
  \xmark_OR_restore_state:nn 
    {
%<*trace>
     \trace:n{copying~ g_xor_hold_page_box~ onto~ MVL}
%</trace>
     \vbox_unpack:N \g_xor_hold_page_box
     \xor_add_badly_guarded_penalty:N \c_xor_output_collect_penalty
    }
    \xor_OR_best_cols_setup:
%<*trace>
  \trace_pop:n{xor_fpc_place_float_loop:}
%</trace>
}
%    \end{macrocode}
% \end{macro}
%
% \begin{macro}{\g_xor_next_fpc_page_num}
% The next page for which there is FPC data. Initial value has to be
% a number to |\ifnum|.
%    \begin{macrocode}
\num_new:N \g_xor_next_fpc_page_num
\num_gset:Nn \g_xor_next_fpc_page_num {-1}
%    \end{macrocode}
% \end{macro}
%
%
% \begin{macro}{\xor_maybe_get_fpc_page_data:}
%
%    \begin{macrocode}
\cs_new_eq:NN \xor_maybe_get_fpc_page_data: \prg_do_nothing:
%    \end{macrocode}
% \end{macro}
%
%
% \begin{macro}{\xor_get_fpc_line_and_recurse:}
%    Get one line of data from an FPC file, calls itself recursively so the
%    effect of a call is to get all the lines for the current page up to and
%    including the Page: line for the next page for which there is data.
%    \begin{macrocode}
\cs_new_nopar:Npn \xor_get_fpc_line_and_recurse:{
%<*trace>
\trace_push:n{xor_get_fpc_line_and_recurse:}
%</trace>
  \cs_set_eq:NN \xor_fpc_next_task: \prg_do_nothing:
  \intexpr_compare:nNnTF \g_xor_next_fpc_page_num > \g_xor_page_absolute_num
  {
%<*trace>
    \trace:n{FPC:~No~ float~ control~ on~ page:~ \num_use:N \g_xor_page_absolute_num}
%</trace>
  }
  {
    \tex_ifeof:D\fpc@file
%<*trace>
      \trace:n{FPC:~ fpc~ file~ended}
%</trace>
      \num_gset:Nn \g_xor_next_fpc_page_num \c_max_int

      \cs_gset_eq:NN \xor_trial_place_float_loop: \xor_trial_place_float_loop_std:
      \cs_gset_eq:NN \xor_try_float_pages:        \xor_try_float_pages_std:

    \else:
      \group_begin:
        \tex_endlinechar:D `\ %
        \tex_catcode:D`\ 10\scan_stop:
        \pref_global:D\read\fpc@file~to~\g_tmpa_tl
      \group_end:
      \cs_set_eq:NN \xor_fpc_next_task:\xor_maybe_get_fpc_page_data:
%<*trace>
    \trace:n{FPC:~ fpc~ line~ read:~ \g_tmpa_tl}
%</trace>
      \tl_if_eq:NNF \g_tmpa_tl \c_empty_tl
      {
        \exp_after:wN \xor_parse_fpc_line:w \g_tmpa_tl \scan_stop:
      }
    \fi:
  }
%<*trace>
  \trace_pop:n{xor_get_fpc_line_and_recurse:}
%</trace>
  \xor_fpc_next_task:
}
%    \end{macrocode}
% \end{macro}
%
%
% \begin{macro}{\xor_parse_fpc_line:w}
%    If float control files are being read,
%    Read the section of the fpc file for this page
%    and directly assign floats to the specified areas
%    with no trials or checks.
%    \begin{macrocode}
\cs_new_nopar:Npn \xor_parse_fpc_line:w #1#2~#3~#4 \scan_stop:{
  \if_meaning:w P#1
%<*trace>
    \trace:n{FPC:~ Next~page~control:~#3}
%</trace>
    \num_gset:Nn \g_xor_next_fpc_page_num {#3}
    \cs_set_eq:NN \xor_fpc_next_task: \prg_do_nothing:
  \else:
    \if_meaning:w S#1
%<*trace>
      \trace:n{FPC:~ Style:~#3}
%</trace>
      \tl_gset:Nn \g_xor_fpc_pagestyle_tl {#3}
      \cs_set_eq:NN \xor_fpc_next_task:\xor_maybe_get_fpc_page_data:
    \else:
    \if_meaning:w A#1
%<*trace>
      \trace:n{FPC:~ Area:~#3}
%</trace>
      \xor_this_area_setup:o{#3}
      \cs_set_eq:NN \xor_fpc_next_task:\xor_maybe_get_fpc_page_data:
    \else:
      \if_meaning:w F#1
        \if_meaning:w \g_xor_this_area_name_tl\c_fpc_here_area_name_tl
%    \end{macrocode}
%    Should allow moving floats into and out of here status means
%    reading before the collecton OR, not after.\footnote{FIX, update,
%    extend --- whatever you want to call it.}
%    \begin{macrocode}
%<*trace>
          \trace:n{FPC:~ here~float}
%</trace>
        \else:
%<*trace>
          \trace:n{FPC:~ not~here~float}
%</trace>
          \xor_fpc_find_and_remove_float_from_active:n {#3}
%    \end{macrocode}
%    If a float number is specified and there is not a float of that
%    number in the active list, something has gone wrong, either we
%    haven't reached the float yet (possible if the new arrangement
%    results in the callout being on a later page than specified) or
%    we have already assigned this float (which is a bug, either in
%    this code, or in the fpc file.) Currently this condition is not
%    handled other than by |\ERROR|. If the float is found, assign it
%    to the current area (as set by an Area: line in the fpc file.
%    \begin{macrocode}
          \tl_if_eq:NNTF
             \g_xor_floats_active_seq \g_xor_floats_active_new_seq
           {
%<*trace>
            \trace:n{FPC:~ Float:~#3~not~found}
%</trace>
            \cs_set_nopar:Npx \fpc@list{\fpc@list#3,\g_xor_this_area_name_tl\scan_stop:}
            \cs_show:N \fpc@list
           }
           {
%<*trace>
            \trace:n{FPC:~ Float:~#3~is~\g_xor_this_float_box_tl}
%</trace>
            \seq_gset_eq:NN \g_xor_floats_active_seq \g_xor_floats_active_new_seq
%    \end{macrocode}
%    Before we add the float, let's do some sanity checks. We allow to
%    place a float into an area which is bigger than the float (spans
%    more columns) but we complain if the area is smaller.
%    \begin{macrocode}
            \intexpr_compare:nNnT \g_xor_this_area_span_tl <
                              \g_xor_this_span_num_tl
              {
               \PackageError{xo}
                {Float~ bigger~ than~ target~ area}
                {Float~\g_xor_this_flseq_num\space spans ~
                  \g_xor_this_span_num_tl\space columns,~
                  but~ target~ area~ \g_xor_this_area_name_tl\space
                  spans~ only~ \g_xor_this_area_span_tl\space
                  columns.\MessageBreak
                  Correct~ the~ data~ in~ file~ \jobname.fpc!
                }
              }
%    \end{macrocode}
%    We also check if the target area is allowed on the current page
%    (it could be the case that the page setup for that page doesn't
%    list it in its |area-list| key!
%    \begin{macrocode}
            \clist_if_in:NoTF
                \g_xor_areas_used_clist
                \g_xor_this_area_name_tl
              {}
              {
                \PackageError{xo}
                  {Target~ area~ not~ available~ on~ current~ page}
                  {Float~ \g_xor_this_flseq_num\space was~ requested~ to~
                    be~ placed~ into~ area~ \g_xor_this_area_name_tl,\MessageBreak
                    but~ this~
                    area~ is~ not~ available~ on~ page~ \thepage.
                    \MessageBreak
                    Correct~ the~ data~ in~ file~ \jobname.fpc!
                  }
%    \end{macrocode}
%    If we can't put it into the requested area we have to put it
%    somewhere else. Otherwise a lot of code might break. So we simply
%    chose the first area in |\g_xor_areas_used_clist|\footnote{Need a solution if
%    the page doesn't allow ``any'' floats! FIX!}
%    \begin{macrocode}
              \clist_if_empty:NTF \g_xor_areas_used_clist    % we have a problem
                \RESOLVE
                { \clist_get:NN\g_xor_areas_used_clist\l_tmpa_tl
                  \xor_this_area_setup:o\l_tmpa_tl
                }
            }
%    \end{macrocode}
%    Typeset and attach the caption and adjust the text size for the
%    current column.
%    \begin{macrocode}
            \xor_append_caption_to_float:
            \xor_construct_and_test_col_hts:
            \bool_if:NT \g_test_bool
                \ERRORFloatAreaToLarge
%    \end{macrocode}
%    We append the float to the area after the above column updates,
%    otherwise |\xor_construct_and_test_col_hts:| would incorrectly think
%    there is already a float present even if we add the first float
%    to an area.
%    \begin{macrocode}
%% big ?
%%            \seq_gput_right:No
%%                \g_xor_floats_free_seq \g_xor_this_float_box_tl

%            \seq_gput_right:co
%                {g_xor_area_ \g_xor_this_area_name_tl _float_seq}
%                \g_xor_this_float_box_tl
\xor_update_float_placement_data:
          }
        \fi:
%    \end{macrocode}
%
%    Loop to next line.
%    \begin{macrocode}
        \cs_set_eq:NN \xor_fpc_next_task: \xor_maybe_get_fpc_page_data:
%    \end{macrocode}
%    end of case on type of fpc line.
%    \begin{macrocode}
      \fi:
    \fi:
    \fi:
  \fi:
}
%    \end{macrocode}
% \end{macro}
%
%
\cs_new_nopar:Npn \xor_fpc_find_and_remove_float_from_active:n #1 {
   \group_begin:
    \seq_gclear:N \g_xor_floats_active_new_seq

    \cs_gset_eq:NN \g_xor_this_float_box_tl \scan_stop:
 
    \seq_map_inline:Nn \g_xor_floats_active_seq
	 {
	  \intexpr_compare:nNnTF
	     {\exp_after:wN\get@float@number\tex_the:D\toks ##1} = {#1}
             {
              \tl_gset:Nn \g_xor_curr_float_box_tl ##1
              \xor_extract_this_float_values:N ##1
             }
             { \seq_gput_right:Nn \g_xor_floats_active_new_seq ##1 }
         }
   \group_end:
}
\seq_new:N \g_xor_floats_active_new_seq

%
%
% \begin{macro}{\c_fpc_here_area_name_tl}
%
%    \begin{macrocode}
\tl_new:Nn\c_fpc_here_area_name_tl{hhh}
%    \end{macrocode}
% \end{macro}

% \begin{macro}{\fpc@list}
%
%    \begin{macrocode}
\cs_set_nopar:Npn \fpc@list{\relax}
%    \end{macrocode}
% \end{macro}
%
%
% \begin{macro}{\get@float@number}
% 
%    \begin{macrocode}
\cs_new_nopar:Npn \get@float@number#1#2#3#4#5#6#7{#7}
%    \end{macrocode}
% \end{macro}
%
%
%

%  \begin{macro}{\enddocument}
%    If we are setting floats via FPC files then it might happen that we
%    delete a float from the file by mistake rather than moving it. In that
%    case it will remain on the active list and never gets typeset unless
%    later in the document norming float placement trials are resumed.
%
%    To account for this situation (at least initially) we check that this
%    sequence is empty when we are ready to finally end the document. We don't
%    make it an error right now but probably it has to be one day.
%
%    \begin{macrocode}
\cs_set_nopar:Npn \enddocument{%
   \cs_set_eq:NN \AtEndDocument \@firstofone
   \@enddocumenthook
   \@checkend{document}%
   \clearpage

    \seq_if_empty:NF \g_xor_floats_active_seq
       {
        \typeout{Warning:~ the~ following~ float(s)~ have~ not~
                 been~ typeset:}
        \seq_map_inline:Nn \g_xor_floats_active_seq
           { \typeout{\@spaces ##1} }
       }

   \begingroup
     \if@filesw
       \immediate\closeout\@mainaux
       \cs_set_eq:NN \@setckpt\@gobbletwo
       \cs_set_eq:NN \@newl@bel\@testdef
       \@tempswafalse
       \makeatletter \input\jobname.aux
     \fi
     \@dofilelist
     \ifdim \font@submax >\fontsubfuzz\relax
       \@font@warning{Size substitutions with differences\MessageBreak
                  up to \font@submax\space have occurred.\@gobbletwo}%
     \fi
     \@defaultsubs
     \@refundefined
     \if@filesw
       \ifx \@multiplelabels \relax
         \if@tempswa
           \@latex@warning@no@line{Label(s)~ may~ have~ changed.~
               Rerun~ to~ get~ cross-references~ right}%
         \fi
       \else
         \@multiplelabels
       \fi
     \fi
   \endgroup
   \deadcycles\z@\@@end}
%    \end{macrocode}
%  \end{macro}


%
%
% \endinput
\endinput
%
