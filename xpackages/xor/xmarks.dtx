% \iffalse
%%
%% (C) Copyright 1999 Frank Mittelbach
%% All rights reserved.
%%
%% Not for general distribution. In its present form it is not allowed
%% to put this package onto CD or an archive without consulting the
%% the authors.
%% 
%    \begin{macrocode}
\def\next#1: #2.dtx,v #3 #4 #5 #6 #7$#8{
%<*dtx>
  \ProvidesFile{#2.dtx}
%</dtx>
%<package>\ProvidesPackage{#2}
%<driver>\ProvidesFile{#2.drv}
  [#4 #3 #8 (#6)]}
\next$Id$
       {multiple marks}
%    \end{macrocode}
%
%<*driver>
 \documentclass{ltxdoc}
%
 \begin{document}
 \DocInput{xmarks.dtx}
 \end{document}
%</driver>
%
% \fi
%
%
% \GetFileInfo{xmarks.dtx}
%
% \title{The \textsf{xmarks} package\thanks{This file
%         has version number \fileversion, last
%         revised \filedate.}}
% \author{FMi}
% \date{\filedate}
%
%  \maketitle
%
% \tableofcontents
%
%
%
% \section{Mark algorithm}
%
% To allow for completely independent marks we use the following
% procedure:
% \begin{itemize}
% \item
%   For every mark of type \meta{type} we have a queue that holds every mark
% info found in memory. Every info is associated with a unique number,
% i.e., we simply count them. The queue always holds at least one item
% which is the previous mark info, i.e., the last mark for that type on
% the previous page.
%
% \item
%   We use \TeX's internal mark mechanism only to record the associated
% numbers for marks for every type on the current page.
%
% \item
%  For every \meta{type} we record the info-number that belongs to the last
% mark of the previous page.
%
% \item
%  By looking at \TeX's internal |\botmark| we can determine the
% info-number of the last mark for every \meta{type} that is on the current
% page.
%
% \item
%   The difference between this number and the info-number from the last
% page gives us the number of marks for every \meta{type} on the current page.
%
% \item
%   That way we are able to decide how to extract items from the queue so
% that first, last, and previous mark for every type can be accessed.
%  \begin{itemize}
%  \item
%    If the difference is zero, no new marks have been added on the
% current page, therefore the only item on the corresponding mark type
% queue holds the info that should be previous, first, and last mark of
% this type for the current page.
%  \item
%    If the difference is one, then the first item in the queue represents
% the info that should become the previous mark (it will be popped off)
% and the second item (which will be kept, since it will become the
% previous mark for the next page) should become first and last mark for
% the current page.
%  \item
%    Otherwise we had at least two marks for this type on the current page.
% Therefore, the first item in the queue will become previous mark (popped
% off), the second will become first mark (popped off), then we pop off
% all but the last item in the queue, which will become last mark, by only
% looking at it.
%  \end{itemize}
%  In other words every queue should hold only one item after this part of
% the algorithm has acted.
% After this is done we save the info-numbers given by |\botmark|
% as the info-numbers of the last page, so that they are available next time.
%
% \item
%   Putting a mark into the galley therefore means to put its info to the
% right of the corresponding queue, increment the current info-number for
% \meta{type} by one and putting a \TeX{} mark into the galley holding the
% current info-numbers for all \meta{types}.
%
%  \end{itemize}
%
%
% \subsection{Restrictions}
%
% One restriction in using this algorithm without further refinements
% is that for every \meta{type} there will be an upper limit of marks allowed
% within one document. This upper limit is given by |\maxdimen| which is
% probably high enough to ignore it, but it could be improved by resetting
% the info numbers whenever all marks have found their way onto the
% current page.
%
% Another restriction is that one better not changes from linearily
% processing the data for the main galley, or, if one modifies that
% processing one needs to carefully adjust the data structures for the
% marks as well. The problem being that the individual marks within a
% preprocessed galley will stay there forever (and thus can be parsed
% several times), the external data structures, e.g., the plists and
% the sequences however need manual corrections in that case.
%
% \subsection{Implementation}
%
% The original implementation was done on a portable PC in a train
% sometime in 1992/93 and is just a straight implementation of the
% above algorithm using plists to hold the current and the last
% info-numbers. The whole stuff could be done much better by analysing
% the algorithm combining common parts (a lot of thing have be done
% twice, etc.) and recoding the rest in a better way.
%
% The implementation below is more or less a straight adaption
% of that code, so it still needs a rewrite one day. It is based on
% modules implementing ``queues'', ``property lists'', and ``quarks''
% most of which have been published as experimental code with a
% slightly different surface syntax, i.e., as \texttt{l3seq.sty},
% \texttt{l3prop.sty}, and \texttt{l3quark.sty}.
%
% New stuff:
%    \begin{macrocode}
\RequirePackage{ldcsetup}
\IgnoreWhiteSpace
%    \end{macrocode}
%
%    \begin{macrocode}
\long\def\@firstofthree#1#2#3{#1}
\long\def\@secondofthree#1#2#3{#2}
\long\def\@thirdofthree#1#2#3{#3}
%    \end{macrocode}
%
%    \begin{macrocode}
\def \tlp@to@str@N {\expandafter \tlp@to@str@aux \meaning}
\def \tlp@to@str@aux #1>{}
\def \tlp@to@str@c #1 {\expandafter \tlp@to@str@N \csname#1\endcsname}
%    \end{macrocode}
%
%
%
%
%
% \subsection{Commands for manipulating queues}
%
% \begin{macro}{\trace@queue}
% \begin{macro}{\trace@queue@internal}
% \begin{macro}{\tracingqueues}
%    \begin{macrocode}
%<*trace>
\def\trace@queue#1{\ifnum \tracingqueues > \z@
      \typeout{Queues:~ #1~ \on@line}\fi}
\def\trace@queue@internal#1{\ifnum\tracingqueues>\@ne
      \typeout{Queues:~ #1~ \on@line}\fi}
\newcount\tracingqueues
%</trace>
%    \end{macrocode}
% \end{macro}
% \end{macro}
% \end{macro}
%
% \begin{macro}{\queue@new@c}
%    \begin{macrocode}
\def\queue@new@c#1{\@namedef{#1}{}}
%    \end{macrocode}
% \end{macro}
%
% \begin{macro}{\queue@gadd@Nn}
%    \begin{macrocode}
\def\queue@gadd@Nn#1#2{\expandafter\gdef\expandafter#1\expandafter
    {#1\queue@elt#2\queue@eelt}
%<*trace>
  \trace@queue@internal{add~ `#2'~ to~ queue~ \string#1}
%</trace>
}
%    \end{macrocode}
% \end{macro}
%
%
% \begin{macro}{\queue@gadd@cn}
%    \begin{macrocode}
\def\queue@gadd@cn#1{\expandafter\queue@gadd@Nn\csname#1\endcsname}
%    \end{macrocode}
% \end{macro}
%
% \begin{macro}{\queue@top@NN}
%    \begin{macrocode}
\def\queue@top@NN#1#2{
  \queue@empty@err@N#1
  \expandafter\queue@top@split@w#1\q@stop{\def#2}
%<*trace>
  \trace@queue@internal{top~ of~ queue~\string#1:~`\tlp@to@str@N#2'}
%</trace>
}
% \end{macro}
%
% \begin{macro}{\queue@top@split@w}
\def\queue@top@split@w\queue@elt#1\queue@eelt#2\q@stop#3{#3{#1}}
%    \end{macrocode}
% \end{macro}
%
% \begin{macro}{\queue@top@cN}
%    \begin{macrocode}
\def\queue@top@cN#1{\expandafter\queue@top@NN\csname#1\endcsname}
%    \end{macrocode}
% \end{macro}
%
%
%
% \begin{macro}{\queue@pop@aux@nnNN}
% \begin{macro}{\queue@pop@aux@w}
%    \begin{macrocode}
\def \queue@pop@aux@nnNN #1#2#3{
  \queue@empty@err@N #3
  \expandafter\queue@pop@aux@w #3\q@stop #1#2#3}
\def \queue@pop@aux@w \queue@elt#1\queue@eelt
                #2\q@stop #3#4#5#6{#3#5{#2}#4#6{#1}}
%    \end{macrocode}
% \end{macro}
% \end{macro}
%
%
% \begin{macro}{\queue@gpop@NN}
%    \begin{macrocode}
\def \queue@gpop@NN #1#2{\queue@pop@aux@nnNN \gdef \def #1 #2
%<*trace>
  \trace@queue@internal{pop~ of~ queue~\string#1:~`\tlp@to@str@N #2'}
%</trace>
}
%    \end{macrocode}
% \end{macro}
%
%
% \begin{macro}{\queue@gpop@cN}
%    \begin{macrocode}
\def \queue@gpop@cN #1{\expandafter\queue@gpop@NN\csname#1\endcsname}
%    \end{macrocode}
% \end{macro}
%
%
% \begin{macro}{\queue@empty@err@N}
%    \begin{macrocode}
\def\queue@empty@err@N #1{\ifx#1\@empty \ERROR \fi}
%    \end{macrocode}
% \end{macro}
%
%
%
%
%
%
%
%
%
% \subsection {Quarks}
%
% A quark is a control sequence that expands into itself@ |\def|
% |\foo{\foo}|.  Quarks provide a cheap way of generating distinct
% constants.  Also, they permit the following ingenious trick: when
% you pick up a token in a temporary, and you want to know whether you
% have picked up a particular quark, all you have to do is compare the
% temporary to the quark using |\ifx|.
%
% \begin{macro}{\quark@new@N}
%    Allocate a new quark.
%    \begin{macrocode}
\def \quark@new@N #1{\def #1{#1}}
%    \end{macrocode}
% \end{macro}
%
% \begin{macro}{\q@stop}
% \begin{macro}{\q@no@value}
% \begin{macro}{\q@nil}
%    |\q@stop| is often used as a marker in parameter text,
%    |\q@no@value| is the canonical missing value, and |\q@nil|
%    represents a nil pointer in some data structures.
%    \begin{macrocode}
\quark@new@N \q@stop
\quark@new@N \q@no@value
\quark@new@N \q@nil
%    \end{macrocode}
% \end{macro}
% \end{macro}
% \end{macro}
%
% \begin{macro}{\q@error}
% \begin{macro}{\q@mark}
%    We need two additional quarks.  |\q@error| delimits the end of
%    the computation for purposes of error recovery.  |\q@mark| is
%    used in parameter text when we need a scanning boundary that is
%    distinct from |\q@stop|.
%    \begin{macrocode}
\quark@new@N\q@error
\quark@new@N\q@mark
%    \end{macrocode}
% \end{macro}
% \end{macro}
%
%
% \begin{macro}{\quark@if@no@value@NTF}
% \begin{macro}{\quark@if@no@value@NF}
% \begin{macro}{\quark@if@no@value@nTF}
% \begin{macro}{\quark@if@no@value@nT}
% \begin{macro}{\quark@if@no@value@nF}
%    Here we test if we found a special quark as the first argument.
%    The argument might contain an arbitrary list of tokens, therefore
%    we have to wrap it up in a token list pointer.
%    \begin{macrocode}
\def \quark@if@no@value@NTF #1{
%    \end{macrocode}
%    We better start with |\q@no@value| as the first argument since
%    the whole thing may otherwise loop if |#1| is wrongly given
%    a string like |aabc| instead of a single token.\footnote{It may
%    still loop in special circumstances however!}
%    \begin{macrocode}
     \ifx\q@no@value#1
          \expandafter\@firstoftwo
     \else \expandafter\@secondoftwo \fi}
%    \end{macrocode}
%    It would be possible to speed up the following commands by
%    providing individual implementations similar to the one above.
%    Should perhaps be done if they are used often!
%    \begin{macrocode}
\def \quark@if@no@value@NF #1{\quark@if@no@value@NTF {#1}\@empty}
\def \quark@if@no@value@nTF #1{\gdef \@gtempa {#1}
    \quark@if@no@value@NTF\@gtempa}
\def \quark@if@no@value@nF #1{\quark@if@no@value@nTF {#1}\@empty}
\def \quark@if@no@value@nT #1#2{\quark@if@no@value@nTF {#1}
        {#2}\@empty}
%    \end{macrocode}
% \end{macro}
% \end{macro}
% \end{macro}
% \end{macro}
% \end{macro}
%
% \begin{macro}{\quark@if@nil@NTF}
%    A function to check for the presence of |\q@nil|.
%    \begin{macrocode}
\def\quark@if@nil@NTF#1{
  \ifx#1\q@nil
    \expandafter\@firstoftwo
  \else
    \expandafter\@secondoftwo\fi}
%    \end{macrocode}
% \end{macro}
%
%
%
%
% \subsection{Commands for manipulating property lists}
%
%
% \begin{macro}{\prop@new@N}    
%    \begin{macrocode}
\def \prop@new@N #1{\def #1{}}
%    \end{macrocode}
% \end{macro}
%
% \begin{macro}{\prop@put@NNn}    
% \begin{macro}{\prop@gput@NNn}    
% \begin{macro}{\prop@gput@NNo}    
% \begin{macro}{\prop@gput@cco}    
% \begin{macro}{\prop@gput@ccn}    
%    \begin{macrocode}
\long\def \prop@put@NNn #1#2{\prop@split@aux@NNn
                             #1#2{\prop@put@aux@w {\def #1}#2}}
\long\def \prop@gput@NNn #1#2{\prop@split@aux@NNn
                                #1#2{\prop@put@aux@w {\gdef #1}#2}}
%  missing commands for galley stuff...
\def \prop@gput@NNo #1#2#3{
  \expandafter\prop@gput@NNn \expandafter #1 \expandafter 
                          #2 \expandafter { #3 } }
\def \prop@gput@cco #1#2#3{
  \expandafter\prop@gput@NNn \csname #1\expandafter\endcsname
                             \csname #2\expandafter\endcsname
                             \expandafter { #3 } }
\def \prop@gput@ccn #1#2{
  \expandafter\prop@gput@NNn \csname #1\expandafter\endcsname
                             \csname #2\endcsname
                              }
%    \end{macrocode}
% \end{macro}
% \end{macro}
% \end{macro}
% \end{macro}
% \end{macro}
%
%
% \begin{macro}{\prop@put@aux@w}    
%    \begin{macrocode}
\long\def \prop@put@aux@w #1#2#3#4#5#6{
  \quark@if@no@value@nTF {#4}
    {#1{#2{#6}#3}}
    {\def\tmp@w ##1#2\q@no@value {#1{#3#2{#6}##1}}
     \tmp@w #5}}
%    \end{macrocode}
% \end{macro}
%    
% \begin{macro}{\prop@split@aux@NNn}    
%    \begin{macrocode}
\long\def \prop@split@aux@NNn #1#2#3{
  \def\tmp@w ##1#2##2##3\q@stop {#3{##1}{##2}{##3}}
%                                       ^   ^ needed!
  \expandafter\tmp@w #1#2\q@no@value \q@stop}
%    \end{macrocode}
% \end{macro}
%    
%
% \begin{macro}{\prop@get@NNN}    
%    \begin{macrocode}
\long\def \prop@get@NNN #1#2{\prop@split@aux@NNn
                                    #1#2\prop@get@aux@w}
%    \end{macrocode}
% \end{macro}
%    
%
% \begin{macro}{\prop@get@aux@w}    
%    \begin{macrocode}
\long\def \prop@get@aux@w #1#2#3#4{\def#4{#2}}
%    \end{macrocode}
% \end{macro}
%    
%
% \begin{macro}{\prop@map@funct@Nn}    
%    \begin{macrocode}
\let \prop@map@funct@Nn \@gobbletwo
%    \end{macrocode}
% \end{macro}
%    
%
% \begin{macro}{\prop@map@NN}    
%    \begin{macrocode}
\def \prop@map@NN #1#2{
  \let \prop@map@funct@Nn #2
  \expandafter\prop@map@aux@w #1\q@stop \q@stop}
%    \end{macrocode}
% \end{macro}
%    
%
% \begin{macro}{\prop@map@aux@w}    
%    \begin{macrocode}
\def \prop@map@aux@w #1#2{
  \ifx #1\q@stop \else
    \prop@map@funct@Nn #1{#2}
    \expandafter\prop@map@aux@w
  \fi}
%    \end{macrocode}
% \end{macro}
%
% \begin{macro}{\prop@map@cN}    
%    \begin{macrocode}
\def \prop@map@cN #1{
  \expandafter \prop@map@NN \csname #1\endcsname }
%    \end{macrocode}
% \end{macro}
%
%
%
%
%
%
%
% \subsection{Commands for manipulating marks}
%
%
%
%  \begin{macro}{\trace@mark}
%  \begin{macro}{\trace@mark@internal}
%  \begin{macro}{\tracingmarks}
%    Tracing is done when |\tracingmarks| is positive.
%    \begin{macrocode}
%<*trace>
\def\trace@mark#1{\ifnum \tracingmarks > \z@
      \typeout{Marks:~ #1~ \on@line}\fi}
\def\trace@mark@internal#1{\ifnum\tracingmarks>\@ne
      \typeout{Marks:~ #1}\fi}
\newcount\tracingmarks
%</trace>
%    \end{macrocode}
%  \end{macro}
%  \end{macro}
%  \end{macro}
%
%  \begin{macro}{\mark@new}
%    Defines a new mark label and retrieval token list pointer.
%    \begin{macrocode}
\def\mark@new#1{
%<*trace>
   \trace@mark
       {new~mark:~\string#1}
%</trace>
%    \end{macrocode}
%    We make sure that the argument is still free and allocate the
%    corresponding queue. To avoid problems with the use of such tlp's
%    when they are used before the first |\mark@update@structure| we initialize
%    them in a way that |\mark@previous| and the like will accept them even
%    then.
%    \begin{macrocode}
   \newcommand* #1 {{}{}{}}
   \queue@new@c{\string#1@seq}
%    \end{macrocode}
%    The last info-number for this type will be set to |1| This value
%    is actually arbitrary and we could use |-\maxdimen+1| to get the
%    range doubled.
%    \begin{macrocode}
   \prop@gput@NNn\mark@last@plist#1{1}
%    \end{macrocode}
%    More important is that the starting value for the current
%    info-number is one less then the one used above.
%    \begin{macrocode}
   \prop@gput@NNn\mark@curr@plist#1{0}
%    \end{macrocode}
%    We now put the very first mark of this type in. This mark will be
%    empty. This will initialize the queue (now holding the empty previous
%    mark) increment the current info number, so that last and current are
%    now equal and also puts a \TeX{} mark into the galley will the
%    additional type present.
%    \begin{macrocode}
   \mark@put@Nn#1{}%
%    \end{macrocode}
%    \begin{macrocode}
   \expandafter\g@addto@macro\expandafter\mark@save@state\expandafter{%
      \expandafter\global\expandafter\let
                 \csname saved\string#1@seq\expandafter\endcsname
                 \csname\string#1@seq\endcsname}
   \expandafter\g@addto@macro\expandafter\mark@restore@state@internal\expandafter{%
      \expandafter\global\expandafter\let
                 \csname\string#1@seq\expandafter\endcsname
                 \csname saved\string#1@seq\endcsname}
}
%    \end{macrocode}
%  \end{macro}
%
%
%
%  \begin{macro}{\mark@save@state}
%  \begin{macro}{\mark@restore@state@internal}
%    Save and restore the current state of the mark data
%    structure. The commands will be extended by |\mark@new|. They are
%    needed if part of the main galley is being reused or several
%    times parsed.
%    \begin{macrocode}
\def\mark@save@state{%
      \global\let\saved@mark@curr@plist\mark@curr@plist
      \global\let\saved@mark@last@plist\mark@last@plist}
\def\mark@restore@state@internal{%
      \global\let\mark@curr@plist\saved@mark@curr@plist
      \global\let\mark@last@plist\saved@mark@last@plist}
%    \end{macrocode}
%  \end{macro}
%  \end{macro}
%
%  \begin{macro}{\mark@restore@state}
%    |\mark@restore@state| will restore the state of the mark
%    mechanism to the one saved by |\mark@save@state|. It has to be
%    called inside an output routine and as it needs an output routine
%    of its own to restore the |\mark| status of the bottom mark (so
%    that \TeX{} correctly updates the mark registers in case it
%    doesn't find a mark in box 255) it has an argument which is
%    executed inside that output routine to regain control.
%
%    The code in the argument has to set |\output| to a new output
%    routine! If not you are in trouble.
%
%    \begin{macrocode}
\def\mark@restore@state#1{%
%<*trace>
  \@tracepush{mark@restore@state}
%</trace>
%    \end{macrocode}
%    We first restore the external data structures by calling
%    |\mark@restore@sate@internal|. Then we have to ensure that \TeX's
%    idea of what the current |\botmark| is, corresponds to what it
%    was when we saved the state. Back then it was the contents of
%    |\mark@last@plist| so we are now adding an empty box and then
%    mark with is contents
%    onto the MVL followed by a penalty that ensures we trigger the
%    output routine and then specify a tiny little output routine
%    whose sole purpose is to get that mark processed.
%    \begin{macrocode}
      \mark@restore@state@internal
      \hbox{}%
%    \end{macrocode}
%    We have to protect |\mark@last@plist| from expanding too far
%    inside the |\mark| so we first put it into a token register
%    before applying |\mark|.
%    \begin{macrocode}
      \@temptokena\expandafter{\mark@last@plist}%
%<*trace>
      \trace@mark@internal{restoring~ botmark~ \the\@temptokena}%
%</trace>
      \mark
         {\the\@temptokena}
      \penalty-20203\relax
%    \end{macrocode}
%    We are not setting |\vsize| assuming its positive. So the moment
%    we return from the current output routine (we are hopefully in
%    right now) we should very very soon find the above penalty and
%    then trigger the output routine we are about to define now.
%
%    In that output routine we have a quick test for the
%    |\outputpenalty| just to ensure that we really grabbed what we
%    are supposed to grab. Then we throw away the contents of box 255
%    since all we wanted todo is to get |\botmark| set back to the
%    value of our mark above. Finally we execute whatever got passed
%    along as argument one to regain control. Since current |\output|
%    has a rather funny definition the code in that argument better
%    change that back to something that does a little bit more
%    typesetting.
%    \begin{macrocode}
      \global\output{%
%<*trace>
  \@tracepush{mark@restore@state@or}
%</trace>
         \ifnum-20203=\outputpenalty\else \ERROR \fi
         \global\setbox\@cclv\box\voidb@x
         #1%
%<*trace>
  \@tracepop{mark@restore@state@or}
%</trace>
        }%
%<*trace>
  \@tracepop{mark@restore@state}
%</trace>
}
%    \end{macrocode}
%  \end{macro}
%
%
%  \begin{macro}{\mark@put@Nn}
%    To put a mark into the current horizontal or vertical \TeX{} list
%    we first update |\mark@curr@plist| and then put its value into the
%    mark without any further expansion. (The code could be
%    optimized!)
%    \begin{macrocode}
%<-trace>\def\mark@put@Nn#1{%    % wtest hack }
%<*trace>
\def\mark@put@Nn#1#2{
   \def\l@tmpa@tlp{#2}
   \trace@mark
       {set~\string#1~<-~\tlp@to@str@N\l@tmpa@tlp}
%</trace>
%    \end{macrocode}
%    First we have to update the current info number for this type. The
%    code is a bit complicated because we have to get the number from the
%    the plist update it and then put it back. If we  would keep the numbers
%    differently this could be improved, probably at the cost of more macro
%    names.
%    \begin{macrocode}
   \@temptokena\expandafter{\mark@curr@plist}
   \prop@get@NNN\mark@curr@plist#1\@tempa
%<*debug>
   \ifx\q@no@value\@tempa
      \ERROR
   \fi
%</debug>
   \@tempcnta\@tempa
   \advance\@tempcnta\@ne
   \expandafter
   \prop@gput@NNn\expandafter\mark@curr@plist\expandafter#1\expandafter
                 {\the\@tempcnta}
%    \end{macrocode}
%    Now we want to put this plist into a \TeX{} mark. We need a
%    one-level expansion, otherwise our pkeys would expand which would
%    produce chaos. So we save it in a toks register.
%    \begin{macrocode}
   \@temptokena\expandafter{\mark@curr@plist}
   \mark
       {\the\@temptokena}
%    \end{macrocode}
%   Finally we shouldn't forget to put the info at the right of the
%   corresponding queue.
%   Without tracing we can omit the second argument to
%   this function.
%    \begin{macrocode}
%<*trace>
   \queue@gadd@cn{\string#1@seq}{#2}
%</trace>
%<-trace>   \queue@gadd@cn{\string#1@seq}%
}
%    \end{macrocode}
%  \end{macro}
%
%
%
%
%  \begin{macro}{\mark@put@Nnn}
%    \begin{macrocode}
\def\mark@put@Nnn#1#2#3{
   \mark@put@Nn#1{#2#3}}
%    \end{macrocode}
%  \end{macro}
%
%
%
%
%  \begin{macro}{\mark@update@structure}
%    This function is for use in output routines to initialize the marks
%    at the beginning. It is only useful after \TeX{} internally has
%    updated |\botmark|.
%    \begin{macrocode}
\def\mark@update@structure{
%    \end{macrocode}
%    We map a function over all keys in |\mark@last@plist|, in other
%    words over all types of marks. This routine will set the marks for
%    all types. Finally we set the plist |\mark@last@plist| to the
%    plist that was stored in |\botmark|, i.e., to the
%    info-numbers from this page.
%    \begin{macrocode}
  \prop@map@NN\mark@last@plist\mark@retrieve@single@Nn
  \expandafter\gdef\expandafter\mark@last@plist\expandafter{\botmark}
}
%    \end{macrocode}
%  \end{macro}
%
%
%
%  \begin{macro}{\mark@retrieve@single@Nn}
%    This function will be mapped over the plist containing the info-numbers
%    from the last page. This means that |#1| will be the key (which is
%    simply our type token list pointer and |#2| is the value, i.e. the
%    info-number from the last page for this type.
%    \begin{macrocode}
\def\mark@retrieve@single@Nn#1#2{
%<*trace>
   \trace@mark@internal
       {queue~for~\string#1~ before:~`\tlp@to@str@c{\string#1@seq}'}
%</trace>
%    \end{macrocode}
%    We retrieve the info number for this type from |\botmark|
%    and store the difference in |\@tempcnta|.
%    \begin{macrocode}
   \@temptokena\expandafter{\botmark}%
%<*trace>
   \trace@mark@internal{picking~ up~ from~ output~ \the\@temptokena}%
%</trace>
   \prop@get@NNN\botmark#1\@tempa
%<*debug>
   \ifx\q@no@value\@tempa
      \ERROR
   \fi
%</debug>
   \@tempcnta\@tempa
   \advance\@tempcnta-#2\relax
%<*trace>
   \trace@mark@internal
       {found~\the\@tempcnta\space entries~ for~ \string#1~
        (new~\@tempa;~old~#2)}
%</trace>
%    \end{macrocode}
%    Then we decide what to do depending on its value.
%    We use |\@tempa|, |\@tempb|, and |\@tempc| to
%    temporarily hold the previous, first and last mark for this type.
%    \begin{macrocode}
   \ifcase\@tempcnta
%    \end{macrocode}
%    If there haven't been any new marks we leave the queue alone,
%    peeking only at its top element and using it for all three marks.
%    \begin{macrocode}
     \queue@top@cN{\string#1@seq}\@tempa
     \let\@tempb\@tempa
     \let\@tempc\@tempa
   \or
%    \end{macrocode}
%    When there was one mark of this type, we pop one item for the
%    previous mark of this type, and use the remaining item for first
%    and last mark.
%    \begin{macrocode}
     \queue@gpop@cN{\string#1@seq}\@tempa
     \queue@top@cN{\string#1@seq}\@tempb
     \let\@tempc\@tempb
   \else
%    \end{macrocode}
%    If we had more than one mark for this type we pop one mark for the
%    previous mark of this type, then pop another mark for the first
%    mark of this type, then pop all but the last item and use the last
%    item as the last mark of this type, keeping it as before on the
%    queue.
%    \begin{macrocode}
     \queue@gpop@cN{\string#1@seq}\@tempa
     \queue@top@cN{\string#1@seq}\@tempb
     \@whilenum \@tempcnta>\@ne\do
        {\advance\@tempcnta\m@ne
         \queue@gpop@cN{\string#1@seq}
            \@tempc
        }
     \queue@top@cN{\string#1@seq}\@tempc
   \fi
%    \end{macrocode}
%    After this process all values for previous, first and last mark for
%    this type are found so that we can store them in the tlp
%    representing this type for further processing.
%    \begin{macrocode}
   \@temptokena\expandafter{\@tempa}
   \@temptokenb\expandafter{\@tempb}
   \@temptokenc\expandafter{\@tempc}
   \edef#1{{\the\@temptokena}{\the\@temptokenb}{\the\@temptokenc}}
%<*trace>
   \trace@mark
       {in~output:~\string#1~->~\tlp@to@str@N#1}
%    \end{macrocode}
%    At this point the queue should show only one item.
%    \begin{macrocode}
   \trace@mark@internal
       {queue~ for~\string#1~after:~`\tlp@to@str@c{\string#1@seq}'}
%</trace>
}
%    \end{macrocode}
%  \end{macro}
%
%  \begin{macro}{\@temptokenb}
%  \begin{macro}{\@temptokenc}
%    \begin{macrocode}
\newtoks\@temptokenb
\newtoks\@temptokenc
%    \end{macrocode}
%  \end{macro}
%  \end{macro}
%
%
%  \begin{macro}{\mark@curr@plist}
%  \begin{macro}{\mark@last@plist}
%    Here are the allocations for the two plists used.
%    \begin{macrocode}
\prop@new@N\mark@curr@plist
\prop@new@N\mark@last@plist
%    \end{macrocode}
%  \end{macro}
%  \end{macro}
%
%
%
%
%
%
%  \begin{macro}{\mark@get@first@N}
%  \begin{macro}{\mark@get@last@N}
%  \begin{macro}{\mark@get@previous@N}
%    To retrieve the first, last or previous page mark, we
%    grap the appropriate value stored in the tlp. The marks are stored
%    in the tlp in the order previous, first, last mark. These functions
%    should be used only in output routines after |\mark@retrieve| has
%    acted, otherwise their value will be wrong. They are all fully
%    expandable.
%
%    Actually, they may bomb when tracing is included and we have
%    material that doesn't work corretly within a write.
%    \begin{macrocode}
%<*trace>
\def\mark@get@first@N#1{
       \expandafter\@secondofthree#1
       \trace@mark@internal{use~first~\string#1~=~
            `\expandafter\@secondofthree#1'}
}
\def\mark@get@last@N#1{
   \expandafter\@thirdofthree#1
   \trace@mark@internal{use~ last~ \string#1~ =~
       `\expandafter\@thirdofthree#1'}
}
\def\mark@get@previous@N#1{
   \expandafter\@firstofthree#1
   \trace@mark@internal{use~prev~\string#1~ =~
       `\expandafter\@firstofthree#1'}
}
%</trace>
%<-trace>\def\mark@get@first@N{%
%<-trace>       \expandafter\@secondofthree}
%<-trace>\def\mark@get@last@N{%
%<-trace>       \expandafter\@thirdofthree}
%<-trace>\def\mark@get@previous@N{%
%<-trace>       \expandafter\@firstofthree}
%    \end{macrocode}
%    above version in trace case doesn't work alway!
%    \begin{macrocode}
\def\mark@get@first@N{%
       \expandafter\@secondofthree}
\def\mark@get@last@N{%
       \expandafter\@thirdofthree}
\def\mark@get@previous@N{%
       \expandafter\@firstofthree}
%    \end{macrocode}
%  \end{macro}
%  \end{macro}
%  \end{macro}
%
%
% \begin{macro}{\DeclareMarkType}
%    \begin{macrocode}
\def\DeclareMarkType#1{
  \expandafter\mark@new\csname mark@#1\endcsname}
%    \end{macrocode}
%  \end{macro}
%
% \begin{macro}{\PutMark}
%    \begin{macrocode}
\def\PutMark#1{
   \@ifundefined{mark@#1}{\ERROR}{}
   \expandafter\mark@put@Nn\csname mark@#1\endcsname}
%    \end{macrocode}
%  \end{macro}
%
%
% \begin{macro}{\PreviousMark}
%    \begin{macrocode}
\def\PreviousMark#1{
  \expandafter\mark@get@previous@N\csname mark@#1\endcsname}
%    \end{macrocode}
%  \end{macro}
%
% \begin{macro}{\FirstMark}
%    \begin{macrocode}
\def\FirstMark#1{
  \expandafter\mark@get@first@N\csname mark@#1\endcsname}
%    \end{macrocode}
%  \end{macro}
%
% \begin{macro}{\LastMark}
%    \begin{macrocode}
\def\LastMark#1{
  \expandafter\mark@get@last@N\csname mark@#1\endcsname}
%    \end{macrocode}
%  \end{macro}
%
% \Finale
%
\endinput
%
% $Log$
% Revision 1.1  2001/07/26 19:55:12  latex3
% original web distrib
%
% Revision 1.6  2000/06/13 20:47:23  latex3
% docu update
% inlined quarks code
%
