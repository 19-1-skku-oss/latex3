% \iffalse
%% File: xparse.dtx (C) Copyright 1999 Frank Mittelbach, Chris Rowley, David Carlisle
%%                  (C) Copyright 2004-2008 Frank Mittelbach, LaTeX3 Project
%%
%% It may be distributed and/or modified under the conditions of the
%% LaTeX Project Public License (LPPL), either version 1.3c of this
%% license or (at your option) any later version.  The latest version
%% of this license is in the file
%%
%%    http://www.latex-project.org/lppl.txt
%%
%% This file is part of the ``xbase bundle'' (The Work in LPPL)
%% and all files in that bundle must be distributed together.
%%
%% The released version of this bundle is available from CTAN.
%%
%% -----------------------------------------------------------------------
%%
%% The development version of the bundle can be found at
%%
%%    http://www.latex-project.org/cgi-bin/cvsweb.cgi/
%%
%% for those people who are interested.
%%
%%%%%%%%%%%
%% NOTE: %%
%%%%%%%%%%%
%%
%%   Snapshots taken from the repository represent work in progress and may
%%   not work or may contain conflicting material!  We therefore ask
%%   people _not_ to put them into distributions, archives, etc. without
%%   prior consultation with the LaTeX Project Team.
%%
%% -----------------------------------------------------------------------
%%
%<*driver|package>
\RequirePackage{l3names}
%</driver|package>
%\fi
\GetIdInfo$Id$
          {generic document command parser}
%\iffalse
%<*driver>
%\fi
\ProvidesFile{\filename.\filenameext}
  [\filedate\space v\fileversion\space\filedescription]
%\iffalse
  \documentclass{ltxdoc}
  \usepackage[T1]{fontenc}
  \usepackage{textcomp}
  \makeatletter
  \providecommand*\eTeX{%
    \if b\expandafter\@car\f@series\@nil\boldmath\fi
    $\m@th\varepsilon$-\TeX}
  \makeatother
% \usepackage{ldcdoc}
  \begin{document}
  \catcode`\_=11
  \catcode`\:=11
  \DocInput{xparse.dtx}
  \end{document}
%</driver>
% \fi
%
% \CheckSum{632}
%
%
% \title{The \textsf{xparse} package\thanks{This file
%         has version number \fileversion, last
%         revised \filedate.}}
% \author{FMi, CAR, DPC, MH}
% \date{\copyright~\filedate}
%  \maketitle
%
% \begin{abstract}
% The interfaces described in this document are not meant to be final
% but only as a basis for discussion. Building productive applications
% using the current code is discouraged.
% \end{abstract}
%
% \section{Interface}
%
% This package implements high-level interface commands for class file
% writers which allows the separation of formatting commands
% (typically instances of so-called `templates') and their arguments
% from the signature of document-level commands.
%
% This works by declaration commands that provide a general
% specification method for the typical \LaTeX{} syntax, e.g.,
% star-form, optional arguments, and mandatory arguments. A command
% (or environment) declared in this way parses the input according to
% its spec and presents its findings in a normalized way for further
% processing.
%
% \subsection{Argument spec}
%
%
%  An argument specification is a list of letters each representing a
%  type of argument, i.e., |m| is a mandatory argument (surrounded by
%  braces), |o| an optional argument (surrounded by brackets if
%  present), and |s| represents a star (which might be present or not).
%  Thus the argument spec for headings as implemented by
%  |\@startsection| in standard \LaTeX{} would be represented by the
%  three letters |som|. Other argument types are available and can be
%  added at will. For a complete list of built-in argument types see
%  the next section.
%
%  There is one important argument specifier worth taking note of. The
%  |P| does not behave as a normal argument specifier but allows the
%  next argument to take an argument containing the |\par| token which
%  is not allowed by default.
%
%
%  \subsection{Parsing results}
%
%  To normalise the result of parsing the input according to an
%  argument specification it is important to uniquely identify all
%  arguments found. For this reason each parsing operation initiated by
%  one of the argument spec letters will result in an identifiable
%  output as follows:
%  \begin{description}
%  \item[m] will return the parsed argument surrounded by a brace
%    pair, i.e., will normally be the identity;
%  \item[o] will return the parsed argument surrounded by a brace pair
%    if present. Otherwise it will return the token |\NoValue|;
%  \item[O\{default\}] will return the parsed argument surrounded by a
%    brace pair if present. Otherwise it will return the |{default}|
%    as specified above.
%  \item[S\{\meta{symbol}\}] will return either the token |\c_true| or
%    |\c_false| depending on whether or not the next token is
%    \meta{symbol}.
%  \item[s] will return either the token |\c_true| or |\c_false|
%    depending on whether or not a star was parsed. This is just a
%    shorthand for |S{*}|.
%  \item[c] will parse the syntax |(|\meta{x}|,|\meta{y}|)|, i.e., a
%    coordinate pair and will return |{{|\meta{x}|}{|\meta{y}|}}| as
%    the result. If no open parentheses is scanned an error is
%    signalled.
%  \item[C\{\{x-default\}\{y-default\}\}] behaves like |c|, i.e.,
%    parses a coordinate pair if present. If the coordinate pair is
%    missing it returns the default values instead.
%  \item[l] Reads everything up to the first left brace as the
%    argument.
%  \end{description}
%  For example, given the spec |soomO{default}| the input
%  |*[Foo]{Bar}| would be parsed as
%  |\c_true{Foo}\NoValue{Bar}{default}|. In other words there will be
%  always exactly the same number of brace groups or tokens as the
%  number of letters in the argument spec.
%
%  There are a few exceptions to this rule as the r\^ole of following
%  letters is to affect how the next argument is read.
%  \begin{description}
%  \item[>] takes a mandatory argument like |{P}| or |{PW}| and
%    inserts these extra argument specifiers so that they act on the
%    next regular argument type. See below.
%  \item[P] will cause the next argument to allow |\par| in it's
%    argument. Thus the specification |m>{P}mm| will cause the command to
%    read three mandatory arguments but only the second can contain the
%    |\par| token.
%  \item[W] will cause the next argument to \emph{not} ignore a space
%    when trying to scan ahead for a special symbol such as a |*| or an
%    optional argument. This is mostly useful when the last argument(s)
%    of a document command are optional (see example later on).
%  \end{description}
%
%
%  \subsection{Applying the parsing results}
%
%  Since the result of the parsing is a well defined number of tokens or
%  brace groups it is easy to pass them on in any order to any
%  processing function. To this end the tokens or brace groups are
%  associated with the standard argument specifiers in \TeX{} macros,
%  i.e., |#1|, |#2|, and so forth. This limits the argument
%  specification to a maximum number of 9 letters, but for practical
%  applications this should be sufficient.
%
%  \subsection{The class designer interface}
%
%  This package provides commands for declaring commands
%  and environments to be used within the document body.
%
%  \DescribeMacro\DeclareDocumentCommand
%  The |\DeclareDocumentCommand| declaration takes three arguments. The
%  first argument is the name of the command to be declared, the second
%  is the argument specification in the syntax described above, and the
%  third is the action to be carried out once the arguments are
%  parsed. Within the third argument |#1|, |#2|, etc.\ denote the
%  result of the parsing, e.g.,
%  \begin{verbatim}
%  \DeclareDocumentCommand{\chapter}{som}
%    { \IfBooleanTF {#1}
%        { \typesetnormalchapter {#2}{#3} }
%        { \typesetstarchapter   {#3} }
%    }
%  \end{verbatim}
%  would be a way to define a |\chapter| command which would
%  essentially behave like the current \LaTeX{} command (except that it
%  would accept an optional argument even when a |*| was parsed). The
%  |\typesetnormalchapter| could test its first argument for being
%  |\NoValue| to see if an optional argument was present.
%
%  Of course something like the |\IfNoValueTF| test could also be
%  placed inside a function that would process all three arguments, thus
%  using the templates and their instances as provided by the
%  \texttt{template} package such a declaration would probably look
%  more like the following example:
%  \begin{verbatim}
%  \DeclareDocumentCommand{\chapter}{som}
%    { \UseInstance {head} {A-head-main} {#1} {#2} {#3} }
%  \end{verbatim}
%
%  Using the |\DeclareDocumentCommand| interface it is easy to modify
%  the document-level syntax while still applying the same
%  layout-generating functions, e.g., a class that would not support
%  optional arguments or star forms for the heading commands could
%  define |\chapter| like this:
%  \begin{verbatim}
%   \DeclareDocumentCommand{\chapter}{m}
%     { \UseInstance {head} {A-head-main} \BooleanFalse \NoValue {#1} }
%  \end{verbatim}
%  while a class that would allow for an additional optional argument
%  (for whatever reason) could define it like that:
%  \begin{verbatim}
%   \DeclareDocumentCommand{\chapter}{somo}
%     { \doSomethingWithTheExtraOptionalArg {#4}
%       \UseInstance {head} {A-head-main} {#1} {#2} {#3} }
%  \end{verbatim}
%  \DescribeMacro\NewDocumentCommand
%  \DescribeMacro\RenewDocumentCommand
%  \DescribeMacro\ProvideDocumentCommand If |\DeclareDocumentCommand|
%  is used on an already defined command, it'll overwrite it. The
%  commands |\NewDocumentCommand|, |\RenewDocumentCommand| and
%  |\ProvideDocumentCommand| have identical syntax to that of
%  |\DeclareDocumentCommand| with the exception that |\New...| will
%  only define the command if it is undefined, |\Renew...| will only
%  redefine an existing command and |\Provide...| will only define the
%  command if it is already undefined.
%
%
%  \DescribeMacro\DeclareDocumentEnvironment
%  \DescribeMacro\NewDocumentEnvironment
%  \DescribeMacro\RenewDocumentEnvironment
%  \DescribeMacro\ProvideDocumentEnvironment The
%  |\DeclareDocumentEnvironment| declaration is similar to
%  |\DeclareDocumentCommand| except that it takes four arguments: the
%  first being the environment name (without a backslash), the second
%  again the argument-spec, and the third and forth are the actions
%  taken at start aqnd end of the environment. The parsed arguments are
%  available to both the start and the finish as |#1|, |#2|, etc.
%
%  All commands and environments defined with the above commands are
%  automatically robust.
%
%
%
% \subsection{A different interface (for class design?)}
%
%  Combining signature and top-level definition of a command |\foo| as
%  in
%  \begin{verbatim}
%  \DeclareDocumentCommand \foo { m c m }
%  { \typeout{1:#1}
%    \typeout{2:#2}
%    \typeout{3:#3}
%  }
%  \end{verbatim}
%  is fine in certain cases, e.g., if the user wants to declare a few
%  commands this way in the preamble of a document. However in a class
%  file it probably better to completely separate signature (i.e.,
%  argument \#1 and \#2) from top-level implementation (i.e., argument
%  \#3).
%
%
%  \begin{verbatim}
%  \DeclareDocumentCommandInterface \foo {bar} { m c m }
%  \end{verbatim}
%  The above now declare only the signature of the command |\foo| and
%  states that the implementation is to be found under the label |bar|.
%
%  A bunch of such statements would then for the first time clearly
%  define a document class (i.e., what a document class instance needs
%  to define to be compliant).
%
%  An instance would then consist of a lot of declarations of the type
%  \begin{verbatim}
%  \DeclareDocumentCommandImplementation {bar} {3}
%  { \typeout{1:#1}
%    \typeout{2:#2}
%    \typeout{3:#3}
%  }
%  \end{verbatim}
%  i.e., defining how to format things being referenced as part of the
%  signature.
%
%  The functions |\DeclareDocumentCommandInterface| and
%  |\DeclareDocumentCommandImplementation| are implemented for now but
%  it is not sure if they will stay. If a document command is defined
%  with |\DeclareDocumentCommandInterface| but there is no matching
%  interface it prints the label in quotes as |``bar''|.
%
%
%
%
%  \subsection{Checking for a value or boolean}
%
%  It would make perfect sense to make |\NoValue| a \LaTeX3 quark but
%  as there is a real danger of it getting executed by accident it is
%  best not to as it would result in an infinite loop.
%
%  \DescribeMacro\NoValue
%  For this reason |\NoValue| is defined to expand to the string
%  |-NoValue-| which would get typeset if ever executed thus clearly
%  indicating the type of error the writer made.
%
%  However this makes testing for this token slightly complicated as in
%  that case the test
%  \begin{verbatim}
%  \tlp_set:Nn \l_seen_tlp {#1}
%  \tlp_set:Nn \l_hidden_novalue_tlp {\NoValue}
%  \tlp_if_eq:NNTF \l_seen_tlp \l_hidden_novalue_tlp
%  \end{verbatim}
%  will be true if |#1| was |\NoValue| but false if |#1| itself
%  contains a macro which contains |\NoValue|; a case that happens
%  unfortunately very often in practice.  Using an unguarded |x| type
%  expansion to define |\l_seen_tlp| is out of the question as |#1| is
%  typically either |\NoValue| or arbitrary user input for which we
%  can't properly guard it unless we are sure people only use robust
%  commands. When running a pure \LaTeX3 format where all document
%  commands are robust this is perhaps something to be looked at again
%  but when running on top of \LaTeXe{} we have to be careful.
%
%  Therefore we use a somewhat different procedure here, which goes
%  like this:
%  \begin{enumerate}
%  \item Store |#1| in a temporary variable and check if it equals a
%    variable containing |\NoValue|. If true, execute the \meta{true}
%    code, otherwise go to~2. 
%  \item Peek at the first token in |#1|. If this is a macro taking no
%    arguments expand |#1| once and go to~1. Otherwise execute the
%    \meta{false} code.
%  \end{enumerate}
%  The reason for the careful peeking ahead is that |#1| may very well
%  be a macro taking arguments and it is not certain that these
%  arguments are present! Hence this could lead to the well-known
% \begin{verbatim}
% Argument of \XXX has an extra }
% \end{verbatim}
%  and we wouldn't want that\dots{} Goes without saying that this
%  procedure is quite tedious but usually it will exit after the first
%  time |#1| has been inspected. One additional test could be added,
%  namely that |#1| should also contain exactly one token but I don't
%  think that is going to matter much.
%
%  \DescribeMacro\IfNoValueTF
%  \DescribeMacro\IfNoValueT
%  \DescribeMacro\IfNoValueF
%  \DescribeMacro\IfValueTF
%  \DescribeMacro\IfValueT
%  \DescribeMacro\IfValueF
%  These macros are used for checking if an optional argument was
%  present as in the example below.
%  \begin{verbatim}
%  \DeclareDocumentCommand\testcmd{om}{
%    \IfNoValueTF{#1}{``#2''}{#1,#2}
%  }
%  \end{verbatim}
%
%
%
%  \DescribeMacro\IfBooleanTF
%  \DescribeMacro\IfBooleanT
%  \DescribeMacro\IfBooleanF
%  As mentioned earlier, the parsing result for a symbol argument like
%  |s| or |S{+}| is represented as the one of the tokens |\c_true| or
%  |\c_false| respectively. To test for these values the macro
%  |\IfBooleanTF| can be used. It expects as its first argument either
%  |\c_true| or |\c_false| and executes its second or third argument
%  depending on this value. |\IfBooleanT| and |\IfBooleanF| are
%  obvious shortcuts.
%
%  At one point in time I thought that one can represent everything
%  using |\NoValue|, e.g., for the star case either return |*| or
%  |\NoValue|. However, this slows down processing of commands like
%  |\\*| considerably since they would then have to use a slower
%  internal test instead of a fast two-way switch.
%
%
%  \subsection{New argument types and shorthands}
%
%  New argument types can be added at in a fairly straight forward
%  manner with the command
%  |\DeclareArgumentType|\DescribeMacro\DeclareArgumentType. This
%  command takes seven arguments: |#1| is a symbol denoting the
%  argument type, |#2| is the token the scanner should look for, |#3|
%  is one of the words |meaning|, |charcode| or |catcode| which is
%  handed down to the underlying |\peek_meaning:NTF| or
%  |\peek_charcode:NTF| etc. functions. For instance you can use the
%  |charcode| version if you want to pick up a literal |*| and not
%  just something that has the same \emph{meaning} as a regular
%  |*|. |#4| is for executing a special action (like an error message)
%  if no argument is found, |#5| is the default value in case of a
%  missing argument, |#6| is what the scanner will encounter, and
%  finally |#7| is what the scanner grabs from the argument type. This
%  is perhaps clearer with an example\dots{} This is how the |o| type
%  is implemented:
%  \begin{verbatim}
%  \DeclareArgumentType o [ {meaning} {} {\NoValue} {#1[#2]} {#2}
%  \end{verbatim}
%  In a similar fashion a |b| type expecting its argument inside |<|
%  and |>| would be defined as
%  \begin{verbatim}
%  \DeclareArgumentType b < {meaning} {} {\NoValue} {#1<#2>} {#2}
%  \end{verbatim}
%
%  \DescribeMacro\DeclareArgumentTypeWithDefault
%  |\DeclareArgumentTypeWithDefault| is even simpler as this type
%  expects the default value to be input at the time of argument
%  specification. Thus arguments |#3| and |#4| from above are
%  unnecessary and are simply absent. Thus the |O| type is
%  implemented\footnote{I would say it is at best wishful thinking
%    that the interface could be something like
%    \texttt{\cs{DeclareArgumentTypeWithDefault}
%      C\{2\}\{meaning\}\{(\#1,\#2)\}\{\{\#1\}\{\#2\}\}} as it would
%    be quite troublesome implementation-wise and the current syntax
%    isn't that difficult to use.} as
%  \begin{verbatim}
%  \DeclareArgumentTypeWithDefault O [ {meaning}{#1[#2]} {#2}
%  \end{verbatim}
%  and again a similar |B| type could be implemented as
%  \begin{verbatim}
%  \DeclareArgumentTypeWithDefault B < {meaning}{#1<#2>} {#2}
%  \end{verbatim}
%
%
%  There is also a possibility to define a shorthand for a specific
%  sequence of arguments with the command
%  \DescribeMacro\DeclareArgumentTypeShorthand
%  |\DeclareArgumentTypeShorthand|. As an example the |s| argument
%  type is just a different way of saying |S{*}| so it's simply
%  implemented as
%  \begin{verbatim}
%  \DeclareArgumentTypeShorthand s {S{*}}
%  \end{verbatim}
%  Similarly one could add a shorthand |M| for inserting five
%  mandatory arguments where the latter is allowed to take |\par| in
%  its argument:
%  \begin{verbatim}
%  \DeclareArgumentTypeShorthand M {mmmm>{P}m}
%  \end{verbatim}
%
%  Finally there is also the command
%  \DescribeMacro\DeclareSymbolArgument|\DeclareSymbolArgument| which
%  defines the low level interface to looking ahead for specific
%  symbols and removing them. For instance the |S| type argument is
%  declared as
% \begin{verbatim}
% \DeclareSymbolArgument S{meaning}
% \end{verbatim}
%  So symbols picked up by this type will compare the meaning of the
%  tokens. One could do a
% \begin{verbatim}
% \DeclareSymbolArgument A{charcode}
% \end{verbatim}
%  and then the |A| type would compare character codes instead. For
%  instance a command which expects a |+| symbol would return
%  \meta{true} if a |+| of any catcode is detected, not just catcode
%  12 as is usual.
%
% \subsection{Some comments on the need for the \texttt{O} specifier}
%
% With |\newcommand| there is the possibility of specifying a default
% for an optional argument which is stored away in a more or less
% efficient manner. For example below is the old definition of
% |\linebreak| as can be found in the \LaTeX2e kernel:
%\begin{verbatim}
%\def\linebreak{\@testopt{\@no@lnbk-}4}
%\def\@no@lnbk #1[#2]{%
%  \ifvmode
%    \@nolnerr
%  \else
%    \@tempskipa\lastskip
%    \unskip
%    \penalty #1\@getpen{#2}%
%    \ifdim\@tempskipa>\z@
%      \hskip\@tempskipa
%      \ignorespaces
%    \fi
%  \fi}
%\end{verbatim}
% Ignoring for the moment that the above is slightly optimised an
% expansion of this code under |\tracingall| will result in about 90
% lines of tracing output.  If we reimplement this using
% |\DeclarDocumentCommand\linebreak{o}| we have to use |\IfNoValueTF|
% to find out if an argument was present which (because of the careful
% expansion we do internally) results in about twice as much of
% tracing lines. In contrast using |O{4}| as below we end up with 110
% lines, which seems roughly the price we have to pay for the extra
% generality available (though this could perhaps even be reduced by a
% better implementation of the parsing machine as originally done by
% David, before i talked him into adding support for arguments in the
% end code of an environment).\footnote{These comments were made
%   before the extensive changes in expl3 and xparse in 2005.}
%
%\begin{verbatim}
%\DeclareDocumentCommand\linebreak { O{4} }
% {
%  \ifvmode
%    \@nolnerr
%  \else
%    \@tempskipa\lastskip
%    \unskip
% %    \IfNoValueTF{#1}
% %       {\break}
% %       {\penalty -\@getpen{#1}}
%    \penalty -\@getpen{#1}
%    \ifdim\@tempskipa>\z@
%      \hskip\@tempskipa
%      \ignorespaces
%    \fi
%  \fi
% }
%\end{verbatim}
%
%
% \subsection{A somewhat complicated example}
%
% This example reimplements the |\makebox| and |\framebox| interface
% of \LaTeX{} both of which are unfortunately quite overloaded
% syntactically. For this reason the example isn't meant to show good
% coding practice but to show the power of the interface even though
% applied in a somewhat bad way.
%
% |\makebox| and |\framebox| support the following different document
% syntax forms:
% \begin{itemize}
% \item
% |\makebox{A}|: only a single mandatory argument.
%
% \item
% |\makebox[20pt]{B}|: one optional argument specify the box width.
%
%
% \item
% |\makebox[30pt][r]{C}|: two optional arguments the second specifying
%    the text position within the box being made (l,c,r being allowed
%    with c being the default).
%
% \item
% |\makebox(20,30){D}|: within picture mode the width is specified not
% as an optional argument in brackets but as a coordinate pair.
%
% \item
% |\makebox(0,0)[lt]{E}|: in that case an optional argument following
% coordinate pair denotes the placement within the box which can have
% different values compared to case C above.
% \end{itemize}
%
% To cater for this overloaded structure we can define |\makebox| to
% be something like the following:
%\begin{verbatim}
%\DeclareDocumentCommand \makebox { C{\NoValue} o O{c} m}
% { \IfNoValueTF{#1}
%     { \ltx@maketextbox{#2}{#3}{#4} }
%     { \ltx@makepicbox #1  {#2}{#4} } }
%\end{verbatim}
% In other words: if we do not see a coordinate pair first (i.e.,
% first argument is |\NoValue| then we expect up to two optional
% arguments (the width of the box or |\NoValue| if not given and the
% placement specifier with a default of |c| if not given) and one
% mandatory one which is the text. In that case we pass argument 2 to
% 4 to an internal function |\ltx@maketextbox| which builds the text
% box.
%
% Otherwise, if the first argument is a coordinate pair we parse an
% optional argument denoting the placement specifier. Since
% \texttt{xparse} doesn't support variant syntax we actually parse for
% another optional argument (which has no meaning in that case and is
% in fact ignored if present) followed by a mandatory one containing
% the box text. In that case we pass the coordinate pair (|#1|), the
% specifier (|#2|), and the text (|#4|) to the function
% |\ltx@makepicbox| which builds the picture box. Note the special
% handling of the coordinatesD: which are passed without surrounding
% braces to |\ltx@makepicbox|: since the coordinate argument looks
% like |{x-val}{y-val}| the receiving function |\ltx@makepicbox| gets
% the |x-val| as argument one and the |y-val| as argument two.
%
% A definition for |\framebox| would look more or less identical
% except that we would need to pass the arguments to slightly
% different internal functions. The alternative is to give the
% internal functions an extra argument that controls whether or not a
% frame is bying built.
%
% \begin{macro}{\ltx@makepicbox}
% So here is a possible implementation of |\ltx@makepicbox| that
% builds a picture box with or without frame. It takes the following
% mandatory arguments:
% \begin{enumerate}
% \item x-part of coordinate
% \item y-part of coordinate
% \item placement specifier, e.g., |lt| or |\NoValue|
% \item text of box
% \item the token |\frame| (if a frame should surround the box) or the
% token |\@firstofone| --- not pretty i agree
% \end{enumerate}
%\begin{verbatim}
%\def\ltx@makepicbox#1#2#3#4#5
%  {
%   #5{
%     \vbox to#2\unitlength
%      {\let\mb@b\vss \let\mb@l\hss\let\mb@r\hss
%       \let\mb@t\vss
%       \IfNoValueF{#3}
%        {
%         \@tfor\reserved@a :=#3\do{
%           \if s\reserved@a
%             \let\mb@l\relax\let\mb@r\relax
%           \else
%             \expandafter\let\csname mb@\reserved@a\endcsname\relax
%           \fi}%
%        }
%       \mb@t
%       \hb@xt@ #1\unitlength{\mb@l #4\mb@r}
%       \mb@b
%       \kern\z@}
%   }
%  }
%\end{verbatim}
% \end{macro}
%
%
%
% \begin{macro}{\ltx@maketextbox}
% For the text case the internal function could  take the following
% mandatory arguments:
% \begin{enumerate}
% \item width of the box or |\NoValue| (denoting to build the box at
% natural width)
% \item placement specifier, e.g., |l|. (No test for |\NoValue| being undertaken)
% \item text of box
% \item the token |\fbox| (if a frame should surround the box) or the
% token |\mbox|
% \end{enumerate}
% The actual code is taken straight from the current \LaTeX{} kernel
% and looks kind of scary.
%\begin{verbatim}
%\def\ltx@maketextbox#1#2#3#4
% {
%  \IfNoValueTF{#1}
%     {#4{#3}}
%     {
%      \leavevmode
%      \@begin@tempboxa\hbox{#3}
%       \setlength\@tempdima{#1}
%       \ifx#4\fbox
%         \setbox\@tempboxa\hb@xt@\@tempdima
%              {\kern\fboxsep\csname bm@#2\endcsname\kern\fboxsep}
%         \@frameb@x{\kern-\fboxrule}
%       \else
%         \hb@xt@\@tempdima{\csname bm@#2\endcsname}
%       \fi
%      \@end@tempboxa
%     }
% }
%\end{verbatim}
% \end{macro}
%
% \begin{macro}{\makebox}
% \begin{macro}{\framebox}
% Given the above internal functions the declarations of |\makebox|
% and |\framebox| would then look like this:
%\begin{verbatim}
%\DeclareDocumentCommand \makebox { C{\NoValue} o O{c} m}
% { \IfNoValueTF{#1}
%     { \ltx@maketextbox{#2}{#3}{#4}\mbox }
%     { \ltx@makepicbox #1  {#2}{#4}\@firstofone } }
%
%\DeclareDocumentCommand \framebox { C{\NoValue} o O{c} m}
% { \IfNoValueTF{#1}
%     { \ltx@maketextbox{#2}{#3}{#4}\fbox }
%     { \ltx@makepicbox #1  {#2}{#4}\frame } }
%\end{verbatim}
% \end{macro}
% \end{macro}
%
%
%
%  Here's another example which shows how one can use the |W| specifier.
%  The \textsf{amsmath} environments use their own definition of |\\|
%  which goes under the name |\math@cr|. As there is a good chance the
%  next line of math begins with either an asterisk or something in
%  square brackets this often leads to errors. The following example
%  definition works (I \emph{did} try it).
%  \begin{verbatim}
%  \ExplSyntaxOn
%  \makeatletter
%  \DeclareDocumentCommand\math@cr{>{W}s >{W}O{\z@}}{
%    \IfBooleanTF{#1}
%    { \global\@eqpen\@M }{
%      \global\@eqpen
%        \ifnum\dspbrk@lvl <\z@
%          \interdisplaylinepenalty
%        \else
%          -\@getpen\dspbrk@lvl
%        \fi
%    }
%    \math@cr@@@
%    \noalign{\vskip#2\relax}
%  }
%  \makeatother
%  \ExplSyntaxOff
%  \end{verbatim}
%  Note that this definition automatically avoids problems within the
%  alignments even if it is called as |\\&|.
%
%
%
%
% \subsection{Open issues}
%
% In this section unresolved issues or ideas to think about and
% perhaps implement are collected. There is no particular order to
% them.
%
% \begin{itemize}
% \item
%   Support for parsing verbatim type of arguments was considered (and
%   actually implemented at one stage. E.g.,
%   |g{|\meta{prepare-parsing}|}| where |g| first run
%   \meta{prepare-parsing} (which might involve |catcode|
%   changes)\footnote{urg horror!}, then look at the next token: if
%   that would be a |{| it would scan a brace delimited argument
%   (reverting catcodes of |{| and |}| if needed) otherwise would scan
%   for an argument delimited by that token so that something like
%   |\verb+%\+| would scan |%\| as its argument assuming that the
%   \meta{prepare-parsing} turned |%| and |\| into non-letters.
%
%   All kind of nasty problems lurking especially no proper error
%   checks possible and of course as we all know such commands would
%   then not work inside arguments of other commands. At least they
%   would have some restrictions.
%
%   Also no way to make the parsing smart by not accepting newlines as
%   part of the code (|\verb| does this right now and this is really
%   helpful as it catches runaways nicely).
%
%
% \item
%   \textsf{xparse} should probably go into the \LaTeX3 kernel in which
%   case it should also provide the definitions of |\begin| and |\end|
%   and this raises a few questions: Will environments be defined as
%   control sequences |\foo| and |\endfoo|? If they are do we then want
%   to include support for a separate command |\foo| and if so what
%   about using the environment forms in special scanner mechanisms
%   where we \emph{can't} use |\begin{foo}|? In those cases one could
%   use a special command like what the \textsf{fancyvrb} package
%   provides with the command |\VerbatimEnvironment| and all packages
%   that define such special arguments should then also make sure that
%   the |\foo| command issued on its own should cause an error.
%   Another solution would be to simply define environments as
%   |\beginfoo| and |\endfoo| and then let |\begin| check if it was
%   used to call an environment or a command. If the latter is the
%   case then insert grouping as that could be handled by |\beginfoo|
%   itself. When |\end| is reached check if |\endfoo| exists. If it
%   does then use that, otherwise we were probably using a normal
%   command which should just close the group.
% \end{itemize}
%
%
% \subsection{Rejected ideas---and why}
%
% Let's say the document command |\testmo| has signature |mo|. If we
% call it as |\testmo{arg} [opt]| a following space will be obeyed.
% However calling it with the mandatory argument only as in
% |\testmo{arg}| will gobble a following space because the peek ahead
% in the |o| type ignores spaces. We can record that a space has been
% gobbled when peeking ahead and we can also put it back in if needed
% but I have only ever seen one request for this and in that case the
% better solution was to use a signature |mWo| so that no spaces are
% allowed between the mandatory and the optional argument. One could
% even argue that this is the most appropriate thing to do if the user
% is really that sensitive to spaces! In any case it is not
% implemented here as it creates an unnecessary overhead for something
% that is never used and a simple solution exists.
%
% \subsection{Experimental features}
%
% Experimental features in an experimental packages\dots{} Needless to
% say that you can't rely on them staying!
%
%
% One new feature is what I decided to call ``pseudo
% arguments''. Strictly speaking these aren't real arguments but
% instead read by removing a begin group token and then using
% |\tex_aftergroup:D| to regain control. The definition of |\footnote|
% in plain \TeX\ uses this so that people can use
% |\verb| and the like inside the footnotes. The command
% |\DeclarePseudoArgument|\DescribeMacro{\DeclarePseudoArgument} takes
% four arguments: |#1| is a name for the pseudo argument, |#2| is the
% number of arguments it takes, |#3| is the action to be taken right
% before the left brace is removed and |#4| is what to be done when
% the right brace has been read and we regain control. Note that both
% |#3| and |#4| may use arguments! Here is a silly example:
% \begin{verbatim}
% \DeclarePseudoArgument{boxtest}{1}
% {Before:~`#1',\hbox_set_inline_begin:N \l_tmpa_box }
% {\hbox_set_inline_end: \text_put_sp:  the~ box:~
%   \hbox_unpack_and_clear:N\l_tmpa_box ,~
%   After:~`#1'}
% \end{verbatim}
% Now when you want to use it, you simply call it with the macro
% |\UsePseudoArgument|\DescribeMacro{\UsePseudoArgument}, which as its
% first argument takes a name and the remaining depend on how many
% arguments the pseudo argument was defined to have. Hence we could
% use our |boxtest| definition like this:
% \begin{verbatim}
% \DeclareDocumentCommand\sillyboxtest{m}{
%   Testing~#1:~\UsePseudoArgument{boxtest}{#1}
% }
% \end{verbatim}
%   Then calling |\sillyboxtest{AB}{a\verb*+% $%&\+b}| produces
%     \begin{quote}
%       Testing AB: Before: `AB', the box:
%       a\verb*+% $%&\+b, After: `AB'
%     \end{quote}
%     |\UsePseudoArgument| should always come last in the document
%     command definition.
%
% \StopEventually{}
%
% \section{Implementation}
%
%  Versions prior to and including 1.19 contained several different
%  implementations. These are gone now but can be looked at by
%  getting an older version of the package from the CVS repository.
%
%  First the required packages.
%    \begin{macrocode}
%<*package>
\ProvidesExplPackage
  {\filename}{\filedate}{\fileversion}{\filedescription}
\RequirePackage{l3tlp,l3num,l3toks,l3prg,l3int,l3seq,l3token}
%    \end{macrocode}
%
%
%
%  \subsection{Error and warning messages}
%
%  Here we define the error message we're going to use in this package.
%
%  \begin{macro}{\xparse_already_defined_error_msg:N}
%  \begin{macro}{\xparse_not_yet_defined_error_msg:N}
%  \begin{macro}{\xparse_begins_with_end_error_msg:N}
%  \begin{macro}{\xparse_unknown_arg_type_error_msg:N}
%  \begin{macro}{\xparse_number_of_arguments_error_msg:Nn}
%  \begin{macro}{\xparse_no_command_implementation_warning:n}
%  The names should give the function's purpose away.
%    \begin{macrocode}
\def_new:NNn \xparse_already_defined_error_msg:N 1 {
  \xparse_error:x {
    Command~name~`\token_to_str:N #1'~ already~defined!
  }
}
\def_new:NNn \xparse_not_yet_defined_error_msg:N 1 {
  \xparse_error:x {
    Command~`\token_to_str:N #1'~ not~ yet~defined!
  }
}
\def_new:NNn \xparse_begins_with_end_error_msg:N 1 {
  \xparse_error:x {
    Command~`\token_to_str:N #1'~begins~with~
    `\token_to_str:N \end'!
  }
}
%    \end{macrocode}
%  This one is for when we've encountered an unknown argument type in
%  the argument specification. We can try to recover by putting in an
%  |m| type instead and hope for the best.
%    \begin{macrocode}
\def_new:NNn \xparse_unknown_arg_type_error_msg:N 1{
  \xparse_error:x {
    Unknown~ argument~ type~ `#1'~
    I'll~ substitute~ it~ with~ `m'~ for~ now.~ Fingers~ crossed...
  }
}
%    \end{macrocode}
%  A message for people using the wrong number of arguments.
%    \begin{macrocode}
\def_new:NNn \xparse_no_command_implementation_warning:n 1 {
  \xparse_warning:x {No~ implementation~ for~ `#1'~ defined}
}
%    \end{macrocode}
%  \end{macro}
%  \end{macro}
%  \end{macro}
%  \end{macro}
%  \end{macro}
%  \end{macro}
%
%  \begin{macro}{\xparse_error:x}
%  \begin{macro}{\xparse_warning:x}
%  Here's how we produce the error messages and warnings currently.
%  This awaits a proper error message module!
%    \begin{macrocode}
\def_new:NNn \xparse_error:x 1{\tex_errmessage:D {xparse~error:~#1}}
\def_new:NNn \xparse_warning:x 1{\io_put_term:x{xparse~warning:~#1}}
%    \end{macrocode}
%  \end{macro}
%  \end{macro}
%
%
%
%  \subsection{Checking for valid command names}
%
%
%  \begin{macro}{\xparse_if_definable:NT}
%  \begin{macro}{\xparse_if_definable:cT}
%  A definable command is either |\c_undefined| or |\scan_stop:| and
%  furthermore we won't allow it to start with |\end|.
%  If the command is definable we do what was asked for otherwise
%  we give an appropriate error message.
%    \begin{macrocode}
\def_new:NNn \xparse_if_definable:NTF 1 {
%    \end{macrocode}
%  First check if the control sequence is free.
%    \begin{macrocode}
  \cs_if_free:NTF #1
%    \end{macrocode}
%  If free check if it begins with |\end|.
%    \begin{macrocode}
  {
    \xparse_begins_with_end:NTF #1
    { \xparse_begins_with_end_error_msg:N #1 \use_ii:nn }
    \use_i:nn
  }
%    \end{macrocode}
%  If not free give an error message.
%    \begin{macrocode}
  { \xparse_already_defined_error_msg:N #1 \use_ii:nn }
}
\def_new:NNn \xparse_if_definable:cTF 0 {
  \exp_args:Nc \xparse_if_definable:NTF
}
%    \end{macrocode}
%  \end{macro}
%  \end{macro}
%  \begin{macro}{\xparse_if_redefinable:NTF}
%  \begin{macro}{\xparse_if_redefinable:cTF}
%  A re-definable command is almost the same as above but here we
%  demand that it already exists.
%    \begin{macrocode}
\def_new:NNn \xparse_if_redefinable:NTF 1 {
  \cs_if_free:NTF #1
  { \xparse_not_yet_defined_error_msg:N #1 \use_ii:nn }
  {
    \xparse_begins_with_end:NTF #1
    { \xparse_begins_with_end_error_msg:N #1 \use_ii:nn }
    \use_i:nn
  }
}
\def_new:NNn \xparse_if_redefinable:cTF  0 {
  \exp_args:Nc \xparse_if_redefinable:NTF
}
%    \end{macrocode}
%  \end{macro}
%  \end{macro}
%
%
%  \begin{macro}{\xparse_begins_with_end:NTF}
%  \begin{macro}{\xparse_begins_with_end_aux:N}
%  Finally we need the magic function for checking if the control
%  sequence begins with |\end|.
%  We wish to determine if the control sequence in question begins with
%  |\end| which is forbidden. The end goal is to get a control sequence
%  consisting of the first three letters to test against |\end| with
%  |\cs_if_eq_name_p:NN|. The function |\cs_to_str:N| will remove the |\|
%  from a command name and return the rest of the characters with
%  category code 12. Thus using an |f| type expansion to get the first
%  three characters is perfectly safe as it'll stop right there. Then
%  we turn these three letters into a control sequence with a simple
%  expansion which is then tested with |\xparse_begins_with_end_aux:N|.
%    \begin{macrocode}
\def_new:NNn \xparse_begins_with_end:NTF 1 {
%    \end{macrocode}
%  We do it all in a group as there is no need to fill up the hash
%  table with weird three letter control sequences.
%    \begin{macrocode}
  \group_begin:
    \exp_args:Nc \xparse_begins_with_end_aux:N {
      \tlist_head_iii:f { \cs_to_str:N #1 ??}
    }
}
%    \end{macrocode}
%  After the above expansion tricks we now have a real control sequence
%  to test against |\end|. We end the group again.
%    \begin{macrocode}
\def_new:NNn \xparse_begins_with_end_aux:N 1 {
  \if:w \cs_if_eq_name_p:NN #1 \end
    \group_end:
    \exp_after:NN \use_i:nn
  \else:
    \group_end:
    \exp_after:NN \use_ii:nn
  \fi:
}
%    \end{macrocode}
%  Note that these definitions make no use of temporary variables.
%  \end{macro}
%  \end{macro}
%
%
%
%
%
%  \subsection{The low level definitions}
%
%
%  First some helper functions
%  \begin{macro}{\l_xparse_grabbed_args_toks}
%  \begin{macro}{\l_xparse_end_environment_args_toks}
%  \begin{macro}{\l_xparse_mandatory_args_int}
%  \begin{macro}{\l_xparse_total_args_int}
%  We need some \meta{token} registers to keep track of things. We also
%  allocate an \meta{int} register to keep track of the number of
%  arguments.
%    \begin{macrocode}
\toks_new:N \l_xparse_grabbed_args_toks
\toks_new:N \l_xparse_end_environment_args_toks
\int_new:N \l_xparse_mandatory_args_int
\int_new:N \l_xparse_total_args_int
%    \end{macrocode}
%  \end{macro}
%  \end{macro}
%  \end{macro}
%  \end{macro}
%
%
%  \begin{macro}{\xparse_declare_document_command:Nnn}
%  \begin{macro}{\xparse_declare_document_command:cnn}
%  |\xparse_declare_document_command:Nnn| is a two step procedure. The
%  user level command |\|\meta{cmd} contains the argument grabbers
%  while the internal command |\\|\meta{cmd} holds the actual
%  definition of what to do with the arguments. Naturally this calls
%  for every such user level command to be robust which is done using
%  the \eTeX{} protection feature.
%    \begin{macrocode}
\def_long:NNn \xparse_declare_document_command:Nnn 3{
%    \end{macrocode}
%  First we prepare the signature.
%    \begin{macrocode}
  \xparse_prepare_signature:n {#2}
%    \end{macrocode}
%  Now for the document command. Make it robust and make sure it starts
%  with |\toks_set:Nn \l_xparse_grabbed_args_toks {\\|\meta{cmd}|}|.
%    \begin{macrocode}
  \def_protected:Npx #1 {
    \exp_not:n { \toks_set:Nn \l_xparse_grabbed_args_toks }
    {\exp_not:c {\token_to_str:N #1}}
%    \end{macrocode}
% Then add the argument grabbers which contain code gathered when
% preparing the signature.
%    \begin{macrocode}
    \toks_use:N \l_xparse_grabbed_args_toks
%    \end{macrocode}
% Finally execute |\toks_use:N \l_xparse_grabbed_args_toks| which now
% holds the grabbed arguments enclosed in braces and starts with the
% internal command |\\|\meta{cmd}.
%    \begin{macrocode}
    \exp_not:n{ \toks_use:N \l_xparse_grabbed_args_toks}
  }
%    \end{macrocode}
%  Define the internal command by making it a |\long| macro with number
%  of arguments equal to |\l_xparse_total_args_int| and definition as
%  given by |#3|. Even though this internal macro allows |\par| in its
%  argument the decision is left to the individual argument grabbers.
%    \begin{macrocode}
  \def_long:cNn {\token_to_str:N #1}\l_xparse_total_args_int{#3}
}
\def_new:Npn \xparse_declare_document_command:cnn {
  \exp_args:Nc \xparse_declare_document_command:Nnn
}
%    \end{macrocode}
%  \end{macro}
%  \end{macro}
%
%  \begin{macro}{\xparse_declare_document_environment:nnnn}
%  |\xparse_declare_document_environment:nnnn| is almost the same. In
%  order to use the grabbed arguments we take them from
%  |\l_xparse_grabbed_args_toks| just before they are executed. We
%  also insert a |\group_begin:| as it will otherwise fail miserably
%  in case a user tries to use it without |\begin ... \end|.
%    \begin{macrocode}
\def_long_new:NNn \xparse_declare_document_environment:nnnn 4 {
  \xparse_declare_document_command:cnn {#1}{#2}
  { \group_begin:
      \toks_set_eq:NN \l_xparse_end_environment_args_toks
                      \l_xparse_grabbed_args_toks
      #3
  }
%    \end{macrocode}
%  We let |\end|\meta{envir} equal to a standard version which will do
%  all the work.
%    \begin{macrocode}
  \let:cN {end #1} \xparse_parsed_end_environment:
%    \end{macrocode}
%  Now define |\end\\|\meta{envir}.
%    \begin{macrocode}
  \def_long:cNn {end \token_to_str:N \\ #1}
    \l_xparse_total_args_int{#4}
}
%    \end{macrocode}
%  \end{macro}
%
%  \begin{macro}{\xparse_parsed_end_environment:}
%  \begin{macro}{\xparse_parsed_end_environment_aux:N}
%  |\end|\meta{envir} is set equal to |\xparse_parsed_end_environment:|
%  which then executes |\end\\|\meta{envir}. The |\\|\meta{envir} comes
%  from the argument grabbing. The remainder of the token list is the
%  grabbed arguments. We better make this robust as well.
%    \begin{macrocode}
\def_protected_new:NNn \xparse_parsed_end_environment: 0{
  \exp_after:NN \xparse_parsed_end_environment_aux:N
    \toks_use:N \l_xparse_end_environment_args_toks
  \group_end:
}
\def_new:NNn \xparse_parsed_end_environment_aux:N 1{
  \use:c {end \token_to_str:N #1 }
}
%    \end{macrocode}
%  \end{macro}
%  \end{macro}
%
%
%  \subsection{The high level commands}
%
%
%  Here we set up the functions for defining and redefining document
%  commands and environments. It's all rather straight forward here so
%  I've saved on the comments.
%
%  \begin{macro}{\DeclareDocumentCommand}
%  \begin{macro}{\NewDocumentCommand}
%  \begin{macro}{\RenewDocumentCommand}
%  In case we can't do what the user wanted we gobble the next
%  arguments from the input stream. For commands there are two
%  arguments waiting.
%    \begin{macrocode}
\let_new:NN \DeclareDocumentCommand  \xparse_declare_document_command:Nnn 
\def_new:NNn \NewDocumentCommand 1 {
  \xparse_if_definable:NTF #1
  { \xparse_declare_document_command:Nnn #1 }
  \use_none:nn
}
\def_new:NNn \RenewDocumentCommand 1 {
  \xparse_if_redefinable:NTF #1
  { \xparse_declare_document_command:Nnn #1 }
  \use_none:nn
}
%    \end{macrocode}
%  \end{macro}
%  \end{macro}
%  \end{macro}
%
%  \begin{macro}{\DeclareDocumentEnvironment}
%  \begin{macro}{\NewDocumentEnvironment}
%  \begin{macro}{\RenewDocumentEnvironment}
%  Three arguments for environments.
%    \begin{macrocode}
\let_new:NN \DeclareDocumentEnvironment
              \xparse_declare_document_environment:nnnn
\def_new:NNn \NewDocumentEnvironment 1 {
  \xparse_if_definable:cTF {#1}
  { \xparse_declare_document_environment:nnnn {#1} }
  \use_none:nnn
}
\def_new:NNn \RenewDocumentEnvironment 1 {
  \xparse_if_redefinable:cTF {#1}
  { \xparse_declare_document_environment:nnnn {#1} }
  \use_none:nnn
}
%    \end{macrocode}
%  \end{macro}
%  \end{macro}
%  \end{macro}
%
%  \begin{macro}{\ProvideDocumentCommand}
%  \begin{macro}{\ProvideDocumentEnvironment}
%  When providing a command we just check if the control sequence is
%  free. If it is let |\DeclareDocument...| do the rest, otherwise
%  gobble the next arguments.
%    \begin{macrocode}
\def_new:NNn \ProvideDocumentCommand 1{
  \cs_if_free:NTF #1
  { \DeclareDocumentCommand #1}
  \use_none:nn
}
\def_new:NNn \ProvideDocumentEnvironment 1{
  \cs_if_free:cTF {#1}
  { \DeclareDocumentEnvironment {#1} }
  \use_none:nnn
}
%    \end{macrocode}
%  \end{macro}
%  \end{macro}
%
%
%  \subsubsection{A class designer interface}
%
%  Below is the implementation of |\DeclareDocumentCommandInterface|
%  and the close cousin |\DeclareDocumentCommandImplementation|.
%  It is an open issue whether this concept should be supported.
%
%  \begin{macro}{\DeclareDocumentCommandInterface}
%  With this macro we define the document command |#1| to have
%  signature |#3| which does the argument grabbing. The arguments are
%  passed on to the internal command |\impl-#2|.
%    \begin{macrocode}
\def_long_new:NNn \DeclareDocumentCommandInterface 3{
  \xparse_prepare_signature:n {#3}
  \def_protected:Npx #1 {
    \exp_not:n { \toks_set:Nn \l_xparse_grabbed_args_toks }
    {\exp_not:c {impl-#2}}
    \toks_use:N\l_xparse_grabbed_args_toks
    \exp_not:n{ \toks_use:N \l_xparse_grabbed_args_toks}
  }
%    \end{macrocode}
%  Define the internal command by making it a |\long| macro as we did
%  for |\DeclareDocumentCommand| but give it just a default definition
%  instead (usually an error message).
%    \begin{macrocode}
  \def_long:cNn {impl-#2} \l_xparse_total_args_int
    {\xparse_undefined_command_implementation:n{#2}}
}
%    \end{macrocode}
%  \end{macro}
%  \begin{macro}{\xparse_undefined_command_implementation:n}
%  What to do if no implementation is made and the command is
%  called. Let's just typeset the name in quotes indicating that
%  something is wrong.
%    \begin{macrocode}
\def_new:NNn \xparse_undefined_command_implementation:n 1{
  ``#1''
%  \ensuremath{\langle\textit{#1}\rangle}
  \xparse_no_command_implementation_warning:n {#1}
}
%    \end{macrocode}
%  \end{macro}
%
%  \begin{macro}{\DeclareDocumentCommandImplementation}
%  Defines the internal command |\impl-#1| with |#2| arguments
%  and definition |#3|.
%    \begin{macrocode}
\def_long_new:NNn \DeclareDocumentCommandImplementation 3{
  \def_long:cNn {impl-#1}#2{#3}
}
%    \end{macrocode}
% \end{macro}
%
%
%  \subsection{Creating the signature}
%
%
% \begin{macro}{\xparse_prepare_signature:n}
%   The actual preparation of the signature is always the same but we
%   have different ways of handling it afterwards, so we make it a
%   separate function.
%    \begin{macrocode}
\def_new:NNn \xparse_prepare_signature:n 1 {
%    \end{macrocode}
% Initialize the counter taking care of the number of arguments. In
% case we reach more than nine we will give an error message later on.
%    \begin{macrocode}
  \int_zero:N \l_xparse_total_args_int
%    \end{macrocode}
% Clear the token register we use when building the signature and the
% number of ``normal'' mandatory arguments. Also clear out the special
% markers and initialize the two booleans to \meta{false}.
%    \begin{macrocode}
  \toks_clear:N \l_xparse_grabbed_args_toks
  \int_zero:N \l_xparse_mandatory_args_int
  \bool_gset_false:N \g_xparse_insert_marker_bool
  \bool_gset_false:N \g_xparse_allow_par_bool
  \tlp_gset_eq:NN \g_xparse_ignore_marker_tlp 
                  \g_xparse_ignore_spaces_marker_tlp
%    \end{macrocode}
%  Then call |\xparse_parse_signature:n|.
%    \begin{macrocode}
  \xparse_parse_signature:n #1 \q_nil
}
%    \end{macrocode}
%  \end{macro}
%
%
%  \subsection{The argument types}
%
%
%
%  All argument types must be denoted by single letters.\footnote{You
%  can use a multi-letter sequence to denote a shorthand but it
%  requires you to put two brace groups around it in the signature and
%  it's not officially supported.} We have three categories of letters:
%  Special markers, basic argument types, and shorthands.
%
%  For certain argument types where we peek ahead in the token stream
%  we have a multitude of choices for how the macro should work:
%  Should it ignore spaces or other tokens? Which test should it use
%  for comparing tokens: meaning, catcode or charcode? Because of
%  these difficulties every argument that require peeking ahead must
%  be declared as one of the three test types.
%
%  \subsubsection{Special markers}
%
%  Since there are different markers that may be used at the same
%  time, we use the a special ``insert'' marker |>| to do this.
%
% \begin{macro}{\g_xparse_ignore_marker_tlp}
% The current ignore marker is contained in this string.
%    \begin{macrocode}
\tlp_new:Nn \g_xparse_ignore_marker_tlp {}
%    \end{macrocode}
% \end{macro}
%
% \begin{macro}{\g_xparse_ignore_marker_seq}
%   We start out by declaring a new sequence to contain all the
%   different types of ignore functions. Each element should be the
%   missing part of a |\peek_meaning|\meta{text}|:NTF| function.
%    \begin{macrocode}
\seq_new:N \g_xparse_ignore_marker_seq
%    \end{macrocode}
% \end{macro}
%
% \begin{macro}{\g_xparse_insert_marker_bool}
% A boolean to check if we just saw an insert marker.
%    \begin{macrocode}
\bool_new:N \g_xparse_insert_marker_bool
%    \end{macrocode}
% \end{macro}
%
% \begin{macro}{\xparse_add_arg_type_>:}
%   The insert marker is called when this is seen. Is has a funny
%   definition but it works. |#1| is some test in the loop function
%   and |#2| is the list of markers.
%    \begin{macrocode}
\def_new:cpn {xparse_add_arg_type_>:} #1 
  \xparse_read_arg_type_or_grab_default:n #2{
%    \end{macrocode}
% Insert pending |m| args first.
%    \begin{macrocode}
  \xparse_add_remaining_m_args:
%    \end{macrocode}
% Then we set the boolean true and subtract one from the argument
% count.
%    \begin{macrocode}
  \bool_gset_true:N \g_xparse_insert_marker_bool
  \int_decr:N \l_xparse_total_args_int
%    \end{macrocode}
% Then go through the list and call each subtype if it exists.
%    \begin{macrocode}
  \tlist_map_inline:nn{#2}{
    \xparse_check_and_add_argument_type:N ##1
  }
%    \end{macrocode}
% Finally call the loop again.
%    \begin{macrocode}
  \xparse_parse_signature:n
}
%    \end{macrocode}
% \end{macro}
%
% \begin{macro}{\xparse_add_ignore_marker:Nnn}
% A small function for adding a new ignore type marker.
%    \begin{macrocode}
\def_new:NNn \xparse_add_ignore_marker:Nnn 3{
  \tlp_new:cn {g_xparse #2 _marker_tlp}{#3}
  \seq_gpush:NC \g_xparse_ignore_marker_seq {g_xparse #2 _marker_tlp}
  \def_new:cpn {xparse_add_arg_type_#1:}{
    \tlp_gset_eq:Nc \g_xparse_ignore_marker_tlp {g_xparse #2 _marker_tlp}
  }
}
%    \end{macrocode}
% \end{macro}
% Here we add some ignore markers. 
% \begin{macrocode}
\xparse_add_ignore_marker:Nnn W{_ignore_nothing}{}
\xparse_add_ignore_marker:Nnn i{_ignore_spaces}{_ignore_spaces}
\xparse_add_ignore_marker:Nnn I{_ignore_pars}{_ignore_pars}
%    \end{macrocode}
%
% \begin{macro}{\xparse_add_arg_type_P:}
% This is simple, just set a boolean.
% \begin{macrocode}
\def_new:NNn \xparse_add_arg_type_P: 0{
  \bool_gset_true:N \g_xparse_allow_par_bool
}
%    \end{macrocode}
% \end{macro}
%
% \begin{macro}{\g_xparse_allow_par_bool}
% The boolean for allowing the |\par| token.
%    \begin{macrocode}
\bool_new:N \g_xparse_allow_par_bool
%    \end{macrocode}
% \end{macro}
%
%
%  \begin{macro}{\xparse_parse_signature:n}
%  |\xparse_parse_signature:n| reads the signature one token/brace
%  group at a time and builds a list of argument grabbers which is
%  stored in |\l_xparse_grabbed_args_toks|. The function is recursive
%  which is why we use a quark to delimit it.
%    \begin{macrocode}
\def_new:NNn \xparse_parse_signature:n 1{
%    \end{macrocode}
%  Check if we reached the end of the signature.
%    \begin{macrocode}
  \quark_if_nil:NTF #1
%    \end{macrocode}
% If we did we may have pending |m| type arguments so we clear them
% out here. Also check if the insert marker is true because if it is,
% something is wrong. Otherwise we just continue.
%    \begin{macrocode}
  { 
    \xparse_add_remaining_m_args: 
  }
%    \end{macrocode}
% Then proceed with adding arguments. 
%    \begin{macrocode}
  {
    \int_incr:N \l_xparse_total_args_int
    \xparse_check_and_add_argument_type:N #1
%    \end{macrocode}
% If the insert marker is true at this point we reset it again and
% also disallow |\par| in arguments and make all |\peek| functions use
% the |_ignore_spaces| versions.
%    \begin{macrocode}
    \bool_if:NT \g_xparse_insert_marker_bool
    {
      \bool_gset_false:N \g_xparse_insert_marker_bool
      \bool_gset_false:N \g_xparse_allow_par_bool    
      \tlp_gset_eq:NN \g_xparse_ignore_marker_tlp 
                      \g_xparse_ignore_spaces_marker_tlp
    }
%    \end{macrocode}
% After adding arguments we better set the boolean false in case we
% don't have an insert marker.
%    \begin{macrocode}
%    \end{macrocode}
% First we add the remaining |m| arguments.
%    \begin{macrocode}
%    \end{macrocode}
%  Then we run the loop again.
%    \begin{macrocode}
    \xparse_read_arg_type_or_grab_default:n
  }
}
%    \end{macrocode}
%  \end{macro}
%
%
%  \begin{macro}{\xparse_check_and_add_argument_type:N}
%  This function checks if the argument type actually exists and gives
%  an error if it doesn't.
%    \begin{macrocode}
\def_new:NNn \xparse_check_and_add_argument_type:N 1 {
  \cs_if_free:cTF {xparse_add_arg_type_#1:}
  { \xparse_unknown_arg_type_error_msg:N #1
    \int_incr:N \l_xparse_mandatory_args_int
  }
%    \end{macrocode}
%  Otherwise we just add it with its dedicated function.
%    \begin{macrocode}
  { \use:c {xparse_add_arg_type_#1:} }
}
%    \end{macrocode}
%  \end{macro}
%
%
%
%  \begin{macro}{\xparse_read_arg_type_or_grab_default:n}
%  \begin{macro}{\xparse_grab_default_arg:n}
%  \begin{macro}{\xparse_grab_default_arg_allow_par:n}
%    The function for grabbing a default value. The default setting is
%    just to tun the loop again. However, some argument types need to
%    add a default argument so we add these as well.
%    \begin{macrocode}
\let_new:NN \xparse_read_arg_type_or_grab_default:n
            \xparse_parse_signature:n
\def_new:NNn \xparse_grab_default_arg:n 1{
  \toks_put_right:Nn \l_xparse_grabbed_args_toks {{#1}}
  \let:NN \xparse_read_arg_type_or_grab_default:n
          \xparse_parse_signature:n
  \xparse_parse_signature:n
}
\def_long_new:NNn \xparse_grab_default_arg_allow_par:n 1{
  \toks_put_right:Nn \l_xparse_grabbed_args_toks {{#1}}
  \let:NN \xparse_read_arg_type_or_grab_default:n
          \xparse_parse_signature:n
  \xparse_parse_signature:n
}
%    \end{macrocode}
%  \end{macro}
%  \end{macro}
%  \end{macro}
%
%
%
%
%
%
%  \subsubsection{The default argument types}
%
%
%
%  When scanning the signature we need to add the argument type which
%  is done by a function |\xparse_add_arg_type_|\meta{X}|:|, where
%  \meta{X} is the letter denoting the argument type. If not found an
%  error message is issued ny |\xparse_unknown_arg_type_error_msg:N|.
%
%  By default the package implements the argument types  |m|, |S|, |c|,
%  |C|, |o|, and |O| argument types as low level ones. The |s| type is
%  a shorthand as we shall see later.
%
%  As we don't make every single |m| argument a separate grabber all
%  other types must clear out any pending |m|s.
%
%  \paragraph{Mandatory arguments}
%
%  Below follows the implementation of mandatory arguments.
%
%  \begin{macro}{\xparse_add_arg_type_m:}
%    Check if the user asked for an |m| argument allowing |\par|
%    tokens and insert it. Otherwise increment the number of
%    non-|\long| argument grabbers.
%    \begin{macrocode}
\def_new:Npn \xparse_add_arg_type_m: {
  \bool_if:NTF \g_xparse_allow_par_bool
  {
    \toks_put_right:Nn \l_xparse_grabbed_args_toks {\xparse_allow_par_m:w}
  }
  { \int_incr:N \l_xparse_mandatory_args_int }
}
%    \end{macrocode}
%  \end{macro}
%
%  \begin{macro}{\xparse_add_remaining_m_args:}
%    And here is the function for clearing out normal pending |m|
%    arguments. There is little point in adding any arguments if none
%    are required.
%    \begin{macrocode}
\def_new:Npn \xparse_add_remaining_m_args: {
  \int_compare:nNnF \l_xparse_mandatory_args_int = \c_zero
  {
    \toks_put_right:Nx \l_xparse_grabbed_args_toks {
      \exp_not:c{xparse_m
        \int_use:N \l_xparse_mandatory_args_int
        :w }
    }
    \int_zero:N \l_xparse_mandatory_args_int
  }
}
%    \end{macrocode}
%  \end{macro}
%
%
%  \begin{macro}{\xparse_m1:w}
%  \begin{macro}{\xparse_m2:w}
%  \begin{macro}{\xparse_m3:w}
%  \begin{macro}{\xparse_m4:w}
%  \begin{macro}{\xparse_m5:w}
%  \begin{macro}{\xparse_m6:w}
%  \begin{macro}{\xparse_m7:w}
%  \begin{macro}{\xparse_m8:w}
%  \begin{macro}{\xparse_m9:w}
%  Grabbing 1--9 mandatory arguments. The one grabbing nine arguments
%  will automatically end with
%  |\toks_use:N \l_xparse_grabbed_args_toks| anyway so we avoid
%  problems with that one.
%    \begin{macrocode}
\def_new:cpn {xparse_m1:w} #1 \l_xparse_grabbed_args_toks#2{
  \toks_put_right:Nn \l_xparse_grabbed_args_toks{{#2}}
  #1 \l_xparse_grabbed_args_toks
}
\def_new:cpn {xparse_m2:w} #1 \l_xparse_grabbed_args_toks #2#3{
  \toks_put_right:Nn \l_xparse_grabbed_args_toks{{#2}{#3}}
  #1 \l_xparse_grabbed_args_toks
}
\def_new:cpn {xparse_m3:w} #1 \l_xparse_grabbed_args_toks #2#3#4{
  \toks_put_right:Nn \l_xparse_grabbed_args_toks{{#2}{#3}{#4}}
  #1 \l_xparse_grabbed_args_toks
}
\def_new:cpn {xparse_m4:w} #1 \l_xparse_grabbed_args_toks #2#3#4#5{
  \toks_put_right:Nn \l_xparse_grabbed_args_toks{{#2}{#3}{#4}{#5}}
  #1 \l_xparse_grabbed_args_toks
}
\def_new:cpn {xparse_m5:w} #1 \l_xparse_grabbed_args_toks #2#3#4#5#6{
  \toks_put_right:Nn \l_xparse_grabbed_args_toks{{#2}{#3}{#4}{#5}{#6}}
  #1 \l_xparse_grabbed_args_toks
}
\def_new:cpn {xparse_m6:w} #1 \l_xparse_grabbed_args_toks #2#3#4#5#6#7{
  \toks_put_right:Nn \l_xparse_grabbed_args_toks
                     {{#2}{#3}{#4}{#5}{#6}{#7}}
  #1 \l_xparse_grabbed_args_toks
}
\def_new:cpn {xparse_m7:w} #1 \l_xparse_grabbed_args_toks#2#3#4#5#6#7#8{
  \toks_put_right:Nn \l_xparse_grabbed_args_toks
                     {{#2}{#3}{#4}{#5}{#6}{#7}{#8}}
  #1 \l_xparse_grabbed_args_toks
}
\def_new:cpn {xparse_m8:w} #1\l_xparse_grabbed_args_toks
                                                       #2#3#4#5#6#7#8#9{
  \toks_put_right:Nn \l_xparse_grabbed_args_toks
                     {{#2}{#3}{#4}{#5}{#6}{#7}{#8}{#9}}
  #1 \l_xparse_grabbed_args_toks
}
\def_new:cpn {xparse_m9:w} \toks_use:N \l_xparse_grabbed_args_toks
                                                     #1#2#3#4#5#6#7#8#9{
  \toks_put_right:Nn \l_xparse_grabbed_args_toks
                     {{#1}{#2}{#3}{#4}{#5}{#6}{#7}{#8}{#9}}
  \toks_use:N \l_xparse_grabbed_args_toks
}
%    \end{macrocode}
%  \end{macro}
%  \end{macro}
%  \end{macro}
%  \end{macro}
%  \end{macro}
%  \end{macro}
%  \end{macro}
%  \end{macro}
%  \end{macro}
%
%  \begin{macro}{\xparse_allow_par_m:w}
%  The |m| type allowing |\par| tokens. We only define one of them as
%  two in a row would be quite rare.
%    \begin{macrocode}
\def_long_new:Npn \xparse_allow_par_m:w #1 \l_xparse_grabbed_args_toks#2{
  \toks_put_right:Nn \l_xparse_grabbed_args_toks{{#2}}
  #1 \l_xparse_grabbed_args_toks
}
%    \end{macrocode}
%  \end{macro}
%
% \paragraph{Arguments delimited by a left brace}
%
% \begin{macro}{\xparse_add_arg_type_l:}
% \begin{macro}{\xparse_l:w}
% \begin{macro}{\xparse_allow_par_l:w}
%   This is almost exactly the same as the |m| type except we read
%   everything up to the first left brace.
%    \begin{macrocode}
\def_new:Npn \xparse_add_arg_type_l: {
  \xparse_add_remaining_m_args:
  \bool_if:NTF \g_xparse_allow_par_bool
  {
    \toks_put_right:Nn \l_xparse_grabbed_args_toks {\xparse_allow_par_l:w}
  }
  { \toks_put_right:Nn \l_xparse_grabbed_args_toks {\xparse_l:w} }
}
\def_new:Npn \xparse_l:w #1 \l_xparse_grabbed_args_toks#2#{
  \toks_put_right:Nn \l_xparse_grabbed_args_toks{{#2}}
  #1 \l_xparse_grabbed_args_toks
}
\def_long_new:Npn \xparse_allow_par_l:w #1 \l_xparse_grabbed_args_toks#2#{
  \toks_put_right:Nn \l_xparse_grabbed_args_toks{{#2}}
  #1 \l_xparse_grabbed_args_toks
}
%    \end{macrocode}
% \end{macro}
% \end{macro}
% \end{macro}
%
%  \paragraph{Adding a symbol argument type}
%
%
%  \begin{macro}{\DeclareSymbolArgument}
% The actual grabber functions just issue a |\peek_meaning_remove:NTF|
% or similar on the symbol chosen. If found, we put a |\c_true| on the
% argument stack and |\c_false| otherwise.
%    \begin{macrocode}
\def_new:NNn \DeclareSymbolArgument 2{
  \seq_map_variable:NNn \g_xparse_ignore_marker_seq \l_tmpa_tlp {
    \def:cpx {xparse \l_tmpa_tlp _#1:w}##1##2\l_xparse_grabbed_args_toks{
      \exp_not:c{peek_ #2 _remove \l_tmpa_tlp :NTF} ##1
      { 
        \exp_not:N \toks_put_right:Nn \exp_not:N
        \l_xparse_grabbed_args_toks \exp_not:N \c_true
        ##2 \exp_not:N\l_xparse_grabbed_args_toks
      }
      { 
        \exp_not:N \toks_put_right:Nn \exp_not:N
        \l_xparse_grabbed_args_toks \exp_not:N \c_false
        ##2 \exp_not:N\l_xparse_grabbed_args_toks
      }
    }
  }
%    \end{macrocode}
%  Implementing a symbol type is fairly straight forward. It needs to
%  grab a default argument. First we must clear out any pending |m|
%  arguments.
%    \begin{macrocode}
  \def:cpn {xparse_add_arg_type_#1:} {
    \xparse_add_remaining_m_args:
    \toks_put_right:Nx \l_xparse_grabbed_args_toks
    { \exp_not:c {xparse \g_xparse_ignore_marker_tlp _#1:w } }
%    \end{macrocode}
% Nothing special when building the signature: We always grab a single
% default argument.
%    \begin{macrocode}
    \let:NN \xparse_read_arg_type_or_grab_default:n \xparse_grab_default_arg:n
  }
}
%    \end{macrocode}
% It is this easy now:
%    \begin{macrocode}
\DeclareSymbolArgument S{meaning}
%    \end{macrocode}
% \end{macro}
%
%
%  \paragraph{Adding delimited arguments}
%
% 
%  \begin{macro}{\DeclareArgumentType}
%    Argument types that read their argument delimited as the |o| type
%    can be defined with this function. |#1| is the letter associated
%    with the type, |#3| is the test type we should use, i.e.,
%    meaning, charcode or catcode. |#2| is the left delimiter \LaTeX{}
%    will look for, |#4| can be used for an error message or such in
%    case of a missing value, and |#5| the value inserted in case the
%    optional argument is missing.  The last two arguments control how
%    the arguments are picked up and parsed on to the inner parsing.
%    \begin{macrocode}
\def_new:NNn \DeclareArgumentType 7{
%    \end{macrocode}
% For instance, we want |\DeclareArgumentType a{charcode}[...| to
% define the function |\xparse_ignore_spaces_a:w| with meaning
% |\peek_charcode_ignore_spaces:NTF [{...}{...}|.  We want to define a
% series of functions for each of these delimited argument types. One
% for each of the |ignore| types plus an identical version of each of
% these allowing the |\par| token in the argument. All the different
% ignore types are stored in the sequence
% |\g_xparse_ignore_marker_seq| so we go through that when doing the
% definitions.
%    \begin{macrocode}
  \seq_map_variable:NNn \g_xparse_ignore_marker_seq \l_tmpa_tlp {
    \def:cpx {xparse \l_tmpa_tlp _#1:w}##1\l_xparse_grabbed_args_toks{
      \exp_not:c{peek_#3 \l_tmpa_tlp :NTF} \exp_not:N #2
      { \exp_not:c{xparse_#1_#3_help:nw}{##1} }
      { 
        \exp_not:n {
          #4 \toks_put_right:Nn \l_xparse_grabbed_args_toks {#5}
        }
        ##1 \exp_not:N\l_xparse_grabbed_args_toks
      }
    }
%    \end{macrocode}
% And now an almost identical version which allows the |\par| token.
%    \begin{macrocode}
    \def_long:cpx {xparse_allow_par \l_tmpa_tlp _#1:w}##1
    \l_xparse_grabbed_args_toks{
      \exp_not:c{peek_#3 \l_tmpa_tlp :NTF} \exp_not:N #2
      { \exp_not:c{xparse_allow_par_#1_#3_help:nw}{##1} }
      { 
        \exp_not:n {
          #4 \toks_put_right:Nn \l_xparse_grabbed_args_toks {#5} 
        }
        ##1 \exp_not:N\l_xparse_grabbed_args_toks
      }
    }
  }
%    \end{macrocode}
% Here is the function for building the argument grabbing. We clear
% out the remaining |m| arguments first. 
%    \begin{macrocode}
  \def:cpn {xparse_add_arg_type_#1:} {
    \xparse_add_remaining_m_args:
    \toks_put_right:Nx \l_xparse_grabbed_args_toks {
      \exp_not:c {xparse 
%    \end{macrocode}
% Insert |_allow_par| in the csname if needed. This also applies to
% the argument grabber (although this should probably just be the
% default).
%    \begin{macrocode}
        \bool_if:NT \g_xparse_allow_par_bool {_allow_par} 
        \g_xparse_ignore_marker_tlp        
        _#1:w
      }
    }
    \let:NN \xparse_read_arg_type_or_grab_default:n \xparse_parse_signature:n
  }
%    \end{macrocode}
% Now all we need to do is to define the helper functions.
%    \begin{macrocode}
  \xparse_define_helper:Nnnn #1{#3}{#6}{#7}
}
%    \end{macrocode}
%  \end{macro}
%
%  \begin{macro}{\DeclareArgumentTypeDefaultValue}
%  The same as above but expects a default value in the signature.
%  Therefore it doesn't have the arguments of the default value and
%  what to do if it is missing like the above.
%    \begin{macrocode}
\def_new:NNn \DeclareArgumentTypeDefaultValue 5{
  \seq_map_variable:NNn \g_xparse_ignore_marker_seq \l_tmpa_tlp {
    \def:cpx {xparse \l_tmpa_tlp _#1:w}##1##2\l_xparse_grabbed_args_toks{
      \exp_not:c{peek_#3 \l_tmpa_tlp :NTF} \exp_not:N #2
      { \exp_not:c{xparse_#1_#3_help:nw}{##2} }
      { 
        \exp_not:N \toks_put_right:Nn 
        \exp_not:N \l_xparse_grabbed_args_toks {{##1}}
        ##2 \exp_not:N \l_xparse_grabbed_args_toks
      }
    }
%    \end{macrocode}
% And now an almost identical version which allows the |\par| token.
%    \begin{macrocode}
    \def_long:cpx {xparse_allow_par \l_tmpa_tlp _#1:w}##1##2
    \l_xparse_grabbed_args_toks{
      \exp_not:c{peek_#3 \l_tmpa_tlp :NTF} \exp_not:N #2
      { \exp_not:c{xparse_allow_par_#1_#3_help:nw}{##2} }
      { 
        \exp_not:N \toks_put_right:Nn
        \exp_not:N \l_xparse_grabbed_args_toks {{##1}}
        ##2 \exp_not:N \l_xparse_grabbed_args_toks
      }
    }
  }
%    \end{macrocode}
% Here is the function for building the argument grabbing. We clear
% out the remaining |m| arguments first. 
%    \begin{macrocode}
  \def:cpn {xparse_add_arg_type_#1:} {
    \xparse_add_remaining_m_args:
    \toks_put_right:Nx \l_xparse_grabbed_args_toks {
      \exp_not:c {xparse 
%    \end{macrocode}
% Insert |_allow_par| in the csname if needed. This also applies to
% the argument grabber (although this should probably just be the
% default).
%    \begin{macrocode}
        \bool_if:NT \g_xparse_allow_par_bool {_allow_par} 
        \g_xparse_ignore_marker_tlp        
        _#1:w
      }
    }
    \let:Nc \xparse_read_arg_type_or_grab_default:n 
    {xparse_grab_default_arg
      \bool_if:NT \g_xparse_allow_par_bool {_allow_par} 
      :n
    }
  }
%    \end{macrocode}
% Now all we need to do is to define the helper functions.
%    \begin{macrocode}
  \xparse_define_helper:Nnnn #1{#3}{#4}{#5}
}
%    \end{macrocode}
% \end{macro}
%
%  \begin{macro}{\xparse_define_helper:Nnnn}
%  All we need now is the helper functions. We define a generic
%  interface that should be fairly easy to use.
%    \begin{macrocode}
\def_new:NNn \xparse_define_helper:Nnnn 4{
  \toks_set:Nn \l_tmpa_toks
  {
    #3
    {
%    \end{macrocode}
% Remember to add an extra set of braces here.
%    \begin{macrocode}
      \toks_put_right:Nn \l_xparse_grabbed_args_toks {{#4}}
      ##1 \l_xparse_grabbed_args_toks
    }
  }
%    \end{macrocode}
%  The next bit looks a little ugly but that's life.
%    \begin{macrocode}
  \toks_set:Nx \l_tmpa_toks {
    \exp_not:n {\def:cpn{xparse_#1_#2_help:nw}}
    \toks_use:N \l_tmpa_toks
    \exp_not:n {\def:cpn{xparse_allow_par_#1_#2_help:nw}}
    \toks_use:N\l_tmpa_toks
  }
  \toks_use:N \l_tmpa_toks
}
%    \end{macrocode}
%  \end{macro}
%
%  All the above may seem like an awful lot of trouble but it is
%  general and it allows the following simple definitions:
%    \begin{macrocode}
\DeclareArgumentType o[{meaning}{}{\NoValue}{#1[#2]}{#2}
%    \end{macrocode}
%  For the |c| type we add give an error message.
%    \begin{macrocode}
\DeclareArgumentType c({meaning}{
  \xparse_error:x{
    Missing~ coordinate~ argument.~ A~ value~ of~ (0,0)~ is~ assumed}
  }
  {{00}}
  {#1(#2,#3)}{{#2}{#3}}
\DeclareArgumentTypeDefaultValue O[{meaning}{#1[#2]}{#2}
\DeclareArgumentTypeDefaultValue C({meaning}{#1(#2,#3)}{{#2}{#3}}
%    \end{macrocode}
% One could also do
% \begin{verbatim}
% \DeclareArgumentTypeDefaultValue O[{charcode}{#1#2#3]}{#3}
% \end{verbatim}
% where argument |#2| is the bracket but not requiring it to have
% catcode~12. However we can't use the same trick for reading the end
% of the argument.
%
% Another interesting one could be
% \begin{verbatim}
% \DeclareArgumentType a\c_group_begin_token{catcode}{}{\NoValue}{#1#2}{#2}
% \end{verbatim}
% which would basically be an optional argument in braces.
%
%
%
%  \subsubsection{Argument shorthands}
%
%
%  \begin{macro}{\DeclareArgumentTypeShorthand}
%    Letting a letter or symbol be a shorthand for a more complicated
%    structure is rather easy actually. All we need to do is to gobble
%    the remainder of the main loop in the signature creation until
%    |\xparse_parse_signature:n|, decrement the argument counter, put
%    in the replacement argument specifiers and start the loop
%    again. The disadvantage of this scheme is that we have to cheat
%    and give it the wrong name as it should have a |w| argument
%    specification but I hope you can forgive this little white lie.
%    \begin{macrocode}
\def_new:NNn \DeclareArgumentTypeShorthand 2{
  \def_new:cpn {xparse_add_arg_type_#1:}
    ##1 \xparse_read_arg_type_or_grab_default:n {
    \int_decr:N \l_xparse_total_args_int
    \xparse_read_arg_type_or_grab_default:n #2
  }
}
%    \end{macrocode}
%  \end{macro}
%  \begin{macro}{\xparse_add_arg_type_s:}
%  Defining the |s| type is easy now since it really means |S{*}|.
%    \begin{macrocode}
\DeclareArgumentTypeShorthand s {S{*}}
%    \end{macrocode}
%  \end{macro}
%
%
%  \subsection{Checking for optional values}
%
%
%  \begin{macro}{\IfBooleanTF}
%  \begin{macro}{\IfBooleanT}
%  \begin{macro}{\IfBooleanF}
%    The logical \meta{true} and \meta{false} statements are just our
%    normal |\c_true| and |\c_false| so testing for them is done with
%    our usual |\bool_if:NTF| functions from \textsf{l3prg}.
%    \begin{macrocode}
\let_new:NN \IfBooleanTF \bool_if:NTF
\let_new:NN \IfBooleanT  \bool_if:NT
\let_new:NN \IfBooleanF  \bool_if:NF
%    \end{macrocode}
%  \end{macro}
%  \end{macro}
%  \end{macro}
%
%  \begin{macro}{\NoValue}
%  \begin{macro}{\c_xparse_hidden_no_value_tlp}
%    |\NoValue| is just a text string and we define it as a token list
%    pointer. However we will actually test for token list pointers
%    with \emph{meaning} |\NoValue|. Hence if an argument is really
%    |-NoValue-| it will not be detected but that shouldn't happen.
%    \begin{macrocode}
\tlp_new:Nn \NoValue {-NoValue-}
\tlp_new:Nn \c_xparse_hidden_no_value_tlp {\NoValue}
%    \end{macrocode}
%  \end{macro}
%  \end{macro}
%
%  \begin{macro}{\xparse_if_no_value:nTF}
%    The test start by making a token list pointer out of the first
%    argument and then check if it is equal to
%    |\c_xparse_hidden_no_value_tlp| which contains |\NoValue|.
%    \begin{macrocode}
\def_long_new:Npn \xparse_if_no_value:nTF #1{
  \tlp_set:Nx \l_tmpa_tlp{\exp_not:n{#1}}
  \tlp_if_eq:NNTF \l_tmpa_tlp \c_xparse_hidden_no_value_tlp
%    \end{macrocode}
% If they are equal we just exit and execute the \meta{true} code.
%    \begin{macrocode}
  { \use_i:nn }
%    \end{macrocode}
% If not, take a closer look at |#1|. Peek ahead at the first token in
% |#1| and then call the function |\xparse_if_no_value_aux:| but only
% if |#1| is not empty or a blank space: in that case we can execute
% the \meta{false} code immediately. We must use this test at some
% point otherwise the macros for checking tokens will break since the
% argument may be empty.
%    \begin{macrocode}
  { \tlist_if_blank:nTF {#1}
    { \use_ii:nn }
    {\peek_after:NN \xparse_if_no_value_aux: #1 \q_nil {#1} }
  }
}
%    \end{macrocode}
%  \end{macro}
%
%  \begin{macro}{\xparse_if_no_value_aux:}
%    Now we simply check the argument specification of the token. If
%    it is not a macro the function |\token_get_arg_spec:N| returns
%    |\scan_stop:|.
%    \begin{macrocode}
\def_long_new:Npn \xparse_if_no_value_aux: {
  \tlp_set:Nx \l_tmpa_tlp{\token_get_arg_spec:N \l_peek_token }
%    \end{macrocode}
% Then check if |\l_tmpa_tlp| is empty, because this means we have a
% macro taking zero arguments and we can expand it once
% safely. Otherwise we just exit and execute the false code. Remember
% that we have the sequence |{#1}| waiting after the |\q_nil| in
% |\xparse_if_no_value:nTF| above.
%    \begin{macrocode}
  \tlp_if_empty:NTF \l_tmpa_tlp
  {\use_i_delimit_by_q_nil:nw {\exp_args:No\xparse_if_no_value:nTF}}
  {\use_i_delimit_by_q_nil:nw {\use_iii:nnn}}
}
%    \end{macrocode}
%  \end{macro}
%
%
%  \begin{macro}{\IfNoValueTF}
%  \begin{macro}{\IfNoValueT}
%  \begin{macro}{\IfNoValueF}
%  \begin{macro}{\IfValueTF}
%  \begin{macro}{\IfValueT}
%  \begin{macro}{\IfValueF}
%  |\IfNoValueTF| and its varieties are implemented as subcases of
%  |\xparse_if_no_value:nTF|.
%    \begin{macrocode}
\let_new:NN \IfNoValueTF \xparse_if_no_value:nTF 
\def_long_new:NNn \IfNoValueT 2 {\xparse_if_no_value:nTF{#1}{#2}{}}
\def_long_new:NNn \IfNoValueF 1 {\xparse_if_no_value:nTF {#1}{}}
%    \end{macrocode}
%  For |\IfValueTF| we just reverse the arguments.
%    \begin{macrocode}
\def_long_new:NNn \IfValueTF 3{\xparse_if_no_value:nTF {#1}{#3}{#2}}
\let_new:NN \IfValueT \IfNoValueF
\let_new:NN \IfValueF \IfNoValueT
%    \end{macrocode}
%  \end{macro}
%  \end{macro}
%  \end{macro}
%  \end{macro}
%  \end{macro}
%  \end{macro}
%
% 
% \subsection{Pseudo arguments}
%
% \begin{macro}{\l_xparse_pseudo_post_arg_tlp}
% A temporary token list pointer to store the post-arguments.
%    \begin{macrocode}
\tlp_new:Nn \l_xparse_pseudo_post_arg_tlp {}
%    \end{macrocode}
% \end{macro}
% \begin{macro}{\DeclarePseudoArgument}
%   Must be called at the end of the replacement text.  Can use the
%   parameters picked up so far. |#1| is a name for it, |#2| is the
%   number of arguments, |#3| is the pre-argument definition, |#4| is
%   the post-argument definition. The idea here is to define two
%   functions: One to execute before the special pseudo argument is
%   found and one to to it afterwards. Hence we define such two
%   functions with |#2| arguments and definitions |#3| and |#4|
%   respectively.
%    \begin{macrocode}
\def_long_new:NNn \DeclarePseudoArgument 4{
  \def_long:cNn {xparse_pseudo_pre_arg_#1:\prg_replicate:nn{#2}{n}}#2{#3}
  \def_long:cNn {xparse_pseudo_post_arg_#1:\prg_replicate:nn{#2}{n}}#2{#4}
%    \end{macrocode}
% In case the user doesn't put the argument in braces, it must be a
% single token which we just grab and then execute the
% post-argument. Surely it won't be a |\par| token!
%    \begin{macrocode}
   \def:cNn {xparse_pseudo_nobrace_arg_#1:N} 1 {
    ##1 \l_xparse_pseudo_post_arg_tlp
  }
%    \end{macrocode}
% The regular version of the command is then this:
% \begin{enumerate}
% \item Execute the pre-code at level $n$.
% \item Store the post-code in a token list pointer at level $n$.
% \item Look for a begin-group token.
% \item[4a] If found, remove it, start a group (level $n+1$) and make
%   sure the token list pointer from 2) is executed after the group
%   has ended (at level $n$).
% \item[4b] If not found, call the |nobrace_arg| version which also
%   executes the token list pointer from 2) at level $n$.
% \end{enumerate}
% This scheme enables these commands to be nested since they all
% appear at different grouping levels.
%
% Next we define the function. Also make sure that it gets all the
% arguments carried over to the post-argument.
%    \begin{macrocode}
  \def_long:cNx {xparse_pseudo_arg_#1:w} #2
  {
    \exp_not:c {
      xparse_pseudo_pre_arg_#1: \prg_replicate:nn{#2}{n}
    }
    \use:c{def_aux_use_\int_use:N \int_eval:n{#2}_parameter:}
    \exp_not:n {\tlp_set:Nn \l_xparse_pseudo_post_arg_tlp}
    {
      \exp_not:c {xparse_pseudo_post_arg_#1:\prg_replicate:nn{#2}{n}}
      \use:c{def_aux_use_\int_use:N \int_eval:n{#2}_parameter:}
    }
    \exp_not:n{ \peek_catcode_remove_ignore_spaces:NTF \c_group_begin_token }
    {
%    \end{macrocode}
% Here we insert the left brace token which was removed removed.
% In case you're wondering then yes, I know that |\c_group_begin_token|
% and |\group_execute_after:N| are unexpandable but I prefer this method
% because I don't want to remember which commands are expandable and 
% which are not.
%    \begin{macrocode}
     \exp_not:n {
        \c_group_begin_token 
        \group_execute_after:N \l_xparse_pseudo_post_arg_tlp
      }
    }
    {
      \exp_not:c{xparse_pseudo_nobrace_arg_#1:N}
    }
  }
}
%    \end{macrocode}
% \end{macro}
% \begin{macro}{\UsePseudoArgument}
%   Just a nice wrapper for calling such an argument.
%    \begin{macrocode}
\def_new:NNn \UsePseudoArgument 1{\use:c{xparse_pseudo_arg_#1:w}}
%    \end{macrocode}
% \end{macro}
%
%
%
%
% \Finale
%
\endinput
