% \iffalse
%%
%% (C) Copyright 1999 Frank Mittelbach, Chris Rowley, David Carlisle
%% All rights reserved.
%%
%% Not for distribution
%% 
%<*dtx>
          \ProvidesFile{xparse.dtx}
%</dtx>
%<package>\NeedsTeXFormat{LaTeX2e}
%<package>\ProvidesPackage{xparse}
%<driver>\ProvidesFile{xparse.drv}
% \fi
%         \ProvidesFile{xparse.dtx}
          [1999/07/23 v0.12 generic document command parser]
%
% \iffalse
%<*driver>
 \documentclass{ltxdoc}
 \usepackage{textcomp}
% \usepackage{ldcdoc}
 \begin{document}
 \DocInput{xparse.dtx}
 \end{document}
%</driver>
% \fi
%
% \GetFileInfo{xparse.dtx}
%
% \title{The \textsf{xparse} package\thanks{This file
%         has version number \fileversion, last
%         revised \filedate.}}
% \author{FMi, CAR, DPC}
% \date{\filedate}
%  \maketitle
%
% \begin{abstract}
% The interfaces described in this document are not meant to be final
% but only as a basis for discussion
% \end{abstract}
%
% \section{Interface}
%
% This package implements high-level interface commands for class file
% writers which allows the separation of formatting commands
% (typically instances of so-called `templates') and their arguments
% from the signature of document-level commands.
%
% This works by declaration commands that provide a general
% specification method for the typical \LaTeX{} syntax, e.g.,
% star-form, optional arguments, and mandatory arguments. A command
% (or environment) declared in this way parses the input according to
% its spec and presents its findings in a normalized way for further
% processing.
%
% \subsection{Argument spec}
%
% An argument specification is a list of letters each representing a
% type of argument, i.e., |m| is a mandatory argument (surrounded by
% braces), |o| an optional argument (surrounded by backets if
% present), and |s| represents a star (which might be present or
% not).\footnote{Other argument types such as \texttt{c} for picture
% coordinates could be integrated in principle --- the last prototype
% now contains support for\texttt{c}.} Thus the argument
% spec for headings as implemented by |\@startsection| in standard
% \LaTeX{} would be represented by the three letters |som|.
%
% With the new proof of concept implementation there is also
% |O{default}| which acts like |o| which also scans for an optional
% argument but allows to specify a default value if the optional
% argument is not present. I.e., it can be used to turn mandatory
% arguments of a template (that is those that do not check for
% |\NoValue|) into optional ones by supplying a default on the
% document command level.
%
% \subsection{Parsing results}
%
% To normalise the result of parsing the input according to an
% argument specification it is important to uniquely identify all
% arguments found. For this reason each parsing operation initiated by
% one of the argument spec letters will result in an identifiable
% output as follows:
% \begin{description}
% \item[m] will return the parsed argument surrounded by a brace pair,
%   i.e., will normally be the identity;
% \item[o] will return the parsed argument surrounded by a brace pair
%   if present. Otherwise it will return the token |\NoValue|;
% \item[O] will return the parsed argument surrounded by a brace pair
%   if present. Otherwise it will return the |{default}| as specified above.
% \item[s] will return either the token |\BooleanTrue| or
%   |\BooleanFalse| depending on whether or not a star was parsed.
% \end{description}
% For example, given the spec |soomO{default}| the input |*[Foo]{Bar}| would be
% parsed as |\BooleanTrue{Foo}\NoValue{Bar}{default}|. In other words
% there will be always exactly the same number of brace groups or
% tokens as the number of letters in the argument spec.
%
% \subsection{Applying the parsing results}
% 
% Since the result of the parsing is a welldefined number of tokens or
% brace groups it is easy to pass them on in any order to any
% processing function. To this end the tokens or brace groups are
% associated with the standard argument specifiers in \TeX{} macros,
% i.e., |#1|, |#2|, and so forth. This limits the argument
% specification to a maximum number of 9 letters, but for practical
% applications this should be sufficient.
%
% \subsection{The class designer interface}
%
% This package provides two commands for declaring commands
% and environments to be used within the document body.
%
% \DescribeMacro\DeclareDocumentCommand
% The |\DeclareDocumentCommand| declaration takes three arguments. The
% first argument is the name of the command to be declared, the second
% is the argument specification in the syntax described above, and the
% third is the action to be carried out once the arguments are
% parsed. Within the third argument |#1|, |#2|, etc.\ denote the
% result of the parsing, e.g.,
%\begin{verbatim}
%  \DeclareDocumentCommand{\chapter}{som}
%    { \IfBooleanTF {#1}
%        { \typesetnormalchapter {#2}{#3} }
%        { \typesetstarchapter   {#3} }
%    }
%\end{verbatim}
% would be a way to define a |\chapter| command which would
% essentially behave like the current \LaTeX{} command (except that it
% would accept an optional argument even when a |*| was parsed). The
% |\typesetnormalchapter| could test its first argument for being
% |\NoValue| to see if an optional argument was present.
%
% Of course something like the |\IfNoValueTF| test could also be
% placed inside a function that would process all three arguments, thus
% using the templates and their instances as provided by the
% \texttt{template} package such a declaration would probably look
% more like the following example:
%\begin{verbatim}
%  \DeclareDocumentCommand{\chapter}{som}
%    { \UseInstance {head} {A-head-main} {#1} {#2} {#3} }
%\end{verbatim}
%
% Using the |\DeclareDocumentCommand| interface it is easy to modify
% the document-level syntax while still applying the same
% layout-generating functions, e.g., a class that would not support
% optional arguments or star forms for the heading commands could
% define |\chapter| like this:
%\begin{verbatim}
%  \DeclareDocumentCommand{\chapter}{m}
%    { \UseInstance {head} {A-head-main} \BooleanFalse \NoValue {#1} }
%\end{verbatim}
% while a class that would allow for an additional optional argument
% (for whatever reason) could define it like that:
%\begin{verbatim}
%  \DeclareDocumentCommand{\chapter}{somo}
%    { \doSomethingWithTheExtraOptionalArg {#4}
%      \UseInstance {head} {A-head-main} {#1} {#2} {#3} }
%\end{verbatim}
%
% Commands declared in this way are automatically robust (in David's
% implementation).
%
% \DescribeMacro\DeclareDocumentEnvironment
% The |\DeclareDocumentEnvironment| declaration is similar to
% |\DeclareDocumentCommand| execpt that it takes four arguments: the
% first being the environment name (without a backslash), the second
% again the argument-spec, and the third and forth are the actions
% taken at start end end of the environment. The parsed arguments are
% available to both the start and the finish as |#1|, |#2|,
% etc.\footnote{It is wishful thinking that there args are available
% to the end of the environment. In the current implementation this
% was buggy and has been taken out again below.}
%
%
% \subsubsection{Comparing tokens in a quarky way}
%
% Something like |\NoValue| would perhaps be best implemented as a
% ltx3 quark, i.e. a token which expands to itself as this is
% can be easily tested even if hidden inside a macro. The unfortunate
% sideeffect however is that it will result in a tight loop if it ever
% gets executed by mistake.
%
% \DescribeMacro\NoValue
% For this reason |\NoValue| is defined to expand to the string
% |-NoValue-| which would get typeset if ever executed thus clearly
% indicating the type of error the writer made.
%
% However this makes testing for this token slightly complicated as in
% that case the test
%\begin{verbatim}
%   \def\seen{#1}
%   \def\containsNoValue{\NoValue}
%   \ifx\seen\containsNoValue
%\end{verbatim}
% will be true if |#1| was |\NoValue| but false if if |#1| itself
% contains a macro which contains |\NoValue|; a case that happens
% unfortunately very often in pratice.  Using unguarded |\edef| to
% define |\seen| is out of question as |#1| typically is either
% |\NoValue| or arbitrary user input (which is likely to die horribly
% even in a |\protected@edef|.
%
% \DescribeMacro\IfSomethingTF 
% \DescribeMacro\IfSomethingT 
% \DescribeMacro\IfSomethingF
% Therefore the |\IfSomethingTF| test uses a slower recursive
% procedure: it tests if |#1| and |#2| are equal. If they are testing
% stops and the true-case (argument |#3|) will be executed. If not, it
% expands the first token of |#2| and checks if the resulting
% token-list is identical to |#2| (this would happen if this first
% token is unexpandable or |#2| is empty). In the latter case testing
% terminates and the false-case (argument |#4|) will be
% executed. However if the token-list created this way differs from
% |#2| the macro recurses using this token-list in place of |#2|.
%
% \DescribeMacro\IfNoValueTF 
% \DescribeMacro\IfNoValueT
% \DescribeMacro\IfNoValueF
% \DescribeMacro\IfValueTF 
% \DescribeMacro\IfValueT 
% \DescribeMacro\IfValueF 
% The test |\IfNoValueTF| is imlpemented as an application of
% |\IfSomethingTF|, i.e.,
%\begin{verbatim}
%   \def\IfNoValueTF{\IfSomethingTF\NoValue}
%\end{verbatim}
% where the three arguments (test token, true and false case) are
% picked up by |\IfSomethingTF|. Similar tests for other quark-like
% tokens could be defined similarly. |\IfValueTF| is the same test
% with the true/false case exchanged.
%
%
% \subsection{Some comments on the need for the \texttt{O} specifier}
%
% With |\newcommand| there is the possibility of specifiying a default
% for an optional argument which is stored away in a more or less
% efficient manner. For example below is the old definition of
% |\linebreak| as can be found in the \LaTeX2e kernel:
%\begin{verbatim}
%\def\linebreak{\@testopt{\@no@lnbk-}4}
%\def\@no@lnbk #1[#2]{%
%  \ifvmode
%    \@nolnerr
%  \else
%    \@tempskipa\lastskip
%    \unskip
%    \penalty #1\@getpen{#2}%
%    \ifdim\@tempskipa>\z@
%      \hskip\@tempskipa
%      \ignorespaces
%    \fi
%  \fi}
%\end{verbatim}
% Ignoring for the moment that the above is slightly optimised an
% expansion of this code under |\tracingall| will result in about 90
% lines of tracing output.
% If we reimplement this using |\DeclarDocumentCommand\linebreak{o}|
% we have to use |\IfNoValueTF| to find out if an argument was present
% which (because of the somewhat slow expansion of |\IfSomethingTF|)
% results in about twice as much of tracing lines. In contrast using
% |O{4}| as below we end up with 110 lines, which seems roughly the
% price we have to pay for the extra generality available (though this
% could perhaps even be reduced by a better implementation of the
% parsing machine as originally done by David, before i talked him
% into adding support for arguments in the end code of an environment).
%
%\begin{verbatim}
%\DeclareDocumentCommand\linebreak { O{4} }
% {
%  \ifvmode
%    \@nolnerr
%  \else
%    \@tempskipa\lastskip
%    \unskip
%%    \IfNoValueTF{#1}
%%       {\break}
%%       {\penalty -\@getpen{#1}}
%    \penalty -\@getpen{#1}
%    \ifdim\@tempskipa>\z@
%      \hskip\@tempskipa
%      \ignorespaces
%    \fi
%  \fi
% }
%\end{verbatim}
%
% \subsubsection{A boolean data type}
%
% \DescribeMacro\BooleanTrue
% \DescribeMacro\BooleanFalse
% The parsing result for a star etc is presented as the token
% |\BooleanTrue| or |\BooleanFalse| respectively!
%
% \DescribeMacro\IfBooleanTF
% \DescribeMacro\IfBooleanT
% \DescribeMacro\IfBooleanF
% To test for these values the macro |\IfBooleanTF| can be used. It
% expects as its first argument either |\BooleanTrue| or
% |\BooleanFalse| and executes its second or third argument depending
% on this value. |\IfBooleanT| and |\IfBooleanF| are obvious
% shortcuts. 
%
% At one point in time i thought that one can represent everything
% using |\NoValue|, eg for the star case either return |*| or
% |\NoValue|. However, this slows down processing of commands like
% |\\*| considerably since they would then have to use the slow
% |\IfSomethingTF| internally instead of a fast twoway switch. So now
% this data type is back in.
%
%
% \StopEventually{}
% \CheckSum{0}
%
% \section{Implementation}
%
% Set up certain defaults including to ignore white space
% within the body of this package.
%    \begin{macrocode}
%<*package>
\RequirePackage{ldcsetup}
\IgnoreWhiteSpace
%    \end{macrocode}
%
%

% \subsection{The Mittelbach/Rowley prove-of-concept
%             implementation for the parsing}
%
% \begin{macro}{\DeclareDocumentCommand}
%    \begin{macrocode}
%<*obsolete>
\def\DeclareDocumentCommand #1 #2  {
%  \parse@countargs {#2} % sets \@tempc    % Not implemented yet
  \def\parse@csname{#1}
  \def\parse@spec{#2}
  \afterassignment \declare@dt@
  \toks@
}
%    \end{macrocode}
% \end{macro}
%
% \begin{macro}{\declare@dt@}
%    \begin{macrocode}
\def \declare@dt@ {
  \expandafter \edef \parse@csname
    { \noexpand \parse@something
        { \parse@spec }
        { \the\toks@ }
        % { \@tempc }
    }
}
%    \end{macrocode}
% \end{macro}
%
% \begin{macro}{\DeclareDocumentEnvironment}
%    \begin{macrocode}
\long\def\DeclareDocumentEnvironment #1 #2  {
%  \parse@countargs {#2} % sets \@tempc    % Not implemented yet
  \def\parse@csname{#1}
  \def\parse@spec{#2}
  \afterassignment \declare@de@
  \toks@
}
%    \end{macrocode}
% \end{macro}
%
% \begin{macro}{\declare@de@}
%    \begin{macrocode}
\def \declare@de@ {
  \expandafter \edef \csname \parse@csname \endcsname
    { \begingroup
      \noexpand \parse@something
        { \parse@spec } 
        { \the\toks@ }
        % { \@tempc }
    }
  \afterassignment \declare@dee@
  \toks@
}
%    \end{macrocode}
% \end{macro}
%
% \begin{macro}{\declare@dee@}
%    \begin{macrocode}
\def \declare@dee@ {
  \expandafter \edef \csname end\parse@csname \endcsname
    { \noexpand \parse@apply
        { \the\toks@ }
        \noexpand \parse@results
      \endgroup
    }
}
%    \end{macrocode}
% \end{macro}
%
% \begin{macro}{\parse@apply}
%    \begin{macrocode}
\def\parse@apply #1 {
      \renewcommand \parse@tempa [\the\parse@cnt] {#1}
      \expandafter \parse@tempa %    #2 implicit
}
\let\parse@tempa\@empty % or \renew... might bulk the first time
%    \end{macrocode}
% \end{macro}
%
% \begin{macro}{\parse@something}
%    \begin{macrocode}
%% CCC here and elsewhere there are some suggestions for
%      not counting args at run-time
\def \parse@something #1 #2 {  % #3
  \parse@body {#2}
  % \parse@cnt #3
  \parse@cnt \m@ne  % REMOVE
  \parse@args #1x \parse@args {}
}
%    \end{macrocode}
%
%    \begin{macrocode}
\newcount\parse@cnt
\newtoks\parse@body
%    \end{macrocode}
%
%    \begin{macrocode}
\def\parse@args #1 #2\parse@args #3 {
  \advance \parse@cnt \@ne % REMOVE
  \csname parse@#1 \endcsname {#2}{#3}
}
%    \end{macrocode}
%
%    \begin{macrocode}
\def \parse@s #1 #2 {
  \@ifstar
      { \parse@args #1 \parse@args { #2 \BooleanTrue } }
      { \parse@args #1 \parse@args { #2 \BooleanFalse } }
}
%    \end{macrocode}
%
%    \begin{macrocode}
\def \parse@m #1 #2 #3 {
      \parse@args #1 \parse@args { #2 {#3} }
}
%    \end{macrocode}
%
%    \begin{macrocode}
\def \parse@x #1 #2 { 
      \def \parse@results {#2}   % for environment
%      \def \parse@tempa {
%        \renewcommand \parse@tempb [\the\parse@cnt] 
%      } 
%      \expandafter \parse@tempa \expandafter { \the \parse@body } 
%      \parse@tempb #2
      \expandafter \parse@apply \expandafter { \the \parse@body } 
                            \parse@results % \parse@apply expands this!
}
%    \end{macrocode}
%
%    \begin{macrocode}
\def \parse@o #1 #2 {
  \@ifnextchar [
      { \parse@o@ {#1}{#2} }
      { \parse@args #1 \parse@args { #2 \NoValue } }
}
%    \end{macrocode}
%
%    \begin{macrocode}
\def \parse@o@ #1 #2 [#3] {
      \parse@args #1 \parse@args { #2 {#3} }
}
%</obsolete>
%    \end{macrocode}
% \end{macro}
%
%
% \subsection{The Carlisle implementation for the parsing}
%
% Not implemented in this version is |\DeclareDocumentEnvironment|
% (excerise for the reader).\footnote{Exercise now completed by dpc :-)
%    (required re-implementing argument grabbers to grab to a toks
%     register rather that into a brace group in the input stream.) }
%
% seemed to need two tokse registers.
%    \begin{macrocode}
\newtoks\@temptokenb
%    \end{macrocode}
%
%    \begin{macrocode}
\newtoks\xparsed@args
%    \end{macrocode}
%
%
% |\DeclareDocumentCommand|
% |#1| csname\\
% |#2| soom argument spec\\
% |#3| code
%
% in |#2| currently supported types are:\\
%     s star\\
%     o \oarg{optional}\\
%     m \marg{mandatory}
%
% |#3| is just grabbed so as to not get a space in the argument spec
%    for def.
%
%    \begin{macrocode}
\long\def\DeclareDocumentCommand #1 #2 #3{
%    \end{macrocode}
%
% needed to count no of arguments
%    \begin{macrocode}
   \@tempcnta\z@
%    \end{macrocode}
%
% builds up list of argument parsers |\@ddc@s\@ddc@m| etc
%    \begin{macrocode}
   \toks@{}
%    \end{macrocode}
%
% builds up |#1#2#3#4| argument spec
%    \begin{macrocode}
   \@temptokena\toks@
%    \end{macrocode}
%
% builds up list of m-argument parsers |\@ddc@m\@ddc@m| 
% occurred since start or since argument of another type.
%    \begin{macrocode}
   \@temptokenb\toks@
%    \end{macrocode}
%
% Start parsing argument spec
%    \begin{macrocode}
   \@ddc#2X
%    \end{macrocode}
%
% Define top level command, this just has the wrapper command
% |\@ddc@| then the argument grabbers from |\toks@| then the original
% command name (in case we need to |\protect|) then the internal
% command with the code.
% The |\long| below could be |\relax| if wanted to have a star-non-long
% form of this, cf newcommand*. The argument preamble for
% |\def| comes from |\@temptokena|.
%    \begin{macrocode}
   \edef#1{
    \noexpand\@ddc@
    {\the\toks@}
    \expandafter\noexpand\csname\string#1\endcsname
    \noexpand#1
    }
   \long\expandafter\def\csname\string#1\expandafter\endcsname
          \the\@temptokena{#3}}
%    \end{macrocode}
%
%
% |\DeclareDocumentEnvironment|\\
% The implementation here could save a csname or two per environment
% if |\begin| and especially |\end| were modified but that not done here
% so each end code responsible for getting its own arguments.\\
% |#1| env name\\
% |#2| soom spec.\\
% |#3| begin code\\
% |#4| end code\\
%    \begin{macrocode}
\long\def\DeclareDocumentEnvironment#1#2#3#4{
  \expandafter\DeclareDocumentCommand\csname #1\endcsname{#2}{
    \xparsed@args\toks@
    #3}
   \expandafter\let\csname end #1\endcsname\@parsed@endenv
  \long\expandafter\def\csname end \string\\#1\expandafter\endcsname\the\@temptokena
    {#4}}
%    \end{macrocode}
%
% All end codes are let to this. (Could be merged into |\end|.)
%    \begin{macrocode}
\def\@parsed@endenv{
  \expandafter\@parsed@endenv@\the\xparsed@args}
%    \end{macrocode}
%
% Helper that just replaces the internal name of the begin code
% with that of the end code.
%    \begin{macrocode}
\def\@parsed@endenv@#1{
  \csname end\string#1\endcsname}
%    \end{macrocode}
%
%
% |\@ddc@|\\
% |#1| set of argument grabbers\\
% |#2| internal command\\
% |#3| top level command for |\protect|ing
%    \begin{macrocode}
\def\@ddc@#1#2#3{
  \ifx\protect\@typeset@protect
    \expandafter\@firstofone
  \else
    \protect#3\expandafter\@gobble
  \fi
%    \end{macrocode}
%
% The command+arguments so far are kept in a token reg until
% the last moment for ease of processing. So need to  initialise
% that and use it at the end.
%    \begin{macrocode}
  {\toks@{#2} #1\the\toks@}}
%    \end{macrocode}
%
%
% |\@ddc|\\
% |#1| one of s o m.
%    \begin{macrocode}
\def\@ddc#1{
  \ifx #1X
%    \end{macrocode}
% fini
%    \begin{macrocode}
  \else
  \ifx #1m
%    \end{macrocode}
%    If doing an m, just stick another |\@ddc@m| in a temporary list.
%    \begin{macrocode}
%  \@temptokenb\expandafter{%
%     \the\@temptokenb
%     \@ddc@m}
   \addto@hook\@temptokenb m
  \else
%    \end{macrocode}
%
% Otherwise for o and s
% in |\toks@| first add any `m' argument parsers saved up
% then add |\@ddc@o| or |\@ddc@s|.
%    \begin{macrocode}
  \toks@\expandafter{%
     \the\expandafter\toks@
     \csname @ddc@\the\@temptokenb\expandafter\endcsname
     \csname @ddc@#1\endcsname}
%    \end{macrocode}
%
% clear list of m's
%    \begin{macrocode}
   \@temptokenb{}
  \fi
%    \end{macrocode}
%
% add one to arg count.
%    \begin{macrocode}
  \advance\@tempcnta\@ne
%    \end{macrocode}
%
% Internally all arguments are non delimited args, so add
% |#|\meta{n} to the list in |\@temptokena|
%    \begin{macrocode}
  \@temptokena\expandafter{
     \the\expandafter\@temptokena\expandafter##\the\@tempcnta}
%    \end{macrocode}
%
% loop
%    \begin{macrocode}
  \expandafter
  \@ddc
%    \end{macrocode}
%
%    \begin{macrocode}
  \fi}
%    \end{macrocode}
%
%
% |\@ddc@s|\\
% |#1| any remaining argument grabbers (+ |\the|)
%    \begin{macrocode}
\long\def\@ddc@s#1\toks@{
%    \end{macrocode}
%
% put back the rest of the argument grabbers, but add new
% (boolean) argument to list of arguments inside the register.
%    \begin{macrocode}
  \@ifstar
    {\addto@hook\toks@\BooleanTrue #1\toks@}
    {\addto@hook\toks@\BooleanFalse #1\toks@}}
%    \end{macrocode}
%
%
% |\@ddc@m|\\
% |#1| any remaining argument grabbers (+ |\the|)
% |#2| argument to be grabbed.
%    \begin{macrocode}
\long\def\@ddc@m#1\toks@#2{
%    \end{macrocode}
% put back the rest of the argument grabbers, but copy
%  argument to list of arguments inside the toks register.
% Any `m' at the end will be discarded and the internal
% command will pick up its own arguments.
%    \begin{macrocode}
 \addto@hook\toks@{{#2}} #1\toks@}
%    \end{macrocode}
%
% |\@ddc@o|\\
% |#1| any remaining argument grabbers (+ |\the|)
%    \begin{macrocode}
\long\def\@ddc@o#1\toks@{
%    \end{macrocode}
%
% put back the rest of the argument grabbers, but copy
% argument (now with |{}| or |\NoValue| to list of arguments
%  inside toks register.
%    \begin{macrocode}
   \@ifnextchar[
     {\@ddc@o@{#1}}
     {\addto@hook\toks@\NoValue #1\toks@}}
%    \end{macrocode}
%
%
% helper to remove []
%    \begin{macrocode}
\long\def\@ddc@o@#1[#2]{
  \addto@hook\toks@{{#2}} #1\toks@}
%    \end{macrocode}
%
%
%
%
% \subsection{A Mittelbach update to the Carlisle implementation
%    supporting O with default for optional arg and c for coordinates
%    --- prove of concept only}
%
% Only changed and new commands are listed.
%    \begin{macrocode}
\def\@ddc#1{
  \ifx #1X
%    \end{macrocode}
%    Normally we do nothing at this point and don't pick up the
%    trailing mandatory args into |\toks@| but if we want to reuse the
%    arg list in an end environment we have to so |\perhaps@grab@ms|
%    is normally |\relax| but might be |\grab@ms| instead.
%    \begin{macrocode}
    \perhaps@grab@ms
  \else
    \ifx #1m
%    \end{macrocode}
%    Just record how many m's seen so far
%    \begin{macrocode}
      \addto@hook\@temptokenb m
    \else
%    \end{macrocode}
%    If anything other than an m is scanned we add to |\toks@| and
%    argument grabber that gets all m's in one go rather than argument
%    grabbers that pick up each m at a time. Right now this
%    unnecessarily adds |\@ddc@x| if no m's have been
%    seen.\footnote{fix, see also below}
%    \begin{macrocode}
      \toks@\expandafter{%
         \the\expandafter\toks@
         \csname @ddc@x\the\@temptokenb\expandafter\endcsname
         \csname @ddc@#1\endcsname}
      \@temptokenb{}
%    \end{macrocode}
%    In case of |O| we have to grab the default next.
%    \begin{macrocode}
      \ifx #1O
         \let\next@ddc\grab@default
      \else
        \ifx #1c
%    \end{macrocode}
%    In case of |c| we are going to parse two not one argument so we
%    have to update |\@tempcnta| and |\@temptokena| twice (second time
%    is done below.
%    \begin{macrocode}
          \advance\@tempcnta\@ne
          \@temptokena\expandafter{
          \the\expandafter\@temptokena
              \expandafter##\the\@tempcnta}
        \fi
      \fi
    \fi
    \advance\@tempcnta\@ne
    \@temptokena\expandafter{
      \the\expandafter\@temptokena\expandafter##\the\@tempcnta}
    \expandafter
    \next@ddc
  \fi
}
\let\next@ddc\@ddc
\def\grab@default #1{
      \toks@\expandafter{%
         \the\toks@
         {#1}}
  \let\next@ddc\@ddc
  \@ddc
}
\long\def\@ddc@O#1#2\toks@{
   \@ifnextchar[
     {\@ddc@o@{#2}}
     {\addto@hook\toks@{{#1}} #2\toks@}}
%    \end{macrocode}
%
%    Parsing the coordinates should most likely get some extra syntax
%    checking, eg to allow \verb*=\foo{1} (1,2)= the code simply drops
%    everything until it sees a |(| which is a bit bold\ldots
%    \begin{macrocode}
\long\def\@ddc@c#1\toks@#2(#3,#4){
 \addto@hook\toks@{{#3}{#4}} #1\toks@}
%    \end{macrocode}
%    So here is an alternative: this accepts spaces but will crash on
%    anything else with an error message.
%    \begin{macrocode}
\long\def\@ddc@c#1\toks@#2{\@dc@c@#1\toks@#2}
\long\def\@ddc@c@#1\toks@(#2,#3){
 \addto@hook\toks@{{#2}{#3}} #1\toks@}
%    \end{macrocode}
%
% Normally we don't pick up trailing m-args but if we do we grab them
% all at once.
%    \begin{macrocode}
\let\perhaps@grab@ms\relax
\def\grab@ms {
     \toks@\expandafter{
        \the\expandafter\toks@
          \csname @ddc@x\the\@temptokenb\endcsname
}}
%    \end{macrocode}
%
%
%    Instead of |\@ddc@m| we use |\@ddc@xm| as with the implementation
%    below we might have a need for |\@ddc@x| (that is no m's pending)
%    and this would result in a name clash. A potentially better
%    implementation (cause faster) would be not to use |\@temptokenb|
%    above to record the number of m's seen but a counter register and
%    generate grabber function names containing the number in their
%    name. This way one could better single out the empty case which
%    currently will always result and a |\@ddc@x| grabber doing
%    nothing.
%    \begin{macrocode}
\let\@ddc@m\undefined
\long\def\@ddc@xm#1\toks@#2{
 \addto@hook\toks@{{#2}} #1\toks@}
%    \end{macrocode}
%
%
% |\@ddc@xmm|\\
% |#1| any remaining argument grabbers (+ |\the|)
% |#2#3| argument to be grabbed; and so on. There can be at most 8
% such mandatory arguments otherwise
%    \begin{macrocode}
\long\def\@ddc@xmm#1\toks@#2#3{
%    \end{macrocode}
% put back the rest of the argument grabbers, but copy
%  argument to list of arguments inside the toks register.
% Any `m' at the end will be discarded and the internal
% command will pick up its own arguments.
%    \begin{macrocode}
 \addto@hook\toks@{{#2}{#3}} #1\toks@}
%    \end{macrocode}
%
% And here are the cases for 3 to eight upcoming mandatory args:
%    \begin{macrocode}
\long\def\@ddc@xmmm#1\toks@#2#3#4{
 \addto@hook\toks@{{#2}{#3}{#4}} #1\toks@}
\long\def\@ddc@xmmmm#1\toks@#2#3#4#5{
 \addto@hook\toks@{{#2}{#3}{#4}{#5}} #1\toks@}
\long\def\@ddc@xmmmmm#1\toks@#2#3#4#5#6{
 \addto@hook\toks@{{#2}{#3}{#4}{#5}{#6}} #1\toks@}
\long\def\@ddc@xmmmmmm#1\toks@#2#3#4#5#6#7{
 \addto@hook\toks@{{#2}{#3}{#4}{#5}{#6}{#7}} #1\toks@}
\long\def\@ddc@xmmmmmmm#1\toks@#2#3#4#5#6#7#8{
 \addto@hook\toks@{{#2}{#3}{#4}{#5}{#6}{#7}{#8}} #1\toks@}
\long\def\@ddc@xmmmmmmmm#1\toks@#2#3#4#5#6#7#8#9{
 \addto@hook\toks@{{#2}{#3}{#4}{#5}{#6}{#7}{#8}{#9}} #1\toks@}
%    \end{macrocode}
%
% If we grab arguments even if they are all mandatory we might have
% even 9 such arguments, thus we also need:
%    \begin{macrocode}
\long\def\@ddc@xmmmmmmmmm\the\toks@#1#2#3#4#5#6#7#8#9{
 \addto@hook\toks@{{#1}{#2}{#3}{#4}{#5}{#6}{#7}{#8}{#9}}\the\toks@}
%    \end{macrocode}
% And in that case we also need to provide |\@ddc@x| as that name will
% be executed if there are no trailing mandatory arguments.
%    \begin{macrocode}
\let\@ddc@x\relax
%    \end{macrocode}
%
%
%    Fixing |\DeclareDocumentEnvironment| to not choke on the spec
%    |m|. Interestingly enough |mm| does work with the old
%    implementation: it simply grabs |\@checkend{name}| whereas the
%    first only grabs |\@checkend| leaving |{name}| around for
%    typesetting :-)
%
%    If we don't make the arguments available in the end code we
%    should most likely revert to David's original code which worked
%    without assigning everything to toks registers (or at least check
%    what is faster!)
%    \begin{macrocode}
\long\def\DeclareDocumentEnvironment#1#2#3#4{
  \expandafter\DeclareDocumentCommand\csname #1\endcsname{#2}{
%    \end{macrocode}
%     After |\DeclareDocumentCommand| has parsed the arguments the
%     parsing result is available in |\@toks| except that trailing
%     |m|'s are picked up directly and their values are therefore not
%     part of this token register. So either we have to slow down
%     everything by individually parsing those as well or this bright
%     idea is not working. For the moment i disabled it again!
%    \begin{macrocode}
%    \xparsed@args\toks@
    #3}
%    \end{macrocode}
%    In that case the following can be simplified as well:
%    \begin{macrocode}
   \@namedef{end #1}{#4}
%   \expandafter\let\csname end #1\endcsname\@parsed@endenv
%   \long\expandafter\def\csname end \string\\#1\expandafter\endcsname\the\@temptokena
%     {#4}
}
%    \end{macrocode}
%    And those are no longer necessary:
%    \begin{macrocode}
\let\@parsed@endenv\undefined
\let\@parsed@endenv@\undefined
%    \end{macrocode}
%
%    The above does the fixing by not providing the arguments to the
%    end command. The alternative is to do full parsing and could
%    perhaps look more or less like this:
%    \begin{macrocode}
\long\def\DeclareDocumentEnvironment#1#2#3#4{
  \let\perhaps@grap@ms\grab@ms
  \expandafter\DeclareDocumentCommand\csname #1\endcsname{#2}{
%    \end{macrocode}
%    I think we need as a safety measure add a group here or else the
%    code will fail over if people use it without |\begin|\ldots
%    |\end|.
%    Or not? I don't really like it but then i don't want to maintain
%    a private stack here.
%    \begin{macrocode}
    \begingroup
    \xparsed@args\toks@
    #3}
  \let\perhaps@grap@ms\relax
  \expandafter\let\csname end #1\endcsname\@parsed@endenv
  \long\expandafter\def\csname end \string\\#1\expandafter\endcsname\the\@temptokena
     {#4}
}
%    \end{macrocode}
%    Same as in David's again except that we have the added group now.
%    \begin{macrocode}
\def\@parsed@endenv{
  \expandafter\@parsed@endenv@\the\xparsed@args\endgroup}
\def\@parsed@endenv@#1{
  \csname end\string#1\endcsname}
%    \end{macrocode}
%
%
%
% \subsection{A Quark-like datatype}
%
% If it turns out that the only quark thingie is going to be
% |\NoValue| then this can be streamlined for speed!
%
% \begin{macro}{\IfSomethingTF}
% \begin{macro}{\IfSomethingT}
% \begin{macro}{\IfSomethingF}
% Setting up the stage \ldots
%    \begin{macrocode}
\def\IfSomethingTF#1{\def\something@in{#1} \If@SomethingTF}
\def\IfSomethingT#1#2#3{\def\something@in{#1} \If@SomethingTF{#2}{#3}\@empty}
\def\IfSomethingF#1#2#3{\def\something@in{#1} \If@SomethingTF{#2}\@empty{#3}}
%    \end{macrocode}
% \end{macro}
% \end{macro}
% \end{macro}
%
% \begin{macro}{\If@SomethingTF}
% \ldots and then for the recursive part:
%    \begin{macrocode}
\def\If@SomethingTF#1{
   \def\something@tmp{#1}
   \ifx\something@tmp\something@in
%fini true
     \expandafter\@secondofthree
   \else
       \expandafter\def\expandafter\something@tmpb\expandafter{#1}
       \ifx\something@tmp\something@tmpb
%fini false
         \expandafter\expandafter\expandafter\@thirdofthree
       \else
%try again expanded
         \expandafter\expandafter\expandafter\@firstofone
       \fi
   \fi
   {\expandafter\If@SomethingTF\expandafter{#1}}
}
%    \end{macrocode}
% \end{macro}
%
% \begin{macro}{\@secondofthree}
% \begin{macro}{\@thirdofthree}
% Some helpers missing in \LaTeX{}:
%    \begin{macrocode}
\def\@secondofthree#1#2#3{#2}
\def\@thirdofthree #1#2#3{#3}
%    \end{macrocode}
% \end{macro}
% \end{macro}
%
%
% \subsection{Testing for \texttt{\textbackslash NoValue}}
%
%
% \begin{macro}{\NoValue}
%    \begin{macrocode}
\def\NoValue{-NoValue-}
%    \end{macrocode}
% \end{macro}
%
% \begin{macro}{\NoValueInIt}
%    might be handy sometimes\ldots
%    \begin{macrocode}
\def\NoValueInIt{\NoValue}
%    \end{macrocode}
% \end{macro}
%
% \begin{macro}{\IfNoValueTF}
% \begin{macro}{\IfNoValueT}
% \begin{macro}{\IfNoValueF}
%    \begin{macrocode}
\def\IfNoValueTF{\IfSomethingTF\NoValue}
\def\IfNoValueT {\IfSomethingT \NoValue}
\def\IfNoValueF {\IfSomethingF \NoValue}
%    \end{macrocode}
% \end{macro}
% \end{macro}
% \end{macro}
%
% \begin{macro}{\IfValueTF}
% \begin{macro}{\IfValueT}
% \begin{macro}{\IfValueF}
%    \begin{macrocode}
\def\IfValueTF #1 #2 #3 { \IfNoValueTF {#1} {#3} {#2} }
\let \IfValueT \IfNoValueF
\let \IfValueF \IfNoValueT
%    \end{macrocode}
% \end{macro}
% \end{macro}
% \end{macro}
%
% \subsection{A Boolean datatype}
%
% \begin{macro}{\BooleanFalse}
% \begin{macro}{\BooleanTrue}
%    \begin{macrocode}
\def\BooleanFalse{TF}
\def\BooleanTrue{TT}
%    \end{macrocode}
% \end{macro}
% \end{macro}
%
% \begin{macro}{\IfBooleanTF}
% \begin{macro}{\IfBooleanT}
% \begin{macro}{\IfBooleanF}
%    \begin{macrocode}
\def\IfBooleanTF #1 {
   \if#1
       \expandafter\@firstoftwo
   \else
       \expandafter\@secondoftwo
   \fi
}
\def\IfBooleanT #1 #2 { 
   \IfBooleanTF {#1} {#2} \@empty
}
\def\IfBooleanF #1 { 
   \IfBooleanTF {#1} \@empty
}
%    \end{macrocode}
% \end{macro}
% \end{macro}
% \end{macro}
%
%
%    \begin{macrocode}
%</package>
%    \end{macrocode}
%
%
% \Finale
%
\endinput

