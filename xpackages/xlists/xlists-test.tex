% \iffalse
%%
%% (C) Copyright 2009 Frank Mittelbach, LaTeX3 Project
%%
%%
%% It may be distributed and/or modified under the conditions of the
%% LaTeX Project Public License (LPPL), either version 1.3c of this
%% license or (at your option) any later version.  The latest version
%% of this license is in the file
%%
%%    http://www.latex-project.org/lppl.txt
%%
%% This file is part of the ``xlists bundle'' (The Work in LPPL)
%% and all files in that bundle must be distributed together.
%%
%% The released version of this bundle is available from CTAN.
%%
%% -----------------------------------------------------------------------
%%
%% The development version of the bundle can be found at
%%
%%    http://www.latex-project.org/svnroot/experimental/trunk/
%%
%% for those people who are interested.
%%
%%%%%%%%%%%
%% NOTE: %%
%%%%%%%%%%%
%%
%%   Snapshots taken from the repository represent work in progress and may
%%   not work or may contain conflicting material!  We therefore ask
%%   people _not_ to put them into distributions, archives, etc. without
%%   prior consultation with the LaTeX Project Team.
%%
%% -----------------------------------------------------------------------
%% 

\documentclass{article}

\usepackage{expl3}

\usepackage{template-new}


\usepackage{xlists}
\usepackage{xlists-samples}
\tracinggalleys=2  % want tracing here

\newcommand\text{Some sample text to file more than one line in the tests
  below. Some sample text to file more than one line in the tests
  below. }

\begin{document}



\section{itemize}

\text\text

\begin{itemize}
\item xx \text
\item yy \text
\end{itemize}
\text\text

\begin{itemize}
\item one \text
\item two
  \begin{itemize}
  \item two a \text
   \item two b \text
  \end{itemize}
\item three \text
\end{itemize}

\text\text

\section{description}

\begin{description}
\item[A] one
\item[B] two \text
  \begin{description}
  \item[A long one] two a \text
   \item[C] two b \text
  \end{description}
\item[Another long one] three \text
\end{description}

\section{enumerate}

\begin{enumerate}
\item xx \text
\item yy \text
\end{enumerate}

\text\text

\begin{enumerate}
\item one \text
\item two
  \begin{enumerate}
  \item two a \text
   \item two b \text
  \end{enumerate}
\item three \text
\end{enumerate}


\end{document}





