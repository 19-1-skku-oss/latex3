% \iffalse
%% File: l3keys2e.dtx (C) Copyright 2009 LaTeX3 Project 
%%
%% It may be distributed and/or modified under the conditions of the
%% LaTeX Project Public License (LPPL), either version 1.3c of this
%% license or (at your option) any later version.  The latest version
%% of this license is in the file
%%
%%    http://www.latex-project.org/lppl.txt
%%
%% This file is part of the ``l3keys2e'' (The Work in LPPL)
%% and all files in that bundle must be distributed together.
%%
%% The released version of this bundle is available from CTAN.
%%
%% -----------------------------------------------------------------------
%%
%% The development version of the bundle can be found at
%%
%%    http://www.latex-project.org/svnroot/experimental/trunk/
%%
%% for those people who are interested.
%%
%%%%%%%%%%%
%% NOTE: %%
%%%%%%%%%%%
%%
%%   Snapshots taken from the repository represent work in progress and may
%%   not work or may contain conflicting material!  We therefore ask
%%   people _not_ to put them into distributions, archives, etc. without
%%   prior consultation with the LaTeX Project Team.
%%
%% -----------------------------------------------------------------------
%% 
%
%<*driver|package>
\RequirePackage{l3names}
%</driver|package>
%\fi
\GetIdInfo$Id: l3keys2e.dtx 1494 2009-08-24 11:05:44Z joseph $
  {Parsing LaTeX3 keyvals as LaTeX2e package options}
%\iffalse
%<*driver>
%\fi
\ProvidesFile{\filename.\filenameext}
  [\filedate\space v\fileversion\space\filedescription]
%\iffalse
\documentclass{l3doc}
\begin{document}
  \DocInput{l3keys2e.dtx}
\end{document}
%</driver>
% \fi
%
% \title{The \textsf{l3keys2e} package\thanks{This file
%         has version number \fileversion, last
%         revised \filedate.}\\
% Parsing \LaTeX3 keyvals as \LaTeXe\ package options}
% \author{\Team}
% \date{\filedate}
% \maketitle
%
%\begin{documentation}
%
%\section{Key--value arguments as \LaTeXe\ class and package options}
%
% The key--value method for optional arguments is very popular, as it
% allows the class or package author to define a large number of
% options with a simple interface. The \pkg{expl3} bundle of \LaTeX3
% base code includes the module \pkg{l3keys} for defining keys, but
% to use these when loading \LaTeXe\ packages and classes requires
% extra support. That support is provided by this small package, which
% is intended to enable \LaTeXe\ packages to benefit from the power of 
% the \LaTeX3 key--value system.
% 
%\subsection{Creating and using keyval options}
%
% As with any key--value input, using key--value pairs as package or
% class options has two parts. The first stage is to define one or
% more keys. This should be done using the \LaTeX3 module \pkg{l3keys}.
% For example, an option which simply stores a value would be created
% using:
%\begin{verbatim} 
%  \keys_define:nn { module } {
%    option .set:N = \l_module_variable_tl
%  }
%\end{verbatim}
% On its own, this will not make the key an option for the package or
% class containing the definition. The second stage is therefore to 
% process the current options, searching for applicable keys.
% 
%\begin{function}{\ProcessKeysOptions}
%  \begin{syntax}
%    "\ProcessKeysOptions" \Arg{module}
%  \end{syntax}
%  The \cs{ProcessKeysOptions} function is used to check the current 
%  option list against the keys defined for <module>. Global (class)
%  options and local (package) options are checked when this function
%  is called in a package. Each option which does match a key name is
%  then used to attempt to set the appropriate key using
%  \cs{keys_set:nn}. For example, the option defined earlier would be
%  processed by the line
%  \begin{verbatim}
%    \ProcessKeysOptions { module }
%  \end{verbatim}
%\end{function}
%
%\subsection{Internal functions and variables}
%
%\begin{variable}{\keys_latexe_options_clist}
%  Local options and applicable global (class) options are collected up
%  here prior to processing using \cs{keys_set:nn}.
%\end{variable}
%
%\begin{function}{\keys_latexe_options:n}
%  \begin{syntax}
%    "\keys_latexe_options:n" \Arg{module}
%  \end{syntax}
%  Examines current global and local options, comparing them to the
%  keys defined for <module>. Option names which match defined keys are
%  then executed by passing the appropriate data to \cs{keys_set:nn}.
%\end{function}
%
%\begin{function}{\keys_latexe_options_global:n}
%  \begin{syntax}
%    "\keys_latexe_options_global:n" \Arg{module}
%  \end{syntax}
%  Makes appropriate checks on global (class) options, then searches for
%  key matches for <module>.
%\end{function}
%
%\begin{function}{\keys_latexe_options_class:n}
%  \begin{syntax}
%    "\keys_latexe_options_class:n" \Arg{module}
%  \end{syntax}
%  Processes global options when the current file is itself a class, 
%  checking the options against keys defined for <module>.
%\end{function}
%
%\begin{function}{\keys_latexe_options_package:n}
%  \begin{syntax}
%    "\keys_latexe_options_package:n" \Arg{module}
%  \end{syntax}
%  Processes global options when the current file is a \LaTeXe\ package,
%  checking the options against keys defined for <module>.
%\end{function}
%
%\begin{function}{\keys_latexe_options_local:}
%  \begin{syntax}
%    "\keys_latexe_options_local:"
%  \end{syntax}
%  Finds local (package) options and adds them to the list to be
%  parsed by \cs{keys_set:nn}.
%\end{function}
%
%\begin{function}{
%  \keys_latexe_remove_equals:n / (EXP) |
%  \keys_latexe_remove_equals:w / (EXP)
%}
%  \begin{syntax}
%    "\keys_latexe_remove_equals:n" \Arg{option}
%    "\keys_latexe_remove_equals:w" <key> = <value> "\q_stop"
%  \end{syntax}
%  Returns only the key part of a key--value option by removing 
%  everything after the first equals sign. It is assumed that, as this
%  will apply in the preamble, equals signs have category code 12.
%\end{function}
%
%\end{documentation}
% 
%\begin{implementation}
%
%\subsection{Implementation}
%
% The usual lead-off, using \pkg{xparse} at the interface level.
%    \begin{macrocode}
%<*package>
\ProvidesExplPackage
  {\filename}{\filedate}{\fileversion}{\filedescription}
\RequirePackage{xparse}
%    \end{macrocode}
% 
%\begin{macro}{\keys_latexe_options_clist}
% The module collects all applicable global and local options into a
% single list before processing them. 
%    \begin{macrocode}
\clist_new:N \keys_latexe_options_clist
%    \end{macrocode}
%\end{macro}
% 
%\begin{macro}{\keys_latexe_options:n}
% The main function calls functions to collect up the global and local
% options into \cs{keys_latexe_options_clist} before calling the main
% \pkg{l3keys} module to actually do the processing. So that a suitable
% message is produced if the option is unknown, the special 
% \texttt{unknown} key is set if it does not already exist for the 
% current module.
%    \begin{macrocode}
\cs_new:Npn \keys_latexe_options:n #1 {
  \clist_clear:N \keys_latexe_options_clist
  \keys_latexe_options_global:n {#1}
  \keys_latexe_options_local:
  \keys_if_exist:nnF {#1} { unknown } {
    \keys_define:nn {#1} {
      unknown .code:n = { 
        \msg_warning:nnxx { Option~Processing } { unknown-option }
          { \l_keys_key_tl } { \@currname }
      }
    }
  }
  \keys_set:nV {#1} \keys_latexe_options_clist
  \AtEndOfPackage { \cs_set_eq:NN \@unprocessedoptions \scan_stop: }
}
\msg_new:nnn { Option~Processing } { unknown-option } {%
  Unknown~option~`#1'\\%
  for~module~#2.%
}
%    \end{macrocode}
%\end{macro}
%
%\begin{macro}{\keys_latexe_options_global:n}
% Global (class) options are handled differently for \LaTeXe\ packages
% and classes. Hence this function is essentially a check on the current
% file type. The initial test is needed as \LaTeXe\ allows variables to
% be equal to \cs{scan_stop:}, which is forbidden in \LaTeX3 code.
%    \begin{macrocode}
\cs_new:Npn \keys_latexe_options_global:n #1 {
  \cs_if_eq:NNF \@classoptionslist \scan_stop: {
    \cs_if_eq:NNTF \@currext \@clsextension {
      \keys_latexe_options_class:n {#1}
    }{
      \keys_latexe_options_package:n {#1}
    }
  }
}
%    \end{macrocode}
%\end{macro}    
%    
%\begin{macro}{\keys_latexe_options_class:n}
% For classes, each option (stripped of any content after `\texttt{=}')
% is checked for existence as a key. If found, the option is added to 
% the combined list for processing. On the other hand, unused options
% are stored up in \cs{@unusedoptionlist}.
%    \begin{macrocode}
\cs_new:Npn \keys_latexe_options_class:n #1 {
  \clist_map_inline:Nn \@classoptionslist {
    \keys_if_exist:nnTF {#1} { \keys_latexe_remove_equals:n {##1} } {
      \clist_put_right:Nn \keys_latexe_options_clist {##1}
    }{
      \clist_put_right:Nn \@unusedoptionlist {##1}
    }
  }
}
%    \end{macrocode}
%\end{macro}
%
%\begin{macro}{\keys_latexe_options_package:n}
% For global options when processing a package, the tasks are slightly
% different from those for a class. The check is the same, but here
% there is nothing to do if the option is not applicable. Each valid
% option also needs to be removed from \cs{@unusedoptionlist}.
%    \begin{macrocode}
\cs_new:Npn \keys_latexe_options_package:n #1 {
  \clist_map_inline:Nn \@classoptionslist {
    \keys_if_exist:nnT {#1} { \keys_latexe_remove_equals:n {##1} } {
      \clist_put_right:Nn \keys_latexe_options_clist {##1}
      \clist_remove_element:Nn \@unusedoptionlist {##1}
    }
  }
}
%    \end{macrocode}
%\end{macro}
%
%\begin{macro}{\keys_latexe_options_local:}
% If local options are found, the are added to the processing list.
% \LaTeXe\ stores options for each file in a macro which may or may not
% exist, hence the need to use \cs{cs_if_exist:c}.
%    \begin{macrocode}
\cs_new_nopar:Npn \keys_latexe_options_local: {
  \cs_if_eq:NNF \@currext \@clsextension {
    \cs_if_exist:cT { opt@ \@currname . \@currext } {
      \exp_args:NNc \clist_put_right:NV \keys_latexe_options_clist 
        { opt@ \@currname . \@currext }
    }
  }
}
%    \end{macrocode}
%\end{macro}   
%    
%\begin{macro}{\keys_latexe_remove_equals:n}
%\begin{macro}{\keys_latexe_remove_equals:w}
% As the name suggests, this is a simple function to remove an equals
% sign from the input. This is all wrapped up in an \texttt{n} function
% so that there will always be a sign available.
%    \begin{macrocode}
\cs_new:Npn \keys_latexe_remove_equals:n #1 {
  \keys_latexe_remove_equals:w #1 = \q_stop
}
\cs_new:Npn \keys_latexe_remove_equals:w #1 = #2 \q_stop {#1}
%    \end{macrocode}
%\end{macro}
%\end{macro}
% 
%\begin{macro}{\ProcessKeysOptions}
% The user macro is simply a wrapper around the internal process. In
% contrast to other similar packages, the module name is always required
% here.
%    \begin{macrocode}
\NewDocumentCommand \ProcessKeysOptions { m } {
  \keys_latexe_options:n {#1}
}
\@onlypreamble \ProcessKeysOptions
%    \end{macrocode}
%\end{macro}
%
%    \begin{macrocode}
%</package>
%    \end{macrocode}
%
% \end{implementation}
% 
% \PrintIndex
