% \iffalse
%% File: l3chk.dtx Copyright (C) 1990-2006,2009 LaTeX3 project
%%
%% It may be distributed and/or modified under the conditions of the
%% LaTeX Project Public License (LPPL), either version 1.3c of this
%% license or (at your option) any later version.  The latest version
%% of this license is in the file
%%
%%    http://www.latex-project.org/lppl.txt
%%
%% This file is part of the ``expl3 bundle'' (The Work in LPPL)
%% and all files in that bundle must be distributed together.
%%
%% The released version of this bundle is available from CTAN.
%%
%% -----------------------------------------------------------------------
%%
%% The development version of the bundle can be found at
%%
%%    http://www.latex-project.org/svnroot/experimental/trunk/
%%
%% for those people who are interested.
%%
%%%%%%%%%%%
%% NOTE: %%
%%%%%%%%%%%
%%
%%   Snapshots taken from the repository represent work in progress and may
%%   not work or may contain conflicting material!  We therefore ask
%%   people _not_ to put them into distributions, archives, etc. without
%%   prior consultation with the LaTeX Project Team.
%%
%% -----------------------------------------------------------------------
%%
%<*driver|package>
\RequirePackage{l3names}
%</driver|package>
%\fi
\GetIdInfo$Id$
          {L3 Experimental check module}
%\iffalse
%<*driver>
%\fi
\ProvidesFile{\filename.\filenameext}
  [\filedate\space v\fileversion\space\filedescription]
%\iffalse
\documentclass[full]{l3doc}
\begin{document}
\DocInput{\filename.\filenameext}
\end{document}
%</driver>
% \fi
%
% \title{The \textsf{l3chk} package\thanks{This file
%         has version number \fileversion, last
%         revised \filedate.}\\
% Checking things\ldots}
% \author{\Team}
% \date{\filedate}
% \maketitle
%
% \begin{documentation}
%
% To ensure that functions and variables are properly used certain
% checking functions are implemented that may or may not be compiled
% into the final format.
%
% \section{Tracing}
%
% \begin{function}{\tracingall           |
%                  \absolutelytracingall |
%                  \donotcheck           |
%                  \tracingoff           |
%                  \traceon              |
%                  \traceoff             }
% Commands to control the level of verbosity in the tracing output.
% \end{function}
%
% \section{Functions}
%
% \begin{function}{%
%                  \chk_var_or_const:N |
% }
% \begin{syntax}
%   "\chk_var_or_const:N" <cs>
% \end{syntax}
% Checks that <cs> is a proper variable or constant which means that its
% name starts out with "\L", "\l", "\G", "\g", "\R", "\C", "\c", or "\q".
% \end{function}
%
% \begin{function}{%
%                  \chk_local:N |
%                  \chk_global:N |
% }
% \begin{syntax}
%   "\chk_local:N" <cs>
% \end{syntax}
% Checks that <cs> is a proper local or global variable. This means that
% its name starts out with "\L", "\l", or "\G", "\g" respectively.
% \end{function}
%
% \begin{function}{%
%                  \chk_local_or_tex_global:N |
%                  \tex_global_chk: |
% }
% To allow implementations where we precede some function with
% "\tex_global:D" without loosing the possibility to check for the
% correct variable type the following helper functions can be used:
% "\chk_local_or_tex_global:N" <cs> is the variable check which is
% usually let to "\chk_local:N", i.e.\ it will check that its argument
% is a local variable. This behavior will be changed by
% "\tex_global_chk:".  This function first changes
% "\chk_local_or_tex_global:N" to check for global variables then it
% issues a "\tex_global:D". After use "\chk_local_or_tex_global:N"
% will restore itself to "\chk_local:N".  So, if we use
% "\chk_local_or_tex_global:N" inside some function "\foo_bar:n" we can
% implement a global version "\foo_gbar:n" by defining
% \begin{quote}
%    "\cs_new_nopar:Npn \foo_gbar:n {\tex_global_chk: \foo_bar:n }"
% \end{quote}
% provided that "\foo_bar:n" is built in a way that prefixing it with
% "\tex_global:D" turns its operation into a global one. See
% implementation for details.
% \end{function}
%
% \end{documentation}
%
% \begin{implementation}
%
% \section{\pkg{l3chk} implementation}
%
%    We start by ensuring that the required packages are loaded.
%    \begin{macrocode}
%<*package>
\ProvidesExplPackage
  {\filename}{\filedate}{\fileversion}{\filedescription}
\package_check_loaded_expl:
%</package>
%<*initex|package>
%    \end{macrocode}
%
% The |\chk| module contains those functions that are used primarily
% during the development of \LaTeX3 for checking that things are not mixed up
% too badly.  All these functions are of type `{\tt e}' since they
% issue error messages if certain conditions are violated.
%
% \subsection{Checking variable assignments}
%
% \begin{macro}{\chk_local:N}
% \begin{macro}[aux]{\chk_local_aux:w}
%    This function checks that its argument is a proper local
%    variable, i.e.\ its name starts with |\l| or |\L|.
%    It is not allowed that the name starts with |\g| or |\G|,
%    which means that we do not allow to update global
%    variables locally. But see |\tex_global_chk:| below for
%    the encoding of functions that might accept global variables
%    in certain situations.  Not checked is the case that the
%    argument isn't a variable at all, i.e.\ it doesn't have
%    a |_| as second letter. Maybe we should add this for safety
%    during the implementation since it will find certain errors
%    involving wrong expansion in earlier stage.
%    \begin{macrocode}
\cs_new_nopar:Npn \chk_local:N #1{
  \exp_after:wN \chk_local_aux:w \token_to_str:N#1\q_stop}

\cs_new_nopar:Npn \chk_local_aux:w #1#2#3\q_stop{
   \if_num:w\tex_uccode:D`#2=`G\scan_stop:
       \msg_kernel_bug:x {Local~mismatch:~local~function~called~with~
                     global~variable:^^J\text_put_four_sp: #1#2#3~
                     on~line~\tex_the:D\tex_inputlineno:D}
   \else:
     \if_num:w\tex_uccode:D`#2=`L\scan_stop:
     \else:
       \msg_kernel_bug:x {Variable~mismatch:~function~not~called~with~
                     proper~variable:^^J\text_put_four_sp: #1#2#3~
                     on~line~\tex_the:D\tex_inputlineno:D}\fi:
   \fi:}
%    \end{macrocode}
%    We set the |\l_iow_new_line_code| at this point, just in case we
%    run into errors.
%    \begin{macrocode}
\tex_newlinechar:D=`\^^J
%    \end{macrocode}
% \end{macro}
% \end{macro}
%
% \begin{macro}{\chk_global:N}
% \begin{macro}[aux]{\chk_global_aux:w}
%    |\chk_global:N| is similar to |\chk_local:N| but looks only
%    for global variables.
%    \begin{macrocode}
\cs_new_nopar:Npn \chk_global:N #1{\exp_after:wN
                           \chk_global_aux:w \token_to_str:N#1\q_stop}
\cs_new_nopar:Npn \chk_global_aux:w #1#2#3\q_stop{
   \if_num:w\tex_uccode:D`#2=`L\scan_stop:
       \msg_kernel_bug:x {Global~mismatch:~global~function~called~with~
                   local~variable:~#1#2#3~
                   on~line~\tex_the:D\tex_inputlineno:D}
   \else:
     \if_num:w\tex_uccode:D`#2=`G\scan_stop:
     \else:
       \msg_kernel_bug:x {Variable~mismatch:~function~not~called~with~
                     proper~variable:~#1#2#3~
                     on~line~\tex_the:D\tex_inputlineno:D}\fi:\fi:}
%    \end{macrocode}
% \end{macro}
% \end{macro}
%
% \begin{macro}{\tex_global_chk:}
% \begin{macro}{\chk_local_or_tex_global:N}
%    To allow implementations where we precede some function with
%    |\tex_global:D| without loosing the possibility to check for the
%    correct type of a variable, the following helper functions can be
%    used: |\chk_local_or_tex_global:N| \meta{variable} is the variable
%    check which is usually |\cs_set_eq:NN| to |\chk_local:N|, i.e.\ it will
%    check for local variables. This behavior will be changed by
%    |\tex_global_chk:|.  This function first changes
%    |\chk_local_or_tex_global:N| to check for global variables, then
%    issues a |\tex_global:D|. After being used,
%    |\chk_local_or_tex_global:N| will restore itself to
%    |\chk_local:N|. So, if we use |\chk_local_or_tex_global:N|
%    inside some function |\foo_bar:n| we can implement a global
%    version |\foo_gbar:n| by defining
%    \begin{verbatim}
%       \cs_new_nopar:Npn \foo_gbar:n {\tex_global_chk: \foo_bar:n }
%\end{verbatim}
%    provided of course, that |\foo_bar:n| is defined in a way that a
%    |\tex_global:D| does work. Such a scheme has to be used
%    carefully, but its advantage is that the checking version has the
%    same structure as a streamlined version, we only have to change
%    |\tex_global_chk:| into |\tex_global:D| and omit the
%    |\chk_local_or_tex_global:N| function in the body.
%    \begin{macrocode}
\cs_new_nopar:Npn \tex_global_chk: {
    \cs_gset_nopar:Npn \chk_local_or_tex_global:N ##1{
          \chk_global:N ##1
          \cs_gset_eq:NN \chk_local_or_tex_global:N \chk_local:N}
    \tex_global:D}
\cs_new_eq:NN \chk_local_or_tex_global:N \chk_local:N
%    \end{macrocode}
% \end{macro}
% \end{macro}
%
% \begin{macro}{\chk_var_or_const:N}
% \begin{macro}[aux]{\chk_var_or_const_aux:w}
%    |\chk_var_or_const:N| is used in situations where we want to check
%    that we have a variable (or a constant) but do not care whether
%    or not it is global (for example, in allocation routines).
%    \begin{macrocode}
\cs_new_nopar:Npn \chk_var_or_const:N #1{\exp_after:wN
     \chk_var_or_const_aux:w \token_to_str:N#1\q_stop }
\cs_new_nopar:Npn \chk_var_or_const_aux:w #1#2#3\q_stop {
    \if_num:w\tex_uccode:D`#2=`L\scan_stop:
    \else:
      \if_num:w\tex_uccode:D`#2=`G\scan_stop:
      \else:
        \if_num:w\tex_uccode:D`#2=`C\scan_stop:
        \else:
%    \end{macrocode}
%    We also allow the beast to be a quark, i.e.\ to start with |\q|.
%    \begin{macrocode}
          \if_charcode:w#2q\scan_stop:
          \else:
%    \end{macrocode}
%    We might also want to allow that it is a user definable
%    variable which means that its name consists of letters only. We
%    could check this by testing that there is no |_|, but this is not
%    implemented so far.
%    \begin{macrocode}
           \msg_kernel_bug:x {Variable~mismatch:~function~not~called~with~
                       proper~variable:^^J\text_put_four_sp: #1#2#3~
                       on~line~\tex_the:D\tex_inputlineno:D}\fi:\fi:\fi:
    \fi:}
%    \end{macrocode}
% \end{macro}
% \end{macro}
%
%
% \subsection{Doing some tracing}
%
%  \begin{macro}{\tracingall}
%  \begin{macro}{\absolutelytracingall}
%  \begin{macro}{\donotcheck}
%    During |\tracingall| we don't want to see all this code coming
%    from the checking functions since something more substantial is
%    probably wrong. Therefore this definition of it turns the
%    functions of as far as possible.
%    \begin{macrocode}
\cs_new_nopar:Npn\donotcheck{
  \cs_set_eq:NN \chk_global:N \use_none:n
  \cs_set_eq:NN \chk_local:N \use_none:n
  \cs_set_eq:NN \chk_local_or_tex_global:N \use_none:n
  \cs_set_eq:NN \tex_global_chk: \tex_global:D
  \cs_set_eq:NN \chk_if_free_cs:N \use_none:n
  \cs_set_eq:NN \chk_exist_cs:N \use_none:n
  \cs_set_eq:NN \chk_var_or_const:N \use_none:n
  \cs_set_eq:NN \cs_record_name:N \use_none:n
  \cs_set_eq:NN \cs_record_name:c \use_none:n
  \cs_set_eq:NN \cs_record_meaning:N \use_none:n
  \cs_set_eq:NN \register_record_name:N \use_none:n
}
\cs_new_nopar:Npn\absolutelytracingall{
%    \end{macrocode}
%  \end{macro}
%
%    We do the settings by hand to avoid uninteresting lines in the
%    log file as much as possible.
%    \begin{macrocode}
  \tex_global:D\g_trace_commands_status\c_two
  \tex_global:D\g_trace_statistics_status\c_two
  \tex_global:D\g_trace_pages_status\c_one
  \tex_global:D\g_trace_output_status\c_one
  \tex_global:D\g_trace_chars_status\c_one
  \tex_global:D\g_trace_macros_status\c_two
  \tex_global:D\g_trace_paragraphs_status\c_one
  \tex_global:D\g_trace_restores_status\c_one
  \tex_global:D\g_trace_box_breadth_int\c_ten_thousand
  \tex_global:D\g_trace_box_depth_int\c_ten_thousand
  \tex_global:D\g_trace_online_status\c_one
  \tex_errorstopmode:D}
%
% Use LaTeX2e definition for now.
%\cs_new_nopar:Npn\tracingall{
%  \donotcheck
%  \absolutelytracingall
%}
%    \end{macrocode}
%  \end{macro}
%  \end{macro}
%
%  \begin{macro}{\tracingoff}
%    This macro turns off all tracing.
%    \begin{macrocode}
\cs_new_nopar:Npn\tracingoff{
%    \end{macrocode}
%    First we turn off |\g_trace_online_status| so that we get minimal
%    rubbish on the terminal. Of course, in the log file all
%    assignments are shown.
%    \begin{macrocode}
  \tex_global:D\g_trace_online_status\c_zero
  \tex_global:D\g_trace_commands_status\c_zero
  \tex_global:D\g_trace_statistics_status\c_zero
  \tex_global:D\g_trace_pages_status\c_zero
  \tex_global:D\g_trace_output_status\c_zero
  \tex_global:D\g_trace_chars_status\c_zero
  \tex_global:D\g_trace_macros_status\c_zero
  \tex_global:D\g_trace_paragraphs_status\c_zero
  \tex_global:D\g_trace_restores_status\c_zero
  \tex_global:D\g_trace_box_breadth_int\c_zero
  \tex_global:D\g_trace_box_depth_int\c_zero
}
%    \end{macrocode}
%  \end{macro}
%
% \subsection{Tracing modules}
%
%  \begin{macro}{\traceon}
%  \begin{macro}{\traceoff}
%    Turning the tracing of modules on or off. (primitive version).
%    \begin{macrocode}
%<*trace>
\cs_new_nopar:Npn\traceon#1{
  \clist_map_inline:nn {#1}{
    \cs_if_free:cF{g_trace_ ##1 _status}
    {\int_gincr:c{g_trace_ ##1 _status}}
  }
}
\cs_new_nopar:Npn\traceoff#1{
  \clist_map_inline:nn {#1}{
    \cs_if_free:cF{g_trace_ ##1 _status}
    {\int_gdecr:c{g_trace_ ##1 _status}}
  }
}
%</trace>
%<-trace>\cs_new_eq:NN\traceon\use_none:n
%<-trace>\cs_new_eq:NN\traceoff\use_none:n
%</initex|package>
%    \end{macrocode}
%  \end{macro}
%  \end{macro}
%
%
%
% \end{implementation}
% \PrintIndex
%
% \endinput
