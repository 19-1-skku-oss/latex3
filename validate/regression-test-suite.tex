%		tb57mitt.ltx
\documentclass{ltugboat}

%\usepackage[T1]{fontenc}
\DeclareTextSymbol{\textbackslash}{OT1}{`\\}

\hyphenation{Mass-a-chu-setts}

\usepackage{calc}
\usepackage{shortvrb}
\MakeShortVerb\|

\begin{document}

\title{A regression test suite for \LaTeXe}

\author{Frank Mittelbach}
\address{\LaTeX3 project}
\netaddress{Frank.Mittelbach@eds.com}


\maketitle

\begin{abstract}
  This paper describes the history of the development of a regression
  test suite for \LaTeX{} and its importance for the release of stable
  and reliable future distributions of that software. 
  A more detailed description of the concepts and the
  implementation of the test suite will be given in~\cite{tub:xxx}.
%  It carries on
%  with describing the concepts and the implementation of this suite
%  showing some examples how to provide additional test files within
%  this scheme.

  As experience shows that there can't be enough test files in such a
  suite, we make a plea to the \TeX{} community to help us in making
  \LaTeX{} distributions even more reliable by joining a new volunteer
  group working on the task of updating and adding to this suite.
\end{abstract}


\section{Introduction}

Back in 1992 when the \LaTeX3 team took over maintenance of \LaTeX{}
and started to work on the current \LaTeX{} version~\cite{A-W:LLa94},
also known as \LaTeXe{}, I had the idea of producing a test
environment for \LaTeX{} that would help us in providing stable and
reliable distributions. My idea originated in the
\texttt{trip} test for the \TeX{} program~\cite{Knuth:1984:TTT}, a
fixed test file which is run through \TeX{}, containing code that tries
to exercise as many boundary cases as Don Knuth could think of.  The
output of this run is then compared with a set of files certified by
Don to contain the correct information. Only if a new implementation
of the \TeX{} program produces the same output (with well defined minor
%derivations in certain places) is it allowed\footnote{An additional
deviations in certain places) is it allowed\footnote{An additional
requirement according to the \texttt{trip} test documentation is that
the author of the \TeX{} implementation has to be satisfied with the
product. In other words, a simple program that throws away all its
input and always output the files needed to satisfy the \texttt{trip}
test would be allowed to call itself \TeX{} as long as the author of
that program is happy with it.} to be called \TeX{}.  The idea behind
this is to ensure that \TeX{} behaves identically on all
implementations and the \texttt{trip} test was the measure proving
this.

With \LaTeX{} the idea was not to ensure that it is identical on all
%platforms---this is automatically the case if the standard
platforms\Dash this is automatically the case if the standard
installation is obtained and the installation procedures are
%applied---but to ensure that that new releases of \LaTeX{} do not
applied\Dash but to ensure that that new releases of \LaTeX{} do not
inadvertently modify the behavior of commands.  Since \LaTeX{} is a
large and complex system, this is definitely a non-trivial
task: in `fixing' one bug, it is often necessary to modify the
definitions of several `internal' commands, and these may in turn
affect many other commands which have no obvious connection with the
original problem.

We have had some pretty disastrous experiences of this type, often
finding that harmless looking corrections had effects on what seemed,
at first glance, completely unrelated areas. This is in part due to
the fact that \LaTeX{} is based on the macro language of \TeX{}, which
allows reuse and redefinition of arbitrary code fragments.

\begin{figure*}
\centering
\setlength\fboxsep{5pt}
\fbox{\begin{minipage}{\linewidth-3em}
\newcommand{\timeestimate}[1]{\par \smallskip\noindent
  {\it Estimated time required:}
  #1.\par}

\newcommand{\task}[1]{\textbf{#1}\par\vspace*{3pt}}
\newcommand{\coordinator}[1]{\par\smallskip
  \noindent{\it Coordinator\/} [#1]:\volunteer}

\newcommand{\othervolunteers}{\par\noindent{\it Other volunteers:}}
\newcommand{\volunteer}[1]{\par#1\quad \ignorespaces}

\newcommand{\email}{\begingroup \catcode`\%=12 \xemail}
%    Auxiliary function for \email. It applies \meaning to the
%    argument to make all the characters category 12.
\newcommand{\xemail}[1]{\def\temp{#1}\tt
  \expandafter\xmeaning\meaning\temp\xmeaning\endgroup}
%    Auxiliary function for \xemail. \newcommand cannot be used here.
\def\xmeaning#1->#2\xmeaning{#2}


\task{Validating \protect\LaTeX 2.09}
Writing test files for regression testing: checking bug fixes and
improvements to verify that they don't have undesirable
side effects; making sure that bug fixes really correct the problem
they were intended to correct; testing interaction with
various document styles, style options, and environments.

We would like three kinds of validation files:
\begin{enumerate}
\item General documents.
\item Exhaustive tests of special environments/modules such as tables,
displayed equations, theorems, floating figures, pictures, etc.
\item Bug files containing tests of all bugs that are supposed to be
fixed (as well as those that are not fixed, with comments about their
status).
\end{enumerate}

A procedure for processing validation files has been devised; details
will be furnished to anyone interested in this task.

\timeestimate{2 to 3 weeks, could be divided up}

\coordinator{25 August 1992}{Daniel Flipo}
                         \email{flipo@citil.citilille.fr}
\othervolunteers
\volunteer{Chris Martin} \email{cs1cwm@sunc.sheffield.ac.uk}

\end{minipage}}
\caption{An excerpt from the volunteer task list 1993}
\label{fig:voltask}
\end{figure*}

For that reason we started working on a concept for automated tests to
detect such problems. When that system was
available,
we asked for volunteers to help us in building up a suitable test suite
for \LaTeX{} (which at that time was \LaTeX~2.09). Part of the rationale
behind this work was to ensure that a future transition from
\LaTeX~2.09 to \LaTeXe{} (for which development was under way) would
become as painless as possible, i.e., these tests were also supposed
to ensure that the new code for \LaTeXe{} would not change the user
interface behavior without detection.

This approach seems to us to have been very successful; this is in
large parts due to the quality and quantity of the work of the
volunteers helping us at that time, in particular Daniel Flipo and
Chris Martin. Figure~\ref{fig:voltask} shows an excerpt from the
volunteer task list from 1993 describing this task (and my rather
optimistic time requirements for it).

When \LaTeXe{} was released for the first time in 1994 we updated the
regression test support macros and tried to improve the test suite by
adding new test files when we fixed bugs or when we added new
functionality to \LaTeX. However, being human, we have not followed
this practice as rigorously as we should have: especially since the
first releases it has become more and more common for us to fix a small
bug without spending the additional time necessary to also write a
test file that exhibits the correct behavior.

Today our test suite has about 300 test files which are automatically
executed and compared before a new release hits the streets. And
indeed, these test files have saved us from embarrassment many times
already.

\section{This year's boo-boos!}
%\section[Chris' boo-boos!]{Chris'
% boo-boos!\footnotemark}\footnotetext{I don't feel responsible for this
% section title: It was added by the man himself (while looking the paper
% over), even though neither of the described problems were entirely his fault.}

However, results show that such a suite can never be large enough to
avoid the need for a patch release once in a while.  It is
particularly important that new
features, such as the release of additional files or the correction of
recently found bugs, get tested and frozen within this suite so that
there is no unexpected change later on.  For example, with the December
1997 release we added the packages \texttt{calc} and \texttt{textcomp}
to the distribution but, due to time constraints, did not add to the suite  
additional test files designed to exercise these packages; 
and, by Murphy's law, \texttt{textcomp} did not contain a necessary
|\ProvidesPackage| command, with the result that it claimed to be
written for a future release\footnote{The technical reason for this
  behavior, for those who wonder, was that the release date of the
  package, which is an optional argument to the (missing)
  \texttt{\textbackslash
    ProvidesPackage} command, was there but was mistakenly picked up the
  \texttt{\textbackslash NeedsTeXFormat} which then produced a warning
  as the release date of \texttt{textcomp} was later than the nominal
  release date (of 1997/12/01) for the format of the
%  distribution.}---something that would have been caught by any test
  distribution.}\Dash something that would have been caught by any test
file exercising the package.

Another embarrassing example of a missing test file in that release
was the |\t| error.  To better support language files from the Babel
suite, some of which make the |"| character active, we changed all
internal definitions of characters and accents from 
hexadecimal notation, such as |"7F|, to decimal, i.e., to |127| in that case.
Unfortunately in the definition for |\t| we did this wrong and |"7F|
became |79|, giving very strange effects when the accent was
used.\footnote{Both errors got found and reported several times within
  two days after the release, so the patch release came out quite
  quickly this time.} An error like this would have been automatically
caught if we had, for each output encoding, a test file to
check that each definition in the encoding results in the `right' glyph or
glyphs.


\typeout{WARNING: **** hardwired new page ****}\newpage

\section{Call for volunteers}

Thus to make the \LaTeX{} system even more reliable we call on you for
help! What we hope to find is a new group of volunteers that is
interested in working on an extension of the \LaTeX{} test suite
system. There is no need to be an expert \TeX{} or \LaTeX{} programmer
for this task though some experience with \LaTeX{} and its inner
workings will be necessary.

\begin{center}
\fbox{\begin{minipage}{\linewidth-2em}
  \bfseries If you are interested in joining this effort, please
  contact Daniel Flipo at 
\begin{center}
  \texttt{Daniel.Flipo@univ-lille1.fr}
\end{center}
  who kindly agreed to act as a coordinator between the individual
  volunteers.
\end{minipage}}
\end{center}




There are a number of areas in which further test files would improve
the system enormously. They are outlined in the following sections.

\subsection{Testing existing interfaces}

Testing existing interfaces is a very important task, one not so far,
for several reasons, adequately covered by the test suite. This will
not only help us to detect problems when fixing errors in \LaTeX{}
but, more importantly, it will help one day in the transition to a
new system since these test files will then clearly identify which
interfaces are compromised (deliberately or by mistake) by the new
system. This in turn will then help to produce, if necessary,
procedures to automatically translate source documents from \LaTeX{}
to its successor.

What we are looking for are test files that describe and test the
current interfaces on all levels. This is certainly an ambitious task,
but perhaps also one of the most interesting and rewarding ones within
this list.

\subsection{Testing corrected bugs}

As described above, several of the bugs reported to us have been fixed and a
test file showing the correct behavior has been added. But for many
this is not the case.

What we are looking for is the provision of test files for all bugs reported
and fixed, so that future releases will not by mistake revert any of
these fixes without
% alarming : nice one, get it?  not quite the same as:
alerting the maintainers.  This means working through our bug
database and devising test files showing the correct behavior. As we
ask submitters of bug reports to send in a test file that shows the
incorrect behavior, and they usually do so, it is often possible to
start from the submitted file and modify it slightly so that it fits into
the regression suite concepts.


\subsection{Testing new extensions}

What is important for the kernel interfaces is also important for the
core packages and extensions: these interfaces should be exercised in
such a way that any future changes will be automatically detected. Again
this provides interesting mental exercise since it isn't always easy
to decide what is pertinent for the interface and how to exercise it
so that enough (but not too much) information ends up in the
\texttt{.log} file.


\subsection{Testing contributed packages}

A final area which is important is the testing of packages which lie
outside the control of the \LaTeX{} maintainers. Although we cannot
in all cases guarantee that corrections to the kernel software
will not harm any such package, we are, of course, very much concerned
to avoid making any change that makes third party packages
invalid. In the past, whenever we noticed (or even suspected)
such a problem we tried either to avoid it, by choosing a different
solution, or, if that was not possible for some reason, to find the
maintainers of the package and give them notice of a possible
clash so that such problems could be avoided.

There is a problem with testing the interfaces of third party packages:
changes by the package author, to either the interface or the
implementation of the package, can upset the test suite as easily as
can changes to the \LaTeX{} kernel by the \LaTeX3 project team.  Thus,
to avoid our limited time resources being used up in chasing after
errors introduced in this way (being neither our fault nor being
correctable by us), it would be necessary to develop clear protocols for
how this part of the test suite should be maintained, e.g., what
requirements a package must fulfill to be included into it, what
obligations an author of such a package agrees to, etc. This is not
yet done and so it is part of the volunteer effort.

\smallskip

We close our plea for help with a quote taken
from~\cite{Knuth:1984:TTT} which shows how the Grand Wizard sees the
task of writing such test files (which does not mean you have to follow
his advice):
%\noindent
%\fbox{\begin{minipage}{\linewidth-2\fboxrule-2\fboxsep}
\begin{quote}
\setlength\rightskip{0pt plus 2em}
To write such a fiendish test routine, one simply gets into a nasty frame
of mind and tries to do everything in the unexpected way. Parameters
that are normally positive are set negative or zero; borderline cases
are pushed to the limit; deliberate errors are made in hopes that the
compiler will not be able to recover properly from them.\\
\mbox{}\hfill\slshape Donald Knuth 1984
\end{quote}
%\end{minipage}}




\begin{thebibliography}{1}
\bibitem{tub:xxx}
David Carlisle and Frank Mittelbach.
\newblock \emph{The \LaTeX{} regression test suite: concepts and
implementation}.
%\newblock To appear in \TUB{} \ldots
\newblock \TUB{}; to appear.

\bibitem{Knuth:1984:TTT}
Donald~E. Knuth.
\newblock A torture test for {\TeX}.
\newblock Report STAN-CS-84-1027, Stanford University, Department of Computer
  Science, Stanford, CA, USA, 1984.

\bibitem{A-W:LLa94}
Leslie Lamport.
\newblock {\em {\LaTeX:} A Document Preparation System}.
\newblock Addison-Wesley, Reading, Massachusetts, second edition, 1994.

\end{thebibliography}

\makesignature

\end{document}

