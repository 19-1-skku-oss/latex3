% \iffalse meta-comment
%
% Copyright (C) 1992-1994 by David Carlisle, Frank Mittelbach.
% Copyright (C) 2008, 2012 LaTeX3 project
% All rights reserved.
%
% This file is part of the validate package.
%
% IMPORTANT NOTICE:
%
% You are not allowed to change this file.  In case of error
% write to the email address mentioned in the file readme.val.
%
% \fi
%                  regression-test.tex
                   %%%%%%%%%%%%%%%%%%%

% This file should not be used as a package or class file,
% it should be \input.

% The scope of this \makeatletter will then be the rest of the
% document.  Put TeX into scroll mode, and stop it showing the
% implementation details of macros in error messages.
\makeatletter
\scrollmode
\errorcontextlines=-1

% Use the same \showbox settings as 2.09, unless they are changed in
% the test file. (2e sets these to -1)
\showboxbreadth=5
\showboxdepth=3

% Start the test, after the optional \documentclass (or \documentstyle)
% \begin{document} commands with \START.  All lines in the .log file
% before this will be ignored. It also prints a docstrip-style
% character table in the .tlg file so the .tlg file can easily be
% checked for email translations.
%
\def\START{\typeout{START-TEST-LOG^^J^^J%
   This is a generated file for the LaTeX (2e + expl3) validation system.%
^^J^^JDon't change this file in any respect.%
^^J^^J\CTable^^J}}

\begingroup
\catcode`\^^\=0
\catcode`\^^A=\catcode`\%
^^\catcode`^^\ =11
^^\catcode`^^\%=11
^^\catcode`^^\#=11
^^\catcode`^^\~=11
^^\endlinechar=`^^\^^J
^^\catcode`^^\\=11^^A
^^\gdef^^\CTable{
%% \CharacterTable
%%  {Upper-case    \A\B\C\D\E\F\G\H\I\J\K\L\M\N\O\P\Q\R\S\T\U\V\W\X\Y\Z
%%   Lower-case    \a\b\c\d\e\f\g\h\i\j\k\l\m\n\o\p\q\r\s\t\u\v\w\x\y\z
%%   Digits        \0\1\2\3\4\5\6\7\8\9
%%   Exclamation   \!     Double quote  \"     Hash (number) \#
%%   Dollar        \$     Percent       \%     Ampersand     \&
%%   Acute accent  \'     Left paren    \(     Right paren   \)
%%   Asterisk      \*     Plus          \+     Comma         \,
%%   Minus         \-     Point         \.     Solidus       \/
%%   Colon         \:     Semicolon     \;     Less than     \<
%%   Equals        \=     Greater than  \>     Question mark \?
%%   Commercial at \@     Left bracket  \[     Backslash     \\
%%   Right bracket \]     Circumflex    \^     Underscore    \_
%%   Grave accent  \`     Left brace    \{     Vertical bar  \|
%%   Right brace   \}     Tilde         \~}
%%
}^^A
^^\endgroup{}%

% The test should end with
% \END or \end{document}
%
\let\@@@end\@@end
%\let\@ED=\enddocument
\def\END
  {%
    \ifnum\currentgrouplevel>0
      \typeout{Bad grouping: \the\currentgrouplevel!}%
    \fi
    \ifnum\currentiflevel>1
      \typeout{Bad conditionals: \the\numexpr\currentiflevel-1!}%
    \fi
    \typeout{END-TEST-LOG}\@@@end
  }
\let\@@end\END


% After the \START should come declarations of the format and style
% options being used.
%
\def\FORMAT#1{\typeout{Format: #1}%
  \def\@tempa{#1}\ifx\@tempa\@EJ\else
   \OMIT\@warning{Declared format #1,^^JActual format \@EJ}\TIMO\fi}

% The old version got this information from everyjob,
% but that does not work with LaTeX2e as \everyjob is cleared.
\edef\@EJ{\fmtname <\fmtversion>}

% Some author info:
\def\AUTHOR#1{\typeout{Author: #1}}
\def\ADDRESS#1{\typeout{Address: #1}}

% Not all packages declare themselves to the log file, and we can not
% rely on TeX`s output as it includes full path names, and does not
% include version numbers etc.  So for each package included give a
% declaration like: \PACKAGES{array v2.0d}
%
\def\STYLE#1{\typeout{Main Style: #1}}%
\def\STYLEOPTIONS#1{\typeout{Style Options: #1}}


% If The class or package is loaded with options, you may
% specify the options in the \CLASS (\PACKAGE) declaration. eg:
%
% \CLASS[german,a4page]{article v2.0 1994/01/02}
% \PACKAGE{ifthen v2.2 1993/11/12}
% \PACKAGE[dvips]{graphics v 3.8 1994/02/02}
%
\def\CLASS{\@ifnextchar[\OPTCLASS\XCLASS}
\def\OPTCLASS[#1]#2{%
  \typeout{Main Class: #2^^J\@spaces Options: #1}}
\def\XCLASS#1{%
  \typeout{Main Class: #1}}

\def\PACKAGE{\@ifnextchar[\OPTPACKAGE\XPACKAGE}
\def\OPTPACKAGE[#1]#2{%
  \typeout{Package: #2^^J\@spaces Options: #1}}
\def\XPACKAGE#1{%
  \typeout{Package: #1}}


% LaTeX2e always uses NFSS2 so new test files need not use
% \FONTSELECTION but it is retained for compatibility for test files
% written for 209/NFSS1.
%
\def\FONTSELECTION#1{%
  \OMIT\@@warning{\noexpand\FONTSELECTION obsolete.^^J%
                 LaTeX2e always uses NFSS2}\TIMO
  \typeout{Font Selection: #1}}


% Surround commands which produce irrelevant lines in the .log file by
% \OMIT
% \TIMO
%
\def\OMIT{\typeout{OMIT}}
\def\TIMO{\typeout{TIMO}}

% After the above declarations, and before the main tests, you may
% optionally `declare' all the commands in the `module' that you are
% about to test. These commands will be registered as defined,
% undefined or relaxed (ie \let to \relax). You may wish to declare
% commands not currently implemented, so that if they are added at a
% later stage, the test will fail, reminding someone to document the
% fact that the user interface has changed. So if you are testing
% array and tabular environments, you may wish do declare
% \extrarowheight. This is undefined in the curent latex, but would
% become defined if Mittelbach's array.sty was incorporated into
% latex.tex.
%
\def\declare@command#1{%
  \ifx#1\@undefined\typeout{Undefined \string#1}\else
  \ifx#1\relax\typeout{Relaxed \space\space\string#1}\else
         \typeout{Defined \space\space\string#1}\fi\fi}


% To allow testing of possible changes, we allow extra code to be read
% in before the test starts. The necessary code should be placed in a
% file regression-test.cfg.
%
\OMIT
\InputIfFileExists{regression-test.cfg}
      {\typeout{^^J***^^Jregression-test.cfg in operation^^J***^^J}}{}
\TIMO

%%%%%%%%%

% We are not starved for space in the log file output, so let's make it as
% verbose as is useful when reading the .diff's.
\newcount \gTESTint

\newcommand \SEPARATOR {%
  \TYPE{%
    ============================================================%
  }%
}

% A long version of \typeout, because tests may contain \par tokens.
% Besides, with that \TYPE, we can do \TYPE { ... \TRUE ... \NEWLINE ... }.
\newcommand \LONGTYPEOUT [1] {%
  \begingroup
    \set@display@protect
    \let \TYPE \@firstofone
    \immediate \write \@unused {#1}%
  \endgroup
}
\let\TYPE\LONGTYPEOUT

\newcommand \TEST [2] {%
  \advance \gTESTint 1
  \SEPARATOR
  \typeout{%
    TEST \the\gTESTint: \detokenize{#1}}%
  \SEPARATOR
  \begingroup
    \let \TYPE \LONGTYPEOUT
    #2%
  \endgroup
  \SEPARATOR \typeout{}%
}

\newcommand \TESTEXP [2] {%
  \advance \gTESTint 1
  \SEPARATOR
  \typeout{%
    TEST \the\gTESTint: \detokenize{#1}}%
  \SEPARATOR
  \begingroup
    \let \TYPE \@firstofone
    \LONGTYPEOUT{#2}%
  \endgroup
  \SEPARATOR \typeout{}%
}

\def \TRUE  {\TYPE{TRUE}}
\def \FALSE {\TYPE{FALSE}}
\def \YES   {\TYPE{YES}}
\def \NO    {\TYPE{NO}}

\def \NEWLINE {\TYPE{^^J}}

% We allocate a large number of registers now: this number
% can be changed to keep register numbers stable in test logs.

\newcount\regression@test@loop@int
\newcommand{\regression@test@alloc}[2]
  {%
    \regression@test@loop@int=\numexpr#1\relax
    \regression@test@loop#2%
  }
\newcommand{\regression@test@loop}[1]
  {
    \ifnum 0<\regression@test@loop@int
      #1\regression@test@dummy
      \advance\regression@test@loop@int by -1\relax
      \expandafter\regression@test@loop
      \expandafter#1%
    \fi
  }
\regression@test@alloc {32} \newcount
\regression@test@alloc {30} \newdimen
\regression@test@alloc {30} \newskip
\regression@test@alloc {30} \newmuskip
\regression@test@alloc {30} \newbox
% \regression@test@alloc {30} \newtoks

\endinput



