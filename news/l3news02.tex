% Copyright 2009 The LaTeX Project
\documentclass{ltnews}
\renewcommand{\LaTeXNews}{\LaTeX3~News}

\publicationmonth{June}
\publicationyear{2009}
\publicationissue{2}

\begin{document}
\maketitle

\section{\TeX Live and the \textsf{expl3} code}

\TeX Live 2009 is almost upon us, and the \LaTeX3 team have been readying a new release. Very dramatic changes have occurred since the public release of the code; no backwards compatibility has been maintained (as warned in the beginning of the documentation) but we believe the changes made are all much for the better.

The \textsf{expl3} code is now considered to be much more stable than it was before; a comprehensive test suite has been written that helps to ensure that we don't make any mistakes as we change things in the future. Many bugs were fixed in the process of writing the test suite. Some small underlying changes are expected in the \textsf{expl3} code, but major, disruptive, changes aren't planned.

\section{Planned updates}

Until now, the last update to \textsc{ctan} of the \textsf{expl3} bundle was for \TeX Live~2008. Now that work on the code is happening on a semi-steady basis, we plan to keep updates rolling out to \textsc{ctan} more frequently. This will allow anyone who wishes to experiment with the new code to use the \TeX Live updater to install a recent version without having to check out the \textsc{svn} repository and install the packages manually.

\section{New members}

We didn't say anything about it in the last status update, but Joseph Wright and Will Robertson are now members of the \LaTeX\ Team. They have been working fairly exclusively on the \textsf{expl3} code. It's worth repeating that \LaTeXe\ is essentially frozen in order to prevent any backwards compatibility problems. As desireable as it is to benefit from the new features offered by new engines Xe\TeX\ and Lua\TeX, we cannot risk the stability of production servers running older versions of \LaTeX\ which will inevitably end up processing documents written into the future.

\section{The next six months}

Having overhauled the \textsf{expl3} code, we now plan to perform an analogous process with the foundations of the \textsf{xpackages}. These are the higher-level packages that will provide the basic needs such as control of the page layout and rich document-level interaction with the user. As the groundwork for this layer of the document processing matures, we will be able to start building more packages for a \LaTeX3 kernel; these packages will also be useable on top of \LaTeXe\ and serve as broadly customisable templates for future document design.

As gaps in the functionality offered by \textsf{expl3} are found (in some cases, we know that they exist already), the programming layer will be extended to support our needs. In other cases, wrappers around \TeX\ functions that can be more usefully handled at a higher level will be written.

\section{Some specifics}

The discussion above is all very broad. Here are some specifics for those interested. New code written and broad changes made to the \textsf{expl3} modules:
\begin{description}
\item [Defining functions] Implicit arguments to \verb|\cs_set:Nn|
\item [Defining conditionals] Morten's ideas for defining predicates and conditionals.
\item [Smart comparisons] Comparisons can be made much more easily now, with familiar notation such as \verb|example goes here|
\item [Data from variables] A new argument specifier for extracting information from variables of different types.
\item [l3msg] New module by Joseph Wright to deal with communication between \LaTeX3 code and the user (info messages, warnings, and errors). These are able to be filtered at a very granular level in order to display only the information that is desired.
\end{description}

Some things to look at next:
\begin{description}
\item [xbase] Code for defining new commands in \LaTeX3 and for handling keyval lists for user input and document specification. Fairly complete but a revisit is planned to see if some concepts need to be updated.
\item [galley2] Sophisticated handling for constructing paragraphs and other document elements. Design needs to be revisited.
\item [xor] Next generation output routine for splitting the galley into page and sub-page sized chunks. Design is sound but there are still some outstanding issues.
\end{description}

\end{document}
