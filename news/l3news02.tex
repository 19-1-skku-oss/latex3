% Copyright 2009 The LaTeX Project
\documentclass{ltnews}
\renewcommand{\LaTeXNews}{\LaTeX3~News}

\publicationmonth{June}
\publicationyear{2009}
\publicationissue{2}

\begin{document}
\maketitle

\section{\TeX Live and the \textsf{expl3} code}

\TeX Live 2009 is almost upon us, and the \LaTeX3 team have been
readying a new release. Very dramatic changes have occurred since the
public release of the code; no backwards compatibility has been
maintained (as warned in the beginning of the documentation) but we
believe the changes made are all much for the better. Almost every
single part of \textsf{expl3} has been scrutinized, resulting in a far
more coherent code base.

The \textsf{expl3} code is now considered to be much more stable than
it was before; a comprehensive test suite has been written that helps
to ensure that we don't make any mistakes as we change things in the
future. Many bugs were fixed in the process of writing the test
suite. Some small underlying changes are expected in the
\textsf{expl3} code, but major, disruptive, changes aren't planned.

\section{Planned updates}

Until now, the last update to \textsc{ctan} of the \textsf{expl3}
bundle was for \TeX Live~2008. Now that work on the code is happening
on a semi-steady basis, we plan to keep updates rolling out to
\textsc{ctan} more frequently. This will allow anyone who wishes to
experiment with the new code to use the \TeX Live or MiK\TeX 
updaters to install a recent version without having to check out the
\textsc{svn} repository and install the packages manually.

\section{New members}

We didn't say anything about it in the last status update, but Joseph
Wright and Will Robertson are now members of the \LaTeX\ Team. They
have been working fairly exclusively on the \textsf{expl3} code. 

It's worth repeating that \LaTeXe\ is essentially frozen in order to
prevent any backwards compatibility problems. As desirable as it is
to benefit from the new features offered by new engines Xe\TeX\ and
Lua\TeX, we cannot risk the stability of production servers running
older versions of \LaTeXe\ which will inevitably end up processing
documents written into the future.

\section{The next six months}

Having overhauled the \textsf{expl3} code, we now plan to perform an
analogous process with the foundations of the
\textsf{xpackages}. These are the higher-level packages that will
provide the basic needs such as control of the page layout and rich
document-level interaction with the user. As the groundwork for this
layer of the document processing matures, we will be able to start
building more packages for a \LaTeX3 kernel; these packages will also
be usable on top of \LaTeXe\ and serve as broadly customisable
templates for future document design.

As gaps in the functionality offered by \textsf{expl3} are found (in
some cases, we know that they exist already), the programming layer
will be extended to support our needs. In other cases, wrappers around
\TeX\ functions that can be more usefully handled at a higher level
will be written.

\section{Some specifics}

The discussion above is all very broad. Here are some specifics for
those interested. New code written and broad changes made to the
\textsf{expl3} modules:
\begin{description}
\item [Defining functions] Implicit arguments to \verb|\cs_set:Nn|,
  rather than needing to count them (the old \verb|\cs_set:NNn|).
\item [Defining conditionals] Morten's ideas for defining predicates 
  and conditionals have been implemented.
\item [Smart comparisons] Comparisons can be made much more easily
  now, with familiar notation such as 
  \verb|\intexpr_compare:nTF{ 5+3 != \l_tmpa_int }|, in addition to
  the existing \verb|\intexpr_compare:nNnTF| syntax.
\item [Data from variables] A new argument specifier \texttt{V} 
  for extracting information from variables of different types, 
  without needing to know the underlying variable structure.
\item [l3msg] New module to deal with communication between \LaTeX3 
  code and the user (info messages, warnings, and errors), including
  message filtering inspired by the \textsf{silence} package.
\end{description}

Some things to look at next:
\begin{description}
\item [xbase] Code for defining new document commands in \LaTeX3 and for
  handling keyval lists for user input and document specification. 
  Functionality coverage is good but concepts need a good ``airing''.
\item [galley2] Sophisticated handling for constructing paragraphs and
  other document elements. Design needs to be revisited.
\item [xor] \LaTeX3 output routine for splitting the galley into 
  page and sub-page sized chunks. Ideas and code need work to move to
  ``production ready'' status.
\end{description}

\end{document}
