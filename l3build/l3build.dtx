% \iffalse
%%
%% File l3build.dtx (C) Copyright 2014-2015 The LaTeX3 Project
%%
%% It may be distributed and/or modified under the conditions of the
%% LaTeX Project Public License (LPPL), either version 1.3c of this
%% license or (at your option) any later version.  The latest version
%% of this license is in the file
%%
%%    http://www.latex-project.org/lppl.txt
%%
%% This file is part of the "l3build bundle" (The Work in LPPL)
%% and all files in that bundle must be distributed together.
%%
%% The released version of this bundle is available from CTAN.
%%
%% -----------------------------------------------------------------------
%%
%% The development version of the bundle can be found at
%%
%%    http://www.latex-project.org/svnroot/experimental/trunk/
%%
%% for those people who are interested.
%%
%%%%%%%%%%%
%% NOTE: %%
%%%%%%%%%%%
%%
%%   Snapshots taken from the repository represent work in progress and may
%%   not work or may contain conflicting material!  We therefore ask
%%   people _not_ to put them into distributions, archives, etc. without
%%   prior consultation with the LaTeX Project Team.
%%
%% -----------------------------------------------------------------------
%%
%
%
%<*driver>
\RequirePackage{expl3}
\def\ExplFileName{l3build}
\def\ExplFileDescription{L3 Regression test suite}
\def\ExplFileDate{2015/09/24}
\def\ExplFileVersion{6172}
\documentclass[full]{l3doc}
\renewcommand\partname{Part}
\usepackage{multicol,needspace}

\makeatletter
\addtolength\columnsep{1em}
\renewcommand\columnseprule{0.4pt}
\def\@starttoc#1{%
  \begin{multicols}{2}
  \small
  \makeatletter
  \@input{\jobname.#1}%
  \expandafter\newwrite\csname tf@#1\endcsname
  \immediate\openout \csname tf@#1\endcsname \jobname.#1\relax
  \@nobreakfalse
  \end{multicols}}
\makeatother

\newenvironment{buildcmd}[1]{%
  \bigskip\par\noindent\hspace{-\parindent}%
  \fbox{\ttfamily \textcolor[gray]{0.5}{\$} texlua build.lua #1}%
  \nopagebreak\smallskip\nopagebreak\par\nopagebreak\noindent\ignorespaces
 }
 {}
\makeatletter
\newcommand\luavar[1]{\@ifundefined{lua_#1}{\ERROR}{\texttt{\@nameuse{lua_#1}}}}
\newcommand\luavarset[3]{\@namedef{lua_#1}{#2}}
\newcommand\luavarseparator{}
\makeatother
\def\allluavars{
\luavarset{module}{""}{The name of the module.}
\luavarset{bundle}{""}{The name of the bundle in which the module belongs.}
\luavarset{ctanpkg}{bundle}{Name of the bundle on CTAN}
\luavarseparator
\luavarset{modules}{\{~\}}{The list of all modules in a bundle (when not auto-detecting)}
\luavarset{exclmodules}{\{~\}}{Directories to be excluded from automatic module detection}
\luavarseparator
\luavarset{maindir}     {"."}{The top level directory for this module or bundle.}
\luavarset{supportdir}  {maindir .. "/support"}     {Where copies of files to support check/doc compilation are stored.}
\luavarset{testfiledir} {maindir .. "/testfiles"}   {Where the tests are.}
\luavarset{testsuppdir} {testfiledir .. "/support"} {Where support files for the tests are.}
\luavarseparator
\luavarset{localdir}   {maindir .. "/build/local"}  {Generated folder where support files are placed to allow \enquote{sandboxed} \TeX{} runs.}
\luavarset{testdir}    {maindir .. "/build/test"}   {Generated folder where tests are run.}
\luavarset{typesetdir} {maindir .. "/build/doc"}    {Generated folder where typesetting is run.}
\luavarset{unpackdir}  {maindir .. "/build/unpack"} {Generated folder where unpacking occurs.}
\luavarset{distribdir} {maindir .. "/build/distrib"}{Generated folder where the archive is created.}
\luavarset{ctandir}    {distribdir .. "/ctan"}      {Generated folder where files are organised for CTAN.}
\luavarset{tdsdir}     {distribdir .. "/tds"}       {Generated folder where files are organised for a TDS.}
\luavarset{tdsroot}    {"latex"}{Root directory of the TDS structure for the bundle/module to be installed into.}
\luavarseparator
\luavarset{bibfiles}        {\{"*.bst"\}}{\BibTeX{} database files.}
\luavarset{binaryfiles}     {\{"*.pdf", "*.zip"\}}
                            {Files to be added in binary mode to zip files.}
\luavarset{bstfiles}        {\{"*.bst"\}}{\BibTeX{} style files.}
\luavarset{checkfiles}      {\{~\}}{Extra files unpacked purely for tests}
\luavarset{checksuppfiles}  { }{Files needed for performing regression tests.}
\luavarset{cmdchkfiles}     {\{~\}}{Files need to perform command checking (\cls{l3doc}-based documentation only).}
\luavarset{cleanfiles}      {\{"*.log", "*.pdf", "*.zip"\}}{Files to delete when cleaning.}
\luavarset{demofiles}       {\{~\}}{Files which show how to use a module.}
\luavarset{docfiles}        {\{~\}}{Files which are part of the documentation but should not be typeset.}
\luavarset{excludefiles}    {\{"*\string~"\}}            {Files to ignore entirely (default for Emacs backup files).}
\luavarset{installfiles}    {\{"*.sty"\}}         {Files to install to the \TeX{} tree and similar tasks.}
\luavarset{makeindexfiles}  {\{"*.ist"\}}{MakeIndex files to be included in a TDS-style zip}
\luavarset{sourcefiles}     {\{"*.dtx", "*.ins"\}}{Files to copy for unpacking.}
\luavarset{textfiles}       {\{"*.md", "*.txt"\}}{Plain text files to send to CTAN as-is.}
\luavarset{typesetfiles}    {\{"*.dtx"\}} {Files to typeset for documentation.}
\luavarset{typesetsuppfiles}{\{~\}}       {Files needed to support typesetting when \enquote{sandboxed}.}
\luavarset{unpackfiles}     {\{"*.ins"\}} {Files to run to perform unpacking.}
\luavarset{unpacksuppfiles} {\{~\}}       {Files needed to support unpacking when \enquote{sandboxed}.}
\luavarseparator
\luavarset{lvtext} {".lvt"} {Extension of test files.}
\luavarset{tlgext} {".tlg"} {Extension of test file output.}
\luavarset{lvtext} {".lve"} {Extension of auto-generating test file output.}
\luavarset{logext} {".log"} {Extension of checking output, before processing it into a \texttt{.tlg}.}
\luavarseparator
\luavarset{checkdeps}   {\{~\}} {List of build unpack dependencies for checking.}
\luavarset{typesetdeps} {\{~\}} {\dots for typesetting docs.}
\luavarset{unpackdeps}  {\{~\}} {\dots for unpacking.}
\luavarseparator
\luavarset{checkengines}{\{"pdftex", "xetex", "luatex"\}} {Engines to check with \texttt{check} by default.}
\luavarset{stdengine}    {"pdtex"} {Engine to generate \texttt{.tlg} file from.}
\luavarset{checkformat}  {"latex"} {Format to use for tests.}
\luavarseparator
\luavarset{typesetexe} {"pdflatex"} {Executable for compiling \texttt{doc(s)}.}
\luavarset{unpackexe}  {"tex"}      {Executable for running \texttt{unpack}.}
\luavarset{zipexe}     {"zip"}      {Executable for creating archive with \texttt{ctan}.}
\luavarseparator
\luavarset{checkopts}  {"-interaction=batchmode"}  {Options based to engine when running checks.}
\luavarset{cmdchkopts} {"-interaction=batchmode"}  {Options based to engine when running command checks.}
\luavarset{typesetopts}{"-interaction=nonstopmode"}{Options based to engine when typesetting.}
\luavarset{unpackopts} {""}                        {Options based to engine when unpacking.}
\luavarset{zipopts}    {"-v -r -X"}                {Options based to zip program.}
\luavarseparator
\luavarset{checksearch}  {true}  {Look in \texttt{tds} dirs for checking?}
\luavarset{typesetsearch}{true}  {Look in \texttt{tds} dirs for typesetting docs?}
\luavarset{unpacksearch} {true}  {Look in \texttt{tds} dirs for unpacking?}
\luavarseparator
\luavarset{glossarystyle}{"gglo.ist"}{MakeIndex style file for glossary/changes creation}
\luavarset{indexstyle}   {"gind.ist"}{MakeIndex style for index creation}
\luavarseparator
\luavarseparator
\luavarset{biberexe}     {"biber"}    {Biber executable}
\luavarset{biberopts}    {""}         {Biber options}
\luavarset{bibtexexe}    {"bibtex8"}  {\BibTeX{} executable}
\luavarset{bibtexopts}   {"-W"}       {\BibTeX{} options}
\luavarset{makeindexexe} {"makeindex"}{MakeIndex executable}
\luavarset{makeindexopts}{""}         {MakeIndex options}
\luavarseparator
\luavarset{asciiengines}{\{"pdftex"\}}{Engines which should log as sure ASCII}
\luavarset{checkruns}   {1}           {How many times to run a check file before comparing the log.}
\luavarset{packtdszip}  {false}       {Build a TDS-style zip file for CTAN?}
\luavarset{scriptname}  {"build.lua"} {Name of script used in dependencies.}
\luavarset{typesetcmds} {""}          {Instructions to be passed to \TeX{} when doing typesetting.}
}
\allluavars
\newcommand\luavartypeset{%
  \begingroup
  \frenchspacing
  \renewcommand\luavarset[3]{
    \texttt{##1} & \texttt{##2} & ##3 \\
  }
  \renewcommand\luavarseparator{\midrule}
  \setlength\LTleft{-0.21\linewidth}
  \begin{longtable}{@{}%
    >{\small}
    p{0.18\linewidth}
    @{\hspace{0.03\linewidth}}
    >{\footnotesize\hangindent=1em}
    p{0.34\linewidth}
    @{\hspace{0.03\linewidth}}
    >{\small\raggedright\arraybackslash}
    p{0.63\linewidth}
    @{}}
  \toprule
  Variable & Default & Description \\
  \midrule\endhead
  \allluavars
  \bottomrule
  \end{longtable}
  \endgroup
}

\newcommand\code{\texttt}
\newcommand\var{\texttt}

\usepackage[procnames]{listings}
\lstset{
    basicstyle=\ttfamily\small,
    numbers=left,
    numberstyle={\tiny\color[gray]{0.4}},
}

\usepackage{shortvrb}
\usepackage{enumitem}
\usepackage{longtable}
\MakeShortVerb\|
\begin{document}
  \DocInput{\jobname.dtx}
\end{document}
%</driver>
% \fi
%
% \title{^^A
%   The \pkg{l3build} package\\ Checking and building packages^^A
%   \thanks{This file describes v\ExplFileVersion,
%      last revised \ExplFileDate.}^^A
% }
%
% \author{^^A
%  The \LaTeX3 Project\thanks
%    {^^A
%      E-mail:
%        \href{mailto:latex-team@latex-project.org}
%          {latex-team@latex-project.org}^^A
%    }^^A
% }
%
% \date{Released \ExplFileDate}
%
% \maketitle
% \tableofcontents
%
% \begin{documentation}
%
% \section{The \pkg{l3build} system}
%
% \subsection{Introduction}
%
% The \pkg{l3build} system is a Lua script for building \TeX{} packages, with particular emphasis on regression testing.
% It is written in cross-platform Lua code, so can be used by any modern \TeX{} distribution with the |texlua| interpreter.
% A package for building with \pkg{l3build} can be written in any \TeX\ dialect; its defaults are set up for \LaTeX\ packages written in the DocStrip style. (Caveat: minimal testing has yet been performed for non-\LaTeX{} packages.)
%
% Test files are written as standalone \TeX{} documents using the |regression-test.tex| setup file; documentation on writing these tests is discussed in Section~\ref{sec:writing-tests}.
%
% The |l3build.lua| script is not designed to be executed directly; each package will define its own |build.lua| script as a driver file which both sets variables (such as the name of the package) and then calls the main |l3build.lua| script internally.
%
% \pagebreak[2]
% A standard package layout might look something like the following:
% \begin{Verbatim}[fontsize=\small]
%  abc/
%      abc.dtx
%      abc.ins
%      build.lua
%      README.md
%      support/
%      testfiles/
% \end{Verbatim}
% Most of this should look fairly self-explanatory.
% The top level |support/| directory (optional) would contain any necessary files for compiling documentation, running regression tests, and so on.
%
% The \pkg{l3build} system is also capable of building and checking \emph{bundles} of packages.
% To avoid confusion, we refer to either a standalone package or a package within a bundle as a \emph{module}.
%
% For example, within the \LaTeX3 project we have the \textsf{l3packages} bundle which contains the \textsf{xparse}, \textsf{xtemplate}, etc., modules.
% These are all built and distributed as one bundle for installation, distribution \emph{via} CTAN and so forth.
%
% Each module in a bundle will have its own build script, and a bundle build script brings them all together.
% A standard bundle layout would contain the following structure.
% \begin{Verbatim}
% mybundle/
%          build.lua
%          support/
%          yyy/                    zoo/
%              build.lua               build.lua
%              README.md               README.md
%              testfiles/              testfiles/
%              yyy.dtx                 zoo.dtx
%              yyy.ins                 zoo.ins
% \end{Verbatim}
% All modules within a bundle must use the same build script name.
%
% In a small number of cases, the name used by CTAN for a module or bundle is
% different from that used in the installation tree. For example, the \LaTeXe{}
% kernel is called \pkg{latex-base} by CTAN but is located inside
% \texttt{\meta{texmf}/tex/latex/base}. This can be handled by using
% \var{ctanpkg} for the name required by CTAN to override the standard
% value.
%
% The |testfiles/| folder is local to each module, and its layout consists of a series of regression tests with their outputs.
% \begin{Verbatim}
% testfiles/
%           test1.lvt
%           test1.tlg
%           ...
%           support/
%                   my-test.cls
% \end{Verbatim}
% Again, the |support/| directory contains any files necessary to run some or all of these tests.
%
% When the build system runs, it creates a directory |build/| for various unpacking, compilation, and testing purposes.
% For a module, this build folder can be in the main directory of the package itself, but for a bundle it should be common for the bundle itself and for all modules within that bundle.
% A |build/| folder can be safety deleted; all material within is re-generated for each command of the \pkg{l3build} system.
%
% \subsection{Main build commands}
%
% In the working directory of a bundle or module, the following commands can be executed:
% \begin{itemize}[noitemsep]\ttfamily
% \item check
% \item check \meta{name(s)}
% \item cmdcheck
% \item clean
% \item doc
% \item install
% \item save \meta{name(s)}
% \end{itemize}
% These commands are described below.
%
% As well as these commands, the system recognises the options
% \begin{itemize}
% \item \texttt{--engine} (\texttt{-e}) Sets the engine to use for
%   testing
% \item \texttt{--halt-on-error} (\texttt{-H}) Specifies that checks
%   should stop as soon as possible, rather than running all requested
%   tests
% \item \texttt{--quiet} (\texttt{-q}) Suppresses output from unpacking
% \end{itemize}
%
% \begin{buildcmd}{check}
% The |check| command runs the entire test suite.
% This involves iterating through each \texttt{.lvt} file in the test directory (specified by the \var{testfiledir} variable), compiling each test in a \enquote{sandbox} (a directory specified by \var{testdir}), and comparing the output against each matching predefined \texttt{.tlg} file.
%
% If changes to the package or the typesetting environment have affected the results, the check for that file fails.
% A |diff| of the expected to actual output should then be inspected to determine the cause of the error; it is located in the \var{testdir} directory (default \luavar{testdir}).
%
% On Windows, the |diff| program is not available and so |fc| is used instead
% (generating an |.fc| file). Setting the environmental variables |diffexe|
% and |diffext| can be used to adjust the choice of comparison made: the standard
% values are
% \begin{itemize}
%   \item[Windows] |diffext = fc|, |diffexe = fc /n|
%   \item[*nix] |diffext = diff|, |diffexe = diff -c --strip-trailing-cr|
% \end{itemize}
%
% The following files are moved into the \enquote{sandbox} for the |check| process:
% \begin{itemize}
% \item all \var{installfiles} after unpacking;
% \item all \var{checkfiles} after unpacking;
% \item any files in the directory \var{testsuppdir};
% \item any files that match \var{checksuppfiles} in the \var{supportdir}.
% \end{itemize}
% This range of possibilities allow sensible defaults but significant flexibility for defining your own test setups.
%
% Checking can be performed with any or all of the `engines' \texttt{pdftex}, \texttt{xetex}, and \texttt{luatex}.
% By default, each test is executed with all three, being compared against the \texttt{.tlg} file produced from the \var{pdftex} engine (these defaults are controlled by the |checkengines| and |stdengine| variable respectively).
% The format used for tests can be altered by setting \var{checkformat}: the default setting \texttt{latex} means that tests are run using \emph{e.g.}~\texttt{pdflatex}, whereas setting to \texttt{plain} will run tests using \emph{e.g.}~\texttt{pdftex}.
% (Currently, this should be one of \texttt{latex} or \texttt{plain}.)
% To perform the check, the engine typesets each test \var{checkruns} times.
% More detail on this in the documentation on |save|.
% Options passed to the binary are defined in the variable \var{checkopts}.
%
% By default, |texmf| trees are searched for input files when checking. This can be disabled by setting \var{checksearch} to |false|: isolation provides confidence that the tests cannot accidentally be running with incorrect files installed in the main distribution or |hometexmf|.
% \end{buildcmd}
%
%
% \begin{buildcmd}{check \meta{name(s)}}
% Checks only the test \texttt{\meta{name(s)}.lvt}.
% All engines specified by \var{checkengines} are tested unless the command
% line option \texttt{--engine} (or \texttt{-e}) has been given to limit
% testing to a single engine.
% \end{buildcmd}
%
% \begin{buildcmd}{cmdcheck}
% For \cls{l3doc}-based sources, allows checking that the commands defined in the code part  (by \var{cmdchkfiles}) are documented in the description part.
% This is performed by passing the |check| option to the \cls{l3doc} class, typesetting the file(s) to check with engine \var{stdengine} with options \var{cmdchkopts}, and checking the resultant |.cmds| file(s).
% Dependencies are specified also with \var{checkdeps}.
% \end{buildcmd}
%
%
% \begin{buildcmd}{clean}
% This command removes all temporary files used for package bundling and regression testing.
% In the standard layout, these are all files within the directories defined by \var{localdir}, \var{testdir}, \var{typesetdir} and \var{unpackdir}, as well as all files defined in the \var{cleanfiles} variable in the same directory as the script.
% The defaults are |.pdf| files from typesetting (|doc|) and |.zip| files from bundling (|ctan|).
% \end{buildcmd}
%
%
% \begin{buildcmd}{ctan}
% Creates an archive of the package and its documentation, suitable for uploading to CTAN
% The archive is compiled in \var{distribdir}, and if the results are successful the resultant |.zip| file is moved into the same directory as the build script.
% If \var{packtdszip} is set true then  the building process includes a |.tds.zip| file containing the `\TeX\ Directory Structure' layout of the package or bundle.
% The archive therefore may contain two `views' of the package:
% \begin{Verbatim}
% abc.zip/
%         abc/
%             abc.dtx
%             abc.ins
%             abc.pdf
%             README.md
%         abc.tds.zip/
%                     doc/latex/abc/
%                                   abc.pdf
%                                   README.md
%                  source/latex/abc/
%                                   abc.dtx
%                                   abc.ins
%                     tex/latex/abc/
%                                   abc.sty
% \end{Verbatim}
% The files copied into the archive are controlled by a number of variables.
% The `root' of the TDS structure is defined by \var{tdsroot}, which is \luavar{tdsroot} by default. Plain users would redefine this to |"plain"| (or perhaps |"generic"|), for example.
% The build process for a |.tds.zip| file currently assumes a `standard'
% structure in which all extracted files should be placed inside the |tex|
% tree in a single directory, as shown above. If the module includes any
% \BibTeX{} or MakeIndex styles these will be placed in the appropriate
% subtrees.
%
% \pagebreak[2]
% The |doc| tree is constructed from:
% \begin{itemize}
% \item all files matched by \var{demofiles},
% \item all files matched by \var{docfiles},
% \item all files matched by \var{typesetfiles} with their extension replaced with |.pdf|,
% \item all files matched by \var{textfiles},
% \item all files matched by \var{bibfiles}.
% \end{itemize}
% The |source| tree is constructed from all files matched by \var{typesetfiles} and \var{sourcefiles}.
% The |tex| tree from all files matched by \var{installfiles}.
%
% Files that should always be excluded from the archive are matched against the \var{excludefiles} variable; by default this is \luavar{excludefiles}, which match Emacs' autosave files.
%
% Binary files should be specified with the \var{binaryfiles} variable (default \luavar{binaryfiles}); these are added to the zip archive without normalising line endings (text files are automatically converted to Unix-style line endings).
%
% To create the archive, by default the binary \var{zipexe} is used (\luavar{zipexe}) with options \var{zipopts} (|-v -r -X|). The intermediate build directories \var{ctandir} and \var{tdsdir} are used to construct the archive.
% \end{buildcmd}
%
%
% \begin{buildcmd}{doc}
% Compiles documentation files in the \var{typesetdir} directory.
% In the absence of one or more file names, all documentation is typeset;
% a file list may be given at the command line for selective typesetting.
% If the compilation is successful the |.pdf| is moved back into the main directory.
%
% The documentation compilation is performed with the \var{typesetexe} binary (default \texttt{pdflatex}), with options \var{typesetopts}.
% Additional \TeX{} material defined in \var{typesetcmds} is passed to the document (e.g., for writing |\\PassOptionsToClass{l3doc}{letterpaper}|, and so on---note that backslashes need to be escaped in Lua strings).
%
% Files that match |typesetsuppfiles| in the |support| directory (|supportdir|) are copied into the |build/local| directory (|localdir|) for the typesetting compilation process.
% Additional dependencies listed in the \var{typesetdeps} variable (empty by default) will also be installed.
%
% If \var{typesetsearch} is \code{true} (default), standard \texttt{texmf} search trees are used in the typesetting compilation. If set to false, \emph{all} necessary files for compilation must be included in the |build/local| sandbox.
% \end{buildcmd}
%
%
% \begin{buildcmd}{install}
% Copies all package files (defined by \var{installfiles}) into the user's home \texttt{texmf} tree in the form of the \TeX\ Directory Structure.
% \end{buildcmd}
%
%
% \begin{buildcmd}{save \meta{name(s)}}
% This command runs through the same execution as |check| for a specific test(s) \texttt{\meta{name(s)}.lvt}.
% This command saves the output of the test to a |.tlg| file.
% This file is then used in all subsequent checks against the \texttt{\meta{name}.lvt} test.
%
% If the \texttt{--engine} (or \texttt{-e}) is specified (one of |pdftex|, |xetex|, or |luatex|), the saved output is stored in \texttt{\meta{name}.\meta{engine}.tlg}. This is necessary if running the test through a different engine produces a different output.
% A normalisation process is performed when checking to avoid common differences such as register allocation; full details are listed in section~\ref{sec:norm}.
% \end{buildcmd}
%
%
% \begin{buildcmd}{unpack}
% This is an internal target that is normally not needed on user level.
% It unpacks all files into the directory defined by \var{unpackdir}. This occurs before other build commands such as |doc|, |check|, etc.
%
% The unpacking process is performed by executing the \var{unpackexe} (default \texttt{tex}) with options \var{unpackopts} on all files defined by the \var{unpackfiles} variable; by default, all files that match \luavar{unpackfiles}.
%
% If additional support files are required for the unpacking process, these can be enumerated in the \var{unpacksuppfiles} variable.
% Dependencies for unpacking are defined with \var{unpackdeps}.
%
% By default this process allows files to be accessed in all standard |texmf| trees; this can be disabled by setting \var{unpacksearch} to |false|.
% \end{buildcmd}
%
% \subsection{Example build scripts}
%
% An example of a standalone build script for a package that uses self-contained |.dtx| files is shown in Figure~\ref{fig:breqn}.
% Here, the |module| only is defined, and since it doesn't use |.ins| files so the variable \var{unpackfiles} is redefined to run |tex| on the |.dtx| files instead to generate the necessary |.sty| files.
% There are some PDFs in the repository that shouldn't be part of a CTAN submission, so they're explicitly excluded, and here unpacking is done `quietly' to minimise console output when building the package.
% Finally, because this is a standalone package, we assume that \pkg{l3build} is installed in the main \TeX{} distribution and find the Lua script by searching for it.
%
% \begin{figure}[!b]
%   \begin{lstlisting}[frame=single,language={[5.2]Lua},gobble = 6]
%     #!/usr/bin/env texlua
%
%     -- Build script for breqn
%
%     module = "breqn"
%
%     unpackfiles = {"*.dtx"}
%     excludefiles = {"*/breqn-abbr-test.pdf",
%                     "*/eqbreaks.pdf"}
%     unpackopts  = "-interaction=batchmode"
%
%     kpse.set_program_name("kpsewhich")
%     dofile(kpse.lookup("l3build.lua"))
%   \end{lstlisting}
% \caption{The build script for the \pkg{breqn} package.}
% \label{fig:breqn}
% \end{figure}
%
% An example of a bundle build script for \pkg{l3packages} is shown in Figure~\ref{fig:bundle}.
% Note for \LaTeX3 we use a common file to set all build variables in one place, and the path to the |l3build.lua| script is hard-coded so we always use our own most recent version of the script.
% An example of an accompanying module build script is shown in Figure~\ref{fig:module}.
%
% \begin{figure}[p]
%   \begin{lstlisting}[frame=single,language={[5.2]Lua},gobble = 6]
%     #!/usr/bin/env texlua
%
%     -- Build script for LaTeX3 "l3packages" files
%
%     -- Identify the bundle: there is no module as this is the "driver"
%     bundle = "l3packages"
%
%     -- Location of main directory: use Unix-style path separators
%     maindir = ".."
%
%     -- Load the common build code: this is the one place that a path
%     -- needs to be hard-coded
%     dofile (maindir .. "/l3build/l3build-config.lua")
%     dofile (maindir .. "/l3build/l3build.lua")
%   \end{lstlisting}
% \caption{The build script for the \pkg{l3packages} bundle.}
% \label{fig:bundle}
% \end{figure}
%
% \begin{figure}[p]
%   \begin{lstlisting}[frame=single,language={[5.2]Lua},gobble = 6]
%     #!/usr/bin/env texlua
%
%     -- Build script for LaTeX3 "xparse" files
%
%     -- Identify the bundle and module:
%     bundle = "l3packages"
%     module = "xparse"
%
%     -- Location of main directory: use Unix-style path separators
%     -- Should match that defined by the bundle.
%     maindir = "../.."
%
%     -- Load the common build code: this is the one place that a path
%     -- needs to be hard-coded
%     dofile (maindir .. "/l3build/l3build-config.lua")
%     dofile (maindir .. "/l3build/l3build.lua")
%   \end{lstlisting}
% \caption{The build script for the \pkg{xparse} module.}
% \label{fig:module}
% \end{figure}
%
% \begin{figure}[p]
%   \begin{lstlisting}[frame=single,gobble = 6]
%     @echo off
%     texlua build.lua %*
%   \end{lstlisting}
% \caption{Windows batch file wrapper for running the build process.}
% \label{fig:batch}
% \end{figure}
%
% Under a Unix-like platform, you may wish to run `|chmod +x build.lua|' on these files, which allows a simpler command line use. Instead of writing\\
% \hspace*{2\parindent}|texlua build.lua check|\\ for example, you would simply write\\
% \hspace*{2\parindent}|./build.lua check|\\
% instead. (Or even omit the |./| depending on your path settings.)
% Windows users can achieve a similar effect by creating a file |build.bat| as show in Figure~\ref{fig:batch}.
%
% \subsection{Variables}
%
% This section lists all variables defined in the |l3build.lua| script that are available for customisation.
%
% \luavartypeset
%
% \subsection{Dependencies}
%
% If you have multiple packages that are developed separately but still interact in some way, it's often desirable to integrate them when performing regression tests.
% For \LaTeX3, for example, when we make changes to \pkg{l3kernel} it's important to check that the tests for \pkg{l3packages} still run correctly, so it's necessary to include the \pkg{l3kernel} files in the build process for \pkg{l3packages}.
%
% In other words, \pkg{l3packages} is \emph{dependent} on \pkg{l3kernel}, and this is specified in \pkg{l3build} by setting appropriately the variables \texttt{checkdeps}, \texttt{typesetdeps}, and \texttt{unpackdeps}.
% The relevant parts of the \LaTeX3 repository is structured as the following.
% \Needspace{3\baselineskip}
% \begin{Verbatim}
% l3/
%    l3kernel/
%               build.lua
%               expl3.dtx
%               expl3.ins
%               ...
%               testfiles/
%    l3packages/
%               build.lua
%               xparse/
%                      build.lua
%                      testfiles/
%                      xparse.dtx
%                      xparse.ins
%    support/
% \end{Verbatim}
% For \LaTeX3 build files, |maindir| is defined as top level folder |l3|, so all support files are located here, and the build directories will be created there.
% To set \pkg{l3kernel} as a dependency of \pkg{l3package}, within |l3packages/xparse/build.lua| the equivalent of the following is set:
% \begin{Verbatim}
%   maindir = "../.."
%   checkdeps = {maindir .. "/l3kernel"}
% \end{Verbatim}
% This ensures that the \pkg{l3kernel} code is included in all processes involved in unpacking and checking and so on.
% The name of the script file in the dependency is set with the |scriptname| variable; by default these are |"build.lua"|.
%
%
% \subsection{Output normalisation}
% \label{sec:norm}
%
% To allow test files to be used between different systems (\emph{e.g.}~when
% multiple developers are involved in a project), the log files are normalised
% before comparison during checking. This removes some system-dependent
% data but also some variations due to different engines. This normalisation
% consists of two parts: removing (\enquote{ignoring}) some lines and modifying
% others to give consistent test. Currently, the following types of line are
% ignored:
% \begin{itemize}
%   \item Lines before the \cs{START}, after the \cs{END} and within
%     \cs{OMIT}/\cs{TIMO} blocks
%   \item Entirely blank lines, including those consisting only of spaces.
%   \item Lines containing file dates in format
%    \texttt{\meta{yyyy}/\meta{mm}/\meta{dd}}.
%  \item Lines starting \cs{openin} or \cs{openout}.
% \end{itemize}
% Modifications made in lines are:
% \begin{itemize}
%   \item Removal of the name of the test file itself.
%   \item Removal of the |pdftex.map| load information given during
%     first page shipout.
%   \item Removal spaces at the start of lines.
%   \item Removal of |./| at start of file names.
%   \item Standardisation of the list of units known to \TeX{} (\pdfTeX{}
%     and \LuaTeX{} add a small number of additional units which are not
%     known to \TeX90 or \XeTeX{}).
%   \item Standardisation of \verb*|\csname\endcsname | to |\csname\endcsname|
%     (the former is formally correct, but the latter was produced for many
%     years due to a \TeX{} bug).
%   \item Conversion of \texttt{on line \meta{number}} to \texttt{on line ...}
%     to allow flexibility in changes to test files.
%   \item
% \end{itemize}
%
% \LuaTeX{} makes several additional changes to the log file. As normalising
% these may not be desirable in all cases, they are handled separately.
% When creating \LuaTeX{}-specific test files (either with \LuaTeX{} as
% the standard engine or saving a \LuaTeX{}-specific |.tlg| file) no further
% normalisation is undertaken. On the other hand, for cross-engine comparison
% the following normalisation is applied:
% \begin{itemize}
%   \item Removal of additional (unused) |\discretionary| points.
%   \item Removal of |U+...| notation for missing characters.
%   \item Removal of |display| for display math boxes
%     (included by \TeX90/\pdfTeX{}/\XeTeX).
%   \item Removal of Omega-like |direction TLT| information.
%   \item Removal of additional whatsit containing local paragraph information
%     (|\localinterlinepenalty|, \emph{etc.}).
%   \item Rounding of glue set to four decimal places (glue set may be
%     slightly different in \LuaTeX{} compared to other engines).
%   \item Conversion of low chars ($1$ to $31$) to |^^| notation.
% \end{itemize}
%
% When making comparisons between 8-bit and Unicode engines it is useful to
% format the top half of the 8-bit range such that it appears in the log as
% |^^|\texttt{\meta{char}} (the exact nature of the 8-bit output is otherwise
% dependent on the active code page). This may be controlled using the
% |asciiengines| option. Any engines named here will use a |.tcx| file to
% produce only ASCII chars in the log output, whilst for other engines
% normalisation is carried out from UTF-8 to ASCII. If the option is set to
% an empty table the latter process is skipped: suitable for cases where only
% Unicode engines are in use.
%
% \section{Writing test files}
% \label{sec:writing-tests}
%
% Test files are written in a \TeX{} dialect using the support file |regression-test.tex|, which should be |\input| at the very beginning of each test.
% Additional customisations to this driver can be included in a local |regression-test.cfg| file, which will be loaded automatically if found.
%
% The macros loaded by |regression-test.tex| set up the test system and provide a number of commands to aid the production of a structured test suite.
% The basis of the test suite is to output material into the |.log| file, from which a normalised test output (|.tlg|) file is produced by the build command |save|.
% A number of commands are provided for this; they are all written in uppercase to help avoid possible conflicts with other package commands.
%
% \subsection{Metadata and structural commands}
%
% Any commands that write content to the |.log| file that should be ignored can be surrounded by |\OMIT| \dots\ |\TIMO|.
% At the appropriate location in the document where the |.log| comparisons should start (say, after |\begin{document}|), the test suite must contain the |\START| macro.
% The test should then include \cs{AUTHOR}\marg{authors details} in case a test file fails in the future and needs to be re-analysed.
%
% Some additional diagnostic information can then be included as metadata for the conditions of the test.
% The desired format can be indicated with \cs{FORMAT}\marg{format name}, and any packages or classes loaded can be indicated with
%
% \noindent\hspace*{2\parindent} \cs{CLASS}\oarg{options}\marg{class name, version}\par
% \noindent\hspace*{2\parindent} \cs{PACKAGE}\oarg{options}\marg{package name, version}
%
% \noindent These do not provide information that is useful for automated checking; after all, packages change their version numbers frequently.
% Rather, including this information in a test indicates the conditions under which the test was definitely known to pass at a certain time in the past.
%
% The |\END| command signals the end of the test (but read on).
% Some additional diagnostic information is printed at this time to debug if the test did not complete `properly' in terms of mismatched brace groups or \cs{if}\dots\cs{fi} groups.
%
% In a \LaTeX{} document, |\end{document}| will implicitly call |\END| at the very end of the  compilation process.
% If |\END| is used directly (replacing |\end{document}| in the test), the compilation will halt almost immediately, and various tasks that |\end{document}| usually performs will not occur (such as potentially writing to the various |.toc| files, and so on). This can be an advantage if there is additional material printed to the log file in this stage that you wish to ignore, but it is a disadvantage if the test relies on various auxiliary data for a subsequent typesetting run.
% (See the \var{checkruns} variable for how these tests would be test up.)
%
% \subsection{Commands to help write tests}
%
% A simple command \cs{CHECKCOMMAND}\cs{\meta{macro}} is provided to check whether a particular \cs{\meta{macro}} is defined, undefined, or equivalent to \cs{relax}.
% This is useful to flag either that internal macros are remaining local to their definitions, or that defined commands definitely are defined, or even as a reminder that commands you intend to define in a future package need to be tested once they appear.
%
% \cs{TYPE} is used to write material to the \texttt{.log} file, like \LaTeX's \cs{typeout}, but it allows `long' input.
% The following commands are defined to use \cs{TYPE} to output strings to the \texttt{.log} file.
% \begin{itemize}
% \item
% \cs{SEPARATOR} inserts a long line of \texttt{=} symbols to break up the log output.
% \item
% \cs{NEWLINE} inserts a linebreak into the log file.
% \item
% \cs{TRUE}, \cs{FALSE}, \cs{YES}, \cs{NO} output those strings to the log file.
% \item
% \cs{ERROR} is \emph{not} defined but is commonly used to indicate a code path that should never be reached.
% \item
% The \cs{TEST}\marg{title}\marg{contents} command surrounds its \meta{contents} with some \cs{SEPARATOR}s and a \meta{title}.
% \item
% \cs{TESTEXP} surrounds its contents with \cs{TYPE} and formatting to match \cs{TEST}; this can be used as a shorthand to test expandable commands.
% \item
% TODO: would a \cs{TESTFEXP} command (based on \cs{romannumeral} expansion) be useful as well?
% \end{itemize}
% An example of some of these commands is shown following.
% \begin{Verbatim}
% \TEST{bool_set,~lazy~evaluation}
%  {
%   \bool_set:Nn \l_tmpa_bool
%    {
%     \int_compare_p:nNn 1=1
%     && \bool_if_p:n
%      {
%       \int_compare_p:nNn 2=3 ||
%       \int_compare_p:nNn 4=4 ||
%       \int_compare_p:nNn 1=\ERROR % is skipped
%      }
%     && \int_compare_p:nNn 2=2
%    }
%   \bool_if:NTF \l_tmpa_bool \TRUE \FALSE
%  }
% \end{Verbatim}
% This test will produce the following in the output.
% \begin{Verbatim}
% ==========================================
% TEST 8: bool_set, lazy evaluation
% ==========================================
% TRUE
% ==========================================
% \end{Verbatim}
% (Only if it's the eighth test in the file of course, and assuming \pkg{expl3}
% coding conventions are active.)
%
% \subsection{Showing box content}
%
% The commands introduced above are only useful for checking algorithmic or logical correctness.
% Many packages should be tested based on their typeset output instead; \TeX{} provides a mechanism for this by printing the contents of a box to the log file.
% The |regression-test.tex| driver file sets up the relevant \TeX{} parameters to produce as much output as possible when showing box output.
%
% A plain \TeX{} example of showing box content follows.
% \begin{Verbatim}[frame=single,fontsize=\small]
% \input regression-test.tex\relax
% \START
% \setbox0=\hbox{\rm hello \it world $a=b+c$}
% \showbox0
% \END
% \end{Verbatim}
% This produces the output shown in Figure~\ref{fig:box-log} (left side).
% It is clear that if the definitions used to typeset the material in the box changes, the log output will differ and the test will no longer pass.
%
% The equivalent test in \LaTeXe{} using \pkg{expl3} is similar.
% \begin{Verbatim}[frame=single,fontsize=\small]
% % \iffalse meta-comment
%
% Copyright (C) 1992-1994 by David Carlisle, Frank Mittelbach.  
% Copyright (C) 2008 LaTeX3 project
% All rights reserved.
% 
% This file is part of the validate package.
% 
% IMPORTANT NOTICE:
% 
% You are not allowed to change this file.  In case of error
% write to the email address mentioned in the file readme.val.
% 
% \fi
%                  regression-test.tex
                   %%%%%%%%%%%%%%%%%%%

% This file should not be used as a package or class file, 
% it should be \input.

% The scope of this \makeatletter will then be the rest of the
% document.  Put TeX into scroll mode, and stop it showing the
% implementation details of macros in error messages.
\makeatletter
\scrollmode
\errorcontextlines=-1

% Use the same \showbox settings as 2.09, unless they are changed in 
% the test file. (2e sets these to -1)
\showboxbreadth=5
\showboxdepth=3

% Start the test, after the optional \documentclass (or \documentstyle)
% \begin{document} commands with \START.  All lines in the .log file
% before this will be ignored. It also prints a docstrip-style
% character table in the .tlg file so the .tlg file can easily be
% checked for email translations.
%
\def\START{\typeout{START-TEST-LOG^^J^^J%
   This is a generated file for the LaTeX (2e + expl3) validation system.%
^^J^^JDon't change this file in any respect.%
^^J^^J\CTable^^J}}

\begingroup
\catcode`\^^\=0
\catcode`\^^A=\catcode`\%
^^\catcode`^^\ =11
^^\catcode`^^\%=11
^^\catcode`^^\#=11
^^\catcode`^^\~=11
^^\endlinechar=`^^\^^J
^^\catcode`^^\\=11^^A
^^\gdef^^\CTable{
%% \CharacterTable
%%  {Upper-case    \A\B\C\D\E\F\G\H\I\J\K\L\M\N\O\P\Q\R\S\T\U\V\W\X\Y\Z
%%   Lower-case    \a\b\c\d\e\f\g\h\i\j\k\l\m\n\o\p\q\r\s\t\u\v\w\x\y\z
%%   Digits        \0\1\2\3\4\5\6\7\8\9
%%   Exclamation   \!     Double quote  \"     Hash (number) \#
%%   Dollar        \$     Percent       \%     Ampersand     \&
%%   Acute accent  \'     Left paren    \(     Right paren   \)
%%   Asterisk      \*     Plus          \+     Comma         \,
%%   Minus         \-     Point         \.     Solidus       \/
%%   Colon         \:     Semicolon     \;     Less than     \<
%%   Equals        \=     Greater than  \>     Question mark \?
%%   Commercial at \@     Left bracket  \[     Backslash     \\
%%   Right bracket \]     Circumflex    \^     Underscore    \_
%%   Grave accent  \`     Left brace    \{     Vertical bar  \|
%%   Right brace   \}     Tilde         \~}
%%
}^^A
^^\endgroup{}%

% The test should end with
% \END or \end{document}
%
\let\@@@end\@@end
%\let\@ED=\enddocument
\def\END{\typeout{END-TEST-LOG}\@@@end}
\let\@@end\END


% After the \START should come declarations of the format and style
% options being used.
%
\def\FORMAT#1{\typeout{Format: #1}%
  \def\@tempa{#1}\ifx\@tempa\@EJ\else
   \OMIT\@warning{Declared format #1,^^JActual format \@EJ}\TIMO\fi}

% The old version got this information from everyjob, 
% but that does not work with LaTeX2e as \everyjob is cleared.
\edef\@EJ{\fmtname <\fmtversion>}

% Some author info:
\def\AUTHOR#1{\typeout{Author: #1}}
\def\ADDRESS#1{\typeout{Address: #1}}

% Not all packages declare themselves to the log file, and we can not
% rely on TeX`s output as it includes full path names, and does not
% include version numbers etc.  So for each package included give a
% declaration like: \PACKAGES{array v2.0d}
%
\def\STYLE#1{\typeout{Main Style: #1}}%
\def\STYLEOPTIONS#1{\typeout{Style Options: #1}}


% If The class or package is loaded with options, you may
% specify the options in the \CLASS (\PACKAGE) declaration. eg:
%
% \CLASS[german,a4page]{article v2.0 1994/01/02}
% \PACKAGE{ifthen v2.2 1993/11/12}
% \PACKAGE[dvips]{graphics v 3.8 1994/02/02}
%
\def\CLASS{\@ifnextchar[\OPTCLASS\XCLASS}
\def\OPTCLASS[#1]#2{%
  \typeout{Main Class: #2^^J\@spaces Options: #1}}
\def\XCLASS#1{%
  \typeout{Main Class: #1}}

\def\PACKAGE{\@ifnextchar[\OPTPACKAGE\XPACKAGE}
\def\OPTPACKAGE[#1]#2{%
  \typeout{Package: #2^^J\@spaces Options: #1}}
\def\XPACKAGE#1{%
  \typeout{Package: #1}}


% LaTeX2e always uses NFSS2 so new test files need not use 
% \FONTSELECTION but it is retained for compatibility for test files
% written for 209/NFSS1.
%
\def\FONTSELECTION#1{%
  \OMIT\@@warning{\noexpand\FONTSELECTION obsolete.^^J%
                 LaTeX2e always uses NFSS2}\TIMO
  \typeout{Font Selection: #1}}


% Surround commands which produce irrelevant lines in the .log file by
% \OMIT
% \TIMO
%
\def\OMIT{\typeout{OMIT}}
\def\TIMO{\typeout{TIMO}}

% After the above declarations, and before the main tests, you may
% optionally `declare' all the commands in the `module' that you are
% about to test. These commands will be registered as defined,
% undefined or relaxed (ie \let to \relax). You may wish to declare
% commands not currently implemented, so that if they are added at a
% later stage, the test will fail, reminding someone to document the
% fact that the user interface has changed. So if you are testing
% array and tabular environments, you may wish do declare
% \extrarowheight. This is undefined in the curent latex, but would
% become defined if Mittelbach's array.sty was incorporated into
% latex.tex.
%
\def\declare@command#1{%
  \ifx#1\@undefined\typeout{Undefined \string#1}\else
  \ifx#1\relax\typeout{Relaxed \space\space\string#1}\else
         \typeout{Defined \space\space\string#1}\fi\fi}


% To allow testing of possible changes, we allow extra code to be read
% in before the test starts. The necessary code should be placed in a
% file regression-test.cfg.
%
\OMIT
\InputIfFileExists{regression-test.cfg}
      {\typeout{^^J***^^Jregression-test.cfg in operation^^J***^^J}}{}
\TIMO

%%%%%%%%% 

% We are not starved for space in the log file output, so let's make it as
% verbose as is useful when reading the .diff's.
\newcount \gTESTint

\newcommand \SEPARATOR {%
  \typeout{%
    ============================================================%
  }%
}

% A long version of \typeout, because tests may contain \par tokens.
% Besides, with that \TYPE, we can do \TYPE { ... \TRUE ... \NEWLINE ... }.
\newcommand \LONGTYPEOUT [1] {%
  \begingroup
    \set@display@protect
    \let \TYPE \@firstofone
    \immediate \write \@unused {#1}%
  \endgroup
}

\newcommand \TEST [2] {%
  \advance \gTESTint 1
  \SEPARATOR
  \typeout{%
    TEST \the\gTESTint: \detokenize{#1}}%
  \SEPARATOR 
  \begingroup
    \let \TYPE \LONGTYPEOUT
    #2%
  \endgroup
  \SEPARATOR \typeout{}%
}

\newcommand \TESTEXP [2] {%
  \advance \gTESTint 1
  \SEPARATOR
  \typeout{%
    TEST \the\gTESTint: \detokenize{#1}}%
  \SEPARATOR 
  \begingroup
    \let \TYPE \@firstofone
    \LONGTYPEOUT{#2}%
  \endgroup
  \SEPARATOR \typeout{}%
}

\def \TRUE  {\TYPE{TRUE}}
\def \FALSE {\TYPE{FALSE}}
\def \YES   {\TYPE{YES}}
\def \NO    {\TYPE{NO}}

\def \NEWLINE {\TYPE{^^J}}

% We allocate a large number of registers now: this number
% can be changed to keep register numbers stable in test logs.

\newcount\regression@test@loop@int
\newcommand{\regression@test@alloc}[2]
  {%
    \regression@test@loop@int=\numexpr#1\relax
    \regression@test@loop#2%
  }
\newcommand{\regression@test@loop}[1]
  {
    \ifnum 0<\regression@test@loop@int
      #1\regression@test@dummy
      \advance\regression@test@loop@int by -1\relax
      \expandafter\regression@test@loop
      \expandafter#1%
    \fi
  }
\regression@test@alloc {30} \newcount
\regression@test@alloc {30} \newdimen
\regression@test@alloc {30} \newskip
\regression@test@alloc {30} \newmuskip
% \regression@test@alloc {30} \newtoks

\endinput




% \documentclass{article}
% \usepackage{expl3}
% \START
% \ExplSyntaxOn
% \box_new:N \l_tmp_box
% \hbox_set:Nn \l_tmp_box {hello~ \emph{world}~ $a=b+c$}
% \box_show:N \l_tmp_box
% \ExplSyntaxOff
% \END
% \end{Verbatim}
% The output from this test is shown in Figure~\ref{fig:box-log} (right side).
% There is marginal difference (mostly related to font selection and different logging settings in \LaTeX) between the plain and \pkg{expl3} versions.
%
% When examples are not self-contained enough to be typeset into boxes, it is possible to ask \TeX{} to output the entire contents of a page.
% Insert \cs{showoutput} for \LaTeX{} or set \cs{tracingoutput} positive for plain \TeX{}; ensure that the test ends with \cs{newpage} or equivalent because \TeX{} waits until the entire page is finished before outputting it.
%
% TODO: should we add something like \cs{TRACEPAGES} to be format-agnostic here? Should this perhaps even be active by default?
%
% \begin{figure}
% \hspace*{-3cm}
%   \begin{BVerbatim}[fontsize=\small]
% > \box0=
% \hbox(6.94444+0.83333)x90.56589
% .\tenrm h
% .\tenrm e
% .\tenrm l
% .\tenrm l
% .\tenrm o
% .\glue 3.33333 plus 1.66666 minus 1.11111
% .\tenit w
% .\tenit o
% .\tenit r
% .\tenit l
% .\tenit d
%
% .\glue 3.57774 plus 1.53333 minus 1.0222
% .\mathon
% .\teni a
% .\glue(\thickmuskip) 2.77771 plus 2.77771
% .\tenrm =
% .\glue(\thickmuskip) 2.77771 plus 2.77771
% .\teni b
% .\glue(\medmuskip) 2.22217 plus 1.11108 minus 2.22217
% .\tenrm +
% .\glue(\medmuskip) 2.22217 plus 1.11108 minus 2.22217
% .\teni c
% .\mathoff
%
% ! OK.
% l.9 \showbox0
%
%
%   \end{BVerbatim}
%   \qquad
%   \begin{BVerbatim}[fontsize=\small]
% > \box71=
% \hbox(6.94444+0.83333)x91.35481
% .\OT1/cmr/m/n/10 h
% .\OT1/cmr/m/n/10 e
% .\OT1/cmr/m/n/10 l
% .\OT1/cmr/m/n/10 l
% .\OT1/cmr/m/n/10 o
% .\glue 3.33333 plus 1.66666 minus 1.11111
% .\OT1/cmr/m/it/10 w
% .\OT1/cmr/m/it/10 o
% .\OT1/cmr/m/it/10 r
% .\OT1/cmr/m/it/10 l
% .\OT1/cmr/m/it/10 d
% .\kern 1.03334
% .\glue 3.33333 plus 1.66666 minus 1.11111
% .\mathon
% .\OML/cmm/m/it/10 a
% .\glue(\thickmuskip) 2.77771 plus 2.77771
% .\OT1/cmr/m/n/10 =
% .\glue(\thickmuskip) 2.77771 plus 2.77771
% .\OML/cmm/m/it/10 b
% .\glue(\medmuskip) 2.22217 plus 1.11108 minus 2.22217
% .\OT1/cmr/m/n/10 +
% .\glue(\medmuskip) 2.22217 plus 1.11108 minus 2.22217
% .\OML/cmm/m/it/10 c
% .\mathoff
%
% ! OK.
% <argument> \l_tmp_box
%
% l.12 \box_show:N \l_tmp_box
%   \end{BVerbatim}
%   \caption{Output from displaying the contents of a simple box to the log file, using plain \TeX{} (left) and \pkg{expl3} (right). Some blank lines have been added to the plain \TeX{} version to help with the comparison.}
%   \label{fig:box-log}
% \end{figure}
%
% \section{Alternative test formats}
%
% \subsection{Generating test files with \pkg{DocStrip}}
%
% It is  possible to pack tests inside source files. Tests generated during the
% unpacking process will be available to the \texttt{check} and \texttt{save}
% commands as if they were stored in the \texttt{testfiledir}. Any explicit
% test files inside \texttt{testfiledir} take priority over generated ones
% with the same names.
%
% \subsection{Specifying expectations}
%
% Regression tests check whether changes introduced in the code modify the test
% output. Especially while developing a complex package there is not yet a
% baseline to save a test goal with. It might then be easier to formulate the
% expected effects and outputs of tests directly. To achieve this, you may
% create an \texttt{.lve} instead of a \texttt{.tlg}
% file.\footnote{Mnemonic: \texttt{lv\textbf{t}}: \textbf{t}est,
% \texttt{lv\textbf{e}}: \textbf{e}xpectation} It is processed exactly like
% the \texttt{.lvt} to generate the expected outcome. The test fails when both
% differ.
%
% Combining both features enables contrasting the test with its expected
% outcome in a compact format. Listing \ref{fig:expect-dtx} exemplary tests
% \TeX{}s counters. Listing \ref{fig:expect-ins} shows the relevant part of an
% \texttt{.ins} file to generate it.
%
% \begin{figure}
%   \begin{lstlisting}[frame=single,language={TeX},gobble = 6]
%     \input regression-test.tex\relax
%     \START
%     \TEST{counter-math}{
%     %<*test>
%       \OMIT
%       \newcounter{numbers}
%       \setcounter{numbers}{2}
%       \addtocounter{numbers}{2}
%       \stepcounter{numbers}
%       \TIMO
%       \typeout{\arabic{numbers}}
%     %</test>
%     %<expect>  \typeout{5}
%     }
%     \END
%   \end{lstlisting}
%   \caption{Test and expectation can be specified side-by-side in a single
%     \texttt{.dtx} file.}
%   \label{fig:expect-dtx}
% \end{figure}
%
%\begin{figure}
%   \begin{lstlisting}[frame=single,language={TeX},gobble = 6]
%      \generate{\file{\jobname.lvt}{\from{\jobname.dtx}{test}}
%                \file{\jobname.lve}{\from{\jobname.dtx}{expect}}}
%   \end{lstlisting}
%   \caption{Test and expectation are generated from a \texttt{.dtx} file of
%     the same name.}
%   \label{fig:expect-ins}
% \end{figure}
%
% \subsection{Typesetting documentation}
%
% As part of the overall build process, \pkg{l3build} will create PDF
% documentation as described earlier. The standard build process for PDFs
% will attempt to run Biber, \BibTeX{} and MakeIndex as appropriate
% (the exact binaries used are defined by \luavar{biberexe}, \luavar{bibtexexe}
% and \luavar{makeindexexe}). However, there is no attempt to create an entire
% PDF creation system in the style of \texttt{latexmk} or similar.
%
% For package authors who have more complex requirements than those covered
% by the standard set up, the Lua script offers the possibility for
% customisation. The Lua function \texttt{typeset} may be defined before
% reading \texttt{l3build.lua} and should take one argument, the name of
% the file to be typeset. Within this function, the auxiliary Lua
% functions \texttt{biber}, \texttt{bibtex}, \texttt{makeindex} and
% \texttt{tex} can be used, along with custom code, to define a PDF
% typesetting pathway. The functions \texttt{biber} and \texttt{bibtex}
% take a single argument: the name of the file to work with \emph{minus}
% any extension. The \texttt{tex} takes as an arugment the full name
% of the file. The most complex function \texttt{makeindex} requires the
% name, input extension, putput extension, log extension and style name.
% For example,  Figure~\ref{fig:PDF} shows a simple script which might
% apply to a case where multiple \BibTeX{} runs are needed (perhaps where
% citations can appear within other references).
%
% \begin{figure}[!b]
%   \begin{lstlisting}[frame=single,language={[5.2]Lua},gobble = 6]
%     #!/usr/bin/env texlua
%
%     -- Build script with custom PDF route
%
%     module = "mymodule"
%
%     function typeset (file)
%       local name = string.match (file, "^(.*)%.") or name
%       local errorlevel = tex (file)
%       if errorlevel == 0 then
%         -- Return a non-zero errorlevel if anything goes wrong
%         errorlevel = (
%           bibtex (name) +
%           tex (file)    +
%           bibtex (name) +
%           tex (file)    +
%           tex (file)
%         )
%       end
%       return errorlevel
%     end
%
%     kpse.set_program_name("kpsewhich")
%     dofile(kpse.lookup("l3build.lua"))
%   \end{lstlisting}
%   \caption{A customised PDF creation script.}
%   \label{fig:PDF}
% \end{figure}
%
% \end{documentation}
%
% \begin{implementation}
%
% \clearpage
% \section{\texttt{regression-test.tex}}
%
% This section describes the code for setting up the regression test system.
% Each test file should start with |\input regression-test.tex\relax|.
%
%    \begin{macrocode}
%<*package>
%    \end{macrocode}
%
% \subsection{Preliminaries}
%
% Unlike in the \LaTeXe{} regression test suite, reset catcodes: each test should set these
% as appropriate.
%    \begin{macrocode}
\expandafter\edef\csname reset\string @catcodes\endcsname{%
  \catcode`\noexpand\@=\the\catcode`\@\relax
}
\catcode`\@=11 %
%    \end{macrocode}
% Put \TeX{} into scroll mode, and stop it showing the
% implementation details of macros in error messages.
%    \begin{macrocode}
\scrollmode
\errorcontextlines=-1 %
%    \end{macrocode}
%
% Show all box details: this avoids getting variable results if boxes
% have different numbers of lines (\LuaTeX{} adds extra information).
%    \begin{macrocode}
\showboxbreadth=\maxdimen
\showboxdepth=\maxdimen
%    \end{macrocode}
%
% The |\showoutput| command of \LaTeX{} calls |\loggingoutput| which in
% turn calls |\errorstopmode| but we want to continue running without
% any stops.
%    \begin{macrocode}
\def\loggingoutput{\tracingoutput1
    \showboxbreadth\maxdimen\showboxdepth\maxdimen}
%    \end{macrocode}

%
% Set the newline character: \LaTeXe{} does this but plain-based formats
% do not.
%    \begin{macrocode}
\newlinechar=`\^^J
%    \end{macrocode}
%
% \subsection{Commands in test files}
%
% A long version of |\typeout|, because tests may contain |\par| tokens.
% Besides, with that |\TYPE|, we can do |\TYPE { ... \TRUE ... \NEWLINE ... }|.
%    \begin{macrocode}
\long\def\LONGTYPEOUT#1{%
  \begingroup
    \long\def\TYPE##1{##1}%
    \immediate\write17{#1}%
  \endgroup
}
\let\TYPE\LONGTYPEOUT
%    \end{macrocode}
%
% Start the test, after the optional |\documentclass|
% |\begin{document}| commands with |\START|.  All lines in the |.log| file
% before this will be ignored.
%
% It also prints a DocStrip-style
% character table in the |.tlg| file.
% Is this necessary any more? Probably too late to remove it.
%    \begin{macrocode}
\def\STARTMESSAGE{This is a generated file for the l3build validation system.}
\def\START{\LONGTYPEOUT{START-TEST-LOG^^J^^J%
   \STARTMESSAGE%
^^J^^JDon't change this file in any respect.%
^^J^^J\CTable^^J}}
\begingroup
\catcode`\^^\=0
\catcode`\^^A=\catcode`\%
^^\catcode`^^\ =11
^^\catcode`^^\%=11
^^\catcode`^^\#=11
^^\catcode`^^\~=11
^^\endlinechar=`^^\^^J
^^\catcode`^^\\=11^^A
^^\gdef^^\CTable{
%% \CharacterTable
%%  {Upper-case    \A\B\C\D\E\F\G\H\I\J\K\L\M\N\O\P\Q\R\S\T\U\V\W\X\Y\Z
%%   Lower-case    \a\b\c\d\e\f\g\h\i\j\k\l\m\n\o\p\q\r\s\t\u\v\w\x\y\z
%%   Digits        \0\1\2\3\4\5\6\7\8\9
%%   Exclamation   \!     Double quote  \"     Hash (number) \#
%%   Dollar        \$     Percent       \%     Ampersand     \&
%%   Acute accent  \'     Left paren    \(     Right paren   \)
%%   Asterisk      \*     Plus          \+     Comma         \,
%%   Minus         \-     Point         \.     Solidus       \/
%%   Colon         \:     Semicolon     \;     Less than     \<
%%   Equals        \=     Greater than  \>     Question mark \?
%%   Commercial at \@     Left bracket  \[     Backslash     \\
%%   Right bracket \]     Circumflex    \^     Underscore    \_
%%   Grave accent  \`     Left brace    \{     Vertical bar  \|
%%   Right brace   \}     Tilde         \~}
%%
}^^A
^^\endgroup{}%
%    \end{macrocode}
% The test should end with |\END| or |\end{document}|
% or |\bye| in plain \TeX{}.
%    \begin{macrocode}
\ifx\@@end\@undefined
  \let\@@@end\end
\else
  \let\@@@end\@@end
\fi
\def\END
  {%
    \ifx\currentiflevel\@undefined\else
      \ifnum\currentgrouplevel>0 %
        \LONGTYPEOUT{Bad grouping: \the\currentgrouplevel!}%
      \fi
      \ifnum\currentiflevel>2 %
        \LONGTYPEOUT{Bad conditionals: \the\numexpr\currentiflevel-2!}%
      \fi
    \fi
    \LONGTYPEOUT{END-TEST-LOG}\@@@end
  }
\ifx\@@end\@undefined
  \let\end\END
\else
  \let\@@end\END
\fi
%    \end{macrocode}
% After the |\START| should come declarations of the format and style
% options being used.
%    \begin{macrocode}
\def\FORMAT#1{\LONGTYPEOUT{Format: #1}%
  \def\@tempa{#1}\ifx\@tempa\@EJ\else
   \OMIT\TYPE{WARNING: Declared format #1,^^JActual format \@EJ}\TIMO\fi}
%    \end{macrocode}
% The old version got this information from everyjob,
% but that does not work with \LaTeXe\ as |\everyjob| is cleared.
%    \begin{macrocode}
\edef\@EJ{\fmtname <\fmtversion>}
%    \end{macrocode}
% Some author info:
%    \begin{macrocode}
\def\AUTHOR#1{\LONGTYPEOUT{Author: #1}}
%    \end{macrocode}
% Surround commands which produce irrelevant lines in the .log file by
% |\OMIT|\dots|\TIMO|
%    \begin{macrocode}
\def\OMIT{\LONGTYPEOUT{OMIT}}
\def\TIMO{\LONGTYPEOUT{TIMO}}
%    \end{macrocode}
% Not all packages declare themselves to the log file, and we can not
% rely on TeX`s output as it includes full path names, and does not
% include version numbers etc.
%
% If the class or package is loaded with options, you may
% specify the options in the |\CLASS| (|\PACKAGE|) declaration. eg:
% \begin{verbatim}
% \CLASS[german,a4page]{article v2.0 1994/01/02}
% \PACKAGE{ifthen v2.2 1993/11/12}
% \PACKAGE[dvips]{graphics v 3.8 1994/02/02}
% \end{verbatim}
%    \begin{macrocode}
\def\CLASS{\@ifnextchar[\OPTCLASS\XCLASS}
\def\OPTCLASS[#1]#2{%
  \TYPE{Main Class: #2^^J\space\space\space\space Options: #1}}
\def\XCLASS#1{%
  \TYPE{Main Class: #1}}
\def\PACKAGE{\@ifnextchar[\OPTPACKAGE\XPACKAGE}
\def\OPTPACKAGE[#1]#2{%
  \TYPE{Package: #2^^J\space\space\space\space Options: #1}}
\def\XPACKAGE#1{%
  \TYPE{Package: #1}}
%    \end{macrocode}
%
% The commands above require 2e's \verb|\@ifnextchar|, so copy that definition verbatim if necessary:
%    \begin{macrocode}
\ifx\@ifnextchar\@undefined
\long\def\@ifnextchar#1#2#3{%
  \let\reserved@d=#1%
  \def\reserved@a{#2}%
  \def\reserved@b{#3}%
  \futurelet\@let@token\@ifnch}
\def\@ifnch{%
  \ifx\@let@token\@sptoken
    \let\reserved@c\@xifnch
  \else
    \ifx\@let@token\reserved@d
      \let\reserved@c\reserved@a
    \else
      \let\reserved@c\reserved@b
    \fi
  \fi
  \reserved@c}
\def\:{\let\@sptoken= } \: % this makes \@sptoken a space token
\def\:{\@xifnch} \expandafter\def\: {\futurelet\@let@token\@ifnch}
\fi
%    \end{macrocode}
% After the above declarations, and before the main tests, you may
% optionally declare' all the commands in the `module' that you are
% about to test. These commands will be registered as defined,
% undefined or relaxed (i.e.~|\let| to |\relax|). You may wish to declare
% commands not currently implemented, so that if they are added at a
% later stage, the test will fail, reminding someone to document the
% fact that the user interface has changed.
%    \begin{macrocode}
\def\CHECKCOMMAND#1{%
  \ifx#1\@undefined\LONGTYPEOUT{Undefined \string#1}\else
  \ifx#1\relax\LONGTYPEOUT{Relaxed \space\space\string#1}\else
         \LONGTYPEOUT{Defined \space\space\string#1}\fi\fi}
%    \end{macrocode}
% To allow testing of possible changes, we allow extra code to be read
% in before the test starts. The necessary code should be placed in a
% file |regression-test.cfg|.
%    \begin{macrocode}
\OMIT
\ifx\InputIfFileExists\@undefined
\newread\@inputcheck
\long\def\InputIfFileExists#1#2#3{%
  \openin\@inputcheck#1\relax
  \ifeof\@inputcheck
    \def\reserved@a{#3}%
  \else
    \def\reserved@a{#2\input #1\relax}%
  \fi
  \closein\@inputcheck
  \reserved@a}
\fi
\InputIfFileExists{regression-test.cfg}
      {\LONGTYPEOUT{^^J***^^Jregression-test.cfg in operation^^J***^^J}}{}
\TIMO
%    \end{macrocode}
%
% \subsection{Formatting the \texttt{.log} file}
%
% We are not starved for space in the log file output, so let's make it as
% verbose as is useful when reading the |.diff|'s.
%    \begin{macrocode}
\newcount \gTESTint
\def\SEPARATOR{%
  \TYPE{%
    ============================================================%
  }%
}
%    \end{macrocode}
%    \begin{macrocode}
\long\def\TEST#1#2{%
  \advance \gTESTint 1 %
  \SEPARATOR
  \LONGTYPEOUT{%
    TEST \the\gTESTint: \detokenize{#1}}%
  \SEPARATOR
  \begingroup
    \let\TYPE\LONGTYPEOUT
    #2%
  \endgroup
  \SEPARATOR \LONGTYPEOUT{}%
}
\long\def\TESTEXP#1#2{%
  \advance \gTESTint 1 %
  \SEPARATOR
  \LONGTYPEOUT{%
    TEST \the\gTESTint: \detokenize{#1}}%
  \SEPARATOR
  \begingroup
    \long\def\TYPE##1{##1}%
    \LONGTYPEOUT{#2}%
  \endgroup
  \SEPARATOR
  \LONGTYPEOUT{}%
}

\def \TRUE  {\TYPE{TRUE}}
\def \FALSE {\TYPE{FALSE}}
\def \YES   {\TYPE{YES}}
\def \NO    {\TYPE{NO}}

\def \NEWLINE {\TYPE{^^J}}
%    \end{macrocode}
% Finish up.
%    \begin{macrocode}
\reset@catcodes
%    \end{macrocode}
%
%    \begin{macrocode}
%</package>
%    \end{macrocode}
%
% \newpage
% \section{\texttt{l3build.lua}}
%
% This section consists of the |l3build.lua| code.
% This code is \emph{not} generated from |l3build.dtx| since it needs to already be extracted to build \pkg{l3build} itself!
% As Frank says, we don't want to end up with a double M\"unchhausen.
%
% \lstinputlisting
%   [
%    basicstyle=\ttfamily\scriptsize,
%    numbers=left,
%    numberstyle={\tiny\color[gray]{0.4}},
%    language={[5.2]Lua},
%    procnamekeys=function,
%    procnamestyle=\color{red},
%   ]
%   {l3build.lua}
%
% \end{implementation}
%
% \PrintIndex


